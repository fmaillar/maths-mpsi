\chapter{Espaces vectoriels euclidiens}
\label{chap:espaceseuclidiens}
\minitoc
\minilof
\minilot

\section{Produits scalaire}

\subsection{Notion de produit scalaire}

Soit un espace vectoriel $E$.
\begin{defdef}
  On appelle produit scalaire sur $E$ toute application
  \begin{equation}
    \fonction{\varphi}{E \times E}{\R}{(x,y)}{\prodscal{x}{y}},
  \end{equation}
  telle que~:
  \begin{enumerate}
  \item pour tous vecteurs $x$ et $y$ de $E$, on a $\prodscal{x}{y}=\prodscal{y}{x}$;
  \item pour tous vecteurs $x$, $x'$ et $y$ de $E$, et tout réels $\lambda$, on a $\prodscal{\lambda x +x'}{y} = \lambda\prodscal{x}{y} + \prodscal{x'}{y}$;
  \item pour tout vecteur $x$ non nul de $E$, $\prodscal{x}{x}>0$.
  \end{enumerate}
\end{defdef}
\emph{Remarque}~:
\begin{itemize}
\item La linéarité à droite est aussi vraie grâce à la symétrie, et alors le produit scalaire est une forme bilinéaire symétrique;
\item on peut aussi remarquer que $\prodscal{x}{x}=0$ si et seulement si $x=0$; alors dans ce cas le produit scalaire est défini.
  \begin{proof}
    $\implies$ Soit $x \in E$, si $x\neq 0$ alors $\prodscal{x}{x}>0$ et $\prodscal{x}{x}\geq 0$. Sinon, alors $x=0$ et $\prodscal{x}{x}=0$.

    $\impliedby$ Si $x \neq 0$ alors avec $\prodscal{x}{x}\geq 0$, on a $\prodscal{x}{x} \neq 0$. Donc $\prodscal{x}{x}>0$.
  \end{proof}
\end{itemize}

Finalement, un produit scalaire est une forme bilinéaire symétrique et définie-positive.

\begin{prop}
  Soit  $(E,\varphi)$ un espace vectoriel réel muni d'un produit scalaire. Pour tout vecteur $x \in E$, on a
  \begin{equation}
    \left(\forall y \in E \quad \prodscal{x}{y}=0\right) \iff x=0.
  \end{equation}
\end{prop}
\begin{proof}
  $\impliedby$ Déjà vu.

  $\implies$ On applique l'égalité en $y=x$ et on a $\prodscal{x}{x}=0$, qui est vrai si et seulement si $x=0$. 
\end{proof}

\subsection{Norme euclidienne associée à un produit scalaire}

Soit  $(E,\varphi)$ un espace vectoriel réel muni d'un produit scalaire.

\subsubsection{Définition}

\begin{defdef}
  Pour tout vecteur $x \in E$, on définit sa norme euclidienne par $\norme{x}=\sqrt{\prodscal{x}{x}}$. Qui est légitime puisque $\prodscal{x}{x} \geq 0$.
\end{defdef}

\danger La norme dépend du produit scalaire choisi.

\emph{Relation entre le produit scalaire et la norme}
Soient $x$ et $y$ deux vecteurs de $E$ et $\lambda$ et $\mu$ deux réels, alors
\begin{gather}
  \norme{\lambda x + \mu y} = \lambda^2 \norme{x}^2 + \mu^2 \norme{y}^2 + 2\lambda\mu\prodscal{x}{y}; \\
  \norme{x + y} = \norme{x}^2 + \norme{y}^2 + 2\prodscal{x}{y}; \\
\norme{x - y} = \norme{x}^2 + \norme{y}^2 - 2\prodscal{x}{y}; \\
\prodscal{x}{y} = \frac{1}{4}\left(\norme{x+y}^2-\norme{x-y}^2 \right) = \frac{1}{2}\left(\norme{x+y}^2-\norme{x}^2-\norme{x}^2 \right);\\
\norme{x + y}^2+\norme{x - y}^2 = 2\left(\norme{x}^2 + \norme{y}^2\right).
\end{gather}
\begin{proof}
  On démontre la première égalité grâce à la bilinéarité du produit scalaire et grâce à la définition de la norme. Les autres points se démontre facilement.
\end{proof}

\subsubsection{Inégalité de Cauchy-Schwarz}

\begin{theo}[Théorème de Cauchy-Schwarz]
  Pour tous vecteurs $x$ et $y$ de $E$, on a
  \begin{equation}
    \abs{\prodscal{x}{y}} \leq \norme{x}\norme{y}.
  \end{equation}
\end{theo}
\begin{proof}
  Soient $x$ et $y$ deux vecteurs de $E$, on définit l'application
  \begin{equation}
    \fonction{P}{\R}{\R}{\lambda}{\norme{x+\lambda y}}.
  \end{equation}
  Elle vérifie~:
  \begin{enumerate}
  \item pour tout réel $\lambda$, $P(\lambda)\geq 0$;
  \item pour tout réel $\lambda$, $P(\lambda) = \norme{x}^2+\lambda^2\norme{y}^2+2\lambda\prodscal{x}{y}$.
  \end{enumerate}
  $P$ est un polynôme de degré inférieur ou égal à $2$. 

  Dans le cas où $\norme{y}=0$, alors $y=0$ et l'inégalité $\abs{\prodscal{x}{y}} \leq \norme{x}\norme{y}$ est vraie.

  Dans le cas où $\norme{y}\neq 0$, alors $P$ est de degré $2$ et soit $\delta$ le discriminant réduit de $P$. Comme $P$ ne change pas de signe ce discriminant est négatif et $\delta=\prodscal{x}{y}^2-\norme{x}^2\norme{y}^2\leq 0$. D'où l'inégalité $\abs{\prodscal{x}{y}} \leq \norme{x}\norme{y}$.
\end{proof}
%
\begin{prop}
  Pour tous vecteurs $x$ et $y$ de $E$,
  \begin{equation}
    \abs{\prodscal{x}{y}}=\norme{x}\norme{y} \iff (x,y) \text{~est liée}.
  \end{equation}
\end{prop}
\begin{proof}
  \begin{align}
    \abs{\prodscal{x}{y}}=\norme{x}\norme{y} &\iff \begin{cases} \abs{\prodscal{x}{y}}=\norme{x}\norme{y} \text{~et~} x=0 \\ \text{~ou~} \\
\abs{\prodscal{x}{y}}=\norme{x}\norme{y} \text{~et~} x\neq 0 \end{cases}\\
&\iff \begin{cases}  x=0 \\ \text{~ou~} \\
 x\neq 0 \text{~et~}  \delta=0\end{cases}\\
&\iff \begin{cases}  x=0 \\ \text{~ou~} \\
 x\neq 0 \text{~et~}  \exists \lambda_0 \in \R \quad P(\lambda_0)=0\end{cases}\\
&\iff \begin{cases}  x=0 \\ \text{~ou~} \\
 x\neq 0 \text{~et~}  \exists \lambda_0 \in \R \quad x=-\lambda_0 y\end{cases}\\
&\iff (x,y) \text{~est liée}.
  \end{align}
\end{proof}

\subsubsection{Inégalité de Minkowski}

\begin{theo}
  Pour tout couple de  vecteurs $(x,y) \in E$ on a
  \begin{equation}
    \norme{x+y} \leq \norme{x}+\norme{y}.
  \end{equation}
\end{theo}
\begin{proof}
  Soit un couple de  vecteurs $(x,y) \in E$, alors en appliquant le théorème de Cauchy-Schwarz on a 
  \begin{align}
    \norme{x+y}^2 &= \norme{x}^2 + \norme{y}^2 + 2\prodscal{x}{y} \\
    & \leq \norme{x}^2 + \norme{y}^2 + 2\abs{\prodscal{x}{y}} \\
    & \leq \norme{x}^2 + \norme{y}^2 + 2\norme{x}\norme{y}=(\norme{x}+\norme{y})^2.
  \end{align}
  Comme $\norme{x+y} \geq 0$, $\norme{x}\geq 0$ et $\norme{y}\ge 0$, le sens de l'inégalité est conservé en passant à la racine carrée et on a bien
  \begin{equation}
    \norme{x+y} \leq \norme{x}+\norme{y}.
  \end{equation}
\end{proof}
%
\begin{prop}
  Pour tout couple de  vecteurs $(x,y) \in E$ on a
  \begin{equation}
    \norme{x+y} = \norme{x}+\norme{y} \iff x=0 \text{~ou~} (x\neq 0 \text{~et~} \exists \mu \geq 0  \quad y=\mu x)
  \end{equation}
\end{prop}
\begin{proof}
  Soit un couple de  vecteurs $(x,y) \in E$, alors
  \begin{align}
    \norme{x+y} = \norme{x}+\norme{y} &\iff \begin{cases} \prodscal{x}{y} = \abs{\prodscal{x}{y}} \\ \text{~et~} \\ \abs{\prodscal{x}{y}} = \norme{x}\norme{y}\end{cases} \\
    &\iff  \begin{cases}  \prodscal{x}{y} \geq 0 \\ \text{~et~} \\ \begin{cases} x=0 \\ \text{~ou~} \\ x\neq 0 \text{~et~} \exists \lambda \in \R \quad y=\lambda x\end{cases} \end{cases}\\
    & \iff \begin{cases} \prodscal{x}{y} \geq 0 \text{~et~} x =0 \\ \text{~ou~} \\  \prodscal{x}{y} \geq 0 \text{~et~} x \neq 0 \text{~et~} \exists \lambda \in \R \quad y=\lambda x \end{cases}\\
    & \iff \begin{cases} x=0 \\ \text{~ou~} \\ x \neq 0 \text{~et~} \exists \lambda \in \R \quad y=\lambda x \text{~et~} \lambda\norme{x}^2 \geq 0 \end{cases} \\
      & \iff \begin{cases} x=0 \\ \text{~ou~} \\ x \neq 0 \text{~et~} \exists \lambda \in \Rpluss \quad y=\lambda x \end{cases}
  \end{align}
\end{proof}

\emph{Conséquences~:} Soit un couple de  vecteurs $(x,y) \in E$, alors
\begin{gather}
  \abs{\norme{x}-\norme{y}} \leq \norme{x+y}\\
  \abs{\norme{x}-\norme{y}} \leq \norme{x-y}.
\end{gather}
\begin{proof}
  On a
  \begin{equation}
    \norme{x} = \norme{x+y-y} \leq \norme{x+y} +\norme{y}
  \end{equation}
  alors
  \begin{equation}
    \norme{x} - \norme{y} \leq \norme{x+y}
  \end{equation}
  et en passant à la valeur absolue on a bien
  \begin{equation}
    \abs{\norme{x}-\norme{y}} \leq \norme{x+y}.
  \end{equation}
  On applique cette inégalité à $-y$ pour avoir la deuxième.
\end{proof}

\subsubsection{Application norme euclidienne}

Soit un $\R$-espace vectoriel $E$ muni d'un produit scalaire $\prodscal{.}{.}$, et soit $\norme{.}$ la norme euclidienne associée.

\begin{prop}
  L'application $\fonction{\norme{.}}{E}{\R}{x}{\norme{x}}$ est une norme, c'est-à-dire qu'elle vérifie
  \begin{enumerate}
  \item la positivité : pour tout vecteur $x \in E$, $\norme{x} \geq 0$;
  \item la séparation : pour tout vecteur $x \in E$, $\norme{x}= 0 \implies x=0$;
  \item l'homogénéité : pour tout réel $\lambda$ et tout vecteur $x \in E$, $\norme{\lambda x}=\abs{\lambda}\norme{x}$;
  \item l'inégalité triangulaire (ou de Minkowski) : pour tout couple de vecteurs $(x,y) \in E^2$, $\norme{x+y}\leq \norme{x}+\norme{y}$.
  \end{enumerate}
\end{prop}
\begin{proof}
  La démonstration repose sur les propriété du produit scalaire, parce que pour tout vecteur $x \in E$, $\norme{x}^2 = \prodscal{x}{x}$.
\end{proof}

\subsubsection{Distance associée à la norme euclidienne}

Avec les mêmes notations, on définit l'application distance
\begin{equation}
  \fonction{d}{E^2}{\R}{(x,y)}{\norme{x-y}}.
\end{equation}
% 
\begin{prop}
  L'application distance $d$ vérifie les propriétés suivantes~: Pour tout triplet $(x,y,z) \in E^3$ on a
  \begin{enumerate}
  \item $d(x,y) \in \Rpluss$;
  \item $d(y,x)=d(x,y)$;
  \item $d(x,y)=0 \implies x=y$;
  \item $d(x,z)\leq d(x,y)+\d(y,z)$.
  \end{enumerate}
\end{prop}

\subsection{Exemples de produits scalaires}

\subsubsection{Produit scalaire sur $\R^n$}

Soit $E=\R^n$, $x=(x_i)_{1\le i\leq n}$ et $y=(y_i)_{1\le i\leq n}$. On pose $\prodscal{x}{y}=\sum_{i=1}^n x_iy_i$. Alors $\fonction{}{\R^n\times\R^n}{\R}{(x,y)}{\prodscal{x}{y}}$ est un produit scalaire. C'est le produit scalaire canonique. La norme euclidienne associée est telle que
\begin{equation}
  \forall x \in \R^n \quad \norme{x} = \sqrt{\sum_{i=1}^n x_i^2}.
\end{equation}
La distance associée est telle que
\begin{equation}
  \forall (x,y) \in \R^n\times\R^n \quad d(x,y)=\norme{x-y} = \sqrt{\sum_{i=1}^n (x_i-y_i)^2}.
\end{equation}
L'inégalité de Cauchy-Scharwz s'écrit
\begin{equation}
  \forall (x,y) \in \R^n\times\R^n \quad \abs{\sum_{i=1}^n x_iy_i} \leq \sqrt{\sum_{i=1}^n x_i^2} \sqrt{\sum_{i=1}^n y_i^2}.
\end{equation}
L'inégalité de Minkowski s'écrit
\begin{equation}
  \forall (x,y) \in \R^n\times\R^n \quad \sqrt{\sum_{i=1}^n (x_i+y_i)^2} \leq \sqrt{\sum_{i=1}^n x_i^2}+ \sqrt{\sum_{i=1}^n y_i^2}.
\end{equation}

\subsubsection{Produit scalaire sur $\cont{\intervalleff{a}{b}}{\R}$}

Soient deux réels $a$ et $b$ tels que $a<b$ et l'application
\begin{equation}
  \fonction{\varphi}{\cont{\intervalleff{a}{b}}{\R}^2}{\R}{(f,g)}{\int_{[a,b]}fg}.
\end{equation}
L'application $\varphi$ est bilinéaire, symétrique, et définie-positive. C'est-à-dire que c'est un produit scalaire. En effet pour toute fonction $f$ continue sur $[a,b]$ on a bien
\begin{enumerate}
\item $\int_{[a,b]}f^2 \geq 0$;
\item $\int_{[a,b]}f^2 = 0 \iff f^2=\tilde{0} \iff f=\tilde{0}$.
\end{enumerate}


\emph{Remarque~: Sur l'ensemble des fonctions continues par morceaux, $\varphi$ n'est pas un produit scalaire car $\varphi$ n'est pas définie positive. On dispose qand même de l'inégalité de Cauchy-Schwarz.}

La norme euclidienne associée à $\varphi$ est telle que
\begin{equation}
  \forall f \in \cont{\intervalleff{a}{b}}{\R} \quad \norme{f}=\sqrt{\int_{[a,b]}f^2}.
\end{equation}
La distance euclidienne associée est 
\begin{equation}
  \forall (f,g) \in \cont{\intervalleff{a}{b}}{\R}^2 \quad d(f,g)=\sqrt{\int_{[a,b]}(f-g)^2}.
\end{equation}
L'inégalité de Cauchy-Schwarz s'écrit
\begin{equation}
  \forall (f,g) \in \cont{\intervalleff{a}{b}}{\R}^2 \quad \abs{\int_{[a,b]}fg} \leq \sqrt{\int_{[a,b]}f^2} \sqrt{\int_{[a,b]}g^2}.
\end{equation}
L'inégalité de Minkowski s'écrit
\begin{equation}
  \forall (f,g) \in \cont{\intervalleff{a}{b}}{\R}^2 \quad \sqrt{\int_{[a,b]}(f+g)^2} \leq \sqrt{\int_{[a,b]}f^2} + \sqrt{\int_{[a,b]}g^2}.
\end{equation}

\emph{Remarque~:} Soit $E$ l'ensemble des fonctions continues $2\pi$-périodiques de $\R$ dans $\R$. On définit l'application $\fonction{\varphi}{E^2}{\R}{(f,g)}{\frac{1}{2\pi}\int_{\intervalleff{0}{2\pi}}fg}$. C'est un produit scalaire.

\subsection{Familles de vecteurs orthogonales et familles de vecteurs orthonormales}

Soit $E$ un $\R$-espace vectoriel muni d'un produit scalaire $\varphi$.

\subsubsection{Vecteur unitaire}

\begin{defdef}
  Soit $x \in E$. On dit que $x$ est unitaire si et seulement $\norme{x}=1$. \emph{C'est une caractéristique qui dépend du produit scalaire $\varphi$}.
\end{defdef}

\emph{Remarque}~: Si $x$ est non nul, alors $\frac{x}{\norme{x}}$ est un vecteur unitaire colinaire à $x$ et de même sens.

\subsubsection{Vecteurs orthogonaux}

\begin{defdef}
  Soient deux vecteurs $x$ et $y$ de $E$. On dit que $x$ et $y$ sont orthogonaux, on note $x \perp y$ si et seulement si $\prodscal{x}{y}=0$.
\end{defdef}

\danger C'est une notion qui dépend du produit scalaire choisi.

\begin{prop}
  \begin{gather}
    \forall (x,y) \in E^2 \quad x \perp y \iff y \perp x;\\
    \forall x \in E \quad x \perp 0;\\
    \forall x \in E \quad x \perp x \iff x=0;\\
    \forall x \in E \quad [(\forall y \in E \quad x \perp y) \iff x=0].
  \end{gather}
\end{prop}

Le seul vecteur orthogonal à tous les vecteurs de l'espace est le vecteur nul.
\begin{proof}
  Il suffit de réécrire les propriétés déjà vues.
\end{proof}
\begin{theo}[Théorème de Pythagore]
  Pour tout couple de vecteurs $(x,y) \in E^2$, on a
  \begin{equation}
    x \perp y \iff \norme{x+y}^2 = \norme{x}^2 + \norme{y}^2.
  \end{equation}
\end{theo}
\begin{proof}
  \begin{align}
    x \perp y &\iff \prodscal{x}{y}=0 \\
    &\iff \frac{1}{2} (\norme{x+y}^2-\norme{x}^2-\norme{y}^2)=0 \\
    &\iff \norme{x+y}^2 = \norme{x}^2 + \norme{y}^2.
  \end{align}
\end{proof}

\subsubsection{Familles orthogonales, familles orthonormales}

\begin{defdef}
  Soit un ensemble fini $I$ ayant au moins deux éléments. Soit $\X=(x_i)_{i \in I}$ une famille de vecteurs de $E$ indéxée par $I$.
  \begin{itemize}
  \item On dit que $\X$ est une famille orthogonale de $E$ (FOG) si et seulement si
    \begin{equation}
      \forall (i,j) \in I^2 \quad i\neq j \implies x_i \perp x_j.
    \end{equation}
    Les vecteurs sont deux à deux orthogonaux.
  \item On dit qe $\X$ est une famille orthonormale de $E$ (FON) si et seulement si $\X$ est une famille orthogonale de $E$ et si pour tout $i \in I$, $\norme{x_i}=1$.
  \end{itemize}
\end{defdef}

\begin{prop}
  Une famille orthogonale finie de vecteurs de $E$ dont aucun n'est nul est libre.
\end{prop}
\begin{proof}
  Soit $\X=(x_i)_{i \in I}$ une famille orthogonale finie de $E$ et $(\alpha_i)_{i \in I} \in \R^I$ telle que $\sum_{i \in I}\alpha_i x_i=0$. Soit $i_0 \in I$, alors $\prodscal{x_{i_0}}{0}=0$. Alors
  \begin{align}
    0 &=\prodscal{\sum_{i \in I}\alpha_i x_i}{x_{i_0}} \\
    &=\sum_{i \in I}\alpha_i \prodscal{x_i}{x_{i_0}} && \text{linéarité à gauche}\\
    &=\alpha_{i_0} \norme{x_{i_0}}^2. && \text{famille orthogonale}
  \end{align}
  Comme tous les vecteurs sont non nuls on a $x_{i_0}\neq 0$ et donc $\norme{x_{i_0}} \neq 0$. Si bien que $\alpha_{i_0}=0$, qui est vrai pour tout $i_0$, alors la famille $\X$ est libre.
\end{proof}

\begin{theo}[Théorème de Pythagore pour une famille orthogonale finie]
  Soit un naturel $n\geq 2$ et une famille orthogonale $(x_i)_{i \in \intervalleentier{1}{n}} \in E^n$. Alors
  \begin{equation}
    \norme{\sum_{i=1}^nx_i}^2 = \sum_{i=1}^n \norme{x_i}^2.
  \end{equation}
\end{theo}
\begin{proof}[Démonstration par récurrence]
  Initialement on prend $n=2$ et c'est le théorème de Pythagore classique énoncé ci-avant. Soit un naturel $n \ge 2$ et supposons l'assertion vraie au rang $n$, montrons qu'elle est encore vraie au rang $n+1$. Soit famille orthogonale $(x_i)_{i \in \intervalleentier{1}{n+1}} \in E^{n+1}$, alors
  \begin{equation}
    \norme{\sum_{i=1}^{n+1}x_i}^2 =  \norme{\sum_{i=1}^{n}x_i+x_{n+1}}^2.
  \end{equation}
  Or
  \begin{align}
    \prodscal{\sum_{i=1}^{n}x_i}{x_{n+1}} &=\sum_{i=1}^{n}\prodscal{x_i}{x_{n+1}} && \text{linéarité} \\
    &=0 && \text{famille orthogonale}.
  \end{align}
  Alors $x_{n+1} \perp \sum_{i=1}^n x_i$. Donc
  \begin{align}
    \norme{\sum_{i=1}^{n+1}x_i}^2 &=  \norme{\sum_{i=1}^{n}x_i+x_{n+1}}^2 \\
    &=\norme{\sum_{i=1}^{n}x_i}^2 + \norme{x_{n+1}}^2 && x_{n+1} \perp \sum_{i=1}^n x_i \\
    &=\sum_{i=1}^{n+1}\norme{x_i}^2 &&  \text{Hypothèse de récurrence}.
  \end{align}
  L'assertion au rang $n+1$ est vraie. Par théorème de récurrence, le résultat est vraie pour tout naturel $n \geq 2$.
\end{proof}

\subsection{Orthogonalité de sous-espaces vectoriels}

\subsubsection{Orthogonal}

Soit $(E,\varphi)$ un $\R$-espace vectoriel muni d'un produit scalaire $\varphi$.

\begin{defdef}
  Soit $F$ un sous-espace vectoriel de $E$. Posons
  \begin{equation}
    F^{\perp}=\enstq{x \in E}{\forall y \in F \quad x \perp y}.
  \end{equation}
  C'est l'ensemble des vecteurs de $E$ qui sont orthogonaux à tous les vecteurs de $F$. $F^{\perp}$ est appelé l'orthogonal de $F$.
\end{defdef}

\emph{Exemples~:}
\begin{itemize}
\item Si $F=E$ alors $F^{\perp}=\{0\}$;
\item Si $F=\{0\}$ alors $F^{\perp}=E$.
\end{itemize}

\begin{prop}
  Pour tout sous-espace vectoriel $F$ de $E$, $F^{\perp}$ est un sous-espace vectoriel de $E$.
\end{prop}
\begin{proof}
  Par définition $F^\perp \subset E$. Comme $0 \in F^\perp$, il est non vide. Soient deux vecteurs $x$ et $x'$ de $F^\perp$ et $\lambda \in \R$. Pour tout $y \in F$ on a
  \begin{equation}
    \prodscal{\lambda x+x'}{y}=\lambda \prodscal{x}{y}+\prodscal{x'}{y}=0.
  \end{equation}
  alors $\lambda x+x' \perp y$ et donc $\lambda x+x' \in F^\perp$.

  Par caractérisation $F^\perp$ est un sous-espace vectoriel de $E$.
\end{proof}

Comme $F^\perp$ est un sous-espace vectoriel, on peut définir son orthogonal $(F^\perp)^\perp$.

\begin{prop}
  Pour tout sous-espace vectoriel $F$ de $E$, on a
  \begin{equation}
    F \subset (F^\perp)^\perp.
  \end{equation}
  \danger En général ce n'est pas une égalité.
\end{prop}
\begin{proof}
  On note $G=F^\perp$. Soit $x \in F$, alors
  \begin{equation}
    \forall y \in G \quad x\perp y.
  \end{equation}
  Alors $x \in G^\perp$. Du coup $F \subset G^\perp=(F^\perp)^\perp$.
\end{proof}

\subsubsection{Sous-espaces vectoriels othogonaux}

\begin{defdef}
  Soient $F$ et $G$ deux sous-espaces vectoriels de $E$. On dit que $F$ et $G$ sont orthogonaux et on note $F \perp G$, si et seulement si
  \begin{equation}
    \forall x \in F \ \forall y \in G \quad x \perp y.
  \end{equation}
\end{defdef}
\begin{prop}
  Soit $F$ un sous-espace vectoriel de $E$. Alors
  \begin{enumerate}
  \item $F \perp F^\perp$;
  \item pour tout sous-espace vectoriel $G$ de $E$,
    \begin{equation}
      G \perp F \iff G \subset F^\perp.
    \end{equation}
    $F^\perp$ est le plus grand sous-espace vectoriel (au sens de l'inclusion) qui est orthogonal à $F$.
  \end{enumerate}
\end{prop}
\begin{proof}
  \begin{enumerate}
  \item Évident par définitin de $F^\perp$;
  \item Si $G \perp F^\perp$ soit alors $x \in G$. Pour tout $y \in F$, on a $x \perp y$ (car $G \perp F$) donc $x \in F^\perp$, par définition de $F^\perp$.
  \end{enumerate}
\end{proof}
\begin{prop}
  Soient $F_1$ et $F_2$ deux sous-espace vectoriels de $E$, alors
  \begin{equation}
    F_1 \subset F_2 \implies F_2^\perp \subset F_1^\perp.
  \end{equation}
\end{prop}
\begin{proof}
  Soit $x \in F_2^\perp$, alors pour tout $y \in F_1$, $y \in F_2$ donc $x \perp y$. Alors $x \in F_1^\perp$.
\end{proof}
\begin{prop}
  Pour tous sous-espaces vectoriels $F$ et $G$ de $E$, on a
  \begin{equation}
    F \perp G \implies F \cap G = \{0\}.
  \end{equation}
\end{prop}
\begin{proof}
  Si $x \in F \cap G$, alors $x (\in F) \perp x (\in G)$ donc $x=0$.  $F \cap G \subset \{0\}$. L'autre inclusion est triviale puisque $F \cap G$ est un sous-espace vectoriel.
\end{proof}
\begin{cor}
  Pour tout sous-espace vectoriel $F$ de $E$, $F \cap F^\perp =\{0\}$.
\end{cor}

\subsubsection{Orthogonalité et supplémentaire}
\label{subsec:orthetsupp}

Soit $F$ un sous-espace vectoriel de $E$. On sait que~:
\begin{itemize}
\item $F$ et $F^\perp$ sont en somme directe mais à priori $F \oplus F^\perp \subset E$;
\item $F \subset (F^\perp)^\perp$.
\end{itemize}

\begin{theo}\label{theo:orthetsupp}
  Soient $F$ et $G$ deux sous-espaces vectoriels de $E$. On suppose qu'ils sont supplémentaires dans $E$. Il y a équivalence entre les assertions suivantes~:
  \begin{enumerate}
  \item $F \perp G$;
  \item $G = F^\perp$;
  \item $F = G^\perp$.
  \end{enumerate}
  Auquel cas $E = F \oplus G = F \oplus F^\perp$.
\end{theo}
\begin{proof}
  $1 \implies 2$ D'après la proposition précédente, on a déjà $G \subset F^\perp$, montrons l'autre inclusion~: soit $x \in F^{\perp}$. Comme $E=F\oplus G$ alors il existe un unique couple $(x_1,x_2) \in F \times G$ tel que $x=x_1+x_2$. Alors $0=\prodscal{x}{x_1}=\prodscal{x_1+x_2}{x_1}$. Comme $x_2 \in G$ et $x_1 \in F$, et $F \perp G$ on a $\prodscal{x_1}{x_2}=0$. Du coup $\norme{x_1}^2=0$ donc $x_1=0$. Finalement $x=x_2 \in G$. Alors $F^\perp \subset G$. Par double inclusion on a $G=F^\perp$.
  
  $1 \implies 3$ C'est la même démonstration en échangeant $F$ et $G$.
  
  $2 \implies 1$ et $3 \implies 1$ C'est clair.
\end{proof}

\begin{corth}[Supplémentaire orthogonal]
\label{corth:supportho}
  Pour tout sous-espace vectoriel $F$ de $E$, il existe \emph{au plus} un sous-espace vectoriel $G$ tel que $\begin{cases} E = F \oplus G \\ F \perp G \end{cases}$. S'il existe alors $G=F^\perp$. On parle de supplémentaire orthogonal.
\end{corth}
\begin{corth}
  Soit $F$ un sous-espace vectoriel de $E$. Si $E=F \oplus F^\perp$ alors $F=(F^\perp)^\perp$.
\end{corth}
\begin{proof}
  On applique le théorème avec $G=F^\perp$ : $F=G^\perp = (F^\perp)^\perp$.
\end{proof}

\section{Espace vectoriel euclidien}

\subsection{Définition d'un espace vectoriel euclidien}

\begin{defdef}
  On appelle espace vectoriel euclidien tout couple $(E,\varphi)$ où $E$ est un $\R$-espace vectoriel de dimension finie non nulle, et $\varphi$ un produit scalaire sur $E$.
\end{defdef}

\emph{Remarque~:} Si $F$ est un sous-espace vectoriel non nul de $E$, on peut montrer que $\varphi_F=\varphi_{|F\times F}$ est un produit scalaire sur $F$. Alors $(F,\varphi_F)$ est un espace vectoriel euclidien.

\subsection{Supplémentaire orthogonal d'un sous-espace vectoriel euclidien}

Soit $(E,\varphi)$ un espace vectoriel euclidien.

\begin{theo}
  Soit $F$ un sous-espace vectoriel de $E$. Alors~:
  \begin{itemize}
  \item $E=F\oplus F^\perp$;
  \item $\Dim F^\perp = \Dim E - \Dim F$;
  \item $F=(F^\perp)^\perp$.
  \end{itemize}
  On dit que $F^\perp$ est le supplémentaire orthogonal de $F$ (l'unicité a été prouvé à la sous-section~\ref{subsec:orthetsupp}). 
\end{theo}
\emph{Ne pas confondre l'unicité du supplémentaire orthogonal avec l'unicité du supplémentaire en général. En général un sous-espace vectoriel admet une infinité de supplémentaire.}
\begin{proof}
  \emph{Cas 1~:} C'est le cas facile où $F=\{0\}$, alors $F^\perp =E$ et alors les trois points sont vérifiés.

  \emph{Cas 2~:} On note $\Dim F=p \in \intervalleentier{1}{n}$ avec $n = \Dim E$. Soit $(f_1, \ldots, f_p)$ une base de $F$. On a montré que $F \oplus F^\perp \subset E$, alors $\Dim F^\perp \leq n-p$. Soit l'application
  \begin{equation}
    \fonction{u}{E}{\R^p}{x}{(\prodscal{x}{f_1}, \ldots, \prodscal{x}{f_p})}.
  \end{equation}
  L'application $u$ est linéaire, grâce à la linéarité à gauche du produit scalaire. Montrons que $\Ker(u)=F^\perp$~:
  \begin{itemize}
  \item Soit $x \in \Ker(u)$. Pour tout $i \in \intervalleentier{1}{p}$, on a $\prodscal{x}{f_i}=0$. Pour tout $y \in F$, il existe un unique $p$-uplet de scalaire $(\alpha_1, \ldots, \alpha_p)$ tel que $y = \sum_{k=1}\alpha_k f_k$. Alors
    \begin{align}
      \prodscal{x}{y} &= \sum_{k=1}\alpha_k \prodscal{x}{f_k} && \text{linéarité} \\
      &=\sum_{k=1}\alpha_k 0 && x \in \Ker(u) \\
      &=0,
    \end{align}
    alors $x \in F^\perp$. Alors $\Ker(u) \subset F^\perp$.
  \item Soit $x \in F^\perp$. Pour tout $i \in \intervalleentier{1}{p}$, on a $x\perp f_i$. Donc $\prodscal{x}{f_i}=0$, et alors $u(x)=0$. Finalement $x \in \Ker(u)$. $F^\perp \subset \Ker(u)$.
  \end{itemize}
  Par double inclusion $F^\perp = \Ker(u)$.

  En appliquant le théorème du rang à l'application $u$, on obtient
  \begin{equation}
    \Dim E = \rg(u) + \Dim \Ker(u),
  \end{equation}
  c'est-à-dire
  \begin{equation}
    \Dim F^\perp = n-\rg(u).
  \end{equation}
  Comme $\Image(u) \subset \R^p$, on a bien $\rg(u) \leq p$ et donc $\Dim F^\perp \geq n-p$. Avec l'inégalité inverse vue plus haut on obtient bien $\Dim F^\perp = n-p$.

  Finalement $F \oplus F^\perp \subset E$ et
  \begin{equation}
    \Dim F \oplus F^\perp = \Dim F + \Dim F^\perp = p+n-p=n=\Dim E
  \end{equation}
  nous donne que $F \oplus F^\perp = E$.
\end{proof}

\subsection{Bases orthogonales et bases orthonormales d'un espace vectoriel euclidien}

\subsubsection{Définition}

Soient $(E,\varphi)$ un espace vectoriel euclidien et $\B$ une famille finie de vecteurs de $E$.
\begin{defdef}
  $\B$ est une base orthogonale (BOG) de $E$ si et seulemet si $\B$ est une base de $E$ et si $\B$ est une famille orthogonale de $E$.

  $\B$ est une base orthonormale (BON) de $E$ si et seulemet si $\B$ est une base de $E$ et si $\B$ est une famille orthonormale de $E$.
\end{defdef}

\emph{Remarque~:} Si $E$ est une droite vectorielle, alors $\B=(x)$ est une BOG si et seulement si $x$ est non nul et c'est une BON si et seulement si $\norme{x}=1$.

\subsubsection{Caractérisation}

Soient $(E,\varphi)$ un espace vectoriel euclidien de dimension $n\in \N^*$ et $\B=(e_i)_{i \in \intervalleentier{1}{n}} \in E^n$.

\begin{prop}
  $\B$ est une base orthogonale de $E$ si et seulement si~:
  \begin{itemize}
  \item pour tout $i \in \intervalleentier{1}{n}$, $e_i \neq 0$;
  \item pour tout $(i,j)\in \intervalleentier{1}{n}^2$ tel que $i \neq j$ on a $e_i \perp e_j$.
  \end{itemize}
\end{prop}
\begin{proof}
  $\implies$ C'est évident : le fait que $\B$ est une base implique que tous les vecteurs de $\B$ sont non nuls, et le fait que la famille soit orthogonale implique trivialement l'orthogonalité des vecteurs deux à deux.

  $\impliedby$ Le deuxième point implique que la famille $\B$ est orthogonale. De plus c'est une famille orthogonale dont aucun des vecteurs n'est nul donc $\B$ est une famille libre. C'est une famille libre de cardinal $n=\Dim E$ donc c'est une base.
\end{proof}
%
\begin{prop}
  $\B$ est une base orthonormale de $E$ si et seulement si~:
  \begin{itemize}
  \item pour tout $i \in \intervalleentier{1}{n}$, $\norme{e_i}=1$;
  \item pour tout $(i,j)\in \intervalleentier{1}{n}^2$ tel que $i \neq j$ on a $e_i \perp e_j$.
  \end{itemize}
  C'est-à-dire
  \begin{equation}
    \B \text{~est une BON} \iff \forall (i,j)\in \intervalleentier{1}{n}^2 \quad \prodscal{e_i}{e_j}=\delta_{ij}.
  \end{equation}
\end{prop}
\begin{proof}
  $\implies$ C'est la définition de la BON.

  $\impliedby$ D'après la proposition précédente, $\B$ est une BOG\@. De plus tous ces vecteurs sont unitaires, donc c'est une BON\@.
\end{proof}
\begin{prop}
  Soit $\B=(e_i)_{1 \leq i \leq n}$ une BOG de $E$. Alors $\B'=\left(\frac{e_i}{\norme{e_i}}\right)_{i \in \intervalleentier{1}{n}}=(b_i)_{i \in \intervalleentier{1}{n}}$ est une BON de $E$.
\end{prop}
\begin{proof}
  Déjà $\B' \subset E$ et $\Card \B' = n = \Dim E$. De plus
  \begin{equation}
    \forall (i,j) \in \intervalleentier{1}{n}^2 \quad \prodscal{b_i}{b_j}=\frac{\prodscal{e_i}{e_j}}{\norme{e_i}\norme{e_j}} = \delta_{ij}.
  \end{equation}
  Par caractérisation, $\B$ est une BON de $E$.
\end{proof}

\emph{Exemple~:} La base canonique de $\R^n$ est une BON de $\R^n$ munique du produit scalaire canonique.

\subsection{Existence de bases orthonormales dans un espace euclidien}

\begin{theo}
  Tout espace vectoriel euclidien admet au moins une base orthonormale.
\end{theo}
\begin{proof}
  On prouve par récurrence sur $n \in \N^{*}$ l'assertion $\P(n)$ ``Si $E$ est un espace vectoriel euclidien de dimension $n$, alors $E$ admet une base orthonormale.''

\emph{Initialisation~:} Soit $E$ un espace vectoriel euclidien de dimension $1$. Soit $x \in E\setminus\{0\}$, alors $\left(\frac{x}{\norme{x}}\right)$ est une bas orthonormale de $E$. $\P(1)$ est vraie.

\emph{Hérédité~:} Soit un naturel $n$ non nul. Supposons que $\P(n)$ est vraie, et démontrons $\P(n+1)$. Soit $E$ un espace vectoriel euclidien de dimension $n+1$. Soit $x \in E\setminus\{0\}$ et posons $F = \VectEngendre(x)$.

$F^\perp$ est un sous-espace vectoriel de $E$ de dimension $n$. Muni du produit scalaire induit par celui de $E$, $F^\perp$ est un sous espace vectoriel euclidien de dimension $n$. Appliquons-lui l'hypothèse de récurrence : Il existe une base orthonormale $\B'=(b_1, \ldots, b_n)$ de $F^\perp$. Soit $\B=\B'\cup\{b_{n+1}\}$ avec $b_{n+1}=\left(\frac{x}{\norme{x}}\right)$. On vérifie que $\B$ est une base orthonormale de $E$~:
\begin{itemize}
\item $\Card(\B)=n+1=\Dim(E)$;
\item Pour tout $(i,j) \in \intervalleentier{1}{n+1}^2$ on a~:
  \begin{equation}
    \prodscal{b_i}{b_j} = \begin{cases}
      \delta_{ij} & \text{si~} i \leq n \text{~et~} j \leq n \\
      & \text{~car~} \B' \text{~est une BON} \\
      0 = \delta_{ij} & \text{si~} (i \leq n, \ j=n+1) \text{~ou~} (j \leq n, \ i=n+1) \\
      &  \text{~car~} F \perp F^{\perp} \\
      1 = \delta_{ij} & \text{si~} i=j=n+1 \text{~car~} \prodscal{b_{n+1}}{b_{n+1}}=1
      \end{cases}
  \end{equation}
\end{itemize}
Donc $\B$ est une base orthonormale de $E$.

\emph{Conclusion~:} Le théorème de récurrence permet de conclure que l'assertion $\P(n)$ est vraie pour tout naturel $n$ non nul.
\end{proof}
\begin{theo}
  Toute famille orthonormale d'un espace vectoriel euclidien $E$ peut être complétée en une base orthonormale de $E$.
\end{theo}
\begin{proof}
  Soit $(E,\varphi)$ un espace vectoriel euclidien de dimension $n \geq 1$. Soit $\F$ une famille orthonormale de $E$. $\F$ est donc une famille libre de $E$. Soit $p \leq n$ le cardinal de $\F$, et notons $\F=(f_1, \ldots, f_p)$. Deux cas se présentent~: Si $p=n$ alors $\F$ est déjà une base orthonormale.

  Sinon, on note $F=\VectEngendre(\F)$, et $\dim F^{\perp} = n-p\geq 1$. Muni du produit scalaire induit par celui de $E$, $F^\perp$ est un espace vectoriel euclidien. D'après le théorème précédent, $F^\perp$ admet une base orthonormale notée $\F'=(f_{p+1}, \ldots, f_{n})$. Soit $\B=\F\cup\F'$. Montrons que $\B$ est une base orthonormale de $E$~:
  \begin{itemize}
  \item $\Card(\B)=n=\Dim(E)$;
  \item Pour tout $(i,j) \in \intervalleentier{1}{n}^2$ on a
    \begin{equation}
      \prodscal{f_i}{f_j} = \begin{cases} \delta_{ij} & \text{si~} (i \leq p, \ j \leq p) \text{~ou~} (i \geq p+1, \ j \geq p+1) \\
        & \text{~car~} \F, \F'\text{~sont des~} BON \\
        0=\delta_{ij} & \text{~si~} (i \leq p, \ j \leq p) \text{~ou~} (i \geq p+1, \ j \geq p+1)\\
        &  \text{~car~} F \perp F^\perp 
      \end{cases}
    \end{equation}
    Donc $\B$ est une base orthonormale de $E$.
  \end{itemize}
\end{proof}
\begin{theo}
  Soit $E$ un espace vectoriel euclidien de dimension $n \geq 1$. Soit $\B=(b_1, \ldots, b_n)$ une base orthonormale de $E$. Soit $p \in \intervalleentier{0}{n}$. On note $\B_1 = (b_i)_{1 \leq i \leq p}$ ($\B_1 = \emptyset$ si $p=0$) et $\B_2 = (b_i)_{p+1 \leq i \leq n}$ ($\B_2 = \emptyset$ si $p=n$). Soient $F_1=\VectEngendre(\B_1)$ et $F_2=\VectEngendre(\B_2)$. Alors
  \begin{equation}
    \begin{cases}
      E = F_1 \oplus F_2 \\
      F_2 = F_1^\perp
    \end{cases}.
  \end{equation}
  De plus $\B_1$ et $\B_2$ sont des bases orthonormales respectives de $F_1$ et $F_2$.
\end{theo}
\begin{proof}
  On a déjà vu au chapitre~\ref{chap:dimensionfinie} (théorème~\ref{theo:theosuppdimfinie}) que $F_1$ et $F_2$ sont supplémentaires dans $E$. Montrons l'orthogonalité~:
  
  Soit $x \in F_1$ et $y \in F_2$. Il existe $(\alpha_i)_{1 \leq i \leq p} \in \R^p$ et $(\beta_i)_{p+1 \leq i \leq n}\in \R^{n-p}$ telles que $x = \sum_{i=1}^p \alpha_i b_i$ et $y = \sum_{i=p+1}^n \beta_i b_i$. Leur produit scalaire vaut
  \begin{align}
    \prodscal{x}{y} &= \prodscal{\sum_{i=1}^p \alpha_i b_i}{\sum_{j=p+1}^n \beta_j b_j}\\
    &= \sum_{i=1}^p \alpha_i \sum_{j=p+1}^n \beta_j \underbrace{\prodscal{b_i}{b_j}}_{=0 \text{~car~} i\neq j}\\
    &=0.
  \end{align}
  Donc $F_1 \perp F_2$, et puisque $E = F_1 \oplus F_2$, alors $F_2 = F_1^\perp$, d'après le corollaire~\ref{corth:supportho}.

  $\B_1$ et $\B_2$ vérifient toutes les deux~:
  \begin{equation}
    \forall (i,j) \in \intervalleentier{1}{n}^2 \quad \prodscal{b_i}{b_j}=\delta_{ij}.
  \end{equation}
  De plus $\B_1$ et $\B_2$ sont des bases de $F_1$ et $F_2$ (libres car extraites de $\B$ et génératrices par définition de $F_1$ et$F _2$). Alors $\B_1$ et $\B_2$ sont des bases orthonormales respectives de $F_1$ et $F_2$.
\end{proof}

\subsection{Construction d'une base orthonormale par le procédé de Gram-Schmidt}

\begin{theo}
  Soit $(E,\varphi)$ un espace vecoriel euclidien de dimension $n \geq 1$. Soit $\E=(e_1, \ldots, e_n)$ une base quelconque de $E$. Alors il existe de façon unique une famille de réels $(\alpha_{ij})_{1 \leq i<j \leq n}$ telle que la famille $(b_j)_{1 \leq j \leq n} \in E^n$ vérifiant
  \begin{equation} \Sigma
    \begin{cases}
      b_1 & = e_1 \\
      b_2 & = \alpha_{12}b_1 +e_2 \\
      \vdots & \\
      b_j & = \sum_{i=1}^{j-1} \alpha_{ij}b_i  + e_j \\
      \vdots & \\
      b_n & = \sum_{i=1}^{n-1} \alpha_{in}b_i  + e_n
    \end{cases}
  \end{equation}
  soit une base orthogonale de $E$. De plus
  \begin{equation}
    \forall (i,j) \in \intervalleentier{1}{n}^2 \ 1 \leq i < j \leq n \quad \alpha_{ij} = -\frac{\prodscal{b_i}{e_j}}{\norme{b_i}^2}.
  \end{equation}
  Il reste à normaliser cette base~: la famille $\left(\frac{b_j}{\norme{b_j}}\right)_{1 \leq j \leq n}$ est une base orthonormale de $E$.
\end{theo}
\begin{proof}[Démonstration par récurrence]
  Pour tout naturel $n$ non nul on pose $\P(n)$ ``pour tout espace vectoriel euclidien de dimension $n$, toute base $\E$, \ldots''

  \emph{Initialisation~:} Soit $E$ un espace vectoriel euclidien de dimension $1$. Soit $\E=(e_1)$ une base de $E$, et elle est orthogonale. $\P(1)$ est vraie.

  \emph{Hérédité~:} Soit $n \in \N^*$, et supposons $\P(n)$. Soit $E$ un espace vectoriel euclidien de dimension $n+1$, $\E=(e_1, \ldots, e_{n+1})$ une base de $\E$. Soient $\E'=\E\setminus\{e_{n+1}\}$ et $E'=\VectEngendre(\E')$ muni du produit scalaire induit. $E'$ est un espace vectoriel euclidien de dimension $n$. Par hypothèse de récurrence, il existe, de façon unique, une famille de réels $(\alpha_{ij})_{1 \leq i<j \leq n}$ telle que la famille $(b_j)_{1 \leq j \leq n} \in E^n$ vérifiant $\Sigma$ formant une base orthogonale de $E'$. 

  Il reste à prouver l'existence et l'unicité de $(\alpha_{i n+1})_{1 \leq i \leq n} \in \R^n$ telle que si $b_{n+1} = \sum_{i=1}^{n} \alpha_{i n+1}b_i  + e_{n+1}$ alors $(b_j)_{1 \leq j \leq n+1}$ est une base orthogonale de $E$.

  \emph{Unicité (Analyse)~:} Si les $\alpha_{i n+1}$ existent alors pour tout $j \in \intervalleentier{1}{n}$, on a
  \begin{align}
    0 &= \prodscal{b_{n+1}}{b_j} \\
    0 &= \prodscal{\sum_{i=1}^{n} \alpha_{i n+1}b_i  + e_{n+1}}{b_j}\\
    0 &= \sum_{i=1}^{n} \alpha_{i n+1} \prodscal{b_i}{b_j} + \prodscal{e_{n+1}}{b_j} && \text{linéarité à gauche}\\
    0 &= \alpha_{j n+1} \prodscal{b_j}{b_j} + \prodscal{e_{n+1}}{b_j} && (b_j)_{1 \leq j \leq n} \text{~est orthogonale}\\
    \alpha_{j n+1} &= -\frac{\prodscal{e_{n+1}}{b_j}}{\norme{b_j}^2}. && b_j \neq 0
  \end{align}
  On a montré l'unicité sous réserve d'existence et on a trouvé la seule valeur possible des $\alpha_{i n+1}$.

  \emph{Existence (Synthèse)~:} Pour tout $i \in \intervalleentier{1}{n}$, on pose $\alpha_{i n+1} = -\frac{\prodscal{e_{n+1}}{b_i}}{\norme{b_i}^2}$ et $b_{n+1}=\sum_{i=1}^{n} \alpha_{i n+1}b_i  + e_{n+1}$. 

  Montrons que $(b_j)_{1 \leq j \leq n+1}$ est une base orthogonale de $E$~:
  \begin{itemize}
  \item Pour tout $(i,j) \in \intervalleentier{1}{n}^2$ si $i \neq j$ alors $\prodscal{b_i}{b_j}=0$ car $(b_j)_{1 \leq j \leq n}$ est une base orthogonale de $E$. Ensuite
    \begin{align}
      \prodscal{b_i}{b_{n+1}} &= \prodscal{b_i}{\sum_{i=1}^{n} \alpha_{i n+1}b_i  + e_{n+1}} \\
      & = \sum_{i=1}^{n} \alpha_{i n+1} \prodscal{b_i}{b_j} + \prodscal{b_i}{e_{n+1}} \\
      &=\alpha_{i n+1} \norme{b_i}^2 + \prodscal{b_i}{e_{n+1}} \\
      &=0.
    \end{align}
    Alors $(b_j)_{1 \leq j \leq n+1}$  est une famille orthogonale. Montrons que c'est une base~:
    \begin{itemize}
    \item Si $i \leq n$, alors $b_i \neq 0$ car $(b_j)_{1 \leq j \leq n}$ est une base orthogonale de $E'$;
    \item Si on avait $b_{n+1}=0$, on aurait $e_{n+1}= -\sum_{i=1}^{n} \alpha_{i n+1}b_i$. C'est-à-dire $e_{n+1} \in \VectEngendre(b_1, \ldots, b_n) = \VectEngendre(e_1, \ldots, e_n)$. Or c'est impossible car $(e_1, \ldots, e_{n+1})$ est une base. Donc $b_{n+1} \neq 0$.
    \end{itemize}
    $(b_j)_{1 \leq j \leq n+1}$  est une famille orthogonale dont tous les vecteurs sont non nuls, elle est donc libre. Finalement c'est une base car $\Card(b_j)_{1 \leq j \leq n+1}=n+1=\Dim E$.
  \end{itemize}
  Alors $\P(n+1)$ est vraie.

  \emph{Conclusion~:} Le théorème de récurrence nous permet de conclure en écrivant que l'assertion $\P(n)$ est vraie pour tout naturel $n$ non nul.
\end{proof}

\emph{Remarques~:}
\begin{enumerate}
\item La démonstration par récurrence fournit un procédé pour déterminer la base $(b_1, \ldots, b_n)$, appelé procédé d'orthogonalisation ou procédé de Gram Schmidt;
\item La matrice de passage $\P_{\B,\E}$ de la base $\E$ à la base $\B$ est triangulaire supérieure avec que des $1$ sur la diagonale, idem pour $\P_{\E,\B}=\P_{\B,\E}^{-1}$;
\item Pour tout naturel $p \in \intervalleentier{1}{n}$, $\VectEngendre(e_1, \ldots, e_p)=\VectEngendre(b_1, \ldots, b_p)$.
\end{enumerate}

\subsection{Expressions analytiques dans une base orthonormée donnée}

Soit $(E,\varphi)$ un espace vectoriel euclidien de dimension $n \geq 1$. Soit $\B=(b_1, \ldots, b_n)$ une base orthonormée de $E$.

\subsubsection{Coordonnées d'un vecteur}

Soit $x \in E$, $\B$ une base de $E$. Il existe un unitque $n$-uplet $(x_i)_{1 \leq i \leq n}\in\R^n$ tel que $x = \sum_{i=1}^n x_ib_i$. Pour tout $j \in \intervalleentier{1}{n}$, on a
\begin{align}
  \prodscal{x}{b_j} &=\sum_{i=1}^n x_i\prodscal{b_i}{b_j} && \text{linéarité} \\
  &=x_j. && \B \text{~est une BON}
\end{align}

Alors pour tout $x \in E$, on a $x = \sum_{i=1}^n \prodscal{x}{b_i}b_i$.

\subsubsection{Produit scalaire}

Soient $x$ et $y$ des vecteurs de $E$. Il existe deux uniques familles de scalaire $(x_i)_{1 \leq i \leq n}\in\R^n$ et $(y_i)_{1 \leq i \leq n}\in\R^n$ telles que
\begin{align}
  x &= \sum_{i=1}^n x_ib_i \\
  y &= \sum_{i=1}^n y_ib_i.
\end{align}
Ainsi le produit scalaire s'écrit
\begin{align}
  \prodscal{x}{y} &=\prodscal{\sum_{i=1}^n x_ib_i}{\sum_{j=1}^n y_jb_j} \\
  &=\sum_{i=1}^n  \sum_{j=1}^n x_i y_j \prodscal{b_i}{b_j} && \text{bilinéarité} \\
  &=\sum_{i=1}^n  x_i y_i. && \B \text{~est une BON}
\end{align}

Finalement pour tout $(x,y) \in E^2$, on a $\prodscal{x}{y} = \sum_{i=1}^n \prodscal{x}{b_i}\prodscal{y}{b_i}$.

\emph{Remarque~:} Si on note $X=\Mat_{\B}(x) \in \Mnp{n}{1}{\R}$ et $Y=\Mat_{\B}(y) \in \Mnp{n}{1}{\R}$, alors $\prodscal{x}{y} = X^{\top}Y$.

\subsubsection{Norme euclidienne}

Soit $x \in E$, il existe un unique $n$-uplet de scalaires $(\lambda_i)_{i \in \intervalleentier{1}{n}}$ tel que $x = \sum_{i=1}^n x_i b_i$. On note $X = \Mat_{\B}(x)$. Alors
\begin{gather}
  \norme{x} = \sqrt{\sum_{i=1}^n x_i^2}=\sqrt{\sum_{i=1}^n \prodscal{x}{b_i}^2} \\
  \norme{x}^2 = X^\top X
\end{gather}

\subsection{Isomorphisme entre un espace euclidien et $\R^n$}

\begin{theo}
  Soit $(E,\varphi)$ un espace vectoriel euclidiende dimension $n \geq 1$. Soit $\B$ une base orthonormale de $E$. On note $\E_c$ la base canonique de $\R^n$ muni du produit scalaire canconique.

  Il existe un unique isomorphisme $u \in \Isom{E}{\R^n}$ tel que $u(\B)=\E_c$. De plus $u$ conserve le produit scalaire~:
  \begin{equation}
    \forall (x,y) \in E^2 \quad \prodscal{u(x)}{u(y)}=\varphi(x,y)=\prodscal{x}{y}.
  \end{equation}
\end{theo}
\begin{proof}
  $\B$ est une base de $E$ et $\Dim(E)=n$, alors il existe un unique $u \in \Lin{E}{\R^n}$ tel que $u(\B)=\E_c$. De plus $\E_c$ est une base de $\R^n$, donc $u$ est bijective, c'est un isomorphisme de $E$ sur $\R^n$.

Pour tout couple $(x,y) \in E^2$, il existe deux uniques $n$-uplets $(x_i)_{i \in \intervalleentier{1}{n}}\in \R^n$ et $(y_i)_{i \in \intervalleentier{1}{n}}\in \R^n$ tels que
\begin{equation}
  x = \sum_{i=1}^n x_i b_i, \quad y = \sum_{i=1}^n y_i b_i.
\end{equation}
Alors
\begin{align}
  \prodscal{u(x)}{u(y)} &=\prodscal{\sum_{i=1}^n x_i u(b_i)}{\sum_{j=1}^n y_j u(b_j)} && u \in \Lin{E}{\R^n} \\
  &=\prodscal{\sum_{i=1}^n x_i e_i}{\sum_{j=1}^n y_j e_j} && \text{définition de } u\\
  &=\sum_{i=1}^n \sum_{j=1}^n x_iy_j \prodscal{e_i}{e_j} && \text{bilinéarité} \\
  &=\sum_{i=1}^n x_i y_i && \E_c \text{~est une  BOG} \\
  &=\varphi(x,y)=\prodscal{x}{y}.
\end{align}
\end{proof}
\begin{theo}
  Soit $E$ un espace vectoriel euclidien de dimension $n \geq 1$. Soit $\B$ une base de $E$. Il existe un unique produit scalaire $\varphi$ sur $E$ tel que la base $\B$ soit une base orthonormale de $E$ pour le produit scalaire $\varphi$.
\end{theo}
\begin{proof}[Analyse \& Unicité]
  Supposons avoir construit le produit scalaire $\varphi$.

  On peut lui appliquer le théorème précédent, alors pour tout $(x,y) \in E^2$, on a $\varphi(x,y)=\prodscal{u(x)}{u(y)}$. De plus $u$ ne dépend que de la base $\B$ et pas du produit scalaire $\varphi$.

  Alors $\varphi$ est défini en fonction unique de $u$, et donc de $\B$.
\end{proof}
\begin{proof}[Synthèse \& Existence]
  On sait grâce au htéorème précédent qu'il existe un unique isomorphisme $u \in \Isom{E}{\R^n}$ tel que $u(\B)=\E_c$. On définit l'application
  \begin{equation}
    \fonction{\varphi}{E^2}{\R}{(x,y)}{\prodscal{u(x)}{u(y)}}.
  \end{equation}
  L'application $\varphi$ est symétrique puisque $\prodscal{.}{.}$ l'est. Pour tout réel $\lambda$ et tout triplet $(x,x',y) \in E^3$ on a
  \begin{align}
    \varphi(\lambda x+x',y) &= \prodscal{u(\lambda x+x')}{u(y)} \\
    &=\lambda \prodscal{u(x)}{u(y)}+ \prodscal{u(x')}{u(y)} && \text{linéarité de } u \text{~et~} \prodscal{.}{.} \\
    &=\lambda \varphi(x,y)+ \varphi(x,y').
  \end{align}
  $\varphi$ est linéaire à gauche. Elle est aussi linéaire à droite et symétrique.

  Soit $x \in E$, $\varphi(x,x)=\norme{u(x)}^2 \geq 0$ et $\varphi(x,x)=0 \iff u(x)=0 \iff x=0$ car $u$ est bijective (donc injective). L'application $\varphi$ est définie positive.

  Au final, $\varphi$ est un produit scalaire.

  Il reste à vérifier que $\B$ est une base orthonormale pour $\varphi$. Pour tout $(i,j) \in \intervalleentier{1}{n}^2$, on a
  \begin{align}
    \varphi(b_i,b_j) &= \prodscal{u(b_i)}{u(b_j)} \\
    &= \prodscal{e_i}{e_j} && u(\B)=\E_c\\
    &=\delta_{ij}. && \E_c \text{~est une BON}
  \end{align}
  Alors $\B$ est une base orthonormale.
\end{proof}

\section{Projecteurs orthogonaux}

Soit $(E, \varphi)$ un espace vectoriel euclidien.

\subsection{Notion de projecteur orthogonal}

Soit $p \in \Endo{E}$ un projecteur. Alors $E = \Image(p)\oplus \Ker(p)$. D'après le théorème~\ref{theo:orthetsupp}, il y a équivalence entre
\begin{enumerate}
\item $\Image(p) \perp \Ker(p)$;
\item $\Image(p) = \Ker(p)^\perp$;
\item $\Ker(p) = \Image(p)^\perp$.
\end{enumerate}
Lorsqu'une de ces conditions est vérifiée, on dit que $p$ est un projecteur orthogonal.

\emph{Vocabulaire~:} Soit $F$ un sous-espace vectoriel de $E$, $E = F \oplus F^\perp$. Le projecteur sur $F$ parallélement à $F^\perp$ et noté $p_F$ est appelé projecteur orthogonal de $E$ sur $F$.

\begin{theo}
  Soit $p$ ine application de $E$ dans $E$. Il y a équivalence entre les assertions suivantes~:
  \begin{enumerate}
  \item $p$ est un projecteur orthognal de $E$;
  \item $p \in \Endo{E}$, $p \circ p =p$, et pour tout $(x,y) \in E^2$ on a $\prodscal{x}{p(y)}=\prodscal{p(x)}{y}$.
  \end{enumerate}
\end{theo}
\begin{proof}
  $1 \implies 2$. Comme $p$ est un projecteur, alors $p \in \Endo{E}$ et $p \circ p =p$. Soient $(x,y) \in E^2$ alors
  \begin{align}
    \prodscal{p(x)}{y} &=  \prodscal{p(x)}{y-p(y)+p(y)} \\
    &= \prodscal{\underbrace{p(x)}_{\in \Image(p)}}{\underbrace{y-p(y)}_{\in \Ker(p)}}+\prodscal{p(x)}{p(y)} \\
    &=  \prodscal{p(x)}{p(y)} && \Image(p) \perp \Ker(p)
  \end{align}
  De la même manière, par symétrie des rôles de $x$ et $y$ on a $\prodscal{x}{p(y)}=\prodscal{p(x)}{p(y)}$. Donc $\prodscal{x}{p(y)}=\prodscal{p(x)}{y}$.

  $2 \implies 1$. $p \in \Endo{E}$ et $p \circ p =p$ impliquent que $p$ est un projecteur de $E$.

  Soit $x \in \Ker(p)$ et $y \in \Image(p)$. Alors
  \begin{align}
    \prodscal{x}{y} &= \prodscal{x}{p(y)} \\
    &=\prodscal{p(x)}{y} \\
    &=\prodscal{0}{y}=0.
  \end{align}
  Donc $\Image(p) \perp \Ker(p)$. Alors le projecteur $p$ est orthogonal.
\end{proof}

\subsection{Expression du projeté orthogonal}

Soit $F$ un sous-espace vectoriel de $E$. Soit $p_F$ le projecteur orthogonal sur $F$. On se donne une \emph{base orthogonale} $(f_1, \ldots, f_p)$ de $F$. On veut obtenir une expression de $p_F(x)$ pour tout vecteur $x$ de $E$.

Soit $x \in E$, alors $p_F(x) \in F$ et
\begin{align}
  p_F(x) &=\sum_{k=1}^n \prodscal{p_F(x)}{f_k}f_k \\
  &=\sum_{k=1}^n \prodscal{x}{p_F(f_k)}f_k \\
  &=\sum_{k=1}^n \prodscal{x}{f_k}f_k. && f_k \in F=\Image(p_F)
\end{align}

\emph{Remarque~:} Pour tout $x \in E$, $x = p_F(x) +(x-p_F(x)) = p_F(x) + p_{F^\perp}(x)$. Parfois c'est plus intéressant de calculer $p_{F^\perp}$, soit parce qu'il est de dimension plus petite, soit parce qu'on connait une de ses bases orthonormales.

\paragraph{Matrice d'un projecteur orthogonal}
Soit $F$ un sous-espace vectoriel de dimension $p \geq 1$. Soit $p_F$ un projecteur orthogonal sur $F$. Soit $\F=(f_1, \ldots, f_p)$ une base orthonormale de $F$ et $\B=(b_1, \ldots, b_n)$ une base orthonormale de $E$.

Soit $A=\Mat_\B(p_F)$. Calculons pour tout $i \in \intervalleentier{1}{n}$ $p_F(b_i)$~: Pour tout $j \in \intervalleentier{1}{n}$, on a
\begin{equation}
  p_F(b_j)=\sum_{k=1}^p \prodscal{b_j}{f_k}f_k.
\end{equation}

Il faut les coordonnées de $p_F(b_j)$ dans la base $\B$~: ce sont les $\prodscal{p_F(b_j)}{b_i}$.
\begin{align}
  a_{ij} &= \prodscal{p_F(b_j)}{b_i} \\
  &= \prodscal{\sum_{k=1}^p \prodscal{b_j}{f_k}f_k}{b_i} \\
  &= \sum_{k=1}^p \prodscal{b_j}{f_k} \prodscal{b_i}{f_k}.
\end{align}

La matrice $A$ est symétrique.

\subsection{Distance d'un point à un sous-espace vectoriel}

Soit $F$ un sous-espace vectoriel de $E$ de dimension $p \geq 1$. Soit $x \in E$, la partie $\enstq{d(x,y)}{y \in F}$ est une partie de $\R$ non vide (car $F$ est non vide) et est minorée par $0$.

Elle admet donc une borne inférieure noté $d(x,F)=\inf\enstq{d(x,y)}{y \in F}$. C'est la distance du point $x$ au sous-esapce vectoriel $F$.

\begin{theo}
  \begin{enumerate}
  \item Cette borne inférieure est en fait un minimum;
  \item pour tout vecteur $y \in F$, on a
    \begin{equation}
      d(x,F) = d(x,p_F(x)) = \norme{x-p_F(x)} \leq \norme{x-y};    
    \end{equation}
  \item pour tout vecteur $y \in F$, on a
    \begin{equation}
      d(x,F) = d(x,y) \implies y=p_F(x).
    \end{equation}
    Le minimum est atteint en $p_F(x)$ seulement.
  \end{enumerate}
\end{theo}
\begin{proof}
  Soit $y \in F$, alors
  \begin{align}
    \norme{x-y}^2 &=\norme{\underbrace{x-p_F(x)}_{\in F^\perp} + \underbrace{p_F(x)-y}_{\in F}}\\
    &=\norme{x-p_F(x)}^2 + \norme{p_F(x)-y}^2. && \text{Pythagore}
  \end{align}
  
  Donc $\norme{x-p_F(x)} \leq \norme{x-y}$, et
  \begin{equation}
    \norme{x-y}^2 = \norme{x-p_F(x)}^2 \iff  \norme{p_F(x)-y}=0.
  \end{equation}
  Cela signifie que $d(x,F)$ est un minimum et qu'il est atteint en $p_F(x)$ : $d(x,F)=\norme{x-p_F(x)}$
\end{proof}

\paragraph{Expressions de $d(x,F)$}

\begin{itemize}
\item
  \begin{gather}
    d(x,F) = \norme{x-p_F(x)} = \norme{p_{F^\perp}(x)} \\
    \norme{x}^2 = \norme{x-p_F(x)+p_F(x)}^2=\norme{x-p_F(x)}^2+\norme{p_F(x)}^2\\
    d(x,F) = \sqrt{\norme{x}^2-\norme{p_F(x)}^2}
  \end{gather}
\item Soit $\B=(b_1,\ldots, b_n)$ une base orthonormale de $E$ et $\F=(f_1,\ldots, f_p)$ une base orthonormale de $F$, alors
  \begin{equation}
    d(x,F) = \sqrt{\sum_{i=1}^n\prodscal{x}{b_i}^2-\sum_{k=1}^p\prodscal{x}{f_k}^2}.
  \end{equation}
\item Il est parfois utile de calculer $p_{F^\perp}(x)$ plutôt que $p_{F^\perp}(x)$.
\end{itemize}

\subsection{Formes linéaires et hyperplans d'un espace euclidien}

\subsubsection{Vecteur normal et forme linéaire}

Soit $(E,\varphi)$ un espace vectoriel euclidien. On note $E^*=\Lin{E}{\R}$ son dual, c'est-à-dire l'ensemble des formes linéaires de $E$. Pour tout $x \in E$, on pose
\begin{equation}
  \fonction{\varphi_x}{E}{\R}{y}{\prodscal{x}{y}}.
\end{equation}
Le produit scalaire est linéaire à droite donc $\varphi_x \in E^*$. Soit l'application
\begin{equation}
  \fonction{\psi}{E}{E^*}{x}{\varphi_x}.
\end{equation}
L'application $\psi$ est bien définie car pour tout vecteur $x \in E$, $\varphi_x \in E^*$. Montrons que $\psi$ est linéaire~: Soit $(x,x') \in E^2$ et $\lambda \in \R$, alors $\psi(\lambda x+x') = \varphi_{\lambda x+x'}$. Soit $y \in E$, alors
\begin{align}
  \psi(\lambda x+x')(y) &= \varphi_{\lambda x+x'}(y) \\
  &=\prodscal{\lambda x+x'}{y} \\
  &=\lambda \prodscal{x}{y} + \prodscal{x'}{y} && \text{linéarité à gauche}\\
  &=\lambda \varphi_x(y)+ \varphi_{x'}(y) \\
  &=\lambda \psi(x)(y)+\psi(x')(y)\\
  &=[\lambda \psi(x)+\psi(x')](y).
\end{align}
Comme l'égalité est vraie pour tout $y \in E$, on a $\psi(\lambda x+x')=\lambda \psi(x)+\psi(x')$, donc $\psi \in \Lin{E}{E^*}$.

Trouvons le noyau de $\psi$.

Soit $x \in E$, alors
\begin{align}
  x \in \Ker(\psi) &\iff \psi(x) = 0_{E^*} \\
  &\iff \forall y \in E \quad \psi(x)(y)=0_{\R} \\
  &\iff \forall y \in E \quad \varphi_x(y)=0_{\R} \\
  &\iff \forall y \in E \quad \prodscal{x}{y}=0_\R \\
  &\iff x=0_E.
\end{align}
Donc $\Ker \psi = \{0_E\}$. L'application $\psi$ est injective et $\Dim E= \Dim E^*$ donc $\psi$ est bijective. On en déduit le théorème suivant
\begin{theo}
  L'application
  \begin{equation}
    \fonction{\psi}{E}{E^*}{x}{\fonction{\varphi_x}{E}{\R}{y}{\prodscal{x}{y}}}
  \end{equation}
  est un isomorphisme de $\R$-espace vectoriels de $E$ dans $E^*$.
\end{theo}
\begin{corth}
  \begin{gather}
    \forall f \in E^* \ \exists! a \in E \quad f=\varphi_a \\
    \forall f \in E^* \ \exists! a \in E \ \forall x \in E \quad f(x)=\prodscal{a}{x}.
  \end{gather}
\end{corth}
\emph{Vocabulaire~:} Le vecteur $a$ est le \emph{vecteur normal} de la forme linéaire $f$.

\subsubsection{Applications aux hyperplans d'unespace vectoriel euclidien}

Soit $H$ un hyperplan de $E$. Il existe une forme linéaire $f$ non nulle telle que $H= \Ker(f)$.

Pour toute forme linéaire $g$,
\begin{equation}
  H = \Ker(g) \iff \exists \lambda \in \R^* \quad g=\lambda f.
\end{equation}

Soit $a$ le vecteur normal de $f$. $a$ est non nul parce que $f$ est non nulle. Soit un vecteur $x \in E$, alors
\begin{equation}
  f(x)= \prodscal{a}{x},
\end{equation}
donc pour tout $x \in E$, $(\lambda f)(x) = \prodscal{\lambda a}{x}$. Le vecteur normal de $\lambda f$ est $\lambda a$.

\begin{defdef}
  Soit $H$ un un hyperplan de $E$ et une forme linéaire $f$ non nulle telle que $H= \Ker(f)$. Soit $a$ le vecteur normal de $f$. $a$ est appelé \emph{un} vecteur normal  de $H$. Les autres sont les $\lambda a$, avec $(\lambda \neq 0)$.
\end{defdef}

\paragraph{Propriétés des vecteurs normaux}

Soit $H$ un hyperplan de $E$. Il existe une forme linéaire $f$ non nulle telle que $H= \Ker(f)$. Soit un vecteur $x \in E$, alors
\begin{align}
  x \in H &\iff f(x) = 0 \\
  &\iff \prodscal{x}{a}=0\\
  &\iff a \in H^\perp.
\end{align}
De plus $\Dim H^\perp =1$ et $a$ est non nul, donc c'est une base de $H^\perp$. Ainsi $\left(\frac{a}{\norme{a}}\right)$ est une base orthonormée de $H^\perp$.

\paragraph{Conséquences}
Soit un vecteur $x \in E$.  Alors
\begin{gather}
  p_{H^\perp}(x) = \prodscal{x}{\frac{a}{\norme{a}}}\frac{a}{\norme{a}} = \frac{\prodscal{x}{a}}{\prodscal{a}{a}}a \\
  p_H(x) = x-\frac{\prodscal{x}{a}}{\prodscal{a}{a}}a.
\end{gather}

La distance de $x$ à l'hyperplan $H$ vaut
\begin{equation}
  d(x,H) = \norme{p_{H^\perp}(x)} = \frac{\abs{\prodscal{x}{a}}}{\norme{a}}.
\end{equation}

\subsection{Orientation d'un hyperplan}

Soit $E$ un espace vectoriel de dimension finie non nulle $n$, supposée orienté. Soient $F$ et $G$ deux sous-espaces vectoriels orthogonaux et supplémentaires de $E$~:
\begin{equation}
  E = F \oplus G \quad F \perp G.
\end{equation}
Soient $\E_F=(e_1, \ldots, e_p)$ une base de $F$, $\E_G=(e_{p+1}, \ldots, e_n)$ une base de $G$ et $\E=\E_G \cup \E_F$ une base de $E$. On suppose que $G$ est orienté par $\E_G$, et que $\E_G$ est une base directe de $G$. On veut orienter $F$. Deux possibilités se présentent~:
\begin{itemize}
\item Si $\E$ est une base directe de $E$, on décide d'orienter $F$ par $\E_F$ ($\E_F$ est une base directe de $F$);
\item Si $\E$ est une base indirecte de $E$, on décide que $\E_F$ est une base indirecte de $F$. On dit que l'on a induit une orientation de $F$ par l'orientation de $G$.
\end{itemize}

En particulier si $F=H$, et un hyperplan $G=\VectEngendre(a)$ avce $a$ un vecteur normal de $H$. Le choix d'un vecteur normal donne une orientation pour $G$ qui induit une orientation de l'hyperplan.

\section{Produit mixte -- produit vectoriel}

Soit $(E,\varphi)$ un espace vectoriel euclidien orienté de dimension $n \in \N^*$.

\subsection{Produit mixte}

\subsubsection{Définition}

\begin{lemme}
  Soient $\B$ et $\B'$ deux bases orthonormales de $E$, alors $\Det_\B(\B') \in \{-1; 1\}$. De plus
  \begin{itemize}
  \item si $\B$ et $\B'$ ont la même orientation alors $\Det_\B(\B')=1$;
  \item sinon $\Det_\B(\B')=-1$.
  \end{itemize}
\end{lemme}
\begin{proof}
  Notons $\B=(b_1, \ldots, b_n)$, $\B'=(b'_1, \ldots, b'_n)$, $A=\P_{\B \B'}=\Mat_\B(\B')$. Alors pour tout $j \in \intervalleentier{1}{n}$, on a
  \begin{equation}
    b'_j = \sum_{i=1}^na_{ij}b_i.
  \end{equation}
  Comme ce sont des bases orthonormées, on a pour tout $(j,k) \in \intervalleentier{1}{n}^2$
  \begin{align}
    \prodscal{b_j}{b_k} &= \delta_{jk} \\
    \prodscal{b'_j}{b'_k} &= \delta_{jk} \label{eq:defBONdeltaij}.
  \end{align}
  Alors
  \begin{align}
    \delta_{jk} &= \prodscal{b'_j}{b'_k} && \text{BON}\\
    &=\prodscal{\sum_{i=1}^na_{ij}b_i}{\sum_{l=1}^na_{lk}b_l} && \text{eq.} \eqref{eq:defBONdeltaij}\\
    &=\sum_{i=1}^n\sum_{l=1}^n a_{ij}a_{kl} \prodscal{b_i}{b_l} && \text{bilinéarité}\\
    &=\sum_{i=1}^na_{ij}a_{il} && \text{BON}\\
    &=\sum_{i=1}^n (A^\top)_{ji}A_{ik} \\
    &=(A^\top A)_{jk}.
  \end{align}
  Alors $A^\top A=I_n$. Lorsqu'on passe au déterminant $\Det(A) \in \{-1; 1\}$. De plus $\B$ et $\B'$ ont la même orientation si et seulement si $\Det(A)>0$ c'est-à-dire $\Det(A)=1$.
\end{proof}
\begin{lemme}
  Soit $\X=(x_1, \ldots, x_n)$ une famille de $n$ vecteurs de $E$. Alors le réel $\Det_\B(\X)$ est indépendant du choix de la base $\B$ dans l'ensemble des bases orthonormées directes de $E$.
\end{lemme}
\begin{proof}
  Soient $\B$ et $\B'$ deux bases orthonormées directes de $E$. Alors
  \begin{equation}
    \Det_{\B'}(\X) = \Det_{\B'}(\B) \Det_{\B}(\X).
  \end{equation}
  Or $\Det_{\B'}(\B)=1$ puisqu'elles sont de même orientations. Alors $\Det_{\B'}(\X) = \Det_{\B}(\X)$.
\end{proof}
\begin{defdef}
  Soit $X=(x_1, \ldots, x_n)$ une famille de $n$ vecteurs de $E$. On appelle produit mixte de $\X$, et on note $\Det(x_1, \ldots, x_n)$ le réel $\Det_\B(\X)$, $\B$ étant une base orthonormée directe quelconque de $E$.
\end{defdef}
\begin{prop}
  L'application produit mixte
  \begin{equation}
    \fonction{\Det}{E^n}{\R}{x}{\Det(x)}
  \end{equation}
  est une forme $n$-linéaire alternée et antisymétrique.
\end{prop}
\begin{proof}
  Ce sont des propriétés de l'application $\Det_\B$, $\B$ étant une base orthonornée directe de $E$.
\end{proof}
\begin{prop}
  Soit $\X$ une famille de $n$ vecteurs de $E$, alors
  \begin{enumerate}
  \item $\Det(\X)=0$ si et seulement si $\X$ est liée;
  \item $\Det(\X)>0$ si et seulement si $\X$ est une base directe;
  \item $\Det(\X)<0$ si et seulement si $\X$ est une base indirecte.
  \end{enumerate}
\end{prop}
\begin{proof}
  Soit $\B$ une base orthonormée directe de $E$. Alors $\Det(\X)=\Det_\B(\X)$. Donc $\X$ est liée si et seulement si $\Det_\B(\X)=0$. Si $\X$ est libre alors $\Det_\B(\X) \neq 0$. $\Det_\B(\X)>0$ signifie que $\X$ et $\B$ ont la même orientation et comme $\B$ est directe alors $\X$ est directe. Idem pour indirecte.
\end{proof}
\begin{prop}
  Si on change l'orientation de $E$, le produit mixte est changé en son opposé.
\end{prop}
\begin{proof}
  Soient $\B_1$ et $\B_2$ deux bases orthonormées d'orientation contraires. On note $\Det_1$ le déterminant dans $\B_1$ et $\Det_2$ le déterminant dans $\B_2$. Pour toute famille $\X$ de $n$ vecteurs de $E$, on a
  \begin{align}
    \Det_1(\X) &= \Det_{\B_1}(\X) \\
    \Det_2(\X) &= \Det_{\B_2}(\X).
  \end{align}
  Donc
  \begin{equation}
    \Det_1(\X) = \Det_{\B_1}(\X) = \Det_{\B_1}(\B_2)\Det_{\B_2}(\X)=-\Det_2(\X).
  \end{equation}
\end{proof}

\subsubsection{Interprétation géométrique}

\paragraph{Cas où $n=2$}

Soit $(x,y)$ une famille libre de $E$ (c.-à-d.\ une base car $\Dim E=n=2$). $\Dr=\VectEngendre(x)$ et $\Dr^\perp$ sont deux droites vectorielles car $x\neq 0$ et $\Dim(E)=2$. Alors $E = \Dr \oplus \Dr^\perp$.

Il existe un unique couple $(y_\Dr, y_{\Dr^\perp}) \in \Dr \times \Dr^\perp$ tel que $y=y_\Dr +y_{\Dr^\perp}$. Le vecteur $x$ est non nul, alors on peut poser $i=\frac{x}{\norme{x}}$. On compléte $i$ en une base orthonormale directe $(i,j)$ de $E$.

\begin{align}
  \Det(x,y) &= \Det(x,y_\Dr +y_{\Dr^\perp}) \\
  &=\Det(x,y_\Dr) + \Det(x,y_{\Dr^\perp}) && \text{linéarité à droite} \\
  &=\Det(x,y_{\Dr^\perp}). && (y_\Dr,x) \text{~est liée}
\end{align}
On a $x=\norme{x}i$ et il existe $\epsilon \in \{-1;1\}$ tel que $y_{\Dr^\perp}=\epsilon\norme{y_{\Dr^\perp}}j$. Alors
\begin{align}
  \Det(x,y) &=\Det(\norme{x}i, \epsilon\norme{y_{\Dr^\perp}}j) \\
  &=\norme{x} \epsilon\norme{y_{\Dr^\perp}} \Det(i,j) && \text{bilinéarité} \\
  &=\norme{x} \epsilon\norme{y_{\Dr^\perp}}.
\end{align}
Donc $\abs{\Det(x,y)}=\norme{x}\norme{y_{\Dr^\perp}}$ et c'est l'aire du parallélogramme formé par $x$ et $y$. D'où la proposition suivante.

\begin{prop}
  Pour tout couple $(x,y) \in E^2$, si $(x,y)$ est une famille libre alors $\abs{\Det(x,y)}$ est égal à l'aire du parallèlogramme bâti sur $(x,y)$.
\end{prop}

\paragraph{Cas où $n=3$}

$E$ est un espace vectorel de dimension 3. Soit $(x,y,z)$ une famille libre de $E$ (donc une base). $\P=\VectEngendre(x,y)$ est un plan vectoriel et $\P^\perp$ est une droite vectoriel (car $\Dim E=3$). Alors $E =\P \oplus\P^\perp$.

Alors il existe un unique couple $(z_\P,z_{\P^\perp}) \in \P \times \P^\perp$ tel que $z=z_\P +z_{\P^\perp}$. Ainsi
\begin{equation}
  \Det(x,y,z) = \Det(x,y,z_\P) +\Det(x,y,z_{\P^\perp})) = \Det(x,y,z_{\P^\perp})),
\end{equation}
car $z_\P \in \P$.

La famille $(x,y,z)$ est libre donc $z \notin \P$ et donc $z_{\P^\perp} \neq 0$. On pose $k=\frac{z_{\P^\perp}}{\norme{z_{\P^\perp}}}$. On compléte $k$ en une base orthonormale $(i,j,k)$ de $E$. $(i,j)$ est une base orthonormale de $\P$. Alors on a
\begin{align}
  \Det(x,y,z) &= \Det(x,y,k\norme{z_{\P^\perp}}) \\
  &=\norme{z_{\P^\perp}}\Det(x,y,k) && \text{trilinéarité}\\
  &=\norme{z_{\P^\perp}}\Det_{(i,j,k)}(x,y,k)\\
  &=\norme{z_{\P^\perp}}\Det_{(i,j)}(x,y) && \norme{k}=1\\
  &=\norme{z_{\P^\perp}}\Det(x,y).
\end{align}
Alors
\begin{equation}
  \abs{\Det(x,y,z)} = \norme{z_{\P^\perp}}\abs{\Det(x,y)}.
\end{equation}
D'où la proposition suivante
\begin{prop}
  Pour tout couple $(x,y,z) \in E^3$, si $(x,y,z)$ est une famille libre alors $\abs{\Det(x,y,z)}$ est égal au volume du parallélépipéde bâti sur $(x,y,z)$.
\end{prop}

\subsection{Produit vectoriel, en dimension 3}

Soit $(E,\varphi)$ un espace euclidien orienté de dimension 3.

\subsubsection{Définition}

\begin{defdef}
  Soit $(x,y) \in E^2$. On appelle produit vectoriel de $x$ et $y$, dans cet ordre, et on note $x \wedge y$ le vecteur normal de la forme linéaire
  \begin{equation}
    \fonction{f}{E}{\R}{z}{\Det(x,y,z)}.
  \end{equation}
  Et c'est légitime, car~:
  \begin{itemize}
  \item le produit vectoriel est défini, car $E$ est orienté;
  \item $f$ est une forme linéaire, car $\Det$ est trilinéaire;
  \item $f$ admet un unique vecteur normal.
  \end{itemize}
  Ainsi $x \wedge y$ est l'unique vecteur $p \in E$ tel que pour tout $z \in E$ on ait $f(z)=\Det(x,y,z)=\prodscal{p}{z}$.
\end{defdef}
\begin{prop}
  Pour triplet de vecteurs $(x,y,z) \in E^3$ on a
  \begin{gather}
    \Det(x,y,z)=\prodscal{x\wedge y}{z}; \\
    \prodscal{y\wedge x}{z} = -\prodscal{x\wedge y}{z}; \\
    \prodscal{y\wedge z}{x} = \prodscal{z\wedge x}{y} =\prodscal{x\wedge y}{z}.
  \end{gather}
\end{prop}
\begin{proof}
  La première égalit'' est due à la définition du produit vectoriel. On montre la deuxième~:
  \begin{align}
    \prodscal{y\wedge x}{z} &=\Det(y,x,z)\\
    &=-\Det(x,y,z)&& \text{antisymétrie}\\
    &=-\prodscal{x\wedge y}{z}.
  \end{align}
  La deuxième~:
  \begin{align}
    \prodscal{y\wedge z}{x} &=\Det(y,z,x) \\
    &=\Det(x,y,z)\\
    &=\prodscal{x\wedge y}{z}.
  \end{align}
  Idem pour le reste.
\end{proof}
\begin{prop}[Effet d'un changement d'orientation de $E$]
  Soit $(x,y) \in E^2$. Si on change l'orientation de $E$, $x \wedge y$ est changé en son opposé.
\end{prop}
\begin{proof}
  Si on change l'orientation de $E$, l'application produit mixte $\Det$ est changée en son opposée. Donc la forme linéaire $f$ est changée en son opposée. Par conséquent le vecteur normal de $f$ est aussi changé en son opposé.
\end{proof}

\subsubsection{Propriétés du vecteur $x \wedge y$}

\begin{theo}
  Soit $(x,y) \in E^2$. Alors
  \begin{enumerate}
  \item $x \wedge y \perp x$ et $x \wedge y \perp y$;
  \item $x \wedge y = 0$ si et seulement si $(x,y)$ est liée;
  \item Si $(x,y)$ est libre, alors $(x,y,x \wedge y)$ est une base directe de $E$.
  \end{enumerate}
\end{theo}
\begin{proof}
  \begin{enumerate}
  \item En effet, on a $\prodscal{x \wedge y}{x}=\Det(x,y,x)=0$ donc $x \wedge y \perp x$. Idem pour $y$.
  \item $\impliedby$. Si $(x,y)$ est liée, alors $(x,y,x\wedge y)$ est liée aussi et donc
    \begin{align}
      \Det(x,y,x\wedge y) &= 0\\
      \prodscal{x\wedge y}{x\wedge y} &= 0 \\
      \norme{x\wedge y}^2=0,
    \end{align}
    et donc $x \wedge y=0$.

    $\implies$. Par contraposée. Supposons que $'x,y)$ est libre. Alors il existe $z \in E$ tel que $(x,y,z)$ est une base de $E$. Ainsi $\Det(x,y,z)=\prodscal{x\wedge y}{z}\neq 0$. Par conséquent $x\wedge y \neq 0$.
  \item Si $(x,y)$ est libre, $\Det(x,y,x \wedge y)=\norme{x \wedge y}^2 >0$ (car $x \wedge y \neq 0$). Donc $(x,y,x \wedge y)$ est une base directe de $E$.
  \end{enumerate}
\end{proof}

\subsubsection{Propriétés du produit vectoriel}

\begin{theo}
  L'application $\fonction{\wedge}{E^2}{E}{(x,y)}{x \wedge y}$ est bilinéaire et alterné (donc antisymétrique).
\end{theo}
\begin{proof}
  Soient $(x,x',y) \in E^3$ et $\lambda \in \R$. On note $p_1=(\lambda x+x')\wedge y$ et $p_2=\lambda x\wedge y+x'\wedge y$. Pour tout vecteur $z \in E$, on a
  \begin{align}
    \prodscal{p_1}{z} &=\prodscal{(\lambda x+x')\wedge y}{z} \\
    &=\Det(\lambda x +x',y,z) && \text{définition}\\
    &=\lambda \Det(x,y,z) +\Det(x',y,z) && \text{trilinéarité}\\
    &=\lambda \prodscal{x \wedge y}{z} + \prodscal{x' \wedge y}{z} \\
    &=\prodscal{\lambda x\wedge y +x'\wedge y}{z}\\
    &=\prodscal{p_2}{z}.
  \end{align}
  Alors $\prodscal{p_1-p_2}{z}=0$ et comme c'est vrai pour tout $z$ alors $p_1=p_2$. C'est la linéarité à gauche.

  Soient $(x,y) \in E^2$. Pour tout $z \in E$, on a
  \begin{align}
    \prodscal{y \wedge x}{z} &= \Det(y,x,z) \\
    &=-\Det(x,y,z) && \text{antisymétrie} \\
    &=-\prodscal{x \wedge y}{z}\\
    &=\prodscal{-x \wedge y}{z}.
  \end{align}
  Alors $\prodscal{y \wedge x+x \wedge y}{z}=0$ et comme c'est vrai pour tout $z \in E$ on a bien $x \wedge y = -y\wedge x$. C'est l'antisymétrie.

  La linéarité à gauche et l'antisymétrie nous donne la linéarité à droite.
\end{proof}

\subsubsection{Coordonnées de $x \wedge y$ dans une base orthonormale directe}

Soit $\B=(b_1,b_2,b_3)$ une base orthonormée directe de $E$. Soient $x$ et $y$ dans $E$ de coordonnées respectives $(x_1,x_2,x_3)$ et $(y_1,y_2,y_3)$ dans la base $\B$. Alors
\begin{equation}
  x \wedge y = \sum_{i=1}^3 \prodscal{x \wedge y}{b_i}b_i.
\end{equation}
Car $\B$ est une base orthonormée. Alors on a
\begin{align}
  \prodscal{x \wedge y}{b_1} &=\Det(x,y,b_1) \\
  &=\Det_{b_1,b_2,b_3}(x,y,b_1)\\
  &=\begin{vmatrix} x_2 & y_2 \\ x_3 & y_3 \end{vmatrix}.
\end{align}
De la même manière on trouve
\begin{equation}
  \prodscal{x \wedge y}{b_2} = - \begin{vmatrix} x_1 & y_1 \\ x_3 & y_3 \end{vmatrix} \qquad \prodscal{x \wedge y}{b_3} =  \begin{vmatrix} x_1 & y_1 \\ x_2 & y_2 \end{vmatrix}.
\end{equation}
D'où la proposition suivante
\begin{prop}
  Avec les notations précédentes, les coordonnées de $x \wedge y$ dans la base orthonormée directe $\B$ sont $\left(\begin{vmatrix} x_2 & y_2 \\ x_3 & y_3 \end{vmatrix}, - \begin{vmatrix} x_1 & y_1 \\ x_3 & y_3 \end{vmatrix},  \begin{vmatrix} x_1 & y_1 \\ x_2 & y_2 \end{vmatrix}\right)$.
\end{prop}

\subsubsection{Propriétés relatives aux bases orthonormées directes}

\begin{prop}
  Soit $(i,j,k)$ une base orthonormée directe. Alors $i \wedge j=k$, $j \wedge k = i$, et $k \wedge i =j$.
\end{prop}
\begin{proof}
  La base $(i,j,k)$ est orthonormée donc
  \begin{align}
    i \wedge j &= \prodscal{i \wedge j}{i}i + \prodscal{i \wedge j}{j}j + \prodscal{i \wedge j}{k}k \\
    &=\prodscal{i \wedge j}{k}k\\
    &=\Det(i,j,k)k\\
    &=\Det_{(i,j,k)}(i,j,k)k\\
    &=k.
  \end{align}
  Idem pour les autres.
\end{proof}
\begin{prop}
  Soient $x$ et $y$ deux vecteurs de $E$ unitaires et orthogonaux. Alors $(x,y,x \wedge y)$ est ue base orthonormée directe.
\end{prop}
\begin{proof}
  La famille $(x,y)$ est orthogonale et $x$ et $y$ sont non nuls. Donc Elle est libre. On sait déjà que $(x,y,x \wedge y)$ est une base directe de $E$. Par hypothèse $x \perp y$ et d'après les propriétés on a $x \perp x \wedge y$ et $y \perp x \wedge y$. Alors $(x,y,x \wedge y)$ est une base orthogonale. Ainsi $\left(x,y,\frac{x \wedge y}{\norme{x \wedge y}}\right)$ est une base orthonormée directe.
\end{proof}
\begin{prop}
  Soit $(x,y) \in E^2$. Si $x \perp y$ alors $\norme{x \wedge y}=\norme{x}\cdot\norme{y}$.
\end{prop}
\begin{proof}
  Deux cas se présentent. Si $x$ est nul ou si $y$ est nul alors $x \wedge y$ est nul. L'égalité est vraie.

  Sinon, c'est-à-dire si $x$ est non nul et si $y$ est non nul, alors $\frac{x}{\norme{x}}$ et $\frac{y}{\norme{y}}$ sont unitaires et orthogonaux. En appliquant la proposition précédente $\left(\frac{x}{\norme{x}}, \frac{y}{\norme{y}}, \frac{x}{\norme{x}}\wedge \frac{y}{\norme{y}}\right)$ est une base orthonormée directe de $E$. En particulier $\frac{x}{\norme{x}}\wedge \frac{y}{\norme{y}}$ est unitaire et donc
  \begin{equation}
    1 = \frac{\norme{x \wedge y}}{\norme{x}\norme{y}}.
  \end{equation}
  Alors $\norme{x \wedge y}=\norme{x}\cdot\norme{y}$.
\end{proof}

\subsubsection{Double produit vectoriel}

\begin{theo}
  Pour tout triplet $(x,y,z) \in E^3$ on a
  \begin{equation}
    (x \wedge y)\wedge z = \prodscal{x}{z}y-\prodscal{y}{z}x.
  \end{equation}
\end{theo}
\begin{proof}
  Voir le chapitre~\ref{chap:geomEspace}.
\end{proof}
\emph{Remarque~:} Pour tout triplet $(x,y,z) \in E^3$ on a
\begin{equation}
  x \wedge (y \wedge z) = \prodscal{x}{z}y-\prodscal{x}{y}z.
\end{equation}
