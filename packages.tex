%pour résoudre le problème "No room for a new dimen" et on le place en premier
\usepackage{etex}
%packages fondamentaux
\usepackage[utf8]{inputenc}
\usepackage[T1]{fontenc}
\usepackage[french]{babel}
%pour les symboles mathématiques
\usepackage{amsmath}
\setlength{\mathindent}{1.5cm}
\usepackage{amsthm}
\usepackage{amssymb}
\usepackage{amsfonts}
\usepackage{mathrsfs}
\usepackage{latexsym}
%Pour faire des maths en francais
%\usepackage{frmath}
%Pour utiliser les tableaux
\usepackage{array}
%fonte vectorielle
\usepackage{lmodern}
%Pour écrire de la physique avec les unités
%\usepackage{siunitx}
%Pour inclure des images avec \includegraphics[scale=]{}
\usepackage{graphics}
%Pour inclure des fichiers .ps ou .eps
\usepackage{pstricks-add}
%Pour dessiner avec Tikz
\usepackage{tikz, pgf}
\usetikzlibrary{arrows,calc,backgrounds}
\usepackage{tkz-tab}
%Pour le trace geogebra
\usepackage{pgfplots}
\pgfplotsset{compat=1.15}
% pour le symbole \danger
\usepackage{fourier-orns}
%pour utiliser les couleurs
\usepackage{color}
%pour utiliser "draft" (brouillon)
%\usepackage{draftcopy}
%pour renvoyer les \es à la fin
\usepackage{endnotes}
%pour les encadrements et les boites
\usepackage{fancybox}
%pour les boucles de programmation en \LaTeX
\usepackage{ifthen}
%Pour le placement de flottants avec FloatBarrier
\usepackage{placeins}
%Pour changer les titres de section
%\makeatletter
%   \renewcommand\section{\@startsection
%   {section}{2}{0mm}
%   {-\baselineskip}{0.5\baselineskip}
%   {\FloatBarrier\normalfont\Large\bfseries}}
%\makeatother
%Calculs arithmétique
%\usepackage{calc}
%Lettrines
\usepackage{lettrine}
%pour le symbole double crochet \rrbracket \llbracket
\usepackage{stmaryrd}
%pour les minitoc{}
\usepackage{minitoc}
\mtcselectlanguage{french}
%pour les tableaux de variations
\usepackage{variations}
%Pour la taille de la page
\usepackage{geometry}
\geometry{%
    a4paper,
    total={170mm,257mm},
    left=20mm,
    top=20mm,
}
%% entêtes et pieds de pages
%Entêtes et pieds de pages
\usepackage{fancyhdr}
%pour inclure des pdf
\usepackage[final]{pdfpages}
%texte en latin
%\usepackage{lipsum}
%pour creer un sommaire
\usepackage{shorttoc}
%pour écrire des arc géométriques
\usepackage{yhmath}
%pour l'index
%\usepackage{makeidx}
%\makeindex
%pour les boites autour des equations
\usepackage{empheq}
%paquet pour les liens hypertextes
\usepackage[colorlinks=true,%
linkcolor=black,%
urlcolor=black,%
citecolor=black,%
pdftitle={Mathématiques},%
pdfauthor={Florian Maillard},%
breaklinks = true%
]{hyperref}
%pour les tableaux
\usepackage{tabularx}
\usepackage{multicol}
\usepackage{caption}
\usepackage{subcaption}
\usepackage{booktabs}
