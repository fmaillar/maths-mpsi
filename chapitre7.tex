\chapter{Coniques}
\minitoc
\minilof
\minilot
\section{Définition monofocale et équation polaire}
Soient \(F\) un point du plan, \(\Dr\) une droite ne passant pas par \(F\), \(e\) un réel strictement positif.
\subsection{Définition monofocale}
\begin{defdef}
   On appelle conique de foyer \(F\), de directrice \(\Dr\) et d'excentricité \(e\) l'ensemble des points \(M\) du plan tels que
  \begin{equation}
    MF=e MH,
  \end{equation}
  où \(H\) est le projeté orthogonal de \(M\) sur \(\Dr\).
\end{defdef}
\begin{defdef}
  On appelle axe focal de la conique \(\con{e}\) la droite \(\Delta\) perpendiculaire à la droite \(\Dr\) passant par \(F\).
\end{defdef}
\begin{prop}
  La conique \(\con{e}\) est symétrique par rapport à l'axe focal \(\Delta\).
\end{prop}
\begin{proof}
  On note \(K\) le projeté orthogonal de \(F\) sur \(\Dr\) : \(K=D \cap \Delta\). Soit \(M \in \con{e}\) et \(H\) son projeté orthogonal sur \(\Dr\). Soit \(M'\) le symétrique de \(M\) par rapport à \(\Delta\), \(H'\) le projeté orthogonal de \(M'\) sur \(\Dr\) (c'est aussi le symétrique de \(H\) par rapport à \(\Delta\)). Alors
\begin{equation}
  \frac{M'F}{M'H'} = \frac{MF}{MH} = e,
\end{equation}

donc le point \(M'\) appartient à la conique \(\con{e}\).
\end{proof}
Si on considère l'application \(\fonction{\varphi}{P\setminus\{\Dr \cup F\}}{\Rplusetoile}{M}{\frac{MF}{MH}}\), où \(H\) est le projeté orthogonal de \(M\) sur \(\Dr\), les coniques \(\con{e}\) de directrice \(\Dr\) sont les lignes de niveau de \(\varphi\). On cherche les points d'intersection entre \(\con{e}\) et l'axe \(\Delta\). On note \(K\) le projeté orthogonal de \(F\) sur \(\Dr\). On place l'origine des abscisses en \(K\) et on oriente \(\Delta\) de \(K\) vers \(F\), on note \(d=KF\).

Soit M sur \(\Delta\) le point d'abscisse \(x\). Le projeté orthogonal de M sur la directrice \(\Dr\) est le point \(K\), puisque \(\Dr\) et \(\Delta\) sont orthogonales et se croisent en \(K\). On a alors la suite d'équivalence suivante~:
\begin{align}
  M \in \con{e} &\iff MF = e MK \\
  &\iff MF^2=e^2 MK^2 \\
  &\iff (x-d)^2=e^2 x^2\\
  &\iff x^2+d^2-2xd=e^2 x^2\\
  &\iff x^2(1-e^2)-2xd+d^2=0.
\end{align}
On distingue deux cas~:
\begin{itemize}
\item si \(e=1\), alors le point M est sur la conique si et seulement si \(x=\frac{d}{2}\). Il y a donc un unique point d'intersection entre la conique \(\con{1}\) est son axe focal \(\Delta\), c'est le milieu de \([KF]\). Cette conique est une parabole;
\item sinon comme \(1-e^2 \neq 0\), on peut calculer le discriminant de ce trinôme qui vaut \((2de)^2\), alors les deux racines réelles du trinôme sont \(x_1=\frac{d}{1+e}\) et \(x_2=\frac{d}{1-e}\). Soient les points d'intersections \(A_1\) et \(A_2\) de \(\con{e}\) et \(\Delta\), d'abscisse respectives \(x_1\) et \(x_2\). Alors \(\vect{KA}=\frac{1}{1+e} \vect{KF}\) et \(\vect{KA'}=\frac{1}{1-e} \vect{KF}\). On peut distinguer deux sous-cas~:
  \begin{itemize}
  \item si \(e<1\) alors \(x_1\) et \(x_2\) sont positifs et les deux points \(A\) et \(A'\) sont du même côté de \(\Dr\) que \(F\). On dira que \(\con{e}\) est une ellipse;
  \item si \(e>1\) alors \(A\) est du même côté que \(F\) mais \(A'\) est de l'autre, dans ce cas là on dira que \(\con{e}\) est une hyperbole.
  \end{itemize}
\end{itemize}

\subsection{Équation polaire de \(\con{e}\)}
On munit le plan de la base orthonormale \((\vi,\vj)\) où \(\vi=\frac{\vect{FK}}{\norme{\vect{FK}}}\) et où \(\vj\) est tel que \((\vi,\vj)=\frac{\pi}{2}\) et \(\norme{\vj}=1\). On se place dans le repère \(\Rep=(F,\vi,\vj)\). Notons \(d=FK\), une équation de \(\Dr\) dans \(\Rep\) est \(x=d\) et son équation polaire dans \(\Rep\) est \(\rho=\frac{d}{\cos \theta}\). Soient un point \(M\) du plan et son système de coordonnées polaires \((\rho,\theta)\) : \(\vect{FM}=\rho\vect{u}_\theta\), alors~:
\begin{align}
  M \in \con{e} &\iff MF = e MH \\
  &\iff \abs{\rho}=e\abs{d-\rho \cos\theta}\\
  &\iff \begin{cases}\rho = ed-e\rho\cos\theta \\ \text{ou}\\ \rho = e\rho\cos\theta-ed \end{cases} \\
  &\iff  \begin{cases}\rho = \frac{ed}{1+e\cos\theta} \\ \text{ou}\\ \rho = \frac{-ed}{1-e\cos\theta} \end{cases}.
\end{align}
On obtient deux équations polaires, mais ces deux équations sont les mêmes~: en effet, si \((\rho,\theta)\) est un s.c.p.\ de \(M\) alors \(\rho=\frac{ed}{1+e\cos\theta}\) et donc en prenant l'opposé \(-\rho=\frac{-ed}{1-e\cos(\theta+\pi)}\) et donc \((-\rho,\theta+\pi)\), étant un deuxième s.c.p.\ de \(M\) dans \(\Rep\) vérifie la deuxième équation. Alors~:
\begin{equation}
  M(\rho,\theta) \in \con{e} \iff \rho=\frac{ed}{1+e\cos\theta}.
\end{equation}
\(\rho=\frac{ed}{1+e\cos\theta}\) est une équation polaire de la conique \(\con{e}\).
\begin{enumerate}
\item Si on oriente la droite (FK) dans l'autre sens on obtient une équation du même type avec un signe négatif;
\item l'équation est la même pour toutes les valeurs de l'excentricité, mais le domaine dans lequel varie \(\theta\) dépend de l'excentricité.
\end{enumerate}

\section{Équations réduites}
\label{sec:eqred}
Soit \(\con{e}\) la conique de foyer \(F\), de directrice \(\Dr\), d'excentricité \(e\). On note \(d=d(F,\Dr)>0\). On pose \(p=ed\), \(p\) est appelé le paramètre de la conique \(\con{e}\). On se place dans le ROND \((F,\vi,\vj)\) avec \(\vi=\frac{\vect{KF}}{KF}\). Dans ce repère, \(F(0,0), K(-d,0)\) et soit \(M(x,y)\) un point du plan, \(H\) le projeté orthogonal de \(M\) sur \(\Dr\), \(H(-d,y)\). Alors~:
\begin{align}
  M \in \con{e} &\iff MF = e MH \\
  &\iff MF^2 = e^2 MH^2\\
  &\iff x^2+y^2=e^2(x+d)^2.
\end{align}

\subsection{Cas de la parabole}
Soit \(S\) l'unique point d'intersection entre \(\con{1}\) et \(\Delta\). Le point \(S\) est le sommet de la parabole, c'est aussi le milieu de \([KF]\), alors \(S\left(\frac{-d}{2}\right)\). On adopte un nouveau repère~: le ROND \((S,\vi,\vj)\), si \(M\) a pour coordonnées \((x,y)\) dans \((F,\vi,\vj)\) et \((X,Y)\) dans \((S,\vi,\vj)\) alors \(X=x+\frac{d}{2}\) et \(Y=y\). Ainsi~:
\begin{align}
  M(X,Y) \in \con{1} &\iff \left(X-\frac{d}{2}\right)^2+Y^2=\left(X+\frac{d}{2}\right)^2\\
  &\iff \left(X-\frac{d}{2}\right)^2 - \left(X+\frac{d}{2}\right)^2+Y^2=0\\
  &\iff Y^2=2dX=2pX,
\end{align}
puisque \(p=ed=d\).
\begin{theo}
Soit \(\P\) une parabole de foyer \(F\) de directrice \(\Dr\). Il existe un repère orthonormal \((S,\vi,\vj)\) dans lequel \(\P\) a pour équation \(Y^2=2pX\), c'est ce qu'on appelle l'équation réduite de la parabole \(\P\).

Réciproquement, si \(p\) est un réel strictement positif, l'ensemble des points représentés par l'équation cartésienne \(Y^2=2pX\) dans un repère orthonormal \(\rond\) est une parabole de foyer \(F\left(\frac{p}{2},0\right)\), de directrice \(\Dr\) et d'équation \(X=\frac{-p}{2}\). Le nombre \(p\) est le paramètre de la parabole. Un paramétrage de la parabole est~:
  \begin{equation}
    \forall t \in \R \quad
    \begin{cases}
      x(t)=\frac{t^2}{2p} \\
      y(t)=t
    \end{cases}.
  \end{equation}
\end{theo}

Plusieurs paraboles sont représentées sur la figure~\ref{fig:parabole}.

\begin{figure}
    \begin{subfigure}{.5\textwidth}
      \centering
      % include first image
      \includegraphics[scale=.45]{Tracé_parabole_0.5.png}  
      \caption{$p=0.5$}
      \label{fig:parabole1}
    \end{subfigure}
    \begin{subfigure}{.5\textwidth}
      \centering
      % include first image
      \includegraphics[scale=.45]{Tracé_parabole_0.75.png}  
      \caption{$p=0.75$}
      \label{fig:parabole2}
    \end{subfigure}
    \newline
    \begin{subfigure}{.5\textwidth}
      \centering
      % include first image
      \includegraphics[scale=.45]{Tracé_parabole_1.png}  
      \caption{$p=1$}
      \label{fig:parabole3}
    \end{subfigure}    
    \begin{subfigure}{.5\textwidth}
      \centering
      % include first image
      \includegraphics[scale=.45]{Tracé_parabole_1.5.png}  
      \caption{$p=1.5$}
      \label{fig:parabole4}
    \end{subfigure}
    \newline
    \begin{subfigure}{.5\textwidth}
      \centering
      % include first image
      \includegraphics[scale=.45]{Tracé_parabole_1.75.png}  
      \caption{$p=1.75$}
      \label{fig:parabole5}
    \end{subfigure}
    \begin{subfigure}{.5\textwidth}
      \centering
      % include first image
      \includegraphics[scale=.45]{Tracé_parabole_2.png}  
      \caption{$p=2$}
      \label{fig:parabole6}
    \end{subfigure}
    \newline
    \begin{subfigure}{.5\textwidth}
      \centering
      % include first image
      \includegraphics[scale=.45]{Tracé_parabole_2.5.png}  
      \caption{$p=2.5$}
      \label{fig:parabole7}
    \end{subfigure}    
    \begin{subfigure}{.5\textwidth}
      \centering
      % include first image
      \includegraphics[scale=.45]{Tracé_parabole_3.png}  
      \caption{$p=3$}
      \label{fig:parabole8}
    \end{subfigure}
%  \centering
%  \includegraphics[width=\textwidth, scale=1]{parabole.png}
  \caption{Représentations graphiques de plusieurs paraboles}
  \label{fig:parabole}
\end{figure}


\subsection{Cas des coniques à centre (\(e\neq 1\))}
Dans ce cas la conique \(\con{e}\) est son axe focal \(\Delta\) on deux points d'intersection. Dans le repère \((K,\vi,\vj)\), ce sont les points \(A(\frac{d}{1+e},0)\) et \(A'(\frac{d}{1-e},0)\). Ces deux points sont les sommets de la conique \(\con{e}\). On définit le milieu O de [AA'] et on se place dans le repère orthonormal \(\rond\). Alors \(\vect{KO}=\frac{1}{2}\left(\frac{d}{1+e}+\frac{d}{1-e}\right)\vi=\frac{d}{1-e^2}\vi\). Si M a pour coordonnées (x,y) dans \((K,\vi,\vj)\) et (X,Y) dans \(\rond\) alors puisque \(\vect{OM}=\vect{OK}+\vect{KM}=\vect{KM}-\frac{d}{1-e^2}\vi\) donc \(X=x-\frac{d}{1-e^2}\) et \(Y=y\). Alors on peut noter les coordonnées dans le nouveau repère~:
\begin{equation}
  K\left(-\frac{d}{1-e^2},0\right) \ F\left(-\frac{de^2}{1-e^2},0\right) \ A\left(-\frac{de}{1-e^2},0\right) \ A'\left(\frac{de}{1-e^2},0\right)
\end{equation}
Soit un point \(M(x,y)\) et son projeté sur la directrice \(H\left(-\frac{d}{1-e^2},Y\right)\). Alors~:
\begin{align}
  M \in \con{e} &\iff MF^2=e^2 MH^2\\
  &\iff \left(X+\frac{de^2}{1-e^2}\right)^2+Y^2=e^2 \left(X+\frac{d}{1-e^2}\right)^2\\
  &\iff X^2+\left(\frac{de^2}{1-e^2}\right)^2+Y^2=e^2 X^2+e^2\left(\frac{d}{1-e^2}\right)^2\\
&\iff X^2(1-e^2)+Y^2=\frac{d^2e^2}{1-e^2}=\frac{p^2}{1-e^2}.
\end{align}
Alors finalement, on distingue deux sous-cas.

\subsubsection{Cas des hyperboles (\(e>1\))}
Si on pose \(a=\frac{p}{e^2-1}\) et \(b=\frac{p}{\sqrt{e^2-1}}\), alors dans le repère \(\rond\) l'équation devient~:
\begin{equation}
  M(X,Y) \in \con{e} \iff \frac{X^2}{a^2}-\frac{Y^2}{b^2}=1.
\end{equation}
\begin{theo}
  Il existe un repère orthonormal direct dans lequel l'hyperbole admet pour équation cartésienne~:
  \begin{equation}
    \frac{X^2}{a^2}-\frac{Y^2}{b^2}=1,
  \end{equation}
avec \(a\) et \(b\) des réels strictement positifs.
\end{theo}
\begin{prop}
  \begin{enumerate}
  \item \(p=\frac{b^2}{a}\) et \(e^2-1=\frac{b^2}{a^2}\);
  \item si on pose \(c=\sqrt{a^2+b^2}\), alors \(e=\frac{c}{a} \ d=\frac{b^2}{c}\);
  \item le foyer F a pour coordonnées \((c,0)\) et la directrice \(\Dr\) a pour équation \(X=\frac{a^2}{c}\)
  \end{enumerate}
\end{prop}
\begin{proof}
  \begin{enumerate}
  \item \(\frac{b^2}{a}=\frac{\frac{p^2}{e^2-1}}{\frac{p}{e^2-1}}=p\) et \(\frac{b^2}{a}=\frac{\frac{p^2}{e^2-1}}{\frac{p^2}{(e^2-1)^2}}=e^2-1\);
  \item \(c=a\sqrt{1+\left(\frac{b}{a}\right)^2}=a\sqrt{1+e^2-1}=ae\) puisque \(e>0\) et \(d=\frac{p}{e}=\frac{b^2}{a} \frac{a}{c}=\frac{b^2}{c}\);
  \item à la base l'abscisse de F est \(-\frac{de^2}{1-e^2}=\frac{-pe}{1-e^2}=ae=c\) et l'équation de la courbe est \(X=-\frac{d}{1-e^2}=-\frac{p/e}{1-e^2}=\frac{a}{e}=\frac{a^2}{c}\)
  \end{enumerate}
\end{proof}
\begin{theo}[Théorème réciproque]
  Soient deux réels strictement positifs \(a\) et \(b\) et \(c=\sqrt{a^2+b^2}\), un repère orthonormal \(R=\rond\). La courbe d'équation cartésienne dans \(R\) \(\frac{X^2}{a^2}-\frac{Y^2}{b^2}=1\) est une hyperbole de foyer \(F(-c,0)\) et de directrice \(\Dr\) d'équation \(X=\frac{a^2}{c}\) d'excentricité \(e=\frac{c}{a}\) et de paramètre \(p=\frac{b^2}{a}\).
\end{theo}
L'hyperbole \(\H\) est aussi l'hyperbole de foyer \(F'(-c,0)\) de directrice \(\Dr ':X=-\frac{a^2}{c}\). Les intersections de \(\H\) avec l'axe focal sont appelés les sommets de l'hyperbole. Ce sont les points \(A(a,0)\) et \(A'(-a,0)\).
\begin{prop}
  L'hyperbole \(\H\) est la réunion des deux courbes paramétrées suivantes~:
  \begin{equation}
    \Gamma_1 : x(t)=a\cosh(t), y(t)=b\sinh(t) \quad \Gamma_2 : x(t)=-a\cosh(t), y(t)=b\sinh(t)
  \end{equation}
\end{prop}
\begin{proof}
  \begin{itemize}
  \item On démontre dans ce premier point l'inclusion \(\Gamma_1 \subset \H\)~: Soit un point \(M(t)(x(t),y(t)) \in \Gamma_1\) alors pour chaque instant \(t \in \R\), \(\frac{x(t)^2}{a^2}-\frac{y(t)^2}{b^2}=\cosh^2(t)-\sinh^2(t)=1\) donc le point \(M(t)\) est sur l'hyperbole.
  \item On démontre de la même manière l'inclusion  \(\Gamma_2 \subset \H\)
  \item Démontrons maintenant l'inclusion inverse \(\H \in \Gamma_1\cup\Gamma_2\). Soit un point \(M(x,y)\) sur l'hyperbole, on pose \(t=\argsh\left(\frac{y}{b}\right)\) alors \(y=b\sinh(t)\) et puisque le point est sur l'hyperbole on écrit~: \(\frac{x^2}{a^2}=1+\frac{y^2}{b^2}=\cosh(t)\) donc \(x^2=(a\cosh(t))^2\). Si \(x\geqslant 0\) alors \(x=a\cosh(t)\) et le point M est sur \(\Gamma_1\) sinon \(x=-a\cosh(t)\) et il est sur \(\Gamma_2\). Dans tous les cas le point M est dans l'union \(\Gamma_1 \cup \Gamma_2\).
  \end{itemize}
Finalement \(\H=\Gamma_1 \cup \Gamma_2\).
\end{proof}
\begin{prop}
  L'hyperbole \(\H\) admet pour asymptote les droites d'équations \(y\frac{b}{a}x\) et \(y\frac{-b}{a}x\).
\end{prop}
\begin{proof}
  Étudions la courbe paramétrée \(\Gamma_1\)~: lorsque \(t \to +\infty\) alors \(x\) et \(y\) deviennent infinis, il y a donc une branche infinie. Le rapport \(\frac{y(t)}{x(t)}=\frac{b}{a} \tanh(t)\) tend vers \(\frac{b}{a}\). Ainsi la courbe admet une direction asymptotique de pente \(\frac{b}{a}\) et la différence \(y(t) -\frac{b}{a}x(t)=b(\sinh t -\cosh t)=-b\e^{-t}\) tend vers 0. Alors la courbe admet bien la droite d'équation \(y=\frac{b}{a}x\) pour asymptote. On raisonne de la même manière en \(t\to -\infty\) et pour la courbe \(\Gamma_2\).
\end{proof}
Si \(a=b\), on dit que l'hyperbole est équilatère et les asymptotes sont les bissectrices.



\subsubsection{Cas des ellipses (\(e<1\))}
Si on pose \(a=\frac{-p}{1-e^2}\) et \(b=\frac{p}{\sqrt{1-e^2}}\), alors dans le repère \(\rond\) l'équation devient~:
\begin{equation}
  M(X,Y) \in \con{e} \iff \frac{X^2}{a^2}+\frac{Y^2}{b^2}=1
\end{equation}
\begin{theo}
  Il existe une repère orthonormal direct dans lequel l'ellipse admet pour équation cartésienne~:
  \begin{equation}
    \frac{X^2}{a^2}+\frac{Y^2}{b^2}=1
  \end{equation}
  avec \(a\) et \(b\) des réels strictement positifs.
\end{theo}
\begin{prop}
  \begin{enumerate}
  \item \(p=\frac{b^2}{a}\) et \(1-e^2=\frac{b^2}{a^2}\);
  \item si on pose \(c=\sqrt{a^2-b^2}\), alors \(e=\frac{c}{a}\) et \(d=\frac{b^2}{c}\);
  \item le foyer F a pour coordonnées \((c,0)\) et la directrice \(\Dr\) a pour équation \(X=\frac{a^2}{c}\)
  \end{enumerate}
\end{prop}
\begin{proof}
  \begin{enumerate}
  \item \(\frac{b^2}{a}=\frac{\frac{p^2}{1-e^2}}{\frac{p}{1-e^2}}=p\) et \(\frac{b^2}{a^2}=\frac{\frac{p^2}{1-e^2}}{\frac{p^2}{(1-e^2)^2}}=1-e^2\);
  \item \(\frac{c}{a}=\sqrt{1-\left(\frac{b}{a}\right)^2}=\sqrt{1-1+e^2}=e\) puisque \(e>0\) et \(d=\frac{p}{e}=\frac{b^2}{a} \frac{a}{c}=\frac{b^2}{c}\);
  \item à la base l'abscisse de \(F\) est \(-\frac{de^2}{1-e^2}=\frac{-pe}{1-e^2}=ae=c\) et l'équation de la courbe est \(X=-\frac{d}{1-e^2}=-\frac{p/e}{1-e^2}=\frac{a}{e}=\frac{a^2}{c}\)
  \end{enumerate}
\end{proof}
\begin{theo}[Théorème réciproque]
  Soient deux réels strictement positifs \(a\) et \(b\) et \(c=\sqrt{a^2-b^2}\), un repère orthonormal \(R=\rond\). La courbe d'équation cartésienne dans \(R\) \(\frac{X^2}{a^2}+\frac{Y^2}{b^2}=1\) est une ellipse de foyer \(F(c,0)\) et de directrice \(\Dr\) d'équation \(X=\frac{a^2}{c}\) d'excentricité \(e=\frac{c}{a}\) et de paramètre \(p=\frac{b^2}{a}\).
\end{theo}
%L'ellipse \(\Elli\) est aussi l' de foyer \(F'(-c,0)\) de directrice \(\Dr ':X=-\frac{a^2}{c}\). Les intersections de \(\H\) avec l'axe focal sont appelés les sommets de l'hyperbole. Ce sont les points \(A(a,0)\) et \(A'(-a,0)\).
\begin{prop}
  L'ellipse \(\Ell\) est la courbe paramétrée suivante~:
  \begin{equation}
    \Gamma : x(t)=a\cos(t), y(t)=b\sin(t).
  \end{equation}
\end{prop}
% \begin{proof}
%   %% À faire
% \end{proof}


\section{Définition bifocale des coniques à centre}
\subsection{Définition bifocale de l'ellipse}
\begin{prop}
  \label{prop:bifellipse}
  Soient \(F\) et \(F'\) deux points distincts et \(a\) un réel tel que \(2a>FF'\). Alors l'ensemble des points M du plan tel que \(MF+MF'=2a\) est une ellipse de foyers \(F\) et \(F'\).
\end{prop}
\begin{proof}
  Soit O le milieu de \([FF']\) et le vecteur \(\vi=\frac{\vect{FF'}}{FF'}\) et le vecteur \(\vj\) tel que le repère \(R=\rond\) soit orthonormal direct. Les coordonnées de F et F' sont tels que \(F'(c,0)\) et \(F(-c,0)\). Soit \(\epsilon=\enstq{M \in \P}{MF+MF'=2a}\), pour tout point \(M(x,y)\),
  \begin{align}
    M \in \epsilon & \iff MF+MF'=2a \\
    & \iff \sqrt{(x+c)^2+y^2} + \sqrt{(x-c)^2+y^2}=2a \\
    & \iff (x+c)^2+y^2+(x-c)^2+y^2 \notag \\
    & \phantom{\iff} + 2\sqrt{((x+c)^2+y^2)((x-c)^2+y^2)}=4a^2 \label{eq:tag:1}\\
    & \iff x^2+c^2+y^2+\sqrt{(x^2+c^2+y^2)^2-4c^2x^2}=2a^2\\
    & \iff \begin{cases} (x^2+c^2+y^2)^2-4c^2x^2 = (2a^2-(c^2+y^2+x^2))^2 \\ x^2+y^2 \leqslant 2a^2 -c^2\\\end{cases}\\
    & \iff \begin{cases} (x^2+c^2+y^2 + 2a^2-(c^2+y^2+x^2)) \\ \times (x^2+c^2+y^2 - 2a^2+(c^2+y^2+x^2))=4c^2x^2  \\ x^2+y^2 \leqslant 2a^2 -c^2\end{cases}\\
    & \iff \begin{cases} 2a^2(2x^2+2y^2+2c^2-2a^2)=4c^2x^2  \\ x^2+y^2 \leqslant 2a^2 -c^2\end{cases}\\
    & \iff \begin{cases} (a^2-c^2)x^2 +a^2y^2=(a^2-c^2)a^2  \\ x^2+y^2 \leqslant 2a^2 -c^2\end{cases}
  \end{align}
L'équation~\eqref{eq:tag:1} est justifiée puisque les deux membres sont positifs. Si on  pose \(b=\sqrt{a^2-c^2}\), \(2a>FF'=2c\) donc \(a>c>0\), alors
\begin{align}
  M \in \epsilon & \iff \begin{cases} b^2x^2 +a^2y^2=b^2a^2 \\ x^2+y^2 \leqslant a^2 +b^2\\\end{cases}\\
  & \iff \begin{cases} \frac{x^2}{a^2} +\frac{y^2}{b^2}=1\\ x^2+y^2 \leqslant a^2 +b^2 \\\end{cases}
\end{align}
Si le couple \((x,y)\) vérifie la première équation du système alors \(\frac{x^2}{a^2}=1-\frac{y^2}{b^2}\leqslant 1\) et \(\frac{y^2}{b^2}=1-\frac{x^2}{a^2} \leqslant 1\) donc \(x^2 \leqslant a^2\) et \(y^2 \leqslant b^2\) et donc \(x^2+y^2 \leqslant a^2+b^2\) ce qui correspond à la deuxième équation du système.
Alors \(M \in \epsilon \iff \frac{x^2}{a^2} +\frac{y^2}{b^2}=1\) donc \(\epsilon\) est une ellipse et \(c=\sqrt{a^2-b^2}>0\), \(\epsilon\) est une ellipse de foyer F et F'.
\end{proof}
Soit \(\fonction{f}{\P}{\R}{M}{MF+MF'}\), un réel \(c\) tel que \(2c=FF'\), \(\lambda>c\) et l'ensemble \(X_\lambda=\enstq{M \in \P}{f(M)=\lambda}\) avec pour tout point M du plan \(FF'\leqslant MF+MF'\).
\begin{itemize}
\item Si \(\lambda < FF'\) alors \(X_\lambda=\emptyset\);
\item sinon si \(\lambda=FF'\) alors \(FF'=MF+MF'\) si et seulement si \(M \in [F,F']\) soit si et seulement si \(X_\lambda=[FF']\);
\item sinon si \(\lambda >FF'\) alors \(X_\lambda\) est une ellipse de foyer \(F\) et \(F'\).
\end{itemize}

On a représenté plusieurs ellipses sur la figure~\ref{fig:ellipse}.
\begin{figure}
    \begin{subfigure}{.5\textwidth}
      \centering
      % include first image
      \includegraphics[scale=.5]{Tracé_ellipse_1.png}  
      \caption{$e=\frac{\sqrt{5}}{3}, p=\frac{2}{3}$}
      \label{fig:ellipse1}
    \end{subfigure}
    \begin{subfigure}{.5\textwidth}
      \centering
      % include first image
      \includegraphics[scale=.5]{Tracé_ellipse_2.png}  
      \caption{$e=\frac{\sqrt{3}}{2}, p=\frac{1}{2}$}
      \label{fig:ellipse2}
    \end{subfigure}
    \newline
    \begin{subfigure}{.5\textwidth}
      \centering
      % include first image
      \includegraphics[scale=.5]{Tracé_ellipse_3.png}  
      \caption{$e=\frac{\sqrt{21}}{5}, p=\frac{2}{5}$}
      \label{fig:ellipse3}
    \end{subfigure}    
    \begin{subfigure}{.5\textwidth}
      \centering
      % include first image
      \includegraphics[scale=.5]{Tracé_ellipse_4.png}  
      \caption{$e=\frac{\sqrt{8}}{3}, p=\frac{1}{3}$}
      \label{fig:ellipse4}
    \end{subfigure}
    \newline
    \begin{subfigure}{.5\textwidth}
      \centering
      % include first image
      \includegraphics[scale=.5]{Tracé_ellipse_5.png}  
      \caption{$e=\frac{3\sqrt{5}}{7}, p=\frac{2}{7}$}
      \label{fig:ellipse5}
    \end{subfigure}    
    \begin{subfigure}{.5\textwidth}
      \centering
      % include first image
      \includegraphics[scale=.5]{Tracé_ellipse_6.png}  
      \caption{$e=\frac{\sqrt{15}}{4}, p=\frac{1}{4}$}
      \label{fig:ellipse6}
    \end{subfigure}
   \caption{Représentations graphiques de plusieurs ellipses}
  \label{fig:ellipse}
\end{figure}


\subsection{Définition bifocale de l'hyperbole}
\begin{prop}
  Soient \(F\) et \(F'\) deux points distincts. Soit \(a\) un réel tel que \(0<2a<FF'\). L'ensemble \(\H\) des points M du plan tels que \(\abs{MF-MF'}=2a\) est une hyperbole de foyer F et F'.
\end{prop}
\begin{proof}
  On se place dans le même repère que pour la démonstration de~\ref{prop:bifellipse}. On note \(F(-c,0)\) et \(F'(c,0)\). Pour tout point \(M(x,y)\)  on a~:
  \begin{align}
    M \in \H & \iff \abs{MF'-MF}=2a \\
    & \iff \abs{\sqrt{(x+c)^2+y^2}-\sqrt{(x-c)^2+y^2}}=2a \\
    & \iff (x+c)^2 +y^2 + (x-c)^2 + y^2 \notag \\ & \phantom{\iff} -2\sqrt{[(x-c)^2+y^2][(x+c)^2+y^2]}=4a^2\\
    & \iff 2x^2+2y^2+2c^2-4a^2=2\sqrt{[(x-c)^2+y^2][(x+c)^2+y^2]}\\
    & \iff \begin{cases}(x^2+y^2+c^2-2a^2)^2=(x^2+y^2+c^2)^2-4c^2x^2 \\ x^2+y^2+c^2-2a^2 \geqslant 0\end{cases}\\
    & \iff \begin{cases}-a^2(x^2+y^2+c^2-a^2)+c^2x^2=0 \\ x^2+y^2+c^2-2a^2 \geqslant 0\end{cases}\\
    & \iff \begin{cases}(c^2-a^2)x^2 - a^2y^2=a^2(c^2-a^2) \\ x^2+y^2 \geqslant 2a^2-c^2\end{cases}.
  \end{align}
  Or \(0<2a<FF'=2c\) donc \(0<a<c\) et on pose \(b=\sqrt{c^2-a^2}\), alors
  \begin{equation}
    M \in \H \iff \begin{cases}b^2x^2 - a^2y^2=a^2b^2 \\ x^2+y^2 \geqslant a^2-b^2\end{cases}\iff\begin{cases}\frac{x^2}{a^2} - \frac{y^2}{b^2}=1 \\ x^2+y^2 \geqslant a^2-b^2\end{cases}.
\end{equation}
Si \((x,y)\) vérifie la première équation alors \(\frac{x^2}{a^2}= \frac{y^2}{b^2}+1>1\) donc \(x^2\geqslant a^2\) et \(y^2\geqslant -b^2\) donc il vérifie l'inégalité \(x^2+y^2\geqslant a^2-b^2\), alors~:
\begin{equation}
  M \in \H \iff \frac{x^2}{a^2} - \frac{y^2}{b^2} =1.
\end{equation}
\(\H\) est donc une hyperbole. De plus \(c=\sqrt{b^2+a^2}\) donc F et F' sont les foyers de \(\H\).
\end{proof}
Soit l'application \(\fonction{g}{\P}{\R}{M}{\abs{MF+MF'}}\) alors d'après l'inégalité triangulaire \(MF \leqslant MF'+F'F\) et \(MF'\leqslant MF+FF'\) donc \(MF-MF'\leqslant FF\) et \(MF'-MF\leqslant FF'\) ainsi \(\abs{MF-MF'}\leqslant F'F\). Il y a égalité si et seulement si \(MF'=MF+FF'\) ou si \(MF=MF'+FF'\), c'est-à-dire si et seulement si M est dans \((FF') \setminus \intervalleoo{F}{F'}\). Soit \(\lambda\) un réel, si on considère l'ensemble \(Y_{\lambda}=\enstq{M \in \P}{g(M)=\lambda}\) alors plusieurs cas sont possibles~:
\begin{itemize}
\item si \(\lambda>FF'\) alors \(Y_{\lambda}=\emptyset\);
\item si \(\lambda=FF'\) alors \(Y_{FF'}=(FF') \setminus \intervalleoo{F}{F'}\);
\item si \(0\leqslant\lambda\leqslant FF'\) alors \(Y_{\lambda}\) est l'hyperbole de foyer \(F\) et \(F'\);
\item si \(\lambda=0\) alors \(Y_{0}\) est la médiatrice de \([FF']\);
\item si \(\lambda<0\) alors \(Y_{\lambda}=\emptyset\) puisqu'une valeur absolue ne peut pas être négative.
\end{itemize}

On a représenté plusieurs hyperboles sur la figure~\ref{fig:hyperbole}.

\begin{figure}
    \begin{subfigure}{.5\textwidth}
      \centering
      % include first image
      \includegraphics[scale=.5]{Tracé_hyperbole_1.png}  
      \caption{$e=\frac{2\sqrt{3}}{3}, p=\frac{1}{3}$}
      \label{fig:hyperbole1}
    \end{subfigure}
    \begin{subfigure}{.5\textwidth}
      \centering
      % include first image
      \includegraphics[scale=.5]{Tracé_hyperbole_2.png}  
      \caption{$e=\frac{\sqrt{6}}{2}, p=\frac{1}{2}$}
      \label{fig:hyperbole2}
    \end{subfigure}
    \newline
    \begin{subfigure}{.5\textwidth}
      \centering
      % include first image
      \includegraphics[scale=.5]{Tracé_hyperbole_3.png}  
      \caption{$e=\sqrt{2}, p=1$}
      \label{fig:hyperbole3}
    \end{subfigure}    
    \begin{subfigure}{.5\textwidth}
      \centering
      % include first image
      \includegraphics[scale=.5]{Tracé_hyperbole_4.png}  
      \caption{$e=\sqrt{3}, p=2$}
      \label{fig:hyperbole4}
    \end{subfigure}
    \newline
    \begin{subfigure}{.5\textwidth}
      \centering
      % include first image
      \includegraphics[scale=.5]{Tracé_hyperbole_5.png}  
      \caption{$e=2, p=3$}
      \label{fig:hyperbole3}
    \end{subfigure}    
    \begin{subfigure}{.5\textwidth}
      \centering
      % include first image
      \includegraphics[scale=.5]{Tracé_hyperbole_6.png}  
      \caption{$e=\sqrt{6}, p=5$}
      \label{fig:hyperbole4}
    \end{subfigure}

%  \centering
%  \includegraphics[width=\textwidth]{./hyperbole.png}
  \caption{Représentations graphiques de plusieurs hyperboles}
  \label{fig:hyperbole}
\end{figure}


\section{Courbes définies par une équation cartésienne de degré deux}
\label{sec:eqcart}
\subsection{Problème}
On se donne six réels, \(a\), \(b\), \(c\), \(d\), \(e\) et \(f\) avec \(a\), \(b\), \(c\) non tous nuls. On veut décrire la courbe \(\con{{}}\) définie dans un repère orthonormal par l'équation cartésienne suivante~:
\begin{equation}
  ax^2+bxy+cy^2+dx+ey+f=0 \label{eq:con2gre2}
\end{equation}
Un paramètre important de~\eqref{eq:con2gre2} est son discriminant \(\Delta=b^2-4ac\). L'idée est de reconnaître l'équation d'une conique par des changements de RON\@. On doit se débarrasser des termes \(xy\) et de degré 1 en \(x\) et \(y\).

\subsection{Étape 1~: si \(b\neq 0\), on se ramène par changement de repère à l'équation où \(b=0\)}
Soit \((\vect{u_{\varphi}},\vect{v_{\varphi}})\) la nouvelle BON\@. On va choisir \(\varphi\) judicieusement pour que l'équation de la courbe \(\con{{}}\) dans la nouvelle base n'ait pas de termes en \(xy\). Si \(M(x,y)\) dans \(\rond\), on note \(M(X,Y)\) ses coordonnées dans \((O,\vect{u}_\varphi,\vect{v}_\varphi)\) alors~:
\(\begin{cases} x&=\cos\varphi X - \sin\varphi Y \\ y&=\sin\varphi X + \cos\varphi Y\end{cases}\) ainsi~:
\begin{align}
  ax^2&=a(\cos^2\varphi X^2 + \sin^2\varphi Y^2 - 2\cos\varphi\sin\varphi XY)\\
  bxy&=b(\cos\varphi X^2 -\sin\varphi\cos\varphi Y^2 + (\cos^2\varphi \sin^2\varphi)XY)\\
  cy^2&=c(\sin^2\varphi Y^2 + \cos^2\varphi X^2 + 2\cos\varphi\sin\varphi XY)
\end{align}
alors l'équation~\eqref{eq:con2gre2} est équivalente à~:
\begin{multline}
 (a\cos^2\varphi + b\cos\varphi\sin\varphi +c\sin^2\varphi)X^2+(a\sin^2\varphi - b\cos\varphi\sin\varphi +c\cos^2\varphi)Y^2\\+(2c\cos\varphi\sin\varphi-2a\cos\varphi\sin\varphi+b(\cos^2\varphi - \sin^2\varphi))XY + \\ (d\cos\varphi + e\sin\varphi)X +(e\cos\varphi - d\sin\varphi)Y +f=0
\end{multline}
c'est à dire qu'il existe six réels A,B,C,D,E et F tels que~:
\begin{equation}
  AX^2+BXY+CY^2+DX+EY+F=0 \label{eq:eqz}
\end{equation}
avec
\begin{equation}
 B=(c-a)\sin(2\varphi)+b\cos(2\varphi).
\end{equation}
Alors
\begin{equation}
 B=0 \iff \cotan(2\varphi)=\frac{a-c}{b}.
\end{equation}
Puisque la cotangente induit une bijection de \(\intervalleoo{0}{\pi}\) sur \(\R\) alors il existe un certain \(\varphi\) tel que \(B=0\). On choisit maintenant \(\varphi\) tel que \(B=0\). Montrons que \(B^2-4AC=-4AC=b^2-4ac\)~:
\begin{align}
  -4AC&=-4(a\cos^2\varphi+b\cos\varphi\sin\varphi +c\sin^2\varphi)(a\sin^2\varphi \\ &- b\cos\varphi\sin\varphi +c\cos^2\varphi) \notag\\
  &=-4(ac\cos^4\varphi + ac\sin^4\varphi+(a^2+c^2-b^2)\cos^2\varphi\sin^2\varphi\\ & +(bc-ab)\cos^3\varphi\sin\varphi+(ab-bc)\cos\varphi\sin^3\varphi) \notag\\
  &=-4(ac(\cos^2\varphi+\sin^2\varphi)^2+(a^2+c^2-b^2)\cos^2\sin^2\varphi \\ & +b(c-a)\cos\varphi\sin\varphi(\cos^2\varphi-\sin^2\varphi)) \notag\\
  &=-4ac-\sin^2(2\varphi)[(a-c)^2-b^2]-2b(c-a)\sin(2\varphi)\cos(2\varphi)
\end{align}
et puisque \(B=0\), \(\varphi\) vérifie \((c-a)\sin(2\varphi)=-b\cos(2\varphi)\), alors
\begin{align}
  -4AC&=-4ac-(a-c)^2\sin^2(2\varphi)+b^2\sin^2(2\varphi)+2b^2\cos^2(2\varphi)\\
  &=b^2-4ac-(a-c)^2\sin^2(2\varphi)+b^2\cos^2(2\varphi)\\
  &=b^2-4ac
\end{align}

\subsection{Étape 2~: disjonction des cas selon la valeur de \(\Delta\)}
\subsubsection{\(\Delta=0\), la courbe \(\con{{}}\) est du type parabole}
C'est à dire que \(A=0\) ou \(C=0\), quitte à faire un changement d'axe on suppose que \(C=0\). Le réel \(A\) est non nul sinon \((A,B,C)=(0,0,0)\). Alors~:
\begin{equation}
  AX^2+DX+EY+F=0 \iff A\left(X+\frac{D}{2A}\right)^2-\frac{AD^2}{4A^2}+EY+F=0
\end{equation}
on pose \(\begin{cases}X'=X+\frac{D}{2A} \\ Y'=Y\end{cases}\) et \(F'=F-\frac{D^2}{4A}\)(on change d'origine) et on a~:
\begin{equation}
  AX'^2+EY'+F'=0.
\end{equation}
Plusieurs cas sont possibles~:
\begin{itemize}
\item si \(E \neq 0\) alors \(Y'=-\frac{A}{E}X'^2-\frac{F'}{E}\) et donc la courbe \(\con{{}}\) est une parabole;
\item sinon alors \(AX'^2+F'=0\)
  \begin{itemize}
  \item si \(\frac{F'}{A}>0\) alors \(\con{{}}=\emptyset\);
  \item sinon si \(F'=0\) alors \(X'=0\) et \(\con{{}}\) est une droite;
  \item sinon si \(\frac{F'}{A}<0\) alors \(X'=\pm \sqrt{\frac{-F'}{A}}\), et \(\con{{}}\) est la réunion de deux droites parallèles.
  \end{itemize}
\end{itemize}

\subsubsection{\(\Delta>0\), la courbe \(\con{{}}\) est du type hyperbole}
Dans ce cas on a \(AC<0\) et quitte à changer l'équation en son opposée, on peut supposer que \(A>0\) et \(C<0\). Alors l'équation~\eqref{eq:eqz} est équivalente à~:
\begin{equation}
  A\left(X+\frac{D}{2A}\right)^2 -A\frac{D^2}{4A^2} + C\left(Y+\frac{E}{2C}\right)^2 -C\frac{E^2}{4C^2}+F=0,
\end{equation}
puis en effectuant le changement de variable
\begin{equation}
  \begin{cases}
    X' = X+\frac{D}{2A} \\
    Y' = Y+\frac{E}{2C}
  \end{cases},
\end{equation}
alors on obtient (en posant \(F'=F-\frac{D^2}{4A}-\frac{E^2}{4C}\)):
\begin{equation}
  AX'^2+CY'^2+F'=0.
\end{equation}
Deux cas sont alors possibles~:
\begin{itemize}
\item si \(F'=0\) alors \(X'=\pm \sqrt{\frac{-C}{A}}Y'\) et la courbe \(\con{{}}\) est la réunion de deux droites sécantes d'équation \(X'=\sqrt{\frac{-C}{A}}Y'\) et \(X'=-\sqrt{\frac{-C}{A}}Y'\);
\item sinon alors \(\frac{-C}{F'}Y^2-\frac{A}{F'}X^2=0\) et \(\con{{}}\) est une hyperbole dont les asymptotes sont les droites précédentes.
\end{itemize}

\subsubsection{\(\Delta<0\), la courbe \(\con{{}}\) est du type ellipse}
Dans ce cas on a \(AC<0\) et quitte à changer l'équation en son opposée, on peut supposer que \(A>0\) et \(C>0\). Alors l'équation~\eqref{eq:eqz} est équivalente à~:
\begin{equation}
  A\left(X+\frac{D}{2A}\right)^2 -A\frac{D^2}{4A^2} + C\left(Y+\frac{E}{2C}\right)^2 -C\frac{E^2}{4C^2}+F=0.
\end{equation}
En effectuant le changement de variable
\begin{equation}
  \begin{cases}
    X' = X+\frac{D}{2A}\\Y' = Y+\frac{E}{2C}
  \end{cases},
\end{equation}
on choisit une nouvelle origine \(\Omega\left(-\frac{D}{2A},-\frac{E}{2C}\right)\). Alors on obtient (en posant \(F'=F-\frac{D^2}{4A}-\frac{E^2}{4C}\))~:
\begin{equation}
  AX'^2+CY'^2+F'=0.
\end{equation}
Trois cas sont alors possibles~:
\begin{itemize}
\item si \(F'>0\) alors \(\con{{}}=\emptyset\);
\item si \(F'=0\) alors \(X'=Y'=0\) et donc \(\con{{}}=\{\Omega\}\);
\item sinon \(\frac{A}{-F'}X'^2+\frac{C}{-F'}Y'^2=1\) et~:
  \begin{itemize}
  \item si \(A=C\) alors \(\con{{}}\) est un cercle;
  \item sinon alors \(\con{{}}\) est une ellipse.
  \end{itemize}
\end{itemize}
On parle de conique propre pour l'ellipse, l'hyperbole et la parabole; sinon on parle de conique dégénérée pour le cercle et les droites.

\section{Tangentes à une conique}
\subsection{Conique définie par une équation cartésienne}
Soient \(a,b,c,d,e\) et \(f\) six réels avec \(a,b,c\) non tous nuls et \(\con{{}}\) la conique d'équation~:
\begin{equation}
  ax^2+bxy+cy^2+dx+ey+f=0
\end{equation}
On suppose de plus que \(\con{{}}\) est une conique propre. Nous avons vu dans la section~\ref{sec:eqred} que \(\con{{}}\) peut être paramétrée par \((I,x,y)\) avec \(x\) et \(y\) dérivables. Nous rappelons que les équations étaient dans un bon repère telles que~:
\begin{itemize}
\item pour une parabole on a \(\begin{cases} x(t) &= \frac{t^2}{2p} \\ y(t) &= t \end{cases}\);
\item pour une ellipse on a \(\begin{cases} x(t) &= a\cos t\\ y(t) &= b\sin t \end{cases}\);
\item  et pour l'hyperbole \(\begin{cases} x(t) &= \pm a\cosh t\\ y(t) &= b\sinh t \end{cases}\).
\end{itemize}
On a aussi vu dans la section~\ref{sec:eqcart} que n'importe quelle conique admettait une équation cartésienne de degré deux telle que~:
\begin{equation}
\forall t \in I \quad   ax(t)^2+bx(t)y(t)+cy(t)^2+dx(t)+ey(t)+f=0
\end{equation}
Les fonctions \(x\) et \(y\) sont dérivables, alors si on dérive cette équation on a~:
\begin{equation}
  \forall t \in I \quad x'(t) \cdot [2ax(t)+by(t)+d]+y'(t) \cdot [2cy(t)+bx(t)+e]=0
\end{equation}
Montrons que si \((x_0,y_0)\) est un point de la courbe \(\con{{}}\) alors \((2ax_0+by_0+d,2cy_0+bx_0+e)\neq (0,0)\).
\begin{proof}
  Soit \(M_0(x_0,y_0)\) un point du plan qui vérifie le système suivant \(\begin{cases}2ax_0+by_0+d &=0 \\ 2cy_0+bx_0+e &=0\end{cases}\). Si on pose les points \(M(x,y)\) et \(M'(2x_0-x,2y_0-y)\) alors après le développement des calculs on a~:
  \begin{align}
    a(2x_0-x)^2+b(2x_0-x)+c(2y_0-y)^2+d(2x_0-x)+e(2y_0-y)+f \notag \\
= ax(t)^2+bx(t)y(t)+cy(t)^2+dx(t)+ey(t)+f
  \end{align}
Alors \(M \in \con{{}} \iff M' \in \con{{}}'\). Un tel point serait un centre de symétrie de \(\con{{}}\). Deux cas sont possibles~:
\begin{itemize}
\item Si \(\con{{}}\) est une parabole, un tel point n'existe pas;
\item sinon \(M_0\) existe mais n'est pas sur \(\con{{}}\).
\end{itemize}
En conclusion si un point \(M_0(x_0,y_0)\) est sur \(\con{{}}\) alors \(2ax_0+by_0+d \neq 0\) ou \(2cx_0+dy_0+e \neq 0\).
\end{proof}
Par conséquent, en un point \(M_0(x_0,y_0)\) la conique \(\con{{}}\) admet une tangente \(T\) orthogonale à \(\vect{u}=\begin{pmatrix} 2ax_0 +by_0 +d \\ 2cy_0+bx_0+e \end{pmatrix}\)  alors~:
\begin{align}
M \in (T) &\iff \vect{M_0 M} \cdot \vect{u}=0 \\
&\iff (x-x_0)(2ax_0+by_0+d)+(y-y_0)(2cy_0+bx_0+e)=0 \\
&\iff 2axx_0 + b(xy_0+yx_0)+2cyy_0 +dx+ey - \alpha=0
\end{align}
avec
\begin{equation}
\alpha=2ax_0^2+  2bx_0y_0+2cy_0^2+dx_0+ey_0=-2f-dx_0-ey_0,
\end{equation}
puisque le point \(M_0\) est sur la conique. Alors
\begin{equation}
  M \in (T) \iff 2axx_0+b(xy_0+x_0y)+2cyy_0+dx+e_y+dx_0+ey_0+2f=0.
\end{equation}
On dit que l'équation de la tangente \(T\) à la courbe \(\con{{}}\) est obtenue à partir de l'équation de \(\con{{}}\) par la règle du dédoublement. En pratique \(\frac{X^2}{a^2} + \frac{Y^2}{b^2}=1\) alors la tangente en un point \((x_0,y_0)\) est telle que
\begin{equation}
  \frac{1}{a^2}(2xx_0) + \frac{1}{b^2}(2yy_0)=2 \iff \frac{1}{a^2}(xx_0) + \frac{1}{b^2}(yy_0)=1.
\end{equation}

\subsection{Coniques définies par une équation polaire}

Soit \(\con{{}}\) la conique d'équation polaire \(\rho=\frac{p}{1+e\cos \theta}\) avec \(e>0\) et \(p=ed>0\). la fonction \(\rho\) est dérivable. La tangente en un point \(M_{\theta}\) de coordonnées polaires \((\rho(\theta),\theta)\) est dirigée par le vecteur \(\rho'(\theta) \vect{u}_\theta + \rho(\theta) \vect{v}_\theta\) et on a~:
\begin{equation}
  \forall \theta \in \R \quad \rho'(\theta)=\frac{pe\sin\theta}{(1+e\cos\theta)^2},
\end{equation}
alors
\begin{equation}
  \rho'(\theta) \vect{u}_\theta + \rho(\theta) \vect{v}_\theta = \frac{p}{(1+e\cos\theta)^2} \vect{w}_\theta,
\end{equation}
avec \(\vect{w}_\theta = e\sin\theta\vect{u}_\theta + (1+e\cos\theta)\vect{v}_\theta\) le vecteur directeur de la tangente \(T\). Dans le repère \((O,\vect{u}_\theta, \vect{v}_\theta)\) soit \(M(X,Y)\), alors
\begin{align}
  M \in T & \iff  \Det(\vect{M_\theta M},\vect{w}_\theta)=0 \\
  & \iff \begin{vmatrix} X-\rho(\theta) & e\sin\theta \\ Y & 1+e\cos\theta \end{vmatrix} = 0 \\
  & \iff X(1+e\cos\theta)-\rho(\theta)(1+e\cos\theta)-e\sin\theta Y = 0\\
  & \iff (1+e\cos\theta)X-e\sin\theta Y=p.
\end{align}
Si on note les coordonnées de \(M(x,y)\) dans le repère initial \(\rond\), alors
\begin{align}
  M \in T &\iff (1+e\cos\theta)(\cos\theta x+\sin\theta y)-e\sin\theta(-\sin\theta x+\cos\theta y)=p\\
&\iff (e+\cos\theta)x+\sin\theta y=p.
\end{align}

\subsection{Caractérisation géométrique des tangentes à une conique}
\subsubsection{Cas de la parabole}
\begin{prop}
  Soient \(\P\) une parabole de foyer \(F\) et de directrice \(\Dr\), un point \(M\) de \(\P\) et son projeté orthogonal \(H\) sur \(\Dr\). Ainsi la tangente à \(\P\) en M est la médiatrice du segment \([FH]\), comme le montre la figure~\ref{fig:tangente_parabole}.
\end{prop}
\begin{proof}
  On se place dans le repère orthonormal direct \(\rond\) dans lequel \(\P\) a pour équation cartésienne \(y^2=2px\), alors \(F\left(\frac{p}{2},0\right)\) et \(\Dr{}~: x=-\frac{p}{2}\). Soit \(M_0(x_0,y_0)\in\P\), l'équation de la tangente à \(\P\) en \(M_0\) est \(yy_0=p(x+x_0)\). Soit \(I\) le point d'intersection de \(T\) avec \((Oy)\) alors l'ordonnée du point I vaut \(p\frac{x_0}{y_0}=\frac{y_0^2}{2y_0}=\frac{y_0}{2}\). Les coordonnées du point H sont \(H\left(\frac{-p}{2}, y_0\right)\) et celle du foyer sont bien sûr \(F\left(\frac{p}{2},0\right)\). Le point I est donc le milieu de \([HF]\) or \(M_0\in\P\) donc \(M_0H=M_0F\) donc \((M_0I)=T\) est la médiatrice de \([HF]\).
\end{proof}
\emph{Remarque}~: le point \(M_0\) de la démonstration est confondu avec I si et seulement si \(M_0=O\) et alors dans ce cas \(T=(Oy)\) et c'est encore la médiatrice de \([HF]\).

\subsubsection{Cas de l'ellipse}
\begin{prop}
  Soit \(\Ell\) une ellipse de foyer \(F\) et \(F'\). Soit \(M\) un point de l'ellipse. La tangente \(T\) en \(M\) à l'ellipse est la bissectrice extérieure de l'angle \(\widehat{F'MF}\), comme le montre la figure~\ref{fig:tangente_ellipse}.
\end{prop}
\begin{proof}
  Soient \((I,f(t))\) un paramétrage de l'ellipse avec \(f\) une application dérivable et les applications \(h=\norme{\vect{F'M}}\) \(g=\norme{\vect{FM}}\). On sait que \(g+h=2a\) avec \(\vect{FM}\) et \(\vect{F'M}\) tous deux non nuls (puisqu'on sait que les foyers ne sont pas sur l'ellipse). Alors \(g\) et \(h\) sont dérivables. Soit un réel \(t\), alors~:
\begin{gather}
  g'(t)+h'(t)=0; \\
  \vect{FM}(t)=\vect{FO}+\vect{OM}(t), \quad \vect{F'M}(t)=\vect{F'O}+\vect{OM}(t);\\
  \derived{\vect{FM}}{t}=\derived{\vect{OM}}{t}, \quad \derived{\vect{F'M}}{t}=\derived{\vect{OM}}{t}.
\end{gather}
  Alors pour tout \(t \in I\)
  \begin{equation}
    0=g'(t)+h'(t)=\frac{\vect{FM(t)}\cdot\derived{\vect{OM}}{t}}{\norme{\vect{FM(t)}(t)}} + \frac{\vect{F'M(t)}\cdot\derived{\vect{OM}}{t}}{\norme{\vect{F'M(t)}(t)}}=(\vect{u}+\vect{v})\cdot \derived{\vect{OM}}{t}.
  \end{equation}
Le vecteur \(\derived{\vect{OM}}{t}\) est orthogonal au vecteur \((\vect{u}+\vect{v})\) avec \(\vect{u}\) unitaire qui dirige \((FM)\) et \(\vect{v}\) unitaire qui dirige \((F'M)\), donc \(\vect{u}+\vect{v}\) dirige la bissectrice intérieure de \(\widehat{F'MF}\). Alors \(T\) est la droite passant par \(M\) de vecteur normal \(\vect{u}+\vect{v}\), donc \(T\) est la bissectrice extérieure de l'angle \(\widehat{F'MF}\).
\end{proof}

\subsubsection{Cas de l'hyperbole}
\begin{prop}
    Soient \(\H\) une hyperbole de foyers \(F\) et \(F'\), \(M\) un point de \(\H\). La tangente \(T\) à \(\H\) en \(M\) est la bissectrice intérieure de l'angle \(\widehat{F'MF}\), comme le montre la figure~\ref{fig:tangente_hyperbole}.
\end{prop}
\begin{proof}
  Soit \((I,f)\) un paramétrage d'une des deux branches de l'hyperbole avec \(f\) dérivable. Soient les applications \(h=\norme{\vect{F'M}}\) \(g=\norme{\vect{FM}}\). On sait que pour tout point \(M\) de l'hyperbole, \(\abs{MF'-MF}=2a\). De plus, si on se place sur une des deux branche alors \(MF'-MF\) est constant. Comme pour l'ellipse, les foyers n'appartiennent à l'hyperbole, donc \(\vect{FM}(t)\) et \(\vect{F'M}(t)\) sont toujours non nuls. Les fonction \(g\) et \(h\) sont ainsi dérivables sur \(I\) et de manière analogue à l'ellipse on a~:
\begin{equation}
\forall t \in I \quad 0=g'(t)-h'(t)=(\vect{u}-\vect{v})\cdot \derived{\vect{OM}}{t}(t).
\end{equation}
 Le vecteur vitesse \(\derived{\vect{OM}}{t}\) est orthogonal au vecteur \((\vect{u}-\vect{v})\) avec \(\vect{u}\) unitaire qui dirige \((FM)\) et \(\vect{v}\) unitaire qui dirige \((F'M)\), donc \(\vect{u}-\vect{v}\) dirige la bissectrice extérieure de \(\widehat{F'MF}\). Alors \(T\) est la droite passant par \(M\) de vecteur normal \(\vect{u}-\vect{v}\), donc \(T\) est la bissectrice intérieure de l'angle \(\widehat{F'MF}\).
\end{proof}
\begin{figure}
    \centering
    \includegraphics[scale=0.7]{./Tangente_parabole.png}
    \caption{Représentation graphique de tangentes d'une parabole}
    \label{fig:tangente_parabole}
\end{figure}
\begin{figure}
    \centering
    \includegraphics[scale=0.7]{./Tangente_hyperbole.png}
    \caption{Représentation graphique de tangentes d'une hyperbole}
    \label{fig:tangente_hyperbole}
\end{figure}
\begin{figure}
    \centering
    \includegraphics[scale=0.7]{./Tangente_ellipse.png}
    \caption{Représentation graphique de tangentes d'une ellipse}
    \label{fig:tangente_ellipse}
\end{figure}
\newpage
\section{Exercices}
\begin{exercice}
    Étudier et représenter la courbe d'équation~: \(x^2-2y^2+4x+12y-16=0\).
\end{exercice}
\begin{exercice}
    Soient \(F\) un point et \(\Dr\) une droite du plan. Déterminer l'ensemble des points \(O\) tels que le cercle de centre \(O\) passant par \(F\) soit tangent à la droite \(\Dr\).
\end{exercice}
\begin{exercice}
    On considère la parabole \((\P)\) d'équation \(y=ax^2\) avec \(a \in \R\). Soient \(p, p'\) deux réels distincts donnés. Par un point \(A\) de \((\P)\), on mène une droite de pente \(p\) et une droite de pente \(p'\) et on note \(B\) et \(C\), respectivement, les points où elles recoupent la parabole \((\P)\).
    \begin{enumerate}
        \item Quel est le lieu des milieux \(I\) de \([BC]\) lorsque \(A\) parcourt \((\P)\) ?
        \item Quel est le lieu des centres de gravité \(G\) des triangles \(ABC\) lorsque \(A\) parcourt \((\P)\) ?
    \end{enumerate}
\end{exercice}
\begin{exercice}
    \begin{enumerate}
        \item Soient \((\mathcal{C})\) un cercle du plan et \(M_0\) un point à l'intérieur du cercle. Si \(\Dr\) est une droite passant par \(M_0\), elle coupe le cercle en deux points \(P, Q\). Quel est le lieu du milieu \(I\) de \([PQ]\) lorsque \(\Dr\) "tourne" autour de \(M_0\) ?
        \item Même question dans le cas d'une ellipse \((\mathcal{E})\) avec \(M_0\) à l'intérieur.
    \end{enumerate}
\end{exercice}
\begin{exercice}
    Dans un repère orthonormal \((O, \vi, \vj)\) on considère les points \(A(-\sqrt{2}, -\sqrt{2}), B(\sqrt{2}, -\sqrt{2})\) et le cercle \(\mathcal{C}\) de centre \(O\) et de rayon \(2\).
    \begin{enumerate}
        \item Soit \(M\) un point du cercle distinct de \(A\) et \(B\). Les droites \((AM)\) et \((BM)\) coupent l'axe \((Ox)\) en \(P\) et \(Q\). On note \(I\) le centre du cercle circonscrit au triangle \(MPQ\). Montrer qu'il existe \(\lambda_M \in \R\) tel que \(\vect{OI} = \lambda_M \vect{OM}\), calculer \(\lambda_M\).
        \item Quel est le lieu des points \(I\) lorsque \(M\) parcourt \(\mathcal{C}\setminus\{A, B\}\) ? donner un système d'équation polaire à la courbe et la reconnaitre.
    \end{enumerate}
\end{exercice}
\begin{exercice}
    Soit \(ABCD\) un rectangle. Déterminer le lieu des points \(M\) tels que les cercles circonscrits à \(MAB\) et à \(MCD\) aient le même diamètre.
\end{exercice}
\begin{exercice}
    Soient \(\mathcal{R} = (O, \vi, \vj)\) un repère orthonormal direct fixé et \(a>b>0\) deux réels fixés. Pour \(\lambda \in \intervalleoo{-\infty}{a}\setminus\{b\}\), on note \(\mathcal{C}_\lambda\) la courbe d'équation cartésienne dans \(\mathcal{R}\) \(\frac{x^2}{a-\lambda} + \frac{y^2}{b-\lambda} = 1\).
    \begin{enumerate}
        \item Caractériser \(\mathcal{C}_\lambda\) selon la valeur de \(\lambda\). Faire un dessin.
        \item Montrer que par un point du plan, situé hors des deux axes, passent exactement deux courbes \(\mathcal{C}_\lambda\) et \(\mathcal{C}_\mu\). Montrer que ces deux courbes sont perpendiculaires.
    \end{enumerate}
\end{exercice}
\begin{exercice}
    Dans un carré, représenter l'ensemble des points plus proches du centre que des bords.
\end{exercice}
\begin{exercice}
    Soient \(F, A, B\) trois points distincts du plan. Déterminer l'ensemble des points \(F'\) tels que \(A, B\) appartiennent à une même conique de foyers \(F\) et \(F'\).
\end{exercice}
