\chapter{Espace géométrique euclidien --- Pré-requis}

L'objet de cette annexe est de rappeler des résultats et notions supposés 
connus à l'issue de la classe de terminale, mais avec une formalisation 
nouvelle. L'espace géométrique euclidien \(\E\) est un ensemble dont les 
éléments sont appelés points et notés en lettre majuscules. On lui associe un 
ensemble noté \(\vect{\E}\) appelé espace vectoriel euclidien. Ses éléments 
sont appelés vecteurs et notés en minuscules surmontés d'une flèche. On appelle 
scalaire les éléments de l'ensemble \(\R\), plus généralement les éléments d'un 
corps.

\section{Espace vectoriel \(\protect\vect{\E}\)}

L'espace vectoriel \(\vect{\E}\) est muni de deux lois
\begin{itemize}
  \item l'addition vectorielle, qui fait correspondre aux vecteurs \(\vu\) et 
    \(\vv\) le vecteur noté \(\vu + \vv\);
  \item la multiplication scalaire, qui au scalaire \(\lambda\) et au vecteur 
    \(\vu\) le vecteur noté \(\lambda \vu\).
\end{itemize}

Il possède de plus un élément appelé vecteur nul noté \(\vec{0}\).

\begin{prop}
  \((\vect{\E}, +, \cdot)\) est un \(\R-\)espace vectoriel, c'est à dire
  \begin{enumerate}
    \item \((\vect{\E}, +)\) est un groupe abélien;
    \item \(\cdot\) admet un élément neutre noté \(1\);
    \item \(\cdot\) est associative;
    \item \(\cdot\) est distributive sur \(+\);
    \item \(0 \cdot \vu = \lambda \vec{0} = \vec{0}\)
  \end{enumerate}
\end{prop}
\begin{defdef}
  Soient \(\vu, \vv\) et \(\vw\) trois vecteurs de \(\vect{\E}\).
  \begin{enumerate}
    \item On dit que \(\vu\) et \(\vv\) sont colinéaires ou liés si
      \begin{equation}
        \vu = \vec{0} \textrm{~ou~} \exists \lambda \in \R \ \vv = \lambda \vu
      \end{equation}
    \item On dit que \(\vu, \vv\) et \(\vw\) sont coplanaires ou liés si
      \begin{equation}
        \vu \textrm{~et~} \vv \textrm{~sont liés ou~} \exists \lambda, \mu 
        \in \R \ \vw = \lambda \vu + \mu \vv
      \end{equation}
  \end{enumerate}
\end{defdef}
\begin{prop}
  Il existe dans \(\vect{\E}\) des triplets \((\vi,\vj,\vk)\) de vecteurs non 
  coplanaires. De plus si \((\vi,\vj,\vk)\) est un tel triplet alors c'est une 
  base de \(\vect{\E}\), c'est à dire que pour tout vecteur \(\vu\) de 
  \(\vect{\E}\), il existe un unique triplet \((x,y,z)\) de réels tel que
  \begin{equation}
    \vu = x\vi + y\vj + z\vk
  \end{equation}
\end{prop}
\begin{defdef}
  Le triplet \((x,y,z)\) est alors appelé coordonnées du vecteur \(\vu\) dans 
  la base \((\vi,\vj,\vk)\).
\end{defdef}

Ce résultat caractérise le fait que \(\vect{\E}\) est un espace vectoriel de 
dimension trois, c'est à dire un espace.

\section{Espace affine \(\E\)}

L'espace affine \(\E\) est muni d'une loi qui aux points \(A\) et \(B\) fait 
correspondre le vecteur \(\vect{AB}\).

\begin{prop}
  Soient \(A\), \(B\) et \(C\) des points de \(\E\), alors
  \begin{enumerate}
    \item \(\vect{AB}+\vect{BC} = \vect{AC}\);
    \item \(\vect{AA}=\vec{0}\);
    \item \(\vect{AB} = -\vect{BA}\);
    \item Pour tout vecteur \(\vu\) de \(\E\) il existe une unique point \(A' 
      \in \E\) tel que \(\vect{AA'}=\vu\). On note alors \(A'=A+\vu\).
  \end{enumerate}
\end{prop}
\begin{defdef}
  \begin{enumerate}
    \item On dit que trois points \(A\), \(B\) et \(C\) sont alignés si les 
      vecteurs \(\vect{AB}\) et \(\vect{BC}\) sont colinéaires.
    \item On dit que quatre points \(A, B, C\) et \(D\) sont coplanaires si les 
      vecteurs \(\vect{AB}\), \(\vect{AC}\) et \(\vect{AD}\) sont coplanaires.
  \end{enumerate}
\end{defdef}
\begin{defdef}
  On appelle points pondéré tout couple \((A,\lambda)\) où \(A\) est point de 
  \(\E\) et \(\lambda\) un réel.
\end{defdef}
\begin{prop}
  Soit un naturel \(n \geq 2\) et \(\{(A_1,\lambda_1), \ldots, 
  (A_n,\lambda_n)\}\) une famille de \(n\) points pondérés telle que 
  \(\sum_i^n\lambda_i \neq 0\). Alors il existe un unique point \(G \in \E\) 
  tel que
  \begin{equation}
    \sum_{i=1}^n \lambda_i \vect{GA_i}=0
  \end{equation}
  Il est appelé barycentre du système de points pondérés \(\{(A_1,\lambda_1), 
  \ldots, (A_n,\lambda_n)\}\) et on note
  \begin{equation}
    G = \Bary\{(A_1,\lambda_1), \ldots, (A_n,\lambda_n)\}
  \end{equation}
  On a alors pour tout point \(M \in \E\)
  \begin{equation}
    \sum_{i=1}^n \lambda_i \vect{MA_i} = \sum_{i=1}^n \lambda_i \vect{MG}
  \end{equation}
\end{prop}
\begin{prop}
  Soient \(A,B, A_1, \ldots, A_p,B_1, \ldots, B_q\) des points de \(\E\) et 
  \(\lambda, \mu, \lambda_1, \ldots, \lambda_p, \mu_1, \ldots, \mu_q\) des 
  réels. En notant \(G = \Bary\{(A_1,\lambda_1), \ldots, (A_p,\lambda_p)\}\), 
  \(\alpha=\sum_{i=1}^p \lambda_i\), \(H = \Bary\{(B_1,\mu_1), \ldots, 
  (B_q,\mu_q)\}\), \(\beta=\sum_{i=1}^q \mu_i\) on a
  \begin{enumerate}
    \item L'associativité du barycentre, c'est à dire que
      \begin{equation}
        \Bary\{(G,\alpha),(H,\beta)\} = \Bary\{(A_1,\lambda_1), \ldots, 
        (A_p,\lambda_p), (B_1,\mu_1), \ldots, (B_q,\mu_q)\}
      \end{equation}
    \item La commutativité du barycentre
      \begin{equation}
        \Bary\{(A,\lambda),(B,\mu)\}=\Bary\{(B,\mu),(A,\lambda)\}
      \end{equation}
    \item
      \begin{equation}
        \Bary\{(A,\lambda),(A,\lambda)\}=A
      \end{equation}
  \end{enumerate}
\end{prop}

\section{Distances et normes}

L'espace vectoriel \(\vect{\E}\) est muni d'une loi qui au vecteur \(\vu\) 
associe le réel noté \(||\vu||\) appelé sa norme ou norme euclidienne.
\begin{prop}
  Soient \(\vu\) et \(\vv\) deux vecteurs de \(\vect{\E}\) et \(\lambda \in 
  \R\), on a les propriétés suivantes
  \begin{enumerate}
    \item positivité
      \begin{equation}
        ||\vu|| \geq 0
      \end{equation}
    \item séparation
      \begin{equation}
        ||\vu|| = 0 \Rightarrow \vu = \vec{0}
      \end{equation}
    \item homogénéité
      \begin{equation}
        ||\lambda \vu||=|\lambda| \cdot ||\vu||
      \end{equation}
    \item inégalité triangulaire
      \begin{equation}
        ||\vu+\vv|| \leq ||\vu|| + ||\vv||
      \end{equation}
      égalité si et seulement s'ils sont colinéaires et de même sens
  \end{enumerate}
\end{prop}
L'espace affine \(\E\) est muni d'une loi qui aux points \(A\) et \(B\) fait 
correspondre le réel \(||\vect{AB}||\), noté \(AB\) et appelé distance de \(A\) 
à \(B\).
\begin{prop}
  Soient \(A\), \(B\) et \(C\) trois points de \(\E\) on a les propriétés 
  suivantes
  \begin{enumerate}
    \item positivité
      \begin{equation}
        AB \geq 0
      \end{equation}
    \item séparation
      \begin{equation}
        AB = 0 \Rightarrow A=B
      \end{equation}
    \item symétrie
      \begin{equation}
        AB=BA
      \end{equation}
    \item inégalité triangulaire
      \begin{equation}
        AC \leq AB + BC
      \end{equation}
      égalité si et seulement si \(A\) \(B\) et \(C\) sont alignés dans cet 
      ordre
  \end{enumerate}
\end{prop}

\section{Angles non orientés, orthogonalité}

L'espace vectoriel \(\vect{\E}\) est muni d'une loi qui aux vecteurs non nuls 
\(\vu\) et \(\vv\) associe le réel noté \(\widehat{(\vu,\vv)}\), appartenant à 
\([0, \pi]\), appelé mesure de l'angle non orienté entre les vecteurs \(\vu\) 
et \(\vv\).
\begin{prop}
  Soient \(\vu\), \(\vv\) et \(\vw\) des vecteurs non nuls de \(\vect{\E}\) et 
  un réel non nul \(\lambda\), on a
  \begin{enumerate}
    \item \(\widehat{(\vu,\vw)} \leq \widehat{(\vu,\vv)} + 
      \widehat{(\vv,\vw)}\)
    \item \(\widehat{(\vu,\vu)} = 0\)
    \item \(\widehat{(\vu,\vv)} = \widehat{(\vv,\vu)}\)
    \item Si \(\lambda>0\) \(\widehat{(\vu,\lambda \vv)}= \widehat{(\vu,\vv)}\) 
      et si \(\lambda < 0\) alors \(\widehat{(\vu,\lambda \vv)}= 
      -\widehat{(\vu,\vv)}\)
  \end{enumerate}
\end{prop}

Deux vecteurs \(\vu\) et \(\vv\) sont donc colinéaires si et seulement si 
\(\vu\) ou \(\vv\) est nul ou si \(\widehat{(\vu,\vv)}=0\) ou \(\pi\).
\begin{defdef}
  On dit que deux vecteurs \(\vu\) et \(\vv\) sont orthogonaux si et seulement 
  si \(\vu\) est nul ou si \(\vv\) est nul ou encore si \(\widehat{(\vu,\vv)} = 
  \frac{\pi}{2}\).
\end{defdef}

Étant donné trois points \(A\), \(B\) et \(C\) de \(\E\) tels que \(A \neq B\) 
et \(C \neq B\), on appelle mesure de l'angle non orienté \(\widehat{ABC}\) la 
mesure de l'angle non orienté \(\widehat{(\vect{BA},\vect{BC})}\).

\begin{prop}
  Pour tout triangle \(ABC\) rectangle en \(B\), on a
  \begin{enumerate}
    \item Le théorème de Pythagore : \(AB^2+BC^2=AC^2\);
    \item \(AB=AC\cos(\widehat{ABC})\);
    \item \(BC=AC\sin(\widehat{ABC})\);
    \item \(BC=AB\tan(\widehat{ABC})\).
  \end{enumerate}
\end{prop}

\section{Bases et repères}

Un triplet \((\vi,\vj,\vk)\) de vecteurs de \(\vect{E}\) est une base de 
\(\vect{\E}\) si et seulement si \(\vi\), \(\vj\) et \(\vk\) sont non 
coplanaires. 

\begin{defdef}
  On appelle repère de l'espace \(\E\) tout quadruplet \((O,\vi,\vj,\vk)\) où 
  \(O\) est un point quelconque de l'espace \(\E\) et \((\vi,\vj,\vk)\) une 
  base de \(\vect{\E}\).

  Si \(M\) est un point de \(\E\), on appelle coordonnées du point \(M\) dans 
  le repère \((O,\vi,\vj,\vk)\) l'unique triplet de réels \((x,y,z)\) tel que
  \begin{equation}
    \vect{OM}=x\vi+y\vj+z\vk
  \end{equation}
\end{defdef}

\begin{defdef}
  On dit que \((\vi,\vj,\vk)\) est une base orthonormée si les vecteurs 
  \(\vi\), \(\vj\) et \(\vk\) sont des vecteurs orthogonaux et de norme 1. Le 
  repère \((O,\vi,\vj,\vk)\) est orthonormé si la base \((\vi,\vj,\vk)\) est 
  orthonormée.
\end{defdef}

On définira plus tard, à l'aide du déterminant, les notions de base directe et 
de base indirecte. Dans ce chapitre, on se contentera d'en donner une 
définition géométrique à l'aide de la ``règle des trois doigts''. 

On retiendra que si \(\B=(\vu,\vv,\vw)\) est une base directe de l'espace alors 
les bases \((\vv,\vw,vu)\) et \((\vw,\vu,\vv)\), obtenues par permutation 
circulaires, sont également directes. Alors que les bases \((\vv,\vu,\vw)\), 
\((\vw,\vv,\vu)\) et \((\vu,\vw,\vv)\) obtenues en échangeant à chaque fois 
deux vecteurs et les bases \((-\vu,\vv,\vw)\) \((\vu,-\vv,\vw)\) et 
\((\vu,\vv,-\vw)\), obtenues en remplaçant à chaque fois un des vecteurs par 
son opposé, sont indirectes.

On dira que le repère \((O,\vi,\vj,\vk)\) est orthonormée direct 
(respectivement indirect) si la base \((\vi,\vj,\vk)\) est orthonormée directe 
(respectivement indirecte).
