\chapter{Étude métrique des courbes planes}
\label{chap:etudemetriquedescourbesplanes}
\minitoc
\minilof
\minilot

On travaille dans \(\R^2\) muni de sa structure canonique d'espace vectoriel orienté et de sa base canonique. Soit \(I\) un intervalle réel et \(k \in \N^* \cup\{+\infty\}\).

\section{Modes de définitions des courbes planes}

\subsection{Représentation paramétrique et cartésiennes}

Soit \(\varphi \in \classe{k}(I, \R)\). La courbe définie par l'équation cartésienne \(y=\varphi(x)\) peut être représentée paramétriquement par \(\fonction{f}{I}{\R^2}{t}{(t, \varphi(t))}\). L'arc \((I, f)\) ainsi défini est de classe \(\classe{k}\). De plus, c'est un arc régulier car pour tout \(t \in I\), \(f'(t)=(1, \varphi'(t))\) qui n'est jamais nul.

De même, si \(\psi \in \classe{k}(I, \R)\), la courbe \(\Gamma\) d'équation cartésienne \(x=\psi(y)\) peut être paramétrée par \(\fonction{g}{I}{\R^2}{t}{(\psi(t), t)}\). La courbe est régulière parce que pour tout \(t \in I\), \(g'(t) \neq (0, 0)\). 

Toute représentation cartésienne admet une représentation paramétrique. Pourtant une courbe définie par une représentation paramétrique peut ne pas forcément s'écrire à l'aide d'une équation cartésienne de la forme \(y=\varphi(x)\) ou \(x=\psi(y)\) ---~un contre-exemple pertinent est le cercle. Sous certaines conditions, on peut trouver au moins localement une équation cartésienne de ce type pour certaines courbes paramétrées ---~théorème des fonctions implicites.

\subsection{Représentation cartésiennes et polaires}

Soient \(\rho\) et \(\theta\) deux applications de \(I\) vers \(\R\) de classe \(\classe{k}\). On peut définir un arc \((I, f)\) par \(\fonction{f}{I}{\R^2}{t}{\rho(t)\vect{u_{\theta(t)}}}\).

Réciproquement, soit \(f\) de \(I\) vers \(\R^2\) une fonction de classe \(\classe{k}\) qui  ne s'annule pas sur \(I\). On pose \(\fonction{\rho}{I}{\R}{t}{\sqrt{f_1^2(t)+f_2^2(t)}}\), où \(f=(f_1, f_2)\), et on voit qu'elle est de classe \(\classe{k}\) (car \(f_1\) et \(f_2\) le sont) et demeure strictement positive. 

On pose \(g=f/\rho\), alors \(g\) est de classe \(\classe{k}\) et \(\norme{g}=1\). En appliquant le théorème du relèvement~: il existe une fonction \(\theta\) de \(I\) vers \(\R\) de classe \(\classe{k}\) telle que \(g=\e^{\ii \theta}\), donc \(f=\rho \e^{\ii \theta}\).

\subsection{Paramétrage admissible}

\begin{defdef}
  Soient \((I, f)\) et \((J, g)\) deux arcs paramétrés de classe \(\classe{k}\). On dit que \((J, g)\) est un paramétrage admissible de classe \(\classe{k}\) de l'arc \((I, f)\) s'il existe une bijection \(\varphi\) de \(I\) vers \(J\) de classe \(\classe{k}\) telle que \(\varphi^{-1}\) est aussi de classe \(\classe{k}\) vérifiant \(f=g \circ \varphi\).

L'application \(\varphi\) est appelée changement de paramétrage admissible de classe \(\classe{k}\).
\end{defdef}

\emph{Remarque~:} Une telle application \(\varphi\) est un \(\classe{k}\)-difféomorphisme.

\begin{prop}
  Soit \(\varphi\) de \(I\) vers \(J\) un changement de paramétrage admissible de classe \(\classe{k}\), alors \(\varphi\) est strictement monotone.
\end{prop}
\begin{proof}
\(\varphi\) et \(\varphi^{-1}\) sont de classe \(\classe{1}\) donc \(\varphi\) est dérivable et \(\varphi'\) ne s'annule pas ---~car \(\varphi^{-1}\) est dérivable. De plus \(\varphi'\) est continue, et comme elle ne s'annule pas alors elle est de signe constant strictement.
\end{proof}

\emph{Conséquence~:} Les arcs \((I, f)\) et \((J, g)\) ont le même support. Lorsque l'application changement de paramétrage \(\varphi\) est strictement croissante, les courbes sont parcourues dans le même sens.

\begin{defdef}
  Orienter un arc paramétrer revient à choisir un des paramétrages et à ne travailler qu'avec des paramétrages de même sens, c'est-à-dire obtenus avec des changements de paramétrages admissibles strictement croissants.
\end{defdef}

\emph{Remarque~:} On peut montrer que les points singuliers, réguliers et biréguliers sont invariants par changement de paramétrage admissibles de classe \(\classe{k}\).

\section{Repère de Frenet}

Soit \((I, f)\) un arc paramétré régulier de classe \(\classe{k}\) et orienté.

\subsection{Vecteur unitaire tangent}

\begin{defdef}
  Pour tout \(t_0 \in I\), on pose \(\vect{T}(t_0) = \frac{\vect{f'(t_0)}}{\norme{\vect{f'(t_0)}}}\). C'est le vecteur unitaire tangent à l'arc \((I, f)\) à la date \(t_0\).
\end{defdef}

\emph{Remarque}~: \(\vect{T}(t_0)\) est bien défini car \(f\) est de classe \(\classe{1}\) donc \(\vect{f'(t_0)}\) existe. De plus l'arc est régulier donc \(\vect{T}(t_0)\neq \vect{0}\).

On peut aussi écrire \(\vect{T}(t_0) = \lim\limits_{t \to t_0} \frac{\vect{f(t)}-\vect{f(t_0)}}{\norme{\vect{f(t)}-\vect{f(t_0)}}}\). Le vecteur \(\vect{T}(t_0)\) est un vecteur unitaire qui dirige la tangente à la courbe en \(M(t_0)\). Il est invariant par changement de paramétrage admissible de même orientation (\(\varphi\) strictement croissante) et changé en son opposé par un changement de paramétrage admissible d'orientation contraire (\(\varphi\) strictement décroissante). L'application \(\fonction{\vect{T}}{I}{\R^2}{t_0}{\vect{T}(t_0)}\) est de classe \(\classe{k}\), car \(f\) est de classe \(\classe{k}\) et \(f'\) ne s'annule pas.

\subsection{Vecteur normal unitaire}

\begin{defdef}
  Pour tout \(t_0 \in I\), on appelle vecteur normal à l'arc \((I, f)\) à la date \(t_0\) l'unique vecteur \(\vect{N}{t_0}\) tel que \((\vect{T}(t_0), \vect{N}(t_0))\) soit une base orthonormale directe de \(\R^2\). Autrement dit \(\vect{N}(t_0)\) est l'image de \(\vect{T}(t_0)\) par la rotation vectorielle d'angle \(\frac{\pi}{2}\) et de centre \(M(t_0)\).
\end{defdef}

\emph{Remarque}~: \(\vect{N}(t_0)\) est invariant par changement de paramétrage admissible de même orientation (\(\varphi\) strictement croissante) et changé en son opposé par un changement de paramétrage admissible d'orientation contraire (\(\varphi\) strictement décroissante).

\subsection{Repère de Frenet}

\begin{defdef}
  On appelle repère de Frenet à la date \(t_0\) le repère orthonormal direct \((M(t_0), \vect{T}(t_0), \vect{N}(t_0))\). Par abus de notation on appelle parfois repère de Frenet la base orthonormale directe \((\vect{T}(t_0), \vect{N}(t_0))\).
\end{defdef}

\section{Abscisse curviligne}

\subsection{Définition}

\begin{defdef}
  Soit \((I, f)\) un arc paramétré de classe \(\classe{k}\) régulier. On appelle abscisse curviligne toute application \(s \in \R^I\) telle que pour tout \(t \in I\), on a \(s'(t)=\norme{f'(t)}\).
\end{defdef}

Par hypothèse, l'application \(f'\) est continue et ne s'annule pas. Alors la fonction \(s'\) est continue strictement positive. Les abscisses curvilignes sont de la forme
\begin{equation}
  \fonction{s}{I}{\R}{t}{s_0 + \int_{t_0}^t \norme{f'(u)}\diff u \quad (s_0, t_0) \in \R \times I}.
\end{equation}
Si \(s_0=0\), on dira que l'origine des dates est prise en \(t_0\). Les abscisses curvilignes différent d'une constantes. On en déduit que pour toutes dates \(t_1\) et \(t_2\) on a
\begin{equation}
  s(t_2)-s(t_1) = \int_{t_1}^{t_2} \norme{f'(t)}\diff t.
\end{equation}

\subsection{Propriétés}

\begin{theo}
  Soit \((I,f)\) un arc paramétré de classe \(\classe{k}\) régulier et orienté. Soit \(s \in \R^I\) une abscisse curviligne de \((I, f)\). Alors \(s\) induit un changement de paramétrage admissible de classe \(\classe{k}\) strictement croissant.
\end{theo}
\begin{proof}
  Soit \(J=s(I)\). Par définition, \(s\) est dérivable et pour tout instant \(t \in I\), on a \(s'(t)=\norme{f'(t)}>0\). La fonction \(s\) est continue, strictement croissante et \(I\) sur \(J\). Donc \(s\) induit une bijection de \(I\) sur \(J\). La fonction \(s\) est de classe \(\classe{k}\) et comme \(s'\) ne s'annule pas, \(s^{-1}\) est aussi de classe \(\classe{k}\). Finalement \(s\) est un changement de paramétrage admissible strictement croissant.
\end{proof}

\begin{corth}
  Soit \((I,f)\) un arc paramétré de classe \(\classe{k}\) régulier et orienté. Soit \(s\) une abscisse curviligne de l'arc \((I,f)\). Alors \((s(I), f\circ s^{-1})\) définit un paramétrage admissible de classe \(\classe{k}\) de \((I, f)\) de même orientation. 

Ce paramétrage s'appelle paramétrage par l'abscisse curviligne.
\end{corth}

\subsection{Formulaire}

On déduit des calculs d'abscisses curvilignes effectués précédemment les formules suivantes.
\begin{itemize}
\item Si l'arc est régulier et représenté paramétriquement par \((I, f)\) avec \(f=(x,y)\) où \(x\) et \(y\) sont des fonctions au moins de classe \(\classe{1}\), alors pour tout \(a\) et \(b\) dans \(I\), on a \(\ell(\intervalleff{a}{b},f)=\int_a^b\sqrt{x'(t)^2+y'(t)^2}\diff t\).
\item En particulier si la courbe est définie par une équation cartésienne de la forme \(y=\varphi(x)\) avec \(\varphi\) une fonction de classe \(\classe{1}\) (au moins), alors pour tout \(a\) et \(b\) dans \(I\), on a \(\ell(\intervalleff{a}{b},f)=\int_a^b\sqrt{1+\varphi'(t)^2}\diff t\).
\item Si la courbe est représentée paramétriquement par \(\vect{f}(t)=\rho(t)\vect{u_{\theta(t)}}\) avec \(\rho\) et \(\theta\) des fonctions de classe \(\classe{1}\) (au moins), alors pour tout \(a\) et \(b\) dans \(I\), on a \(\ell(\intervalleff{a}{b},f)=\int_a^b\sqrt{\rho'(t)^2+\rho(t)^2\theta'(t)^2}\diff t\).
\item En particulier si la courbe est définie par une équation polaire \(r=\rho(\theta)\) avec \(\rho\) une fonction de classe \(\classe{1}\) (au moins), alors pour tout \(\theta_1\) et \(\theta_2\) dans \(I\), on a \(\ell(\intervalleff{\theta_1}{\theta_2})=\int_{\theta_1}^{\theta_2}\sqrt{\rho'(\theta)^2+\rho(\theta)^2}\diff \theta\).
\end{itemize}

\section{Courbure}

On se place dans le plan \(\R^2\) muni de la base orthonormée directe \((\vi, \vj)\). Soit \(I\) un intervalle réel et \((I,f)\) un arc paramétré de classe \(\classe{k}\) (\(k>1\)) supposé régulier et orienté. 

On dispose d'une application abscisse curviligne \(s\) qui induit un changement de paramétrage admissible de classe \(\classe{k}\) de \(I\) vers \(J=s(I)\), dont on note \(\varphi\) la bijection réciproque.

L'arc \((J, g)\) avec \(g=f \circ \varphi\) est un paramétrage admissible de \((I,f)\) qui à la même orientation et qui est normal.

\subsection{Application angulaire \(\alpha\)}

\begin{prop}
  Il existe une application \(\alpha \in \classe{k-1}(I, \R)\) telle que pour tout réel \(t \in I\), on a
  \begin{equation}
    \vect{T}(t) = \vect{u_{\alpha(t)}} = \cos(\alpha(t)) \vi + \sin(\alpha(t)) \vj,
  \end{equation}
  où \(\vect{T}(t)\) est le vecteur unitaire tangent à la courbe à l'instant \(t\). L'application \(\alpha\) est l'application angulaire.
\end{prop}
\begin{proof}
  Pour tout \(t \in I\), on a \(\norme{\vect{T}(t)}=1\). L'application \(\fonction{\vect{T}}{I}{\R^2}{t}{\vect{T}(t)}\) est de classe \(\classe{k-1}\) et par application du théorème du relèvement, il existe une application \(\alpha \in \classe{k-1}(I, \R)\) telle que pour tout réel \(t \in I\), on a
  \begin{equation}
    \vect{T}(t) = \vect{u_{\alpha(t)}} = \cos(\alpha(t)) \vi + \sin(\alpha(t)) \vj.
  \end{equation}
\end{proof}

L'application \(\alpha\) représente une mesure d'angle de la rotation qui transforme \(\vi\) en \(\vect{T}(t)\).

\emph{Conséquence}~: Pour tout réel \(t \in I\), notons \(\vect{N}(t)\) le vecteur normal unitaire tel que \((\vect{T}(t), \vect{N}(t))\) soit une base orthonormée directe. Alors 
\begin{equation}
  \vect{T}(t) = \vect{v_{\alpha(t)}} = -\sin(\alpha(t)) \vi + \cos(\alpha(t)) \vj.
\end{equation}

\subsection{Formulaire}

Si l'arc est défini paramétriquement par \((I,f)\) avec \(f=(x,y)\), \(x\) et \(y\) des fonctions de classe \(\classe{k}\) alors pour tout \(t \in I\), on a
\begin{align}
  \vect{T}(t) &= \vect{f'(t)}{s'(t)} = \frac{x'(t)\vi+y'(t)\vj}{\sqrt{x'(t)^2+y'(t)^2}} \\
  \vect{N}(t) &= \frac{-y'(t)\vi+x'(t)\vj}{\sqrt{x'(t)^2+y'(t)^2}}.
\end{align}
Ainsi
\begin{equation}
  \cos(\alpha(t)) = \frac{x'(t)}{s'(t)} \quad \sin(\alpha(t)) = \frac{y'(t)}{s'(t)}.
\end{equation}

Si l'arc, supposé régulier, est défini par une équation polaire \(r=\rho(\theta)\) avec \(\rho\) une application de classe \(\classe{1}\) alors pour tout \(\theta \in I\)
\begin{align}
  \vect{T}(t) &= \vect{f'(t)}{s'(t)} = \frac{\rho'(\theta)\vu_\theta+\rho(\theta)\vv_\theta}{\sqrt{\rho'(\theta)^2+\rho(\theta)^2}} \\
  \vect{N}(t) &= \frac{-\rho(\theta)\vv_\theta+\rho'(\theta)\vu_\theta}{\sqrt{\rho'(\theta)^2+\rho(\theta)^2}}.
\end{align}
En pratique, pour déterminer \(\alpha(\theta)\) on commence par trouver l'angle entre \(\vu_\theta\) et \(\vect{T}\).

\subsection{Courbure}

\subsubsection{Définition}

On appelle courbure de l'arc \((I, f)\) l'application \(\gamma : J=s(I) \longrightarrow \R\) définie par \(\gamma=\derived{\alpha}{s}\). Plus précisément, on dispose de~:
\begin{itemize}
\item \(\alpha \in \classe{k-1}(I, \R)\) ;
\item \(\fonction{s}{I}{J}{t}{s}\);
\item \(\fonction{\varphi}{J}{I}{s}{t}\).
\end{itemize}

Soit \(t \in I\), alors \(t=\varphi(s)\) et \(\vect{f}(t) = \vect{g}(s)\). Ainsi \(\alpha(t) = \alpha \circ \varphi(s)\). L'application \(\alpha \circ \varphi\) est au moins de classe \(\classe{k-1}\).

On définit \(\gamma=(\alpha \circ \varphi)'\). Pour tout \(s \in J\), on a
\begin{align}
  \gamma(s) &=(\alpha \circ \varphi)'(s) \\
  &=\varphi'(s) \alpha'(\varphi(s))\\
  &=\frac{1}{s'(\varphi(s))} \alpha'(\varphi(s))\\
  &=\frac{\alpha'(t)}{s'(t)}.
\end{align}

Pour tout \(s \in J\), on a \(t=\varphi(s)\) alors
\begin{align}
  \vect{g}(s) &= \vect{f}(t) = \vect{f \circ \varphi}(s) \\
  \vect{g'}(s) &= \varphi'(s) \vect{f' \circ \varphi}(s) \\
  &=\frac{1}{s'(t)} f'(t) =\vect{T}(t).
\end{align}
De plus \(\vect{T}(t) = \vu_{\alpha\circ \varphi(s)}\), donc \(\vect{g''}(s)=\gamma(s) \vv_{\alpha\circ \varphi(s)}=\gamma(s) \vv_{\alpha(t)}\). Au final \(\vect{g''}(s)=\gamma(s) \vect{N}(t)\).

On remarque que \(\gamma(s)=\Det(\vect{g'}(s), \vect{g''}(s))\).

\subsubsection{Points biréguliers}

\begin{prop}
  Le point \(M(t)\) de l'axe \((I, \vect{f})\) est birrégulier si et seulement si \(\gamma(s)\neq 0\) (avec \(t=\varphi(s)\)).
\end{prop}
\begin{proof}
  \begin{align}
    M(t) \text{~est birrégulier} &\iff (\vect{f'}(t), \vect{f''}(t)) \text{~est libre}\\
    &\iff (\vect{g'}(s), \vect{g''}(s)) \text{~est libre}\\
    &\iff \gamma(s)=\Det(\vect{g'}(s), \vect{g''}(s)) \neq 0.
  \end{align}
\end{proof}

\emph{Remarque}~: À quoi ressemble un arc dont la courbure est nulle ? C'est un arc inclus dans une droite.

\begin{defdef}
  Si l'arc \((I,f)\) est birrégulier, on défini pour tout \(t \in I\) le rayon de courbure à la date \(t\) par \(R(t) = \frac{1}{\gamma(s)}\) avec \(t=\varphi(s)\).
\end{defdef}

\emph{Remarque}~: La définition est légitime car en un point birrégulier la courbure est non nulle.

\begin{defdef}
  Si l'arc \((I,f)\) est birrégulier, on définit le centre de courbure à la date \(t\) comme étant le point \(I\) tel que
  \begin{equation}
    \vect{M(t)I} = R \vect{N}(t).
  \end{equation}
\end{defdef}
L'ensemble des centres de courbures de l'arc \((I, f)\) s'appelle la développée de l'arc \((I, f)\).

\subsubsection{Formules de Frenet}

Soient \(s \in J\) et \(t \in I\) tels que \(t=\varphi(s)\) et \(f(t)=g(s)\). Alors
\begin{align}
  \derived{\vect{T}}{s} = \gamma(s) \vect{N} \\
  \derived{\vect{N}}{s} = -\gamma(s)\vect{T}=-\frac{\vect{T}}{R}.
\end{align}

\subsection{Application écart angulaire}

On dispose de \(\alpha \in \classe{k-1}(I, \R)\). On a défini la courbure par
\begin{equation}
  \forall s \in J \quad \gamma(s) = (\alpha \circ \varphi)'(s) ) \varphi'(s)\alpha'(t).
\end{equation}

Si on suppose que l'arc est birrégulier, la courbure ne s'annule pas ; la dérivée de \(\alpha\) ne s'annule pas. \(\alpha'\) est donc de signe constant et continue. Ainsi \(\alpha\) induit une bijection de \(I\) vers \(K=\alpha(I)\). De plus \(\alpha^{-1}\) est dérivable et est de classe \(\classe{k-1}\). 

Au final \(\alpha\) induit un changement de paramétrages admissible de classe \(\classe{k-1}\). Les formules de Frenet deviennent~:
\begin{equation}
  \derived{\vect{T}}{\alpha}=\vect{N} \qquad \derived{\vect{N}}{\alpha}=-\vect{T}.
\end{equation}

\subsection{Accélération et vitesse dans le repère de Frenet}

Si \(\vect{OM}(t)\) représente la position d'un point mobile à l'instant \(t\), alors (en notant\(v=\derived{s}{t}\)) la vitesse vaut
\begin{equation}
  \derived{\vect{OM}}{t} = v \vect{T},
\end{equation}
et l'accélération vaut
\begin{equation}
  \deriveds{\vect{OM}}{t} = \derived{v}{t} \vect{T} + \frac{v^2}{R} \vect{N}.
\end{equation}
