\chapter{Automorphismes orthogonaux}
\label{chap:automorphismesorthogonaux}
\minitoc
\minilof
\minilot

\section{Automorphismes orthogonaux, groupes \(\Orth{E}\) et \(\SOrth{E}\)}

\subsection{Conservation du produit scalaire et de la norme euclidienne}

Soit \((E,\varphi)\) un espace vectoriel euclidien de dimension \(n \in \N^*\).

\begin{defdef}
  Soit \(u \in E^E\).
  \begin{enumerate}
    \item On dit que \(u\) conserve le produit scalaire si et seulement si pour 
      tout couple de vecteurs \((x,y) \in E^2\) on a
      \begin{equation}
        \prodscal{u(x)}{u(y)}=\prodscal{x}{y}.
      \end{equation}
    \item On dit que \(u\) conserve la norme euclidienne si et seulement si pour 
      tout couple de vecteurs \((x,y) \in E^2\) on a
      \begin{equation}
        \norme{u(x)} = \norme{x}.
      \end{equation}
  \end{enumerate}
\end{defdef}
\begin{theo}
  Soit \(u \in E^E\). Il y a équivalence entre les assertions suivantes~:
  \begin{enumerate}
    \item \(u \in \Endo{E}\) et \(u\) conserve la norme euclidienne;
    \item \(u \in \Endo{E}\) et \(u\) conserve le produit scalaire;
    \item \(u\) conserve le produit scalaire.
  \end{enumerate}
\end{theo}
\begin{proof}
  \(1 \implies 2\). L'application est déjà linéaire. Montrons qu'elle conserve 
  le produit scalaire. Soit \((x,y) \in E^2\), alors
  \begin{align*}
    \prodscal{u(x)}{u(y)} &= \frac{1}{2}\left(\norme{u(x)+u(y)}^2 - 
    \norme{u(x)}^2 - \norme{u(y)}^2\right) && \text{Polarisation} \\
    &=\frac{1}{2}\left(\norme{u(x+y)}^2 - \norme{u(x)}^2 - \norme{u(y)}^2\right) 
    && u \in \Endo{E} \\
    &=\frac{1}{2}\left(\norme{x+y}^2 - \norme{x}^2 - \norme{y}^2\right) && u 
    \text{~conserve la norme} \\    &=\prodscal{x}{y}.
  \end{align*}
  Ainsi \(u\) conserve le produit scalaire.

  \(2 \implies 3\). Trivial

  \(3 \implies 1\). Soit \(x \in E\). On a
  \begin{align*}
    \norme{u(x)} &= \sqrt{\prodscal{u(x)}{u(x)}}\\
    &=\sqrt{\prodscal{x}{x}} && u\text{~conserve le produit scalaire}\\
    &=\norme{x}.
  \end{align*}
  Alors \(u\) conserve la norme. 

  Montrons qu'elle est linéaire. Soient \((x,x') \in E^2\) et \(\lambda \in 
  \R\). On veut montrer que
  \begin{equation}
    u(\lambda x+x')=\lambda u(x)+u(x').
  \end{equation}
  Posons \(p=\norme{u(\lambda x+x')-\lambda u(x)-u(x')}\). Il faut montrer que 
  \(p\) est nul.

  En effet,
  \begin{align*}
    p^2 &=\prodscal{u(\lambda x+x')-\lambda u(x)-u(x')}{u(\lambda x+x')-\lambda 
    u(x)-u(x')} \\
    &=\prodscal{u(\lambda x+x')}{u(\lambda x+x')} - \lambda\prodscal{u(\lambda 
    x+x')}{u(x)} - \prodscal{u(\lambda x+x')}{u(x')} \notag \\
    &- \lambda\prodscal{u(x)}{u(\lambda x+x')} +\lambda^2\prodscal{u(x)}{u(x)} 
    +\lambda\prodscal{u(x)}{u(x')} \notag \\
  &-\prodscal{u(x')}{u(\lambda 
  x+x')}+\lambda\prodscal{u(x)}{u(x)}+\prodscal{u(x')}{u(x')}.    \end{align*}
  L'application \(u\) conserve le produit scalaire, alors
  \begin{align*}
    p^2&=\prodscal{\lambda x+x'}{\lambda x+x'} - \lambda\prodscal{\lambda 
    x+x'}{x} - \prodscal{\lambda x+x'}{x'} \notag \\
    &- \lambda\prodscal{x}{\lambda x+x'} +\lambda^2\prodscal{x}{x} 
    +\lambda\prodscal{x}{x'} \notag \\
    &-\prodscal{x'}{\lambda x+x'}+\lambda\prodscal{x}{x}+\prodscal{x'}{x'} \\
    &=\prodscal{\lambda x+x'}{\lambda x +x'-\lambda x-x'} +\lambda 
    \prodscal{x}{-(\lambda x+x')+\lambda x+x'} \notag\\
    & +\prodscal{x'}{-(\lambda x+x')+\lambda x+x'} \\
    &=0.
  \end{align*}
  Donc \(p=0\), alors \( u(\lambda x+x')=\lambda u(x)+u(x')\). L'application 
  \(u\) est linéaire.
\end{proof}

\subsection{Automorphismes orthogonaux d'un espace vectoriel euclidien}

\begin{defdef}
  Soit \((E,\varphi)\) un espace vectoriel euclidien de dimension \(n\). On 
  appelle automorphisme orthogonal de \(E\) toute automorphisme linéaire de 
  \(E\) qui conserve le produit scalaire.

  On note \(\Orth{E}\) l'ensemble des authomorphismes orthogonaux de \(E\).
\end{defdef}
\begin{equation}
  \label{eq:O0}
  \forall u \in E^E \quad u \in \Orth{E} \iff u \in \GL{E} \text{~et~} u 
  \text{~conserve le produit scalaire} \tag{\(O_0\)}.
\end{equation}
\begin{theo}
  Soit \(u \in E^E\). Il y a équivalence entre les assertions suivantes~:
  \begin{enumerate}
    \item[- \(O_0\)] \(u\) est un automorphisme orthogonal;
    \item[- \(O_1\)] \(u\) est linéaire et conserve le produit scalaire;
    \item[- \(O_2\)] \(u\) est linéaire et conserve la norme euclidienne;
    \item[- \(O_3\)] \(u\) conserve le produit scalaire;
    \item[- \(O_4\)] \(u\) est un automorphisme de \(E\) et conserve la norme.
  \end{enumerate}
\end{theo}
\begin{proof}
  D'après le premier paragraphe, on a déjà \(O_1 \iff O_2 \iff O_3\) et \(O_0 
  \iff O_4\). De plus \(O_0 \implies O_1\) est évident. 

  Il nous reste à montrer que \(O_1 \implies O_0\)~: Déjà \(u\) est linéaire et 
  conserve le produit scalaire, donc la norme. Comme \(E\) est de dimension 
  finie (car euclidien), il suffit de montrer que \(u\) est injective pour 
  démontrer qu'elle est bijective.

  Soit \(x \in E\), alors
  \begin{align*}
    x \in \Ker(u) &\iff u(x)=0 \\
    &\iff \norme{u(x)} = 0 \\
    &\iff \norme{x}=0 \\
    &\iff x=0.
  \end{align*}
  Donc le noyau est nul, l'application \(u\) est donc injective. Comme \(E\) est 
  de dimension finie, \(u\) est bijective.

  Finalement \(O_1 \iff O_0\). Toutes les équivalences sont démontrées.
\end{proof}
\begin{theo}
  Soit \(u \in E^E\). Il y a équivalence entre les assertions suivantes~:
  \begin{enumerate}
    \item[- \(O_0\)] \(u\) est un automorphisme orthogonal;
    \item[- \(O_5\)] \(u\) est linéaire et pour toute base orthonormée \(\B\) de 
      \(E\), \(u(\B)\) est une base orthonormée de \(E\);
    \item[- \(O_6\)] \(u\) est linéaire et il existe une base orthonormée 
      \(\B_0\) de \(E\) telle que \(u(\B_0)\) est une base orthonormée de \(E\).
  \end{enumerate}
\end{theo}
\begin{proof}
  On peut démontrer ces équivalences de manières circulaire.

  \(O_0 \implies O_5\). Soit \(\B=(b_1, \dotsc, b_n)\) une base orthonormée de 
  \(E\). D'une part \(\Card u(\B)=\Card \B=\Dim E\). D'autre part pour tout 
  \((i,j) \in \intervalleentier{1}{n}^2\), on a
  \begin{equation}
    \prodscal{u(b_i)}{u(b_j)}=\prodscal{b_i}{b_j}=\delta_{ij}.
  \end{equation}
  Donc \(u(\B)\) est une base orthonormée de \(E\).

  \(O_5 \implies O_6\). On a montré qu chapitre précédent qu'il existait des 
  base orthonormées de \(E\). Il suffit d'en choisir une~: \(\B_0\) et de lui 
  appliquer \(O_5\).

  \(O_6 \implies O_0\). On va plutôt montrer \(O_2\)~\footnote{car \(O_2 \iff 
  O_0\)}. Déjà on sait que \(u\) est linéaire. Soit \(\B_0 = (b_1, \dotsc, 
  b_n)\) une base orthonormée de \(E\). Soit \(x \in E\) de coordonnées \((x_1, 
  \dotsc, x_n) \in \R^n\) dans \(\B_0\). Alors
  \begin{equation}
    \norme{x} = \sqrt{\sum_{i=1}^n x_i^2},
  \end{equation}
  car \(\B_0\) est une base orthonormée. Ensuite
  \begin{equation}
    u(x) = \sum_{i=1}^n x_i u(b_i),
  \end{equation}
  car \(u\) est linéaire. Enfin
  \begin{equation}
    \norme{u(x)} = \sqrt{\sum_{i=1}^n x_i^2}=\norme{x},
  \end{equation}
  car \(u(\B_0)\) est une base orthonormée. Finalement \(u\) est linéaire et 
  conserve la norme.
\end{proof}
\begin{prop}
  Soit \(u \in \Orth{E}\). Soit \(F\) un sous-espace vectoriel de \(E\) stable 
  par \(u\)~\footnote{c'est-à-dire que \(u(F)\subset F\)}. Alors l'orthogonal de 
  \(F\), \(F^\perp\), est aussi stable par \(u\).
\end{prop}
\begin{proof}
  On sait que \(u(F) \subset F\), et de plus \(\Dim(u(F)) = \Dim F\) (car \(u\) 
  est un isomorphisme) donc \(u(F)=F\).

  Soit \(x \in F^\perp\), montrons que \(u(x) \in F^\perp\). Soit \(y \in F\), 
  il existe \(z \in F\) tel que \(y=u(z)\) (car \(u(F)=F\)). Alors
  \begin{align*}
    \prodscal{u(x)}{y} &= \prodscal{u(x)}{u(z)} \\
    &= \prodscal{x}{z} && u \in \Orth{E} \\
    &=0. && x \in F^\perp \ z \in F
  \end{align*}
  Donc \(u(x) \perp y\) et \(y \in F\), alors \(u(x) \in F^\perp\).
\end{proof}

\subsection{Exemple d'automorphisme orthogonal~: les symétries orthogonlaes}

\subsubsection{Symétries orthogonales}

\begin{defdef}
  Soit \((E,\varphi)\) un espace vectoriel euclidien. On appelle symétrie 
  orthogonale de \(E\) toute symétrie de \(E\) dont le projecteur associé est un 
  projecteur orthogonal.
\end{defdef}

Soit \(s \in E^E\). L'application \(s\) est une symétrie orthogonale si et 
seulement si \(s \in \Endo{E}\), \(s \circ s = \Id\) et si on note 
\(p=\frac{1}{2}(s+\Id)\) le projecteur associé on a \(\Ker(p) \perp \Image(p)\). 
Ce qui est équivalent à dire que \(s \in \Endo{E}\), \(s \circ s = \Id\) et 
\(\Inv(s)\perp \Opp(s)\) (\(S_0\)).

\emph{Vocabulaire}~: Si \(F\) est un sous-espace vectoriel de \(E\), la symétrie 
orthogonale dont le projecteur associé est \(p_F\)\footnote{un projecteur 
orthogonal sur \(F\) est un projecteur sur \(F\) parallélement à \(F^\perp\)} 
sera notée \(s_F\) et appelée symétrie orthogonlae par rapport à \(F\).

\begin{theo}
  Soit \(s \in E^E\). Les assertions suivantes sont équivalentes~:
  \begin{enumerate}
    \item[- \(S_0\)] \(s\) est une symétrie orthogonale de \(E\);
    \item[- \(S_1\)] \(s\in \Endo{E}\), \(s \circ s=\Id\) et quelque soit 
      \((x,y) \in E^2\), on a \(\prodscal{s(x)}{y}=\prodscal{x}{s(y)}\);
    \item[- \(S_2\)] \(s\in \Endo{E}\), \(s \circ s=\Id\) et quelque soit \(x 
      \in E^2\), on a \(\norme{s(x)}=\norme{x}\);
    \item[- \(S_3\)] \(s\in \Endo{E}\), \(s \circ s=\Id\) et quelque soit 
      \((x,y) \in E^2\), on a \(\prodscal{s(x)}{s(y)}=\prodscal{x}{y}\);
    \item[- \(S_4\)] \(s\in \Orth{E}\) et \(s \circ s=\Id\).
  \end{enumerate}
\end{theo}
\begin{proof}
  \(S_0 \implies S_1\). Par définition, \(s \in \Endo{E}\) et \(s \circ s=\Id\). 
  Posons \(p=\frac{1}{2}(s+\Id)\). Pour tout couple \((x,y) \in E^2\) on a
  \begin{align*}
    \prodscal{s(x)}{y} = \prodscal{2p(x)-x}{y} & = 
    2\prodscal{p(x)}{y}-\prodscal{x}{y} && \text{bilinéarité}\\
    &=2\prodscal{x}{p(y)}-\prodscal{x}{y} && p\text{~est un projecteur 
    orthogonal}\\
    &=\prodscal{x}{2p(y)-y} && \text{bilinéarité}\\
    &=\prodscal{x}{s(y)}.
  \end{align*}

  \(S_1 \implies S_2\). On a déjà \(s \in \Endo{E}\) et \(s \circ s=\Id\). Soit 
  \(x \in E\), alors
  \begin{align*}
    \norme{s(x)}^2 &=\prodscal{s(x)}{s(x)} \\
    &=\prodscal{x}{s\circ s(x)} &&\text{d'après~} S_1 \\
    &=\prodscal{x}{x} && s\circ s=\Id \\
    &\norme{x}^2.
  \end{align*}
  Comme les normes sont positives, on a bien \(\norme{s(x)}=\norme{x}\).

  \(S_2 \implies S_3\). On a déjà \(s \in \Endo{E}\) et \(s \circ s=\Id\). Comme 
  \(s\) conserve la norme, alors elle conserve le produit scalaire.

  \(S_3 \implies S_0\). On a déjà \(s \in \Endo{E}\) et \(s \circ s=\Id\). 
  Montrons que \(\Inv(s)\perp \Opp(s)\). Soit \((x,y) \in \Opp(s) \times 
  \Inv(s)\). Alors
  \begin{align*}
    \prodscal{x}{y} &= \prodscal{s(x)}{-s(y)} && (x,y) \in \Opp(s) \times 
    \Inv(s) \\
    &=-\prodscal{s(x)}{-s(y)} && \text{bilinéarité} \\
    &=-\prodscal{x}{y}. && S_3
  \end{align*}
  Donc \(\prodscal{x}{y}=0\). Finalement \(\Inv(s)\perp \Opp(s)\), \(s\) est une 
  symétrie orthogonale.

\(S_3 \iff S_4\). On a déja \(s \circ s=\Id\) et par caractérisation \(s \in 
\Orth{E}\). \end{proof}

\subsubsection{Réflexions}

\begin{defdef}
  On appelle réflexion de l'espace vectoriel euclidien \((E,\varphi)\) toute 
  symétrie orthogonale de \(E\) par rapport à un hyperplan de \(E\).
\end{defdef}
\begin{prop}[Expression de la réflexion]
  Soient \(H\) un hyperplan de \(E\), \(s_H\) la réflexion par rapport à 
  \(H^\perp\), \(a\) un vecteur normal à \(H\), \(k=\frac{a}{\norme{a}}\). Pour 
  tout \(x \in E\), on a
  \begin{equation}
    s_H(x)=x-2\frac{\prodscal{x}{a}}{\norme{a}^2}a = x-2\prodscal{x}{k}k.
  \end{equation}
\end{prop}
\begin{proof}
  On a \(H^\perp = \VectEngendre(k)\), et on a vu que 
  \(p_{H^\perp}(x)=\prodscal{x}{k}k\), alors
  \begin{align*}
    s_{H}(x)&=2p_H(x)-x\\
    &=2(x-p_{H^\perp}(x))-x\\
    &=x-2p_{H^\perp}(x)\\
    &=x-2\prodscal{x}{k}k.
  \end{align*}
\end{proof}

\begin{theo}
  Soient \(a\) et \(b\) différents dans \(E\) tels que \(\norme{a}=\norme{b}\). 
  Il existe une unique réflexion \(s\) de \(E\) qui échange \(a\) et 
  \(b\)\footnote{c'est-à-dire \(s(b)=a\) et \(s(a)=b\)}. De plus \(s\) est la 
  réflexion par rapport à \(H=\VectEngendre(b-a)^\perp\).
\end{theo}
\begin{proof}[Analyse \& unicité]
  On suppose l'existence d'un hyperplan \(H\) de \(E\) tel que la réflexion 
  \(s\) par rapport à \(H\) vérifi \(s(a)=b\) et \(s(b)=a\). Soit \(c\) un 
  vecteur normal unitaire à \(H\). Alors pour tout \(x \in E\), 
  \(s(x)=x-2\prodscal{x}{c}c\). Alors
  \begin{equation}
    s(a) = b \iff b=a-2\prodscal{a}{c}c \iff b-a = -2\prodscal{a}{c}c.
  \end{equation}
  Comme~:
  \begin{itemize}
    \item \(b-a\) est non nul;
    \item \(b-a\) est colinéaire à \(c\);
    \item \(H^\perp = \VectEngendre(c)\);
  \end{itemize}
  on a \(H^\perp = \VectEngendre(b-a)\). Comme \(E\) est de dimension finie, en 
  passant à l'orthogonal il vient que 
  \(H=(H^\perp)^\perp=(\VectEngendre(b-a))^\perp\).
\end{proof}
\begin{proof}[Synthèse \& Existence]
  Soit \(H=(\VectEngendre(b-a))^\perp\). Ce qui est légitime car \(b \neq a\), 
  et \(H\) est un hyperplan.

  Soit \(s\) la réflexion par rapport à \(H\), et \(b-a\) est un vecteur normal 
  à \(H\), on a
  \begin{equation}
    \forall x \in E \quad s(x) = x -2 
    \frac{\prodscal{x}{b-a}}{\norme{b-a}^2}(b-a).
  \end{equation}
  Calculons \(s(a)\)~: \(s(a) = a -2 
  \frac{\prodscal{a}{b-a}}{\norme{b-a}^2}(b-a)\). Alors
  \begin{align*}
    s(a)-b &=(a-b)\left(1+2\frac{\prodscal{a}{b-a}}{\norme{b-a}^2}\right) \\
    &=\frac{a-b}{\norme{b-a}^2}(\norme{b}^2+\norme{a}^2-2\prodscal{b}{a}+2\prodscal{a}{b}-2\prodscal{a}{a}) 
    \\
    &=\frac{a-b}{\norme{b-a}^2}(2\norme{a}^2-2\norme{a}^2)\\
    &=0.
  \end{align*}
  Donc \(s(a)=b\). D'où \(a=s\circ s(a)=s(b)\). L'application \(s\) convient.
\end{proof}

\begin{defdef}
  Soient \(a\) et \(b\) deux vecteurs distincts de \(E\) tels que 
  \(\norme{a}=\norme{b}\). \(H=\VectEngendre(b-a)^\perp\) est appelé hyperplan 
  médiateur de \(a\) et \(b\)
  \begin{equation}
    H = \enstq{x\in E}{d(x,a)=d(x,b)}.
  \end{equation}
\end{defdef}
\begin{proof}
  Soit \(x \in E\). Alors
  \begin{align*}
    d(x,a)=d(x,b) &\iff \norme{x-a}=\norme{x-b} \\
    &\iff \norme{x-a}^2=\norme{x-b}^2 \\
    &\iff \norme{x}+\norme{a}^2-2\prodscal{x}{a}= 
    \norme{x}+\norme{b}^2-2\prodscal{x}{b}\\
    &\iff \prodscal{x}{b-a}=0\\
    &\iff x \in H=\VectEngendre(b-a)^\perp.
  \end{align*}
\end{proof}

\subsection{Groupes \(\Orth{E}\) et \(\SOrth{E}\)}

Soit \((E,\varphi)\) un espace vectoriel euclidien. On rappelle que \(\Orth{E}\) 
est l'ensemble des automorphismes de \(E\) qui conservent le produit scalaire.

\begin{theo}
  \((\Orth{E},\circ)\) est un sous-groupe du groupe linéaire \((\GL{E},\circ)\).
\end{theo}
\begin{proof}
  Par définition \(\Orth{E} \subset \GL{E}\). \(\Orth{E}\) est non vide puisque 
  \(\Id \in \Orth{E}\). Soient \(u\) et \(v\) dans \(\Orth{E}\), alors on peut 
  définir \(u\circ v\) et \(u \circ v \in \GL{E}\). Montrons que \(u \circ v\) 
  conserve le produit scalaire~: Pour tout \((x,y) \in E^2\), on a
  \begin{align*}
    \prodscal{u\circ v(x)}{u\circ v(y)} &= \prodscal{u(v(x))}{u(v(y))} \\
    &=\prodscal{v(x)}{v(y)} && u \in \Orth{E}
    &=\prodscal{x}{y}. && v \in \Orth{E}
  \end{align*}
  Donc \(u \circ v \in \Orth{E}\). Soit \(u \in \Orth{E}\), alors \(u\) est 
  inversible et \(u^{-1} \in \GL{E}\). Montrons que \(u^{-1}\) conserve le 
  produit scalaire~: Pour tout \((x,y) \in E^2\), on a
  \begin{align*}
    \prodscal{u^{-1}(x)}{u^{-1}(y)}&= \prodscal{u(u^{-1}(x))}{u(u^{-1}(y))} && u 
    \in \Orth{E} \\
    &=\prodscal{x}{y}.
  \end{align*}
  Donc \(u^{-1} \in \Orth{E}\).

  Par caractérisation, \((\Orth{E}, \circ)\) est un sous-groupe de 
  \((\GL{E},\circ)\).
\end{proof}
%
\begin{defdef}
  On appelle rotation de \(E\) toute application \(u \in \Orth{E}\) tel que 
  \(\Det u=1\). On note \(\SOrth{E}\) leurs ensemble~:
  \begin{equation}
    \SOrth{E}=\enstq{u \in \Orth{E}}{\Det u=1}.
  \end{equation}
\end{defdef}
\begin{proof}
  Soit \(\fonction{f}{\Orth{E}}{\R^*}{u}{\Det u}\). L'application \(f\) est 
  bien définie car \(\Orth{E}\subset \GL{E}\), \(f\) est un morphisme de 
  \((\Orth{E}, \circ)\) sur \((\R^*,\times)\) car
  \begin{equation}
    \forall (u,v) \in \Orth{E}^2 \quad \Det(u\circ v)= \Det u \times \Det v.
  \end{equation}
  Par définition, \(\SOrth{E}=\Ker(f)\) donc \(\SOrth{E}\) est un sous-groupe 
  de \((\Orth{E}, \circ)\).
\end{proof}

\emph{Remarque}~: Soient \(H\) un hyperplan de \(E\), \(s\) la réflexion par 
rapport à \(H\). Soient \(\B\) une base orthonormée de \(E\) adaptée à 
\(H\oplus H^\perp\)~:
\begin{equation}
  \B=(b_1, \dotsc, b_{n-1}, b_n).
\end{equation}
Si pour \(j \in \intervalleentier{1}{n-1}\) on a \(s(b_j)=b_j\) et 
\(s(b_n)=-b_n\) alors \(\Det_\B(s)=-1\).

\section{Matrices orthogonales, groupes \(\Orth{n}\) et \(\SOrth{n}\)}

\subsection{Notion de matrice orthogonal}

Soient \(n \in \N^*\), \(A=(a_{ij})_{(i,j) \in \intervalleentier{1}{n}^2} \in 
\Mn{n}{\R}\). On note \(\Lb_A\) la famille des lignes de \(A\) et 
\(\courbe_A\) la famille des colonnes de \(A\). Soit \(u\) l'endomorphisme 
canoniquement associé.

\emph{Calcul de \(A^\top A\) et \(A A^\top\)}

On note \(A^\top = (\alpha_{ij})_{1\leqslant i,j \leqslant n}\), \(A^\top A= 
(c_{ij})_{1\leqslant i,j \leqslant n}\) et \(A A^\top=(d_{ij})_{1\leqslant i,j 
\leqslant n}\). Alors pour tout \((i,j) \in \intervalleentier{1}{n}^2\) on a
\begin{gather}
  \alpha_{ij}=a_{ji};\\
  c_{ij} = \sum_{k=1}^n \alpha_{ik}a_{kj}=\sum_{k=1}^n a_{ki}a_{kj} = 
  \prodscal{\courbe_i}{\courbe_j};\\
  d_{ij} = \sum_{k=1}^n a_{ik}\alpha_{kj}=\sum_{k=1}^n a_{ik}a_{jk} = 
  \prodscal{\Lb_i}{\Lb_j}.
\end{gather}
\begin{theo}
  Soit \(A \in \Mn{n}{\R}\). Les assertions suivantes sont équivalentes~:
  \begin{enumerate}
    \item \(A \in \GLn{n}{\R}\) et \(A^{-1}=A^\top\);
    \item \(A^\top A = I_n\);
    \item \(A A^\top = I_n\);
    \item \(\Lb_A\) est une base orthonormées de \((\R^n, \varphi_c)\);
    \item \(\courbe_A\) est une base orthonormées de \((\R^n, \varphi_c)\);
    \item \(u \in \Orth{\R^n}\).
  \end{enumerate}
\end{theo}
\begin{proof}
  De manière évidente, on voit que \(1 \iff 2 \iff 3\). Ensuite~:

  \(2 \iff 5\). En effet, on a
  \begin{align*}
    A A^\top = I_n &\iff \forall (i,j) \in \intervalleentier{1}{n}^2 \quad 
    \prodscal{\Lb_i}{\Lb_j}=\delta_{ij} \\
    &\iff \Lb_A \text{~BON de~} \R^n.
  \end{align*}

  \(5 \implies 6\). Soit \(\E_c\) la base canonique de \(\R^n\). L'application 
  \(u\) est canoniquement associée à \(A\), donc \(u(\E_c)=\courbe_A\). Les 
  bases \(\E_c\) et \(u(\E_c)=\courbe_A\) sont orthonormées, donc \(u\) (par 
  l'assertion \(O_6\)) est un automorphisme orthogonal~: \(u \in 
  \Orth{\R^n}\).

  \(6 \implies 5\). Soit \(u \in \Orth{\R^n}\) et \(\E_c\) la base canonique 
  de \(\R^n\). Alors (par l'assertion \(O_5\)) \(\courbe_A = u(\E_c)\) est une 
  base orthonormale de \(\R^n\).
\end{proof}

\begin{defdef}
  Une matrice \(A \in \Mn{n}{\R}\) qui vérifie l'une des assertions du 
  théorème précédent est appelée matrice orthogonale. On note \(\Orth{n}\) 
  leur ensemble.
  \begin{equation}
    \Orth{n}=\enstq{A \in \Mn{n}{\R}}{A^\top A = I_n}.
  \end{equation}
\end{defdef}
\begin{prop}
  Pour toute matrice \(A \in \Mn{n}{\R}\), \(A\) est orthogonale si et 
  seulement si \(A^\top\) est orthogonale.
\end{prop}
\begin{proof}
  \begin{align*}
    A \in \Orth{n} &\iff A^\top A = I_n \\
    & \iff A^\top (A^\top)^\top = I_n^\top=I_n \\
    & \iff A^\top \in \Orth{n}.
  \end{align*}
\end{proof}
\begin{prop}
  \begin{equation}
    \forall A \in \Mn{n}{\R} \quad A \in \Orth{n} \implies \Det(A) \in \{-1, 
    1\}.
  \end{equation}
\end{prop}
\begin{proof}
  Soit \(A \in \Orth{n}\), alors
  \begin{equation}
    1 = \Det(I_n) = \Det(A^\top A) = \Det(A)^2,
  \end{equation}
  donc \(\Det(A) \in \{-1, 1\}\).
\end{proof}
\emph{La réciproque est fausse}, par exemple \(A = \begin{pmatrix} 1 & 1 \\ 0 
& 1 \end{pmatrix}\), on a bien \(\Det(A)=1\) mais on a 
\(\prodscal{\courbe_1}{\courbe_2}=1 \neq 0\) donc \(A \notin \Orth{2}\).

\begin{defdef}
  On définit les ensembles suivants~:
  \begin{gather}
    \SOrth{n}   = \enstq{A \in \Orth{n}}{\Det(A)=1}; \\
    \Orthmin{n} = \enstq{A \in \Orth{n}}{\Det(A)=-1}; \\
    \Orthmin{E} = \enstq{u \in \Orth{E}}{\Det(u)=-1}.
  \end{gather}
\end{defdef}

\subsection{Nouvelles caractérisations des automorphismes orthogonaux}

\begin{theo}
  Soit \(E\) un espace vectoriel euclidien, et \(u \in E^E\). Alors il y a 
  équivalence entre les assertions suivantes~:
  \begin{itemize}
    \item[- \(O_0\)] \(u \in \Orth{E}\);
    \item[- \(O_7\)] \(u \in \Endo{E}\), et \(\forall \B\) base orthonormale 
      de E on a \(\Mat_\B(u) \in \Orth{n}\);
    \item[- \(O_8\)] \(u \in \Endo{E}\), et \(\exists \B_0\) base orthonormale 
      de E on a \(\Mat_{\B_0}(u) \in \Orth{n}\).
  \end{itemize}
\end{theo}
\begin{proof}
  \(O_0 \implies O_7\). Déjà \(u \in \Endo{E}\) et pour tout \((x,y)\in E^2\) 
  on a \(\prodscal{u(x)}{u(y)}=\prodscal{x}{y}\). Soit \(\B\) une base 
  orthonormée de E. On note \(A=\Mat_\B(u)\), \(X=\Mat_\B(x)\) er 
  \(Y=\Mat_\B(y)\). Alors \(\prodscal{x}{y}=\X^\top Y\) et 
  \(\prodscal{u(x)}{u(y)} = (AX)^\top AY = X^\top A^\top A Y\). Alors
  \begin{align*}
    X^\top Y &= X^\top A^\top A Y\\
    X^\top(I_n - A^\top A)Y &=0.
  \end{align*}
  Cette égalité est vrai pour tout couple \((x,y) \in E^2\) donc \(A^\top A= 
  I_n\), alors \(A \in \Orth{n}\).

  \(O_7 \implies O_8\). Vrai car \(E\) admet au moins une base orthonormée.

  \(O_8 \implies O_0\). Déjà \(u \in \Endo{E}\). Montrons que \(u(\B_0)\) est 
  une base orthonormée. On a \(\Card(u(\B_0))=\Card(\B_0)=\Dim E\), car la 
  matrice est carrée. On sait que \(A=\Mat_{\B_0}(u)\) est orthogonale. On 
  pose que \(\B_0 = (b_1, \ldots, b_n)\) et pour tout \((i,j) \in 
  \intervalleentier{1}{n}^2\) on a
  \begin{align*}
    \prodscal{u(b_i)}{u(b_j)} &=\prodscal{\sum_{k=1}^n a_{ki}b_k}{\sum_{l=1}^n 
    a_{lj}b_l} \\
    &=\sum_{k=1}^n \sum_{l=1}^n a_{ki} a_{lj} \prodscal{b_k}{b_l} && 
    \text{bilinéarité} \\
    &=\sum_{k=1}^n a_{ki}a_{kj} && \B_0 \text{~est une BON} \\
    &=(A^\top A)_{ij} \\
    &=\delta_{ij}. && A \in \Orth{n}
  \end{align*}
  La base \(u(\B_0)\) est orthonormée, donc \(u \in \Orth{E}\).
\end{proof}

\subsection{Caractérisation des rotations}

Soit \((E,\varphi)\) un espace vectoriel euclidien de dimension \(n \in 
\N^*\).
\begin{theo}
  Soit \(r \in E^E\), alors il y a équivalence entre les assertions 
  suivantes~:
  \begin{itemize}
    \item[- \(R_0\)] \(r \in \SOrth{E}\);
    \item[- \(R_1\)] \(r \in \Endo{E}\) et pour toute base orthonormée directe 
      \(\B\), \(r(\B)\) est aussi orthonormée directe;
    \item[- \(R_2\)] \(r \in \Endo{E}\) et il existe une base orthonormée 
      directe \(\B_0\) telle que \(r(\B_0)\) est aussi orthonormée directe;
    \item[- \(R_3\)] \(r \in \Endo{E}\) et pour toute base orthonormée \(\B\), 
      \(r(\B)\) est aussi orthonormée;
    \item[- \(R_4\)] \(r \in \Endo{E}\) et il existe une base orthonormée 
      \(\B_0\) telle que \(r(\B_0)\) est aussi orthonormée.
  \end{itemize}
\end{theo}
\begin{proof}
  \(R_0 \implies R_1\). L'application \(r \in \Orth{E}\), donc pour toute base 
  orthonormée directe \(\B\) de \(E\), \(r(\B)\) est une base orthonormée de 
  \(E\). De plus \(\Det_\B(r(\B))=\Det(r) =1\), alors \(\B\) et \(r(\B)\) on 
  la même orientation. Ainsi \(r(\B)\) est une base orthonormée directe.

  \(R_1 \implies R_2\). En dimension finie (\(E\) est euclidien) il existe des 
  bases orthonormées directes.

  \(R_2 \implies R_0\). D'après l'assertion \(O_6\), l'hypothèse \(R_2\) 
  implique que \(r \in \Orth{E}\). De plus \(\B_0\) et \(r(\B_0)\) ont la même 
  orientation donc \(\Det(r)>0\). Or on a \(\Det r = \Det_{\B_0}r(\B_0) \in 
  \{-1; 1\}\), alors finalement \(\Det r = 1\) et ainsi \(r \in \SOrth{E}\) 
  (\(r\) est une rotation).

  \(R_0 \implies R_3\). L'application \(r \in \SOrth{E} \subset \Orth{E}\), 
  alors \(r \in \Endo{E}\) et pour toute base orthonormée \(\B\) de \(E\) on a 
  \(\Mat_\B(r) \in \Orth{n}\). De plus \(\Det \Mat_\B(r) = \Det r = 1\), donc 
  \(\Mat_\B(r) \in \SOrth{n}\).

  \(R_3 \implies R_4\). En dimension finie (\(E\) est euclidien) il existe des 
  bases orthonormées.

  \(R_4 \implies R_0\). D'après l'assertion \(O_8\), on sait que \(r \in 
  \Orth{E}\) et \(\Det r = \Det \Mat_{\B_0}(r) =1\) donc \(r \in \SOrth{E}\).
\end{proof}

\subsection{Changement de base orthonormée}

Soit \((E,\varphi)\) un espace euclidien de dimension \(n \in \N^*\). Soit 
\(\B\) une base orthonormée de \(E\).

\begin{prop}
  Soit \(\X\) une famille de \(n\) vecteurs de \(E\). \(\X\) est une base 
  orthonormée de \(E\) si et seulement si \(\Mat_\B(\X) \in \Orth{n}\).
\end{prop}
\begin{proof}
  Soit \(u \in \Endo{E}\) l'unique endomorphisme de \(E\) tel que 
  \(u(\B)=\X\)\footnote{possible car \(\B\) est une base de \(E\) et \(\X\) 
  une famille de \(n\) vecteurs de \(E\).}. Alors
  \begin{equation}
    \Mat_\B(\X) = \Mat_\B(u(\B))=\Mat_\B(u).
  \end{equation}
  \(\X=u(\B)\) est une base orthonormée de \(E\) si et seulement si \(u\) est 
  orthogonale, et donc si et seulement si \(\Mat_\B(u)=\Mat_\B(\X) \in 
  \Orth{n}\).
\end{proof}
\begin{prop}
  Soit \(\X\) une base de \(E\). \(\X\) est une base orthonormée de \(E\) si 
  et seulement si \(\P_{\B\X} \in \Orth{n}\).
\end{prop}
\begin{proof}
  Idem.
\end{proof}
\begin{prop}
  Supposons que \((E,\varphi)\) est un espace vectoriel euclidien orienté et 
  que \(\B\) est une base orthonormée directe de \(E\). Soit \(\E\) une base 
  de \(E\), alors \(\E\) est directe si et seulement si \(\P_{\B\E} \in 
  \SOrth{n}\).
\end{prop}
\begin{proof}
  \begin{align*}
    \E \text{~est directe} &\iff \E \text{~est une base de~} E \text{~et~} 
    \Det_\B(\E) > 0 \\
    &\iff \P_{\B\E} \in \Orth{n} \text{~et~} \Det_\B(\E) > 0 \\
    &\iff \P_{\B\E} \in \SOrth{n}.
  \end{align*}
\end{proof}

\subsection{Groupes \(\Orth{n}\) et \(\SOrth{n}\)}

On rappelle que~:
\begin{gather}
  \Orth{n} = \enstq{A \in \GLn{n}{\R}}{A^\top=A^{-1}}, \\ \SOrth{n}=\enstq{A 
  \in \Orth{n}}{\Det A = 1}.
\end{gather}

Soit \(\B\) une base orthonormée de l'espace vectoriel euclidien \((E, 
\varphi)\). On dispose de l'isomorphisme de groupes 
\(\fonction{\Mat_\B}{\GL{E}}{\GLn{n}{\R}}{u}{\Mat_\B(u)}\).

\begin{lemme}
  \begin{equation}
    \Mat_\B(\Orth{E})=\Orth{n}.
  \end{equation}
\end{lemme}
\begin{proof}
  Si \(u \in \Orth{E}\), alors comme \(\B\) est une base orthonormée, 
  \(\Mat_\B(u) \in \Orth{n}\).

  Si \(A \in \Orth{n}\), alors \(A \in \GLn{n}{\R}\) donc il existe un 
  isomorphisme \(u \in \GL{E}\) tel que \(A=\Mat_\B(u)\). La base \(\B\) est 
  orthonormée et la matrice \(A\) est orthogonale donc \(u \in \Orth{E}\) et 
  donc \(A \in \Mat_\B(\Orth{E})\).
\end{proof}

\emph{Conséquence~:} \(\Orth{E}\) est un sous-groupe de \(\GL{E}\) et 
l'application \(\Mat_\B\) est un morphisme de groupe donc \(\Orth{n}\) est un 
sous-groupe de \(\GLn{n}{\R}\). Ainsi \((\Orth{n}, \cdot)\) est un groupe.

On dispose de \(\fonction{\Mat_\B}{\Orth{E}}{\Orth{n}}{u}{\Mat_\B(u)}\), c'est 
un isomorphisme de groupes\footnote{injectif car c'est la restriction d'un 
isomorphisme et surjectif d'après le lemme.}.

\begin{lemme}
  \begin{equation}
    \Mat_\B(\SOrth{E})=\SOrth{n}.
  \end{equation}
\end{lemme}
\begin{proof}
  Soit \(r \in \SOrth{E}\), alors \(r \in \Orth{E}\) et donc \(\Mat_\B(r) \in 
  \Orth{n}\) (d'après le lemme précédent). Or \(A=\Det r = \Det \Mat_\B(r)\). 
  Ainsi \(\Mat_\B(r) \in \SOrth{n}\).

  Soit \(A \in \SOrth{n}\), alors \(A \in \Orth{n}\) et donc il existe \(u \in 
  \Orth{E}\) telle que \(A = \Mat_\B(u)\) (d'après le lemme précédent). Alors 
  \(1=\Det_\B(A)=\Det u\) et donc \(u \in \SOrth{E}\), c'est-à-dire que \(A 
  \in \Mat_\B (\SOrth{E})\).
\end{proof}

\emph{Conséquence~:} \(\SOrth{n}\) est un sous-groupe de \(\Orth{n}\).
\begin{proof}
  \(\SOrth{E}\) est un sous-groupe de \(\GL{E}\). L'image directe d'un 
  sous-groupe par un morphisme de groupes est un sous-groupe.
\end{proof}

\emph{Remarque~:} On aurait pu utiliser l'application 
\(\fonction{\Det}{\Orth{n}}{\{-1;1\}}{A}{\Det A}\) et 
\(\SOrth{n}=\Ker(\Det)\). De même que pour \(\Orth{E}\) et \(\SOrth{E}\), la 
restriction
\begin{equation}
\fonction{\Mat_\B}{\SOrth{E}}{\SOrth{n}}{u}{\Mat_\B(u)} \end{equation}
est un isomorphisme de groupes. 

\section{Automorphismes orthogonaux du plan euclidien}

\subsection{Matrices de \(\Orth{2}\)}

Soit \(A = 
\begin{pmatrix} 
  a & b \\
  c & d
\end{pmatrix} \in \Mn{2}{\R}
\). On note \(\courbe_A\) la famille des colonnes de \(A\).

\begin{theo}
  \begin{equation}
    \Orth{2} = \enstq{A_{\alpha, \epsilon}=\begin{pmatrix} \cos \alpha & 
      -\epsilon\sin\alpha \\ \sin\alpha & 
    \epsilon\cos\alpha\end{pmatrix}}{(\alpha, \epsilon)\in \R\times\{-1; 
    1\}}.
  \end{equation}
  \(\Orth{2}\) est un sous-groupe non-abélien. De plus pour tout \((\alpha, 
  \epsilon)\in \R\times\{-1; 1\}\), \(\Det A_{\alpha, \epsilon} = 
  \epsilon\).
\end{theo}
\begin{proof}
  Soit \(A=\begin{pmatrix} a & b \\ c & d \end{pmatrix} \in \Mn{2}{\R}\). 
    Alors
    \begin{align*}
      A \in \Orth{2} &\iff \courbe_A \text{~est une BON} \\
      &\iff \begin{cases} a^2+c^2=1 \\ b^2+d^2=1 \\ ab+cd=0 \end{cases}\\
        &\iff \exists (\alpha, \beta) \in \R^2 \quad \begin{cases} 
          a=\cos\alpha, \ c=\sin\alpha \\ b=\sin\beta, \ d=\cos\beta \\ 
          \cos\alpha\sin\beta+\sin\alpha\cos\beta=\sin(\alpha+\beta)=0 
        \end{cases}\\
        &\iff \exists (\alpha, \beta) \in \R^2 \quad \begin{cases} 
          a=\cos\alpha, \ c=\sin\alpha \\ b=\sin\beta, \ d=\cos\beta \\ 
          \congru{\beta}{-\alpha}{2\pi} \\ \text{~ou~} 
        \congru{\beta}{\pi-\alpha}{2\pi} \end{cases}\\
        &\iff \exists \alpha \in \R \quad \begin{cases} a=\cos\alpha, \ 
        c=\sin\alpha \\ b=-\sin\alpha, \ d=\cos\alpha \end{cases} 
        \text{~ou~} \begin{cases} a=\cos\alpha, \ c=\sin\alpha \\ 
        b=\sin\alpha, \ d=-\cos\alpha \end{cases} \\
        &\iff \exists (\alpha, \epsilon) \in \R \times \{-1; 1\} \quad 
        \begin{cases} a=\cos\alpha, \ c=\sin\alpha \\ 
        b=-\epsilon\sin\alpha, \ d=\epsilon\cos\alpha \end{cases}\\
        &\iff \exists (\alpha, \epsilon) \in \R \times \{-1; 1\} \quad 
        A=\begin{pmatrix} \cos \alpha & -\epsilon\sin\alpha \\ \sin\alpha 
        & \epsilon\cos\alpha\end{pmatrix}
    \end{align*}
\end{proof}
\begin{theo}
  \begin{equation}
    \SOrth{2} = \enstq{R_\theta=\begin{pmatrix}\cos\theta & -\sin\theta \\ 
    \sin\theta & \cos\theta\end{pmatrix}}{\theta \in \R},
  \end{equation}
  qui est abélien, et
  \begin{equation}
    \Orthmin{2} = \enstq{S_\varphi=\begin{pmatrix}\cos\varphi & \sin\varphi 
    \\ \sin\varphi & -\cos\varphi \end{pmatrix}}{\varphi \in \R},
  \end{equation}
  qui n'est pas stable par multiplication.
\end{theo}
\begin{theo}
  Pour tout quadruplet \((\theta, \theta', \varphi, \varphi') \in \R^4\), on 
  a
  \begin{gather}
    R_\theta R_{\theta'} = R_{\theta+\theta'}; \\
    S_\varphi S_{\varphi'} = S_{\varphi-\varphi'}; \\
    R_\theta S_\varphi = S_{\theta+\varphi'}; \\
    S_\varphi R_\theta = S_{\varphi-\theta}.
  \end{gather}
\end{theo}
\begin{proof}
  La démonstration découle des formules d'addition trigonométriques.
\end{proof}
\begin{corth}
  L'application \(\fonction{\varphi}{\R}{\SOrth{2}}{\theta}{R_\theta}\) est 
  un morphisme de groupes du groupe \((\R,+)\) sur le groupe 
  \((\SOrth{2},\cdot)\).
\end{corth}
Pour tout réel \(\theta\), on a
\begin{equation}
  \varphi(\theta)=I_2 \iff \begin{cases} \cos\theta =1 \\ 
  \sin\theta=0\end{cases} \iff \theta \in 2\pi\Z
\end{equation}
Le morphisme \(\varphi\) est surjectif mais n'est pas injectif. Pour tout 
réel \(\theta\), \(R_\theta^{-1} = R_{-\theta}\).

\begin{corth}
  Pour tout réel \(\varphi\), \(S_\varphi^2=I_2\) donc 
  \(S_\varphi^{-1}=S_\varphi^\top=S_\varphi\). La matrice \(S_\varphi\) est 
  orthogonale, symétrique et involutive.
\end{corth}

\subsection{Classification des éléments de \(\Orth{E_2}\)}

Soit \((E,\varphi)\) le plan vectoriel euclidien.

\subsubsection{Éléments de \(\Orthmin{E_2}\), les symétries vectorielles}

Soit \(\B\) une base orthonormée de \(E_2\) et \(u \in \Orthmin{E_2}\). 
Alors \(\Det u=-1\) et \(\Mat_\B(u) \in \Orthmin{2}\). Il existe un réel 
\(\varphi\) tel que \(\Mat_\B(u)=S_\varphi\) (\(u \circ u=\Id\)). De plus 
\(u \in \Orth{E_2}\), donc c'est une symétrie orthogonale. 

Dans le cas général, on classe les symétries vectorielles \(u\) de \(E_2\) 
en fonction de \(\Dim \Inv(u)\) comme dans le tableau~
\ref{tab:SymVect}.

\begin{table}[!h]
  \centering
  \begin{tabular}{|c|c|c|}\hline
    \(\Dim\Inv(u)\) & Nature de \(u\) & \(\Det u\) \\ \hline
    0 & \(-\Id\) & \((-1)^2=1\) \\
    1 & symétrie / droite & \\
    2 & \(\Id\) & 1 \\
    \hline
  \end{tabular}
  \caption{Classification des symétries vectorielles \(u\) selon 
  \(\Dim\Inv(u)\)}
  \label{tab:SymVect}
\end{table}

Comme \(u \in \Orthmin{E_2}\), alors \(u\) est une symétrie orthogonale par 
rapport à une droite. Donc c'est une réflexion (dans \(E82\) les hyperplans 
sont des droites). Alors \(\Orthmin{E_2}\) est inclus dans l'ensemble des 
réflexions de \(E_2\).

\begin{theo}
  \(\Orthmin{E_2}\) est l'ensemble des réflexions du plan \(E_2\).
\end{theo}
\begin{proof}
  Précisons la réflexion \(u\) dont la matrice dans la base orthonormée 
  \(\B\) est \(S_\varphi\). On cherche \(\Inv(u)=\enstq{x \in 
  E_2}{(u-\Id)(x)=0}\). On travaille matriciellement~: soit 
  \(X=\begin{pmatrix} x_1 \\ x_2 \end{pmatrix}=\Mat_\B(x)\).
    \begin{align*}
      x \in \Inv(u) &\iff (S_\varphi-I_2)x=0 \\
      &\iff \begin{cases} (\cos \varphi -1)x_1 +\sin\varphi x_2 = 0 \\ 
      \sin\varphi x_1 -(\cos\varphi+1)x_2=0 \end{cases} \\
      &\iff \begin{cases} -2\sin^2\frac{\varphi}{2}x_1 
        +2\sin\frac{\varphi}{2}\cos\frac{\varphi}{2} x_2 = 0 \\ 
        2\sin\frac{\varphi}{2}\cos\frac{\varphi}{2} x_1 - 
      2\cos^2\frac{\varphi}{2}x_2 = 0 \end{cases} \\
      &\iff \begin{cases} \sin\frac{\varphi}{2} 
        \left(\cos\frac{\varphi}{2}x_2 - \sin\frac{\varphi}{2}x_1\right) = 0 
        \\ \cos\frac{\varphi}{2} \left(\sin\frac{\varphi}{2}x_1 - 
      \cos\frac{\varphi}{2}x_2\right) = 0\end{cases} \\
      &\iff \cos\frac{\varphi}{2} x_2 = \sin\frac{\varphi}{2}x_1.
    \end{align*}
\end{proof}
\begin{prop}
  La réflexion \(u\) traduite par la matrice \(S_\varphi\) dans la base 
  \(\B\) est la réflexion par rapport à la droite \(\Dr\) d'équation~: 
  \(\cos\frac{\varphi}{2} x_2 = \sin\frac{\varphi}{2}x_1\) dans \(\B\).
  \begin{equation}
    \Dr = \VectEngendre\left(\cos\frac{\varphi}{2}\imath + 
    \sin\frac{\varphi}{2}\jmath\right).
  \end{equation}
\end{prop}

\subsubsection{Éléments de \(\SOrth{E_2}\), les rotations de \(E_2\)}

Soit une rotation \(r \in \SOrth{E_2}\), pour toute réflexion \(s \in 
\Orthmin{E_2}\) on a \(r \circ s \in \Orth{E_2}\). De plus \(\Det(r \circ 
s)=\Det(r) \Det(s)=-1\). donc \(r \circ s \in \Orthmin{E_2}\).

Il existe une réflexion \(s'\) de \(E_2\) telle que \(r \circ s=s'\), donc 
\(r=r \circ s \circ s= s'\circ s\).

\begin{theo}
  Toute rotation de \(E_2\) peut s'écrire d'une infinité de façons, comme le 
  produit de deux réflexions, la première étant choisie de manière 
  quelconque.
\end{theo}

\emph{Remarque}~: Soit \(\B\) une base orthonormée, \(r \in \Orth{E_2}\) et 
\((s,s') \in \Orthmin{E_2}\) telles que \(r=s' \circ s\). Il existe trois 
réels \(\theta\), \(\varphi\) et \(\varphi'\) tels que 
\(\Mat_\B(r)=R_\theta\), \(\Mat_\B(s)=S_\varphi\) et 
\(\Mat_\B(s')=S_{\varphi'}\). Ainsi
\begin{align*}
  r = s' \circ s &\iff  R_\theta = S_\varphi 
  S_{\varphi'}=R_{\varphi'-\varphi} \\
  &\iff \begin{cases} \cos \theta = \cos(\varphi'-\varphi) \\\sin \theta = 
  \sin(\varphi'-\varphi) \end{cases}\\
  &\iff \congru{\theta}{\varphi'-\varphi}{2\pi}.
\end{align*}
Si on connait \(\varphi\) et \(\theta\), il suffit de prendre 
\(\varphi'=\theta +\varphi\).

Autrement dit, la rotation \(r\) (représentée par \(R_\theta\) dans \(\B\)) 
peut s'écrire \(s' \circ s\) avec~:
\begin{itemize}
  \item \(s\) la réflexion par rapport à \(\Dr = 
    \VectEngendre\left(\cos\frac{\varphi}{2}\imath + 
    \sin\frac{\varphi}{2}\jmath\right)\);
  \item \(s'\) la réflexion par rapport à \(\Dr' = 
    \VectEngendre\left(\cos\frac{\varphi+\theta}{2}\imath + 
    \sin\frac{\varphi+\theta}{2}\jmath\right)\).
\end{itemize}

\subsubsection{Synthèse}

\begin{theo}
  Tout automorphisme orthogonal du plan euclidien \(E_2\) peut s'écrire 
  comme le produit d'au plus deux réflexions.
  \begin{table}[!h]
    \centering
    \begin{tabular}{|c|c|c|c|}\hline
      \(\Dim\Inv(u)\) & Nature de \(u\) & 
      \(\Orthmin{E_2}\setminus\SOrth{E_2}\)& Produit de \\ \hline
      0 & rotation & \(\SOrth{E_2}\) & 2 réflexions \\
      1 & réflexion & \(\Orthmin{E_2}\) & 1 réflexion \\
      2 & \(\Id\) & \(\SOrth{E_2}\) & 0 réflexion \\ \hline
    \end{tabular}
    \caption{Classification des automorphismes orthogonaux du plan 
    euclidien}
    \label{tab:ClassOrth}
  \end{table}
\end{theo}

Il reste à montrer que  que si \(u\) est une rotation distincte de 
l'identité on a \(\Inv u =\{0\}\). 

On note \(R_\theta=\Mat_\B(u)\) avec \(\B\) une base orthonormée. On sait 
que
\begin{equation}
  u \neq \Id \iff \theta \not\in 2\pi\Z.
\end{equation}
De plus pour tout \(x \in E_2\), on note \(X=\Mat_\B(x)=\begin{pmatrix} x_1 
\\ x_2 \end{pmatrix}\). Alors
\begin{align*}
  x \in \Inv(u) &\iff (R_\theta-I_2)X=0 \\
  &\iff \begin{cases} (\cos\theta -1)x_1 -\sin\theta x_2 = 0 \\ \sin\theta 
  x_1 +(\cos\theta -1)x_2=0 \end{cases}.
\end{align*}
Le déterminant du système vaut \(\begin{vmatrix} \cos\theta -1 & -\sin\theta 
\\ \sin\theta & \cos\theta -1\end{vmatrix} = 2(1-\cos\theta)\neq 0\). Alors 
c'est un système de Cramer homogène~: il n'y a qu'une solution, c'est la 
solution nulle.
\begin{equation}
  x \in \Inv(u) \iff x=0.
\end{equation}

\subsection{Mesure de l'angle d'une rotation dans le plan euclidien orienté}

\subsubsection{Définition}

Soient \(\B\) et \(\B'\) deux bases orthonormées du plan euclidien orienté 
\(E_2\). Soit \(r \in \SOrth{E_2}\), il existe \((\theta, \theta') \in 
\R^2\) tels ques \(\Mat_\B(r)=R_\theta\) et \(\Mat_{\B'}(r)=R_{\theta'}\). 
Soit \(P=\P_{\B,\B'}\) la matrice de passage de \(\B\) à \(\B'\)~: 
\(R_{\theta'}=P^{-1} R_{\theta} P\). Or \(P\) est orthogonale donc 
\(R_{\theta'}=P^\top R_{\theta} P\). On distingue deux cas~:
\begin{itemize}
  \item \(\B\) et \(\B'\) ont la même orientation et alors \(P \in 
    \SOrth{2}\). Il existe un réel \(\alpha\) tel que \(P=R_\alpha\). Alors
    \begin{align*}
      R_{\theta'}=P^{-1} R_{\theta} P &\iff 
      R_{\theta'}=R_{-\alpha}R_{\theta}R_{\alpha} \\
      &\iff R_{\theta'} = R_{-\alpha+\theta+\alpha}\\
      &\iff R_{\theta'} =R_{\theta}\\
      &\iff \congru{\theta}{\theta'}{2\pi}.
    \end{align*}
  \item \(\B\) et \(\B'\) n'ont pas la même orientation et alors \(P \in 
    \Orthmin{2}\). Il existe un réel \(\varphi\) tel que \(P=S_\varphi\). 
    Alors
    \begin{align*}
      R_{\theta'}=P^{-1} R_{\theta} P &\iff 
      R_{\theta'}=S_{\varphi}R_{\theta}S_{\varphi} \\
      &\iff R_{\theta'} = S_{\varphi}S_{\varphi+\theta}\\
      &\iff R_{\theta'} = R_{\varphi-\varphi-\theta}\\    &\iff R_{\theta'} 
      =R_{-\theta}\\
      &\iff \congru{\theta}{-\theta'}{2\pi}.
    \end{align*}
\end{itemize}
On en déduit le théorème suivant~:
\begin{theo}
  Pour toute rotation \(r\) du plan vectoriel euclidien \(E_2\) orienté, il 
  existe un unique réel \(\theta\) modulo \(2\pi\) tel que pour toute base 
  orthonormée \(\B\) de \(E_2\), on a \(\Mat_\B(r)=R_\theta\).

  Ce réel est appelé mesure de l'angle de la rotation \(r\).
\end{theo}
\emph{Remarque~:} Si on change l'orientation de \(E_2\), la mesure de 
l'angle d'une rotation est changé en son opposé. Elle ne dépend pas du choix 
de la base \(\B\) mais elle dépend quand même de l'orientation de l'espace.

\subsubsection{Détermination de l'angle d'une rotation}

Soient \(E_2\) le plan euclidien orienté, \(r \in \SOrth{E_2}\) et 
\(\theta\) l'angle de la rotation \(r\). Pour toute base orthonormée directe 
\(\B\) de \(E_2\), \(\Mat_\B(r) = \begin{vmatrix} \cos\theta & -\sin\theta 
\\ \sin\theta & \cos\theta \end{vmatrix}\). Soit \(a\) un vecteur unitaire 
quelconque. On peut compléter \(a\) en une base orthonormée directe 
\((a,b)\). Comme on a
\begin{gather}
  \Mat_{(a,b)}(r)=\begin{vmatrix} \cos \theta & -\sin \theta \\ \sin\theta & 
  \cos\theta \end{vmatrix} \\
  \Det(a,r(a))= \Det_{(a,b)}(a,r(a)) = \begin{vmatrix} 1 & \cos \theta \\ 0 
  & \sin\theta \end{vmatrix}=\sin\theta,
\end{gather}
alors
\begin{gather}
  \cos \theta = \prodscal{r(a)}{a} \\
  \sin \theta = \Det(a,r(a)).
\end{gather}

\subsubsection{Angles orientés de vecteurs}

\begin{prop}
  Soient \(a\) et \(b\) deux vecteurs unitaires du plan euclidien \(E_2\) 
  orienté. Il existe une unique rotation \(r\) de \(E_2\) telle que 
  \(r(a)=b\).
\end{prop}
\begin{proof}
  Comme dans la sous sous-section précédente, le vecteur \(a\) est unitaire; 
  alors il existe un vecteur \(c \in E_2\) tel que \((a,c)\) soit une base 
  orthonormée directe de \(E_2\). Alors il existe un unique couple \((b_1, 
  b_2) \in \R^2\) tel que \(b=b_1 a+b_2 c\). \(1=\norme{b}^2=b_1^2+b_2^2\). 
  Il existe donc un réel \(\alpha\) tel que \(\begin{cases} b_1=\cos \alpha 
  \\ b_2=\sin\alpha\end{cases}\).

  Soit \(r \in \SOrth{E_2}\) d'angle \(\theta \in \R\). Comme \((a,c)\) est 
  une base orthonormée directe, on a \(\Mat_{(a,c)}(r)=R_\theta\). Et
  \begin{align*}
    \Mat_{(a,c)}(a) &= \begin{pmatrix} 1 \\ 0 \end{pmatrix} \\
      \Mat_{(a,c)}(b) &= \begin{pmatrix} \cos\alpha\\ \sin\alpha 
      \end{pmatrix}.
  \end{align*}
  Alors
  \begin{align*}
    r(a) = b &\iff R_\theta \Mat_{(a,c)}(a) = \Mat_{(a,c)}(b) \\
    &\iff R_\theta \begin{pmatrix} 1 \\ 0 \end{pmatrix} = \begin{pmatrix} 
  \cos\alpha\\ \sin\alpha \end{pmatrix} \\
  & \iff \begin{cases} \cos\theta = \cos\alpha \\ \sin\theta = \sin\alpha 
  \end{cases} \\
  &\iff \congru{\theta}{\alpha}{2\pi} \\
  &\iff R_\theta = R_\alpha.
\end{align*}
Il existe donc une unique rotation \(r\) telle que \(r(a)=b\), c'est la 
rotation d'angle \(\alpha\).
\end{proof}
On peut alors donner la définition suivante.
\begin{defdef}
  Soient \(u\) et \(v\) deux vecteurs non nuls du plan euclidien \(E_2\) 
  orienté. On appelle angle orienté des vecteurs \(u\) et \(v\), noté 
  \(\widehat{(u,v)}\), l'angle de l'unique rotation \(r\) de \(E_2\) telle que 
  \(r\left(\frac{u}{\norme{u}}\right)=\frac{v}{\norme{v}}\).
\end{defdef}

\emph{Conséquence~:} Soient \(u\) et \(v\) deux vecteurs unitaires, alors
\begin{gather}
  \cos\widehat{(u,v)}=\prodscal{u}{v} \\
  \sin\widehat{(u,v)}=\Det(u,v).
\end{gather}

\section{Automorphismes orthogonaux de l'espace euclidien}

Soit \(E_3\) l'espace euclidien supposé orienté.

\subsection{Décomposition en produits de réflexions}

\emph{Rappel~:} Si \(a\) et \(b\) sont deux vecteurs unitaires différents de 
\(E_3\), il existe une unique réflexion \(s\) telle que \(s(a)=b\) et 
\(s(b)=a\). C'est la réflexion par rapport à l'hyperplan médiateur \(H\) de 
\(a\) et \(b\)~:
\begin{align*}
  H &=\VectEngendre(b-a)^\perp \\
  &=\enstq{x \in E_3}{\norme{a-x}=\norme{b-x}}.
\end{align*}

\begin{theo}
  \label{theo:orthreflexions}
  Tout automorphisme orthogonal de l'espace euclidien \(E_3\) peut s'écrire 
  comme le produit d'au plus trois réflexions.
\end{theo}
\begin{proof}
  Soit \(u \in \Orth{E_3}\). On pose
  \begin{equation}
    F = \Inv u = \ker(u-\Id).
  \end{equation}
  Alors \(\Dim F \in \{0, 1, 2, 3\}\). On distingue les cas selon la dimension 
  de \(F\)~:
  \begin{itemize}
    \item Cas 1~: \(\Dim F=3\), alors \(u=\Id\) est le produit de \(0\) 
      réflexion.
    \item Cas 2~: \(\Dim F=2\), alors \(F\) est un hyperplan de \(E_3\). Soit 
      \(a\) un vecteur normal à \(F\). Alors \(F=\VectEngendre(a)^\perp\) avec 
      \(a \neq 0\). On a \(u_{|F}=\Id\). L'hyperplan \(F\) est stable par 
      \(u\) donc \(F^\perp\) aussi. Or \(F^\perp=\VectEngendre(a)\) donc il 
      existe un réel \(\lambda\) tel que \(u(a)=\lambda a\). en passant à la 
      norme on a \(\abs{\lambda}=1\).

      Suppososons que \(\lambda=1\), alors \(u(a)=a\) et donc \(a \in F\). 
      Finalement \(a \in F \cap F^\perp =\{0\}\). Or \(a \neq 0\) par 
      hypothèse (Contradiction). Finalement \(\lambda =-1\). L'application 
      \(u\) est la symétrie orthogonale par rapport à \(F\).

    \item Cas 3~: \(\Dim F=1\), alors il existe un vecteur \(a\) non nul tel 
      que \(F=\VectEngendre(a)\). Soit \(b \notin F\) tel que \(\norme{b}=1\) 
      et posons \(c=u(b)\)\footnote{alors \(c\neq b\) car \(b \notin F\)}. 
      Alors \(\norme{c}=\norme{b}=1\). Soit \(H\) l'hyperplan médiateur de 
      \(c\) et \(b\)~:
      \begin{equation}
        H = \VectEngendre(b-c)^\perp.
      \end{equation}
      Soit \(s\) la réflexion par rapport à \(H\)\footnote{\(s(b)=c\) et 
      \(s(c)=b\)}. Alors
      \begin{align*}
        \prodscal{a}{u(b)-b} &=\prodscal{a}{u(b)} - \prodscal{a}{b} \\
        &=\prodscal{u(a)}{b} - \prodscal{a}{b} && u \in \Orth{E_3}\\
        &=\prodscal{a}{b} - \prodscal{a}{b} && a \in \Inv(u)=F \\
        &=0.
      \end{align*}
      Donc \(a \in H\), \(a=s(a)\). Soit \(v=s \circ u\), alors 
      \(v(a)=s(u(a))=s(a)=a\) et \(v(b)=s(u(b))=s(c)=b\). Comme 
      \(F=\VectEngendre(a)\) et \(b \notin F\) alors \((a,b)\) est une famille 
      libre. \(\Dim \VectEngendre(a,b)=2\) alors \(\VectEngendre(a,b) \subset 
      \Inv(v)\) et donc \(\Dim\Inv(v) \in \{2,3\}\). D'après les deux premiers 
      cas, \(v\) est le produit d'au plus une réflexion.
      \begin{align*}
        v &=s \circ u \\
        u &=(s \circ s) \circ v = s \circ v.
      \end{align*}
      Alors \(u\) est le produit d'au plus deux réflexions.
    \item Cas 4~: \(\Dim F=0\), soit \(a \in E_3 \setminus \{0\}\) tel que 
      \(\norme{a}=1\) et soit \(b=u(a)\neq a\). Alors \(\norme{b}=1\), car 
      \(u\) est orthogonal. Soit \(s\) l'unique réflexion telle que \(a=s(b)\) 
      et \(b=s(a)\). Soit \(v=s \circ u\), alors \(v(a)=s(b)=a\). Comme \(a 
      \neq 0\), on a \(\VectEngendre(a) \subset \Inv(v)\) et donc \(\Dim 
      \Inv(v) \geqslant 1\). L'application \(v\) est le produit d'au plus deux 
      réflexions, et alors \(u=s \circ v\) est le produit d'au plus trois 
      réflexions.
  \end{itemize}
\end{proof}

\subsection{Éléments de \(\SOrth{E_3}\)~: les rotations}

Soit \(r \in \SOrth{E_3}\). D'après la sous-section précédente, \(r\) est le 
produit de \(p\) réflexions, avec \(p \in \intervalleentier{0}{3}\). Le 
déterminant de \(r\) vaut \(\Det r =1=(-1)^p\). Alors \(p\) est pair, \(p \in 
\{0,2\}\). Les rotations de \(E_3\) sont~:
\begin{itemize}
  \item l'identité;
  \item le produit de deux réflexions distinctes.
\end{itemize}

\begin{defdef}
  Soit \(r \in \SOrth{E_3}\setminus\{\Id\}\), alors \(\Inv(r)\) est une 
  droite, appelée axe de la rotation.
\end{defdef}

\begin{prop}
  Soit \(r \in \SOrth{E_3}\). Il existe une base orthonormale directe \((i, j, 
  k)\) de \(E_3\) et il existe un réel \(\theta\) tels que
  \begin{equation}
    \Mat_{(i, j, k)}(r) = \begin{pmatrix} \cos \theta & -\sin\theta & 0 \\ 
    \sin\theta & \cos\theta & 0 \\ 0& 0 &1 \end{pmatrix}.
  \end{equation}
  C'est la rotation d'axe \(\VectEngendre{k}\) d'angle \(\theta\). Si 
  \(\congru{\theta}{\pi}{2\pi}\), alors \(r\) est appelée un retournement.
\end{prop}
\begin{proof}
  Si \(r=\Id\), alors le résultat est évident (\(\congru{theta}{0}{2\pi}\)). 
  Sinon, il existe deux hyperplans distincts \(P\) et \(H\) tels que \(r=s_P 
  \circ s_H\) où \(s_P\)(\(s_H\)) est la réflexion par rapport à \(P\)(\(H\)). 
  Soit \(\Dr = P \cap H\).

  Comme \(p \neq H\) alors \(\Dim \Dr \leqslant 1\). \(P+H\) est un 
  sous-espace vectoriel de \(E_3\), donc
  \begin{align*}
    3 &\geqslant \Dim(P+H) = \Dim P +\Dim H -\Dim(Dr) \\
    3 &\geqslant 4 - \Dim Dr
    \Dim \Dr &\geqslant 1.
  \end{align*}
  Alors \(\Dr\) est de dimension 1, c'est un droite. Soit \(k \in E_3\) 
  unitaire tel que \(\Dr = \VectEngendre(k)\).
  \begin{equation}
    r(k) = s_P(s_H(k))=s_P(k)=k.
  \end{equation}
  alors \(\Dr \subset \Inv(r)\) et alors \(\Dim \Inv(r) \geqslant 1\). Si 
  \(\Dim \Inv(r)=2\) alors \(r\) est une réflexion (aburde) et si \(\Dim 
  \Inv(r)=3\) alors \(r=\Id\) (absurde). Donc \(\Dim\Inv(r)=1\).

  Finalement \(\Dr = \Inv(v)\). Soit \(F=\Dr^\perp\). \(\Dr\) est stable par 
  \(r\), donc \(F\) est stable par \(r\). Soit \(\varphi=r_{|F}^{|F} \in 
  \Orth{F}\). On induit une orientation sur \(F\) à l'aide de l'orientation de 
  \(\Dr\) par \(k\).

  Si \((i, j)\) est une base orthonormée directe de \(F\), \((i, j, k)\) est 
  une base orthonormée directe de \(E_3\). Donc \(\Det(\varphi)=\Det(r)=1\). 
  Alors \(\varphi\) est une rotation. Il existe un réel \(\theta\) tel que 
  pour toute base orthonormée directe \((i, j)\) on a \([\varphi]_{(i, 
  j)}=R_\theta\).
\end{proof}

\begin{theo}
  Soient \(r \in \SOrth{E_3}\subset\{\Id\}\), \(k\) un vecteur directeur 
  unitaire tel que \(\Inv(r)=\VectEngendre(k)\). Soit \(\theta\) l'angle de 
  \(r\). Pour tout vecteur \(x \in E_3\), on a
  \begin{gather}
    x \perp k \implies r(x) = \cos\theta x + \sin\theta k\wedge x \\
    x \perp k \text{~et~} \norme{x}=1 \implies  \cos\theta = 
    \prodscal{x}{r(x)} \text{~et~} \sin\theta=\Det(x,r(x),k).
  \end{gather}
\end{theo}
\begin{proof}
  Par linéarité on peut supposer que \(x\) est unitaire, on note 
  \(F=\VectEngendre(k)^\perp\). Alors \(x \in F\) et il existe un \(y \in F\) 
  tel que \((x, y)\) soit une base orthonormée directe de \(F\). Ainsi \((x, 
  y, k)\) est une base orthonormée directe de \(E_3\) puisque \(k \wedge 
  x=y\). Alors
  \begin{align*}
    \Mat_{(x, y)}(r_{|F}^{|F}) &=R_\theta \\
    &=\begin{pmatrix} \cos \theta & -\sin\theta \\ \sin\theta & \cos\theta 
    \end{pmatrix},
  \end{align*}
  alors on a bien \(r(x) = \cos\theta x+ \sin\theta y=\cos\theta x+ \sin\theta 
  k\wedge x\)

  D'après la première égalité~:
  \begin{align*}
    \prodscal{x}{r(x)} &= \prodscal{x}{\cos\theta x+ \sin\theta k\wedge x} \\
    &=\cos\theta \norme{x}^2+\sin\theta \prodscal{x}{k\wedge x}\\
    &=\cos\theta.
  \end{align*}
  Et
  \begin{align*}
    \Det(x, r(x), k) &= \Det_{(x, y, k)}(x, r(x), k) \\
    &= \begin{vmatrix} 1 & \cos\theta & 0 \\ 0 &\sin\theta & 0 \\ 0 & 0 & 
    1\end{vmatrix} =\sin\theta.
  \end{align*}
\end{proof}

\subsection{Classification des éléments de \(\Orth{E_3}\)}

\begin{table}
  \centering
  \begin{tabular}{|c|c|c|c|}\hline
    \(\Dim\Inv(r)\) & Nature & \(\Orthmin{E_3}\setminus\SOrth{E_3}\) & Produit 
    de \ldots réflexions \\ \hline
    0 & composée rot. réf. & \(\Orthmin{E_3}\) & 3 \\
    1 & rotation & \(\SOrth{E_3}\) & 2 \\
    2 & réflexion & \(\Orthmin{E_3}\) & 1 \\
    3 & \(\Id\) & \(\SOrth{E_3}\) & 0 \\ \hline
  \end{tabular}
  \caption{Classification des automorphismes orthogonaux de l'espace 
  euclidien}
  \label{tab:classE3}
\end{table}

\subsection{Exemples}

\emph{Exemple 1~:} \(M = \frac{1}{9} \begin{pmatrix} -8 & 4 & 1 \\ 4 & 7 & 4 
\\ 1 & 4 & -8 \end{pmatrix}\). Soit \(u \in \Endo{\R^3}\) canoniquement 
associée à \(M\). Déterminons la nature de \(u\) et ses caractéristiques.

Déjà \(\norme{C_1}^2=\norme{C_2}^2=\norme{C_3}^2=1\) et 
\(\prodscal{C_1}{C_2}=\prodscal{C_2}{C_3}=\prodscal{C_1}{C_3}=0\). \((C_1, 
C_2, C_3)\) est une base orthonormale, \(M \in \Orth{3}\). De plus \(C_1 
\wedge C_2=C_3\) et \(\Det M=1\).

Cherchons \(\Inv(u)\).
\begin{align*}
  X \in \Inv(u) &\iff MX = X \\
  &\iff (M-\Id)X =0 \\
  &\iff \begin{cases} -17x_1 +4x_2 +x_3 =0 \\ 4x_1 -2x_2 +4x_3=0 \\ x_1 +4x_2 
  -17 x_3=0 \end{cases}\\
  &\iff \begin{cases} x_1 = x_3 \\ x_2=4x_3 \end{cases}.
\end{align*}
Donc \(\Inv(u) = \VectEngendre(1, 4, 1)\), \(k = \frac{1}{3\sqrt{2}}(1, 4, 
1)\).

Cherchons un vecteur normal, on prend par exemple 
\(\vect{r}=\frac{1}{\sqrt{2}}(1, 0, -1)\). Alors 
\(u(v)=\frac{1}{9\sqrt{2}} (-9, 0, 9)=\frac{1}{\sqrt{2}}(-1, 0, 1)\). 
Ainsi
\begin{equation}
  \cos\theta = \frac{1}{\sqrt{2}} \begin{pmatrix} -1 \\ 0 \\ 1 
  \end{pmatrix} \cdot \frac{1}{\sqrt{2}} \begin{pmatrix} 1 \\ 0 \\ -1 
  \end{pmatrix} = -1
\end{equation}
donc \(\congru{\theta}{\pi}{2\pi}\).

\(r\) est la rotation d'axe \(\VectEngendre(1, 4, 1)\) d'angle \(\pi\). 
C'est un retournement.

\emph{Exemple 2~:} \(M=\frac{1}{4} \begin{pmatrix} 1 & \sqrt{3} & 
2\sqrt{3} \\ \sqrt{3} & 3 & -2 \\ 2\sqrt{3} & -2 & 0 \end{pmatrix}\). On a 
bien
\begin{gather}
  \norme{C_1}^2=\norme{C_2}^2=\norme{C_3}^3=1 \\
  \prodscal{C_1}{C_2}= \prodscal{C_1}{C_3}=  \prodscal{C_3}{C_2}=0 \\
  C_1 \wedge C_2 = -C_3.
\end{gather}
Alors \(M \in \Orthmin{3}\). Trouvons \(\Inv(M)\)~:
\begin{align*}
  X \in \Inv(M) &\iff  MX=X \\
  &\iff (M-I_3) X=0 \\
  &\iff \begin{pmatrix} -3 & \sqrt{3} & 2\sqrt{3} \\ \sqrt{3} & -1 & -2 \\ 
  2\sqrt{3} & -2 & 4 \end{pmatrix} \begin{pmatrix} x_1 \\ x_2 \\ x_3 
  \end{pmatrix} =0 \\
&\iff x_2 = \sqrt{3} x_1 -2 x_2 \end{align*}
\begin{equation}
  \Inv(u) = \VectEngendre((1, \sqrt{3}, 0), (0, -2, 1)).
\end{equation}
