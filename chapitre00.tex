%&latex
\addtocounter{chapter}{-1}
\chapter{Éléments de logique et quantificateurs}
\minitoc
\minilof
\minilot

Nous présentons dans ce très court chapitre les notions de base de logiques, les
tables de vérités et l'utilisation des quantificateurs. On définit aussi ce
qu'est une assertion et un théorème.

\section{Assertion et théorème}
\label{chap0sec:assertionettheoremes}

Nous allons préciser à un premier niveau quelques notions mathématiques qui sont
relativement intuitives mais nécessitent quand même des définitions rigoureuses.
L’idée étant de préciser schématiquement comment se présente une théorie
mathématique ainsi que la notion essentielle de démonstration. 

La première notion est celle d’assertion. Une assertion est un énoncé 
mathématiques \(P\) aussi rigoureux que possible qui ne prend qu'une valeur de 
vérité qui est soit vraie (V) soit fausse (F). On consigne ces valeurs dans une 
table de vérité.

En mathématique, on appelle \emph{tautologie} une assertion qui est toujours 
vraie. Par exemple, l'assertion \(1+1=2\) est une tautologie. Par contre 
l'assertion \(1+1=3\) est une assertion dont la valeur de vérité est F. \(3=2 
\times\) n'est pas une assertion. \(3=1+x\) est une assertion dont la valeur de 
vérité dépend de la valeur de \(x\).

Un théorème est une assertion vraie déduite grâce à d'autres assertions, en
général les théorèmes sont des résultats importants à retenir. Un lemme est un
résultat préliminaire qui permet de démontrer un théorème. Un corollaire est une
conséquence directe d'un théorème.

Pour écrire un énoncé mathématique on utilise le langage courant mais aussi
quelques symboles et très souvent on manipule des lettres dans l'alphabet grec.
On présente cet alphabet et ces symboles dans l'annexe~\ref{chap:qqsymboles}.

\section{Négation et connecteurs logiques}
\label{chap0sec:negationetconnecteurs}

À partir d'une assertion \(P\), on construit une assertion appelée (non \(P\)),
qui peut être noté \(\neg P\) dont la table de vérité est donnée par la
table~\ref{tab:tabveriteP}.

\begin{table}[!h]
  \centering
  \begin{tabular}{|c|c|}\hline
    \(P\) & \(\neg P\) \\ \hline
    V & F \\ F & V \\ \hline
  \end{tabular}
  \caption{Table de vérité de P}
  \label{tab:tabveriteP}
\end{table}

À partir de deux assertions \(P\) et \(Q\), on peut en fabriquer de nouvelles à
l'aide des connecteurs logiques suivants~:
\begin{itemize}
  \item la conjonction notée ``et'' ou \(\wedge\);
  \item la disjonction notée ``ou'' ou \(\vee\);
  \item l'implication notée \(\implies\);
  \item l'équivalence notée \(\iff\).
\end{itemize}

Les tables de vérités de ces connecteurs logiques sont données dans la
table~\ref{tab:tabverconn}.

\begin{table}[!h]
  \centering
  \begin{tabular}{|c|c|c|c|c|c|}\hline
    P & Q & \(P \text{~et~} Q\) & \(P \text{~ou~} Q\) & \(P \implies Q\) &
    \(P \iff Q\) \\ \hline
    V & V & V & V & V & V \\
    V & F & F & V & F & F \\
    F & V & F & V & V & F \\
    F & F & F & F & V & V \\ \hline
  \end{tabular}
  \caption{Table de vérité des connecteurs logiques}
  \label{tab:tabverconn}
\end{table}

Dans l'assertion \(P \implies Q\) l'assertion \(P\) est l'hypothèse et 
l'assertion \(Q\) est la conclusion. On a l'équivalence logique
\begin{empheq}[box=\shadowbox*]{equation}
  (P \implies Q) \iff (\neg P \text{~ou~} Q).
\end{empheq}

L'assertion \(Q \implies P\) est l'assertion réciproque de \(P \implies 
Q\).Attention à ne pas confondre les deux.

\section{Assertions logiquement équivalentes}
\label{chap0sec:assertionslogiquementequiv}

Deux assertions sont dites logiquement équivalentes lorsqu'elles ont la même
table de vérité. On écrit par exemple les équivalences suivantes~:
\begin{align}
  \neg(\neg P) &\iff P; \\
  \neg(P \text {~et~} Q) & \iff \neg P \text{~ou~} \neg Q; \\
  \neg(P \text{~ou~} Q) & \iff \neg P \text {~et~} \neg Q; \\
  \neg(P \implies Q) & \iff P \text {~et~} \neg Q.
\end{align}

Le raisonnement par \emph{contraposée} repose sur l'équivalence logique
\begin{empheq}[box=\shadowbox*]{equation}
  (P \implies Q) \iff (\neg Q \implies \neg P).
\end{empheq}

Le raisonnement par l'absurde est le suivant~: pour montrer que \(P \implies
Q\), on suppose que \(P\) est vraie et que \(Q\) est fausse puis on montre que
cela aboutit à une contradiction. Ainsi, le \emph{raisonnement par l'absurde}
repose sur l'équivalence logique
\begin{empheq}[box=\shadowbox*]{equation}
  (P \implies Q) \iff \neg(P \text {~et~} \neg Q).
\end{empheq}

\section{Quantificateurs et leurs négations}
\label{chap0sec:quantificateursetnegation}

L'expression mathématique ``\(\forall x \in E\)'' se lit ``quelque soit
l'élément \(x\) de E'' ou encore ``pour tout élément \(x\) de E''. L'expression
mathématique ``\(\exists x \in E\)'' se lit ``il existe un élément \(x\) de E'',
au moins un. ``\(\forall x \in E \ P(x)\)'' signifie que pour tout élément pris
dans E, la propriété P est vraie. ``\(\exists x \in E \ P(x)\)'' signifie qu'il
existe au moins un élément de E pour lequel P est vérifiée.
Les négations des quantificateurs sont les suivantes~:
\begin{empheq}[box=\shadowbox*]{align}
  \neg(\forall x \in E \quad P(x)) &\iff \exists x \in E \quad \neg P(x); \\
  \neg(\exists x \in E \quad P(x)) &\iff \forall x \in E \quad \neg P(x).
\end{empheq}

La variable \(x\) est dite muette.

\clearpage
\section{Exercices}

\begin{exercice}
  Soit \((a, b, c) \in \R^3\). Écrire les négations des propositions suivantes~:
  \begin{enumerate}
    \item \(a=b=c\);
    \item \(ab=0\);
    \item \(-3 \leqslant a < 3 < c\).
  \end{enumerate}
\end{exercice}

\begin{exercice}
  Soit \(x\) un réel quelconque. Que signifient les propositions suivantes et 
  que peut on en conclure?
  \begin{enumerate}
    \item \(\forall \epsilon>0 \quad x<\epsilon\);
    \item \(\forall \epsilon>0 \quad \abs{x}<\epsilon\);
    \item \(\exists \epsilon>0 \quad x<\epsilon\);
  \end{enumerate}
\end{exercice}

\begin{exercice}[Fonctions monotones et strictement monotones]
  \begin{enumerate}
    \item Rappeler la définition d'une fonction croissante, d'une fonction
      strictement croissante.
    \item Soit \(f\) une fonction définie sur un intervalle \(I\) contenant au 
      moins deux points. Écrire avec des quantificateurs que \(f\) n'est pas 
      croissante, puis que \(f\) n'est pas strictement croissante, que \(f\) 
      n'est pas monotone et enfin que \(f\) n'est pas strictement monotone.
    \item Une fonction peut-elle être à la fois croissante et décroissante?
      Strictement croissante et strictement décroissante?
    \item Soient \(a\) et \(b\) des éléments de \(I\). On suppose que \(f\) est
      croissante sur \(I\).
      \begin{enumerate}
        \item On suppose que \(f(a) \leqslant f(b)\). Que peut on dire de \(a\) 
          et \(b\)?
        \item On suppose que \(f(a) < f(b)\). Que peut on dire de \(a\) et 
          \(b\)?
      \end{enumerate}
    \item Reprendre la question précédente avec \(f\) strictement croissante sur 
      \(I\).
  \end{enumerate}
\end{exercice}

\begin{exercice}[Fonction nulle, fonction qui s'annule]
  Soient \(I\) un intervalle contenant au moins deux points et \(f\) une 
  fonction de \(I\) vers \(\R\). Exprimer les propositions suivantes ainsi que 
  leurs négations à l'aide des quantificateurs~:
  \begin{enumerate}
    \item \(f\) est l'application nulle ;
    \item \(f\) s'annule ;
    \item \(f\) est à valeurs positive.
  \end{enumerate}
  Quel(s) sens peut on donner à l'assertion \(f>0\)?
\end{exercice}

\begin{exercice}
  Soit \(E\) un ensemble de nombres. Écrire en langage courant la signification 
  des propositions suivantes~:
  \begin{enumerate}
    \item \(\forall x \in E \ \exists y \in E \quad x \neq y\);
    \item \(\forall x \in E \ \exists y \in E \quad x = y\);
    \item \(\exists x \in E \ \forall y \in E \quad x = y\);
    \item \(\forall x \in E \ \forall \ \forall z \in E \quad (x=y \textrm{~ou~}
      y=z \textrm{~ou~} z=x)\).
  \end{enumerate}
\end{exercice}

\begin{exercice}[Fonctions paires et impaires]
  \begin{enumerate}
    \item Donner la définition d'une fonction paire, d'une fonction impaire,
      d'une fonction non paire et d'une fonction non impaire.
    \item Toute fonction est-elle soit paire soit impaire?
    \item Existe t il des fonctions de \(\R\) vers \(\R\) qui sont à la fois
      paires  et impaires? Si oui les expliciter.
    \item Soit \(f\) de \(\R\) vers \(\R\). Prouver qu'il existe un unique
      couple \((\varphi, \psi)\) de fonctions de \(\R\) vers \(\R\) tel que
      \(f=\varphi+\psi\) avec \(\varphi\) paire et \(\psi\) impaire.
    \item Peut on généraliser ce résultat aux fonctions de \(\R\) vers \(\C\)?
      Si c'est possible, l'appliquer à la fonction
      \(\fonctionL{f}{\R}{\C}{x}{\exp(\ii x)}\).
  \end{enumerate}
\end{exercice}

\begin{exercice}
  Donner la valeur de vérité (Vrai ou Faux) des propositions suivantes~:
  \begin{enumerate}
    \item \(2+2=4\) et \(1=0\);
    \item \(2+2=4\) ou \(1=0\);
    \item \(\forall x \in \R \quad x=1\);
    \item \(\exists x \in \R \quad x\neq 1\);
    \item \(\forall x \in \R \ \exists y \in \R \quad xy=1\);
    \item \(\exists x \in \R \ \exists y \in \R \quad xy=1\);
    \item \(\forall n \in \N \quad (n^2 \textrm{~est pair}) \implies n
      \textrm{~est pair})\);
    \item \(\forall x \in \R \quad (8=7 \implies x=1)\);
    \item \(\forall x \in \R \quad (x \geqslant 7 \implies x^2 \geqslant 49)\).
  \end{enumerate}
\end{exercice}
