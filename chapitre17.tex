\chapter{Espaces vectoriels de dimension finie}
\label{chap:dimensionfinie}
\minitoc
\minilof
\minilot

Dans tout le chapitre, \((\K,+,\cdot)\) désigne un corps et \((E,+,\perp)\) un \(\K\)-espace vectoriel et \(I\) un ensemble quelconque lorsque ce n'est pas précisé.
\section{Familles de vecteurs}

\subsection{Familles génératrices}

\begin{defdef}
  On appelle combinaison linéaire de la famille \((x_i)_{i \in I} \in E^I\) (\(I\) fini) tout vecteur de \(E\) de la forme \(\sum_{i \in I}\alpha_i x_i\) avec \((\alpha_i)_{i\in I} \in \K^I\).

  Soit \(\X=(x_i)_{i \in I} \in E^I\) (\(I\) fini) et \(E_1 = \{\sum_{i \in I}\alpha_i x_i, (\alpha_i)_{i\in I} \in \K^I\}= \VectEngendre(\X)\).

\(E_1\) est le sous-espace vectoriel engendré par la famille \(\X\).
\end{defdef}

\begin{defdef}
  Soient \(E_1\) un sous-espace vectoriel de \(E\), \(\X = (x_i)_{i \in I}\in E^I\) une famille finie de vecteurs de \(E\). 

  Si \(E_1=\VectEngendre(\X)\), on dit que \(\X\) est une famille génératrice de \(E_1\).
\end{defdef}
On a les hypothèses suivantes~:
\begin{itemize}
\item \(\X=(x_i)_{i \in I} \in E^I\) avec \(I\) fini;
\item \(E_1\) est un sous-espace vectoriel de \(E\);
\item si \(E_1 = \VectEngendre(\X)\) alors \(\X \subset E_1\).
\end{itemize}
Alors finalement
\begin{equation}
  \X \text{~est génératrice de } E_1 \iff \begin{cases} \forall i \in I \quad x_i \in E_1 \\ \forall x \in E_1 \ \exists (\alpha_i)_{i \in I} \in \K^I \quad x=\sum_{i \in I} \alpha_i x_i \end{cases}.
\end{equation}

\emph{Exemples}~: La famille \((1,\ii)\) est génératrice du \(\R\)-espace vectoriel \(\C\). Soit \((x_i)_{i \in I} \in E^I\). On appelle sous-famille de \((x_i)_{i \in I}\) toute famille \((x_i)_{i \in J}\) avec \(J \subset I\). On appelle sur-famille de \((x_i)_{i \in I}\) toute famille \((x_i)_{i \in L}\) avec \(I \subset L\).

\begin{prop}
  Toute sur-famille d'une famille génératrice est aussi génératrice. Plus précisément~: soit \(E_1\) un sous-espace vectoriel de \(E\), \(I\) et \(L\) deux ensembles finis tels que \(I \subset L\) et \(\X=(x_i)_{i \in I} \in E^I\), \(\Y=(x_i)_{i \in L} \in E^L\). Si~:
  \begin{itemize}
  \item \(\X\) est génératrice de \(E_1\);
  \item pour tout \(i \in L\setminus I \quad x_i \in E_1\);
  \end{itemize}
  alors \(\Y\) est génératrice de \(E_1\).
\end{prop}
\begin{proof}
  Soit \(i \in L\) tel que \(x_i \in E_1\) alors soit \(i \in L\setminus I\) et c'est l'hypothèse, soit \(i \in I\) et c'est la conséquence de \(\X\) est génératrice de \(E_1\). 

  Soit \(x \in E_1\) alors il existe une famille de scalaire \((\alpha_i)_{i \in I}\) telle que
  \begin{equation}
    x= \sum_{i \in I} \alpha_i x_i = \sum_{i \in L} \alpha_i x_i + \sum_{i \in L\setminus I} 0 x_i = \sum_{i \in L} \lambda_i x_i.
  \end{equation}
  Avec \(\lambda_i = \alpha_i\) si \(i \in I\) et \(\lambda_i=0\) sinon. Ainsi \(x\) est une combinaison linéaire de \((x_i)_{i \in L}\). Donc \(\Y\) est une famille génératrice de \(E_1\).
\end{proof}
%
\begin{prop}
  Soit \(\X=(x_i)_{i \in I}\) (\(I\) fini) une famille génératrice d'un sous-espace vectoriel \(E_1\). Soit \(i_0 \in I\). Alors la famille \((x_i)_{i \in I\setminus\{i_0\}}\) est génératrice si et seulement si \(x_{i_0}\) est une combinaison linéaire de  \((x_i)_{i \in I\setminus\{i_0\}}\).
\end{prop}
\begin{proof}
  Supposons que \((x_i)_{i \in I\setminus\{i_0\}}\) est génératrice de \(E_1\). Puisque \(x_{i_0}\) est un vecteur de \(E_1\), il s'écrit comme une combinaison linéaire de \((x_i)_{i \in I\setminus\{i_0\}}\).

  Supposons maintenant que \(x_{i_0}\) est une combinaison linéaire de  \((x_i)_{i \in I\setminus\{i_0\}}\). Alors pour tout \(i \in I\setminus\{i_0\}\) \(x_i \in E_1\) (puisque \(\X\) est génératrice de \(E_1\) et donc tous les \(x_i\) sont dans \(E_1\)). De plus pour tout \(x \in E_1\), il existe une famille de scalaires \((\alpha_i)_{i \in I} \in \K^I\) telle que
  \begin{equation}
    x = \sum_{i \in I} \alpha_i x_i = \alpha_{i_0} x_{i_0} + \sum_{i \in I\setminus\{i_0\}} \alpha_i x_i.
  \end{equation}
Par hypothèse, il existe une famille de scalaires \((\lambda_i)_{i \in I} \in K^I\) telle que \(x_{i_0} = \sum_{i \in I\setminus\{i_0\}} \lambda_i x_i\). Alors finalement
\begin{equation}
  x = \alpha_{i_0}\sum_{i \in I\setminus\{i_0\}} \lambda_i x_i + \sum_{i \in I\setminus\{i_0\}} \alpha_i x_i = \sum_{i \in I\setminus\{i_0\}} (\alpha_{i_0} \lambda_i +\alpha_i) x_i,
\end{equation}
alors \((x_i)_{i \in I\setminus\{i_0\}}\) est une famille génératrice de \(E_1\).
\end{proof}

\subsection{Familles libres, familles liées}

\begin{defdef}
  Soit \(\X=(x_i)_{i \in I} \in E^I\) (\(I\) fini). On appelle relation de dépendance linéaire (RDL) toute famille de scalaires \((\alpha_i)_{i \in I} \in \K^I\) telle que \(\sum_{i \in I} \alpha_i x_i =0\).
\end{defdef}
\begin{defdef}
  Soit \(\X=(x_i)_{i \in I} \in E^I\) (\(I\) fini). La famille \(\X\) est libre si et seulement si toute relation de dépendance linéaire est nulle (les vecteurs sont linéairement indépendants). La famille \(\X\) est liée si et seulement s'il existe une relation de dépendance linéaire non nulle de \(\X\). Mathématiquement on écrit~:
\begin{itemize}
\item La famille \(\X\) est libre si pour toute famille de scalaires \((\alpha_i)_{i \in I} \in \K^I\) telle que \(\sum_{i \in I} \alpha_i x_i =0\) alors pour tout \(i \in I\), \(\alpha_i =0\);
\item la famille \(\X\) est liée s'il existe une famille de scalaires \((\alpha_i)_{i \in I} \in \K^I\) telle que \(\sum_{i \in I} \alpha_i x_i =0\) et s'il existe \(i_0 \in I \alpha_{i_0} \neq 0\).
\end{itemize}
\end{defdef}

\begin{prop}
  Si \(I = \emptyset\) alors toute famille indexée sur \(I\) est libre. Si \(I\) est un singleton alors on a un vecteur \(a \in E\) tel que \((x_i)_{i \in I}=(a)\), alors \((a)\) est libre si et seulement si \(a \neq 0\) et \((a)\) est liée si et seulement si \(a=0\).
\end{prop}

\begin{proof}
  Si \(a = 0\) alors \(1 \times a = 0\) or \(1 \neq 0\) \((1)\) est une relation de dépendance linéaire non nulle de \((a)\), donc elle est liée. Si \(a\) est non nul alors pour tout scalaire \(\alpha\) si on a\(\alpha a=0\) alors \(\alpha=0\) donc \((a)\) est libre.
\end{proof}

\begin{prop}
Soit \(\X= (x_i)_{i \in I} \in E^I\) (\(I\) fini). La famille \(\X\) est liée si et seulement s'il existe \(i_0 \in I\) tel que \(x_{i_0}\) est combinaison linéaire de \((x_i)_{i \in I \setminus \{i_0\}}\).
\end{prop}
\begin{proof}
  S'il existe \(i_0 \in I\) tel que \(x_{i_0}\) est combinaison linéaire de \((x_i)_{i \in I \setminus \{i_0\}}\) alors il existe une famille de scalaires \((\lambda_i)_{i \in I\setminus\{i_0\}} \in \K^{I\setminus\{i_0\}}\) tel que \(x_{i_0}= \sum_{i \in I\setminus\{i_0\}} \lambda_i x_i\) alors en posant \(\lambda_{i_0}=1\) on a\(\sum_{i \in I} \lambda_i x_i=0\) comme la famille de scalaire n'est pas la famille alors \(\X\) est liée.

  Si \(\X\) est liée, alors par définition il existe une famille de scalaires non tous nul \((\alpha_i)_{i \in I} \in \K^I\) et un élément \(i_0\), \(\alpha_{i_0} \neq 0\) tels que
  \begin{align}
    0&= \sum_{i \in I} \alpha_i x_i = \sum_{i \in I\setminus\{i_0\}} \alpha_i x_i +\alpha_{i_0} x_{i_0} \\
    x_{i_0} &= -\frac{1}{\alpha_{i_0}} \sum_{i \in I\setminus\{i_0\}} \alpha_i x_i && (\alpha_{i_0} \neq 0) \\
    x_{i_0} &= \sum_{i \in I\setminus\{i_0\}} \lambda_i x_i && \left(\lambda_i = -\frac{\alpha_i}{\alpha_{i_0}}\right)
  \end{align}
\end{proof}

\begin{prop}
  Toute sous famille d'une famille libre est libre. Toute sur famille d'une famille liée est liée.
\end{prop}
\begin{proof}
  Soit \(\X=(x_i)_{i \in I} \in E^I\) une famille libre (\(I\) fini). Soit \(J \subset I\) et \(\Y=(x_i)_{i \in J} \in E^J\), \((\alpha_i)_{i \in J} \in \K^J\) telles que \(\sum_{i \in J} \alpha_i x_i =0\). On définit la famille de scalaires \(\lambda\) comme étant nulle sur \(I\setminus J\) et égale à \(\alpha\) sur \(J\). Ainsi
\begin{equation}
  \sum_{i \in I} \lambda_i x_i = \sum_{i \in J} \alpha_i x_i +0 =0
\end{equation}
Comme la famille \(\X\) est libre on en déduit que la famille de scalaires \(\lambda\) est nulle. En particulier \(\forall i \in J \quad \alpha_i = 0\) donc \(\Y\) est libre.

Soit \(\X\) une famille liée et \(\mathcal{Z}\) une sur famille de \(\X\). Si \(\mathcal{Z}\) était libre, alors \(\X\) serait libre (en tant que sous famille de \(\mathcal{Z}\)). Ce qui contredit l'hypothèse. Donc la famille \(\mathcal{Z}\) est liée. 
\end{proof}

\begin{prop}
  Soit \(\X=(x_i)_{i \in I}\) une famille libre de vecteurs de \(E\) (\(I\) fini) et \(i_0 \notin I\). Alors la famille \((x_i)_{i \in I \cup \{i_0\}}\) est libre si et seulement si \(x_{i_0}\) n'est pas combinaison linéaire de la famille \(\X\).
\end{prop}
\begin{proof}
 Montrons la première implication~: si \(x_{i_0}\) est une combinaison linéaire de \(\X\) alors \((x_i)_{i \in I \cup \{i_0\}}\) est liée. Alors par contraposée si \((x_i)_{i \in I \cup \{i_0\}}\) est libre alors \(x_{i_0}\) n'est pas une combinaison linéaire de \(\X\).

  Montrons la deuxième implication~: Soit une famille de scalaire \(\alpha\) indexée sur \(I \cup\{i_0\}\) telle que \(\sum_{i \in I \cup\{i_0\}} \alpha_i x_i =0\). Alors
  \begin{equation}
    \alpha_{i_0} x_{i_0} + \sum_{i \in I} \alpha_i x_i =0
  \end{equation}
  Si \(\alpha_{i_0} \neq 0\) alors \(x_{i_0}\) est une combinaison linéaire de \((x_i)_{i \in I}\), ce qui est impossible par hypothèse. Ainsi \(\alpha_{i_0} = 0\) et 
  \begin{equation}
    \sum_{i \in I} \alpha_i x_i =0
  \end{equation}
  et comme \(\X\) est libre alors \(\forall i \in I \quad \alpha_i=0\). Ainsi \(\forall i \in I\cup\{i_0\} \quad \alpha_i=0\). Alors \((x_i)_{i \in I \cup \{i_0\}}\) est libre.
\end{proof}

\emph{Cas particulier important}~: On dit que deux vecteur \(x,y\) sont colinéaires s'il existe un scalaire \(\lambda\) tel que \(x=ky\) ou s'il existe un scalaire \(\mu\) tel que \(x=\mu y\).

\begin{prop}
  Soit deux vecteurs de \(E\) \(x,y\), alors ils sont colinéaires si et seulement si la famille \((x,y)\) est liée.
\end{prop}
\begin{proof}
  On peut trouver une combinaison linéaire \ldots
\end{proof}

\emph{Remarque}~: Dans une famille libre on ne peut pas trouver
\begin{itemize}
\item le vecteur nul ;
\item deux vecteurs colinéaires a fortiori égaux
\end{itemize}

\begin{prop}
  Soit \(\X=(x_i)_{i \in I}\) une famille libre de vecteurs de \(E\) (\(I\) fini). Soient deux familles \(\alpha\) et \(\beta\) de scalaires indexées par \(I\). Alors
  \begin{equation}
    \sum_{i \in I} \alpha_i x_i = \sum_{i \in I} \beta_i \implies \forall i \in I \quad \alpha_i = \beta_i
  \end{equation}
\end{prop}
\begin{proof}
  Par hypothèse \(\sum_{i \in I} (\alpha_i-\beta_i) x_i=0\). Comme la famille \(\X\) est libre on a bien \(\forall i \in I \ \alpha_i = \beta_i\).
\end{proof}

\subsection{Bases d'un \(\K\)-espace vectoriel}

\begin{defdef}
  On appelle base d'un \(\K\)-espace vectoriel \(E\) toute famille \(\X \in E^I\) libre et génératrice de \(E\).
\end{defdef}

\begin{theo}
  Soit \(\E = (e_i)_{i \in I} \in E^I\) une famille finie de vecteurs de \(E\). Alors c'est une base de \(E\) si et seulement si pour tout \(x \in E\) il existe une unique famille de scalaires \((x_i)_{i \in I}\) telle que \(x=\sum_{i \in I}x_i e_i\).
\end{theo}
\begin{proof}
  Si \(\E\) est une base de \(E\), soit un vecteur \(x \in E\) alors
  \begin{itemize}
  \item elle est génératrice donc il existe une famille de scalaires \((x_i)_{i \in I}\) telle que \(x=\sum_{i \in I} x_i e_i\)
  \item elle est libre donc si \((x_i)_{i \in I}\) et \((y_i)_{i \in I}\) sont deux familles de scalaires telles que \(x=\sum_{i \in I} x_i e_i=\sum_{i \in I} y_i e_i\) alors pour tout \(i \in I\) \(x_i =y_i\).
  \end{itemize}

  Supposons que pour tout \(x \in E\) il existe un unique famille de scalaires \((x_i)_{i \in I}\) telle que \(x=\sum_{i \in I}x_i e_i\). Alors \(\E\) est génératrice (puisqu'il existe \ldots). Montrons qu'elle est libre. Soit une famille de scalaires \((\alpha_i)_{i \in I}\) telle que \(\sum_{i \in I} \alpha_i e_i=0\), or \(0=\sum_{i \in I} 0 e_i\) et par unicité d'une telle écriture, \(\forall i \in I \ \alpha_i=0\). Donc \(\E\) est libre.
\end{proof}

\emph{Exemples}~:
\begin{enumerate}
\item \((1,\ii)\) est une base du \(\R\)-espace vectoriel \(\C\);
\item deux vecteurs non colinéaires forment une base du plan;
\item trois vecteurs non coplanaires forment une base de l'espace;
\item dans \(E=\K^n\) (\(n \in \N^*\)), pour tout \(i \in \intervalleentier{1}{n}\) on a \(e_i = (\delta_{ij})_{j \in \intervalleentier{1}{n}}\) avec \(\delta\) le symbole de Kronecker défini par \(\delta_{ij}=1\) si \(i=j\) et zéro sinon. La famille \(\E=(e_i)_{i \in \intervalleentier{1}{n}}\) définie ainsi est une base appelée la base canonique.
\end{enumerate}

\subsection{Applications linéaires et familles de vecteurs}

Soient \(E\) et \(F\) deux \(\K\)-espaces vectoriels et \(u \in \Lin{E}{F}\). Pour toute famille \(\X=(x_i)_{i \in I} \in E^I\) on dispose de la famille \(u(\X)= (u(x_i))_{i \in I} \in F^I\) appelée famille image de la famille \(\X\).

\subsubsection{Propriétés de la famille \(u(\X)\) déduites de celle de \(u\) et de \(\X\)}

\begin{prop}\label{prop:propdeduitesu(x)}
  Soit \(u \in \Lin{E}{F}\) et \(\X=(x_i)_{i \in I} \in E^I\). Alors
    \begin{center}
      \begin{tabular}{|l|l|l|}\hline
        si \(\X\) est \ldots & si \(u\) est \ldots & alors \(u(\X)\) est \ldots \\\hline
        liée & & liée\\
        génératrice de \(E\) & & génératrice de \(\Image(u)\) \\
        génératrice de \(E\) & surjective & génératrice de \(F\) \\
        libre & injective & libre \\
        base de \(E\) & bijective & base de \(F\) \\ \hline
      \end{tabular}
    \end{center}

  \end{prop}
\begin{proof}
  Si \(\X\) est liée, alors par définition il existe une famille de scalaires \(\alpha\) indexée par \(I\) telle que \(\sum_{i \in I} \alpha_i x_i =0\) et il existe \(i_0 \in I\) tel que \(\alpha_{i_0} \neq 0\). L'application \(u\) est linéaire donc
  \begin{align}
    u\left(\sum_{i \in I} \alpha_i x_i \right) &= u(0) \\
    \sum_{i \in I} \alpha_i u(x_i) =0
  \end{align}
  Donc \(u(\X)\) est liée.

  Soit \(y \in \Image(u)\), il existe donc un vecteur \(x \in E\) tel que \(y=u(x)\). Comme \(\X\) est génératrice de \(E\) alors il existe une famille de scalaires \(\alpha\) indexée par \(I\) telle que \(x = \sum_{i \in I} \alpha_i x_i\). L'application \(u\) est linéaire donc \(y = u\left(\sum_{i \in I} \alpha_i x_i \right)=\sum_{i \in I} \alpha_i u(x_i)\). Alors \(u(\X)\) est génératrice de \(\Image(u)\).

  Si \(u\) est surjective et que \(\X\) est génératrice de \(E\) alors \(u(\X)\) est génératrice de \(\Image(u)=F\).

  Si \(\X\) est libre et \(u\) injective. Soit une famille de scalaires \(\alpha\) indexée par \(I\) telle que \(x = \sum_{i \in I} \alpha_i u(x_i)=0\). Comme \(u\) est linéaire on a \(u\left(\sum_{i \in I} \alpha_i x_i \right) =0\). Ainsi \(\sum_{i \in I} \alpha_i x_i \in \Ker(u)=\{0\}\) (puisque \(u\) est injective). Alors \(\sum_{i \in I} \alpha_i x_i = 0\) et comme \(\X\) est libre \(\alpha\) est la famille nulle. Donc \(u(\X)\) est libre.

  Si \(\X\) est une base de \(E\) et si \(u\) est bijective, alors~:
  \begin{itemize}
  \item \(\X\) est libre et \(u\) est injective donc \(u(\X)\) est libre;
  \item \(\X\) est génératrice de \(E\) donc \(u(\X)\) est génératrice de \(u(\X)\).
  \end{itemize}
  alors \(u(\X)\) est une base de \(F\).
\end{proof}

\subsubsection{Propriétés de \(u\) déduites de son effet sur une famille de vecteurs de \(E\)}

\begin{prop}\label{prop:propdeduitesu}
  Soit \(u \in \Lin{E}{F}\) et \(\X=(x_i)_{i \in I} \in E^I\). Alors
  \begin{center}
    \begin{tabular}{|l|l|l|}\hline
      si \(\X\) est \ldots & si \(u(\X)\) est \ldots & alors \(u\) est \ldots \\\hline
      & génératrice de \(F\) & surjective \\
      génératrice de \(E\) & libre & injective \\
      génératrice de \(E\) & base de \(F\) & bijective \\ \hline
    \end{tabular}
  \end{center}

\end{prop}
\begin{proof}
  Soit \(y \in F\). La famille \(u(\X)\) est génératrice de \(F\) donc il existe une famille de scalaire \(\alpha\) indexée par \(I\) telle que
  \begin{equation}
    y = \sum_{i \in I} \alpha_i u(x_i)=u \left( \sum_{i \in I} \alpha_i x_i \right),
  \end{equation}
  car \(u\) est linéaire. Alors \(y \in \Image(u)\) et donc on a \(F \subset \Image(u)\). On a aussi \(u \in \Lin{E}{F}\) donc \(\Image(u) \subset F\). Par double inclusion \(\Image(u)=F\). L'application \(u\) est donc surjective.

  Soit \(x \in \Ker(u) \subset E\). La famille \(\X\) est génératrice de \(E\) alors il existe une famille de scalaires \(\alpha\) indexée sur \(I\) telle que \(x = \sum_{i \in I} \alpha_i x_i\). Comme \(u\) est linéaire et comme \(x\) est dans le noyau de \(u\), on a \(0=u(x) = \sum_{i \in I} \alpha_i u(x_i)\). Par hypothèse la famille \(u(\X)\) est libre donc \(\alpha\) est la famille nulle et donc \(x=0\). On a montré que \(\Ker(u) \subset \{0\}\) et comme l'autre inclusion est triviale on a \(\Ker(u)=\{0\}\) et ainsi l'application \(u\) est injective.

  Si \(\X\) est génératrice et si \(u(\X)\) est une base alors \(u\) est bijective (conséquence des points précédents).
\end{proof}

\begin{theo}[Caractérisation des isomorphismes par l'image d'une base]
Soient \(E\) et \(F\) deux \(\K\)-espace vectoriels. On suppose que \(E\) admet une base finie. Soit \(u \in \Lin{E}{F}\). Il y a équivalence entre les assertions suivantes~:
\begin{enumerate}
\item \(u\) est un isomorphisme de \(E\) dans \(F\);
\item pour toute base \(\E\) de \(E\), \(u(\E)\) est une base de \(F\);
\item il existe une base \(\E_0\) de \(E\) telle que \(u(\E_0)\) soit une base de \(F\).
\end{enumerate}
\end{theo}
\begin{proof}
  \(1 \implies 2\), c'est une conséquence de la proposition~
ef
ef
ef
ef
ef
ef
ef
ef
ef
ef
ef
ef
ef
ef
ef
ef
ef
ef
ef
ef
ef
ef
ef
ef
ef
ef
ef
ef
ef
ef
ef
ef\ref\ref\ref\ref\ref\ref\ref\ref\ref\ref\ref\ref\ref\ref\ref\ref\ref\ref\ref\ref\ref\ref\ref\ref\ref\ref\ref\ref\ref\ref\ref\ref\ref\ref\ref\ref\ref\ref\ref\ref\ref\ref\ref\ref\ref\ref\ref\ref\ref\ref{prop:propdeduitesu(x)} cinquième ligne.

  \(2 \implies 3\), par hypothèse il existe une base \(\E_0\) de \(E\) à laquelle on applique \(2\).

  \(3 \implies 1\), c'est une conséquence de la proposition~
ef
ef
ef
ef
ef
ef
ef
ef
ef
ef
ef
ef
ef
ef
ef
ef
ef
ef
ef
ef
ef
ef
ef
ef
ef
ef
ef
ef
ef
ef
ef
ef\ref\ref\ref\ref\ref\ref\ref\ref\ref\ref\ref\ref\ref\ref\ref\ref\ref\ref\ref\ref\ref\ref\ref\ref\ref\ref\ref\ref\ref\ref\ref\ref\ref\ref\ref\ref\ref\ref\ref\ref\ref\ref\ref\ref\ref\ref\ref\ref\ref\ref{prop:propdeduitesu} troisième ligne.
\end{proof}

\subsubsection{Caractérisation d'une application linéaire par la donnée de l'image d'une base}

\begin{lemme}
  Soient \(E\) un \(\K\)-espace vectoriel, \(p \in \N\), \(p>1\) et \(\E=(e_i)_{i=1, \ldots, p}\) une base de \(E\). Pour tout vecteur \(x \in E\), il existe un unique \(p\)-uplet \((x_1, \ldots, x_p)\) tel que \(x = \sum_{i=1}^p x_ie_i\). On définit pour tout entier \(i \in \intervalleentier{1}{p}\) l'application ``coordonnée'' \(\fonction{C_i}{E}{\K}{x}{x_i}\). Alors pour tout \(i \in \intervalleentier{1}{p}\) l'application \(C_i\) est linéaire.
\end{lemme}
\begin{proof}
  Soient deux vecteurs \(x,z\) de \(E\) et un scalaire \(\lambda\). Comme \(\E\) est une base de \(E\), il existe deux uniques \(p\)-uplets \((x_1, \ldots, x_p)\) et \((z_1, \ldots, z_p)\) tels que \(x = \sum_{i=1}^p x_ie_i\) et \(z = \sum_{i=1}^p z_ie_i\). Alors
  \begin{equation}
    x+\lambda z = \sum_{i=1}^p (x_i + \lambda z_i)e_i
  \end{equation}
  Par unicité des coordonnées on a bien la linéarité quelque soit \(i \in \intervalleentier{1}{p}\) de l'application \(C_i\)~:
  \begin{equation}
    C_i(x+\lambda z)=C_i(x)+\lambda C_i(z)
  \end{equation}
\end{proof}

\begin{theo}\label{theo:caracapplinimagebase}
  Soient \(E\) et \(F\) deux \(\K\)-espaces vectoriels et \(p\) un naturel plus grand que \(1\). Soit \(\E=(e_i)_{i=1, \ldots, p}\) une base de \(E\). Soit \(\Y=(y_i)_{i=1, \ldots, p}\) une famille quelconque de \(p\) vecteurs de \(F\). Alors il existe une unique application \(u \in \Lin{E}{F}\) telle que \(u(\E)=\Y\).
\end{theo}
\begin{proof}[Unicité]
  Supposons qu'il existe \(u \in \Lin{E}{F}\) telle que \(u(\E)=\Y\). Comme \(\E\) est une base de \(E\), pour tout vecteur \(x \in E\) il existe un unique \(p\)-uplet \((x_1, \ldots, x_p)\) de scalaires tel que
  \begin{equation}
    x = \sum_{i=1}^p x_ie_i
  \end{equation}
  Alors comme \(u\) est linéaire on a
  \begin{equation}
    u(x)=\sum_{i=1}^px_i u(e_i) = \sum_{i=1}^px_i y_i
  \end{equation}
  donc \(u\) est unique (puisque les \(x_i\) et les \(y_i\) sont uniques).
\end{proof}
\begin{proof}[Existence]
  Si on prend les définitions du lemme, on prend l'application \(\fonction{u}{E}{F}{x}{\sum_{i=1}^pC_i(x)y_i}\), alors
  \begin{itemize}
  \item \(u\) est linéaire puisque pour tout \(i\) l'application \(C_i\) est linéaire (d'après le lemme)~: Soient deux vecteurs \(x,z\) de \(E\) et un scalaire \(\lambda\).
    \begin{align}
      u(x+\lambda z)&=\sum_{i=1}^p C_i(x+\lambda z)y_i \\
      &=\sum_{i=1}^p [C_i(x)+\lambda C_i(z)]y_i \\
      &=\sum_{i=1}^p C_i(x) y_i +\lambda \sum_{i=1}^p C_i(z)y_i\\
      &=u(x)+\lambda u(z)
    \end{align}
  \item Pour tout \(i_0 \in \intervalleentier{1}{p}\) \(u(e_{i_O})=\sum_{i=1}^p C_i(e_{i_0}) y_i\) et \(e_{i_0}=\sum_{i=1}^pc_i(e_{i_0}) e_i\) puisque \(c_i(e_{i_0})=1\) si \(i=i_0\) et zéro sinon. Donc \(u(e_{i_0})=c_{i_0}(e_{i_0})y_{i_0}=y_{i_0}\). Alors \(u(\E)=\Y\).
  \end{itemize}
\end{proof}

\emph{Remarque}~: La famille \(\E\) est une base de \(E\) donc elle est génératrice de \(E\), \(\Y=u(\E)\)
\begin{itemize}
\item Si \(\Y\) est libre alors \(u\) est injective;
\item Si \(\Y\) est génératrice de \(F\) alors \(u\) est surjective;
\item Si \(\Y\) est une base de \(F\) alors \(u\) est bijective.
\end{itemize}

\emph{Cas particulier}~: Soit un \(\K\)-espace vectoriel \(E\), \(p \in \N\) \(p>1\) et une famille \(\X=(x_1, \ldots, x_p)\) de \(p\) vecteurs de \(E\). On considère l'espace \(\K^p\) et on dispose de la base canonique \(\E=(e_1, \ldots, e_p)\) où pour tout \(i \in \intervalleentier{1}{p}\), \(e_i = (\delta_{i,j})_{j \in \intervalleentier{1}{p}}\). Il existe une unique application linéaire \(u : \K^p \rightarrow E\) telle que \(u(\E)=\X\).

\emph{Que peut on dire de \(u\)?}

Noyau~: Pour toute famille de \(p\) éléments \(\alpha \in \K^p\) on a
\begin{align}
  u(\alpha)=0 &\iff u\left(\sum_{i=1}^p \alpha_i e_i \right)=0\\
  &\iff \sum_{i=1}^p \alpha_i u(e_i) =0\\
  &\iff \sum_{i=1}^p \alpha_i x_i =0\\
  &\iff \alpha \text{~est une RDL de } \X.
\end{align}
Alors \(u\) est injective si et seulement si \(\X\) est libre.

Image~: Pour tout vecteur \(x \in E\) on a
\begin{align}
  y \in \Image(u) &\iff \exists \alpha \in \K^p \quad y=u(\alpha)\\
  &\iff \exists \alpha \in \K^p \quad y=u\left(\sum_{i=1}^p \alpha_i e_i \right)= \sum_{i=1}^p \alpha_i u(e_i) \\
  &\iff \exists \alpha \in \K^p \quad y=\sum_{i=1}^p \alpha_i x_i\\
  &\iff y \in \VectEngendre(\X).
\end{align}
Alors \(u\) est surjective si et seulement si \(E=\VectEngendre(\X)\), c'est-à-dire si et seulement si \(\X\) est génératrice de \(E\).

Finalement \(u\) est bijective si et seulement si \(\X\) est une base de \(E\).

\section{Espaces vectoriels de dimension finie}

\subsection{Base d'un espace vectoriel de dimension finie}

\begin{defdef}
  On appelle \(\K\)-espace vectoriel de dimension finie tout \(\K\)-espace vectoriel qui admet au moins une famille génératrice finie. Sinon il est dit de dimension infinie.
\end{defdef}

\begin{theo}[Extraction d'une base]
  Soient \(E\) un \(\K\)-espace vectoriel de dimension finie, \(\G\) une famille génératrice finie de \(E\), \(\Lb\) une famille libre de \(E\) telle que \(\Lb \subset \G\). Alors il existe une base \(\B\) de \(E\) telle que
  \begin{equation}
    \Lb \subset \B \subset \G
  \end{equation}
\end{theo}
\begin{proof}
  Premièrement, comme \(E\) est de dimension finie, il existe bien une telle famille \(\G\). Soit alors \(\G=(x_i)_{i \in I}\) avec \(I\) fini. Il existe une famille \(\Lb=(x_i)_{i \in \emptyset}\). La famille \(\Lb\) est libre et incluse dans \(\G\). Donc elles existent bien.

  Soit \(I\) et \(J\) finis tels que \(J \subset I\), \(\G=(x_i)_{i \in I}\) et \(\Lb=(x_i)_{i \in J}\). Soit
  \begin{equation}
    S = \enstq{\Card(J')}{J \subset J' \subset I \text{~et } (x_i)_{i \in J'} \text{~est libre}},
  \end{equation}
  alors~:
  \begin{itemize}
  \item \(S \subset \N\);
  \item \(S \neq \emptyset\) puisque \(\Card J \in S\) (\(\Lb\) est libre);
  \item \(S\) est majorée par \(\Card(I)\).
  \end{itemize}
  La partie \(S\) est une partie de \(\N\) non vide et majorée, elle admet donc un plus grand élément noté \(n_0\). Soit \(J_0\) l'ensemble tel que \(n_0=\Card(J_0)\). Soit \(\B=(x_i)_{i \in J_0}\). Montrons que \(\B\) est une base de \(E\). On sait déjà qu'elle est libre, il faut montrer qu'elle est génératrice de \(E\). Prouvons que tous les vecteurs de \(\G\) sont des combinaisons linéaires de vecteurs de \(\B\). 

  Soit \(i \in I\) et \(x_i \in \G\), alors~:
  \begin{itemize}
  \item si \(i \in J_0\) alors \(x_i \in \B\)
  \item si \(i \in I\setminus J_0\) alors si \(x_i\) n'est pas une combinaison linéaire de \(\B\) alors \(\B \cup \{x_i\}\) est une famille libre de cardinal \(n_0+1 >n_0\), ce qui est contradictoire puisque \(n_0\) est le plus grand cardinal d'une famille libre de \(E\). Alors \(x_i\) est une combinaison linéaire de vecteur de \(\B\) : \(x_i \in \VectEngendre(\B)\).
  \end{itemize}
  On a montré que \(\G \subset \VectEngendre(\B)\).

  Soit \(x \in E\). Comme la famille \(\G\) est génératrice, il existe une famille de scalaires \(\alpha\) indexée sur \(I\) telle que \(x = \sum_{i \in I} \alpha_i x_i\). Or \(\G \subset \VectEngendre(\B)\) donc pour tout \(i \in I\setminus J_0\) il existe une famille de scalaires \(\beta_i\) indexée sur \(J_0\) telle que \(x_i = \sum_{j \in J_0} \beta_{ij}x_j\). Alors
  \begin{equation}
    x= \sum_{i \in J_0} \alpha_i x_i + \sum_{i \in I\setminus J_0} \alpha_i \left(\sum_{j \in J_0} \beta_{ij}x_j \right) \in \VectEngendre(\B)
  \end{equation}
  Alors \(\B\) est une famille génératrice de \(E\). Finalement c'est une base de \(E\).
\end{proof}

\begin{theo}[Existence de base en dimension finie]
  Tout \(\K\)-espace vectoriel de dimension finie admet au moins une base (finie).
\end{theo}
\begin{proof}
  Soit un \(\K\)-espace vectoriel \(E\) de dimension finie. Alors par définition \(E\) admet au moins une famille génératrice finie \(\G=(x_i)_{i \in I}\) avec \(I\) fini. On prend une famille libre \(\Lb=(x_i)_{i \in \emptyset}\) et c'est une sous famille de \(G\). D'après le théorème d'extraction de base, il existe une base \(\B\) de E telle que \(\Lb \subset \B \subset \G\). Donc \(E\) admet au moins une base.
\end{proof}

\begin{theo}[Théorème de la base incomplète]
    \label{theo:baseincomplete}
  Soient \(E\) un \(\K\)-espace vectoriel de dimension finie et \(\Lb\) une famille libre de \(E\). Alors il existe une base \(\B\) de \(E\) telle que \(\Lb\) soit une sous famille de \(\B\).
\end{theo}
\begin{proof}
  Par définition, \(E\) admet une famille génératrice finie \(\G\). Soit \(\mathcal{H}=\G \cup \Lb\). Alors \(\mathcal{H}\) est une famille finie de vecteurs de \(E\) et de plus génératrice telle que \(\Lb \subset \mathcal{H}\).

  Par théorème d'extraction de base, il existe une base \(\B\) de \(E\) telle que
  \begin{align}
    \Lb \subset \B \subset \mathcal{H} \\
    \Lb \subset \B \subset \G \cup \Lb
  \end{align}
  \(\B\) est obtenue en complétant la famille libre \(\Lb\) par des vecteurs de la famille génératrice \(\G\)
\end{proof}

\subsection{Dimension d'un espace vectoriel de dimension finie}

\begin{lemme}
  Soit \(n \in N\) on pose \(\P(n)\) ``Soient \(E\) un \(\K\)-espace vectoriel, \(\G\) une famille de \(n\) vecteurs de \(E\), \(\X\) une famille de \(n+1\) vecteurs de \(E\). On suppose que \(\X \subset \VectEngendre(\G)\). Alors la famille \(\X\) est liée''
\end{lemme}
\begin{proof}[Démonstration par récurrence]
  \emph{Initialisation}~: pour \(n=0\) \(\G=\emptyset\), \(\X=\{x_1\}\) \(\X \subset \VectEngendre(\emptyset)=\{0\}\) donc \(x_1=0\) et \(\X\) est liée. \(\P(0)\) est vraie.

  \emph{Hérédité}~: Soit \(n \in \N\) on suppose \(\P(n)\). Démontrons \(\P(n+1)\). Soit \(\G=(g_1, \ldots, g_{n+1})\) et \(\X=(x_1, \ldots, x_{n+2})\). \(\X \subset \VectEngendre(\G)\) c'est-à-dire que pour tout \(j \in \intervalleentier{1}{n+2} \quad x_j \in \VectEngendre(\G)\) donc il existe \(y_j \in E\) et \(\lambda_j \in \K\) tel que
  \begin{equation}
    \begin{cases} x_j = y_j + \lambda_j g_{n+1} \\ y_j \in \VectEngendre(g_1, \ldots, g_n) \end{cases}
  \end{equation}
  Deux cas de figure se présentent~:
  \begin{enumerate}
  \item Soit pour tout \(j \in \intervalleentier{1}{n+1}\) \(\lambda_j=0\) et donc pour tout \(j \in \intervalleentier{1}{n+1}\) \(x_j=y_j \in \VectEngendre(g_1, \ldots, g_n)\), donc par hypothèse de récurrence la famille \((x_1, \ldots, x_{n+1})\) est liée. Par conséquent \(\X\) est liée aussi.
  \item Soit il existe \(j_0 \in \intervalleentier{1}{n+2}\) \(\lambda_j \neq 0\) et quitte à ré-indexer, supposons que c'est \(\lambda_1\). Ainsi
    \begin{align}
      x_1 &= y_1 + \lambda_1 g_{n+1} \\
      g_{n+1} &= \lambda_1^{-1}(x_1-y_1) \\
      \forall j \in \intervalleentier{2}{n+2} \quad x_j = y_j + \lambda_j g_{n+1} &= y_j + \lambda_j \lambda_1^{-1}(x_1-y_1)\\
      \forall j \in \intervalleentier{2}{n+2} \quad x_j - \lambda_j\lambda_1^{-1}x_1 &= y_j - \lambda_j \lambda_1^{-1} y_1
    \end{align}
    Alors pour tout \(j \in \intervalleentier{2}{n+2}\)
    \begin{equation}
      x_j - \lambda_j\lambda_1^{-1}x_1 \in \VectEngendre(y_1, \ldots, y_{n+2}) \subset \VectEngendre(g_1, \ldots, g_n)
    \end{equation}
    Alors par hypothèse de récurrence la famille \((x_j-\lambda_j\lambda_1^{-1}x_1)_{j \in \intervalleentier{2}{n+2}}\) est liée. Il existe donc une famille de scalaires \(\alpha\) indexée sur \(\intervalleentier{2}{n+2}\) telle que
    \begin{equation}
      \sum_{j=2}^{n+2} \alpha_j (x_j-\lambda_j\lambda_1^{-1}x_1) = 0 \quad \exists j_0 \in \intervalleentier{2}{n+2} \ \alpha_{j_0} \neq 0
    \end{equation}
    si on pose \(\alpha_1=-\sum_{j=2}^{n+2} \alpha_j \lambda_j \lambda_1^{-1}\) on a alors
    \begin{equation}
      \sum_{j=1}^{n+2} \alpha_j x_j =0 \quad \exists j_0 \in \intervalleentier{2}{n+2} \ \alpha_{j_0} \neq 0
    \end{equation}
    Donc \(\X\) est liée. \(\P(n+1)\) est vraie.
  \end{enumerate}

  \emph{Conclusion}~: Comme \(\P(0)\) est vraie et que \(\P(n) \implies \P(n+1)\) est aussi vraie, alors par théorème de récurrence, l'assertion \(\P(n)\) est vraie pour tout naturel \(n\).
\end{proof}

\begin{cor}
  Soit \(E\) un \(\K\)-espace vectoriel de dimension finie. Soit \(\Lb\) une famille libre de \(E\), \(\G\) une famille génératrice de \(E\). Alors
  \begin{equation}
    \Card(\Lb) \leqslant \Card(\G)
  \end{equation}
\end{cor}
\begin{proof}[Démonstration par l'absurde]
  Soit \(n = \Card(\G)\). Supposons que \(\Card(\Lb) \geqslant n+1\), alors il existe une sous famille \(\Lb'\) de \(\Lb\) telle que \(\Card(\Lb')=n+1\) et \(\Lb' \subset \VectEngendre(\G)=E\). D'après le lemme \(\Lb'\) serait liée, ce qui est impossible puisque c'est une sous famille d'une famille libre, donc \(\Lb'\) est libre. Absurde.  Ainsi 
  \begin{equation}
    \Card(\Lb) \leqslant \Card(\G)
  \end{equation}
\end{proof}

\begin{theo}[Théorème de la dimension]
    \label{theo:dim}
  Soit un \(\K\)-espace vectoriel \(E\) de dimension finie. Toutes les bases de \(E\) sont finies de même cardinal. Ce cardinal est appelé la dimension de \(E\) sur \(\K\) noté \(\Dim_\K E\).
\end{theo}
\begin{proof}
  Soient deux bases de \(E\) \(\B_1\) et \(\B_2\). Comme \(\B_1\) et libre et \(\B_2\) est génératrice d'après le corollaire \(\Card(\B_1) \leqslant \Card(\B_2)\) et en inversant les rôles on obtient l'inégalité réciproque. Finalement \(\Card(\B_1)=\Card(\B_2)\).
\end{proof}

\emph{Remarque}~: Ne pas oublier de préciser le corps de l'espace vectoriel, puisque \(\Dim_\C \C=1\) et \(\Dim_\R \C=2\). La dimension de l'espace nul est nulle car il est engendré par l'ensemble vide.

\begin{theo}[Caractérisation des bases parmi les familles génératrices]
  Soit un \(\K\)-espace vectoriel \(E\) de dimension finie. Soit \(\G\) une famille génératrice de \(E\). Alors
  \begin{equation}
    \Card(\G) \geqslant \Dim_\K E
  \end{equation}
et
\begin{equation}
  \Card(\G)=\Dim_\K E \iff \G \text{~est une base de } E
\end{equation}
\end{theo}
\begin{proof}
  Soit \(\B\) une base de \(E\). Alors \(\Card(\B)=\Dim_\K E\) et \(\B\) est libre donc d'après le corollaire \(\Card(\B) = \Dim_\K E \leqslant \Card(\G)\).

  \(\impliedby\)~: Démontrable grâce au théorème de la dimension, théorème~
ef
ef
ef
ef
ef
ef
ef
ef
ef
ef
ef
ef
ef
ef
ef
ef
ef
ef
ef
ef
ef
ef
ef
ef
ef
ef
ef
ef
ef
ef
ef
ef\ref\ref\ref\ref\ref\ref\ref\ref\ref\ref\ref\ref\ref\ref\ref\ref\ref\ref\ref\ref\ref\ref\ref\ref\ref\ref\ref\ref\ref\ref\ref\ref\ref\ref\ref\ref\ref\ref\ref\ref\ref\ref\ref\ref\ref\ref\ref\ref\ref\ref{theo:dim}.
  \(\implies\)~: La famille \(\G\) est génératrice de \(E\) donc il existe une base \(\B\) de \(E\) telle que \(\B \subset \G\) or \(\Card(\B) = \Dim_\K E\) (puisque c'est une base) et par hypothèse \(\Card(\G) = \Dim_\K E\) alors
  \begin{equation}
    \begin{cases} \B \subset \G \\ \Card(\B) = \Card(\G) \end{cases} \implies \B=\G
  \end{equation}
  alors \(\G\) est une base.
\end{proof}

\begin{theo}[Caractérisation des bases parmi les familles libres]
  Soit un \(\K\)-espace vectoriel \(E\) de dimension finie. Soit \(\Lb\) une famille libre de \(E\). Alors
  \begin{equation}
    \Card(\Lb) \leqslant \Dim_\K E,
  \end{equation}
  et
  \begin{equation}
    \Card(\Lb)=\Dim_\K E \iff \Lb \text{~est une base de } E
  \end{equation}
\end{theo}
\begin{proof}
  Soit \(\B\) une base de \(E\), alors elle est génératrice donc \(\Card(\Lb) \leqslant \Card(\B)=\Dim_\K E\).

  \(\impliedby\)~: Démontrable grâce au théorème de la dimension, théorème~
ef
ef
ef
ef
ef
ef
ef
ef
ef
ef
ef
ef
ef
ef
ef
ef
ef
ef
ef
ef
ef
ef
ef
ef
ef
ef
ef
ef
ef
ef
ef
ef\ref\ref\ref\ref\ref\ref\ref\ref\ref\ref\ref\ref\ref\ref\ref\ref\ref\ref\ref\ref\ref\ref\ref\ref\ref\ref\ref\ref\ref\ref\ref\ref\ref\ref\ref\ref\ref\ref\ref\ref\ref\ref\ref\ref\ref\ref\ref\ref\ref\ref{theo:dim}.  
  \(\implies\)~: Si \(\Lb\) est une famille libre, on peut la compléter en une base \(\B\) de \(E\) (\(\Lb \subset \B\)) par théorème de la base incomplète, théorème~
ef
ef
ef
ef
ef
ef
ef
ef
ef
ef
ef
ef
ef
ef
ef
ef
ef
ef
ef
ef
ef
ef
ef
ef
ef
ef
ef
ef
ef
ef
ef
ef\ref\ref\ref\ref\ref\ref\ref\ref\ref\ref\ref\ref\ref\ref\ref\ref\ref\ref\ref\ref\ref\ref\ref\ref\ref\ref\ref\ref\ref\ref\ref\ref\ref\ref\ref\ref\ref\ref\ref\ref\ref\ref\ref\ref\ref\ref\ref\ref\ref\ref{theo:baseincomplete}. Alors \(\Card(\B)=\Dim_\K E = \Card(\Lb)\). Ainsi \(\Lb=\B\) et c'est une base de \(E\).
\end{proof}

\subsection{Théorème d'isomorphisme}

\begin{prop}
  Soit un \(\K\)-espace vectoriel \(E\) de dimension finie et \(F\) un autre \(\K\)-espace vectoriel. On suppose qu'il existe un isomorphisme de \(E\) dans \(F\). Alors \(F\) est de dimension finie et \(\Dim_\K F = \Dim_\K E\).
\end{prop}
\begin{proof}
  Soit \(n = \Dim_\K E\), \(\B\) une base de \(E\) (\(\Dim_\K E = \Card(\B)\)) et \(u\) un isomorphisme de \(E\) dans \(F\). Alors la famille \(u(\B)\) est une base de \(F\) (puisque \(u\) est bijective et \(\B\) est une base). La famille \(u(\B)\) est de cardinal fini le même que \(\B\). Ainsi \(\Dim_\K F = \Dim_\K E\).
\end{proof}

\begin{theo}[Théorème d'isomorphisme]
  Soient deux \(\K\)-espaces vectoriels de dimension finie \(E\) et \(F\). Ces deux espaces vectoriels sont isomorphes si et seulement s'ils ont la même dimension.
\end{theo}
\begin{proof}
  S'ils sont isomorphes, la proposition précédente nous dit qu'ils ont la même dimension.

  Supposons qu'ils ont la même dimension \(n\). Soient \(\B=(b_1, \ldots, b_n)\) une base de \(E\) et \(\E=(e_1, \ldots, e_n)\) une base de \(F\). D'après le théorème~
ef
ef
ef
ef
ef
ef
ef
ef
ef
ef
ef
ef
ef
ef
ef
ef
ef
ef
ef
ef
ef
ef
ef
ef
ef
ef
ef
ef
ef
ef
ef
ef\ref\ref\ref\ref\ref\ref\ref\ref\ref\ref\ref\ref\ref\ref\ref\ref\ref\ref\ref\ref\ref\ref\ref\ref\ref\ref\ref\ref\ref\ref\ref\ref\ref\ref\ref\ref\ref\ref\ref\ref\ref\ref\ref\ref\ref\ref\ref\ref\ref\ref{theo:caracapplinimagebase} il existe une unique application \(u \in \Lin{E}{F}\) telle que \(u(\B)=\E\).

  Comme \(\B\) est une base de \(E\) (donc génératrice), alors \(\E\) une base de \(F\) et \(u(\B)=\E\). Donc \(u\) est bijective (proposition~
ef
ef
ef
ef
ef
ef
ef
ef
ef
ef
ef
ef
ef
ef
ef
ef
ef
ef
ef
ef
ef
ef
ef
ef
ef
ef
ef
ef
ef
ef
ef
ef\ref\ref\ref\ref\ref\ref\ref\ref\ref\ref\ref\ref\ref\ref\ref\ref\ref\ref\ref\ref\ref\ref\ref\ref\ref\ref\ref\ref\ref\ref\ref\ref\ref\ref\ref\ref\ref\ref\ref\ref\ref\ref\ref\ref\ref\ref\ref\ref\ref\ref{prop:propdeduitesu}). Ainsi \(E\) et \(F\) sont isomorphes.
\end{proof}

\begin{corth}
  Soient \(E\) un \(\K\)-espace vectoriel et \(n \in \N\). Alors \(E\) et \(\K^n\) sont isomorphes si et seulement si \(E\) est de dimension finie égale à \(n\).
\end{corth}
\begin{proof}
  Si \(E\) est de dimension finie, on applique le théorème précédent avec \(F=\K^n\). Si \(E\) n'est pas de dimension finie alors il n'est pas isomorphe à \(\K^n\) et il n'est pas de dimension finie égale à \(n\).
\end{proof}

\subsection{Dimension du produit cartésien \(E \times F\)}

\begin{theo}
  Soient \(E\) et \(F\) deux \(\K\)-espaces vectoriels de dimension finie. Alors le produit cartésien \(E \times F\) est de dimension fine et
  \begin{equation}
    \Dim_\K E \times F = \Dim_\K E + \Dim_\K F.
  \end{equation}
\end{theo}
\begin{proof}
On note \(p=\Dim_\K E\) et \(n=\Dim_\K F\). Soient \(\B=(b_1,\ldots, b_p)\) une base de \(E\) et \(\E=(e_1, \ldots, e_n)\) une base de \(F\). Pour tous vecteurs \(x \in E\) et \(y \in F\) il existe deux uniques familles de scalaires \((x_i)_{i \in \intervalleentier{1}{p}}\) et \((y_i)_{i \in \intervalleentier{1}{n}}\) telles que
    \begin{equation}
      x = \sum_{i=1}^p x_i b_i \quad y = \sum_{i=1}^n y_i e_i.
    \end{equation}
    Donc on a
    \begin{equation}
      (x,y) = \sum_{i=1}^p x_i (b_i,0) + \sum_{i=1}^n y_i (0,e_i).
    \end{equation}
Soit la famille \(\mathcal{F}=(f_k)_{k \in \intervalleentier{1}{n+p}}\) définie par~:
\begin{itemize}
\item si \(1 \leqslant k \leqslant p\) alors \(f_k=(b_k,0)\);
\item si \(p+1 \leqslant k < n\) alors \(f_k=(0,e_{k-p})\).
\end{itemize}
Ainsi
\begin{equation}
  (x,y)=\sum_{i=1}^p x_i f_i + \sum_{j=p+1}^{n+p} y_{j-p} f_j.
\end{equation}
On vient de montrer que la famille \(\mathcal{F}\) est génératrice de \(E \times F\) de cardinal \(n+p\). On a montré que \(E \times F\) est de dimension finie et que \(\Dim_\K E\times F \leqslant n+p\). 

Montrons qu'elle est libre. Soit \((\lambda_k)_{k \in \intervalleentier{1}{n+1}} \in \K^{n+p}\) telle que \(\sum_{i=1}^{n+p} \lambda_i f_i =(0,0)\). Alors
\begin{align}
 &\iff \sum_{k=1}^n \lambda_k (b_k,0) + \sum_{k=p+1}^{n+p} \lambda_k (0,e_{k-p}) =(0,0)\\
 &\iff \left(\sum_{k=1}^n \lambda_k b_k,0 \right) + \left(0,\sum_{k=p+1}^{n+p} \lambda_k e_{k-p}\right) =(0,0)\\
 &\iff \left(\sum_{k=1}^n \lambda_k b_k,\sum_{k=p+1}^{n+p} \lambda_k e_{k-p}\right) = (0,0) \\
 &\iff \begin{cases} \sum_{k=1}^n \lambda_k b_k = 0 \\ \sum_{k=p+1}^{n+p} \lambda_k e_{k-p} =0 \end{cases}
\end{align}
Or \(\B=(b_1,\ldots, b_p)\) est une base de \(\E\) donc a fortiori \(\B\) est libre, ainsi pour tout \(i \in \intervalleentier{1}{p} \ \lambda_k=0\). De même \(\E\) est une base de de \(F\) donc elle est libre, ainsi pour tout \(i \in \intervalleentier{1}{n} \ \lambda_{j+p}=0\). Alors \(\mathcal{F}\) est une famille libre.

Finalement c'est une base de \(E \times F\) et elle est de cardinal \(n+p\) donc \(E \times F\) est de dimension \(n+p\).
\end{proof}

\subsection{Dimension de \(\Lin{E}{F}\)}

\begin{theo}
  Soient \(E\) et \(F\) deux \(\K\)-espaces vectoriels de dimension finie. Alors l'ensemble \(\Lin{E}{F}\) des applications linéaires de \(E\) vers \(F\) est de dimension fine et
  \begin{equation}
    \Dim_\K \Lin{E}{F} = \Dim_\K E \times \Dim_\K F.
  \end{equation}
\end{theo}
\begin{proof}
    On note \(p=\Dim_\K E\) et \(n=\Dim_\K F\). Soit \(\B=(b_1, \ldots, b_p)\) une base de \(E\) et \(\B'=(f_1, \ldots, f_n)\) une base de \(F\). On définit une famille \(\Phi\) d'applications linéaires de \(E\) dans \(F\)~: \(\Phi = (\varphi_{i,j})_{(i,j) \in \intervalleentier{1}{n} \times \intervalleentier{1}{p}}\) telle que pour tout couple \((i,j)\), \(\varphi_{i,j}\) soit définie par son action sur la base \(\B\)~:
  \begin{equation}
    \forall k \in \llbracket 1,p \rrbracket \quad \varphi_{i,j}(e_k)= \delta_{jk} f_i
  \end{equation}
  avec \(\delta\) le symbole de Kronecker. Montrons que la famille \(\Phi\) est une base de \(\Lin{E}{F}\). On le montre en deux temps~:
  \begin{enumerate}
  \item Montrons que \(\Phi\) est libre. Soit \((\lambda_{ij})_{(i,j) \in \intervalleentier{1}{n} \times \intervalleentier{1}{p}} \in \K^{np}\) telle que
    \begin{equation}
        \sum_{i=1}^n \sum_{j=1}^p \lambda_{ij} \varphi_{ij} =0_{\Lin{E}{F}}.
    \end{equation}
    C'est-à-dire que pour tout \(k \in \intervalleentier{1}{p}\) on a
    \begin{align}
      \sum_{i=1}^n \sum_{j=1}^p \lambda_{ij} \varphi_{ij}(e_k) &=0_F\\
      \sum_{i=1}^n \sum_{j=1}^p \lambda_{ij} \delta_{jk} f_i &=0_F\\
      \sum_{i=1}^n \lambda_{ik} f_i &=0_F.
    \end{align}
    Or \(\B'=(f_1, \ldots, f_n)\) est une base de \(F\) donc a fortiori libre. Ainsi pour tout \(k \in \intervalleentier{1}{p}\) et pour tout \(i \in \intervalleentier{1}{n}\), \(\lambda_{ik}=0\). Alors \(\Phi\) est libre.
  \item  Montrons que \(\Phi\) est génératrice de \(\Lin{E}{F}\). Soit \(f \in \Lin{E}{F}\). Pour tout \(k \in \intervalleentier{1}{p}\), on a \(f(e_k) \in F\). \(\B'\) est une base de \(F\) donc il existe une famille de scalaires \((\lambda_{ik})_{i \in  \intervalleentier{1}{n}} \in \K^n\) telle que
    \begin{equation}
      f(e_k)= \sum_{i=1}^n \lambda_{ik} f_i.
    \end{equation}
    Comme \(\sum_{j=1}^p \delta_{jk}=1\) on peut l'introduire
    \begin{align}
      f(e_k) = \sum_{i=1}^n \lambda_{ik} \left(\sum_{j=1}^p \delta_{jk}\right) f_i &= \sum_{i=1}^n \sum_{j=1}^p \lambda_{ik} \delta_{jk} f_i \\
      &=\sum_{i=1}^n \sum_{j=1}^p \lambda_{ij} \delta_{jk} f_i\\
      &=\sum_{i=1}^n \sum_{j=1}^p \lambda_{ij} \varphi_{ij}(e_k)\\
      &=\left(\sum_{i=1}^n \sum_{j=1}^p \lambda_{ij} \varphi_{ij}\right)(e_k)
    \end{align}
    Donc \(f \in \VectEngendre(\Phi)\). Alors \(\Phi\) est génératrice de \(\Lin{E}{F}\).
  \end{enumerate}
  Donc \(\Phi\) est une base de \(E\) de cardinal \(np\) donc \(\Lin{E}{F}\) est de dimension finie égale à \(\Dim_\K E \times \Dim_\K F\).
\end{proof}

\emph{Remarque}~: il existe une démonstration avec les matrices.

\begin{corth}
  Si \(E\) est un \(\K\)-espace vectoriel de dimension finie alors \(\Endo{E}\) est de dimension finie égale à \((\Dim_\K E)^2\). Le dual \(E^*\) est de dimension finie égale à celle de \(E\).
\end{corth}

\emph{Remarque, notion de la base duale} Si \(E\) est un \(\K\)-espace vectoriel de dimension finie \(p\) et soit \(\B=(e_1, \ldots, e_p)\) une base de \(E\). Pour tout \(j \in  \intervalleentier{1}{p}\) soit \(e_j^*\) l'unique forme linéaire telle que pour tout \(i \in  \intervalleentier{1}{p}\), on ait \(e_j^*(e_i)=\delta_{ij}\).

\begin{prop}
  Si \(E\) est un \(\K\)-espace vectoriel de dimension finie \(p\) et soit \(\B=(e_1, \ldots, e_p)\) une base de \(E\). Pour tout \(j \in  \intervalleentier{1}{p}\), soit \(e_j^*\) l'unique forme linéaire telle que pour tout \(i \in  \intervalleentier{1}{p}\), on ait \(e_j*(e_i)=\delta_{ij}\). Alors la famille \(\B^* = (e_1^*, \ldots, e_p^*)\) est une base de \(E^*\) appelée base duale de la base \(\B\).
\end{prop}
\begin{proof}
  On sait déjà que \(E^*\) est de dimension finie égale à \(p\). Il suffit de montrer que \(\B^*\) est génératrice ou qu'elle est libre. Soit \(\varphi \in E^*\). Pour tout \(k \in \llbracket 1,p \rrbracket\) on a~:
  \begin{align}
    \varphi(e_k)&= \varphi(e_k) \sum_{i=1}^p \delta_{ik} \\
                &= \sum_{i=1}^p \varphi(e_i) \delta_{ik} \\ 
                &=\sum_{i=1}^p \varphi(e_i) e_i^*(e_k)  \\
                &=\left(\sum_{i=1}^p \varphi(e_i) e_i^* \right)(e_k)
  \end{align}
  donc \(\varphi = \sum_{i=1}^p \varphi(e_i) e_i^*\) et alors \(\B^*\) est génératrice. Alors c'est une base.
\end{proof}

\emph{Coordonnées d'un vecteur de \(\B\)}~: Si \(x \in E\) il existe un unique \(p\)-uplet \((x_1, \ldots, x_p) \in \K^p\) tel que \(x = \sum_{i=1}^p x_i e_i\) (coordonnées de \(x\) dans \(\B\)). Pour tout \(j \in  \intervalleentier{1}{p}\), l'application \(e_j^*\) est telle que
\begin{equation}
  e_j^*(x)=\sum_{i=1}^p x_i e_j^*(e_i) = \sum_{i=1}^p x_i \delta_{ij} = x_j.
\end{equation}
La famille \(\B^*\) est la famille des applications coordonnées de la base \(\B\).

\section{Sous-espaces vectoriels en dimension finie}

\subsection{Sous-espace vectoriel d'un espace vectoriel de dimension finie}

\begin{theo}
  Tout sous-espace vectoriel \(E_1\) d'un \(\K\)-espace vectoriel \(E\) de dimension finie est de dimension finie et
  \begin{equation}
    \Dim_\K E_1 \leqslant \Dim_\K E
  \end{equation}
de plus \(\Dim_\K E_1 = \Dim_\K E \iff E_1 =E\).
\end{theo}
\begin{proof}
  Soit \(\B\) une base de \(E\). On définit 
\begin{equation}
  \courbe=\enstq{\Card \Lb}{\Lb \text{~est une famille libre de } E_1}. 
\end{equation}
Déjà \(\courbe\) est une partie non vide (\(0 \in \courbe\), \(\Lb=\emptyset\)) de \(\N\) et elle est majorée par \(\Dim_\K E\). Alors elle admet un plus grand élément noté \(p\). Soit \(\L\) une famille libre de \(E_1\) de cardinal \(p\). 

Montrons qu'elle est génératrice. Soit \(x \in E_1\), si \(x \in \Lb\) alors on a directement \(x \in \VectEngendre(\Lb)\). Sinon on pose \(\Lb' = \Lb \cup \{x\}\). La famille \(\Lb'\) est dans \(E_1\) et admet \(p+1\) vecteurs. Comme \(p+1 > \max \courbe\), elle ne peut pas être libre. Alors comme \(\Lb' =\Lb\cup\{x\}\) est liée et que \(\Lb\) est libre, on sait que \(x \in \VectEngendre(\Lb)\). Alors \(\Lb\) est génératrice de \(E_1\).

Finalement \(\Lb\) est une base de \(E_1\). Alors \(E_1\) est est de dimension finie égale à \(p\).
\end{proof}
\begin{proof}[Démonstration de l'équivalence]
  S'ils sont égaux alors ils ont la même dimension (trivial). Supposons qu'ils aient la même dimension et soit \(\B\) une base de \(E_1\). Elle est libre dans \(E_1\) donc dans \(E\) et \(\Card(\B)=\Dim_\K E_1=\Dim_\K E\) alors \(\B\) est une base de \(E\). Alors a fortiori \(\B\) est génératrice de \(E\). \(E=\VectEngendre(\B)=E_1\).
\end{proof}

\emph{Vocabulaire}~: Soit \(E_1\) un sous-espace vectoriel d'un \(\K\)-espace vectoriel \(E\). Si \(\Dim_\K E_1=1\) alors \(E_1\) est une droite vectorielle, si \(\Dim_\K E_1=2\) alors \(E_1\) est un plan vectoriel et si \(\Dim_\K E_1 = \Dim_\K E -1\) alors \(E_1\) est un hyperplan de \(E\).

\begin{defdef}
  Soit \(W_1\) un sous-espace affine d'un espace vectoriel réel \(E\) de dimension finie. On appelle dimension de \(W_1\) la dimension de sa direction \(E_1\). Si \(\Dim_\R E_1=1\) alors \(W_1\) est une droite affine, si \(\Dim_\R E_1=2\) alors \(W_1\) est un plan affine et si \(\Dim_\R E_1=\Dim_\R E -1 \) alors \(W_1\) est un hyperplan affine.
\end{defdef}

\emph{Remarque}~: Le corps \(\K\) est un espace vectoriel sur lui-même de dimension \(1\). Les sous-espaces vectoriels de \(\K\) sont de dimension \(0\) ou \(1\). \(\K\) admet deux sous-espaces vectoriels~: \(\K\) et \(\{0\}\).

\subsection{Rang d'une famille de vecteurs}

\begin{defdef}
  Soit un \(\K\)-espace vectoriel \(E\) et soit \(\X\) une famille de vecteurs de \(E\). On dit que \(\X\) est de rang fini si le sous-espace vectoriel \(\VectEngendre(\X)\) est de dimension finie, auquel cas on définit le rang de la famille de vecteurs \(\X\) par
  \begin{equation}
    \rg(\X)=\Dim_\K \VectEngendre(\X).
  \end{equation}
  Sinon on dit que \(\X\) est de rang infini.
\end{defdef}

\begin{prop}
  Soit un \(\K\)-espace vectoriel \(E\) de dimension finie. Alors toute famille \(\X\) de vecteurs de \(E\) est de rang fini et vérifie
  \begin{equation}
    \rg(\X) \leqslant \Dim_\K E.
  \end{equation}
De plus \(\rg(\X) = \Dim_\K E\) si et seulement si \(\X\) est génératrice de \(E\).
\end{prop}
\begin{proof}
  Soit \(E_1=\VectEngendre(\X)\). \(E_1\) est un sous-espace vectoriel de \(E\), donc il est de dimension finie et \(\Dim_\K E_1 \leqslant \Dim_\K E\). De plus
  \begin{align}
    \rg(\X) = \Dim_\K E &\iff \Dim_\K E_1 = \Dim_\K E\\
    & \iff E_1=\VectEngendre(\X)=E,
  \end{align}
  ce qui est équivalent à ce que \(\X\) est génératrice de \(E\).
\end{proof}

\begin{prop}\label{prop:caracrangbase}
   Soit un \(\K\)-espace vectoriel \(E\). Soit \(\X\) une famille génératrice finie de vecteurs de \(E\). Alors \(\X\) est de rang fini et \(\rg(\X) \leqslant \Card(\X)\). De plus \(\rg(\X)=\Card(\X)\) si et seulement si \(\X\) est libre.
\end{prop}
\begin{proof}
  Soit \(E_1=\VectEngendre(\X)\). Comme \(\X\) est une famille génératrice finie de \(E_1\), \(E_1\) est de dimension finie et \(\Dim_\K E_1 \leqslant \Card(\X)\) et donc \(\rg(\X) \leqslant \Card(\X)\). De plus 
  \begin{align}
    \rg(\X) = \Card(\X) &\iff \Dim_\K E_1 = \Card(\X)\\
    &\iff \X \text{~est une base de } E_1\\
    &\iff \X \text{~est libre dans } E_1\\
    &\iff \X \text{~est libre dans } E\\
  \end{align}
\end{proof}

\begin{theo}[Caractérisation du rang]
  Soient \(E\) un \(\K\)-espace vectoriel, \(\X\) une famille de vecteurs de \(E\) de rang fini. Alors le rang de \(\X\) est égal au plus grand nombre de vecteurs linéairement indépendant que l'on peut extraire de \(\X\), ou encore au cardinal de la plus grande famille libre que l'on peut extraire de \(\X\).
\end{theo}
\begin{proof}
  Soit l'ensemble
  \begin{equation}
    \courbe = \enstq{\Card \Lb}{\Lb \text{~famille libre }, \Lb \subset \X},
  \end{equation}
  si \(\Lb\) est une famille libre telle que \(\Lb \subset \X\), alors \(\L\) est une famille libre de \(\VectEngendre(\X)\). Par hypothèse, \(\VectEngendre(\X)\) est de dimension finie donc \(\Card \Lb \leqslant \Dim_\K \VectEngendre(\X)=\rg(\X)\). Alors \(\courbe\) est une partie de \(\N\) non vide (puisque \(0 \in \courbe\)) et majorée. Alors elle admet un plus grand élément noté \(p\).

  On a \(p \leqslant \rg(\X)\), puisque c'est un majorant, et il existe une famille \(\Lb \subset \X\) telle que \(\Card \Lb =p\). Comme \(\X\) est génératrice de \(\VectEngendre(\X)\) et que \(\Lb\) est libre incluse dans \(\X\) il existe une base \(\B\) de \(\VectEngendre(\X)\) telle que \(\Lb \subset \B \subset \X\). Comme la base \(\B\) est une famille libre on a \(\Card \B \in \courbe\). Comme c'est aussi une base \(\Card \B = \Dim_\K \VectEngendre (\X) = \rg(\X)\). D'où \(\rg(\X) \leqslant p = \max \courbe\).

On a alors les deux inégalité
\begin{equation}
  \begin{cases}
    \rg(\X) \leqslant p \\ \rg(\X) \geqslant p,
  \end{cases}
\end{equation}
d'où l'égalité \(\rg(\X) = p= \max \courbe\).
\end{proof}

\subsection{Bases et sous-espaces vectoriels supplémentaires}

\begin{theo}\label{theo:theosuppdimfinie}
  Soient \(E\) un \(\K\)-espace vectoriel de dimension finie. Soit \(\E=(e_i)_{i \in I}\) (\(I\) fini) une base de \(E\). Soit \((I_1, I_2)\) une partition de \(I\), c'est à dire que \(I_1 \subset I\), \(I_2 \subset I\), \(I_1 \cup I_2 =I\) et \(I_1 \cap I_2 = \emptyset\). On note \(E_1 = \VectEngendre((e_i)_{i \in I_1})\) et \(E_2 = \VectEngendre((e_i)_{i \in I_2})\). Alors
  \begin{equation}
    E = E_1 \oplus E_2.
  \end{equation}
\end{theo}
\begin{proof}
  \(E_1\) et \(E_2\) sont des sous-espace vectoriel de \(E\) donc \(E_1+E_2\) est un sous-espace vectoriel de \(E\). Soit \(x \in E\). Comme \(\E\) est une base de \(E\) il existe \((x_i)_{i \in I} \in \K^I\) telle que \(x= \sum_{i \in I} x_i e_i\). 

  Comme \((I_1,I_2)\) est une partition de \(I\), on a aussi \(x = \sum_{i \in I_1} x_i e_i + \sum_{i \in I_2} x_i e_i\) donc \(x \in E_1+E_2\). 

  On a montré \(E \subset E_1+E_2\). Comme l'autre inclusion est établie, on a l'égalité \(E=E_1+E_2\).

\medskip
Il reste à démontrer que la somme est directe. Montrons que le vecteur nul s'écrit de manière unique sous la forme \(0=x_1+x_2\) avec \((x_1,x_2) \in E_1 \times E_2\). 

Soit \((x_1,x_2) \in E_1 \times E_2\) tel que \(0=x_1+x_2\). Il existe alors deux familles \((\alpha_i)_{i \in I_1} \in \K^{I_1}\) et \((\alpha_i)_{i \in I_1} \in \K^{I_2}\) telles que \(x_1 = \sum_{i \in I_1} \alpha_i e_i\) et  \(x_2 = \sum_{i \in I_2} \alpha_i e_i\). Comme \((I_1,I_2)\) est une partition de \(I\), leur somme vaut \(0 = \sum_{i \in I} \alpha_i e_i\). Comme \(\E\) est une base de \(E\), elle est libre et donc la famille \((\alpha_i)_{i \in I}\) est nulle. Donc \(x_1\) et \(x_2\) sont tous les deux nuls.

La somme est donc directe \(E = E_1 \oplus E_2\).
\end{proof}

\begin{theo}
  Soient \(E\) un \(\K\)-espace vectoriel de dimension finie, \(E_1\) et \(E_2\) deux sous-espaces vectoriels supplémentaires dans \(E\). Soient \(\E_1\) une base de \(E_1\) et \(\E_2\) une base de \(E_2\) alors la famille \(\E=\E_1 \cup \E_2\) est une base de \(E\), dite adaptée à la somme directe \(E=E_1 \oplus E_2\).
\end{theo}
\begin{proof}
  Notons \(\E_1=(e_i)_{i \in I_1}\), \(\E_2=(e_i)_{i \in I_2}\) avec \(I_1 \cap I_2 = \emptyset\) et \(\E=(e_i)_{i \in I}\) avec \(I = I_1 \cup I_2\).

  \(\E\) est une famille de vecteurs de \(E\). Soit \(x \in E\). Comme \(E=E_1 \oplus E_2\), il existe un unique couplet \((x_1,x_2) \in E_1 \times E_2\) tel que \(x=x_1+x_2\). \(\E_1\) est une base de \(E_1\) donc il existe une famille de scalaires \((\alpha_i)_{i \in I_1}\) telle que \(x_1 = \sum_{i \in I_1} \alpha_i e_i\). De la même manière il existe une famille de scalaires \((\alpha_i)_{i \in I_2}\) telle que \(x_1 = \sum_{i \in I_2} \alpha_i e_i\) (puisque \(I_1 \cap I_2 = \emptyset\)). Comme \((I_1,I_2)\) est une partition de \(I\), on a \(x=x_1+x_2=\sum_{i \in I_1} \alpha_i e_i+\sum_{i \in I_2} \alpha_i e_i = \sum_{i \in I} \alpha_i e_i\). Ainsi \(x \in \VectEngendre(\E)\). La famille \(\E\) est génératrice.

Soit \((\alpha_i)_{i \in I} \in \K^I\) telle que \(\sum_{i \in I} \alpha_i e_i =0\). Alors \(\sum_{i \in I_1} \alpha_i e_i = -\sum_{i \in I_2} \alpha_i e_i \in E_1 \cap E_2 = \{0\}\). Alors \(\sum_{i \in I_1} \alpha_i e_i=0\) et \(\sum_{i \in I_2} \alpha_i e_i\). Les familles \(\E_1\) et \(\E_2\) sont des bases donc elles sont libres alors \(\forall i \in I=I_1 \cup I_2 \ \alpha_i=0\). La famille \(\E\) est donc libre.

\(\E\) est libre et génératrice donc c'est une base de \(E\).
\end{proof}

\begin{corth}
  Soient \(E\) un \(\K\)-espace vectoriel de dimension finie, \(E_1\) et \(E_2\) deux sous-espaces vectoriels supplémentaires dans \(E\). Alors
  \begin{equation}
    \Dim_\K E = \Dim_\K E_1 + \Dim_\K E_2.
  \end{equation}
\end{corth}
\begin{proof}
  Soient \(\E\), \(\E_1\) et \(\E_2\) des bases respectives de \(E\), \(E_1\) et \(E_2\) alors comme on peut établir une partition \(I=(I_1,I_2)\) on a
  \begin{equation}
    \Dim_\K E = \Card \E = \Card \E_1 + \Card \E_2 = \Dim_\K E_1 + \Dim_\K E_2.
  \end{equation}
\end{proof}

\subsection{Existence de supplémentaire}

\begin{theo}
  Soit \(E\) un \(\K\)-espace vectoriel de dimension finie. Tout sous-espace vectoriel de \(E\) admet au moins un supplémentaire dans \(E\).
\end{theo}
\begin{proof}
  Soit \(E_1\) un sous-espace vectoriel de \(E\). Il est de dimension finie puisque \(E\) est de dimension finie. Soit \(\E_1\) une base de \(E_1\). La famille \(\E_1\) est une famille libre de \(E\). Le théorème de la base incomplète permet de la compléter en une base de \(E\). Alors il existe une famille de vecteur \(\E_2\) de \(E\) telle que
  \begin{equation}
    \E=\E_1 \cup \E_2 \quad \E_1 \cap \E_2 = \emptyset.
  \end{equation}
  Soit \(S= \VectEngendre(\E_2)\). D'après le théorème~
ef
ef
ef
ef
ef
ef
ef
ef
ef
ef
ef
ef
ef
ef
ef
ef
ef
ef
ef
ef
ef
ef
ef
ef
ef
ef
ef
ef
ef
ef
ef
ef\ref\ref\ref\ref\ref\ref\ref\ref\ref\ref\ref\ref\ref\ref\ref\ref\ref\ref\ref\ref\ref\ref\ref\ref\ref\ref\ref\ref\ref\ref\ref\ref\ref\ref\ref\ref\ref\ref\ref\ref\ref\ref\ref\ref\ref\ref\ref\ref\ref\ref{theo:theosuppdimfinie}, \(E_1\) et \(S\) sont supplémentaires dans \(E\)~:
  \begin{equation}
    E = E_1 \oplus S.
  \end{equation}
\end{proof}

\emph{Remarque}~: Il n'y a pas unicité du supplémentaire. On retrouve facilement, dans le cas de la dimension finie, le résultat ``Les supplémentaires d'un même espace vectoriel sont isomorphes'' vu au chapitre précédent. Si \(E\) est un \(\K\)-espace vectoriel de dimension finie et soient trois sous-espaces vectoriels tels que \(E=E_1 \oplus S_1\) et \(E=E_1 \oplus S_2\) alors \(\Dim_\K E = \Dim_\K E_1 + \Dim_\K S_1=\Dim_\K E_1 + \Dim_\K S_2\). Alors \(\Dim_\K S_1=\Dim_\K S_2\). \(S_1\) et \(S_2\) sont isomorphes.

\subsection{Formule de Grassmann}

\begin{theo}
  Soient \(E\) un \(\K\)-espace vectoriel, \(E_1\) et \(E_2\) deux sous-espaces vectoriels de dimensions finies. Alors le sous-espace vectoriel \(E_1+E_2\) est de dimension finie et
  \begin{equation}
    \Dim_\K(E_1+E_2) = \Dim_\K E_1 + \Dim_\K E_2 - \Dim_\K(E_1 \cap E_2).
  \end{equation}
\end{theo}
\begin{proof}
  \(E_1 \cap E_2\) est un sous-espace vectoriel de \(E_1\) et \(E_1\) est de dimension finie. Alors d'après la sous section précédente, il existe un sous-espace vectoriel \(S_1\) de \(E_1\) tel que \(E_1 = (E_1 \cap E_2) \oplus S_1\).

  Montrons que \(E_1+E_2 = S_1 \oplus E_2\).

  Déjà \(S_1 \subset E_1 \subset E_1+E_2\). Puis \(E_2 \subset E_1+E_2\). Alors \(S_1+E_2\) est un sous-espace vectoriel de \(E_1+E_2\) : \(S_1 +E_2 \subset E_1+E_2\).

  Soit \(x \in E_1+E_2\), il existe un couple \((x_1,x_2) \in E_1 \times E_2\) tel que \(x = x_1+x_2\). Comme \(E_1 = (E_1 \cap E_2) \oplus S_1\) il existe un couple \((a_1,b_1) \in (E_1 \cap E_2) \times S_1\) tel que \(x_1=a_1+b_1\). Alors finalement \(x=a_1+b_1+x_2\) avec \(b_1 \in S_1\), \(a_1 \in E_1 \cap E_2 \subset E_2\) et \(x_2 \in E_2\). Comme \(E_2\) est un sous-espace vectoriel on a \(a_1 + x_2 \in E_2\) et donc \(x \in S_1+E_2\). Alors \(E_1 +E_2 \subset S_1 + E_2\).

  Comme on a les deux inclusions, on a l'égalité \(E_1+E_2 = S_1+E_2\).

  Montrons que l'intersection est nulle.
  \begin{align}
    S_1 \cap E_1 &= (S_1 \cap E_1) \cap E_2 \\
    &= S_1 \cap (E_1 \cap E_2)\\
    &=\{0\},
  \end{align}
  car \(S_1\) est un supplémentaire de \((E_1 \cap E_2)\) dans \(E_1\).

  Les sous-espaces vectoriels \(E_2\) et \(S_1\) sont de dimension finies donc \(S_1 \oplus E_2\) est de dimension finie égale à
  \begin{equation}
    \Dim_\K (E_1+E_2)=\Dim_\K (S_1 \oplus E_2) = \Dim_\K S_1 + \Dim_\K E_2
  \end{equation}
  On a aussi \(E_1 = (E_1 \cap E_2) \oplus S_1\) et comme \(E_1\) est aussi de dimension finie on a
  \begin{equation}
    \Dim_\K E_1 = \Dim_\K (E_1 \cap E_2) + \Dim_\K S_1
  \end{equation}
  alors au final
  \begin{equation}
    \Dim_\K(E_1+E_2) = \Dim_\K E_1 + \Dim_\K E_2 - \Dim_\K(E_1 \cap E_2).
  \end{equation}
\end{proof}

\section{Applications linéaires en dimension finie}

\subsection{Théorème du rang}

\begin{theo}
    \label{theo:rang}
  Soient \(E\) un \(\K\)-espaces vectoriel de dimension finie, \(F\) un \(\K\)-espace vectoriel de dimension quelconque et \(u \in \Lin{E}{F}\). Alors le sous-espace vectoriel \(\Image(u)\) de \(F\) est de dimension finie et on a
  \begin{equation}
    \Dim_\K E = \Dim_\K \Ker(u) + \Dim_\K \Image(u).
  \end{equation}
\end{theo}

\emph{Le théorème du rang fait intervenir la dimension de l'espace vectoriel de départ, l'espace d'arrivée n'est pas supposé de dimension finie. Le théorème du rang ne signifie pas que l'image et le noyau de \(u\) sont supplémentaires. Ce qui n'aurait aucun sens puisqu'ils ne sont pas dans le même espace vectoriel. C'est faux même si \(E=F\).}
\begin{proof}
  \(E\) est de dimension finie, le noyau \(\Ker(u)\) est un sous-espace vectoriel de \(E\) donc il existe un supplémentaire \(S\) du noyau dans \(E\) : \(E= S \oplus \Ker(u)\).
  
  D'après le théorème de préparation au théorème du rang (Théorème~
ef
ef
ef
ef
ef
ef
ef
ef
ef
ef
ef
ef
ef
ef
ef
ef
ef
ef
ef
ef
ef
ef
ef
ef
ef
ef
ef
ef
ef
ef
ef
ef\ref\ref\ref\ref\ref\ref\ref\ref\ref\ref\ref\ref\ref\ref\ref\ref\ref\ref\ref\ref\ref\ref\ref\ref\ref\ref\ref\ref\ref\ref\ref\ref\ref\ref\ref\ref\ref\ref\ref\ref\ref\ref\ref\ref\ref\ref\ref\ref\ref\ref{theo:preptheorang})~: \(S\) est isomorphe à \(\Image(u)\). Comme \(E\) est de dimension finie, \(S\) est aussi de dimension finie comme étant un sous-espace vectoriel de \(E\). Alors \(\Image(u)\) est aussi de dimension finie et 
  \begin{equation}
    \Dim_\K \Image(u) = \Dim_\K S = \Dim_\K E - \Dim_\K \Ker(u).
  \end{equation}
\end{proof}

\subsection{Rang d'une application linéaire}

\begin{defdef}
  Soient \(E\) et \(F\) deux \(\K\)-espace vectoriels et \(u \in \Lin{E}{F}\). On dit que \(u\) est de rang fini lorsque \(\Image(u)\) est un sous-espace vectoriel de dimension finie, auquel cas le rang est défini par
  \begin{equation}
    \rg(u) = \Dim_\K \Image(u).
  \end{equation}
  Si \(\Image(u)\) est de dimension infinie, on dit que le rang de \(u\) est infini.
\end{defdef}

\begin{theo}
  Soient \(E\) et \(F\) deux \(\K\)-espace vectoriels et \(u \in \Lin{E}{F}\). On suppose que \(E\) est de dimension finie. Alors \(u\) est de rang fini et
  \begin{equation}
    \rg(u) \leqslant \Dim_\K E.
  \end{equation}
De plus \(\rg(u) = \Dim_\K E\) si et seulement si \(u\) est injective.
\end{theo}
\begin{proof}
  Les hypothèses du théorème du rang sont satisfaites donc \(\rg(u)\) est fini et
  \begin{equation}
    \Dim_\K E = \Dim_\K \Ker(u) + \rg(u),
  \end{equation}
  alors l'inégalité est vraie~: \(\rg(u) \leqslant \Dim_\K E\) et
  \begin{align}
    \rg(u) = \Dim_\K E &\iff \Dim_\K \Ker(u) = 0\\
    &\iff \Ker(u)=\{0\}.
  \end{align}
  Ce qui est équivalent à \(u\) est injective.
\end{proof}

\begin{theo}
  Soient \(E\) et \(F\) deux \(\K\)-espace vectoriels et \(u \in \Lin{E}{F}\). On suppose que \(F\) est de dimension finie. Alors \(u\) est de rang fini et
\begin{equation}
    \rg(u) \leqslant \Dim_\K F.
  \end{equation}
De plus \(\rg(u) = \Dim_\K F\) si et seulement si \(u\) est surjective.
\end{theo}
\begin{proof}
  \(\Image(u)\) est un sous-espace vectoriel de \(F\), or \(F\) est de dimension finie donc \(\Image(u)\) aussi. Alors
  \begin{equation}
    \rg(u) = \Dim_\K \Image(u) \leqslant \Dim_\K F.
  \end{equation}
  De plus \(\rg(u) = \Dim_\K F\) si et seulement si \(\Image(u)=F\) c'est à dire si et seulement \(u\) est surjective.
\end{proof}

\begin{prop}
  Soient deux \(\K\)-espaces vectoriels \(E\) et \(F\) et \(u \in \Lin{E}{F}\). On suppose qu'il existe une famille génératrice finie \(\G\) de \(E\) (donc \(E\) est de dimension finie). Alors
  \begin{align}
    \Image(u) &= \VectEngendre(u(\G)) \\
    \rg(u) &= \rg(u(\G)).
  \end{align}
\end{prop}
\begin{proof}
  \(E\) est un espace vectoriel de dimension finie donc \(\rg(u)\) est fini. La première égalité est une conséquence de la proposition~
ef
ef
ef
ef
ef
ef
ef
ef
ef
ef
ef
ef
ef
ef
ef
ef
ef
ef
ef
ef
ef
ef
ef
ef
ef
ef
ef
ef
ef
ef
ef
ef\ref\ref\ref\ref\ref\ref\ref\ref\ref\ref\ref\ref\ref\ref\ref\ref\ref\ref\ref\ref\ref\ref\ref\ref\ref\ref\ref\ref\ref\ref\ref\ref\ref\ref\ref\ref\ref\ref\ref\ref\ref\ref\ref\ref\ref\ref\ref\ref\ref\ref{prop:propdeduitesu(x)}. La deuxième égalité découle de la définition du rang d'une application linéaire et du rang d'une famille de vecteurs~:
  \begin{equation}
    \rg(u) = \Dim_\K \Image(u) \qquad \rg(u(\G))=\Dim_\K \VectEngendre(u(\G)),
  \end{equation}
  et comme les deux espaces vectoriels sont égaux, on a l'égalité \(\rg(u) = \rg(u(\G))\).
\end{proof}

\begin{prop}
  Soient deux \(\K\)-espaces vectoriels \(E\) et \(F\) et \(u \in \Lin{E}{F}\). Soit \(\X\) une famille de vecteurs de \(E\) supposée de rang fini. Alors la famille \(u(\X)\) est de rang fini et
  \begin{equation}
    \rg(\X) = \rg(u(\X)) + \Dim_\K (\Ker(u) \cap \VectEngendre(\X)).
  \end{equation}
\end{prop}
\begin{proof}
  Soit \(E_1=\VectEngendre(\X)\). C'est un \(\K\)-espace vectoriel de dimension finie et \(\Dim_\K E_1 = \rg(\X)\). Soit \(u_1=u_{|E_1} \in \Lin{E_1}{F}\). Comme \(E_1\) est de dimension finie, on peut appliquer le théorème du rang à \(u_1\) et
  \begin{equation}
    \Dim_\K E_1 = \Dim_\K \Ker(u) + \rg(u_1),
  \end{equation}
  avec~:
  \begin{itemize}
  \item  \(\Dim_\K E_1 = \rg(\X)\);
  \item Comme \(u_1 = u_{|E_1}\) on a \(\Ker(u_1)=\Ker(u) \cap E_1\) et alors
    \begin{equation}
      \Dim_\K \Ker(u_1) =  \Dim_\K(E_1 \cap \Ker(u)) = \Dim_\K(\Ker(u) \cap \VectEngendre(\X))
    \end{equation}
  \item Par définition du rang on a
    \begin{align}
      \rg(u_1)& = \Dim_\K \Image(u_1) \\
      &= \Dim_\K u(E_1) \\
      &= \Dim_\K u(\VectEngendre(\X))\\
      &= \Dim_\K \VectEngendre(u(\X)) \quad u \in \Lin{E}{F} \\
      &=\rg u(\X)
    \end{align}
  \end{itemize}
  Alors finalement en remplaçant tous les membres dans la formule du théorème du rang
  \begin{equation}
    \rg(\X) = \rg(u(\X)) + \Dim_\K (\Ker(u) \cap \VectEngendre(\X)).
  \end{equation}
\end{proof}

\begin{theo}
  La composition par un isomorphisme ne change pas le rang. Soient \(E\), \(F\) et \(G\) trois \(\K\)-espaces vectoriel et \(u \in \Lin{E}{F}\) et \(v\in \Lin{F}{G}\). On suppose que \(u\) et \(v\) sont de rang finis. Alors
  \begin{align}
    u \in \Isom{E}{F} &\implies \rg(v \circ u)=\rg(v) \\
    v \in \Isom{F}{G} &\implies \rg(v \circ u)=\rg(u)
  \end{align}
\end{theo}
\begin{proof}
  \(\Image(v \circ u) \subset \Image(v)\) et \(\Image(v \circ u)\) est de dimension finie (car \(v\) est de rang fini).
  \begin{align}
    \Image(v \circ u) &= \{y \in \G, \exists x \in E \ y=v \circ u(x) \}\\
    &=\{y \in \G, \exists z \in \Image(u) \ y=v(z)\}\\
    &=v(\Image(u))
  \end{align}

  Supposons que \(u \in \Isom{E}{F}\), alors \(\Image(u) = F\) (puisque \(u\) est bijective). L'application \(u\) est aussi de rang fini, donc \(F\) est de dimension finie et donc \(E\) est aussi de dimension finie (puisque \(\Dim_\K E = \Dim_\K F\)). Alors
  \begin{align}
    \Image(v \circ u) &= v(\Image(u)) \\
    &=v(F) \\
    &=\Image(v) \quad \text{~par définition}
  \end{align}
  Alors en passant aux dimensions dans l'égalité \(\rg(v \circ u) = \rg(v)\).

  Supposons maintenant que ce soit \(v \in \Isom{F}{G}\), alors \(\Image(u) = G\) (puisque \(v\) est bijective).  L'application \(v\) est aussi de rang fini, donc \(G\) est de dimension finie et donc \(F\) est aussi de dimension finie (puisque \(\Dim_\K F = \Dim_\K G\)). L'application \(v\) est bijective, donc sa restriction \(v_{|\Image(u)}^{|v(\Image(u))}\) est aussi bijective. Alors \(\Image(u)\) et \(v(\Image(u))\) ont la même dimension alors \(\rg(v \circ u) =\Dim_\K v(\Image(u))= \Dim_\K \Image(u) = \rg(u)\)
\end{proof}

\subsection[Caractérisation des applications linéaires bijectives]{Caractérisation des applications linéaires bijectives entre deux espaces vectoriels de même dimension finie}

\begin{theo}
  Soient \(E\) et \(F\) deux \(\K\)-espaces vectoriels \emph{de même dimension finie} et \(u \in \Lin{E}{F}\). Alors les assertions suivantes sont équivalentes~:
  \begin{enumerate}
  \item \(u\) est bijective;
  \item \(u\) est injective;
  \item \(u\) est surjective;
  \item \(\rg(u) = \Dim_\K E = \Dim_\K F\);
  \item il existe \(v \in \Lin{E}{F}\) telle que \(v \circ u = \Id\);
  \item il existe \(v \in \Lin{F}{E}\) telle que \(u \circ v = \Id\).
  \end{enumerate}
\end{theo}
\begin{proof}
  \(1 \implies 2\) : C'est évident

  \(2 \implies 3\) : \(E\) est de dimension finie et en appliquant le théorème du rang à \(u\) on obtient
  \begin{equation}
    \rg(u) + \Dim \Ker(u) = \Dim E
  \end{equation}
  Or \(u\) est injective donc \(\Ker(u)=\{0\}\) et sa dimension est nulle donc \(\rg(u) = \Dim F\). Ce qui est équivalent (par théorème) à \(u\) est surjective.

  \(3 \implies 4\) : Comme \(\Dim E = \Dim F\) alors si \(u\) est surjective alors \(\rg(u) = \Dim E\).

  \(4 \implies 1\) : Déjà \(\rg(u) = \Dim E = \Dim F\) alors \(u\) est surjective. De plus par théorème du rang \(\Dim E = \Dim \Ker(u) + \rg(u)\). Donc \(\Dim \Ker(u)=0\) ce qui implique que \(u\) est injective. Au final \(u\) est bijective.

  On a montré que les quatre premières assertions sont équivalentes. 

 Ensuite La première implique la cinquième et la sixième par caractérisation des bijections. 

 \(5 \implies 2\) puisque \(v \circ u = \Id \) implique que \(v \circ u\) est injective ce qui implique que \(u\) est injective. 

 Enfin \(6 \implies 3\) puisque \(u \circ v=\Id\) implique que \(u\circ v\) est surjective ce qui implique que \(u\) est surjective.
\end{proof}

\begin{corth}
  Soient un \(\K\)-espace vectoriel \(E\) de dimension finie et \(u \in \Endo{E}\). Alors
  \begin{equation}
    u \text{~bijective} \iff u \text{~injective} \iff u \text{~surjective}.
  \end{equation}
C'est faux en dimension infinie.
\end{corth}

\subsection{Formes linéaires et hyperplans}

Soit un \(\K\)-espace vectoriel \(E\) quelconque. On ne suppose pas pour l'instant qu'il est de dimension finie.

\subsubsection{Rang d'une forme linéaire}

\begin{prop}
  Soit \(f \in E^*\) alors~:
  \begin{itemize}
  \item soit \(f\) est la forme linéaire nulle et elle est de rang nul;
  \item soit \(f\) est non nulle et \(\rg(f)=1\).
  \end{itemize}
\end{prop}
\begin{proof}
  On avait déjà vu qu'une forme linéaire non nulle est surjective. On peut aussi utiliser la dimension : \(\Image(f)\) est un sous-espace vectoriel de \(\K\) donc \(\Image(f) = \{0\}\) ou \(\Image(f) = \K\) alors en passant à la dimension on a le résultat voulu.
\end{proof}

\subsubsection{Hyperplan vectoriel}

\begin{prop}
  Soit \(H\) un sous-espace vectoriel de \(E\). Il y a équivalence entre~:
  \begin{enumerate}
  \item \(H\) est le noyau d'une forme linéaire non nulle;
  \item Il existe une droite vectorielle qui est le supplémentaire de \(H\) dans \(E\)
    \begin{equation}
      \exists a \in E\setminus\{0\} \quad H \oplus \VectEngendre(a) = E
    \end{equation}
  \end{enumerate}
  Si l'une des deux assertions est vérifiée, on dit que \(H\) est un hyperplan vectoriel de \(E\).
\end{prop}
\begin{proof}
  \(1 \implies 2\) : il existe une forme linéaire \(f\) non nulle telle que \(H = \Ker(f)\). Comme \(f\) est non nulle il existe un vecteur \(a\) de \(E\) tel que \(f(a) \neq 0\) et comme \(f\) est linéaire \(a \neq 0\). Montrons que \(E= H \oplus \VectEngendre(a)\).

  \(H\) et \(\VectEngendre(a)\) sont tous les deux des sous-espaces vectoriels de \(E\) donc \(H + \VectEngendre(a) \subset E\).

  Soit \(x \in H \cap \VectEngendre(a)\) alors
  \begin{align}
    x \in H &\iff x \in \Ker(f) \\
    &\iff f(x)=0
  \end{align}
  \begin{align}
    x \in \VectEngendre(a) &\iff \exists \lambda \in \K \ x=\lambda a
  \end{align}
  Donc \(0=f(x)=f(\lambda a)=\lambda f(a)\). Comme \(f(a) \neq 0\) et que \(\K\) est intègre on a \(\lambda=0\), et donc \(x=0\).

  On a montré \(H \cap \VectEngendre(a) \subset \{0\}\). Comme ce sont des sous-espaces vectoriel, leur intersection est un sous-espace vectoriel et on a l'autre inclusion donc au final l'égalité : \(H \cap \VectEngendre(a) = \{0\}\).

  \emph{Analyse}~: Soit \(x \in E\). Supposons qu'il existe \(x_H \in H\) et \(\lambda \in \K\) tels que \(x = x_H + \lambda a\). Alors
  \begin{align}
    f(x) &= f(x_H) +\lambda f(a) \\
    &=\lambda f(a)
  \end{align}
  comme \(f(a) \neq 0\), son inverse existe et donc \(\lambda = f(x) f(a)^{-1}\) (unique) et donc \(x_h = x-f(x) f(a)^{-1}\) (unique).

  \emph{Synthèse}~: Soit \(x \in E\), \(x = (x- f(x) f(a)^{-1} a) + f(x) f(a)^{-1} a\). De plus
  \begin{align}
    f(x- f(x) f(a)^{-1} a)  &= f(x) -f(x)f(a)f(a)^{-1} = 0
  \end{align}
  donc \(x- f(x) f(a)^{-1} a \in \Ker(f) = H\). Ainsi \(E \subset H + \VectEngendre(a)\).

  Finalement la somme est directe et ils sont supplémentaires : \(E = H \oplus \VectEngendre(a)\).

  \(2 \implies 1\) : Soit \(a \in E \setminus\{0\}\) tel que \(E = H \oplus \VectEngendre(a)\). Soit \(f \in E^*\) définie par ses restrictions sur \(H\) et sur \(\VectEngendre(a)\) :
  \begin{itemize}
  \item \(\forall x \in H  \quad f(x)=0\);
  \item \(\forall x \in \VectEngendre(a) \ \exists! \lambda \in \K \quad x=\lambda a \ f(x)=\lambda\).
  \end{itemize}
  Sur \(\VectEngendre(a)\), on associe à \(x\) sa coordonnée dans la base \((a)\) et c'est bien linéaire.

  De plus \(f\) est non nulle (\(f(a)=1\)) et pour tout pour tout \(x \in E\) il existe un unique couple \((x_H, \lambda) \in H \times K\) tel que \(x=x_H+\lambda a\). Alors
  \begin{align}
    x \in \Ker(f) & \iff f(x) = f(x_H) +\lambda f(a) =0\\
    &\iff \lambda f(a) =0\\
    &\iff \lambda =0\\
    &\iff x=x_H\\
    &\iff x \in H
  \end{align}
  donc \(\Ker(f) =H\).
\end{proof}

Soient \(H\) un hyperplan vectoriel de \(E\), \(f \in E^*\setminus\{0\}\) telle que \(H = \Ker(f)\) et \(a \in E\setminus H\) tel que \(E = H \oplus \VectEngendre(a)\).

\begin{prop}
  Pour tout \(D'\) sous-espace vectoriel de \(E\), \(D'\) est un supplémentaire de \(H\) dans \(E\) si et seulement s'il existe \(a' \in E \setminus H\) tel que \(D'=\VectEngendre(a')\).
\end{prop}
\begin{proof}
  S'il existe \(a' \in E \setminus H\) tel que \(D'=\VectEngendre(a')\), on a déjà montrer qu'alors \(D'\) est un supplémentaire de \(H\) dans \(E\) (démontré au \(1 \implies 2\) de la proposition précédente et on avait choisi \(a\) quelconque dans \(E \setminus H\)).

  Si \(D'\) est un supplémentaire de \(H\) dans \(E\) alors on a
  \begin{equation}
    E=D' \oplus H = \VectEngendre(a) \oplus H
  \end{equation}
  alors \(D'\) et \(\VectEngendre(a)\) sont isomorphes et donc \(D'\) est de dimension finie égale à \(1\). Il existe alors un vecteur non nul \(a'\) de \(E\) tel que \(D'=\VectEngendre(a')\). De plus \(a' \notin H\) car sinon \(D' \cap H \supset \{a'\} \neq \{0\}\) ce qui est impossible.
\end{proof}

\begin{prop}
  Pour toute forme linéaire \(g \in E^*\),
  \begin{equation}
    H = \Ker(g) \iff \exists \lambda \in \K\setminus\{0\} \ g=\lambda f.
  \end{equation}
\end{prop}
\begin{proof}
  \(\impliedby\) : Pour tout \(x \in E\), on a
  \begin{align}
    x \in \Ker(g) &\iff (\lambda f)(x) = 0\\
    &\iff \lambda f(x) =0\\
    &\iff f(x) =0 && (\lambda \neq 0)\\
    &\iff x \in \Ker(f)
  \end{align}
  Alors \(\Ker(f)=H=\Ker(g)\).

  \(\implies\) : Si \(H=\Ker(g)\) comme \(a \notin H\) on pose \(\lambda = g(a) f(a)^{-1}\). Pour tout \(x \in E\) il existe un unique couple \((x_H, \lambda) \in H \times \K\) tel que \(x=x_H +\mu a\). On calcule \(f(x)\) et \(g(x)\)

  \begin{align}
    f(x) &= f(x_H)+\mu f(a) = \mu f(a) \\
    g(x) &= g(x_H) +\mu g(a) = \mu (\lambda f(a)) = \lambda (\mu f(a))=\lambda f(x)
  \end{align}
  Alors comme c'est vrai pour tout \(x \in E\), on a \(g=\lambda f\). Le scalaire \(\lambda\) est non nul sinon on aurait \(g\) constante nulle et on aurait aussi \(H = \Ker(g)=E\). Ainsi \(H \neq E\) car \(H\) est un hyperplan.
\end{proof}

\emph{Équations linéaires d'un hyperplan}

Soit \(H\) un hyperplan. Soit \(f \in E^* \setminus\{0\}\) et \(H=\Ker(f)\). Pour tout \(x \in E\), \(x \in H\) si et seulement si \(f(x)=0\). L'écriture ``\(f(x)=0\)'' est une équation linéaire de l'hyperplan \(H\). Les autres équations de \(H\) sont \(\lambda f(x)=0\) avec \(\lambda\) un scalaire non nul.

\emph{Hyperplan affine}

Soit un scalaire \(b\) et \(\mathcal{H}=\{x \in E, f(x)=b\}\). La forme linéaire \(f\) est non nulle, alors elle est surjective, donc il existe \(x_0 \in E\) tel que \(f(x_0)=b\). On sait que \(\mathcal{H}=x_0 + H\). \(\mathcal{H}\) est un hyperplan affine passant par \(x_0\) et dirigé par l'hyperplan vectoriel \(H\).

\subsubsection{Hyperplans en dimension finie}

Soit \(E\) un \(\K\)-espace vectoriel de dimension finie \(n \in \N^*\). Dans ce cadre on avait définit les hyperplans vectoriels par~:
\begin{enumerate}
\item \(H\) est un hyperplan de \(E\) si et seulement si c'est le noyau d'une forme linéaire non nulle;
\item \(H\) est un hyperplan si et seulement si c'est le supplémentaire d'une droite vectorielle dans \(E\);
\item \(H\) est un hyperplan si et seulement si sa dimension vaut \(n-1\).
\end{enumerate}

Vérifions que ces définitions sont équivalentes.

\(1 \implies 3\) : \(H=\Ker(f)\) avec \(f \in E^* \setminus\{0\}\). \(f\) est non nulle donc son rang vaut \(1\) et \(E\) est de dimension finie alors en appliquant le théorème du rang on a
\begin{align}
  \Dim E &= \Dim \Ker(f) +\rg(f) \\
       n &= \Dim H +1
\end{align}
alors \(\Dim H = n-1\).

\(3 \implies 1\) : Il existe \((e_1, \ldots, e_{n-1})\) une base de \(H\). On complète cette famille libre de \(E\) en posant une base \((e_1, \ldots, e_{n})\) de \(E\). On définit \(f\in E^*\) par son action sur cette base~:
\begin{equation}
  \begin{cases}
    i \in \llbracket 1, n-1 \rrbracket & f(e_i)=0 \\
    i=n & f(e_n)=1
  \end{cases}
\end{equation}
alors \(f\) est non nulle et \(\Ker(f)=H\) (\(\forall x \in E\) \(\exists! (x_1, \ldots, x_n) \in \K^n\) \(x = \sum_{i=1}^n x_i e_i\), alors \(f(x)=0\) \(\iff\) \(\sum_{i=1}^n x_i f(e_i)=0\) \(\iff\) \(x_n=0\) \(\iff\) \(x \in H\)).

On peut également montrer que \(2 \iff 3\), mais on n'y est pas forcé (\(1 \iff 2\) et \(1 \iff 3\)).

\(2 \implies 3\) : Il existe un vecteur \(a\) de \(E\) non nul tel que \(E=H \oplus \VectEngendre(a)\). Comme \(E\) est de dimension finie on a
\begin{align}
  \Dim E &= \Dim H +\Dim \VectEngendre(a)\\
  n &= \Dim H +1
\end{align}
soit alors \(\Dim H = n-1\).

\(3 \implies 2\) : On travaille en dimension finie donc \(H\) admet un supplémentaire \(S\) et \(\Dim H = \Dim E -1\) donc \(\Dim S=1\) et alors \(S\) est une droite.

\emph{Équations d'un hyperplan dans une base}

Soit \(\E=(e_1, \ldots, e_n)\) une base de \(E\). Soit \(H\) un hyperplan vectoriel de \(E\) et \(f \in E^*\setminus\{0\}\) telle que \(H=\Ker(f)\). Pour tout vecteur \(x \in E\) il existe un unique \(n\)-uplet de scalaires \((x_1,\ldots, x_n)\) tel que \(x = \sum_{i=1}^n x_i e_i\). Alors

\begin{align}
  x \in H &\iff f(x)=0\\
  &\iff \sum_{i=1}^n x_i f(e_i)=0
\end{align}
Si on note pour tout \(i \in \llbracket 1,n \rrbracket\) \(a_i=f(e_i) \in K\). Comme \(f\) est non nulle il existe un naturel \(i_0 \in \llbracket 1,n \rrbracket\) tel que \(a_{i_0} \neq 0\) et donc
\begin{equation}
  x \in H \iff
  \begin{cases}
    \sum_{i=1}^n a_i x_i=0 \\ \exists i_0 \in \intervalleentier{1}{n} \quad a_{i_0} \neq 0
  \end{cases}
\end{equation}
Alors \(\sum_{i=1}^n a_i x_i=0\) est une équation de l'hyperplan \(H\) dans la base \(\E\). Les autres équations de \(H\) dans cette base sont \(\lambda \left(\sum_{i=1}^n a_i x_i\right)=0\) avec \(\lambda\) un scalaire non nul.

Réciproquement, soit \((b_1, \ldots, b_n) \in \K^n\) tel qu'il existe \(i_0 \in \intervalleentier{1}{n}\) tel que \(b_{i_0} \neq 0\). Soit
\begin{equation}
  K = \enstq{x = \sum_{i=1}^n x_i e_i}{\sum_{i=1}^n b_i x_i =0},
\end{equation}
alors \(K\) est un hyperplan vectoriel.
\begin{proof}
  Soit \(f \in E^*\) définie par : \(\forall i \in \intervalleentier{1}{n} \quad f(e_i)=b_i\). La forme linéaire \(f\) est non nulle puisqu'il existe \(i_0 \in \intervalleentier{1}{n}\) tel que \(b_{i_0} \neq 0\). Soit \(x \in E\), alors
  \begin{align}
    x \in K &\iff \sum_{i=1}^n x_i f(e_i)=0 \\
    &\iff f\left(\sum_{i=1}^n x_i e_i \right) =0\\
    &\iff f(x)=0\\
    &\iff x \in \Ker(f).
  \end{align}
Donc \(K=\Ker(f)\) et c'est un hyperplan.
\end{proof}

\section{Suites récurrentes linéaires d'ordre 2}

Soient \(\K=\R\) ou \(\C\), \((a,b,c) \in \K^3\) tel que \(ac \neq 0\) et \(d \in \K^\N\). On se propose de résoudre dans \(\K^\N\) l'équation linéaire de récurrence d'ordre 2 à coefficients constants :
\begin{equation}
  au_{n+2}+bu_{n+1}+cu_n = d_n \label{eq:S} \tag{\(\Lb\)},
\end{equation}
c'est-à-dire de déterminer l'ensemble \(\S_\K(\Lb)\) des suites \((u_n)_{n \in \N} \in \K^N\) telles que
\begin{equation}
  \forall n \in \N \quad au_{n+2}+bu_{n+1}+cu_n = d_n.
\end{equation}
Dans le cas où \(d=0\), l'équation sera dite homogène et noté \eqref{eq:H}~:
\begin{equation}
  au_{n+2}+bu_{n+1}+cu_n = 0 \label{eq:H} \tag{\(\H\)}
\end{equation}

\subsection{Résolution de l'équation homogène}

\begin{prop}
  L'ensemble \(\S_\K(\H)\) des solutions de l'équation homogène est un sous-espace vectoriel du \(\K\)-espace vectoriel \(\K^\N\).
\end{prop}
\begin{proof}
  \(\S_\K(\H)\) est une partie de \(\K^\N\) non vide, puisqu'elle contient la suite nulle et stable par combinaison linéaire~:
  soient \((u,v) \in \S_\K(\H)^2\) et \(\lambda \in \K\) alors \(\lambda u +v \in \K^\N\) et pour tout entier \(n\)
  \begin{equation}
    a(\lambda u+v)_{n+2} +b(\lambda u+v)_{n+1} +c(\lambda u+v)_n = \lambda (au_{n+2}+bu_{n+1}+cu_n) + (au_{n+2}+bu_{n+1}+cu_n)=0
  \end{equation}
  La suite \(\lambda u+v\) est bien dans \(\S_\K(\H)\). Par théorème de caractérisation des sous-espaces vectoriel, \(\S_\K(\H)\) est un sous-espace vectoriel de \(\K^\N\).
\end{proof}

\begin{prop}
  Le sous-espace vectoriel \(\S_\K(\H)\) est en fait un plan vectoriel, c'est-à-dire qu'il est de dimension 2.
\end{prop}
\begin{proof}
  Soit l'application \(f\) de \(\S_\K(\H)\) sur \(\K^2\) définie par \(f(u)=(u_0,u_1)\). Prouvons que \(f\) est un isomorphisme de \(\K\)-espaces vectoriels de \(\S_\K(\H)\) sur \(\K^2\).
  \begin{itemize}
  \item La fonction \(f\) est bijective puisque pour tout couple \((\alpha, \beta) \in \K^2\), il existe une seule suite \(u \in \S_\K(\H)\) telle que \(f(u)=(u_0,u_1)=(\alpha,\beta)\). En effet, on prouve par récurrence sur \(n\) que le terme \(u_n\) est défini sans ambiguïté par la donnée de \(u_0\), \(u_1\) et de la relation de récurrence \(au_{n+2}+bu_{n+1}+cu_n=0\) avec \(\alpha \neq 0\).
  \item La fonction \(f\) est linéaire car, pour tout \((u,v) \in \S_\K(\H)^2\) et tout \(\lambda \in \K\),
    \begin{align}
      f(\lambda u+v) &=(\lambda u_0 +v_0, \lambda u_1+v_1) \\
      &=\lambda (u_0,u_1) +(v_0,v_1)\\
      &=\lambda f(u)+f(v)
    \end{align}
  \end{itemize}
On a démontré que \(f \in \Isom{\S_\K(\H)}{\K^2}\). Le \(\K\)-espace vectoriel \(\S_\K(\H)\) est donc, comme le \(\K\)-espace vectoriel \(\K^2\), de dimension 2.
\end{proof}

On cherche alors une base de \(\S_\K(\H)\). Par analogie avec les équations de récurrences d'ordre 1 (de la forme \(au_{n+1}+bu_n =0\) avec \(ab \neq 0\) qui peuvent se mettre sous la forme \(u_{n+1}=qu_n\)), on cherche les suites géométriques, de premier terme égal à 1, qui sont des éléments de \(\S_\K(\H)\).

\begin{prop}
  Pour tout élément \(r\) pris dans \(\K\), la suite géométrique \((r^n)_{n \in \N}\) est une solution de \eqref{eq:H} si, et seulement si, \(r\) est racine de l'équation de degré 2 suivante, dite équation caractéristique de \eqref{eq:H}.
  \begin{equation}
    ar^2+br+c=0 \label{eq:EC} \tag{E.C.}
  \end{equation}
\end{prop}
\begin{proof}
  \begin{align}
    (r^n)_{n \in  \N} \in \S_\K(\H) &\iff \forall n \in \N \ ar^{n+2} +br^{n+1}+cr^n = 0 \\
    &\iff \forall n \in \N \ r^n(ar^2+br+c)=0\\
    &\iff ar^2+br+c=0 \text{~car } r^0=1 \neq 0
  \end{align}
\end{proof}

On note \(\Delta=b^2-4ac\) le discriminant de l'équation caractéristique

\subsubsection{Premier cas : l'équation caractéristique admet deux racines distinctes sur le corps \(\K\)}

Soit \(\K=\C\) et \(\Delta \neq 0\), soit \(\K=\R\) avec \(\Delta >0\).

Notons \(r_1\) et \(r_2\) avec \(r_1 \neq r_2\), les deux racines distinctes de \eqref{eq:EC} sur \(\K\). Alors \((r_1^n)_{n \in \N}\) et \((r_2^n)_{n \in \N}\) sont deux éléments de \(\S_\K(\H)\). Démontrons qu'ils forment une famille libre de \(\S_\K(\H)\). Pour tout \((\lambda, \mu) \in \K^2\),
\begin{align}
  \lambda (r_1^n)_{n \in \N} + \mu (r_2^n)_{n \in \N} = 0_{\K^\N} &\implies \forall n \in \N \ \lambda r_1^n + \mu r_2^n=0\\
  &\implies \begin{cases} \lambda + \mu =0 & (n=0) \\ \lambda r_1 +\mu r_2 = 0 & (n=1) \end{cases}\\
  &\implies \begin{cases} \lambda = -\mu \\ \lambda(r_1-r_2)=0 \end{cases}\\
  &\implies \lambda=\mu=0 \text{~car } r_1-r_2 \neq 0
\end{align}
Comme la dimension de \(\S_\K(\H)\) est de deux, on en déduit que \(((r_1^n)_{n \in \N},(r_2^n)_{n \in \N})\) est une base de \(\S_\K(\H)\).

\begin{theo}
  Si l'équation caractéristique de \eqref{eq:H} admet deux racines distinctes sur \(\K\), \(r_1\) et \(r_2\), alors le plan vectoriel \(\S_\K(\H)\) admet \(((r_1^n)_{n \in \N},(r_2^n)_{n \in \N})\) comme base. Autrement dit, pour toute suite \(u \in \K^\N\),
  \begin{equation}
    u \in \S_\K(\H) \iff \exists! (\lambda,\mu) \in \K^2 \ \forall n \in \N \quad u_n = \lambda r_1^n +\mu r_2^n
  \end{equation}
\end{theo}

La donnée des conditions initiales \(u_0=\alpha\) et \(u_1=\beta\) permet de trouver l'unique suite \(u\) solution de \eqref{eq:H} vérifiant ces conditions. Le couple \((\lambda, \mu)\) est l'unique solution du système de Cramer \(\begin{cases} \lambda + \mu =\alpha \\ \lambda r_1+\mu r_2 =\beta \end{cases}\).

\subsubsection{Deuxième cas : l'équation caractéristique admet une racine double dans \(\K\)}

C'est le cas \(\K=\R\) ou \(\C\) avec \(\Delta=0\). Notons \(r_0\) la racine double de \eqref{eq:EC} dans \(\K\) : \(r_0 = -b/2a\). Alors \((r_0^n)_{n \in \N}\) est un élément de \(\S_\K(\H)\). Il faut en trouver un second non colinéaire au premier pour obtenir une base de \(\S_\K(\H)\).

Vérifions que c'est le cas de la suite \((nr_0^n)_{n \in \N}\).

Pour tout naturel \(n\),
\begin{align}
  a(n+2)r_0^{n+2} +b(n+1)r_0^{n+1} +cnr_0^n &= nr_0^n(ar_0^2+br_0+c) +r_0^{n+1}(2ar_0+b)\\
  &= nr_0^n \times 0 + r_0^{n+1} \times 0 =0.
\end{align}
La suite \((nr_0^n)_{n \in \N}\) est bien un élément de \(\S_\K(\H)\).

La famille \(((r_0^n)_{n \in \N},(nr_0^n)_{n \in \N})\) est une famille libre de \(\S_\K(\H)\). En effet, pour tout couple de scalaires \((\lambda,\mu)\),
\begin{align}
  \lambda (r_0^n)_{n \in \N} + \mu (nr_0^n)_{n \in \N}) = 0_{\K^\N} &\implies \forall n \in \N \ \lambda r_0^n +\mu n r_0^n = 0\\
  &\implies \begin{cases} \lambda =0 & (n=0) \\ \lambda r_0 +\mu r_0 =0 & (n=1) \end{cases}\\
  &\implies \lambda=\mu=0 \ \text{~car } r_0 \neq 0.
\end{align}
En effet, puisque \(\Delta = b^2-4ac=0\) alors \(b^2 = 4ac \neq 0\). Comme \(\dim_\K(\S_\K(\H))=2\), alors \(((r_0^n)_{n \in \N},(nr_0^n)_{n \in \N})\) est une base de \(\S_\K(\H)\).

\begin{theo}
  Si l'équation caractéristique de \eqref{eq:H} admet une racine double \(r_0\) dans \(K\) alors le plan vectoriel \(\S_\K(\H)\) admet \(((r_0^n)_{n \in \N},(nr_0^n)_{n \in \N})\) pour base. Autrement dit pour toute suite \(u \in \K^\N\),
  \begin{equation}
    u \in \S_\K(\H) \iff \exists! (\lambda,\mu) \in \K^2 \ \forall n \in \N \quad u_n = \lambda r_0^n +\mu nr_0^n.
  \end{equation}
\end{theo}
La donnée des conditions initiales \(u_0=\alpha\) et \(u_1=\beta\) permet de trouver l'unique suite \(u\) solution de \eqref{eq:H} vérifiant ces conditions. Le couple \((\lambda, \mu)\) est l'unique solution du système de Cramer \(\begin{cases} \lambda =\alpha \\ (\lambda +\mu) r_2 =\beta \end{cases}\).

\subsubsection{Troisième cas : l'équation caractéristique n'admet pas de racine dans \(\K\)}

Il s'agit du cas \(\K=\R\) et \(\Delta <0\). Il n'y a donc pas ici de suite géométrique \((r^n)_{n \in \N}\) avec \(r\) réel qui soit solution de \eqref{eq:H}. Il faut trouver un autre moyen de déterminer une base de \(\S_\R(\H)\). Par contre, la résolution de \eqref{eq:H} avec \(\K=\C\), c'est à dire dans l'ensemble \(\C^\N\) des suites complexes, conduirait à l'ensemble \(\S_\C(\H)\) qui relève du premier cas.

Notons \(r_1\) et \(r_2\) avec \(r_1 \neq r_2\), les racines complexes distinctes de \eqref{eq:EC}. Comme \(a\), \(b\) et \(c\) sont des réels et \(\Delta <0\), on sait que \(r_1\) et \(r_2\) sont des complexes non réels et conjugués. Notons \(\rho\) le module de \(r_1\) et \(\theta\) un argument de \(r_1\), alors \(r_1 = \rho\e^{\ii \theta}\) et \(r_2 = \rho\e^{-\ii \theta}\). Comme ils ne sont pas réels \(\sin \theta \neq 0\).

On a vu que les suites \((r_1^n)_{n \in \N}\) et \((r_2^n)_{n \in \N}\) sont dans \(\S_\C(\H)\). Comme \(\S_\C(\H)\) est un espace vectoriel, la suite \(\frac{1}{2}((r_1^n)_{n \in \N} + (r_2^n)_{n \in \N})\) est également dans \(\S_\C(\H)\). Or, pour tout naturel \(n\),
\begin{align}
  \frac{1}{2} r_1^n + \frac{1}{2} r_2^n & =\frac{1}{2} \rho^n \e^{\ii n \theta}+ \frac{1}{2} \rho^n \e^{-\ii n \theta}\\
  &=\rho^n \cos(n \theta).
\end{align}
Cette suite est donc à valeurs réelles et par conséquent c'est un élément de \(\S_\R(\H)\). De même, la suite \(\frac{1}{2\ii}((r_1^n)_{n \in \N} - (r_2^n)_{n \in \N})\) est également dans \(\S_\C(\H)\). Or, pour tout naturel \(n\),
\begin{align}
  \frac{1}{2\ii} r_1^n - \frac{1}{2\ii} r_2^n & =\frac{1}{2\ii} \rho^n \e^{\ii n \theta}- \frac{1}{2\ii} \rho^n \e^{-\ii n \theta}\\
  &=\rho^n \sin(n \theta).
\end{align}
Cette suite est donc à valeurs réelles et par conséquent c'est un élément de \(\S_\R(\H)\).

On a trouvé deux éléments de \(\S_\R(\H)\), les suites réelles \((\rho^n \sin(n \theta))_{n \in \N}\) et \((\rho^n \cos(n \theta))_{n \in \N}\). Montrons qu'elles constituent une famille libre du \(\R\)-espace vectoriel \(\S_\R(\H)\) et on aura trouvé une base de \(\S_\R(\H)\) (car sa dimension est de deux). Pour tout couple de scalaires \((\lambda,\mu)\),
\begin{align}
  &\lambda (\rho^n \cos(n \theta))_{n \in \N} + \mu (\rho^n \sin(n \theta))_{n \in \N} = 0_{\K^\N} \\
  &\implies \forall n \in \N \ \lambda \rho^n \cos(n \theta) +\mu \rho^n \sin(n \theta) = 0\\
  &\implies \begin{cases} \lambda =0 & (n=0) \\ \lambda \rho \cos \theta +\mu \rho \sin \theta =0 & (n=1) \end{cases}\\
  &\implies \lambda=\mu=0 \text{~car } \rho\sin \theta \neq 0 (r_1 \notin \R).
\end{align}

\begin{theo}
  Si \((a,b,c) \in \R^3\) avec \(ac \neq 0\) et \(b^2-4ac <0\), l'équation caractéristique de \eqref{eq:H} a deux racines complexes non réelles et conjuguées \(\rho \e^{\ii \theta}\) et \(\rho \e^{-\ii \theta}\) avec \(\rho>0\) et \(\sin \theta \neq 0\).

  Alors \(S_\R(\H)\) admet pour base \(((\rho^n \sin(n \theta))_{n \in \N},(\rho^n \cos(n \theta))_{n \in \N})\). 
Autrement dit pour toute suite \(u \in \K^\N\),
  \begin{equation}
    u \in \S_\K(\H) \iff \exists! (A,B) \in \R^2 \ \forall n \in \N \quad u_n = \rho^n(A \cos(n\theta) +B\sin(n\theta)).
  \end{equation}
\end{theo}
La donnée des conditions initiales \(u_0=\alpha\) et \(u_1=\beta\) permet de trouver l'unique suite \(u\) solution de \eqref{eq:H} vérifiant ces conditions. Le couple \((A,B)\) est l'unique solution du système de Cramer \(\begin{cases} A =\alpha \\ \rho(A\cos \theta +B \sin\theta) =\beta \end{cases}\).


\subsection{Résolution de l'équation complète}

Soit \(f\) l'application de \(\K^\N\) dans lui-même qui à toute suite \(u\) lui associe la suite \((au_{n+2}+bu_{n+1}+cu_n)_{n \in \N}\). On vérifie facilement que \(f\) est linéaire, d'où \(f \in \Endo{\K^\N}\).

Pour toute suite \(u \in \K^\N\),
\begin{equation}
  u \in \S_\K(\L) \iff f(u) = (d_n)_{n \in \N}.
\end{equation}
On sait alors que~:
\begin{prop}
  Si \((d_n)_{n \in \N} \notin \Image(f)\) alors \(\S_\K(\L) = \emptyset\). Si \((d_n)_{n \in \N} \in \Image(f)\) alors \(\S_\K(\L)\) est le sous-espace affine de \(\K^\N\) de direction \(\S_\K(\H)\) passant par n'importe laquelle des solutions de \eqref{eq:S}.
\end{prop}

\subsubsection{Recherche d'une solution particulière pour \((d_n)_{n \in \N}\) suite constante de valeur \(d\)}

Soit un scalaire \(d\). Il reste donc à trouver une solution particulière de
\begin{equation}
  au_{n+2}+bu_{n+1}+cu_n = d \label{eq:S1} \tag{\(\Lb\)}.
\end{equation}

On commence par chercher s'il existe des suites constantes dans \(\S_\K(\Lb)\). Soit un scalaire \(\alpha\). La suite constante de valeur \(\alpha\) est solution de \eqref{eq:S} si, et seulement si, \((a+b+c)\alpha=d\).
\begin{itemize}
\item[Cas 1] : si \(a+b+c \neq 0\), alors une solution de \eqref{eq:S1} est la suite constante de valeur \(\frac{d}{a+b+c}\) et
  \begin{equation}
    \S_\K(\Lb) = \S_\K(\H) +\frac{d}{a+b+c}
  \end{equation}
\item[Cas 2] : si \(a+b+c=0\), cherchons s'il existe des suites du type \((\alpha+\beta n)_{n \in \N}\) solutions de \eqref{eq:S1}. Pour tout \((\alpha,\beta) \in \K^2\), \((\alpha + \beta n)_{n \in \N}\) est dans \(\S_\K(\L)\) si, et seulement si,
  \begin{align}
    \forall n \in \N \ a(\alpha +(n+2)\beta)+b(\alpha+(n+1)\beta)+c(\alpha +n\beta) &=d \\
    \forall n \in \N \ (a+b+c)\alpha +(a+b+c)n\beta +(2a+b)\beta &=d\\
    (2a+b)\beta &=d
  \end{align}
  On observe que \(\alpha\) est quelconque. En effet, lorsque \(a+b+c=0\), toutes les suites constantes \(u\) sont solutions de l'équation homogène donc vérifient \(f(u)=0\). On ne change donc pas la valeur de \(f(v)\) en ajoutant une suite constante \(u\) à la suite \(v\). Il suffit ainsi de chercher des solutions de \eqref{eq:S1} sous la forme \((\beta_n)_{n \in \N}\).
  \begin{itemize}
  \item[Cas 2-1]~: si \(a+b+c=0\) et \(2a+b \neq 0\), une solution de \eqref{eq:S1} est la suite \(\left(\frac{d}{2a+b}n\right)_{n \in \N}\). Donc~:
    \begin{equation}
      \S_\K(\Lb) = \S_\K(\H) + \left(\frac{d}{2a+b}n\right)_{n \in \N}
    \end{equation}
  \item[Cas 2-2]~: si \(a+b+c=0\) et \(2a+b = 0\). On a alors \(b=-2a\) et \(c=-a-b=a\), d'où \(\Delta=0\) et \(r_0=-b/2a=1\). L'ensemble des solutions de l'équation homogène est alors~:
    \begin{equation}
      \S_\K(\H) = \enstq{u \in \K^\N}{\exists (\lambda_1, \lambda_2) \in \K^2 \ \forall n \in \N \quad u_n = \lambda_1+\lambda_2 n}.
    \end{equation}
    Cherchons une solution de \eqref{eq:S1} du type \((\alpha+\beta n + \gamma n^2)_{n \in \N}\). On peut choisir \(\alpha\) et \(\beta\) quelconques puisque les suites \((\alpha+\beta n)_{n \in \N}\) sont solutions de l'équation homogène. Pour tout scalaire \(\gamma\), la suite \((\gamma n^2)_{n \in \N}\) est dans \(\S_\K(\Lb)\) si et seulement si~:
    \begin{align}
      \forall n \in \N a\gamma(n+2)^2 +b\gamma(n+1)^2 +c\gamma n^2 &=d\\
      \forall n \in \N (a+b+c)\gamma n^2 +2(2a+b)\gamma n +(4a+b)\gamma &=d \\
      (4a+b)\gamma &=d.
    \end{align}
    Or \(4a+b=4a-2a=2a \neq 0\) par hypothèse, donc la suite \(\left(\frac{d}{2a} n^2\right)_{n \in \N}\) est solution de \eqref{eq:S1}. D'où
    \begin{equation}
      \S_\K(\Lb) = \S_\K(\H) + \left(\frac{d}{2a} n^2\right)_{n \in \N}.
    \end{equation}
  \end{itemize}
\end{itemize}
