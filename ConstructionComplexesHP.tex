\chapter{Construction du corps des complexes}

On munit l'ensemble \(\R^2\) de deux lois internes, une addition notée \(+\) et une multiplication notée \(\times\) définies pour tout \(x,x',y,y' \in \R\) par
\begin{align}
  (x,y)+(x',y') &= (x+x',y+y')\\
  (x,y)\times(x',y') &= (xx'-yy',xy'+x'y)
\end{align}

\begin{prop}
  \((\R^2,+)\) est un groupe abélien, c'est à dire
  \begin{enumerate}
  \item la loi \(+\) est associative et commutative. Soient \(x,x',x'',y,y',y'' \in \R\)
    \begin{align}
      [(x,y)+(x',y')]+(x'',y'')&=(x,y)+[(x',y')+(x'',y'')] \\
      (x,y)+(x',y')=(x',y')+(x,y);
    \end{align}
  \item \(\R^2\) possède un élément neutre pour \(+\) : \((0,0)\);
    \begin{equation}
      \forall x,y \in \R \quad (x,y)+(0,0)=(x,y)
    \end{equation}
  \item Tout élément de \(\R^2\) est inversible par \(+\);
    \begin{equation}
      \forall x,y \in \R \quad (-x,-y)+(x,y)=(0,0)
    \end{equation}
  \end{enumerate}
\end{prop}
\begin{prop}
  \((\R^2,+,\times)\) est un anneau commutatif, c'est à dire
  \begin{enumerate}
  \item \((\R^2,+)\) est un groupe abélien;
  \item la loi \(\times\) est associative et commutative, c'est à dire que pour tout \(x,x',x'',y,y',y'' \in \R\)
    \begin{align}
      [(x,y)\times(x',y')]\times (x'',y'')&=(x,y)\times [(x',y')\times (x'',y'')] \\
      (x,y)\times (x',y')&=(x',y')\times (x,y);
    \end{align}
  \item la loi \(\times\) est distributive par rapport à la loi \(+\). Soient \(x,x',x'',y,y',y'' \in \R\) 
    \begin{equation}
      (x,y) \times [(x',y')+(x,y)]= (x,y) \times (x',y') +(x,y) \times (x'',y'');
    \end{equation}
  \item \(\R^2\) possède un élément neutre pour la loi \(\times\) : \((1,0)\). Soient \(x,y \in \R\)
    \begin{equation}
      (x,y)\times (1,0)=(x,y);
    \end{equation}
  \end{enumerate}
\end{prop}
\begin{proof}
  \begin{enumerate}
  \item Déjà vu;
  \item La commutativité est triviale d'après la définition de \(\times\). Montrons l'associativité. Soient \(x,x',y',x'',y'' \in \R\) alors
    \begin{align}
      [(x,y) \times (x',y')] \times (x'',y'') &= (xx'-yy',xy'+x'y) \times (x'',y'')\\
      &=(xx'x''-yy'x''-xy'y''-x'yy'', \notag \\
      &xx'y''-yy'y''+xy'x''+x'yx'')
    \end{align}
    \begin{align}
      (x,y)\times [(x',y') \times (x'',y'')] &= (x,y) \times (x'x''-y'y'',x'y''+x''y')\\
      &=(xx'x''-xy'y''-yy'x''-x'yy'', \notag \\
      & xx'y''+xy'x''+x'yx''-yy'y'')
    \end{align}
    d'où l'égalité.
  \item Soient \(x,x',y',x'',y'' \in \R\) alors
    \begin{align}
      (x,y) \times [(x',y')+(x'',y'')] &= (x,y) \times (x'+x'',y'+y'')\\
      &=(xx'+xx''-yy'-yy'', \notag \\ &xy'+xy''+yx'+yx'')\\
      &=(xx'-yy',xy'+yx') \notag \\
      & + (xx'-yy'',xy''+yx'')\\
      &=(x,y) \times (x',y') +(x,y)\times (x'',y'')
    \end{align}
  \end{enumerate}
\end{proof}
\begin{prop}
  \((\R^2,+,\times)\) est un corps commutatif, c'est à dire
  \begin{enumerate}
  \item \((\R^2,+,\times)\) est un anneau commutatif;
  \item Tous les éléments non nuls sont inversibles pour \(\times\) et le neutre et \((1,0)\);
  \item Le neutre pour \(+\) est différent du neutre pour \(\times\).
  \end{enumerate}
\end{prop}
\begin{proof}
  \begin{enumerate}
  \item Déjà vu;
  \item Soit \((x,y) \in \R^2 \ (x,y) \neq (0,0)\). Alors \(x^2+y^2>0\) et
    \begin{equation}
      (x,y) \times \frac{1}{\sqrt{x^2+y^2}} (x,-y) = (1,0)
    \end{equation}
  \end{enumerate}
\end{proof}
Soit l'application \(\fonction{\varphi}{\R}{\R^2}{x}{(x,0)}\). Elle est injective et c'est un morphisme d'anneaux :
\begin{equation}
  \forall (x,y) \in \R^2 \quad \varphi(x+y)=\varphi(x)+\varphi(y), \ \varphi(xy)\varphi(x)\varphi(y), \ \varphi(1)=(1,0)
\end{equation}
On peut donc identifier \(\R\) au sous-ensemble \(\R \times \{0\}\) de \(\R^2\) et pour tout \(x \in \R\), \(x\) avec \((x,0)\).
\begin{defdef}
  On note \(\C\) l'ensemble \(\R^2\) muni des deux lois \(+\) et \(\times\).
\end{defdef}
Ainsi, \(\C\) est un corps commutatif. On appelle \(\ii\) le complexe \((0,1)\) et donc \(\ii^2 = (0,1) \times (0,1)=(-1,0)=-1\). On a alors pour tout couple \((x,y) \in \R^2\)
\begin{equation}
  (x,y)=(x,0) + (0,y) = (x,0) \times (1,0) + (y,0) \times (0,1) = x \times 1 + y \times \ii = x+ \ii y
\end{equation}
On adopte la notation définitive du complexe \((x,y)=x+\ii y\). L'écriture \(z=x+\ii y, (x,y) \in \R^2\) s'appelle la forme normale ou la forme algébrique du complexe \(z\).
\begin{prop}
  Pour tout élément \(z \in \C\), il existe un unique couple \((x,y) \in \R^2\) tel que \(z=x+\ii y\).
\end{prop}
\begin{proof}
  Pour tout \((x,y,x',y') \in \R^4\), on a
  \begin{equation}
    z=x+\ii y =x'+\ii y' \Longleftrightarrow z=(x,y)=(x',y') \Longleftrightarrow \begin{cases} x=x' \\ y=y'\end{cases}
  \end{equation}
\end{proof}
