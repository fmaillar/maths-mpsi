\chapter{Suites  réelles}\label{chap:suites}
\minitoc%
\minilof%
\minilot%
\section{Convergence et divergence d'une suite numérique}

\subsection{\(\R\)-espace vectoriel \((\R^{\N}, +, \perp)\) des suites réelles à 
valeurs dans le corps \((\R, +, \cdot )\)}

\begin{defdef}
    On appelle suite de nombres réels ou suites à valeurs réelles, toute famille 
    de réels indexée par \(\N\), c'est à dire toute application de \(\N\) dans 
    \(\R\).
\end{defdef}
\(u_n\) est appelé le terme général de la suite réelle \(u\). La suite réelle 
\(u\) est notée \((u_n)_{n\in\N}\). \emph{Ne pas confondre la suite réelle \(u\) 
et son terme général \(u_n\)}. On notera \(\R^{\N}\) l'ensemble des suite 
réelles. Par extension on parlera aussi de suite réelle pour des familles 
indexées par \(\N^*\) ou même indexées par \(\N \cap 
\intervallefo{n_0}{+\infty}\) pour un \(n_0 \in \N\). On munit \(\R^\N\) d'une 
loi de composition interne, noté \(+\), définie par
\begin{equation}
    \forall u,v \in \R^\N \quad \fonction{u+v}{\N}{\R}{n}{u_n+v_n}.
\end{equation}
L'ensemble \((\R^\N, +)\) est un groupe abélien. L'élément neutre pour l'addition 
est la suite réelle nulle de terme général égal à \(0\). On munit aussi 
\(\R^\N\) d'une loi de composition externe appelée multiplication par un réel et 
notée \(\perp\) ou rien du tout définie telle que
\begin{equation}
    \fonction{\perp}{\R\times\R^\N}{\R^\N}{(\lambda,u)}{(\lambda u_n)_{n\in\N}}.
\end{equation}
\begin{prop}
    L'ensemble \((\R^\N,+,\perp)\) est un \(\R\)-espace vectoriel.
\end{prop}
On munit \(\R^\N\) d'une deuxième loi de composition interne appelée 
multiplication et notée \(\times\) ou rien du tout, définie par
\begin{equation}
    \forall u,v \in \R^\N \quad uv = (u_nv_n)_{n\in \N}.
\end{equation}
\begin{prop}
    L'ensemble \((\R^\N,+, \cdot )\) est un anneau commutatif non intègre. De plus
    \begin{equation}
        \forall \lambda \in \R \ \forall u,v \in \R^\N \quad \lambda(uv) = (\lambda 
        u)v = u(\lambda v).
    \end{equation}
\end{prop}
Ainsi, \((\R^\N,+,\perp, \cdot )\) est une \(\R\)-algèbre. La suite réelle 
constante égale à \(1\) est neutre pour la multiplication.
\begin{defdef}
    On dit qu'une suite réelle \(u\) est constante s'il existe un réel \(k\) tel 
    que pour entier \(n\) \(u_n = k\). On dit qu'une suite réelle \(u\) est 
    stationnaire s'il existe un entier \(n_0\) et un réel \(k\) tels que pour tout 
    entier naturel \(n\), si \(n\geqslant n_0\) alors \(u_n = k\).
\end{defdef}

\subsection{\(\R\)-espace vectoriel \(B(\R)\) des suites réelles bornées}

\begin{defdef}
    Soit \(u\) une suite réelle. On dit que~:
    \begin{enumerate}
        \item la suite \(u\) est majorée s'il existe un réel \(M\) tel que pour tout 
            entier naturel \(n\) \(u_n\leqslant M\)
        \item la suite \(u\) est minorée s'il existe un réel \(m\) tel que pour tout 
            entier naturel \(n\) \(u_n\geqslant m\)
        \item la suite \(u\) est bornée si elle est majorée et minorée.
    \end{enumerate}
\end{defdef}
\begin{prop}
    Soit une suite réelle \(u\), alors \(u\) est bornée si et seulement s'il 
    existe un réel positif \(M\) tel que pour tout naturel \(n\) 
    \(\abs{u_n}\leqslant M\).
\end{prop}
\begin{proof}
    \begin{itemize}
        \item[\(\impliedby\)] S'il existe un réel \(M\geqslant 0\) tel que pour tout 
            naturel \(n\) \(\abs{u_n}\leqslant M\), alors \(u\) est majorée par \(M\) 
            et minoré par \(-M\) donc \(u\) est bornée.
        \item[\(\implies\)] Si \(u\) est bornée, il existe deux réels \(m\) et \(M\) 
            tels que pour tout naturel \(n\) \(m\leqslant u_n \leqslant M\). Posons 
            \(M_0 = \max(\abs{m},\abs{M})\), alors pour tout naturel \(n\)
            \begin{equation}
                -M_0\leqslant -\abs{m} \leqslant m\leqslant u_n\leqslant M\leqslant 
                \abs{M}\leqslant M_0,
            \end{equation}
            donc en passant à la valeur absolue
            \begin{equation}
                \forall n \in \N \quad \abs{u_n}\leqslant M_0.
            \end{equation}
    \end{itemize}
\end{proof}
On note \(B(\R)\) l'ensemble des suites réelles bornées.
\begin{prop}
    L'ensemble \(B(\R)\) est un espace vectoriel.
\end{prop}
\begin{proof}
    Premièrement \(B(\R)\) est non vide, puisque la suite réelle nulle est bornée. 
    La combinaison linéaire de deux suites réelles bornées est bornée, en effet 
    soient \(u\) et \(v\) bornées et un réel \(\lambda\). Il existe donc deux 
    réels positifs \(M\) et \(M'\) tel que pour tout naturel \(n\) 
    \(\abs{u_n}\leqslant M\) et \(\abs{v_n}\leqslant M'\). Alors pour tout naturel 
    \(n\)
    \begin{equation}
        \abs{\lambda u_n + v_n}\leqslant \abs{\lambda}\abs{u_n}+\abs{v_n}\leqslant 
        \abs{\lambda} M+M'.
    \end{equation}
    Donc \(\lambda u+v\) est bornée. Ainsi \(B(\R)\) est un espace vectoriel.
\end{proof}
\begin{prop}
    Le produit de deux suites réelles bornées est bornée.
\end{prop}
\begin{proof}
    Soient \(u\) et \(v\) bornées. Il existe donc \(M\) et \(M'\) des réels positifs 
    tels que pour tout naturel \(n\) \(\abs{u_n}\leqslant M\) et 
    \(\abs{v_n}\leqslant M'\) donc \(\abs{u_n v_n} = \abs{u_n}\abs{v_n}\leqslant 
    MM'\).
\end{proof}

\subsection{Notion de suite réelle convergente et de suite réelle divergente}

\subsubsection{Suites réelles convergentes}

\begin{defdef}
    Soit une suite réelle \(u\). On dit que la suite réelle \(u\) converge ou tend 
    vers \(0\) si et seulement si
    \begin{equation}
        \forall \epsilon >0 \ \exists n_0\in \N \ \forall n \in \N \quad n\geqslant 
        n_0 \implies \abs{u_n}\leqslant \epsilon
    \end{equation}
\end{defdef}
\begin{defdef}
    Soient une suite réelle \(u\) et un réel \(l\). On dit que \(u\) tend vers 
    \(l\) si et seulement si \(u-l\) tend vers 0. On dit que \(u\) est convergent 
    si et seulement s'il existe un réel \(l\) tel que \(u\) tende vers \(l\). Le 
    réel \(l\), s'il existe, est unique et il est appelé la limite de la suite 
    réelle \(u\). On note
    \begin{equation}
        l = \lim\limits_{n\to\infty} u_n \quad l = \lim u \quad u_n \rightarrow l.
    \end{equation}
\end{defdef}
\begin{proof}[Unicité]
    Soit une suite réelle \(u\). On suppose qu'il existe deux réels différents 
    \(l\) et \(l'\) tels que \(u\) tende vers ces deux réels. Alors
    \begin{align*}
        \forall \epsilon >0 \ \exists n_0\in \N \ \forall n\in\N \quad n\geqslant 
        n_0 \implies \abs{u_n-l}\leqslant \epsilon \\
        \forall \epsilon >0 \ \exists n_1\in \N \ \forall n\in\N \quad n\geqslant 
        n_1 \implies \abs{u_n-l'}\leqslant \epsilon.
    \end{align*}
    Puisque \(l\) et \(l'\) sont différents, prenons \(\epsilon = 
    \frac{\abs{l-l'}}{3}\) et \(n_2 = max(n_1,n_0)\) alors pour tout naturel \(n\) 
    si \(n\geqslant n_2\) alors
    \begin{equation}
        \abs{l-l'}\leqslant \abs{l-u_n}+\abs{u_n-l'}\leqslant \frac{2}{3} 
        \abs{l-l'},
    \end{equation}
    alors puisque \(\abs{l-l'}>0\) on obtient \(1\leqslant \frac{2}{3}\), ce qui 
    est absurde. Donc \(l = l'\).
\end{proof}
\begin{defdef}
    Une suite réelle qui n'est pas convergente est divergente.
\end{defdef}
\begin{prop}
    Soient une suite réelle \(u\) et un réel \(l\). Si \(u\) converge vers \(l\), 
    alors \(\abs{u}\) converge vers \(\abs{l}\). \emph{La réciproque est fausse}
\end{prop}
\begin{proof}
    La suite réelle \(u\) converge vers \(l\) donc
    \begin{equation}
        \forall \epsilon>0 \ \exists n_0 \in \N \ \forall n \in \N \quad n\geqslant 
        n_0 \implies \abs{u_n-l}\leqslant \epsilon.
    \end{equation}
    Alors pour tout naturel \(n\), si \(n\geqslant n_0\) on a
    \begin{equation}
        \abs{\abs{u_n}-\abs{l}} = \abs{\abs{u_n}-\abs{-l}}\leqslant 
        \abs{u_n+(-l)}\leqslant \epsilon.
    \end{equation}
    Donc \(\abs{u}\) tend vers \(\abs{l}\).
\end{proof}
La réciproque est fausse, en effet il peut arriver que \(\abs{u}\) converge sans 
que \(u\) converge comme par exemple la suite réelle de terme général \(u_n = 
(-1)^n\).
\begin{prop}
    Toute suite réelle convergente est bornée. \emph{La réciproque est fausse}.
\end{prop}
\begin{proof}
    Soit \(u\in \R^\N\), on suppose que \(u\) converge vers \(l\). On écrit la 
    définition avec \(\epsilon = 1\). Il existe alors un naturel \(n_0\) tel que 
    pour tout \(n\geqslant n_0\) \(\abs{u_n-l}\leqslant 1\). Alors
    \begin{equation}
        \abs{u_n} = \abs{u_n-l+l}\leqslant \abs{u_n-l}+\abs{l},
    \end{equation}
    alors
    \begin{equation}
        \abs{u_n}\leqslant 1+\abs{l}.
    \end{equation}
    Soient \(M_0 = \max(\abs{u_0},\ldots,\abs{u_{n_0-1}})\) et \(M = 
    \max(1+\abs{l},M_0)\), alors
    \begin{equation}
        \forall n \in \N \quad
        \begin{cases}
            n \geqslant n_0 & \abs{u_n}\leqslant 1+\abs{l}\leqslant M \\
            n > n_0 & \abs{u_n}\leqslant M_0\leqslant M
        \end{cases}.
    \end{equation}
    Dans tous les cas, on majore la valeur absolue de \(u\) par \(M\). Donc \(u\) 
    est bornée.
\end{proof}

Si on sait qu'une suite réelle n'est pas bornée, on peut en conclure directement 
qu'elle diverge.
\begin{prop}
    Soient une suite réelle \(u\) et deux réels \(a\) et \(b\). On suppose que 
    \(u\) converge vers une limite \(l\in\R\)
    \begin{enumerate}
        \item si \(a<l\) alors il existe un naturel \(n_0\) tel que pour tout 
            naturel \(n\), si \(n\geqslant n_0\) alors \(u_n\geqslant a\)
        \item si \(b>l\) alors il existe un naturel \(n_1\) tel que pour tout 
            naturel \(n\), si \(n\geqslant n_1\) alors \(u_n\leqslant b\)
        \item si \(a\leqslant l\leqslant b\) alors il existe un naturel \(n_2\) tel 
            que pour tout naturel \(n\), si \(n\geqslant n_2\) alors \(a \leqslant u_n 
            \leqslant b\)
    \end{enumerate}
\end{prop}
\begin{proof}
    \begin{enumerate}
        \item si \(a<l\) alors on pose \(\epsilon = l-a>0\), il existe un naturel 
            \(n_0\) tel que pour tout naturel \(n\) si \(n \geqslant n_0\) alors 
            \(\abs{u_n-l}\leqslant l-a\) alors \(a-l \leqslant u_n \leqslant l-a\) 
            donc \(a \leqslant u_n\)
        \item De la même manière, si \(b>l\) on pose \(\epsilon = b-l\) et il existe 
            un naturel \(n_1\) tel que pour tout naturel \(n\) si \(n \geqslant n_1\) 
            alors \(\abs{u_n-l} \leqslant b-l\) alors \(u_n-l \leqslant 
            \abs{u_n-l}\leqslant b-l\) donc \(u_n \leqslant b\)
        \item Soit \(n_2 = \max(n_1,n_0)\) donc pour tout naturel \(n\), si \(n 
            \geqslant n_2\) alors \(u_n \geqslant a\) et \(u_n \leqslant b\) donc \(a 
            \leqslant u_n \leqslant b\).
    \end{enumerate}
\end{proof}
\begin{defdef}
    Soit une propriété \(P\) portant sur un entier \(n\). On dira que \(P\) est 
    vraie ``à partir d'un certain rang'' s'il existe un naturel \(n_0\) tel que 
    pour tout naturel \(n\) si \(n\geqslant n_0\) alors \(P(n)\) est vraie.
\end{defdef}
\begin{cor}
    Soit une suite réelle \(u\). Si \(u\) converge vers une limite \(l\) positive 
    alors il existe un réel \(a\) positif tel que \(u\) soit minorée par \(a\) ``à 
    partir d'un certain rang''
\end{cor}
\begin{proof}
    Soit \(a = \frac{l}{2}\) alors \(0<a<l\). On applique la proposition avec ce 
    réel \(a\). Il existe un naturel \(n_0\) tel que pour tout naturel \(n\) si 
    \(n \geqslant n_0\) alors \(u_n \geqslant a\). La suite réelle \(u\) est donc 
    minorée ``à partir d'un certain rang''.
\end{proof}
\begin{cor}
    Soit une suite réelle \(u\). Si \(u\) converge vers une limite \(l\) non 
    nulle, alors il existe un réel \(a\) strictement positif tel que la suite 
    réelle \(\abs{u}\) soit minorée par \(a\) ``à partir d'un certain rang''.
\end{cor}
\begin{proof}
    On sait que \(\abs{u}\)  converge vers \(\abs{l}>0\). On applique le 
    corollaire précédent à cette suite réelle.
\end{proof}

\subsubsection{Suites réelles tendant vers l'infini}

\begin{defdef}
    Soit une suite réelle \(u\), on dit que \(u\) tend vers \(+\infty\) si et 
    seulement si
    \begin{equation}
        \forall A \in \R \ \exists n_0 \in \N \ \forall n \in \N \quad n \geqslant 
        n_0 \implies u_n \geqslant A,
    \end{equation}
    et on note \(\lim\limits_{n\to\infty}u_n = +\infty\). 

    On dit aussi que \(u\) tend vers \(-\infty\) si et seulement si
    \begin{equation}
        \forall A \in \R \ \exists n_0 \in \N \ \forall n \in \N \quad n \geqslant 
        n_0 \implies u_n \leqslant A
    \end{equation}
    et on note \(\lim\limits_{n\to\infty}u_n = -\infty\).
\end{defdef}
Une suite réelle qui tend vers \(\pm\infty\) diverge. De telles suites réelles 
sont dites divergentes de première espèce. Les suites réelles divergentes qui 
n'admettent pas de limites sont dites divergentes de deuxième espèce, par 
exemple telle que la suite réelle de terme général \(u_n = (-1)^n\).
\begin{prop}
    Soit une suite réelle \(u\). Si \(u\) tend vers \(\pm\infty\) alors 
    \(\abs{u}\) tend vers \(+\infty\).
\end{prop}
\begin{proof}
    En effet, si la limite de \(u\) est \(+\infty\), alors pour tout réel \(A\) il 
    existe un naturel \(n_0\)  tel que pour tout naturel \(n\) si \(n \geqslant 
    n_0\) alors \(u_n \geqslant A\), donc \(\abs{u_n} \geqslant u_n \geqslant A\). 
    Ainsi \(\abs{u}\) tend vers \(+\infty\). 

    De la même manière si \(u\) tend \(-\infty\) alors  pour tout réel \(B\) il 
    existe un naturel \(n_1\)  tel que pour tout naturel \(n\) si \(n \geqslant 
    n_1\) alors \(u_n \leqslant -B\), alors \(-u_n \geqslant B\) donc \(\abs{u_n} 
    \geqslant -u_n \geqslant B\). Ainsi \(\abs{u}\) tend vers \(+\infty\).  
\end{proof}
\begin{prop}
    Soit une suite réelle \(u\).
    \begin{enumerate}
        \item Si \(u\) tend vers \(+\infty\) alors \(u\) est minorée mais n'est pas 
            majorée.
        \item Si \(u\) tend vers \(-\infty\) alors \(u\) est majorée mais n'est pas 
            minorée.
    \end{enumerate}
\end{prop}
\begin{proof}
    \begin{enumerate}
        \item D'après la définition, \(u\) n'est pas majorée. En particulier pour 
            tout réel \(A\) il existe un entier \(n_0\) tel que \(u_{n_0} \geqslant 
            A\). Si on applique la définition avec \(A = 0\) alors pour tout entier 
            \(n\) plus grand que \(n_0\) \(u_{n_0} \geqslant 0\). Soient \(m_0 = 
            \min(u_0,\ldots,u_{n_0-1},0)\) et un naturel \(n\). Si \(n > n_0\) alors 
            \(u_n \geqslant 0 \geqslant m_0\). Si \(n \leqslant n_0\) alors \(u_n 
            \geqslant m_0\). Donc pour tout entier \(n\) \(u_n \geqslant m_0\). La 
            suite réelle \(u\) est minorée
        \item idem
    \end{enumerate}
\end{proof}
La suite réelle \(\abs{u}\) peut tendre vers \(+\infty\) sans que \(u\) tende 
vers \(+\infty\) ou \(-\infty\). Comme par exemple la suite réelle de terme 
général \(u_n = (-1)^n n\)

\subsection{Suites réelles extraites}

\begin{defdef}
    Soit une suite réelle \(u\), on dit que \(v\) est une suite réelle extraite de 
    \(u\) s'il existe une application \(\varphi:\N \longmapsto \N\) strictement 
    croissante telle que \(\forall n\in \N \quad v_n = u_{\varphi(n)}\).
\end{defdef}
\begin{lemme}
    Soit une application \(\varphi:\N \longmapsto \N\) strictement croissante, 
    alors pour tout naturel \(n\) \(\varphi(n) \geqslant n\).
\end{lemme}
\begin{proof}
    On montre par récurrence sur \(n \in \N\) la propriété \(\P(n)\) \(\varphi(n) 
    \geqslant n\).
    \begin{itemize}
        \item[\emph{Initialisation}] \(n = 0\), puisque \(\varphi(0) \in \N\) alors 
            \(\varphi(0) \geqslant 0\).
        \item[\emph{Hérédité}] soit \(n \in \N\), supposons \(\P(n)\) alors puisque 
            \(\varphi\) est strictement croissante \(\varphi(n+1) > \varphi(n)\) et 
            par hypothèse de récurrence \(\varphi(n+1) > n\) et comme \(\varphi(n+1) 
            \in \N\) on a \(\varphi(n+1) \geqslant n+1\). Alors \(\P(n+1)\) est vraie.
        \item[\emph{Conclusion}] par théorème de récurrence, la propriété \(\P\) est 
            vraie pour tout naturel \(n\).
    \end{itemize}
\end{proof}
\begin{prop}
    Soient \(u\) une suite réelle et \(v\) une suite réelle extraite de \(u\). Si 
    \(u\) converge vers un réel \(l\) alors \(v\) converge aussi vers \(l\). La 
    réciproque est fausse.
\end{prop}
\begin{proof}
    Puisque \(v\) est extraite de \(u\), il existe une application \(\varphi:\N 
    \longmapsto \N\) strictement croissante telle que pour tout naturel \(n\) 
    \(v_n = u_{\varphi(n)}\). Puisque \(u\) converge vers \(l\), pour tout 
    \(\epsilon>0\) il existe un naturel \(n_0\) tel que pour tout naturel \(n\), 
    si \(n \geqslant n_0\) alors \(\abs{u_n-l} \leqslant \epsilon\). D'après le 
    lemme, pour tout naturel \(n_0\), \(\varphi(n) \geqslant n \geqslant n_0\) 
    donc \(\abs{v_n-l} = \abs{u_{\varphi(n)}-l} \leqslant \epsilon\). Alors \(v\) 
    converge vers \(l\).
\end{proof}
Pour démontrer qu'une suite réelle diverge, on peut parfois montrer soit qu'une 
de ses sous-suites réelles diverge, soit que deux de ses sous-suites réelles ont 
des limites différentes. Comme par exemple la suite réelle de terme général 
\(u_n = (-1)^n\).
\begin{prop}
    Soit une suite réelle \(u\). On suppose que les deux suite réelle extraites 
    \((u_{2n})_{n \in \N}\) et \((u_{2n+1})_{n \in \N}\) convergent toutes deux 
    vers un réel \(l\). Alors \(u\) converge vers \(l\).
\end{prop}
\begin{proof}
    Soit un réel \(\epsilon>0\). Il existe alors deux naturels \(n_1\) et \(n_2\) 
    tels que pour tout naturel \(n\)
    \begin{align}
        n \geqslant n_0 \implies \abs{u_{2n}-l} \leqslant \epsilon \\  n \geqslant 
        n_1 \implies \abs{u_{2n+1}-l} \leqslant \epsilon
    \end{align}
    On note \(n_2 = \max(2n_0,2n_1+1)\) et si \(n \geqslant n_2\) alors
    \begin{itemize}
        \item \(n\) est pair, il existe donc un naturel \(p\) tel que \(n = 2p\) 
            donc \(p\geqslant n_0\) d'où \(\abs{u_n-l} = \abs{u_{2p}-l} \leqslant 
            \epsilon\)
        \item \(n\) est impair, il existe donc un naturel \(p\) tel que \(n = 2p+1\) 
            donc \(p\geqslant n_1\) d'où \(\abs{u_n-l} = \abs{u_{2p+1}-l} \leqslant 
            \epsilon\)
    \end{itemize}
    Alors pour tout naturel \(n\) si \(n \geqslant n_2\) alors \(\abs{u_n-l} 
    \leqslant \epsilon\). Donc \(u\) converge vers \(l\).
\end{proof}
\begin{prop}
    Soit une suite réelle \(u\) et \(v\) une suite réelle extraite de \(u\). Si 
    \(u\) est bornée, alors \(v\) est aussi bornée. La réciproque est fausse.
\end{prop}

\subsection{\(\R\)-espace vectoriel des suites réelles de limite nulle}

\begin{prop}
    L'ensemble des suites réelles de limite nulle est un sous espace vectoriel de 
    \(\R^\N\).
\end{prop}
\begin{proof}
    La suite réelle nulle est de limite nulle, donc il est non vide. Cet espace 
    est stable par combinaison linéaire, en effet soient deux suites réelles de 
    limite nulle et un réel \(\lambda\). Montrons que \(\lambda u+v\) est de 
    limite nulle. Déjà si \(\lambda = 0\) alors \(\lambda u+v = v\) est de limite 
    nulle. On suppose désormais que \(\lambda\) est non nul. Soit \(\epsilon>0\) 
    il existe deux naturels \(n_0\) et \(n_1\) tels que pour tout naturel \(n\)
    \begin{align}
        n \geqslant n_0 \implies \abs{u_n} \leqslant \frac{\epsilon}{2\abs{\lambda}} 
        \\ n \geqslant n_1 \implies \abs{v_n} \leqslant \frac{\epsilon}{2}
    \end{align}
    Posons \(n_2 = \max(n_0,n_1)\) alors si \(n \geqslant n_2\) on a
    \begin{equation}
        \abs{\lambda u_n + v_n}\leqslant \abs{\lambda}\abs{u_n}+\abs{v_n}\leqslant 
        \epsilon
    \end{equation}
    donc \(\lambda u+v\) tend vers 0.
\end{proof}
%
\begin{prop}
    Le produit d'une suite réelle de limite nulle par une suite réelle bornée est 
    de limite nulle. En particulier le produit d'une suite réelle de limite nulle 
    et d'une suite réelle convergente est de limite nulle.
\end{prop}
\begin{proof}
    Soient \(u\) et \(v\) deux suites réelles telles que \(u\) soit de limite 
    nulle et \(v\) bornée. Il existe alors un réel strictement positif \(M\) tel 
    que pour tout entier \(n\) \(v_n \leqslant M\). Puisque \(u\) est de limite 
    nulle, pour tout \(\epsilon>0\) il existe un entier naturel \(n_0\) tel que 
    pour tout entier naturel \(n\) si \(n \geqslant n_0\) alors \(\abs{u_n} 
    \leqslant \frac{\epsilon}{M}\). Pour tout entier naturel \(n\)  si \(n 
    \geqslant n_0\) \(\abs{u_n v_n} = \abs{u_n}\abs{v_n}\leqslant M 
    \frac{\epsilon}{M} = \epsilon\). Donc la suite réelle \(uv\) tend vers 0.
\end{proof}

\subsection{Opérations sur les limites}

\subsubsection{Suites réelles convergentes}

\begin{prop}
    Soient \(u\) et \(v\) deux suites réelles convergentes de limites respectives 
    \(l\) et \(l'\). Alors pour tout réels \(\lambda\) et \(\mu\)~:
    \begin{itemize}
        \item \(\lambda u + \mu v\) tend vers \(\lambda l + \mu l'\)
        \item \(uv\) tend vers \(ll'\)
    \end{itemize}
\end{prop}
\begin{proof}
    Soit un naturel \(n\), alors \((\lambda u_n + \mu v_n) - (\lambda l +\mu l') = 
    \lambda(u_n-l)+\mu(v_n-l')\). Puisque \((u_n-l)_{n\in\N}\) et 
    \((v_n-l')_{n\in\N}\) sont de limite nulle. Le produit d'une suite réelle de 
    limite de limite nulle par une constante est une suite réelle de limite nulle. 
    Ainsi \(\lambda(u_n-l)\) et \(\mu(v_n-l')\) sont de limite nulle. Leur somme 
    converge donc vers zéro. Pour tout entier naturel \(n\)
    \begin{equation}
        u_nv_n-ll' = (u_n-l)v_n+l(v_n-l').
    \end{equation}
    \(l(v_n-l)\) tend vers zéro, puisque c'est une suite réelle de limite nulle 
    multipliée par une constante. De même \((u_n-l)v_n\) tend vers zéro, puisque 
    c'est le produit d'une suite réelle de limite nulle par une suite réelle 
    convergente. La somme de deux suites réelles de limite nulle est donc de 
    limite nulle.
\end{proof}
L'ensemble des suites réelles convergentes est un sous espace vectoriel de 
\(\R^\N\).
\begin{prop}
    Soit une suite réelle \(u\). On suppose que \(u\) converge vers une limite 
    \(l\) non nulle. Alors il existe un entier naturel \(n_0\) tel que pour tout 
    \(n \geqslant n_0\) \(u_n \neq 0\). La suite réelle 
    \(\left(\frac{1}{u_n}\right)_{n \geqslant n_0}\) est convergente et tend vers 
    \(\frac{1}{l}\)
\end{prop}
\begin{proof}
    \(l\) est non nulle donc il existe un réel \(a\) strictement positif et un 
    naturel \(n_0\) tels que pour tout entier \(n\) si \(n \geqslant n_0\) alors 
    \(\abs{u_n} \geqslant a >0\). En particulier si \(n \geqslant n_0\) alors 
    \(u_n \neq 0\) donc on peut considérer la suite réelle 
    \(\left(\frac{1}{u_n}\right)_{n \geqslant n_0}\). Soit un entier \(n\) tel que 
    \(n \geqslant n_0\) alors
    \begin{equation}
        \frac{1}{u_n}-\frac{1}{l} = \frac{1}{lu_n}(l-u_n).
    \end{equation}
    Or on sait que
    \begin{align}
        \forall n \geqslant n_0 \quad &\abs{u_n} \geqslant a > 0 \\
        & 0 \leqslant \frac{1}{\abs{u_n}} \leqslant 
        \frac{1}{a} \\
        & 0 \leqslant \frac{1}{\abs{lu_n}} \leqslant 
        \frac{1}{a\abs{l}}.
    \end{align}
    La suite réelle \(\left(\frac{1}{\abs{lu_n}}\right)_{n \geqslant n_0}\) est 
    bornée et la suite réelle \((l-u_n)_{n \geqslant n_0}\) tend vers 0. Leur 
    produit est une suite réelle de limite nulle. Autrement dit la suite réelle de 
    terme général \(\frac{1}{u_n}\) tend donc vers \(\frac{1}{l}\).
\end{proof}

\subsubsection{Suites réelles tendant vers l'infini}

\begin{prop}
    Soient \(u\) et \(v\) deux suites réelles telles que \(u\) tende vers 
    \(+\infty\) et \(v\) soit minorée. Alors \(u+v\) tend vers \(+\infty\).
\end{prop}
\begin{proof}
    \(v\) est minorée, il existe alors un réel \(a\) tel que pour tout entier 
    naturel \(n\) tel que \(u_n \geqslant a\). Comme \(u\) tend vers \(+\infty\), 
    soit un réel \(A\), il existe un entier naturel \(n_0\) tel que pour tout 
    entier naturel \(n\) si \(n \geqslant n_0\) alors \(u_n \geqslant A-a\). Ainsi 
    si \(n \geqslant n_0\) \(u_n+v_n \geqslant A-a+a =A\) alors \(u+v\) tend vers 
    \(+\infty\).
\end{proof}
\begin{prop}
    De la même manière on montre que si \(u\) tend vers \(-\infty\) et si \(v\) 
    est majorée, alors \(u+v\) tend vers \(-\infty\).
\end{prop}
\begin{prop}
    Soient deux suites réelles \(u\) et \(v\). On suppose que \(u\) tend vers 
    \(+\infty\) et que \(v\) est minorée par un réel strictement positif, au moins 
    à partir d'un certain rang. Alors \(uv\) tend vers \(+\infty\).
\end{prop}
\begin{proof}
    \(v\) est minorée donc il existe un réel \(a>0\) et il existe un naturel 
    \(n_0\) tel que pour tout entier \(n\) si \(n \geqslant n_0\) alors \(v_n 
    \geqslant a > 0\). La suite réelle \(u\) tend vers \(+\infty\), donc pour tout 
    réel \(A\) positif il existe un naturel \(n_1\) tel que pour tout naturel 
    \(n\) si \(n \geqslant n_1\) alors \(u_n \geqslant \frac{A}{a}\). Soit \(n_2 = 
    \max(n_0,n_1)\) alors pour tout naturel \(n\), si \(n \geqslant n_2\) alors 
    \(u_nv_n \geqslant \frac{A}{a} a = A\). Alors \(uv\) tend vers \(+\infty\).
\end{proof}
\begin{prop}
    Soit une suite réelle \(u\), on suppose que \(u\) diverge vers \(+\infty\). 
    Alors il existe un naturel \(n_0\) tel que pour tout naturel \(n\), si \(n 
    \geqslant n_0\) alors \(u_n>0\). La suite réelle \(\left(\frac{1}{u_n}\right)_{n 
\geqslant n_0}\) est de limite nulle.  \end{prop}
\begin{proof}
    Soit \(\epsilon > 0\), il existe un naturel \(n_1\) tel que pour tout entier 
    \(n\) si \(n \geqslant n_1\) alors \(\frac{1}{u_n} \geqslant 
    \frac{1}{\epsilon} > 0\). D'où
    \begin{align}
        \forall n \in \N \quad n \geqslant n_1 \implies 0 < \frac{1}{u_n} \geqslant 
        \epsilon \\
        \forall n \in \N \quad n \geqslant n_1 \implies \abs{\frac{1}{u_n}}= 
        \frac{1}{u_n} \geqslant \epsilon.
    \end{align}
    Donc \(\frac{1}{u}\) tend vers 0.
\end{proof}
\begin{prop}
    Soit une suite réelle \(u\). On suppose que \(u\) est de limite nulle et qu'il 
    existe un naturel \(n_0\) tel que pour tout naturel \(n\) si \(n \geqslant 
    n_0\) alors \(u_n > 0\), alors la suite réelle \(\left(\frac{1}{u_n}\right)_{n 
    \geqslant n_0}\) tend vers \(+\infty\).
\end{prop}
\begin{proof}
    Soit \(A>0\) alors il existe un naturel \(n_1\) tel que pour \(n \geqslant 
    n_1\) \(\abs{u_n} \leqslant \frac{1}{A}\). Soit \(n_2 = \max(n_0,n_1)\) alors 
    pour tout naturel \(n \geqslant n_2\) on a
    \begin{equation}
        0 < u_n = \abs{u_n} \leqslant \frac{1}{A}.
    \end{equation}
    D'où
    \begin{equation}
        0 < A \leqslant \frac{1}{u_n},
    \end{equation}
    donc \(\frac{1}{u}\) tend vers \(+\infty\)
\end{proof}

\subsubsection{Récapitulatif}

\paragraph{Somme}

%\begin{table}[!h]
%  \centering
\begin{center}
    \begin{tabular}{|c|c|c|c|c|}\hline
        \(+\) & \(l'\) & \(+\infty\) & \(-\infty\) & PL \\ \hline
        \(l\) & \(l+l'\) & \(+\infty\) & \(-\infty\) & PL \\ \hline
        \(+\infty\) & \(+\infty\) & \(+\infty\) & FI & FI \\ \hline
        \(-\infty\) & \(-\infty\) & FI & \(-\infty\) & FI \\ \hline
        PL & PL & FI & FI & FI \\ \hline
    \end{tabular}
\end{center}
%\end{table}

avec PL ``pas de limite'' et FI ``forme indéterminée''

\paragraph{Produit}
%\begin{table}[!h]
%  \centering
\begin{center}
    \begin{tabular}{|c|c|c|c|c|c|c|}\hline
        \(\times\) & \(l'>0\) & \(l' = 0\) & \(l'<0\) & \(+\infty\) & \(-\infty\) & 
        PL \\ \hline
        \(l>0\) & \(ll'\) & \(0\) & \(ll'\) & \(+\infty\) & \(-\infty\) & PL \\ 
        \hline
        \(l = 0\) & \(0\) & \(0\) & \(0\) & FI & FI & FI \\ \hline
        \(l<0\) & \(ll'\) & \(0\) & \(ll'\) & \(-\infty\) & \(+\infty\) & PL \\ 
        \hline
        \(+\infty\) & \(+\infty\) & FI & \(-\infty\) & \(+\infty\) & \(-\infty\) & 
        FI \\ \hline
        \(-\infty\) & \(-\infty\) & FI & \(+\infty\) & \(-\infty\) & \(+\infty\) & 
        FI \\ \hline
        PL & PL & FI & PL & FI & FI & FI \\ \hline
    \end{tabular}
\end{center}
%\end{table}

\subsubsection{Compatibilité avec la relation d'ordre}

\begin{prop}
    Soient une suite réelle \(u\) et un réel \(l\). On suppose qu'il existe une 
    suite réelle \(\alpha\) et un naturel \(n_0\) tels que
    \begin{itemize}
        \item \(\forall n \in \N \quad n \geqslant n_0 \implies \abs{u_n-l} 
            \leqslant \alpha_n\);
        \item \(\alpha\) est de limite nulle.
    \end{itemize}
    Alors \(u\) tend vers \(l\).
\end{prop}
\begin{proof}
    Il existe un naturel \(n_0\) tel que pour tout naturel \(n\), si \(n \geqslant 
    n_0\) alors \(\abs{u_n-l} \leqslant \alpha_n\). Pour tout \(\epsilon>0\) il 
    existe un naturel \(n_1\) tel que pour tout naturel \(n\) si \(n\geqslant 
    n_1\) alors \(\abs{\alpha_n} \leqslant \epsilon\). Soit \(n_2 = 
    \max(n_1,n_0)\) alors pour tout naturel \(n\), si \(n \geqslant n_2\) alors 
    \(\abs{u_n-l}\leqslant \abs{\alpha_n}\leqslant \epsilon\). Donc \(u\) tend 
    vers \(l\).
\end{proof}
\begin{prop}[Passage à la limite]
    Soient \(u\) et \(v\) deux suites réelles convergentes et on suppose qu'il 
    existe un naturel \(n_0\) tel que pour tout naturel \(n\) si \(n \geqslant 
    n_0\) alors \(u_n \leqslant v_n\). Alors \(\lim u \leqslant \lim v\).
\end{prop}
\begin{proof}
    Soit un naturel \(n \geqslant n_0\) et soit la suite réelle \(w = v-u\). 
    Alors, au-delà de \(n_0\), la suite réelle \(w\) est positive, \(w_n = 
    \abs{w_n}\). Alors
    \begin{equation}
        \lim w_n = \lim \abs{w_n} = \abs{\lim w_n}.
    \end{equation}
    Alors la limite de \(w\) est positive. On sait aussi d'après la proposition 
    sur la limite d'une somme de suites réelles convergentes que
    \begin{equation}
        \lim w_n = \lim v_n - \lim u_n \geqslant 0.
    \end{equation}
    Finalement
    \begin{equation}
        \lim u_n \leqslant \lim v_n.
    \end{equation}
\end{proof}
\begin{theo}[Théorème des gendarmes]
    Soient \(u\) et \(v\) deux suites réelles convergentes de même limite \(l\) et 
    une troisième suite réelle \(w\) telle qu'il existe un entier \(n_0\) et que 
    pour tout naturel \(n\) si \(n \geqslant n_0\) on ait \(u_n \leqslant w_n 
    \leqslant v_n\). Alors \(w\) converge vers \(l\).
\end{theo}
\begin{proof}
    Soit un entier \(n\), si \(n \geqslant n_0\) alors \(0 \leqslant w_n-u_n 
    \leqslant v_n -u_n\). Puisque \(u\) et \(v\) ont la même limite, \(v-u\) est 
    de limite nulle. Donc pour \(n \geqslant n_0\) \(\abs{w_n-u_n} \leqslant 
    \abs{v_n-u_n}\). Alors par passage à la limite, \(\lim (w-u) \leqslant \lim 
    (v-u)\). Par conséquent, \(w-u\) est convergente de limite nulle. Puisque \(w 
    = (w-u)+u\) avec \(\lim (w-u) = 0\) et \(\lim u = l\) on a \(\lim w =l\).
\end{proof}
\begin{prop}
    Soient \(u\) et \(v\) deux suites réelles telles qu'il existe un naturel 
    \(n_0\) tel que pour tout naturel \(n\) si \(n \geqslant n_0\) alors 
    \(u_n<v_n\). Alors
    \begin{gather}
        \lim u =+\infty \implies \lim v =+\infty,\\
        \lim v =-\infty \implies \lim u =-\infty.
    \end{gather}
\end{prop}
\begin{proof}
    \begin{itemize}
        \item Pour tout réel \(A\) il existe un naturel \(n_1\) tel que pour tout 
            naturel \(n\) si \(n \geqslant n_1\) alors \(u_n \geqslant A\). Soit \(n_2 
            = \max(n_0,n_1)\), donc pour tout naturel \(n\), si \(n \geqslant n_2\) 
            alors \(v_n \geqslant u_n \geqslant A\) donc \(\lim v = +\infty\).
        \item Pour tout réel \(A\) il existe un naturel \(n_1\) tel que pour tout 
            naturel \(n\) si \(n \geqslant n_1\) alors \(v_n \leqslant A\). Soit \(n_2 
            = \max(n_0,n_1)\), donc pour tout naturel \(n\), si \(n \geqslant n_2\) 
            alors \(u_n \leqslant v_n \leqslant A\) donc \(\lim v = +\infty\).
    \end{itemize}
\end{proof}
Dans ce paragraphe bien faire la différence entre~:
\begin{itemize}
    \item des résultats qui permettent de comparer les limites lorsqu'on sait déjà 
        qu'elles existent (passages à la limite dans les inégalités);
    \item des résultats qui permettent de conclure à l'existence d'une limite 
        (gendarmes).
\end{itemize}

\section{Suites réelles monotones, Théorème de Bolzano-Weiertrass}

\subsection{Étude de la convergence des suites réelles monotones}

\begin{defdef}
    Soit une suite réelle \(u\). On dit que \(u\) est
    \begin{itemize}
        \item croissante, si pour tout naturel \(n\) \(u_n \leqslant u_{n+1}\);
        \item décroissante, si pour tout naturel \(n\) \(u_n \geqslant u_{n+1}\);
        \item monotone, si \(u\) est croissante ou décroissante.
    \end{itemize}
\end{defdef}
\begin{defdef}
    Soit une suite réelle \(u\). On dit que \(u\) est
    \begin{itemize}
        \item strictement croissante, si pour tout naturel \(n\) \(u_n < u_{n+1}\);
        \item strictement décroissante, si pour tout naturel \(n\) \(u_n > 
            u_{n+1}\);
        \item strictement monotone, si \(u\) est strictement croissante ou 
            strictement décroissante.
    \end{itemize}
\end{defdef}
\begin{theo}
    Soit une suite réelle \(u\) croissante. Alors
    \begin{itemize}
        \item soit \(u\) est majorée et alors \(u\) converge de limite \(l = 
            \sup\{u_n, n \in \N\}\);
        \item soit \(u\) n'est pas majorée et alors \(u\) tend vers \(+\infty\).
    \end{itemize}
\end{theo}
\begin{proof}
    Supposons, dans un premier temps, que \(u\) soit majorée. Il existe alors un 
    réel \(M\) tel que pour tout naturel \(n\) \(u_n \leqslant M\). Soit \(A = 
    \{u_n, n \in \N\}\) A est une partie non vide et majorée de \(\R\), donc elle 
    admet une borne supérieure qu'on note \(l\). Par caractérisation de la borne 
    supérieure
    \begin{align}
        \forall \epsilon > 0 & \ \exists a \in A \quad l-\epsilon < a \leqslant l \\
        \forall \epsilon > 0 & \ \exists n_0 \in \N \quad l-\epsilon < u_{n_0} 
        \leqslant l \\
        \forall \epsilon > 0 & \ \exists n_0 \in \N \ \forall n \geqslant n_0 \quad 
        l \geqslant u_n \geqslant u_{n_0} > l-\epsilon
    \end{align}
    donc pour tout naturel \(n\) si \(n \geqslant n_0\) alors \(\abs{u_n-l} = 
    l-u_n \leqslant \epsilon\) donc \(\lim u =l\).

    Supposons, dans un deuxième temps, que la suite réelle \(u\) ne soit pas 
    majorée. Alors pour tout réel \(M\) il existe un naturel \(n_0\) tel que 
    \(u_{n_0} \geqslant M\). Or \(u\) est croissante donc pour tout naturel \(n 
    \geqslant n_0\) alors \(u_n \geqslant u_{n_0} \geqslant M\). Donc \(\lim u 
    =+\infty\).
\end{proof}
En appliquant ce théorème à la suite réelle \(-u\) on obtient le théorème 
suivant
\begin{theo}
    Soit une suite réelle \(u\) décroissante. Alors
    \begin{itemize}
        \item soit \(u\) est minorée et elle converge et \(\lim u = \inf\{u_n, n \in 
            \N\}\);
        \item soit \(u\) n'est pas minorée et elle tend vers \(-\infty\).
    \end{itemize}
\end{theo}

\subsection{Suites réelles adjacentes}

\begin{defdef}
    Soient \(u\) et \(v\) deux suites réelles. On dit que \(u\) et \(v\) sont 
    adjacentes si
    \begin{itemize}
        \item \(u\) est croissante;
        \item \(v\) est décroissante;
        \item et \(\lim u-v =0\).
    \end{itemize}
\end{defdef}
\begin{prop}
    Soient \(u\) et \(v\) deux suites réelles adjacentes, alors \(u\) et \(v\) 
    convergent de même limite \(l\), de plus
    \begin{equation}
        \forall n \in \N \quad u_n \leqslant l \leqslant v_n
    \end{equation}
\end{prop}
\begin{proof}
    La suite réelle \(u\) est croissante et la suite réelle \(v\) est 
    décroissante, alors \(u-v\) est croissante. Or \(u-v\) converge vers zéro. 
    Alors \(u-v\) est majorée et
    \begin{equation}
        \sup\enstq{u_n-v_n}{n \in \N} = \lim u-v = 0.
    \end{equation}
    Par conséquent, pour tout naturel \(n\) on a \(u_n-v_n \leqslant 0\) donc 
    \(u_n \leqslant v_n\) or \(v\) décroît par conséquent \(u_n \leqslant v_n 
    \leqslant v_0\). La suite réelle \(u\) est alors croissante et majorée donc 
    elle converge. On note \(l\) sa limite. 

    Ensuite, \(v = (v-u)+u\) et \(v-u\) est de limite nulle et \(u\) tend vers 
    \(l\) donc \(v\) tend vers \(l\). On a montré que \(u\) et \(v\) converge de 
    même limite \(l\). De plus
    \begin{equation}
        l = \sup\enstq{u_n}{n\in \N} = \inf\enstq{v_n}{n\in \N}.
    \end{equation}
    Ainsi
    \begin{equation}
        \forall n \in \N \quad u_n \leqslant l \leqslant v_n.
    \end{equation}
\end{proof}

\subsection{Valeurs décimales approchées d'un réel}

Soit un réel \(x\). Pour tout naturel \(n\), on définit \(P_n = E(10^n x)\). 
\(P_n\) est l'unique entier relatif tel que \(P_n \leqslant 10^n x \leqslant 
p_n+1\). Autrement dit \(\frac{P_n}{10^n} \leqslant x \leqslant 
\frac{P_n+1}{10^n}\).
\begin{defdef}
    Pour tout naturel \(n\), on appelle \begin{itemize}
        \item valeur décimale approchée de \(x\) par défaut à \(10^{-n}\) près le 
            réel \(u_n = \frac{P_n}{10^n}\);
        \item valeur décimale approchée de \(x\) par excès  à \(10^{-n}\) près le 
            réel \(v_n = \frac{P_n+1}{10^n}\).
    \end{itemize}
\end{defdef}
\begin{prop}
    Les suites réelles \(u\) et \(v\) des valeurs décimales approchées de \(x\) 
    par défaut et par excès sont adjacentes et convergent vers \(x\).
\end{prop}
\begin{proof}
    Soit un naturel \(n\), alors \(v_n-u_n = \frac{1}{10^n}\) et \(\lim v-u = 0\). 
    De plus
    \begin{align}
        P_n \leqslant 10^n x \leqslant P_n +1 \\
        10 P_n \leqslant 10^{n+1} x \leqslant 10 P_n +10.
    \end{align}
    \(P_{n+1}\) est l'unique entier tel que
    \begin{equation}
        P_{n+1} \leqslant 10^{n+1} x \leqslant P_{n+1}+1.
    \end{equation}
    \(10P_n\) et \(10P_n +10\) sont des entiers, donc \(\begin{cases} 10 P_n 
    \leqslant P_{n+1} \\ p_{n+1}+1 \leqslant 10 P_n +10\end{cases}\). Alors 
    \begin{equation}
        u_{n+1} = \frac{P_{n+1}}{10^{n+1}} \geqslant \frac{10 P_n}{10^{n+1}} = 
        \frac{P_n}{10^n} = u_n.
    \end{equation}
    La suite réelle \(u\) est donc croissante. Et
    \begin{equation}
        v_{n+1} = \frac{P_{n+1}+1}{10^{n+1}} \leqslant \frac{10 P_n+10}{10^{n+1}} 
        = \frac{P_n+1}{10^n}= v_n.
    \end{equation}
    La suite réelle \(v\) est donc décroissante. Ainsi les suites réelles \(u\) 
    et \(v\) sont adjacentes. Elles admettent donc une même limite notée \(l\). 
    De plus par définition
    \begin{equation}
        u_n \leqslant x \leqslant v_n.
    \end{equation}
    En passant à la limite on a \(l = x\)
\end{proof}
\begin{cor}
    \begin{enumerate}
        \item Tout réel est limite d'une suite réelle de rationnels;
        \item tout réel est limite d'une suite réelle d'irrationnels.
    \end{enumerate}
\end{cor}
\begin{proof}
    \begin{enumerate}
        \item Soit un réel \(x\), la suite réelle \(\left(\frac{E(10^n 
            x)}{10^n}\right)_{n \in \N}\) est une suite réelle de rationnels qui 
            tend vers \(x\);
        \item soit un réel \(y\), pour tout naturel \(n\) on définit \(w_n = 
            \frac{E(10^n y)+\sqrt{2}}{10^n}\) donc \(\lim w =y\). Puisque \(\sqrt{2} 
            \in \R \setminus \Q\) alors \(\forall n \in \N \ w_n \in \R \setminus 
            \Q\).
    \end{enumerate}
\end{proof}
Ce corollaire est en fait une formulation différente de la densité de \(\Q\) 
et \(\R \setminus \Q\) dans \(\R\).

\subsection{Théorème des segments emboîtés}

\begin{theo}[Théorème des segments emboîtés]
    Soit \((I_n = \intervalleff{a_n}{b_n})_{n \in \N}\) une suite réelle 
    décroissante par inclusion de segments non vides dont la longueur tend vers 
    zéro. C'est-à-dire que
    \begin{itemize}
        \item pour tout naturel \(n\), \(\intervalleff{a_{n+1}}{b_{n+1}} \subset 
            \intervalleff{a_{n}}{b_{n}}\);
        \item \(\lim b-a =0\).
    \end{itemize}
    Alors il existe un réel \(l\) tel que \(\bigcap_{n \in \N} I_n = \{l\}\).
\end{theo}
\begin{proof}
    On sait que \(\lim b-a = 0\). L'hypothèse de décroissance par inclusion 
    s'écrit aussi
    \begin{equation}
        \forall n \in \N \quad a_n \leqslant a_{n+1} \leqslant b_{n+1} \leqslant 
        b_n.
    \end{equation}
    La suite réelle \(a\) est croissante et la suite réelle \(b\) est 
    décroissante. Les suites réelles \(a\) et \(b\) sont donc adjacentes. Elle 
    convergent donc toutes les deux vers une limite \(l\) et pour tout naturel 
    \(n\), \(a_n \leqslant l \leqslant b_n\). On a donc montré que les suites 
    réelles sont convergentes de même limite \(l\). Prouvons maintenant que 
    \(\bigcap_{n \in \N} I_n = \{l\}\)

    D'une part, pour tout naturel \(n\) on a \(l \in I_n\) alors \(l \in 
    \bigcap_{n \in \N} I_n\).

    D'autre part soit \(x \in \bigcap_{n \in \N} I_n\) alors pour tout naturel 
    \(n\), \(x \in I_n\) donc \(a_n \leqslant x \leqslant b_n\). Puisque \(a\) 
    et \(b\) sont convergentes de limite \(l\), en passant à la limite dans 
    l'inégalité il vient que \(x = l\). Donc \(\bigcap_{n \in \N} I_n \subset 
    \{l\}\).

    Par conséquent, par double inclusion, \(\bigcap_{n \in \N} I_n = \{l\}\).
\end{proof}

\subsection{Théorème de Bolzano-Weiertrass}

\begin{theo}[Théorème de Bolzano-Weiertrass]
    De toute suite réelle bornée on peut extraire une sous-suite réelle 
    convergente.
\end{theo}
\begin{proof}
    La démonstration s'effectue en deux mouvements principaux. Le premier 
    mouvement va construire par récurrence une suite réelle décroissante de 
    segments emboîtés dont la longueur tend vers zéro. On en conclura grâce au 
    théorème des segments emboîtés que la suite réelle tend vers un réel \(l\). 
    Le deuxième mouvement va construire une sous-suite réelle de \(u\) qui 
    converge vers \(l\) en définissant par récurrence une application \(\varphi 
    : \N \longmapsto \N\) strictement croissante. On montrera que la sous-suite 
    réelle \((u_{\varphi(n)})_{n \in \N}\) converge.

    Soit une suite réelle bornée \(u\). Soient \(m\) un minorant de \(u\) et 
    \(M\) un majorant de \(u\). Alors pour tout naturel \(n\), \(m \leqslant u_n 
    \leqslant M\).

    \emph{Premier mouvement}~:

    Construisons par récurrence une suite réelle de segments emboîtés 
    \((\intervalleff{a_n}{b_n})_{n \in \N}\) telle que
    \begin{itemize}
        \item \(\lim b-a =0\)
        \item pour tout naturel \(n\) l'ensemble \(I_n = \enstq{k \in \N}{u_k \in 
            \intervalleff{a_n}{b_n}}\) soit infini.
    \end{itemize}

    \emph{Attention}, cela signifie que \(u_k \in \intervalleff{a_n}{b_n}\) pour 
    une infinité d'indices \(k\), mais surtout pas que 
    \(\intervalleff{a_n}{b_n}\) contient une infinité de termes de la suites 
    réelles \(u\). Un contre exemple est une suite réelle constante.

    Construction par récurrence~:
    \emph{Initialisation}~: Soit \(a_0 = m\) et \(b_0 = M\) alors pour tout 
    naturel \(k\) \(a_0 \leqslant u_k \leqslant b_0\), alors \(I_0 = \enstq{k 
    \in \N}{a_0 \leqslant u_k \leqslant b_0} = \N\) est infini.

    \emph{Hérédité}~: Supposons avoir construit les segments vérifiant les 
    hypothèses. C'est-à-dire que
    \begin{itemize}
        \item \(\lim b-a =0\)
        \item pour tout naturel \(n\) l'ensemble \(I_n = \enstq{k \in \N}{u_k \in 
            \intervalleff{a_n}{b_n}}\) soit infini.
    \end{itemize}

    Soit deux ensembles \(J_n = \enstq{k \in \N}{u_k \in 
    \intervalleff{a_n}{\frac{a_n+b_n}{2}}}\) et \(K_n = \enstq{k \in \N}{u_k \in 
    \intervalleff{\frac{a_n+b_n}{2}}{b_n}}\). Alors comme \(I_n = J_n \cup K_n\) 
    l'un des deux au moins est forcément infini. Puisque s'ils étaient finis, 
    \(I_n\) serait donc l'union de deux ensembles finis, donc \(I_n\) serait fini 
    (Contradiction avec l'hypothèse de récurrence). On choisit alors comme segment 
    suivant \(I_{n+1}\) l'un des deux segments \(J_n\) ou \(K_n\) qui est infini.

    \emph{Conclusion}~: On dispose donc d'une suite de segments réels 
    \((\intervalleff{a_n}{b_n})_{n \in \N}\) telle que pour tout naturel \(n\)
    \begin{itemize}
        \item la suite de segments \(I\) est décroissante par inclusion \(I_{n+1} 
            \subset I_n\), \(\intervalleff{a_{n+1}}{b_{n+1}} \subset 
            \intervalleff{a_{n}}{b_{n}}\);
        \item \(I_{n+1} = \enstq{k \in \N}{u_k \in 
            \intervalleff{a_{n+1}}{b_{n+1}}}\) est infini;
        \item \(b_{n+1}-a_{n+1} = \frac{1}{2}(b_n-a_n)\).
    \end{itemize}

    La suite réelle \(b-a\) est géométrique de raison \(\frac{1}{2}\) de premier 
    terme \(b_0-a_0 = M-m\). Alors pour tout naturel \(n\), \(b_n-a_n = (M-m) 
    \left(\frac{1}{2}\right)^n\) d'où \(\lim b-a = 0\). Par théorème des segments 
    emboîtés, il existe un réel \(l\) tel que \(\bigcap_{n \in \N} 
    \intervalleff{a_{n}}{b_{n}} = \{l\}\) et \(l = \lim a = \lim b\).


    \emph{Deuxième mouvement}~: On construit une sous-suite réelle de \(u\) qui 
    converge vers \(l\). On définit par récurrence une application \(\varphi : \N 
    \longmapsto \N\) strictement croissante telle que
    \begin{equation}
        \forall n \in \N \quad a_n \leqslant u_{\varphi(n)} \leqslant b_n.
    \end{equation}

    \emph{Initialisation}~: On pose \(\varphi(0) = 0\) alors \(a_0 = m \leqslant 
    u_{\varphi(0)} = u_{0} \leqslant b_0 = M\).

    \emph{Hérédité}~: Supposons avoir construit \(\varphi(0) < \ldots < 
    \varphi(n)\) tels que
    \begin{equation}
        \forall p \in \intervalleentier{0}{n} \quad a_p \leqslant u_{\varphi(p)} 
        \leqslant b_p.
    \end{equation}

    On sait d'après le premier mouvement de la démonstration que l'ensemble 
    \(\enstq{k \in \N}{u_k \in \intervalleff{a_{n+1}}{b_{n+1}}}\) est infini. 
    C'est une partie de \(\N\) infinie donc elle n'est pas majorée.

    Il existe \(k_0 \in \N\) tel que \(u_{k_0}\in 
    \intervalleff{a_{n+1}}{b_{n+1}}\) et \(k_0 > \varphi(n)\). On pose \(k_0 = 
    \varphi(n+1)\). Ainsi \(\varphi(n+1) > \varphi(n)\) et \(u_{\varphi(n+1)} = 
    u_{k_0}\in \intervalleff{a_{n+1}}{b_{n+1}}\).

    \emph{Conclusion}~: On a défini une application \(\varphi : \N \longmapsto 
    \N\) strictement croissante telle que pour tout naturel \(n\), 
    \(u_{\varphi(n)} \in \intervalleff{a_{n}}{b_{n}}\). La suite réelle 
    \((u_{\varphi(n)})_{n \in \N}\) est une suite réelle extraite de \(u\) et pour 
    tout naturel \(n\) \(a_n \leqslant u_{\varphi(n)} \leqslant b_n\). On sait que 
    les suites réelles \(a\) et \(b\) convergent de même limite \(l\). On déduit 
    du théorème des gendarmes que \((u_{\varphi(n)})_{n \in \N}\) converge vers 
    \(l\).
\end{proof}

Le procédé utilisé ici s'appelle la dichotomie. Il permet de prouver l'existence 
d'une suite réelle extraite qui converge mais ne donne pas de méthode pratique 
pour en trouver une.

\section{Relations de comparaison}

\subsection{Relation de domination}

\begin{defdef}
    Soient \(u\) et \(v\) deux suites réelles. On dit que \(u\) est dominée par 
    \(v\) et on écrit \(u = \grandO{v}\) ou encore \(u_n = \grandO{v_n}\) lorsque 
    \(n\) tend vers l'infini si et seulement s'il existe une suite réelle \(w\) 
    bornée et un naturel \(n_0\) tels que pour tout naturel \(n\) si \(n \geqslant 
n_0\) alors \(u_n = v_n w_n\).  \end{defdef}
\begin{prop}
    La définition de la relation de domination est équivalente à : soient deux 
    suites réelles \(u\) et \(v\), alors
    \begin{equation}
        u = \grandO{v} \iff \exists M \in \Rpluss \ \exists n_0 \in \N \ \forall n 
        \in \N \quad n \geqslant n_0 \implies \abs{u_n} \leqslant M \abs{v_n}.
    \end{equation}
\end{prop}
\begin{proof}
    Si \(u = \grandO{v}\) il existe une suite réelle bornée \(w\) telle qu'à 
    partir d'un certain rang \(n_0\) pour tout naturel \(n\geqslant n_0\) on a 
    \(u_n = v_n w_n\). Alors il existe un réel \(M\) positif tel que pour tout 
    entier \(n\) \(\abs{w_n} \leqslant M\). Alors pour tout entier \(n \geqslant 
    n_0\), \(\abs{u_n}\leqslant M \abs{v_n}\). Construisons la suite \(w\)~:
    \begin{itemize}
        \item Si \(n < n_0\) on pose \(w_n = 0\).  \item Si \(n \geqslant n_0\) et 
            si \(v_n \neq 0\) on pose \(w_n = \frac{u_n}{v_n}\) ainsi \(u_n = v_n 
            w_n\).
        \item Si \(n \geqslant n_0\) et si \(v_n = 0\) alors \(\abs{u_n} \leqslant M 
            \abs{v_n}\) équivaut à \(u_n = 0\), donc on pose \(w_n = 0\) ainsi \(u_n = 
            v_n w_n\).
    \end{itemize}
    On a construit une suite réelle \(w\) telle que pour tout \(n \in \N\) si \(n 
    \geqslant n_0\) alors \(u_n = v_n w_n\). De plus \(w\) est bornée puisque pour 
    tout naturel \(n\)~:
    \begin{itemize}
        \item si \(n < n_0\) alors \(w_n = 0\) \item si \(n \geqslant n_0\) et \(v_n 
        = 0\) alors \(w_n = 0\) \item si \(n \geqslant n_0\) et \(v_n \neq 0\) 
            alors \(\abs{w_n} = \abs{\frac{u_n}{v_n}} \leqslant M\).
    \end{itemize}
    Dans tous les cas, la suite réelle \(w\) est bornée, donc \(u = \grandO{v}\).
\end{proof}
\begin{prop}
    Soient \(u\) et \(v\) deux suites réelles. On suppose que \(v\) ne s'annule 
    pas, alors \(u\) est dominée par \(v\) si et seulement si \(\frac{u}{v}\) est 
    bornée.
\end{prop}
\begin{proof}
    La suite réelle \(v\) ne s'annule pas donc \(u\) est dominée par \(v\) si et 
    seulement s'il existe un réel positif \(M\) et un entier \(n_0\) tel que pour 
    tout entier \(n\) si \(n \geqslant n_0\) alors \(\abs{u_n} \leqslant M 
    \abs{vn}\), et comme \(v\) ne s'annule pas, c'est équivalent à 
    \(\abs{\frac{u_n}{v_n}} \leqslant M\). Ce qui veut dire que la suite réelle 
    \(\frac{u}{v}\) est bornée à partir d'un certain rang \(n_0\) ce qui équivaut 
    à dire que la suite réelle \(\frac{u}{v}\) est bornée pour tout les rangs.
\end{proof}

En particulier \(u = \grandO{1} \iff u\) est bornée.

\begin{prop}[Règles de calculs]
    Soient quatre suites réelles \(u_1\), \(u_2\), \(v_1\) et \(v_2\), alors
    \begin{gather}
        u_1 = \grandO{v_1} \text{~et~} u_2 = \grandO{v_1} \implies u_1+u_2 = 
        \grandO{v_1}\\
        u_1 = \grandO{v_1} \text{~et~} u_2 = \grandO{v_2} \implies u_1u_2 = 
        \grandO{v_1 v_2}\\
        u_1 = \grandO{v_1} \text{~et~} v_1 = \grandO{v_2} \implies u_1 = 
        \grandO{v_2}.
    \end{gather}
\end{prop}
\begin{proof}
    \begin{enumerate}
        \item Il existe deux suites bornées \(w_1\), \(w_2\) et deux naturels 
            \(n_1\) et \(n_2\) tels que pour tout naturel \(n\)
            \begin{align}
                n \geqslant n_1 \implies u_{1,n} = v_{1,n}w_{1,n} \\
                n \geqslant n_2 \implies u_{2,n} = v_{1,n}w_{2,n}.
            \end{align}
            Soit \(n_0 = \max(n_1,n_2)\) donc si \(n \geqslant n_0\) alors \(u_{1,n} + 
            u_{2,n}= v_{1,n} (w_{2,n}+w_{1,n})\). La suite \(w_{1} + w_{2}\) est 
            bornée donc \(u_1+u_2 = \grandO{v_1}\).
        \item Il existe deux suites bornées \(w_1\), \(w_2\) et deux naturels 
            \(n_1\) et \(n_2\) tels que pour tout naturel \(n\)
            \begin{align}
                n \geqslant n_1 \implies u_{1,n} = v_{1,n}w_{1,n} \\
                n \geqslant n_2 \implies u_{2,n} = v_{1,n}w_{2,n},
            \end{align}
            alors pour tout naturel \(n \geqslant \max(n_1,n_2)\) on a
            \begin{equation}
                u_{1,n} u_{2,n} = v_{1,n} v_{2,n} w_{1,n} w_{2,n}.
            \end{equation}
            La suite \(w_1 w_2\) est le produit de deux suites bornées donc elle est 
            bornée, alors \(u_1u_2 = \grandO{v_1v_2}\).
        \item Il existe deux suites bornées \(w_1\), \(w_2\) et deux naturels 
            \(n_1\) et \(n_2\) tels que pour tout naturel \(n\) \begin{align}
                n \geqslant n_1 \implies u_{1,n} = v_{1,n}w_{1,n} \\
            n \geqslant n_2 \implies v_{1,n} = u_{2,n}w_{2,n}, \end{align}
            alors pour tout naturel \(n \geqslant \max(n_1,n_2)\) on a
            \begin{equation}
                u_{1,n} = (v_{2,n} w_{2,n})w_{1,n} = (w_{2,n} w_{1,n}) v_{2,n}.
            \end{equation}
            La suite \(w_1 w_2\) est bornée (comme étant le produit de deux suites 
            bornées) donc \(u_1 = \grandO{v_2}\).
    \end{enumerate}
\end{proof}

\subsection{Relation de négligeabilité}

\begin{defdef}
    Soient \(u\) et \(v\) deux suites réelles, on dit que \(u\) est négligeable 
    devant \(v\) et on note \(u = \petito{v}\) lorsque \(n\) tend vers l'infini 
    s'il existe une suite \(w\) de limite nulle et un naturel \(n_0\) à partir 
    duquel on ait pour tout naturel \(n \geqslant n_0\) \(u_n = v_n w_n\).
\end{defdef}
\begin{prop}
    La définition de la relation de négligeabilité est équivalente à : soient deux 
    suites réelles \(u\) et \(v\) telles que
    \begin{equation}
        u = \petito{v} \iff \forall \epsilon > 0 \ \exists n_0 \in \N \ \forall n 
        \in \N \quad n \geqslant n_0 \implies \abs{u_n} \leqslant \epsilon 
        \abs{v_n}.
    \end{equation}
\end{prop}
\begin{proof}
    Supposons que \(u = \petito{v}\) alors il existe un naturel \(n_0\) tel que 
    pour tout entier, si \(n \geqslant n_0\) alors \(u_n = v_n w_n\). La suite 
    \(w\) tend vers zéro donc
    \begin{equation}
        \forall \epsilon > 0 \ \exists n_1 \in \N \ \forall n \in \N \ n \geqslant 
        n_1 \implies \abs{w_n} \leqslant \epsilon.
    \end{equation}
    Alors pour tout naturel \(n\), si \(n \geqslant \max(n_0, n_1)\) alors 
    \(\abs{u_n} = \abs{v_n w_n} \leqslant \epsilon \abs{v_n}\).

    D'autre part, pour tout naturel \(n\), on définit la suite \(w\)~:
    \begin{equation}
        \begin{cases}
            w_n = \frac{u_n}{v_n} & v_n \neq 0 \\ w_n = 0 & v_n = 0
        \end{cases}.
    \end{equation}
    Avec \(\epsilon = 1\) il existe un naturel \(n_0\) tel que pour tout entier 
    \(n\), si \(n \geqslant n_0\) alors \(\abs{u_n} \leqslant \abs{v_n}\). 
    Montrons que \(w\) tend vers zéro. On sait que
    \begin{align}
        & \forall \epsilon > 0 \ \exists n_1 \in \N \ \forall n \in \N \quad n 
        \geqslant n_1 \implies \abs{u_n} \leqslant \epsilon \abs{v_n} \\ \iff & 
        \forall \epsilon > 0 \ \exists n_1 \in \N \ \forall n \in \N \quad n 
        \geqslant n_1 \implies \abs{w_n}= \begin{cases} 0 \leqslant \epsilon & v_n 
        = 0 \\ \abs{\frac{u_n}{v_n}} \leqslant \epsilon & v_n\neq 0 \end{cases} \\
        \iff & \forall \epsilon > 0 \ \exists n_1 \in \N \ \forall n \in \N 
        \quad n \geqslant n_1 \implies \abs{w_n} \leqslant \epsilon.
    \end{align}
    Alors \(w\) est de limite nulle donc \(u = \petito{v}\). Les définitions 
    sont équivalentes.
\end{proof}
\begin{prop}
    Soient \(u\) et \(v\) deux suites réelles, on suppose que \(v\) ne 
    s'annule pas. Alors
    \begin{equation}
        u = \petito{v} \iff \lim \frac{u}{v} = 0.
    \end{equation}
\end{prop}
\begin{proof}
    \begin{align}
        u = \petito{v} & \iff \forall \epsilon > 0 \ \exists n_0 \in \N \ 
        \forall n \in \N \quad n \geqslant n_0 \implies \abs{u_n} \leqslant 
        \epsilon \abs{v_n} \\
        & \iff \forall \epsilon > 0 \ \exists n_0 \in \N \ 
        \forall n \in \N \quad n \geqslant n_0 \implies 
        \abs{\frac{u_n}{v_n}} \leqslant \epsilon \\
        & \iff \lim \frac{u}{v} =0.
    \end{align}
\end{proof}
En particulier \(u = \petito{1} \iff \lim u =0\).
\begin{prop}
    Soient \(u\) et \(v\) deux suites réelles, si \(u = \petito{v}\) alors \(u 
    = \grandO{v}\).
\end{prop}
\begin{proof}
    Cela découle des deux définitions, puisqu'une suite de limite nulle est a 
    fortiori bornée.
\end{proof}
\begin{prop}
    Soient quatre suites réelles \(u_1, u_2, v_1\) et \(v_2\). Alors
    \begin{gather}
        u_1=\petito{v_1}\text{~et~}u_2 = \petito{v_1}\implies u_1+u_2 = 
        \petito{v_1};\\
        u_1=\petito{v_1}\text{~et~}u_2 = \grandO{v_2}\implies u_1u_2 = 
        \petito{v_1 v_2};\\
        u_1=\petito{v_1}\text{~et~}u_2 = \petito{v_2}\implies u_1u_2 = 
        \petito{v_1 v_2};\\
        u_1=\grandO{v_1}\text{~et~}v_1 = \grandO{v_2}\implies u_1 = 
        \grandO{v_2};\\
        u_1=\petito{v_1}\text{~et~}v_1 = \grandO{v_2}\implies u_1 = 
        \petito{v_2};\\
        u_1=\grandO{v_1}\text{~et~}v_1 = \petito{v_2}\implies u_1 = 
        \petito{v_2};\\
        u_1=\petito{v_1}\text{~et~}v_1 = \petito{v_2}\implies u_1 = 
        \petito{v_2}.
    \end{gather}
\end{prop}
\begin{proof}
    La preuve est presque identique à la preuve pour la relation de 
    domination.
    \begin{itemize}
        \item idem. La somme de deux suites de limites nulle est une suite de 
            limite nulle.
        \item Il existe une suite \(w_1\) de limite nulle et un naturel \(n_1\) 
            tel que si \(n \geqslant n_1\) alors \(u_{1,n} = v_{1,n} w_{1,n}\). Il 
            existe une suite \(w_2\) bornée et un naturel \(n_2\) tel que si \(n 
            \geqslant n_2\) alors \(u_{2,n} = v_{2,n} w_{2,n}\). Pour tout naturel 
            \(n \geqslant \max(n_1,n_2)\) on a \(u_{1,n}u_{2,n} = (w_{1,n}w_{2,n}) 
            v_{1,n}v_{2,n}\) Or \(w_{1,n}w_{2,n}\) est une suite de limite nulle 
            (puisque c'est le produit d'une suite bornée par une suite de limite 
            nulle), donc \(u_1u_2 = \petito{v_1 v_2}\).
        \item idem pour les autres points.
    \end{itemize}
\end{proof}
\begin{prop}
    Soient \(u\) et \(v\) deux suites réelles. Alors
    \begin{itemize}
        \item si \(u = \grandO{v}\) et si \(\lim v = 0\) alors \(\lim u = 0\),
        \item si \(u = \petito{v}\) et si \(v\) est bornée alors \(\lim u = 0\).
    \end{itemize}
\end{prop}
\begin{proof}
    \begin{itemize}
        \item \(u = \grandO{v}\) et \(v = \petito{1}\) donc 
            \(u=\grandO{\petito{1}} = \petito{1}\), alors \(\lim u =0\),
        \item \(u = \petito{v}\) et \(v = \grandO{1}\) donc 
            \(u=\petito{\grandO{1}} = \petito{1}\), alors \(\lim u =0\).
    \end{itemize}
\end{proof}

\emph{Attention}, les notations \(u = \grandO{0}\) et \(u = \petito{0}\) 
signifient toutes les deux que \(u\) est nulle à partir d'un certain rang. 
Ce qui ne se produit que très rarement. Il ne sera pas autorisé d'écrire 
cette équivalence.

\subsection{Relation d'équivalence}

\begin{defdef}
    Soient \(u\) et \(v\) deux suites réelles. On dit que \(u\) est 
    équivalente à \(v\) et on note \(u \sim v\) en l'infini s'il existe une 
    suite \(w\) qui tend vers \(1\) et un naturel \(n_0\) tel que pour tout 
    naturel \(n\), si \(n \geqslant n_0\) alors \(u_n =w_n v_n\).
\end{defdef}

Il s'agit d'une relation reflexive. Si \(u\) est équivalente à \(v\), alors 
\(v\) est équivalente à \(u\). En effet \(w\) tend vers \(1\), alors il 
existe un naturel \(n_1\) tel que pour tout naturel \(n\), \(n \geqslant 
n_1\) implique \(w_n >0\). Donc pour tout naturel \(n \geqslant 
\max(n_0,n_1)\), \(v_n = \frac{1}{w_n} u_n\) et la suite \(\frac{1}{w}\) 
tend vers 1. On dira que \(u\) et \(v\) sont équivalentes.

\begin{prop}
    Soient \(u\) et \(v\) deux suite réelles. Alors
    \begin{equation}
        u \sim v \iff u-v = \petito{v}.
    \end{equation}
\end{prop}
\begin{proof}
    \begin{align}
        u \sim v & \iff \exists w \in \R^\N \ \lim w =1 \ \exists n_0 \in \N \ 
        \forall n \in \N \quad n \geqslant n_0 \implies u_n= v_n w_n \\
        & \iff \exists w \in \R^\N \ \lim w =1 \ \exists n_0 \in \N \ 
        \forall n \in \N \quad n \geqslant n_0 \implies u_n-v_n= v_n 
        (w_n -1) \\
        & \iff \exists z \in \R^\N \ \lim z =0 \ \exists n_0 \in \N \ 
        \forall n \in \N \quad n \geqslant n_0 \implies u_n-v_n= v_n 
        z_n \\
        u \sim v & \iff u-v = \petito{v}.
    \end{align}
\end{proof}
\begin{prop}
    Soient \(u\) et \(v\) deux suites réelles. On suppose que \(v\) ne 
    s'annule pas, alors
    \begin{equation}
        u \sim v \iff \lim \frac{u}{v} = 1.
    \end{equation}
\end{prop}
\begin{proof}
    La suite \(v\) ne s'annule pas, donc
    \begin{align}
        u \sim v & \iff u-v = \petito{v} \\
        & \iff \lim \frac{u-v}{v} =0 \\
        & \iff \lim \frac{u}{v} =1.
    \end{align}
\end{proof}
En particulier pour tout réel \(l\) non nul, \(u \sim l\) signifie que 
\(\lim u =l\). Cependant \(u \sim 0\) signifie que \(u\) est nulle à partir 
d'un certain rang, ce qui n'arrive que très rarement. En bref, on n'écrit 
jamais \(\sim 0\).
\begin{prop}
    Soient quatre suites réelles \(u_1, u_2, v_1, v_2\).
    \begin{enumerate}
        \item La relation d'équivalence est transitive~: si \(u_1 \sim u_2\) et 
        \(u_2 \sim v_1\) alors \(u_1 \sim v_1\);  \item La relation 
            d'équivalence conserve le produit~: si \(u_1 \sim u_2\) et \(v_1 \sim 
            v_2\) alors \(u_1 v_1 \sim u_2 v_2\);
        \item La relation d'équivalence conserve le quotient~: si \(u_1 \sim 
            u_2\) et \(v_1 \sim v_2\) et si \(v_1\) et \(v_2\) ne s'annulent pas 
            (au moins à partir d'un certain rang) alors \(\frac{u_1}{v_1} \sim 
            \frac{u_2}{v_2}\);
        \item La relation d'équivalence conserve les puissance~: si \(u_1 \sim 
            u_2\) et si elles sont strictement positives (au moins à partir d'un 
            certain rang) alors pour tout réel \(\alpha\) \(u_1^\alpha \sim 
            u_2^\alpha\).
    \end{enumerate}
\end{prop}
\begin{proof}
    \begin{enumerate}
        \item Il existe deux suites \(w_1\) et \(w_2\) de limite égale à \(1\) 
            et deux entiers \(n_1\) et \(n_2\) tels que pour tout entier naturel 
            \(n\)
            \begin{align}
                n \geqslant n_1 \implies u_{1,n} = w_{1,n} u_{2,n}, \\
                n \geqslant n_2 \implies u_{2,n} = w_{2,n} v_{2,n}.
            \end{align}
            Donc pour tout \(n \geqslant \max(n_1,n_2)\) on a \(u_{1,n} = 
            (w_{1,n}w_{2,n}) v_{1,n}\). La suite \(w_1 w_2\) tend vers \(1\) donc 
            \(u_1 \sim v_1\).
        \item Il existe deux suites \(w_1\) et \(w_2\) de limite égale à \(1\) 
            et deux entiers \(n_1\) et \(n_2\) tels que pour tout entier naturel 
            \(n\)
            \begin{align}
                n \geqslant n_1 \implies u_{1,n} = w_{1,n} u_{2,n}, \\
                n \geqslant n_2 \implies u_{2,n} = w_{2,n} v_{2,n}.
            \end{align}
            Donc pour tout \(n \geqslant \max(n_1,n_2)\) on a \(u_{1,n}v_{1,n} = 
            (w_{1,n}w_{2,n}) u_{2,n}v_{2,n}\). La suite \(w_1 w_2\) tend vers 
            \(1\) donc \(u_1 v_1\sim u_2 v_2\).
        \item idem pour les autres points.
    \end{enumerate}
\end{proof}

On peut faire des produits et des quotients d'équivalents, mais on ne peut 
pas faire des sommes, des différences ou des composition (par exponentielle 
ou logarithme par exemple). Cependant, on peut démontrer quelque 
propositions.
\begin{prop}
    Soient \(u\) et \(v\) deux suites réelles telles que \(u = \petito{v}\) 
    alors \(u+v \sim v\).
\end{prop}
\begin{proof}
    On a \((u+v)-v=u = \petito{v}\) donc \(u+v \sim v\).
\end{proof}
\begin{prop}
    Étant données deux suites réelles \(u\) et \(v\) équivalentes, si \(v\) 
    tend vers une limite \(l\) réelle ou infinie, alors \(u\) admet la même 
    limite \(l\).
\end{prop}
\begin{proof}
    Puisque \(u\) et \(v\) sont équivalentes, il existe une suite réelle \(w\) 
    de limite égale à \(1\) et un naturel \(n_0\) tels que pour tout naturel 
    \(n\) si \(n \geqslant n_0\) alors \(u_n = w_n v_n\) En passant à la 
    limite, on obtient \(\lim u = \lim w \lim v =l\).
\end{proof}
\begin{prop}
    Soient \(u\) et \(v\) deux suites réelles équivalentes. Alors
    \begin{enumerate}
        \item si \(v\) ne s'annule pas à partir d'un certain rang alors \(u\) ne 
            s'annule pas à partir d'un certain rang;
        \item à partir d'un certain rang \(u\) et \(v\) sont de même signe.
    \end{enumerate}
\end{prop}
\begin{proof}
    Il existe une suite \(w\) de limite égale à \(1\) telle qu'il existe un 
    naturel \(n_0\) et si pour tout naturel \(n\) \(n \geqslant n_0\) alors 
    \(u_n = v_n w_n\). Comme \(w\) tend vers \(1\), il existe un naturel 
    \(n_1\) à partir duquel \(w\) est strictement positive. Donc pour tout 
    naturel \(n\), si \(n \geqslant \max(n_0,n_1)\) alors \(u_n = w_n v_n\) 
    avec \(w_n >0\). On en déduit les deux points de la proposition.
\end{proof}

La relation d'équivalence est symétrique et transitive. Pour toute suite 
réelle \(u\), on peut écrire que \(u \sim u\), elle est donc réflexive. 
C'est donc une \emph{vraie} relation d'équivalence.

\section{Suites de référence}

\subsection{Suites géométriques, arithmétiques \& 
arithmético-géo\-métriques}

\subsubsection{Suites géométriques}

\begin{defdef}
    Soit une suite réelle \(u\). S'il existe un réel \(r\) telle que pour tout 
    naturel \(n\) \(u_{n+1} = r u_n\) alors la suite \(u\) est dite 
    géométrique de raison \(r\).
\end{defdef}
\begin{prop}
    Soit une suite géométrique \(u\) de raison \(r\). Alors pour tout naturel 
    \(n\)
    \begin{align}
        u_n &= u_0 r^n \\ S_n &=\sum_{k = 0}^n u_k = \begin{cases} (n+1) u_0 & r 
        = 1 \\ u_0 \frac{1-r^{n+1}}{1-r} & r \neq 1 \end{cases}.
    \end{align}
\end{prop}
\begin{prop}[Convergence d'une suite géométrique]
    Soit une suite réelle \(u\) géométrique de raison \(r\).
    \begin{enumerate}
        \item si \(\abs{r} < 1\) alors \(u\) est de limite nulle;
        \item si \(r = 1\), alors \(u\) est constante égale à \(u_0\);
        \item si \(r = -1\) et \(u_0 \neq 0\) alors \(u\) est divergente de 
            deuxième espèce (limite indéterminée) et \(\abs{u}\) est constante 
            égale à \(u_0\);
        \item si \(\abs{r}>1\) et \(u_0 \neq 0\) alors \(\abs{u}\) est 
            divergente de première espèce \(\lim \abs{u} = +\infty\).
    \end{enumerate}
\end{prop}

Pour le quatrième cas, la suite \(u\) n'admet pas forcément de limite.
\begin{prop}[Convergence d'une série géométrique]
    Soit une suite \(u\) géométrique de raison \(r\) et \(S\) la suite 
    définie comme étant la somme partielle de \(u\). Alors
    \begin{enumerate}
        \item si \(\abs{r} < 1\) la série converge et \(\lim S = 
            \frac{u_0}{1-r}\);
        \item si \(\abs{r} \geqslant 1\) et \(u_0 \neq 0\) la série diverge.
    \end{enumerate}
\end{prop}
\begin{proof}
    \begin{enumerate}
        \item \((S_n)\) converge puisque \(\lim\limits_{n \to \infty} r^{n+1} 
            = 0\);
        \item \begin{itemize} \item si \(\abs{r}>1\) alors \(\lim_{n \to 
                    \infty} \abs{r}^{n+1} = +\infty\) donc \((S_n)\) diverge;
                \item si \(r = 1\) alors \(\forall n \in \N \ S_n = (n+1)u_0\) donc 
                    \((S_n)\) diverge;
                \item si \(r = -1\), comme \((-1)^{n+1}\) diverge, alors \((S_n)\) 
                    diverge.
        \end{itemize}
    \end{enumerate}
\end{proof}

\subsubsection{Suites arithmétiques}

\begin{defdef}
    Soit une suite réelle \(u\). On dit que \(u\) est arithmétique s'il existe 
    un réel \(r\) tel que pour tout naturel \(n\), \(u_{n+1} = u_n +r\). La 
    suite \(u\) est dite arithmétique de raison \(r\).
\end{defdef}
\begin{prop}
    Soit une suite \(u\) arithmétique de raison \(r\). Alors pour tout entier 
    naturel \(n\)
    \begin{align}
        u_n &= u_0+nr \\
        S_n &=\sum_{k = 0}^n u_k = (n+1)u_0 + \frac{n(n+1)}{2}r.
    \end{align}
\end{prop}
\begin{prop}[Convergence d'une suite arithmétique]
    Soit une suite \(u\) arithmétique de raison \(r\). Alors
    \begin{enumerate}
        \item si \(r = 0\) alors \(u\) est constante égale à \(u_0\);
        \item si \(r>0\) alors \(\lim u =+\infty\);
        \item si \(r<0\) alors \(\lim u =-\infty\).
    \end{enumerate}
\end{prop}

Les séries arithmétiques ne convergent pas.

\subsubsection{Suites arithmético-géométriques}

\begin{defdef}
    Soit une suite réelle \(u\). On dit que \(u\) est arithmético-géométrique 
    s'il existe deux entiers \(a\) et \(b\) tels que pour tout naturel \(n\) 
    \(u_{n+1} = au_n +b\)
\end{defdef}

Si \(a = 1\), alors \(u\) est arithmétique de raison \(b\). Si \(b = 0\) 
alors \(u\) est géométrique de raison \(a\).

\begin{prop}
    Si \(a\neq 1\) il existe une unique réel \(\alpha\) tel que \(\alpha = 
    a\alpha+b\). %C'est \(\alpha = \frac{b}{1-a}\). 
    % La suite \(u-\alpha\) est géométrique de raison \(a\). En effet, soit 
    % \(n \in \N\), alors
    %   \begin{align}
    %     (u-\alpha){n+1} = u_{n+1}-\alpha & = au_n+b - \frac{b}{1-a} \\
    %     & = a\left(u_n - \frac{b}{1-a}\right)
    %     & = a(u-\alpha)_n.
    %   \end{align}
    %   Ainsi, le terme général s'exprime pour tout naturel \(n\), 
    %   \(u_n-\alpha = (u_0-\alpha)a^n\). Donc
    \begin{equation}
        \forall n \in \N \quad u_n = \alpha + a^{n}(u_0-\alpha).
    \end{equation}
\end{prop}
\begin{proof}
    Puisque \(a \neq 1\) on a \(\alpha = \frac{b}{1-a}\). Soit \(v\), la suite 
    définie comme
    \begin{equation}
        \forall n \in \N \quad v_n = u_n -\alpha.
    \end{equation}
    Alors
    \begin{equation}
        \forall n \in \N \quad v_{n+1} = u_{n+1}-\alpha = a u_n +b - a\alpha -b 
        = a(u_n-\alpha) = a v_n.
    \end{equation}
    Donc \(v\) est géométrique de raison \(a\), ainsi pour tout naturel \(n\), 
    on a \(u_n = \alpha + (u_0-\alpha)a^n\).
\end{proof}

\subsubsection{Suites récurrentes}

Elles sont définies par une relation de récurrence de la forme \(u_{n+1} = 
f(u_n)\). Elles seront étudiées plus tard dans le cours.

\subsection{Comparaison des suites de référence}

\begin{prop}
    Soient \(\alpha\) et \(\beta\) deux réels tels que \(\alpha < \beta\) 
    alors \(n^\alpha = \petito{n^\beta}\)
\end{prop}
\begin{proof}
    \begin{equation}
        \forall n \geqslant 1, \ n^\beta \neq 0, \quad \frac{n^\alpha}{n^\beta} 
        = \frac{1}{n^{\beta-\alpha}} \underset{n\to \infty}{\longrightarrow} 0
    \end{equation}
\end{proof}

\begin{prop}
    Soient \(\alpha\) et \(\beta\) deux réels avec \(\alpha>0\), alors 
    \(\ln^\beta n = \petito{n^\alpha}\)
\end{prop}
\begin{proof}
    \begin{equation}
        \forall n \geqslant 1, \ n^\alpha \neq 0, \quad \frac{\ln^\beta 
        n}{n^\alpha} \underset{n\to \infty}{\longrightarrow} 0.
    \end{equation}
    En effet
    \begin{equation}
        \begin{cases}
            \beta \neq 0 & \frac{\ln^\beta n}{n^\alpha} = \frac{1}{\ln^{-\beta} n 
            n^\alpha} \underset{n\to \infty}{\longrightarrow} 0 \\
            \beta > 0 & \frac{\ln^\beta n}{n^\alpha} = \left(\frac{\ln 
            n}{n^{\frac{\alpha}{\beta}}}\right)^\beta = 
            \left(\frac{\frac{\beta}{\alpha}\ln 
            n^{\alpha/\beta}}{n^{\frac{\alpha}{\beta}}}\right)^\beta 
            \underset{n\to \infty}{\longrightarrow} 0
        \end{cases}.
    \end{equation}
\end{proof}

\begin{prop}
    Pour tout réel \(a\), \(a^n = \petito{n!}\).
\end{prop}
\begin{proof}
    Soit un naturel \(n\), alors puisque \(n! \neq 0\) on écrit 
    \(\frac{a^n}{n!}=\prod_{k = 1}^n \left(\frac{a}{k}\right)\). Soit un 
    naturel \(N\) tel que \(N \geqslant \abs{a}\) alors pour tout 
    naturel \(n \geqslant N\)
    \begin{equation}
        \frac{a^n}{n!} = \frac{a^N}{N!} \prod_{k = N+1}^n 
        \left(\frac{a}{k}\right),
    \end{equation}
    et on a
    \begin{equation}
        \abs{\prod_{k = N+1}^n \left(\frac{a}{k}\right)} = \prod_{k = 
        N+1}^n \left(\frac{\abs{a}}{k}\right).
    \end{equation}
    Pour tout naturel \(k\) tel que \(N+1 \leqslant k \leqslant n\) ou 
    \(\frac{1}{n} \leqslant \frac{1}{k} \leqslant \frac{1}{N+1}\) on a
    \begin{equation}
        \abs{\prod_{k = N+1}^n \left(\frac{a}{k}\right)} \leqslant 
        \prod_{k = N+1}^n \frac{\abs{a}}{N+1} = 
        \left(\frac{\abs{a}}{N+1}\right)^{n-N}.
    \end{equation}
    Alors du coup
    \begin{equation}
        \forall n \in \N \quad \abs{\frac{a^n}{n!}} \leqslant 
        \frac{a^N}{N!} \left(\frac{\abs{a}}{N+1}\right)^{n-N}.
    \end{equation}
    De plus \(0 \leqslant \frac{\abs{a}}{N+1} < 1\) donc 
    \(\left(\frac{a^N}{N!} 
    \left(\frac{\abs{a}}{N+1}\right)^{n-N}\right)_{n \geqslant N}\) 
    converge de limite nulle. Par théorème des gendarmes 
    \(\left(\frac{a^n}{n!}\right)_{n\in \N}\) tend vers zéro.
\end{proof}

\subsection{Exemples d'équivalents}

Soit une suite réelle \(u\) telle que \(\lim u = 0\). On suppose que 
\(u\) n'est pas stationnaire à zéro. Alors~:
\begin{itemize}
    \item \(\ln(1+u_n) \sim_\infty u_n\). En effet, puisque 
        \(\frac{\ln(1+u_n)}{u_n}= \frac{\ln(1+u_n) - \ln(1+0)}{u_n-0} \to 
    \frac{1}{1+0} =1\) par taux d'accroissement.  \item \(\e^{u_n}-1 
        \sim_\infty u_n\), puisqu'on sait aussi que \(\lim\limits_{x \to 
        0} \frac{\e^{x}-1}{x} = 1\).
    \item Soit \(\alpha \neq 0\), alors \((1+u_n)^\alpha \sim \alpha 
        u_n\). Si on pose 
        \(\fonction{f}{\intervalleoo{-1}{1}}{\R}{x}{(1+x)^\alpha}\), alors 
        \(f\) est dérivable et pour tout réel \(x\) de 
        \(\intervalleoo{-1}{1}\), \(f'(x) = \alpha(1+x)^{\alpha -1}\). Et 
        on a \(f'(0) = \alpha = \lim\limits_{x \to 
        0}\frac{(1+x)^\alpha-1}{x}\).
\end{itemize}

\section{Brève extension aux suites complexes}

\subsection{Notion de suite à valeurs dans \(\C\)}

\begin{defdef}
    On appelle suite complexe ou suite à valeur complexe toute famille 
    de complexe indexée par \(\N\) c'est à dire une application de 
    \(\N\) dans \(\C\). On note \(\C^\N\) leur ensemble.
\end{defdef}
\begin{defdef}
    Soit \(u\) une suite complexe, on lui associe les suites suivantes : 
    \(\abs{u}\), \(\Re(u)\), \(\Im(u)\) et \(\bar{u}\).
\end{defdef}
\begin{defdef}
    Soit une suite complexe \(u\). On dit que \(u\) est bornée si la 
    suite réelle \(\abs{u}\) est bornée.
\end{defdef}

\begin{prop}
    Soit une suite complexe \(u\). Alors \(u\) est bornée si et 
    seulement si sa partie imaginaire et sa partie réelle sont bornées.
\end{prop}
\begin{proof}
    Supposons que \(u\) est bornée, alors il existe un réel \(M\) tel 
    que pour tout naturel \(n\) \(\abs{u_n} \leqslant M\) alors 
    \(\abs{\Re(u_n)} \leqslant \abs{u_n}\) et \(\abs{\Im(u_n)} \leqslant 
    \abs{u_n}\). Alors \(\Re(u)\) et \(\Im(u)\) sont bornées.

    Supposons maintenant que \(\Re(u)\) et \(\Im(u)\) sont bornées. 
    Alors il existe deux réels \(M\) et \(N\) qui majorent 
    respectivement \(\Re(u)\) et \(\Im(u)\). Du coup le réel 
    \(\sqrt{N^2+M^2}\) majore \(\abs{u}\) donc \(u\) est bornée.
\end{proof}

\begin{prop}
    Soient deux suites complexe \(u\) et \(v\) bornées et deux complexes 
    \(\lambda\) et \(\mu\), alors
    \begin{itemize}
        \item \(\lambda u + \mu v\) est bornée;
        \item \(uv\) est bornée.
    \end{itemize}
    Il existe alors deux réels \(M\) et \(N\) qui majorent 
    respectivement \(\abs{u}\) et \(\abs{v}\) alors le réel 
    \(\abs{\lambda}M+\abs{\mu}N\) majore la suite \(\abs{\lambda u + \mu 
    v}\) et le réel \(MN\) majore \(\abs{uv}\). Donc ces deux suites 
    sont bornées.
\end{prop}

\subsection{Convergence d'une suite complexe}

\begin{defdef}
    Soit une suite complexe \(u\) et un complexe \(\lambda\). On dit que 
    \(u\) converge vers \(\lambda\) si la suite \(\abs{u-\lambda}\) tend 
    vers zéro. Le complexe \(\lambda\) est alors l'unique limite de 
    \(u\). On note \(\lambda = \lim u\).
\end{defdef}
\begin{proof}
    Supposons que \(u\) tendent vers deux complexes \(\lambda\) et 
    \(\lambda'\) alors on a
    \begin{equation}
        0 \leqslant \abs{\lambda - \lambda'} \leqslant \abs{u_n-\lambda} + 
        \abs{u_n-\lambda'}.
    \end{equation}
    Par théorème des gendarmes comme les deux suites tendent vers zéro, 
    alors \(\lambda = \lambda'\).
\end{proof}

\begin{prop}
    Soit une suite complexe \(u\), elle converge si et seulement si 
    \(\Re(u)\) et \(\Im(u)\) convergents. Auquel cas \(\lim u = \lim 
    \Re(u) + i \lim \Im(u)\).
\end{prop}
\begin{proof}
    Si \(u\) converge vers \(\lambda = \alpha+ i \beta\) on sait que 
    pour tout naturel \(n\), \begin{equation}
        \abs{u_n-\lambda} = \abs{\Re(u_n)-\alpha + i(\Im(u_n)-\beta)}.  
    \end{equation}
    Alors
    \begin{align}
        0 \leqslant \abs{\Re(u_n) -\alpha} \leqslant \abs{u_n-\lambda}, \\ 0 
        \leqslant \abs{\Im(u_n) -\beta} \leqslant \abs{u_n-\lambda}.
    \end{align}
    Par théorème des gendarmes, on en déduit que les deux suites réelles 
    \(\Re(u)\) et \(\Im(u)\) tendent respectivement vers \(\alpha\) et 
    \(\beta\).

    Supposons désormais que deux suites réelles \(\Re u\) et \(\Im u\) 
    tendent respectivement vers deux réels \(\alpha\) et \(\beta\). Posons 
    \(\lambda = \alpha + i \beta\), alors pour tout naturel \(n\) on a par 
    inégalité triangulaire
    \begin{equation}
        \abs{u_n-\lambda} \leqslant \abs{\Re(u_n) - \alpha} + \abs{\Im(u_n) 
        - \beta}.
    \end{equation}
    En appliquant le théorème des gendarmes on en conclue que \(u\) tend 
    vers le complexe \(\lambda\).
\end{proof}

\begin{prop}
    Soit une suite complexe \(u\) convergente vers \(\lambda\), alors 
    \(\abs{u}\) tend vers \(\abs{\lambda}\), \(\Re(u)\) tend vers 
    \(\Re(\lambda)\), \(\Im(u)\) tend vers \(\Im(\lambda)\), \(\bar{u}\) 
    tend vers \(\bar{\lambda}\).
\end{prop}
\begin{proof}
    Soit un naturel \(n\), alors \begin{itemize}
        \item \(0 \leqslant \abs{}u_n\abs{-}\lambda\abs{} \leqslant \abs{u_n 
            - \lambda}\), donc \(\abs{u}\) tend vers \(\abs{\lambda}\);
        \item \(0 \leqslant \abs{\bar{u_n} - \bar{\lambda}}\leqslant 
            \abs{u_n - \lambda}\), donc \(\bar{u}\) tend vers 
            \(\bar{\lambda}\);
        \item voir proposition ci-avant;
        \item idem.
    \end{itemize}
\end{proof}
%Soit une suite complexe \(u\) qui ne s'annule pas alors il existe deux 
%suites réelles \(r\) et \(\theta\) telles que \(u = r \e^{i \theta}\). 
%La convergence de \(u\) n'est pas équivalente à la convergence \(r\) et 
%\(\theta\).
\begin{prop}
    Soit une suite complexe \(u\), si elle converge alors elle est bornée.
\end{prop}
\begin{proof}
    Il existe un complexe \(\lambda\)  tel que \(\abs{u-\lambda}\) tende 
    vers zéro, alors \(\abs{u-\lambda}\) est bornée. Il existe un réel 
    \(M\) tel que pour tout naturel \(n\) \(\abs{u_n-\lambda} \leqslant 
    M\) donc \(\abs{u_n} \leqslant \abs{u_n-\lambda}+\abs{\lambda} 
    \leqslant M+\abs{\lambda}\) donc \(u\) est bornée.
\end{proof}

\subsection{Opérations sur les limites}

\begin{prop}
    Soient deux suites complexes \(u\) et \(v\) qui convergent de limite 
    respectives \(\lambda\) et \(\lambda'\). Soit \(\alpha\) et \(\beta\) 
    deux complexes, alors
    \begin{itemize}
        \item \(\alpha u + \beta v\) converge de limite \(\alpha \lambda + 
            \beta \lambda'\);
        \item \(uv\) est convergente de limite \(\lambda \lambda'\).
    \end{itemize}
\end{prop}
\begin{proof} Voir la démonstration pour les suite réelles.
    %Soit un naturel \(n\)
\end{proof}
\begin{prop}
    Soit une suite complexe \(u\), supposons que \(u\) converge vers un 
    complexe \(\lambda\). Il existe donc un naturel \(n_0\) à partir 
    duquel \(u\) est non nulle. Donc la suite 
    \(\left(\frac{1}{u_n}\right)_{n \geqslant n_0}\) tend vers 
    \(\frac{1}{\lambda}\).
\end{prop}
\begin{proof}
    On sait que \(\abs{u}\) tend vers \(\abs{\lambda}>0\) et pour tout 
    naturel \(n\) si \(n \geqslant n_0\) alors \(\abs{u_n} \geqslant 0\), 
    on peu alors définir la suite inverse \(\left(\frac{1}{u_n}\right)_{n 
    \geqslant n_0}\) et on sait que
    \begin{equation}
        \forall n \in \N \quad n \geqslant n_0 \implies \frac{1}{u_n} = 
        \frac{\bar{u_n}}{\abs{u_n}^2}.
    \end{equation}
    La suite réelle \(\frac{1}{\abs{u_n}^2}\) tend vers 
    \(\frac{1}{\abs{\lambda}^2}\) et de plus on sait que \(\lim \bar{u} = 
    \bar{\lambda}\). Alors par produit \(\lim \frac{1}{u} = 
    \frac{1}{\lambda}\)
\end{proof}
\subsection{Suites extraites et théorème de Bolzano-Weiertrass}
\begin{defdef}
    Soit une suite complexe \(u\). Une suite complexe \(v\) est dite 
    extraite de \(u\) s'il existe une application \(\varphi:\N \longmapsto 
    \N\) strictement croissante telle que pour tout naturel \(n\) \(v_n = 
    u_{\varphi(n)}\).
\end{defdef}
\begin{prop}
    Soit une suite complexe \(u\) et \(v\) une suite extraite de \(u\). 
    Alors
    \begin{itemize}
        \item si \(u\) converge alors \(v\) converge de même limite;
        \item si \(u\) est bornée, alors \(v\) est bornée.
    \end{itemize}
    Les deux réciproques sont fausses.
\end{prop}
\begin{proof}
    \begin{itemize}
        \item il existe un complexe \(\lambda\) tel que \(\abs{u-\lambda}\) 
            tende vers zéro. La proposition appliqué aux suites réelles donne 
            que la suite \(\abs{v-\lambda}\) tend vers zéro. La suite \(v\) 
            tend vers \(\lambda\).
        \item La suite réelle \(\abs{u}\) est bornée et la suite réelle 
            \(\abs{v}\) est extraite de \(\abs{u}\) alors \(v\) est bornée.
    \end{itemize}
\end{proof}
\begin{theo}[théorème de Bolzano-Weiertrass]
    De toute suite complexe bornée on peut extraire une sous-suite 
    convergente.
\end{theo}
\begin{proof}
    Soit une suite complexe \(u\) bornée. La suite réelle \(\Re(u)\) est 
    bornée, on applique donc le théorème pour les suites réelles pour 
    affirmer qu'il existe une application \(\varphi:\N \longmapsto \N\) 
    strictement croissante telle que \((\Re(u_{\varphi(n)})_{n \in \N}\) 
    converge. La suite réelle \(v = \Im(u_{\varphi(n)})_{n \in \N}\) est 
    bornée puisqu'elle est extraite de la suite \(\Im(u)\). le théorème 
    pour les suites réelles permet d'affirmer qu'il existe une application 
    \(\psi:\N \longmapsto \N\) strictement croissante telle que 
    \((v_{\psi(n)})_{n \in \N}\). Alors \(v_{\psi(n)} = \Im(u_{\varphi 
    \circ \psi(n)})\) et l'application \(\varphi \circ \psi\) est 
    strictement croissante. Ainsi \(u_{\varphi \circ \psi(n)}\) est 
    extraite de \(u\). La suite \(\Im(u_{\varphi \circ \psi(n)})_{n \in 
    \N}\) converge donc. La suite \(\Re(u_{\varphi \circ \psi(n)})_{n \in 
    \N}\) est extraite de \(\Re(u_{\varphi(n)})_{n \in \N}\), donc elle 
    converge aussi. De cette manière la suite \((u_{\varphi \circ 
    \psi(n)})_{n \in \N}\) converge.
\end{proof}

Ne pas étendre aux suites complexes les propositions, vues sur les 
suites réelles, relatives à la notion d'ordre. Puisqu'il n'y a pas 
d'ordre sur \(\C\).

\section{Exercice}
\subsection{Convergence d'une suite numérique}
\begin{exercice}
    \begin{enumerate}
        \item Donner des exemples de suites qui divergent telles que la 
            somme converge.
        \item Donner des exemples de suites qui divergent telles que le 
            produit converge.
    \end{enumerate}
\end{exercice}
\begin{exercice}
    Soient \(u\) et \(v\) deux suites réelles telles que \(\forall n \in 
    \N \ u_n \leqslant v_n\). On suppose que la suite \(v\) converge dans 
    \(\R\) de limite \(\ell\). Les assertions suivantes sont elles 
    nécessairement vraies? Justifier les réponses.
    \begin{enumerate}
        \item \(\forall n \in \N \quad u_n \leqslant \ell\)
        \item \(u\) est majorée.
        \item \(u\) est convergente.
    \end{enumerate}
\end{exercice}
\emph{Résolution~:}
Le point 2 est vrai. En effet, la suite \(v\) converge, donc \(v\) est 
majorée par un réel \(M\). Alors comme \(u\) est majorée par \(v\), 
alors \(u\) est aussi majoré par ce réel \(M\). Néanmoins les points 1 
et 3 sont faux. Prenons un contre-exemple : la suite \(v\) définie pour 
tout naturel \(n\) par \(v_n = (-2)^{-n}\) est convergente en \(\ell = 
0\), et la suite \(u\) définie pour \(n<2\) par \(u_n = v_n - 1/100\) et 
pour \(n\geqslant 2\) par \(u_n = -10n\). Alors on a \(u_0 = v_0 - 1/100 
= 1 - 1/100 > \ell\) (qui contredit le point 1) et on voit aussi que 
\(u\) est divergente de première espèce (qui contredit le point 3).

\begin{exercice}
    Soient \(u\) une suite numérique (réelle ou complexe) et \(f \in 
    \N^{\N}\).
    \begin{enumerate}
        \item On suppose que \(u\) est convergente et \(f\) injective. 
            Montrer que la suite extraite \((u_{f(n)})_{n \in \N}\) converge
        \item On suppose que \((u_{f(n)})_{n \in \N}\) est convergente et 
            \(f\) surjective. Montrer que la suite \(u_n\) converge
        \item Si \(f\) est bijective, montrer que \(u\) converge si et 
            seulement si \((u_{f(n)})_{n \in \N}\) converge.
    \end{enumerate}
\end{exercice}
\begin{exercice}
    Soit \(u\) une suite réelle à valeurs strictement positives. On 
    suppose que pour tout naturel \(n\) \(\frac{u_{n+1}}{u_n}\) tend vers 
    \(\ell\) finie ou infinie.
    \begin{enumerate}
        \item On suppose que \(\ell \in \intervallefo{0}{1}\). Montrer que 
            \(u\) tend vers zéro.
        \item On suppose \(\ell > 1\) ou \(\ell = +\infty\). Montrer que 
            \(u\) tend vers \(+\infty\).
        \item On suppose que \(\ell = 1\). Montrer que tous les cas sont 
            possible~: \(u\) converge de limite nulle ou non, ou est 
            divergente de première ou seconde espèce.
    \end{enumerate}
\end{exercice}
\paragraph{Cesaro, variantes et applications}
\begin{exercice}[Lemme de Cesar\`o]
    Soient \(u\) une suite réelle ou complexe et \(v\) la suite des 
    moyennes des valeurs de \(u\) définie par~:
    \[ \forall n \in \N \quad v_n = \frac{1}{n+1}\sum_{k = 0}^n u_k \]
    \begin{enumerate}
        \item Démontrer que si \(u\) converge de limite \(\ell\) finie alors 
            \(v\) converge aussi vers \(\ell\).
        \item La réciproque est elle vraie?
        \item Démontrer que si la suite \(u\) diverge et tend vers 
            \(+\infty\) alors \(v\) diverge aussi et tend vers \(+\infty\)
    \end{enumerate}
\end{exercice}
\begin{exercice}
    Soient une suite \(u\), réelle ou complexe, et une suite \(a\) à 
    termes strictement positifs telle que \[\lim_{n \to \infty} \sum_{k 
    = 0}^n a_n = +\infty\] et soit la suite \(v\) définie par~:
    \[\forall n \in \N \quad v_n = \frac{\sum_{k = 0}^n a_k u_k}{\sum_{i 
    = 0}^n a_i}.\]
    Démontrer que si la suite \(u\) a pour limite \(\ell\) alors la 
    suite \(v\) a aussi pour limite \(\ell\).
\end{exercice}
\begin{exercice}
    Soient un réel \(a\) et une suite \(u\) à valeurs réelles ou 
    complexes. On suppose que \(\lim_{n\to\infty} (u_{n+1}-u_n) = a\), 
    montrer que \(\lim_{n\to\infty} \frac{u_n}{n} = a\).
\end{exercice}
\paragraph{Suites extraites}
\begin{exercice}
    Soit \(u\) une suite, réelle ou complexe, telle que les suites 
    extraites \((u_{2n})_{n \in \N}\), \((u_{2n+1)})_{n \in \N}\), 
    \((u_{3n})_{n \in \N}\) soient convergentes. Montrer que \(u\) 
    converge.
\end{exercice}
\begin{exercice}
    Soit une suite réelle \(u\) non majorée. Montrer que \(u\) admet une 
    suite extraite qui tend vers \(+\infty\).
\end{exercice}
\begin{exercice}
    Soit \(u\) une suite, réelle ou complexe, telle que les suites 
    extraites \((u_{3n})_{n \in \N}\), \((u_{3n+1)})_{n \in \N}\), 
    \((u_{3n+2})_{n \in \N}\) admettent une même limite \(\ell\) finie 
    ou infinie. Montrer que \(u\) possède la même limite \(\ell\).
\end{exercice}
\begin{exercice}
    Soit une suite \((u_n)_{n \in \N^*}\) une suite réelle telle que~:
    \[\forall (m,n) \in (\N^*)^2 \quad 0 \leqslant u_{m+n} \leqslant 
    \frac{m+n}{mn}\]
    Montrer que \((u_n)_{n\geqslant 1}\) tend vers zéro.
\end{exercice}
\begin{exercice}[Diverses études de suites]
    Étudier la convergence de chacune des suites suivantes et dans le 
    cas où elles convergent préciser leurs limites. \(x\) est un réel.
    \begin{align*}
        a_n &= \frac{n+1}{n+\cos n} \qquad & e_n & = \frac{3n^2+\cos 
        n}{4(n+1)^2+\sin(3n)} \\
        b_n &= \frac{\sum_{k = 1}^n \Ent(kx)}{n^2} \ (n \geqslant 1) 
        \qquad & f_n &= \sum_{k = 1}^n \frac{n}{\sqrt{n^4+k}} (n 
        \geqslant 1) \\
        c_n &= \frac{n+1}{n} - \frac{n}{n+\ii} \ (n \geqslant 1) \quad & 
        g_n &= \frac{1-n}{\ii^n n \ln n} (n\geqslant 2) \\
        d_n &= (\ln n)^{1/n} \ (n \geqslant 1) \quad & h_n &= 
        \cos\left(2\pi\sqrt{n^2+1}+\frac{2\pi}{3}\right)
    \end{align*}
\end{exercice}
\subsection{Suites monotones}
\begin{exercice}
    Soient \(u\) et \(v\) des suites réelles telles que \(\forall n 
    \in \N \ u_n \leqslant v_n\). On suppose que la suite \(v\) est 
    convergente dans \(\R\) de limite \(\ell\) et que \(u\) est 
    croissante. Les assertions suivantes sont elles nécessairement 
    vraies? Justifier les réponses.
    \begin{enumerate}
        \item \(\forall n \in \N \ u_n \leqslant \ell\)
        \item \(u\) est majorée.
        \item \(u\) est convergente.
    \end{enumerate}
\end{exercice}
\emph{Résolution~:} Tous les points sont vrais. En effet, la suite 
\(v\) converge, donc elle est bornée, particulièrement majorée par 
un réel \(M\). Ensuite, d'après l'inégalité de l'hypothèse pour 
tout \(n \ u_n \leqslant v_n \leqslant M\). Donc \(u\) est 
majorée. Le point 2 est vrai. La suite \(u\) est aussi croissante, 
donc elle converge vers \(\ell' = \sup\enstq{u_n}{n\in \N}\). Le 
point 3 est vrai. En passant à la limite dans l'inégalité de 
l'hypothèse, on obtient \(\ell' \leqslant \ell\). Et on a aussi 
par définition de la borne supérieure, pour tout \(n\), \(u_n 
\leqslant \ell' \leqslant \ell\). Le point 1 est vrai.
\begin{exercice}
    Soit la suite \(u\) telle que \(u_0 = 0\) et pour tout \(n 
    \geqslant 1\)~:
    \[u_n = \sqrt{1 + \sqrt{1 + \sqrt{1 + \cdots +\sqrt{1 + \sqrt{1}}}}}\]
    où l'on a écrit \(n\) fois le nombre \(1\). Prouver que la suite 
    converge et calculer sa limite.
\end{exercice}
\begin{exercice}[Moyenne arithmético-harmonique]
    Soient deux réels \(a, b\) tels que \(0 < a < b\) et soient des 
    suites \(u, v\) telles que~:
    \[ \begin{cases} u_0 = a \, v_0 = b \\ \forall n \in \N \quad 
        u_{n+1} = \frac{2 u_n v_n}{u_n + v_n} \, v_{n+1} = 
    \frac{u_n+v_n}{2} \end{cases}\]
    Démontrer que ces suites convergent et trouver leurs limites en 
    fonction de \(a\) et \(b\).
\end{exercice}
\begin{exercice}[Moyenne arithmético-géométrique]
    Soient deux réels \(a, b\) tels que \(0 < a < b \) et soient des 
    suites \(u, v\) telles que~:
    \[ \begin{cases} u_0 = a \, v_0 = b \\ \forall n \in \N \quad 
        u_{n+1} = \sqrt{u_n v_n} \, v_{n+1} = \frac{u_n+v_n}{2} 
    \end{cases}\]
    Démontrer que ces suites sont adjacentes et que leur limite commune 
    notée \(\ell(a,b)\) vérifie~:
    \[ \sqrt{ab} \leqslant \ell(a,b) \leqslant \frac{a+b}{2}\]
\end{exercice}
\begin{exercice}
    Soit un naturel \(k \geqslant 2\). Montrer que la suite de terme 
    général~:
    \[ \forall n \in \N^* \quad u_n = \sum_{i = 1}^{(k-1)n} 
    \frac{1}{n+i} \]
    est convergente.
\end{exercice}
\begin{exercice}
    Soit \((u_n)_{n \in \N}\) une suite de réels strictement positifs, 
    décroissante et convergente de limite nulle. Montrer que la suite 
    alternée des somme partielles \((S_n)_{n\in\N}\) définie pour tout 
    naturel \(n\) par \(S_n = \sum_{k = 0} (-1)^k u_k\) converge.
\end{exercice}
\begin{exercice}
    Soit \(u\) une suite réelle bornée telle que pour tout naturel \(n\) 
    non nul~: \(2u_n \leqslant u_{n+1} + u_{n-1} \). Montrer que \(u\) 
    converge.
\end{exercice}
\emph{Résolution~:}
En appliquant la relation au rang suivant, on obtient~: \(u_{n+2} - 
u_{n+1} \geqslant u_{n+1} - u_n\). Soient les sous-suites \(v\) et 
\(w\) des rangs pairs et impairs définies par~: \(v_n = u_{2n+1} \ w_n 
= u_{2n}\).
\begin{exercice}[Suites de Cauchy]
    Soit une suite réelle \(u\) qui vérifie la propriété (elle est dite 
    de Cauchy)~:
    \[ \forall \epsilon > 0 \ \exists n_0 \in \N \ \forall (p,q) \in 
    \N^2 \quad p \geqslant q \geqslant n_0 \implies \abs{u_p-u_q} < 
    \epsilon \]
    Les termes de la suite se rapprochent uniformément les uns des autres.
    \begin{enumerate}
        \item Montrer que la suite est bornée
        \item Pour tout naturel \(n\) on pose \(E_n = \enstq{u_p}{p\geqslant 
            n}\). Montrer que \(E_n\) admet une borne supérieure et une borne 
            inférieure notée respectivement \(a_n\) et \(b_n\).
        \item Montrer que les suites \(a\) et \(b\) sont adjacentes.
        \item Prouver que la suite \(u\) est convergente.
    \end{enumerate}
\end{exercice}
\emph{Résolution~:}
\begin{enumerate}
    \item Pour \(\epsilon = 1 \ q = n_0\), on a par inégalité 
        triangulaire~:
        \[ \forall p \geqslant n_0 \quad \abs{u_p} \leqslant \abs{u_p - 
        u_{n_0}} + \abs{u_{n_0}} \leqslant 1 + \abs{u_{n_0}}\]
        Donc \(u\) est bornée.
    \item La suite \(u\) est bornée, donc \(E_n\) est une partie non vide 
        et bornée de \(\R\), donc \(E_n\) admet une borne supérieure (notée 
        \(a_n\)) et une borne inférieure (notée \(b_n\)). La suite 
        d'ensemble \(E_n\) est décroissante par inclusion. En effet, soit un 
        naturel \(p\). Si \(u_p \in E_{n+1}\) alors \(p \geqslant n+1\), 
        donc \(p > n\), et donc \(u_p \in E_n\). Alors pour tout naturel 
        \(n\), \(E_{n+1} \subset E_n\). Donc par les propriétés des bornes, 
        on a bien : \(b_n \leqslant b_{n+1} \leqslant a_{n+1} \leqslant 
        a_n\). De plus, par caractérisation, on peut écrire~:
        \begin{align*}
            \forall \epsilon_1 >0 \ \exists p_1 \geqslant n & \quad 0 
            \leqslant a_n - u_{p_1} \leqslant \epsilon_1 \\
            \forall \epsilon_2 >0 \ \exists p_2 \geqslant n & \quad 0 
            \leqslant u_{p_2} - b_n \leqslant \epsilon_2
        \end{align*}
        Si on pose \(p = \max(p_1,p_2)\) alors pour tout \(\epsilon>0\), on 
        a \(0 \leqslant a_n-u_p \leqslant \epsilon/2\) et \(0 \leqslant 
        u_p-b_n \leqslant \epsilon/2\), donc par inégalité triangulaire~:
        \[ \abs{a_n - b_n} = \abs{(a_n-u_p) - (b_n-u_p)} \leqslant 
        \abs{a_n-u_p} + \abs{u_p-b_n} = \epsilon. \]
        Donc la suite \((a-b)\) tend vers zéro. Ainsi les suites \(a\) et 
        \(b\) sont adjacentes.
    \item Les suites \(a\) et \(b\) sont adjacentes, donc elles convergent 
        vers une même limite notée \(\ell\). Pour tout naturel \(n\), on a 
        \(b_n \leqslant u_n \leqslant a_n\). En appliquant le théorème des 
        gendarmes, il vient que la suite \(u\) tend aussi vers \(\ell\).
\end{enumerate}
\begin{exercice}
    Étudier, selon la valeur de l'entier \(p\), la convergence de la suite 
    \(u\) définie par~: \[\forall n \in \N \quad u_n = \frac{\sum_{k = 
    1}^n k!}{(n+p)!}.\]
\end{exercice}

\begin{exercice}
    Deux parties \(A, B\) de \(\R\) sont dites adjacentes si et seulement 
    si~:
    \begin{align}
        \forall (a,b) \in A \times B \quad a & \leqslant b \\       \forall 
        \epsilon >0 \ \exists (a,b) \in A \times B \quad b-a & < \epsilon
    \end{align}
    \begin{enumerate}
        \item Montrer que deux parties \(A, B\) de \(\R\) sont adjacentes si 
            et seulement si \(\sup A\) et \(\inf B\) existent et sont égaux.
        \item Montrer que si \(u\) et \(v\) sont deux suites adjacentes 
            alors leurs ensemble de valeurs (\(A = \enstq{u_n}{n\in\N}, B = 
            \enstq{v_n}{n\in\N}\)) sont adjacents. La réciproque est telle 
            vraie?
    \end{enumerate}
\end{exercice}
\begin{exercice}
    Soit une naturel \(k\). Étudier la monotonie et la convergence de la 
    suite définie pour tout naturel \(n \geqslant k\) par~: \(u_n 
    =\frac{\binom{n}{k}}{n^k}\).
\end{exercice}
\emph{Résolution~:}
Montrons que \(u\) est croissante~: pour tout naturel \(n\geqslant k\), 
on a~:
\begin{align*}
    \frac{u_{n+1}}{u_n} &= \frac{(n+1)!}{(n+1-k)!}\frac{(n-k)!}{n!} 
    \left(\frac{n+1}{n}\right)^k \\
    &= \frac{n+1}{n+1-k}\left(\frac{n+1}{n}\right)^k 
    \geqslant 1
\end{align*}
Donc \(u_{n+1} \geqslant u_n\), la suite \(u\) est croissante. De plus~:
\[u_n = \frac{1}{k!} \prod_{i = 1}^k \frac{n-(i-1)}{n}\]
si on note \(v_{n,i} = \frac{n-(i-1)}{n}\), alors lorsque \(n\) tend 
vers l'infini, \(v_{n,i} \sim 1\). Alors par produit (fini) 
d'équivalent, \(u_n \sim \frac{1}{k!} \times 1^k\). Donc \(\lim u = 
\frac{1}{k!}\).
\begin{exercice}[Densité]
    Soit \(A\) une partie de \(\R\). Montrer l'équivalence entre les deux 
    assertions suivantes~:
    \begin{enumerate}
        \item Entre deux éléments distincts de \(\R\), il existe au moins un 
            élément de \(A\);
        \item Tout élément de \(\R\) est limite d'une suite d'éléments de 
            \(A\).
    \end{enumerate}
    Si l'une ou l'autre des propositions est vérifiée, on dit que la 
    partie \(A\) est dense dans \(\R\).
\end{exercice}
\subsection{Comparaison des suites réelles}
\begin{exercice}
    Soit une suite réelle \(u\).
    \begin{enumerate}
        \item On suppose que \(u_{n+1} \sim_{\infty} u_n\), est-ce que 
            \(u_{2n} \sim_{\infty} u_n\)?
        \item Soit \(\ell \in \R\). On suppose que \(u_n \sim_{\infty} 
            \ell\), est-ce que \(u_n^n \sim_{\infty} \ell^n\)?
    \end{enumerate}
\end{exercice}
\begin{exercice}
    Déterminer des équivalents simples des suites de terme général~:
    \begin{align*}
        u_n & = \frac{1}{n-1} - \frac{1}{n+1} \qquad & u_n & = 
        (1+\frac{x}{n})^n (x \in \R) \\
        u_n & = (n+1)^{\alpha} - (n-1)^{\alpha} \qquad & u_n & = 2^{n^2+2} 
        -2^{n^2} \\
        u_n & = \frac{4^n+7}{2^n-n^18} \qquad & u_n & = (n+3\ln 
        n)\e^{-(n+1)}
    \end{align*}
\end{exercice}
\begin{exercice}
    Démontrer que, lorsque \(n\) tend vers l'infini, \[\sum_{k = 1}^n k! 
    \sim n!\]
\end{exercice}
\begin{exercice}
    \begin{enumerate}
        \item Soient deux suites de limite nulle \(u, v\). Montrer que, 
            lorsque \(n\) tend vers l'infini, \[\e^{u_n}-\e^{v_n} \sim u_n - 
            v_n\]
        \item Que peut on dire si \(\lim u = \lim v = \ell \neq 0\)?
        \item Déterminer la limite de la suite \(u\) définie par~:
            \[ \forall n \in \N^* \quad u_n = n^2 (\e^{1/n} - \e^{1/(n+1)})\]
    \end{enumerate}
\end{exercice}
\emph{Résolution~:}
\begin{enumerate}
    \item On sait d'après le cours que lorsque \(n\) tend vers l'infini, 
        \(\e^{u_n-v_n}-1 \sim u_n-v_n\). Alors en en multipliant par 
        \(e^{v_n}\), on obtient \(\e^{u_n}-\e^{v_n} = \e^{v_n} (u_n - 
        v_n)\). Donc par définition de l'équivalence~: \(\lim\limits_{n\to 
        \infty} \frac{\e^{u_n}-\e^{v_n} }{u_n-v_n} = \lim\limits_{n\to 
        \infty} \e^{v_n}\). Comme la fonction exponentielle est continue 
        \(\lim\limits_{n\to \infty} \e^{v_n} = \e^{\lim v_n} = \e^{0} = 1\). 
        Alors \(\lim\limits_{n\to \infty} \frac{\e^{u_n}-\e^{v_n} }{u_n-v_n} 
        = 1\), ce qui veut dire que \(\e^{u_n}-\e^{v_n} \sim u_n - v_n\) 
        lorsque \(n\) tend à l'infini.
    \item Si \(\lim u = \lim v = \ell \neq 0\), alors les suites 
        \((u-\ell)\) et \((v-\ell)\) tendent vers zéro et en appliquant la 
        relation précédente, lorsque \(n\) tend à l'infini~: 
        \(\e^{(u_n-\ell)} - \e^{(v_n-\ell)} \sim (u_n-\ell) - (v_n-\ell)\), 
        et lorsqu'on multiplie par \(\e^{\ell}\), il vient~: 
        \(\e^{u_n}-\e^{v_n} \sim \e^{\ell}(u_n - v_n)\).
    \item Si on pose pour tout naturel non nul, \(n\), \(a_n = 1/n\) et 
        \(b_n = 1/(n+1)\), avec \(\lim a = \lim b = 0\) alors 
        \(\frac{u_n}{n^2} = \e^{a_n} -\e^{b_n}\), d'après le premier point, 
        lorsque \(n\) tend à l'infini, \(\frac{u_n}{n^2} \sim a_n - b_n = 
        \frac{1}{n(n+1)}\). En multipliant par \(n^2\), il vient lorsque 
        \(n\) tend à l'infini, \(u_n \sim \frac{n^2}{n(n+1)} = 
        \frac{n}{n+1}\). Le deuxième terme de l'équivalent tend vers \(1\), 
        donc \(\lim u = 1\).
\end{enumerate}
\begin{exercice}
    Soit le suite réelle \(u\) définie par~: \(u_0 \in 
    \intervalleoo{0}{1}\) et \(\forall n \in \N \ u_{n+1} = u_n - u_n^2\).
    \begin{enumerate}
        \item Montrer que \(u\) converge et déterminer sa limite.
        \item Prouver que la suite \(\left(\frac{1}{u_{n+1}} - 
            \frac{1}{u_n}\right)_{n \in \N}\) converge et déterminer sa 
            limite.
        \item Démontrer que \(u_n \sim \frac{1}{n}\).
    \end{enumerate}
\end{exercice}
\emph{Résolution~:}
\begin{enumerate}
    \item La suite \(u\) est bornée. Montrons par récurrence \(\P_n 
        \forall n \in \N \ 0<u_n<1\). L'initialisation est vraie puisque 
        \(0<u_0<1\) par hypothèse donc \(\P_0\). Soit un naturel \(n\) et 
        supposons \(\P_n\), alors comme \(0<u_n<1\), on a aussi 
        \(0<1-u_n<1\), donc par produit (\(u_{n+1} = u_n(1-u_n)\)), 
        \(0<u_{n+1}<1\). Donc \(\P_{n+1}\) est vraie. Le théorème de 
        récurrence affirme que la la proposition est vraie pour tout naturel 
        \(n\). La suite \(u\) est bornée. On a aussi \(u_{n+1}-u_n = -u_n^2 
        < 0\) donc la suite \(u\) décroît strictement. Finalement la suite 
        \(u\) décroît et est minorée, donc elle converge. Sa limite \(\ell\) 
        est telle qu'en passant à la limite dans l'équation de récurrence~: 
        \(\ell = \ell(1-\ell)\), donc \(\ell^2 = 0\), donc \(\ell = 0\). La 
        suite \(u\) converge de limite nulle.
    \item Soit la suite \(v\) définie pour tout naturel \(n\), par \(v_n 
        =\frac{1}{u_{n+1}} - \frac{1}{u_n}\), alors grâce à l'équation qui 
        définit \(u\), on trouve \(v_n = \frac{1}{1-u_n}\). Soit un réel 
        \(a<1\) et 
        \(\fonction{f}{\intervalleff{0}{a}}{\R}{x}{\frac{1}{1-x}}\). \(f\) 
        est continue, particulièrement en \(0 = \lim u\), et \(v_n = 
    f(u_n)\), donc \(\lim v_n = f(\lim u_n) = f(0) = 1\).  \item On a 
        aussi, pour tout naturel \(n\),  \(v_n = \frac{1 - 
        \frac{u_{n+1}}{u_n}}{u_{n+1}} = \frac{1-(1-u_n)}{u_{n+1}} = 
        \frac{u_n}{u_{n+1}}\). Comme la suite \(v\) tend vers \(1\), on peut 
        dire que lorsque \(n\) tend vers l'infini, \(u_{n+1} \sim u_n\), et 
        \(n+1 \sim n\). En multipliant ces deux équivalents, on a 
        \((n+1)u_{n+1} \sim n u_n\). En divisant cet équivalent par \(n 
        u_{n+1}\), on obtient lorsque \(n\) tend vers l'infini~: 
        \(\frac{u_n}{u_{n+1}} \sim \frac{n+1}{n}\). Le premier terme de 
        l'équivalent vaut \(v_n = \frac{1}{1-u_n}\). Comme \(\lim u = 0\), 
        on peut écrire que lorsque \(n\) tend vers l'infini, 
        \(\frac{1}{1-u_n} \sim u_n+1\). Donc en reprenant l'équation~: 
        \(u_n+1 \sim 1+\frac{1}{n}\). Ce qui peut s'écrire aussi~: \(u_n+1 
        -(1/n+1) = o(1+1/n) = o(1/n)\) et finalement \(u_n-1/n = o(1/n)\). 
        Ce qui est équivalent de dire que \(u_n \sim \frac{1}{n}\).
\end{enumerate}
\begin{exercice}
    Soient \(a\) et \(b\) deux suites réelles à valeurs strictement 
    positives telles que \(\lim\limits_{n\to\infty} a_n^n = a > 0\) et 
    \(\lim\limits_{n\to\infty} b_n^n = b > 0\). Soient deux réels 
    strictement positifs, \(p, q\) tels que \(p+q = 1\). Déterminer~:
    \[\lim\limits_{n\to\infty} (pa_n+qb_n)^n\]
\end{exercice}
\begin{exercice}
    Démontrer que \[\lim\limits_{n\to \infty} (\cosh n)^{1/n} = e\] puis 
    déterminer un équivalent simple, lorsque \(n\) tend à l'infini, de 
    \((\cosh n)^{1/n} - e\).
\end{exercice}
\begin{exercice}
    Démontrer, lorsque \(n\) tend à l'infini,
    \[ \frac{1}{n-1} = \frac{1}{n} + o\left(\frac{1}{n}\right); \quad 
    \frac{1}{n-1} = \frac{1}{n} + \frac{1}{n^2} + 
    o\left(\frac{1}{n^2}\right); \quad \frac{1}{n-1} = \frac{1}{n} + 
    \frac{1}{n^2}+ \frac{1}{n^3} + o\left(\frac{1}{n^3}\right)\]
\end{exercice}
\begin{exercice}[Développement asymptotique]
    \begin{enumerate}
        \item Démontrer que, pour tout naturel \(n \geqslant 2\), l'équation 
            \(x^n-x-1 = 0\) a une seule solution dans l'intervalle 
            \(\intervalleff{1}{2}\), qu'on notera \(x_n\).
        \item Démontrer que la suite \(x = (x_n)_{n \in \N\setminus\{0,1\}}\) 
            est convergente et que sa limite vaut \(1\).
        \item On associe à la suite \(x\) la suite \(u\) telle que pour tout \(n 
            \geqslant 2\), \(u_n = x_n-1\).
            \begin{enumerate}
                \item Démontrer que \[\forall n \geqslant 2 \quad u_n = 
                    \exp\left(\frac{1}{n}\ln(2+u_n)\right)-1\]
                \item Déterminer un équivalent de \(u_n\) lorsque \(n\) tend à 
                    l'infini, du type \(\frac{a}{n}\) où \(a\) est une constante 
                    réelle qu'on précisera.
                \item Déterminer le réel \(b\) tel que, lorsque \(n\) tend à 
                    l'infini,
                    \[u_n = \frac{a}{n} + \frac{b}{n^2} 
                    +o\left(\frac{1}{n^2}\right).\]
                    Une telle écriture est appelée un \emph{développement 
                    asymptotique} de \(u_n\).
            \end{enumerate}
    \end{enumerate}
\end{exercice}
