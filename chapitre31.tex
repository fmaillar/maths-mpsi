\chapter{Compléments d'analyse}\label{chap:complementsanalyse}
\minitoc%
\minilof%
\minilot%

\section{Intégrales doubles}

\subsection{Intégrales doubles sur un segment}

\subsubsection{Théorème de Fubini --- Définitions}

\begin{theo}[Admis]
  Soient quatre réels \(a\), \(b\), \(c\) et \(d\) tels que \(a\leqslant b\) et
  \(c \leqslant d\), \(f\) une fonction continue sur le rectangle
  \(\intervalleff{a}{b} \times \intervalleff{c}{d}\). Alors
  \begin{equation}
  \int_a^b\left(\int_c^d f(x,y)\diff y \right)\diff x = \int_c^d\left(\int_a^b
  f(x,y)\diff x\right)\diff y. \end{equation}
  Cette valeur commune est appelée intégrale double de \(f\) sur le rectangle
  \(\intervalleff{a}{b} \times \intervalleff{c}{d}\) et notée
  \begin{equation}
    \iint_{\intervalleff{a}{b} \times \intervalleff{c}{d}}f(x,y) \diff x \diff y
    = \iint_{\intervalleff{a}{b} \times \intervalleff{c}{d}}f.
  \end{equation}
\end{theo}

\emph{Interprétation}~:
\begin{enumerate}
  \item Lorsque \(f\) est à valeurs positives, \(\iint_{\intervalleff{a}{b}
    \times \intervalleff{c}{d}}f\) représente le volume de l'ensemble
    \(\enstq{(x,y,z)\in\R^3}{(x,y) \in \intervalleff{a}{b} \times
    \intervalleff{c}{d} \ 0 \leqslant z \leqslant f(x,y)}\).
  \item Lorsque \(f=\tilde{1}\) sur le rectangle \(\intervalleff{a}{b} \times
    \intervalleff{c}{d}\),
    \begin{equation}
      \iint_{\intervalleff{a}{b} \times
      \intervalleff{c}{d}}\tilde{1}=(a-b)(c-d).
    \end{equation}
\end{enumerate}
\begin{corth}
  Avec les hypothèses du théorème, on suppose en outre qu'il existe deux
  fonctions continues \(u \in \R^{\intervalleff{a}{b}}\) et \(v \in
  \R^{\intervalleff{c}{d}}\) telles que pour tout \((x, y) \in
  \intervalleff{a}{b} \times \intervalleff{c}{d}\) on a \(f(x,y)=u(x)v(y)\),
  alors
  \begin{equation}
    \iint_{\intervalleff{a}{b} \times \intervalleff{c}{d}} f(x,y)\diff x \diff y
    = \int_a^b u(x)\diff x \int_c^d v(y)\diff y.
  \end{equation}
\end{corth}

\subsubsection{Propriétés}

\begin{prop}
  Soient \(R\) un rectangle du plan, deux fonctions continues \(f\) et \(g\), et
  un réel \(\lambda\). Alors
  \begin{equation}
    \iint_R (\lambda f+g) = \lambda \iint_R f + \iint_R g.
  \end{equation}
\end{prop}
\begin{prop}
  Soient \(R_1\) et \(R_2\) deux rectangles disjoints du plan, \(f\) une
  fonction continue sur \(R_1 \cup R_2\). Alors
  \begin{equation}
    \iint_{R_1 \cup R_2} f = \iint_{R_1} f + \iint_{R_2} f.
  \end{equation}
\end{prop}
\begin{prop}
  Soient \(R\) un rectangle du plan, deux fonctions continues \(f\) et \(g\).
  Supposons que \(f \leqslant g\), alors
  \begin{equation}
    \iint_R f \leqslant \iint_R g.
  \end{equation}
\end{prop}

\subsection{Intégrale double sur un domaine simple}

\subsubsection{Théorème de Fubini}

\begin{defdef}
  On appelle domaine simple du plan toute partie \(\Ddomaine\) de \(\R^2\)
  telle~:
  \begin{itemize}
    \item qu'il existe quatre réels \(a\), \(b\), \(c\) et \(d\) (\(a\leqslant
      b\) et \(c \leqslant d\)) ;
    \item qu'il existe des fonctions continues \(\varphi_1\) et \(\varphi_2\) de
      \(\intervalleff{a}{b}\) vers \(\R\); et \(\psi_1\) et \(\psi_2\) de
      \(\intervalleff{c}{d}\) vers \(\R\) ;
  \end{itemize}
  telles que
  \begin{align}
    \Ddomaine&=\enstq{(x,y)\in\R^2}{a\leqslant x\leqslant b \quad
    \varphi_1(x)\leqslant y\leqslant \varphi_2(x)}\\
    &=\enstq{(x,y)\in\R^2}{c \leqslant y \leqslant d \quad \psi_1(y) \leqslant x
    \leqslant \psi_2(y)}
  \end{align}
\end{defdef}

\begin{defdef}
  Soient \(\Ddomaine\) un domaine simple du plan (avec les notations de la
  définition) et \(f\) une fonction continue sur \(\Ddomaine\). Alors
  \begin{equation}
    \int_{a}^{b} \left( \int_{\varphi_1(x)}^{\varphi_2(x)} f(x,y)\diff
    y\right)\diff x = \int_{c}^{d} \left( \int_{\psi_1(y)}^{\psi_2(y)}
    f(x,y)\diff x\right)\diff y.
  \end{equation}
  Cette valeur commune est appelée intégrale double de \(f\) dans le domaine
  \(\Ddomaine\), et elle est notée \(\iint_{\Ddomaine} f(x,y)\diff x \diff y =
  \iint_{\Ddomaine} f\).
\end{defdef}

\emph{Interprétation}~: On admet ce théorème, en particulier que l'aire de
\(\Ddomaine\) vaut \(\iint_{\Ddomaine}\diff x \diff y\).

\subsubsection{Propriétés}

\begin{prop}[Linéarité]
  Soient \(\Ddomaine\) un domaine simple du plan, \(f\) et \(g\) deux fonctions
  continues sur \(\Ddomaine\), et \(\lambda\) un réel. Alors
  \begin{equation}
    \iint_{\Ddomaine}(\lambda f+g) = \lambda \iint_{\Ddomaine} f +
    \iint_{\Ddomaine} g.
  \end{equation}
\end{prop}
\begin{prop}[Additivité des domaines]
  Soient \(\Ddomaine_1\) et \(\Ddomaine_2\) des domaines simples du plan, \(f\)
  une fonction continue sur \(\Ddomaine_1 \cup \Ddomaine_2\). Alors
  \begin{equation}
    \iint_{\Ddomaine_1 \cup \Ddomaine_2} f = \iint_{\Ddomaine_1} f +
    \iint_{\Ddomaine_2} f.
  \end{equation}
\end{prop}
\begin{prop}[Croissance]
  Soient \(\Ddomaine\) un domaine simple du plan, \(f\) et \(g\) deux fonctions
  continues sur \(\Ddomaine\). Alors
  \begin{equation}
    f \leqslant g \implies \iint_{\Ddomaine} f \leqslant \iint_{\Ddomaine} f.
  \end{equation}
  En particulier si \(f \geqslant \tilde{0}\) alors \(\iint_{\Ddomaine} f
  \geqslant 0\).
\end{prop}

\subsection{Changements de variables}

\subsubsection{Notion de \(\classe{1}\)-difféomorphisme}

\begin{defdef}
  Soient \(U\) un ouvert de \(\R^2\) et
  \(\fonction{\varphi}{U}{\R^2}{(x,y)}{(\varphi_1(x,y),\varphi_2(x,y))}\). On
  suppose que \(\varphi\) est de classe \(\classe{1}\) sur \(U\) et on définit
  pour tout \((x,y) \in U\) la matrice jacobienne de \(\varphi\) en \((x,y)\)
  notée
  \begin{equation}
    J_\varphi(x,y) = \begin{pmatrix} \derivep{\varphi_1}{x}(x,y) &
    \derivep{\varphi_1}{y}(x,y) \\ \derivep{\varphi_2}{x}(x,y) &
    \derivep{\varphi_2}{y}(x,y) \end{pmatrix}.
  \end{equation}

  On appelle le jacobien (ou le déterminant jacobien) de \(\varphi\) au point
  \((x,y)\) le réel
  \begin{equation}
    \Det J_\varphi(x,y) = \derivep{\varphi_1}{x}(x,y)
    \derivep{\varphi_2}{y}(x,y) - \derivep{\varphi_2}{x}(x,y)
    \derivep{\varphi_1}{y}(x,y).
  \end{equation}
\end{defdef}
%
\begin{defdef}
  On dit que \(\varphi\) est un \(\classe{1}\)-difféomorphisme d'un ouvert \(U\)
  de \(\R^2\) sur un ouvert \(V\) de \(\R^2\) si et seulement si~:
  \begin{enumerate}
    \item \(\varphi\) est de classe \(\classe{1}\) sur \(U\) ;
    \item \(\varphi\) est bijective ;
    \item \(\varphi^{-1}\) est de classe \(\classe{1}\) sur \(V\).
  \end{enumerate}
\end{defdef}
%
\begin{prop}[Admise]
  Si \(\varphi\) est un \(\classe{1}\)-difféomorphisme de \(U\) sur \(V\), alors
  pour tout \((x,y) \in U^2\) la matrice jacobienne \(J_\varphi(x,y)\) est
  inversible et~:
  \begin{equation}
    J_\varphi(x,y)^{-1} = J_{\varphi^{-1}}(x,y).
  \end{equation}
\end{prop}

\subsubsection{Théorème de changement de variable}

\begin{theo}
  Soient \(U\) et \(V\) deux ouverts de \(\R^2\), \(\varphi\) un
  \(\classe{1}\)-difféomorphisme de \(U\) sur \(V\). Soit \(\Ddomaine\) un
  domaine simple de \(\R^2\) tel que \(\Ddomaine \subset U\) et \(\Delta =
  \varphi(\Ddomaine)\) soit un domaine simple inclus dans \(V\).

  Soit \(f\) une fonction continue sur \(\Ddomaine\) à valeurs dans \(\R\),
  alors
  \begin{equation}
    \iint_\Delta f(x,y)\diff x \diff y = \iint_\Ddomaine f(\varphi(x,y))
    \abs{\Det J_\varphi(x,y)}\diff x \diff y.
  \end{equation}
  \emph{Valeur absolue.}
\end{theo}

\subsubsection{Changement de variable affine}

\begin{theo}
  Soient quatre réels \(a\), \(b\), \(c\) et \(d\) tels que \(ad \neq bc\). On
  adopte comme nouvelles variables \(u\) et \(v\) telles que \(\begin{cases}
  x=au+bv \\ y=cu+dv \end{cases}\). Soit une bijection affine de classe
  \(\classe{1}\)~: \(\fonction{\varphi}{\R^2}{\R}{(u,v)}{(au+bv, cu+dv)}\).
  Ainsi que \(\varphi^{-1}\) (puisque \(ad-bc\neq 0\).

  Soit \(\Ddomaine\) un domaine simple de \(\R^2\),
  \(\Delta=\varphi(\Ddomaine)\) un domaine simple et \(f\) une fonction continue
  sur \(\Delta\). Alors
  \begin{equation}
    \iint_\Delta f(x,y)\diff x \diff y = \iint_\Ddomaine f(au+bv, cu+dv)
    \abs{ad-bc}\diff u \diff v.
  \end{equation}
\end{theo}

\subsubsection{Changement de variables en polaire}

\begin{theo}
  On adopte comme nouvelles variables \(r\) et \(\theta\) telles que
  \(\begin{cases} x=r\cos\theta \\ y=r\sin\theta \end{cases}\). Soient \(U\) et
    \(V\) deux ouverts de \(\R^2\) tels que \(\fonction{\varphi}{U}{V}{(r,
    \theta)}{(r\cos\theta, r\sin\theta)}\) soit un
    \(\classe{1}\)-difféomorphisme.

    Soit \(\Ddomaine\) un domaine simple de \(\R^2\) tel que \(\Ddomaine \subset
    U\) et \(\Delta=\varphi(\Ddomaine)\) soit un domaine simple inclus dans
    \(V\). Soit \(f\) une fonction continue sur \(\Ddomaine\), alors
    \begin{equation}
      \iint_\Delta f(x,y)\diff x \diff y = \iint_\Ddomaine f(r\cos\theta,
      r\sin\theta) \abs{r}\diff r \diff \theta.
    \end{equation}
\end{theo}

\emph{En pratique}~:
\begin{itemize}
  \item toujours dessiner les domaines \(\Ddomaine\) et \(\Delta\) pour éviter
    de se tromper;
  \item essayer, pour des changements de variables en polaire, de choisir les
    domaines de sorte que \(r>0\).
\end{itemize}

\emph{Exemple --- Aire d'un disque}~: Soit \(\Delta\) le disque de centre \(O\)
et de rayon \(R>0\), tels que
\begin{align}
  \text{Aire}(\Delta) = \iint_\Delta \diff x \diff y \quad
  \Delta=\enstq{(x,y)\in\R^2}{x^2+y^2\leqslant R^2} \\
  \fonction{\varphi}{\Ddomaine}{\Delta}{(r, \theta)}{(r\cos\theta, r\sin\theta)}
  \quad \Ddomaine=\enstq{(r,\theta)\in\R^2}{0 \leqslant r \leqslant R \ 0
  \leqslant \theta \leqslant 2\pi}
\end{align}
Alors
\begin{align}
  \text{Aire}(\Delta) &= \iint_\Ddomaine \abs{r}\diff r \diff \theta \\
  &= \iint_\Ddomaine r\diff r \diff \theta \\
  &=\int_0^R r\diff r \int_0^{2\pi}\diff \theta \\
  &=\pi R^2.
\end{align}

\section{Champs de vecteurs et champs de scalaires}\label{sec:champvec}
On suppose que \(n \in \{2, 3\}\) et on considère l'espace vectoriel \(\R^n\)
muni de sa structure canonique d'espace euclidien. On note \(e=(e_i)_{1
\leqslant i \leqslant n}\) sa base canonique. Soit \(U\) un ouvert de \(\R^n\).

\subsection{Gradient d'un champ de scalaires}

On appelle champ de scalaire sur \(U\) toute application \(f \in \R^U\) de
classe \(\classe{1}\). On rappelle que le gradient de \(f\) au point \(M \in U\)
est défini par
\begin{equation}
  \grad{f}(M) = \sum_{i=1}^n \derivep{f}{x_i} \vect{e_i}.
\end{equation}

\subsection{Champ de vecteur}

On appelle champ de vecteurs sur \(U\) toute application de \(U\) dans \(\R^n\).

\emph{Exemple}~: Si \(f\) est un champ de scalaire sur \(U\) de classe
\(\classe{1}\), alors le gradient \(\grad{f}\) est un champ de vecteurs sur
\(U\).


\subsection{Divergence}

Soit \(\vect{F} : U \longrightarrow \R^n\) un champ de scalaires sur \(U\) de
classe \(\classe{1}\).

Si \(n=2\), on note pour tout \(M(x,y) \in U\) \(\vect{F}=P(x,y)\vect{e_1} +
Q(x,y)\vect{e_2}\).

Si \(n=3\), on note pour tout \(M(x,y,z) \in U\) \(\vect{F}=P(x,y,z)\vect{e_1} +
Q(x,y,z)\vect{e_2}+R(x,y,z)\vect{e_3}\).

\begin{defdef}
  On appelle divergence du champ de vecteurs \(\vect{F}\) au point \(M \in U\),
  le réel noté \(\divgc \vect{F}(M)\) défini par~:
  \begin{itemize}
    \item si \(n=2\), \(M=(x,y)\) \(\divgc \vect{F}(M)=\derivep{P}{x}(x,y) +
      \derivep{Q}{y}(x,y)\);
    \item si \(n=3\), \(M=(x,y,z)\) \(\divgc \vect{F}(M)=\derivep{P}{x}(x,y,z) +
      \derivep{Q}{y}(x,y,z) + \derivep{R}{z}(x,y,z)\).
  \end{itemize}
  L'application \(\fonction{\divgc}{U}{\R}{M}{\divgc \vect{F}(M)}\) est un champ
  de scalaire
\end{defdef}
%
\begin{prop}
  Pour tout point \(M\) de \(U\), l'application divergence notée \(\divgc\),
  \(\fonction{\divgc}{\classe{1}(U, \R^n)}{\R}{\vect{F}}{\divgc\vect{F}(M)}\)
  est une forme linéaire. Autrement dit pour tout champs de vecteurs \(F\) et
  \(G\) dans \(\classe{1}(U, \R^n)\) et pour tout réel \(\lambda\), on a
  \begin{equation}
    \divgc(\lambda \vect{F} + \vect{G})(M) = \lambda \divgc \vect{F}(M) + \divgc
    \vect{G}(M)
  \end{equation}
\end{prop}
\begin{proof}
  C'est une conséquence de la linéarité des dérivées partielles.
\end{proof}
%
\begin{prop}
  Soient \(\vect{F} \in \classe{1}(U, \R^n)\) un champ de vecteurs et \(\varphi
  \in \classe{1}(U, \R)\). Alors pour tout point \(M \in U\), on a
  \begin{equation}
    \divgc(\varphi \vect{F})(M)=\prodscal{\grad{\varphi}}{\vect{F}}(M) +
    (\varphi \divgc \vect{F})(M).
  \end{equation}
\end{prop}
\begin{proof}
  Dans le cas où \(n=2\) avec les notations précédentes, on a
  \begin{align}
    (\varphi \vect{F})(M)& = (\varphi(M) P(M), \varphi(M) Q(M)) \\
    \divgc(\varphi \vect{F})(M)& = \derivep{\varphi}{x}(M)  P(M) + \varphi(M)
    \derivep{P}{x}(M) + \derivep{\varphi}{y}(M)  Q(M) + \varphi(M)
    \derivep{P}{y}(M) \\
    &=\prodscal{\grad{\varphi}(M)}{\vect{F}(M)} + \varphi(M)\divgc\vect{F}(M)
  \end{align}
\end{proof}

\subsection{Rotationnel}

Dans cette sous-section, \(n=3\).

\begin{defdef}
  Soit \(\vect{F} \in \classe{1}(U, \R^3)\) un champ de vecteur. Pour tout point
  \(M \in U\), on définit un vecteur appelé rotationnel de \(\vect{F}\) au point
  \(M\), noté \(\rot \vect{F}(M)\), défini par
  \begin{equation}
    \rot \vect{F}(M) = \begin{pmatrix} \derivep{R}{y}(M) - \derivep{Q}{z}(M) \\
    \derivep{P}{z}(M) - \derivep{R}{x}(M) \\ \derivep{Q}{x}(M) -
    \derivep{P}{y}(M)\end{pmatrix}.
  \end{equation}
\end{defdef}

L'application \(\fonction{\rot \vect{F}}{U}{\R^n}{M}{\rot \vect{F}(M)}\) est un
champ de vecteurs.

\begin{prop}
  Soient \(\vect{F} \in \classe{1}(U, \R^3)\) un champ de vecteurs et
  \(\varphi\) un champs de scalaire de classe \(\classe{1}(U, \R)\). Alors pour
  tout point \(M \in U\), on a
  \begin{equation}
    \rot(\varphi \vect{F})(M) = \grad \varphi(M) \wedge \vect{F}(M) +
    \varphi(M)\rot(\vect{F})(M).
  \end{equation}
\end{prop}
\begin{proof}
  Soit \(M \in U\). On a
  \begin{equation}
    (\varphi \vect{F})(M) = \begin{pmatrix} \varphi(M)P(M) \\ \varphi(M)Q(M) \\
    \varphi(M)R(M)\end{pmatrix}.
  \end{equation}
  Alors
  \begin{align}
    \rot(\varphi \vect{F})(M)) &= \begin{pmatrix} \derivep{\varphi}{y}(M)R(M) +
    \varphi(M)\derivep{R}{y}(M) - \derivep{\varphi}{z}(M)Q(M) -
    \derivep{Q}{z}(M)\varphi(M) \\ \derivep{\varphi}{z}(M)P(M) +
    \varphi(M)\derivep{P}{z}(M) - \derivep{\varphi}{x}(M)R(M) -
    \derivep{R}{x}(M)\varphi(M) \\ \derivep{\varphi}{x}(M)Q(M) +
    \varphi(M)\derivep{Q}{x}(M) - \derivep{\varphi}{y}(M)P(M) -
    \derivep{P}{y}(M)\varphi(M)\end{pmatrix} \\
    &=\begin{pmatrix} \derivep{\varphi}{y}(M)R(M) - \derivep{\varphi}{z}(M)Q(M)
    + \varphi(M)\left(\derivep{R}{y}(M)  - \derivep{Q}{z}(M)\right) \\
    \derivep{\varphi}{z}(M)P(M) - \derivep{\varphi}{x}(M)R(M) +
    \varphi(M)\left(\derivep{P}{z}(M) - \derivep{R}{x}(M)\right) \\
    \derivep{\varphi}{x}(M)Q(M) -\derivep{\varphi}{y}(M)P(M) +
    \varphi(M)\left(\derivep{Q}{x}(M)  - \derivep{P}{y}(M)\right)\end{pmatrix}\\
    &=\grad \varphi(M) \wedge \vect{F}(M) + \varphi(M)\rot(\vect{F})(M).
  \end{align}
\end{proof}
%
\begin{prop}
  Soit \(\vect{F} \in \classe{1}(U, \R^3)\) un champ de vecteurs et soit
  \(\vu\in\R^3\), alors
  \begin{equation}
    \divgc(\vect{F}\wedge \vu)(M) = \prodscal{\rot \vect{F}(M)}{\vu}.
  \end{equation}
\end{prop}
\begin{proof}
  Si on note \(\vu =(a, b, c)\) alors
  \begin{equation}
    \vect{F}(M) \wedge \vu = \begin{pmatrix} cQ-bR \\ aR-cP
    \\bP-aQ\end{pmatrix}(M).
  \end{equation}
  En calculant la divergence~:
  \begin{align}
    \divgc(\vect{F} \wedge \vu)(M) &=c \derivep{Q}{x}(M) -b\derivep{R}{x}(M) + a
    \derivep{R}{y}(M) \notag\\
    &-c\derivep{P}{y}(M) + b \derivep{P}{z}(M) -a\derivep{Q}{z}(M) \\
    &=\prodscal{\rot \vect{F}(M)}{\vu}.
  \end{align}
\end{proof}

\subsection{Potentiel scalaire}

\begin{defdef}
  Soient \(\vect{F} \in \classe{1}(U, \R^n)\) un champ de vecteurs. On dit que
  \(\vect{F}\) dérive d'un potentiel scalaire s'il existe un champ de scalaires
  \(f \in \classe{1}(U, \R)\) tel que \(\vect{F}=\grad{f}\).
\end{defdef}
\begin{prop}[Dans le cas où \(n=3\)]
  Tout champ de vecteurs de classe \(\classe{1}\) qui dérive d'un potentiel
  scalaire admet un rotationnel nul.
\end{prop}
\begin{proof}
  Notons \(\vect{F}=\grad f\) avec \(f \in \classe{1}(U, \R)\). Pour tout \(M
  \in U\), on a
  \begin{equation}
    \vect{F}(M) = \derivep{f}{x}(M)\vect{e_1} +  \derivep{f}{y}(M)\vect{e_2} +
    \derivep{f}{z}(M)\vect{e_3}.
  \end{equation}
  Lorsqu'on calcule la première composante du rotationnel de \(\vect{F}\),
  puisque \(f\) est de classe \(\classe{2}\), on a
  \begin{equation}
    \derivepc{f}{y}{z}(M)-\derivepc{f}{z}{y}(M)=0,
  \end{equation}
  d'après le théorème de Schwarz.

  Le calcul donne le même résultat pour la deuxième et troisième composante du
  vecteur, de telle sorte que \(\rot\vect{F}(M)=\vect{0}\).
\end{proof}

\begin{cor}
  Pour tout champ scalaire de classe \(\classe{1}\) \(f \in classe{1}(U, \R)\),
  on a pour tout point \(M \in U\)
  \begin{equation}
    \rot\grad f(M)=\vect{0}.
  \end{equation}
\end{cor}

\begin{defdef}
  On appelle ouvert étoilé de \(\R^n\) tout ouvert \(U\) de \(\R^n\) pour lequel
  il existe \(\Omega \in U\) tel que pour tout \(M \in U\), \([\Omega M] \subset
  U\).
\end{defdef}

En particulier les convexes sont étoilés.

\begin{theo}[Admis]
  Soit \(U\) un ouvert étoilé de \(\R^3\), \(\vect{F} \in \classe{1}(U, \R^3)\)
  un champ de vecteurs dont le rotationnel est identiquement nul sur \(U\).
  Alors \(\vect{F}\) dérive d'un potentiel scalaire.
\end{theo}

\subsection{Laplacien}

\begin{defdef}
  Soit un champ de scalaires \(f \in \classe{1}(U, \R)\), on appelle laplacien
  de \(f\) au point \(M\), le réel noté \(\Delta f(M)\), défini par~:
  \begin{equation}
    \Delta f(M) = \divgc(\grad f)(M) = \sum_{i=1}^n \deriveps{f}{x_i}(M).
  \end{equation} \end{defdef}
  \begin{defdef}
    Avec les mêmes hypothèse, on dit que \(f\) est harmonique si son laplacien
    est nul sur \(U\).
  \end{defdef}

  \subsection{Opérateur Nabla}

  D'un point de vue mnémotechnique, posons
  \begin{equation}
    \vect{\nabla} = \sum_{i=1}^n \derivep{}{x_i}\vect{e_i}.
  \end{equation}
  Alors, formellement, on a
  \begin{align}
    \grad f &= \vect{\nabla} f \\
    \divgc \vect{F} &= \prodscal{\vect{\nabla}}{\vect{F}} \\
    \rot \vect{F} &= \vect{\nabla} \wedge \vect{F} \\
    \Delta f &= \vect{\nabla}^2 f.
  \end{align}

  \section{Intégrale curviligne}

  Supposons que la notion de courbe paramétrée s'étend à l'espace (\(n \in \{2,
  3\}\)). Soient \(U\) un ouvert de \(\R^n\), \(\Gamma=([a,b], f)\) un arc
  paramétré de \(\R^n\), supposé de classe \(\classe{1}\), régulier et orienté,
  dont le support est inclus dans l'ouvert \(U\). Soit \(\vect{F}\) un champ de
  vecteurs continu sur \(U\).

  \begin{prop}
    L'intégrale \(\int_a^b \prodscal{\vect{F}(f(t))}{\vect{f'(t)}} \diff t\) ne
    dépend pas du paramétrage choisi de l'arc \(\Gamma\).
  \end{prop}
  \begin{defdef}
    Cette intégrale est appelée intégrale curviligne (ou circulation) de
    \(\vect{F}\) sur \(\Gamma\), elle est notée \(\oint_\Gamma
    \vect{F}(M)\vect{\diff M}\).
  \end{defdef}
  \begin{proof}
    Soit \(([\alpha, \beta], g)\) un paramétrage admissible de \(\Gamma\). Il
    existe un \(\classe{1}\)-difféomorphisme \(\varphi : [a, b] \rightarrow
    [\alpha, \beta]\) strictement croissant tel que \(f=g \circ \varphi\).
    Alors
    \begin{align}
      \int_a^b \prodscal{\vect{F}(f(t))}{\vect{f'(t)}} \diff t &= \int_a^b
      \prodscal{\vect{F}(g \circ \varphi(t))}{\varphi'(t)\vect{g'(\varphi(t))}}
      \diff t \\
      &=\int_a^b \prodscal{\vect{F}(g \circ \varphi(t))}{\vect{g'(\varphi(t))}}
      \varphi'(t)\diff t \\
      &=\int_\alpha^\beta \prodscal{\vect{F}(g(u))}{\vect{g'(u)}} \diff u.
    \end{align}
  \end{proof}

  \begin{prop}
    Sous les mêmes hypothèses, on suppose de plus que le champ de vecteurs
    \(\vect{F}\) dérive d'un potentiel scalaire \(h\). Alors
    \begin{equation}
      \oint_\Gamma \vect{F}(M)\vect{\diff M} = h(B)-h(A),
    \end{equation}
    où \(A=f(a)\) et \(B=f(B)\) sont les extrémités de la courbes.
  \end{prop}

  Particulièrement la circulation d'un champ de vecteurs qui dérive d'un
  potentiel sur une courbe fermée est nulle.

  \begin{proof}[Dans le cas où \(n=2\)]
    \begin{align}
      \oint_\Gamma \vect{F}(M)\vect{\diff M}  &= \int_a^b
      \prodscal{\vect{F}(f(t))}{\vect{f'(t)}} \diff t \\
      &= \int_a^b \prodscal{\grad h(x(t), y(t))}{(x'(t), y'(t))} \diff t  \\
      &= \int_a^b \derivep{h}{x}(x(t), y(t))x'(t) + \derivep{h}{y}(x(t),
      y(t))y'(t) \diff t \\
      &=\int_a^b (h \circ f)'(t)\diff t \\
      &=h\circ f(b)-h\circ f(a) \\
      &=h(B)-h(A).
    \end{align}
  \end{proof}

  \section{Formule de Green-Riemann}

  \begin{theo}[Admis]
    Soit \(\Ddomaine\) un domaine simple de \(\R^2\) dont la frontière
    \(\Gamma\) est supposée de classe \(\classe{1}\) par morceaux et orientée
    positivement.

    Soient \(P\) et \(Q\) deux fonctions de classe \(\classe{1}\) sur un ouvert
    \(U\) qui contient \(\Ddomaine\). Alors
    \begin{equation}
      \oint_\Gamma P(x,y) \diff x +Q(x,y)\diff y = \iint_\Ddomaine
      \left(\derivep{Q}{x}(x,y) - \derivep{Q}{y}(x,y)\right)\diff x \diff y.
    \end{equation}
  \end{theo}
