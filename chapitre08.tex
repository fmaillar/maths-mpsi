\chapter{Géométrie élémentaire de l'espace}
\label{chap:geomEspace}
\minitoc
\minilof
\minilot
\section{Modes de repérage dans l'espace}
Soit \(\E\) l'espace géométrique euclidien. On munit \(\E\) d'un repère cartésien \(\Rep=\rondtrois\).

\subsection{Coordonnées cartésiennes}
\begin{defdef}
  Soit M un point de l'espace \(\E\). Il existe un unique triplet \((x,y,z)\in\R^3\) tel que \(\vect{OM}=x\vi+y\vj+z\vk\), ce sont les coordonnées cartésiennes du point M dans le repère \(\rondtrois\). Réciproquement pour tout triplet \((x,y,z)\) de \(\R^3\) il existe un unique point M de l'espace tel que \(\vect{OM}=x\vi+y\vj+z\vk\)
\end{defdef}
Ainsi la donnée d'un repère cartésien de l'espace permet de définir une 
bijection entre \(\R^3\) et \(\E\). La représentation du repère cartésien est 
donné par la figure~\ref{fig:repcart}.

\emph{Changement de repère en coordonnées cartésiennes}
Soient \(\Rep=\rondtrois\) et \(\Rep'=\rondtroisu\) deux repères cartésiens. Dans \(\Rep\) \(\Omega(x_0,y_0,z_0)\) et les vecteurs \(\vu(a,b,c)\) \(\vv(a',b',c')\) et \(\vw(a'',b'',c'')\). Soit M un point de l'espace. On note \((x_0,y_0,z_0)\) ses coordonnées dans \(\Rep\) et \((X,Y,Z)\) ses coordonnées dans \(\Rep'\). D'une part~:
\begin{equation}
  \vect{OM}=x\vi+y\vj+z\vk,
\end{equation}
et d'autre part~:
\begin{equation}
  \vect{OM}=\vect{O\Omega}+\vect{\Omega M}.
\end{equation}
Alors par unicité des coordonnées dans \(\Rep\) on a~:
\begin{equation}
  \begin{cases}
    x=x_0+aX+a'Y+a''Z\\
    y=y_0+bX+b'Y+b''Z\\
    z=z_0+cX+c'Y+c''Z
  \end{cases}.
\end{equation}

\danger Ces formules donnent les \og{}anciennes\fg{} coordonnées en fonction des \og{}nouvelles\fg{}.

\begin{figure}
    \centering
    \includegraphics[scale=1]{coord-cartesiennes.png}
    \caption{Représentation des coordonnées cartésiennes}
    \label{fig:repcart}
\end{figure}

\subsection{Coordonnées cylindriques}
\label{subsec:coordcyl}
Pour tout réel \(\varphi\) on définit 
\(\begin{cases}\vu_\varphi=\cos\varphi\vi+\sin\varphi\vj\\ 
\vv_\varphi=-\sin\varphi\vi+\cos\varphi\vj\end{cases}\). Soit un point \(M\) de 
l'espace de coordonnées \((x,y,z)\) et \(P\) le projeté orthogonal de \(M\) sur 
le plan \(\rond\), \(P(x,y,0)\). Soit \((r,\varphi)\) un système de coordonnées 
polaires de \(P\), alors \(\vect{OP}=r\vu_\varphi\) et donc 
\(\vect{OM}=r\vu_\varphi+z\vk\). Une représentation graphique des du repère 
cylindrique est donné par la figure~\ref{fig:repcyl}.
\begin{defdef}
  Soit \(M\) un point de l'espace \(\E\), on appelle système de coordonnées cylindriques de \(M\) dans \(\Rep\) tout triplet de réels \((r,\varphi,z)\) avec \(r>0\) tel que \(\vect{OM}=r\vu_\varphi+z\vk\). Alors \(\vect{OM}=r\cos\varphi\vi+r\sin\varphi\vj+z\vk\), on peut retrouver les coordonnées cartésiennes. On a \(r=OP\), l'unicité de \(z\) est claire car c'est la conséquence de l'unicité des coordonnées cartésiennes. 
\begin{itemize}
\item Si \(M\notin(Oz)\) alors \(P\neq O\) et donc \(\congru{\varphi}{\widehat{(\vi;\vect{OP})}}{2\pi}\);
\item par contre si \(M\in(Oz)\) alors on définit le repère cylindrique associé à \(M\), \((0,\vu_\varphi,\vv_\varphi,\vk)\) et c'est un repère orthonormal direct de l'espace.
\end{itemize}
\end{defdef}

\begin{figure}
    \centering
    \includegraphics[scale=1]{coord-cylindriques.png}
    \caption{Représentation des coordonnées cylindriques}
    \label{fig:repcyl}
\end{figure}


\emph{Remarque}~: Soit \(r_0>0\), l'ensemble des points \(M\) de l'espace dont un système de coordonnées cylindriques est \((r_0,\varphi,z)\) lorsque \((\varphi,z)\) varie dans \(\R^2\) est un cylindre de rayon \(r_0\) d'axe \((Oz)\) d'où le nom de coordonnées cylindriques.


\subsection{Coordonnées sphériques}
On conserve les notations de la sous-section~\ref{subsec:coordcyl}. On sait que \(\vect{OM}=r\vu_\varphi+z\vk\), donc si on note \(\rho=OM\) on a \(\rho=\sqrt{r^2+z^2}\). Un représentation des coordonnées sphériques est donnée par la figure \ref{fig:repsphere}
\begin{itemize}
\item Si le point \(M\) est différent de l'origine alors \(\rho>0\) et \(\frac{r^2}{\rho^2}+\frac{z^2}{\rho^2}=1\); alors il existe \(\theta\in \intervalleff{0}{2\pi}\) tel que \(\begin{cases}r=\rho\sin\theta\\ z=\rho\cos\theta\end{cases}\) on a \(r\geqslant 0\) et \(z\geqslant 0\) donc on peut choisir \(\theta\in \intervalleff{0}{\pi}\) et on a \(\vect{OM}=\rho\vu_{\varphi,\theta}\) où \(\vu_{\varphi,\theta}=\sin\theta\vu_\varphi+\cos\theta\vk\).
\item Si le point \(M\) est l'origine alors \(\rho=r=z=0\), on peut prendre n'importe quel \(\theta \in \intervalleff{0}{\pi}\) et on a encore \(\begin{cases}r=\rho\sin\theta\\ z=\rho\cos\theta\end{cases}\).
\end{itemize}
\begin{defdef}
  Étant donné un point \(M\) de l'espace, on appelle système de coordonnées sphérique de \(M\) tout triplet \((\rho,\varphi,\theta)\in\Rpluss \times \R \times \intervalleff{0}{\pi}\) tel que \(\vect{OM}=\rho\vu_{\varphi,\theta}\). On appelle \(r\) la rayon, \(\theta\) la colatitude (\(\frac{\pi}{2}-\theta\) est la latitude) et \(\varphi\) la longitude.
\end{defdef}
Alors \(\vect{OM}=\rho\vu_{\varphi,\theta}=\rho\sin\theta\cos\varphi\vi+\rho\sin\theta\sin\varphi\vj+\rho\cos\theta\vk\). On peut retrouver les coordonnées cartésiennes à partir des coordonnées cylindriques grâce à~:
\begin{equation}
  \begin{cases}
    x&=\rho\sin\theta\cos\varphi\\ y&=\rho\sin\theta\sin\varphi\\ z&=\rho\cos\theta
  \end{cases}
\quad
  \begin{cases}
    r&=\rho\sin\theta\\ \varphi&=\varphi\\ z&=\rho\cos\theta
  \end{cases}.
\end{equation}
On a \(\rho=OM\) et si \(M\) est différent de l'origine alors \(\theta\) est la mesure de l'angle non-orienté \((\vect{z},\vect{OM})\) dans l'espace (défini de manière unique dans \(\intervalleff{0}{\pi}\)).
\begin{itemize}
\item Si le point \(M\) est sur l'axe \((Oz)\) alors l'angle \(\varphi\) est unique modulo \(2\pi\)~: \(\congru{\varphi}{(\vi,\vect{OM})}{2\pi}\);
\item par contre si le point \(M\) n'est pas sur cet axe alors on définit le repère sphérique attaché à \(M\), c'est le ROND \((O,\vu_{\varphi,\theta},\vv_{\varphi,\theta},\vw_{\varphi,\theta})\) où \(\vv_{\varphi,\theta}=\vv_{\varphi}\) et \(\vw_{\varphi,\theta}=-\cos\theta\vu_\varphi+\sin\theta\vk\).
\end{itemize}

\emph{Remarque}~: Soit \(\rho_0>0\), l'ensemble des points \(M\) de l'espace \(\E\) dont un système de coordonnées sphérique est \((\rho_0,\varphi,\theta)\) avec \(\varphi\in\R\) et \(\theta\in \intervalleff{0}{\pi}\) est la sphère de centre \(O\) et de rayon \(r_0\) (d'où le nom de coordonnées sphériques).

\begin{figure}
    \centering
    \includegraphics[scale=1]{coord-spheriques.png}
    \caption{Représentation des coordonnées sphériques}
    \label{fig:repsphere}
\end{figure}


\section{Produit scalaire}
\subsection{Définition géométrique}
\begin{defdef}
  Soient \(\vu\) et \(\vv\) deux vecteurs de l'espace. On définit le réel \(\vu\cdot\vv\) appelé \emph{produit scalaire} des vecteurs \(\vu\) et \(\vv\) par~:
  \begin{itemize}
  \item Si \(\vu=0\) ou \(\vv=0\) on pose \(\vu\cdot\vv=0\);
  \item si \(\vu\neq 0\) ou \(\vv\neq 0\) on pose \(\vu\cdot\vv=\norme{\vu}\norme{\vv}\cos(\widehat{\vu,\vv})\).
  \end{itemize}
\end{defdef}
\begin{prop}
Soient \(\vu\) et \(\vv\) deux vecteurs de l'espace, alors~:
\begin{equation}
  \vu\cdot\vv=0 \iff \vu \bot \vv.
\end{equation}
\end{prop}
\begin{proof}
  Soient \(\vu\) et \(\vv\) deux vecteurs de l'espace, alors~:
  \begin{align}
    \vu \bot \vv &\iff \vu=0 \text{~ou~} \vv=0 \text{~ou~} (\widehat{\vu,\vv})=\frac{\pi}{2}\\
&\iff \vu=0 \text{~ou~} \vv=0 \text{~ou~} \cos(\widehat{\vu,\vv})=0\\
&\iff \vu\cdot\vv=0.
  \end{align}
\end{proof}

\emph{Remarque}~: Soient \(\vu\) et \(\vv\) deux vecteurs de l'espace. Si on se place dans le plan qui contient ces deux vecteurs, on comprend que la définition du produit scalaire dans le plan coïncide avec la définition du produit scalaire dans l'espace.

\subsection{Propriétés algébriques}
\begin{prop}
  \label{prop:propalgprodsc}
  pour tout vecteur \(\vu\), \(\vv\), et \(\vw\) de l'espace \(\E\) et pour tout réel \(\lambda\) on a~:
  \begin{enumerate}
  \item \(\vu\cdot\vv=\vv\cdot\vu\), c'est la symétrie;
  \item \(\vu\cdot(\lambda\vv)=\lambda(\vu\cdot\vv)\) et \(\vu\cdot(\vu+\vw)=\vu\cdot\vv+\vu\cdot\vw\), c'est la linéarité à gauche;
    \item \((\lambda\vu)\cdot\vv=\lambda(\vu\cdot\vv)\) et \((\vu+\vv)\cdot\vw=\vu\cdot\vv+\vu\cdot\vw\), c'est la linéarité à droite;
    \item \(\vu\cdot\vu=\norme{\vu}^2\).
  \end{enumerate}
\end{prop}
\begin{proof}
  La démonstration est donnée au chapitre~\ref{chap:geomplan}.
\end{proof}
\begin{prop}
  Soit \(\rondtrois\) une BOND de l'espace et \(\vu\) et \(\vv\) deux vecteurs de coordonnées respectives \((x,y,z)\) et \((x',y',z')\) dans cette BON\@. Alors~:
  \begin{equation}
    \vu\cdot\vv=xx'+yy'+zz'.
  \end{equation}
\end{prop}
\begin{proof}
  De la même manière que pour la proposition~\ref{prop:propalgprodsc}, la démonstration est la même qu'au chapitre~\ref{chap:geomplan}.
\end{proof}

\emph{Distance de deux points dans un repère orthonormal}~: Soit \(\Rep=\rondtrois\) un ROND et deux points de l'espace \(A(a,b,c)\) et \(B(a',b',c')\) dans \(\Rep\) alors \(AB=\sqrt{(a-a')^2+(b-b')^2+(c-c')^2}\) puisque \(\vect{AB}\cdot\vect{AB}=AB^2\).

\section{Produit vectoriel}
\label{sec:prodvec}
\subsection{Définition géométrique}
Il y a plusieurs façons de définir le produit vectoriel~:
\begin{itemize}
\item on peut le définir \emph{en coordonnées}, c'est-à-dire en écrivant ses coordonnées en fonction de celle de \(\vu\) et \(\vv\) dans une BOND;
\item on peut aussi le définir de manière théorique à partir du déterminant;
\item on peut finalement le définir de façon géométrique qu'on effectuera ici (selon les indications du programme).
\end{itemize}
Dans ce chapitre, on admet certains résultats concernant le produit vectoriel.
\begin{prop}[admise]
 Soient deux vecteurs \(\vu\) et \(\vv\) de l'espace non colinéaires. Il existe un troisième vecteur \(\vw\) directement orthogonal au plan engendré par les deux premiers vecteurs. De plus les vecteurs directement orthogonaux à \((\vu,\vv)\) sont alors engendrés par \(\vw\), c'est-à-dire qu'ils sont tous colinéaires à \(\vw\).
\end{prop}
\begin{defdef}
  Soient deux vecteurs \(\vu, \vv\) de l'espace. On définit le vecteur \(\vu \wedge \vv\) appelé produit vectoriel de \(\vu\) et \(\vv\) par~:
  \begin{itemize}
  \item si \(\vu\) et \(\vv\)  sont colinéaires alors \(\vu \wedge \vv=0\);
  \item sinon \(\vu \wedge \vv\) est l'unique vecteur directement orthogonal à \((\vu,\vv)\) de norme égale à \(\norme{\vu}\norme{\vv}\sin(\widehat{\vu,\vv})\).
  \end{itemize}
\end{defdef}

\subsubsection{Remarque}
Notons que si \(\vu\) et \(\vv\) ne sont pas colinéaires alors en particulier ils sont non nuls et donc \(\norme{\vu}>0\), \(\norme{\vv}>0\) et \((\widehat{\vu,\vv})\in \intervalleoo{0}{\pi}\). Ainsi \(\sin(\widehat{\vu,\vv})>0\). La quantité \(\norme{\vu}\norme{\vv}\sin(\widehat{\vu,\vv})\) est donc positive.
\begin{prop}
  \label{prop:1}
  Soient \(\vu\) et \(\vv\) deux vecteurs de \(\E\), alors~:
  \begin{enumerate}
  \item \(\vu \wedge \vv \bot (\vu,\vv)\);
  \item \(\vu \wedge \vv = 0 \iff \vu \textrm{ et } \vv \text{ sont colinéaires}\).
  \end{enumerate}
\end{prop}
\begin{proof}
  \begin{enumerate}
  \item si \(\vu\) et \(\vv\) sont colinéaires alors \(\vu \wedge \vv = O\) est orthogonal à tout vecteur; sinon, c'est la définition.
  \item
    \begin{itemize}
    \item[\(\impliedby\)] C'est la définition.
    \item[\(\implies\)] Si \(\vu\) et \(\vv\) ne sont pas colinéaires, on a vu que \(\vu \wedge \vv\) possède une norme strictement positive donc il n'est pas nul.\qedhere
    \end{itemize}
  \end{enumerate}
\end{proof}

\subsubsection{Interprétation géométrique}
Si \(\vu\) et \(\vv\) sont deux vecteurs non colinéaires, \(\norme{\vu\wedge\vv}\) est l'aire du parallélogramme construit sur ces deux vecteurs. Si \(ABC\) est un triangle non plat, alors \(\frac{1}{2}\norme{\vect{AB}\wedge\vect{AC}}\) est son aire.
\begin{prop}
  Soit \(\bondtrois\) une BON de l'espace, alors~:
  \begin{align}
    \bondtrois\text{~est directe~}\iff \vk=\vi\wedge\vj\\
    \bondtrois\text{~est indirecte~}\iff \vk=-\vi\wedge\vj.
  \end{align}
\end{prop}
\begin{proof}
  En effet, on a \(\norme{\vi\wedge\vj}=1\) et on sait que \(\vk\) est orthogonal à \((\vi,\vj)\) de norme \(1\), donc \(\vk=\pm\vi\wedge\vj\).  
\end{proof}

\subsection{Propriétés algébriques}
\begin{prop}
  pour tout vecteurs \(\vu\), \(\vv\), \(\vw\) de l'espace \(\E\) et tout réel \(\lambda\) on a~:
  \begin{enumerate}
  \item l'antisymétrie~: \(\vu\wedge\vv=-\vv\wedge\vu\);
  \item la linéarité à droite~: \(\vu\wedge(\lambda\vv+\vw)=\lambda\vu\wedge\vv+\vu\wedge\vw\);
  \item la linéarité à gauche~: \((\vu+\lambda\vv)\wedge\vw=\vu\wedge\vw+\lambda\vv\wedge\vw\).
  \end{enumerate}
\end{prop}
\begin{prop}
  Si deux vecteurs \(\vu\) et \(\vv\) sont orthogonaux, alors~:
\begin{equation}
  \norme{\vu\wedge\vv}=\norme{\vu}\cdot\norme{\vv}.
\end{equation}
\end{prop}
\begin{proof}
  Si l'un des deux vecteur est nul, alors c'est trivial; sinon ils sont orthogonaux et non nuls et donc non colinéaires et \(\sin(\vu,\vv)=1\).
\end{proof}
\begin{prop}
  Soit deux vecteurs quelconques \(\vu\) et \(\vv\), alors~:
  \begin{equation}
    \norme{\vu\wedge\vv}^2+(\vu\cdot\vv)^2=\norme{\vu}^2\cdot\norme{\vv}^2.
  \end{equation}
\end{prop}
\begin{proof}
  \begin{enumerate}
  \item Si \(\vu\) et \(\vv\) sont colinéaires, alors soit \(\vu=0\) soit il existe un réel \(\lambda\) tel que \(\vu=\lambda\vv\) si et seulement s'il existe  un réel \(\lambda\) tel que \(\vu=\lambda\vv\) ou qu'il existe un autre réel \(\mu\) tel que \(\vv=\mu\vu\). Quitte à intervertir \(\vu\) et \(\vv\), on peut supposer qu'il existe un réel \(\lambda\) tel que \(\vv=\lambda\vu\). Alors~:
    \begin{align}
      \norme{\vu\wedge\vv}^2+(\vu\cdot\vv)^2&=0+\lambda^2\norme{\vu}^4\\
      &=\norme{\lambda\vu}^2\cdot\norme{\vu}^2\\
      &=\norme{\vu}^2\cdot\norme{\vv}^2
    \end{align}
  \item S'ils ne sont pas colinéaires, alors ils sont tous les deux non nuls et donc puisque \(\sin^2+\cos^2=1\) on a bien l'égalité.
  \end{enumerate}
\end{proof}
\begin{prop}
  Soit \(\bondtrois\) une BOND, \(\vu\) et \(\vv\) deux vecteurs de coordonnées respectives \((x,y,z)\) et \((x',y',z')\) dans cette base. Alors \(\vu\wedge\vv\) a pour coordonnée \((yz'-y'z,x'z-xz',xy'-yx')\)
\end{prop}
\begin{proof}
  Puisque \(\vu=x\vi+y\vj+z\vk\) et \(\vv=x'\vi+y'\vj+z'\vk\) alors \(\vu\wedge\vv=(xy'-x'y)\vi\wedge\vj+(xz'-zx')\vi\wedge\vk+(yz'-zy')\vj\wedge\vk\) et comme \(\bondtrois\) est une base orthonormale directe on a \(\vu\wedge\vv=(xy'-x'y)\vk-(xz'-zx')\vj+(yz'-zy')\vi\).
\end{proof}
\begin{prop}[Double produit vectoriel]
  Soient trois vecteurs \(\vu,\vv\) et \(\vw\) de l'espace \(\E\), alors~:
  \begin{equation}
    (\vu\wedge\vv)\wedge\vk=(\vu\cdot\vw)\vv-(\vv\cdot\vw)\vu.
  \end{equation}
\end{prop}
\begin{proof}
  Si on suppose dans un premier temps que \(\vu\) et \(\vv\) sont colinéaires et on prétend qu'il existe un réel \(\lambda\) tel que \(\vu=\lambda\vv\) et alors le premier terme de l'égalité est nul d'après la proposition~\ref{prop:1} et le deuxième terme vaut \((\vu\cdot\vw)\vv-(\vv\cdot\vw)\vu=\lambda(\vu\cdot\vw)\vu-\lambda(\vu\cdot\vw)\vu=0\).
  
  On suppose dans un dernier temps que \(\vu\) et \(\vv\) ne sont pas colinéaires. On se place dans une base orthonormale directe adaptée à notre problème. On pose le premier vecteur de la BOND comme \(\vi=\frac{\vu}{\norme{\vu}}\) et le deuxième \(\vj\) comme étant unitaire directement orthogonal à \(\vi\) dans le plan \((O,\vu,\vv)\) et pour finir la base on pose \(\vk=\vi\wedge\vj\). Dans cette base les coordonnées de nos vecteurs de départs sont les suivantes~: \(\vu(\alpha,0,0)\) \(\vv(\beta,\gamma,0)\) et \(\vw(\delta,\epsilon,\eta)\). Alors si on calcule on obtient~: \(\vu\wedge\vv(0,0,\alpha\gamma)\) \((\vu\wedge\vv)\wedge\vw(-\alpha\gamma\epsilon,\alpha\gamma\delta,0)\) et les produit scalaires~: \(\vu\cdot\vw=\alpha\delta\) et donc \((\vu\cdot\vw)\vv(\alpha\delta\beta, \alpha\delta\gamma,0)\) puis \(\vv\cdot\vw=\beta\delta+\gamma\epsilon\) et donc \((\vv\cdot\vw)\vu(\alpha\beta\delta+\alpha\gamma\epsilon,0,0)\). Au final on a bien l'égalité.
\end{proof}

\section{Déterminant ou produit mixte}
\subsection{Définition}
\begin{defdef}
  Soient trois vecteurs de \(\E\) \(\vu,\vv \& \vw\). On leur associe un réel appelé déterminant ou produit mixte de \(\vu,\vv\) et \(\vw\) noté \(\Det(\vu,\vv,\vw)\) ou \([\vu,\vv,\vw]\) définie par \(\Det(\vu,\vv,\vw)=(\vu\wedge\vv)\cdot\vw\).
\end{defdef}

\emph{Interprétation géométrique}~: Le nombre \(\abs{\Det(\vu,\vv,\vw)}\) est le volume du parallélépipède construit sur \(\vu\), \(\vv\) et \(\vw\) noté \(ABCDEFGH\) avec \(\vu=\vect{AB}\), \(\vv=\vect{AD}\), \(\vw=\vect{AE}\) alors le volume est égal à l'aire \(\A\) de \(ABCD\) multiplié par la hauteur \(h\) du parallélépipède. L'aire n'est rien d'autre que \(\A=\norme{\vect{AB}\wedge\vect{AD}}\) et si on note \(\vk\) un vecteur unitaire directement orthogonal à \((\vect{AB},\vect{AD})\) alors \(\vect{AB}\wedge\vect{AD}=\A \vk\). La hauteur est \(h=\abs{\vk\cdot\vw}\). On a bien~:
\begin{equation}
  \abs{\Det(\vu,\vv,\vw)}=\abs{(\vu\wedge\vv)\cdot\vw}=\abs{\A\vk\cdot\vw}=\A h.
\end{equation}
\begin{prop}
  Soient \(\vu\), \(\vv\) et \(\vw\) trois vecteurs de \(\E\), alors~:
  \begin{equation}
    \Det(\vu,\vv,\vw)=0 \iff \vu,\vv\&\vw \text{~sont coplanaires}.
  \end{equation}
\end{prop}
\begin{proof}
  On rappelle que trois vecteurs \(\vu\), \(\vv\), \(\vw\) sont coplanaires si et seulement si deux parmi ces trois vecteurs sont colinéaires ou si un des trois est une combinaison linéaire des deux autres (à savoir s'il existe deux réels \(\lambda,\mu\) tel que \(\vw=\lambda\vu+\mu\vv\)).
  \begin{itemize}
  \item[\(\impliedby\)] On suppose ces trois vecteurs coplanaires alors~:
    \begin{itemize}
    \item Si \(\vu\) et \(\vv\) sont colinéaires alors leur produit vectoriel est nul donc le déterminant des trois vecteurs est nul;
    \item sinon s'il existe deux réels \(\lambda,\mu\) tels que \(\vw = \lambda\vu+\mu\vv\) alors \((\vu\wedge\vv)\cdot\vw=\lambda(\vu\wedge\vv)\cdot\vu+\mu(\vu\wedge\vv)\cdot\vv=0\) puisque \(\vu\wedge\vv\) est orthogonal à \(\vu\) et à \(\vv\).
    \end{itemize}
  \item[\(\implies\)] On suppose que le produit mixte est nul, alors \(\vu\wedge\vv\) et \(\vw\) sont orthogonaux. 
    \begin{itemize}
    \item Soit \(\vu\wedge\vv=0\) donc \(\vu\) et \(\vv\) dont colinéaires donc \(\vu,\vv\) et \(\vw\) sont coplanaires;
    \item soit \(\vw=0\) alors ils sont aussi coplanaires;
    \item soit \(\vu\wedge\vv \neq 0\) et \(\vw\neq 0\) et donc \((\vu\wedge\vv)\) et \(\vw\) sont \og orthogonaux \fg{} et donc \(\vu,\vv\) et \(\vw\) sont orthogonaux au même vecteur non nul \(\vu\wedge\vv\) donc ils sont coplanaires.
    \end{itemize}
  \end{itemize}
\end{proof}
\begin{prop}
  Soit \(\bondtrois\) une BON, c'est une base (in)directe si et seulement si \(\Det\bondtrois=\pm 1\)
\end{prop}
\begin{proof}
  D'après la section~\ref{sec:prodvec}, \(\bondtrois\) est (in)directe si et seulement si \(\vk=\pm\vi\wedge\vj\) donc si et seulement si \(\Det\bondtrois=(\vi\wedge\vj)\cdot\vk=\pm\norme{\vk}^2=\pm 1\). Il vaut un lorsque la base est directe.
\end{proof}

\subsection{Propriétés algébriques}
\begin{prop}[Antisymétrie, trilinéarité]
  pour tout vecteurs \(\vu\), \(\vv\), \(\vw\) et \( \vz\) de l'espace \(\E\) et tout réel \(\lambda\) on a~:
  \begin{gather}    
    \Det(\vu,\vw,\vv)=\Det(\vv,\vu,\vw)=\Det(\vw,\vv,\vu)=-\Det(\vu,\vv,\vw)\label{eq:antisym};\\
    \Det(\lambda\vu,\vv,\vw)=\Det(\vu,\lambda\vv,\vw)=\Det(\vu,\vv,\lambda\vw)=\lambda\Det(\vu,\vv,\vw)\label{eq:trilin1};\\
    \Det(\vu+\vv,\vw,\vz)=\Det(\vu,\vw,\vz)+\Det(\vv,\vw,\vz)\label{eq:trilin2};\\
    \Det(\vu,\vv+\vw,\vz)=\Det(\vu,\vv,\vz)+\Det(\vu,\vw,\vz)\label{eq:trilin3};\\
    \Det(\vu,\vv,\vw+\vz)=\Det(\vu,\vv,\vw)+\Det(\vu,\vv,\vz)\label{eq:trilin4}.
  \end{gather}
\end{prop}
\begin{proof}
  Les équations~\eqref{eq:trilin1},~\eqref{eq:trilin2},~\eqref{eq:trilin3} et~\eqref{eq:trilin4} sont des conséquences de la bilinéarité du produit scalaire et du produit vectoriel. Par exemple, on peut montrer l'équation~\eqref{eq:trilin3}~:
  \begin{align}
    \Det(\vu,\vv+\vw,\vz)&=(\vu\wedge(\vv+\vw))\cdot\vz\\
    &=(\vu\wedge\vv+\vu\wedge\vw)\cdot\vz\\
    &=(\vu\wedge\vv)\cdot\vz+(\vu\wedge\vw)\cdot\vz\\
    &=\Det(\vu,\vv,\vz)+\Det(\vu,\vw,\vz)
  \end{align}
Puisque le produit vectoriel est antisymétrique alors le produit mixte est antisymétrique. On peut démontrer un morceau de l'équation~\eqref{eq:antisym}~:
\begin{align}
  \Det(\vv,\vu,\vw)&=(\vv\wedge\vu)\cdot\vw\\
  &=(-\vu\wedge\vv)\cdot\vw\\
  &=-(\vu\wedge\vv)\cdot\vw\\
  &=-\Det(\vu,\vv,\vw).
\end{align}
Pour les autres résultats, on utilise le résultat suivant~: si deux des trois vecteurs sont égaux alors le déterminant est nul. Cela signifie que le déterminant est alterné
\begin{equation}
  0=\Det(\vu,\vv+\vw,\vv+\vw)=\Det(\vu,\vv,\vw)+\Det(\vu,\vw,\vv).
\end{equation}
D'où le résultat.
\end{proof}
\begin{cor}
  pour tout vecteur \(\vu\), \(\vv\), \(\vw\), on a~:
  \begin{equation}
    \Det(\vu,\vv,\vw)=\Det(\vv,\vw,\vu)=\Det(\vw,\vu,\vv).
  \end{equation}
\end{cor}
\begin{proof}
  On échange deux vecteurs à chaque fois et on multiplie par \(-1\times(-1)=1\)~:
  \begin{equation}
   \Det(\vv,\vw,\vu)=-\Det(\vu,\vw,\vv) =-(-\Det(\vu,\vv,\vw)))=\Det(\vu,\vv,\vw).
  \end{equation}
\end{proof}
\begin{prop}[Règle de Sarrus]
  Soit \(\bondtrois\) une BOND de l'espace. On considère trois vecteurs \(\vu,\vv,\vw\) de coordonnées respectives \((x,y,z)\), \((x',y',z')\), \((x'',y'',z'')\) dans cette base, alors~:
  \begin{equation}
    \Det(\vu,\vv,\vw)=xy'z''+yz'x''+zx'y''-zy'x''-xz'y''-yx'z''.
  \end{equation}
\end{prop}
\begin{proof}
  La démonstration se construit sur le détail des calculs du produit vectoriel et du produit scalaire puisque \(\Det(\vu,\vv,\vw)=(\vu\wedge\vv)\cdot\vw\).
\end{proof}

\section{Droites et plans}
On place dans toute cette section un ROND \(\rondtrois\).

\subsection{Définition}
\begin{defdef}
  On appelle droite de l'espace \(\E\) toute partie \(\Dr\) de \(\E\) telle que~:
  \begin{itemize}
  \item Tout barycentre d'une famille de points de \(\Dr\) est dans \(\Dr\), c'est-à-dire que \(\Dr\) est un sous-espace affine;
  \item la partie \(\Dr\) contient au moins deux points, c'est-à-dire que \(\Dr\) est au moins de dimension 1;
  \item trois points de \(\Dr\) sont alignés, c'est-à-dire que \(\Dr\) est au plus de dimension 1.
  \end{itemize}
Par deux points passe une et une seule droite. On dit qu'un vecteur non nul \(\vu\) est directeur d'une droite \(\Dr\) s'il existe deux points \(A\) et \(B\) de \(\Dr\) tels que \(\vu=\vect{AB}\).
\end{defdef}
\begin{defdef}
  On appelle plan de l'espace \(\E\) toute partie \(\P\) de \(\E\) telle que~:
  \begin{itemize}
  \item tout barycentre de points de \(\P\) est dans \(\P\), c'est-à-dire que \(\P\) est un sous-espace affine;
  \item quatre points quelconque de \(\P\) sont nécessairement coplanaires, c'est-à-dire que \(\P\) est au plus de dimension \(2\);
  \item \(\P\) contient au moins trois points non alignés, c'est-à-dire que \(\P\) est au moins de dimension \(2\).
  \end{itemize}
Par trois points \(A\), \(B\), et \(C\) non alignés passe un unique plan. Un vecteur non nul \(\vect{n}\) est dit normal au plan s'il est orthogonal à tous les vecteur du plan.
\end{defdef}

\subsection{Représentations paramétriques}
\subsubsection{Représentation paramétrique d'une droite}
\paragraph{Droite définie par un point et un vecteur directeur}
Soit \(\Dr\) passant par \(A(x_0,y_0,z_0)\) dirigé selon \(\vu(a,b,c)\) alors \(\Dr\) est paramétré 
\begin{equation}
  \Dr : \forall t\in\R
  \begin{cases}
    x(t)=at+x_0\\y(t)=bt+y_0\\z(t)=ct+z_0
  \end{cases}.
\end{equation}
Réciproquement un tel paramétrage avec \(a\), \(b\), \(c\) non tous nuls représente une droite passant par le point \((x_0,y_0,z_0)\) de direction \((a,b,c)\).

\paragraph{Droite définie par deux points distincts A et B}
C'est la droite passant par A de vecteur directeur \(\vect{AB}\) avec \(A(x_0,y_0,z_0)\) et \(B(x_1,y_1,z_1)\) alors \(\Dr\) est paramétrée
\begin{equation}
  \Dr : \forall t\in\R
  \begin{cases}
    x(t)=(x_1-x_0)t+x_0\\y(t)=(y_1-y_0)t+y_0\\z(t)=(z_1-z_0)t+z_0
  \end{cases}.
\end{equation}

\subsubsection{Représentation paramétrique d'un plan}
On considère le plan \(\P\) passant par le point \(A(x_0,y_0,z_0)\) de vecteurs directeurs \(\vu(a_1,b_1,c_1)\) et \(\vv(a_2,b_2,c_2)\). Un paramétrage de \(\P\) est~:
\begin{equation}
  \P : \forall (t,s)\in\R^2
  \begin{cases}
    x(t,s)=a_1t+a_2s+x_0\\y(t,s)=b_1t+b_2s+y_0\\z(t,s)=c_1t+c_2s+z_0
  \end{cases}.
\end{equation}
C'est moins pratique que pour une droite, puisqu'il y a deux paramètres.

\subsection{Équations cartésiennes}

\subsubsection{Représentation cartésienne d'un plan}

\paragraph{Plan défini par un point et deux vecteur non colinéaires}

Soit \(\P\) le plan passant par \(A(x_0,y_0,z_0)\) dirigé par les vecteurs non colinéaires \(\vu(a_1,b_1,c_1)\) et \(\vv(a_2,b_2,c_2)\). Alors
\begin{align}
  M(x,y,z) \in \P &\iff  \vect{AM}, \vu, \vv \text{~sont coplanaires}\\
  &\Det(\vect{AM},\vu,\vv)=\begin{vmatrix} x-x_0 & a_1 & a_2 \\ y-y_0 & b_1 & b_2 \\ z-z_0 & c_1 & c_2\end{vmatrix}=0.
\end{align}

\paragraph{Plan défini par trois points non alignés}

Soit \(\P\) le plan passant par \(A\), \(B\) et \(C\), alors \(\P\) est le plan passant par \(A\) et dirigé par \(\vect{AB}\) et \(\vect{BC}\),
\begin{equation}
  M \in \P \iff \Det(\vect{AM},\vect{AB},\vect{BC})=0.
\end{equation}

\paragraph{Plan défini par un point et un vecteur normal}

Soit \(\P\) le plan passant par \(A(x_0,y_0,z_0)\) de vecteur normal \(\vect{n}(a,b,c)\) non nul. Soit un point \(M(x,y,z)\), alors
\begin{align}
  M \in \P &\iff \vect{AM}\cdot\vect{n}=0 \\
  &\iff ax+by+cz=ax_0+by_0+cz_0.
\end{align}

Réciproquement une équation cartésienne de la forme \(ax+by+cz=\alpha\) avec \(a\), \(b\) et \(c\) non tous nuls définit un plan dont un vecteur normal est \(\vect{n}(a,b,c)\).

\subsubsection{Représentation cartésienne d'une droite}

Une droite est définie comme l'intersection de deux plans sécants. Soient deux plans sécants \(\P\) et \(\P'\) d'équation cartésiennes respectives~:
\begin{align}
  ax+by+cz=\alpha \\ a'x+b'y+c'z=\alpha',
\end{align}
et leur intersection \(\Dr\) est définie par~:
\begin{equation}
 \begin{cases} ax+by+cz&=\alpha \\ a'x+b'y+c'z&=\alpha'\end{cases}.
\end{equation}
Réciproquement, un tel système d'équation cartésiennes peut définir~:
\begin{itemize}
\item un plan si \(\P\) et \(\P'\) sont confondus;
\item une droite, s'ils sont sécants;
\item le vide, s'ils sont paralléles mais non-confondus.
\end{itemize}

\subsection{Positions relatives}

\subsubsection{Positions relatives de deux plans -- Parallélisme et orthogonalité}

\begin{defdef}
  Deux plans \(\P\) et \(\P'\) sont dits parallèles s'ils admettent un vecteur normal, non nul, commun.
\end{defdef}
\begin{prop}
  Soient \(\P\) et \(\P'\) deux plans d'équations cartésiennes respectives \(ax+by+cz=\alpha\) et \(a'x+b'y+c'z=\alpha'\). Alors \(\P\) et \(\P'\) sont paralléles si et seulement si \((a,b,c)\) et \((a',b',c')\) sont proportionnels et \(\P\) et \(\P'\) sont égaux si et seulement si \((a,b,c,\alpha)\) et \((a',b',c',\alpha')\) sont proportionnels.
\end{prop}
\begin{defdef}
  Deux plans sont orthogonaux s'ils admettent des vecteurs normaux orthogonaux.
\end{defdef}
\begin{prop}
  Deux plans \(\P\) et \(\P'\) d'équations cartésiennes respectives \(ax+by+cz=\alpha\) et \(a'x+b'y+c'z=\alpha'\) avec des vecteurs normaux non nuls; alors \(\P\) et \(\P'\) sont orthogonaux si et seulement si \(aa'+bb'+cc'=0\).
\end{prop}

\subsubsection{Positions relatives d'une droite et d'un plan}

\begin{defdef}
  Soient \(\Dr\) une droite et \(\P\) un plan. On dit que \(\Dr\) est parallèles à \(\P\) si pour tout vecteur directeur \(\vu\) de \(\Dr\) il existe deux points du plan \(\P\) tels que \(\vu=\vect{AB}\).
\end{defdef}
\begin{prop}
  Soient \(\Dr\) une droite et \(\P\) un plan.
  \begin{enumerate}
  \item Si \(\Dr\) est paralléles à \(\P\)~:
    \begin{itemize}
    \item soit \(\Dr\subset\P\);
    \item soit \(\Dr\cap\P=\emptyset\).
    \end{itemize}
  \item Si \(\Dr\) n'est pas paralléle à \(\P\), alors l'intersection de \(\Dr\) et de \(\P\) est un singleton.
  \end{enumerate}
\end{prop}
\begin{proof}
  \begin{enumerate}
  \item Si \(\Dr\) est parallèle à \(\P\) alors soit \(\vu\neq 0\) un vecteur directeur de \(\Dr\) et on suppose qu'il existe \(A\in\Dr\cap\P\), alors il existe trois points C,D et E tels que \(\vu=\vect{CD}=\vect{AE}\). Pour tout point M du plan, il existe un réel \(\lambda\) tel que \(\vect{AM}=\lambda\vu=\lambda\vect{AE}\). Les point A et E sont dans le plan \(\P\), alors la droite \((AE)\) est incluse dans le plan \(\P\) donc si M est sur \((AE)\) alors M est sur le plan \(\P\), donc \(D \subset \P\).
  \item Si \(\Dr\) n'est pas parallèle à \(\P\), alors \(\Dr\) est la droite passant par A et dirigé par \(\vu\). \(\P\) est le plan passant par B de vecteur normal \(\vv\). Les vecteurs \(\vu\) et \(\vv\) ne sont pas orthogonaux
    \begin{align}
      M \in \Dr\cap\P &\iff \begin{cases} \exists \lambda \in \R \ \vect{AM}=\lambda\vu \\ \vect{BM}\cdot\vv=0\end{cases} \\
      &\iff \begin{cases} \exists \lambda \in \R \ \vect{AM}=\lambda\vu \\ \vect{BA}\cdot\vv+\vect{AM}\cdot\vv=0\end{cases}\\
      &\iff \begin{cases} \exists \lambda \in \R \ \vect{AM}=\lambda\vu \\ \lambda\vu\cdot\vv=\vect{AB}\cdot\vv\end{cases}\\
      &\iff \begin{cases} \exists \lambda \in \R \ \vect{AM}=\lambda\vu \\\lambda=\frac{\vect{AB}\cdot\vv}{\vu\cdot\vv}\end{cases}\\
      M \in \Dr\cap\P &\iff \vect{AM}=\frac{\vect{AB}\cdot\vv}{\vu\cdot\vv}\vu.
    \end{align}
Il y a donc un unique point d'intersection.
  \end{enumerate}
\end{proof}

\subsection{Distace d'un point à un plan ou à une droite}
\subsubsection{Projeté orthogonal}
\begin{defdef}
  Soit \(M\) un point de l'espace \(\epsilon\) et \(X\) une droite ou un plan. On appelle distance de \(M\) à \(X\) le réel~:
  \begin{equation}
    d(M,X)=\Inf\enstq{MM'}{M' \in X}.
  \end{equation}
Il s'agit de la borne inférieur, la plus petite distance de \(M\) à un point de \(X\).
\end{defdef}
%
\begin{prop}
  Soit \(M\) un point de l'espace \(\epsilon\) et \(X\) une droite ou un plan. Il existe un unique point \(H\) de \(X\) tel que \(\vect{MH}\) soit orthogonal à \(X\). On l'appelle projeté orthogonal de \(M\) sur \(X\) et il vérifie \(d(M,X)=MH\) et la distance de \(M\) à \(X\) n'est atteinte qu'en \(H\).
\end{prop}
\begin{proof}
  \begin{itemize}
  \item \(X\) est une droite. Soit \(\vu\) un vecteur directeur de \(X\), puis \(A\) et \(N\) des points de X. Alors~:
    \begin{align}
      \vect{MN}\perp X &\iff \vect{MN}\cdot\vu=\vect{MA}\cdot\vu+\vect{AN}\cdot\vu=0\\
      N \in X &\iff \exists \lambda \in \R \ \vect{AN}=\lambda\vu\\
      \vect{MN}\perp X &\iff \lambda = \frac{\vect{AM}\cdot\vu}{\norme{\vu}^2}.
    \end{align}
    On a montré qu'il existe un unique point H de X tel que \(\vect{AH}\perp X\) et de plus \(\vect{AH}=\frac{\vect{AM}\cdot\vu}{\norme{\vu}^2}\vu\)
  \item X est un plan. Soit A un point de X et \(\bond\) une BON de X. Alors~:
    \begin{align}
      N \in X &\iff \exists (\lambda,\mu)\in\R^2 \ \vect{AN}=\lambda\vi+\mu\vj\\
      \vect{MN}\perp X &\iff \vect{MN}\cdot\vi=0 \wedge \vect{MN}\cdot\vj=0\\
      &\iff \vect{MA}\cdot\vi+\vect{AN}\cdot\vi=0 \wedge \vect{MA}\cdot\vj+\vect{AN}\cdot\vj=0.
    \end{align}
    Or on sait que~:
    \begin{align}
      \vect{AN}\cdot\vu&=\lambda\vi\cdot\vi+\mu\vj\cdot\vi=\lambda\\
      \vect{AN}\cdot\vv&=\lambda\vi\cdot\vj+\mu\vj\cdot\vj=\mu.
    \end{align}
    Alors finalement, 
    \begin{equation}
      \vect{MN}\perp X \iff \lambda=\vect{AM}\cdot\vi \wedge \mu=\vect{AM}\vj.
    \end{equation}
    Il existe un seul point H de X tel que \(\vect{MH}\perp X\).
  \item \(X\) est une droite ou un plan. Soit \(M'\) un point de \(X\), \(\vect{MH}\) est orthogonal à \(\vect{HM'}\); le théorème de Pythagore nous donne \(MM'^2=MH^2+M'H^2\) donc \(MM'^2\geqslant MH^2\). Et \(MM'=MH \iff HM'=0 \iff H=M'\), la distance est atteinte pour \(M'=H\) et seulement pour \(M'=H\)
  \end{itemize}
\end{proof}

\subsubsection{Distance à un plan}

\paragraph{Plan défini par un point A et deux vecteurs non colinéaires}

Soit \(\P\) un plan, \(M\) un point de l'espace, \(H\) son projeté orthogonal sur \(\P\). \(\vect{HM}\) est orthogonal à \(\vu\) et \(\vv\) donc il existe un réel \(\lambda\) tel que \(\vect{HM}=\lambda \vu\wedge\vv\). 
\begin{equation}
  \vect{AM}\cdot(\vu\wedge\vv)=\vect{AH}\cdot(\vu\wedge\vv)+\vect{HM}\cdot(\vu\wedge\vv)=\lambda \norme{\vu\wedge\vv}^2.
\end{equation}
Donc du coup,
\begin{equation}
  HM=\abs{\lambda} \norme{\vu\wedge\vv}=\frac{\abs{\vect{AM}\cdot(\vu\wedge\vv)}}{\norme{\vu\wedge\vv}}=\frac{\abs{\Det(\vu,\vv,\vect{AM})}}{\norme{\vu\wedge\vv}}.
\end{equation}
Donc la distance vaut~:
\begin{equation}
  \label{eq:freq}
  d(M,\P)=\frac{\abs{\Det(\vu,\vv,\vect{AM})}}{\norme{\vu\wedge\vv}}.
\end{equation}

\paragraph{Plan défini par trois points non alignés A,B et C}

En appliquant la formule~\eqref{eq:freq} avec ces trois points, on obtient~:
\begin{equation}
  d(M,\P)=\frac{\abs{\Det(\vect{AB},\vect{AC},\vect{AM})}}{\norme{\vect{AB}\wedge\vect{AC}}}.
\end{equation}

\paragraph{Plan défini par un point A et un vecteur normal}

Ce vecteur normal est noté \(\vn(a,b,c)\) et \(A(x_0,y_0,z_0)\), soient \(\vu\) et \(\vv\) deux vecteurs directeurs directeurs de \(\P\). Il existe un réel non nul \(\lambda\) tel que \(\lambda\vn=\vu\wedge\vv\), alors
\begin{equation}
  HM=\frac{\abs{\vect{AM}\cdot(\vu\wedge\vv)}}{\norme{\vu\wedge\vv}}=\frac{\abs{\vect{AM}\cdot\lambda\vn}}{\norme{\lambda\vn}}=\frac{\abs{\vect{AM}\cdot\vn}}{\norme{\vn}}.
\end{equation}
Une équation cartésienne de \(\P\) est \(ax+by+cz=\alpha\) avec \(\alpha=ax_0+by_0+cz_0\), finalement~:
\begin{equation}
  d(M,\P)=\frac{\abs{ax+by+cz-\alpha}}{\sqrt{a^2+b^2+c^2}}.
\end{equation}
\begin{defdef}
  On dit qu'une équation d'un plan \(\P\) du type \(ax+by+cz=\alpha\) est une équation normale lorsque \(a^2+b^2+c^2=1\), c'est-à-dire lorsque le vecteur normal \(\vn\) est de norme 1. Dans ce cas \(d(M,\P)=\abs{ax+by+cz-\alpha}\), où \(\abs{\alpha}\) est la distance de l'origine au plan \(\P\).
\end{defdef}

\subsubsection{Distance à une droite}

Si la droite est définie par un point \(A\) et un vecteur \(\vu\) non nul, soit M un point de l'espace \(\epsilon\), H son projeté orthogonal sur \(\Dr\). Alors \(d(M,\Dr)=HM\) et \(\vect{AM}\wedge\vu=\vect{HM}\wedge\vu\), donc \(\norme{\vect{AM}\wedge\vu}=\norme{\vect{HM}}\norme{\vu}\), alors \(d(M,\D)=\frac{\norme{\vect{AM}\wedge\vu}}{\norme{\vu}}\).

Si la droite est définie par deux points distincts \(A\) et \(B\), alors \(d(M,\D)=\frac{\norme{\vect{AM}\wedge\vect{AB}}}{\norme{\vect{AB}}}\).

\subsection{Perpendiculaire commune et distance entre deux droites non parallèles}

Soient \(\Dr\) et \(\Dr'\) deux droites de l'espace non parallèles.
\begin{defdef}
  Soient \(\Dr\) et \(\Dr'\) deux droites de l'espace,
  \begin{itemize}
  \item on dit qu'elles sont orthogonales si et seulement si leur vecteurs directeurs sont orthogonaux;
  \item on dit qu'elles sont perpendiculaires si et seulement si elles sont orthogonales et sécantes.
  \end{itemize}
\end{defdef}

\subsubsection{Perpendiculaire commune}

\begin{theo}
  Soient \(\Dr\) et \(\Dr'\) deux droites de l'espace non parallèles. Il existe une unique droite \(\Delta\) perpendiculaire à \(\Dr\) et à \(\Dr'\), c'est la perpendiculaire commune de \(\Dr\) et \(\Dr'\).
\end{theo}
\begin{proof}[Existence]
  Soient A un point de \(\Dr\) et \(\vu\) un de ses vecteur directeur et \(A'\) un point de \(\Dr'\) et \(\vu'\) un vecteur directeur de \(\Dr'\). Ces deux vecteurs ne sont pas colinéaires. Soit le vecteur non nul \(\vv=\vu\wedge\vu'\). Le triplet \((\vu,\vu',\vv)\) est alors une base directe de l'espace. Il existe donc trois réels \(a,b,c\) tel que \(\vect{AA'}=a\vu+b\vu'+c\vv\). Posons le point H tel que \(\vect{AH}=a\vu\), alors H est un point de \(\Dr\). De la même manière, on pose le point \(H'\) tel que \(\vect{H'A'}=b\vu'\) alors \(H'\) est un point de \(\Dr'\). Soit \(\Delta\) la droite passant par \(H\) et dirigée par \(\vv\). Premièrement, \(\Delta\) passe par \(H\in\Dr\), \(\Delta\) et \(\Dr\) sont sécantes. \(\Delta\) est dirigée par \(\vv\) qui est orthogonal à \(\vu\) et \(\vv\) donc \(\Delta\) est orthogonale à \(\Dr\) et \(\Dr'\). Il reste à montrer qu'elle est sécante avec \(\Dr'\). Secondement,
\begin{equation}
  \vect{HH'}=\vect{HA}+\vect{AA'}+\vect{A'H'}=-a\vu+(a\vu+b\vu'+c\vv)-b\vu'=c\vv.
\end{equation}
Puisque \(H\in\Delta\) et que \(\vv\) dirige \(\Delta\) alors \(H'\in\Delta\). Ainsi \(\Delta\) et \(\Dr'\) sont sécantes. On a prouvé l'existence d'une droite perpendiculaire à \(\Dr\) et à \(\Dr'\).
\end{proof}
\begin{proof}[Unicité]
  Soit \(\Delta'\) une perpendiculaire commune à \(\Dr\) et à \(\Dr'\). Notons \(C\) et \(C'\) ses points d'intesections avec \(\Dr\) et \(\Dr'\). Alors \(\vect{CC'}=\vect{CH}+\vect{HH'}+\vect{H'C'}\). Puisque \(C\) et \(H\) sont des points de \(\Dr\), il existe un réel \(\alpha\) tel que \(\vect{CH}=\alpha\vu\) et puisque \(C'\) et \(H'\) sont des points de \(\Dr'\), il existe un réel \(\beta\) tel que \(\vect{C'H'}=\beta\vu'\); nous savons aussi que \(\vect{HH'}=c\vv\). Alors \(\vect{CC'}=\alpha\vu+c\vv+\beta\vu'\), or \(\vect{CC'}\) est orthogonal à \(\vu\) et à \(\vu'\) donc il est colinéaire à \(\vv=\vu\wedge\vu'\). Par unicité des coordonnées, on a forcément \(\alpha=\beta=0\) alors \(C=H\) et \(C'=H'\). Ainsi \(\Delta=\Delta'\).
\end{proof}

\subsubsection{Distance entre deux droites non parallèles}

\begin{defdef}
  Soient \(\Dr\) et \(\Dr'\) deux droites non parallèles. On définit la distance entre \(\Dr\) et \(\Dr'\) par~:
  \begin{equation}
    d(\Dr,\Dr')=\inf\enstq{MM'}{M \in \Dr \text{~et~} M'\in \Dr'}.
  \end{equation}
C'est la plus petite distance d'un point de \(\Dr\) à un autre point de \(\Dr'\).
\end{defdef}
\begin{prop}
  Soient deux droites \(\Dr\) et \(\Dr'\) non parallèles.
  \begin{equation}
    d(\Dr,\Dr')=HH'
  \end{equation}
\end{prop}
\begin{proof}
  Pour tout point \(M\) de \(\Dr\) et tout point \(M'\) de \(\Dr'\), 
  \begin{equation}
    \vect{MM'}=\vect{MH}+\vect{HH'}+\vect{H'M'}
  \end{equation}
Il existe des réels \(\alpha\) et \(\beta\) tel que \(\vect{MH}=\alpha\vu\) puisque M et H sont sur \(\Dr\) et \(\vect{M'H'}=\beta\vu'\) puisque \(M'\) et \(H'\) sont sur \(\Dr'\); il reste aussi \(\vect{HH'}=c\vv\). Alors \(\vect{MM'}=(\alpha\vu+\beta\vu')+c\vv\). Le vecteur \(\vv\) est orthogonal à \(\alpha\vu+\beta\vu'\).

Le théorème de Pythagore affirme donc que \(MM'^2=\norme{\alpha\vu+\beta\vu'}^2+HH'^2\). Alors pour tout M de \(\Dr\) et tout \(M'\) de \(\Dr'\) on a \(MM'\geqslant HH'\).

\begin{align}
  MM'=HH' &\iff \norme{\alpha\vu+\beta\vu'}^2=0\\
&\iff \alpha\vu+\beta\vu'=0\\
&\iff \alpha=0 \wedge \beta=0\\
&\iff M=H \wedge M'=H'
\end{align}
\(d(\Dr,\Dr')=HH'\) et cette distance n'est atteinte qu'en \(H\) et \(H'\).
\end{proof}
\begin{prop}
  \begin{equation}
    d(\Dr,\Dr')=\frac{\abs{\Det(\vu,\vu',\vect{AA'}}}{\norme{\vu\wedge\vu'}}.
  \end{equation}
\end{prop}
\begin{proof}
  \begin{align}
    \Det(\vu,\vu',\vect{AA'})&=\Det(\vu,\vu',\vect{AH})+\Det(\vu,\vu',\vect{HH'})+\Det(\vu,\vu',\vect{H'A'})\\
    &=\Det(\vu,\vu',\vect{HH'})\\
    &=(\vu\wedge\vu')\cdot\vect{HH'}\\
    &=c\norme{\vu\wedge\vu'}^2.
  \end{align}
  Donc finalement, 
  \begin{equation}
    HH'=\norme{c\vu\wedge\vu'}=\abs{c}\norme{\vu\wedge\vu'}=\frac{\abs{\Det(\vu,\vu',\vect{AA'})}}{\norme{\vu\wedge\vu'}}.
  \end{equation}
\end{proof}

\section{Cercles et sphères}

\subsection{Équations d'une sphère en ROND}

On munit l'espace d'un ROND \(\rondtrois\).
\begin{defdef}
  Soient \(A\) un point de l'espace et \(r\) un réel strictement positif. La sphère de centre \(A\) et de rayon \(r\), \(S(a,r)\), est l'ensemble des points \(M\) de l'espace tels que \(AM=r\). Si le point \(A\) est de coordonnées \((a,b,c)\), alors une équation cartésienne de \(S(A,r)\) est~:
  \begin{gather}
    (x-a)^2+(y-b)^2+(z-c)^2=r^2; \\
    x^2+y^2+z^2-2ax-2by-2cz=d \quad d=r^2-a^2-b^2-c^2.
  \end{gather}
\end{defdef}
Réciproquement, soit \(X\) l'ensemble défini par l'équation cartésienne suivante
\begin{equation}
  x^2+y^2+z^2-2ax-2by-2cz=d \quad (a,b,c,d)\in \R^4,
\end{equation}
si et seulement si
\begin{equation}
  (x-a)^2+(y-b)^2+(z-c)^2=d+a^2+b^2+c^2.
\end{equation}
\begin{itemize}
\item si \(d+a^2+b^2+c^2<0\) alors \(X=\emptyset\);
\item si \(d+a^2+b^2+c^2=0\) alors \(X=A\);
\item si \(d+a^2+b^2+c^2>0\) alors \(X=S(A,\sqrt{d+a^2+b^2+c^2})\).
\end{itemize}

\emph{Représentation paramétrique}~: Si \(S\) est la sphère de centre O de rayon \(\rho_0>0\), alors \(S\) est l'ensemble des points qui admettent un système de coordonnées sphériques de la forme \((\rho_0,\varphi,\theta)\), avec \(\varphi\in[0,\pi]\).
On en déduit une représentation paramétrique de \(S\)~:
\begin{equation}
  \begin{cases}
    x(\varphi,\theta)=\rho_0\sin\varphi\sin\theta\\
    y(\varphi,\theta)=\rho_0\cos\varphi\sin\theta\\
    z(\varphi,\theta)=\rho_0\cos\theta
  \end{cases}.
\end{equation}
Soient A et B deux points distincts de l'espace, l'ensemble S tel que \(S=\{M \in \epsilon, \vect{MA}\cdot\vect{MB}=0\}\) et I le milieu de \([AB]\). Alors
\begin{align}
  M \in S & \iff (\vect{MI}+\vect{IA})\cdot(\vect{MI}+\vect{IB})= (\vect{MI}+\vect{IA})\cdot(\vect{MI}-\vect{IA})=0\\
  &\iff MI^2-IA^2=0\\
  &\iff MI^2=\frac{AB^2}{4}.
\end{align}
\(S\) est la sphère de centre \(I\) de rayon \(\frac{AB}{2}\), c'est aussi la sphère de diamètre \([AB]\).

\subsection{Problème d'intersection}
Soit \(S\) la sphère de centre \(A(a,b,c)\) de rayon \(\rho>0\).

\subsubsection{Intersection d'une sphère et d'une droite}

\begin{prop}
  Soit \(\Dr\) une droite de l'espace. L'intersection de \(S\) et de \(\Dr\) est~:
  \begin{itemize}
  \item vide, si \(d(A,\Dr)>\rho\);
  \item réduite au projeté orthogonal H de A sur \(\Dr\), si \(d(A,\Dr)=\rho\) et dans ce cas la sphère et la droites sont tangentes;
  \item constituée de deux point symétrique par rapport au point H, si \(d(A,\Dr)<\rho\).
  \end{itemize}
\end{prop}

\begin{proof}
  Soit \(H\) le projeté orthogonal de \(A\) sur \(\Dr\). Soit \(M\) un point de \(\Dr\), alors \(\vect{AM}\) et \(\vect{HM}\) sont orthogonaux. Le théorème de Pythagore dit que \(AM^2=d(A,\Dr)^2+HM^2\). Alors
  \begin{align}
    M \in S &\iff AM=\rho\\
    &\iff AM^2=d(A,\Dr)^2+HM^2=\rho^2\\
    &\iff HM^2=\rho^2-d(A,\Dr)^2.
  \end{align}
Si \(d(A,\Dr)>\rho\), alors l'intersection est vide; si \(d(A,\Dr)=\rho\) alors \(HM=0\) donc l'intersection est réduite au point H; si \(d(A,\Dr)<\rho\) alors \(HM=\pm\sqrt{\rho^2-d(A,\Dr)^2}\). Si on note, pour le dernier cas, \(\vu\) un vecteur unitaire de \(\Dr\) alors \(\vect{HM_1}=\sqrt{\rho^2-d(A,\Dr)^2}\vu\) et \(\vect{HM_2}=-\sqrt{\rho^2-d(A,\Dr)^2}\vu=-\vect{HM_1}\), alors les points d'intersection sont symétriques par rapport au point \(H\).
\end{proof}

\subsubsection{Intersection d'une sphère et d'un plan}

\begin{prop}
  Soit \(\P\) un plan de l'espace \(\epsilon\), l'intersection de S et \(\P\) est~:
  \begin{itemize}
  \item vide si \(d(A,\P)>\rho\);
  \item réduite au projeté orthogonal \(H\) de A sur \(\P\) si \(d(A,\P)=\rho\) et dans ce cas la sphère et le plan sont dit tangents;
  \item un cercle de centre H, si \(d(A,\P)<\rho\).
  \end{itemize}
\end{prop}
\begin{proof}
  Soit le point H le projeté orthogonal du point A sur le plan \(\P\). Soit M un point du plan \(\P\). Les vecteurs \(\vect{AH}\) et \(\vect{HM}\) sont orthogonaux, alors le théorème de Pythagore affirme que \(AM^2=d(A,\P)^2+HM^2\). Alors
  \begin{align}
    M \in S &\iff AM^2=d(A,\P)^2+HM^2=\rho^2\\
    &\iff HM^2=\rho^2-d(A,\P)^2.
  \end{align}
Ainsi
\begin{itemize}
\item si \(d(A,\P)>\rho\), alors l'intersection est vide;
\item si \(d(A,\P)=\rho\), alors M est dans l'intersection si et seulement si \(HM=0\) donc si et seulement si \(M=H\) donc l'intersection est réduite au point H;
\item si \(d(A,\P)<\rho\) alors M est dans l'intersection si et seulement si M est dans le plan et si \(HM=\sqrt{\rho^2-d(A,\P)^2}\), donc l'intersection est le cercle de centre H inclus dans \(\P\) de rayon \(\sqrt{\rho^2-d(A,\P)^2}\).
\end{itemize}
\end{proof}

\subsubsection{Intersection de deux sphères}

\begin{prop}
  Soit \(S'\) une sphère de centre \(A'\neq A\) et de rayon \(\rho'\). L'intersection de \(S\) et de \(S'\) est~:
  \begin{itemize}
  \item vide si \(AA'>\rho+\rho'\) ou si \(AA'<\abs{\rho'-\rho}\);
  \item réduite à un point si \(AA'=\rho+\rho'\) ou si \(AA'=\abs{\rho-\rho'}\), dans ce cas les sphères sont tangentes en un point H, qui est sur la droite \((AA')\);
  \item un cercle si \(\abs{\rho-\rho'}<AA'<\rho+\rho'\), le centre de ce cercle est sur \((AA')\) et se situe dans un plan orthogonal à \((AA')\).
  \end{itemize}
\end{prop}
\begin{proof}
  Puisque \(A\neq A'\), soient \(\vc=\frac{\vect{AA'}}{\norme{\vect{AA'}}}\) un vecteur directeur de \((AA')\) \(\va\) un vecteur unitaire orthogonal à \(\vc\) et \(\vb=\va\wedge\vc\). Ainsi \((A,\va,\vb,\vc)\) est un ROND de l'espace. L'équation de \(S\) dans ce ROND est : \(x^2+y^2+z^2=\rho^2\). et \(\vect{AA'}=\norme{\vect{AA'}}\vc=d\vc\) et donc \(A'(0,0,d)\) et l'équation de \(S'\) est telle que \(x^2+y^2+(z-d)^2=\rho'^2\). Soit un point \(M(x,y,z)\) de l'espace \(\epsilon\), alors
  \begin{align}
    M \in S\cap S' &\iff \begin{cases} x^2+y^2+z^2=\rho^2 \\ x^2+y^2+(z-d)^2=\rho'^2 \end{cases} \\
    &\iff \begin{cases} z = \frac{\rho^2-\rho'^2+d^2}{2d}\\ x^2+y^2+\left(\frac{\rho^2-\rho'^2+d^2}{2d}\right)^2=\rho^2 \end{cases}.
  \end{align}
  Alors
  \begin{align} 
    M \in S\cap S' &\iff x^2+y^2=\rho^2-\left(\frac{\rho^2-\rho'^2+d^2}{2d}\right)^2\\
    &\iff x^2+y^2=\left(\rho-\frac{\rho^2-\rho'^2+d^2}{2d}\right)\left(\rho+\frac{\rho^2-\rho'^2+d^2}{2d}\right)\\
    &\iff x^2+y^2=\frac{1}{4d^2}(\rho'-d+\rho)(\rho'+d-\rho)(\rho+d-\rho')(\rho+d+\rho')=\alpha.
  \end{align}
  Le réel \(\alpha\) est du signe de \((\rho'-d+\rho)(\rho'+d-\rho)(\rho+d-\rho')=(\rho'+d-\rho)(d^2-\abs{\rho-\rho'}^2)\). Alors
  \begin{itemize}
  \item si \(d>\rho+\rho'\) alors \(d>\abs{\rho-\rho'}\) donc \(\alpha<0\) et \(S\cap S'=\emptyset\);
  \item si \(d<\abs{\rho+\rho'}\) alors \(d<\rho+\rho'\) donc \(\alpha<0\) et \(S\cap S'=\emptyset\);
  \item si \(d=\rho+\rho'\) ou si \(d=\abs{\rho-\rho'}\) alors \(\alpha=0\) donc
    \begin{equation}
      M\in S\cap S' \iff \begin{cases} z=\frac{\rho^2-\rho'^2+d^2}{2d} \\x^2+y^2=0\end{cases},
    \end{equation}
    et l'intersection est constituée d'un seul point H dont les coordonnées sont de la forme \((0,0,z_H)\) et \(\vect{AH}=z_H \vc\), le point H est sur la droite \((AA')\);
  \item si \(\abs{\rho-\rho'}\leqslant d \leqslant \rho+\rho'\) alors \(\alpha>0\) donc
    \begin{equation}
      M\in S\cap S' \iff \begin{cases} z=\frac{\rho^2-\rho'^2+d^2}{2d} \\x^2+y^2=\alpha\end{cases},
    \end{equation}
    et l'intersection est constituée d'un cercle dans le plan d'équation \(z=\frac{\rho^2-\rho'^2+d^2}{2d}\) et soit \(B\) le centre du cercle, qui est sur la droite \((AA')\). Ce cercle est dans le plan \((B,\va,\vb)\) qui est orthogonal à \(\vc\) qui dirige \((AA')\).
  \end{itemize}
\end{proof}
Dans le cas particulier où \(S\) et \(S'\) ont le même centre, alors on choisit un ROND de centre \(A\) et on a soit \(\rho=\rho'\) et donc \(S\cap S'=S=S'\) ou alors \(\rho\neq \rho'\) et donc \(\S\cap S' =\emptyset\).

\subsection{Projection orthogonale d'un cercle sur un plan}

Soient \(\courbe\) un cercle contenu dans un plan \(\P\) et \(\P'\) un deuxième plan. On définit l'application \(\pi:\epsilon \longmapsto \P'\) qui à un point M associe son projeté orthogonal sur \(\P'\). On veut déterminer l'ensemble \(X=\pi(\courbe)\). On note A le centre du cercle \(\courbe\). On pose \(H=\pi(A)\) le projeté orthogonal de A sur \(\P'\). 

Soit \(\vn\) un vecteur normal unitaire du plan \(\P\). Soit \(\Dr\) la droite passant par \(A\) de vecteur directeur \(\vn\). Soit \(\Dr'\) l'image par \(\pi\) de la droite \(\Dr\). Soit \(\vi\) un vecteur unitaire de \(\Dr'\) et soit \(\vj\) orthogonal à \(\vi\) unitaire dans le plan \(\P'\). On pose \(\vk=\vi\wedge \vj\). Le repère \((H,\vi,\vj,\vk)\) est donc un ROND de l'espace \(\epsilon\).

Soit un point B tel que \(\vect{AB}=\vn\), alors B est un point de la droite \(\Dr\) et soit \(B'=\pi(B)\) son projeté orthogonal sur \(\P'\), alors \(B'\) est un point de \(\Dr'\). Le produit scalaire \(\vn\cdot\vj=(\vect{AH}+\vect{HB'}+\vect{B'B})\cdot\vj\) vaut \(\vect{HB'}\cdot\vj\) puisque \(\vect{AH}\) et \(\vect{B'B}\) sont orthogonaux à \(\P'\) donc à \(\vj\). Les point \(H\) et \(B'\) sont sur la droite \(\Dr'\), le vecteur \(\vi\) dirige \(\Dr'\) donc \(\vect{HB'}\) est colinéaire à \(\vi\). Par définition \(\vi\cdot \vj=0\) alors \(\vn\cdot \vj=0\), ainsi \(\vn\) est dans le plan \((\vi,\vk)\), il existe alors un réel \(\theta\in[0,2\pi[\) tel que \(\vn=\cos\theta\vi+\sin\theta\vk\). Alors si on résume~:
\begin{itemize}
\item l'équation de \(\P'\) est telle que \(z=0\),
\item les coordonnées du point A sont telles que \(A(0,0,d)\),
\item les coordonnées du projeté orthogonal de \(M(x,y,z)\) sur \(\P'\) sont telles que \(\pi(M)(x,y,0)\),
\item l'équation de \(\P\) est telle que 
  \begin{align}
    M\in\P &\iff \vect{AM}\cdot\vn=0\\ &\iff x\cos\theta+(z-d)\sin\theta=0
  \end{align}
\item l'équation de \(\courbe\) est telle que 
  \begin{align}
    M\in\courbe &\iff \begin{cases} AM=r\\M\in\P \end{cases}\\
    &\iff \begin{cases} x^2+y^2+(z-d)^2=r^2\\ \cos\theta x+\sin\theta (z-d)=0\end{cases}.
  \end{align}
\end{itemize}
À partir de ce point, on peut déterminer l'ensemble \(X\).

Soit \(N(x,y,0)\), alors~:
\begin{align}
  N \in X &\iff \exists M \in \courbe \ N=\pi(M)\\
&\iff \exists (\alpha,\beta,\gamma) \in \R^3 \ M(\alpha,\beta,\gamma) \ N=\pi(M)\\
&\iff \exists \gamma \in \R \ M(x,y,\gamma)\in\courbe\\
&\iff \exists \gamma \in \R \begin{cases} x^2+y^2+(\gamma-d)^2=r^2\\ \cos\theta x+\sin\theta (\gamma-d)=0\end{cases}.
\end{align}
On distingue plusieurs cas selon la valeur de \(\theta\).
\begin{enumerate}
\item Dans le cas où \(\sin \theta=0\) alors \(\vn=\pm\vi\), alors \(\vn\) et \(\vk\) sont orthogonaux, et les plans \(\P\) et \(\P'\) aussi. Alors
  \begin{equation}
    N \in X \iff \exists \gamma \in \R \begin{cases} x^2+y^2+(\gamma-d)^2=r^2\\ x=0\end{cases},
  \end{equation}
donc 
\begin{equation}
  N \in X \iff \exists \gamma \in \R \ \begin{cases} x=0 \\ \gamma=d\pm\sqrt{r^2-y^2}\end{cases}.
\end{equation}
Finalement 
\begin{equation}
  N \in X \iff  \begin{cases}x=0 \\ -r\leqslant y \leqslant r\end{cases}.
\end{equation}
On trouve un segment de la droite \((H,\vj)\)
\item Dans le cas où \(\sin \theta\neq 0\), alors les plans \(\P\) et \(\P'\) ne sont pas orthogonaux. Alors
\begin{equation}
  N \in X \iff \exists \gamma \in \R \ \begin{cases} x^2+y^2+(\gamma-d)^2=r^2\\ \cos\theta x+\sin\theta (\gamma-d)=0\end{cases}.
\end{equation}
Alors 
\begin{equation}
  N \in X \iff \exists \gamma \in \R \ \begin{cases} \gamma=d-\frac{\cos\theta}{\sin\theta}x \\ x^2+y^2+\frac{\cos^2\theta}{\sin^2\theta}x^2=r^2\end{cases}.
\end{equation}
Si la deuxième équation est vraie alors il existe un réel \(\gamma\) tel que la première et la deuxième soient vraies. Si la deuxième équation est fausse alors les deux équations sont fausses. Donc
\begin{equation}
  N \in X \iff x^2\left(1+\frac{\cos^2\theta}{\sin^2\theta}\right)+y^2=r^2.
\end{equation}
Soit alors 
\begin{equation}
  N \in X \iff \frac{x^2}{r^2\sin^2\theta}+\frac{y^2}{r^2}=1.
\end{equation}
\begin{enumerate}
\item si \(\sin^2\theta\neq 1\) alors \(\vn\) et \(\vk\) ne sont pas colinéaires donc les plans \(\P\) et \(\P'\) ne sont pas parallèles, \(X\) est alors une ellipse;
\item si \(\sin^2\theta= 1\) alors \(\vn\) et \(\vk\) sont colinéaires donc les plans \(\P\) et \(\P'\) sont parallèles, \(X\) est alors le cercle de centre \(H=\pi(A)\) de même rayon que \(\courbe\).
\end{enumerate}
On vient de trouver \(X\) quelque soit \(\theta\).
\end{enumerate}
Dans tous les cas, l'ensemble \(X\) est trouvé.
\clearpage
\section{Exercices}
\begin{exercice}
    Soient \(\Dr_1 = \begin{cases} y = x+1 \\z = -1 \end{cases}\), \(\Dr_2 = \begin{cases} y = -x+1 \\z = 0 \end{cases}\) et \(\Dr_3 = \begin{cases} y = x-1 \\z = 1 \end{cases}\). On note \(\vu\) le vecteur de coordonnées \((1,2,3)\). Déterminer l'ensemble des droites \(\Dr\) de l'espace rencontrant les trois droites \(\Dr_1, \Dr_2, \Dr_3\) et orthogonales à \(\vu\).
\end{exercice}
\begin{exercice}
    \begin{enumerate}
        \item Étudier la position relative d'une droite et d'une sphère en fonction du rayon \(R\) de la sphère et de la distance \(d\) du centre de la sphère à cette droite.
        \item Soit \(ABC\) un triangle non aplati. Déterminer l'ensemble \(\mathcal{E}\) des points \(M\) de l'espace tels que \(\vect{MA}, \vect{MB}\) et  \(\vect{MC}\) soient deux à deux orthogonaux.
    \end{enumerate}
\end{exercice}
\begin{exercice}
    On se place dans l'espace muni d'un repère orthonormal \((O, \vi, \vj, \vk)\).
    \begin{enumerate}
        \item Donner une équation cartésienne du plan \(\P\) passant par \(A(1,3,0)\) et orthogonal au vecteur \(\vu(-2,3,1)\).
        \item Donner une équation cartésienne du plan \(\mathcal{Q}\) passant par le point \(B(5,1,-2)\) et contenant les vecteurs \(\vv(4,4,4)\) et \(\vw(1, -1, 0)\).
        \item Soit \(\Dr\) la droite d'intersection des plans \(\P\) et \(\mathcal{Q}\). Donner un point et un vecteur directeur de \(\Dr\).
    \end{enumerate}
\end{exercice}
\begin{exercice}
    Soit \(ABCD\) un tétraèdre. On suppose que les quatre face on la même aire. Montrer que les arêtes opposées ont la même longueur.
\end{exercice}
\begin{exercice}
    On se place dans l'espace muni d'un repère orthonormal direct \((O, \vi, \vj, \vk)\). Soient \(A(1,2,0)\), \(B(-1,0,3)\), et \(C(0,2,-1)\) trois points.
    \begin{enumerate}
        \item Donner une équation cartésienne du plan \(\P\) contenant les points \(A, B, C\).
        \item Donner une équation cartésienne du plan \(\mathcal{Q}\) passant par \(A\) et orthogonal à \((BC)\) ; du plan \(\mathcal{R}\) passant par \(B\) et orthogonal à la droite \((AC)\).
        \item Déterminer les coordonnées cartésiennes de l'orthocentre du triangle \(ABC\).
        \item Déterminer l'ensemble des points équidistants de \(A, B, C\).
    \end{enumerate}
\end{exercice}
\begin{exercice}
    Soit \(\mathcal{R} = (O, \vi, \vj, \vk)\) un repère orthonormal direct de l'espace. On définit le point \(\Omega\) de coordonnées \((-1,2,4)\) dans \(\mathcal{R}\) et les vecteurs \(\vect{I} = \vi+\vj, \vect{J} = \vi - \vj, \vect{K} = \vi - \vk\).
    \begin{enumerate}
        \item Montrer que \(\mathcal{R}' = (\Omega, \vect{I}, \vect{J}, \vect{K})\) est un repère de l'espace. Est-il direct ou indirect ? Est-il orthogonal, orthonormal ?
        \item Soit \(A\) le point de coordonnées \((1, 2, 3)\) dans \(\mathcal{R}\). Déterminer ses coordonnées dans \(\mathcal{R}'\).
        \item Donner une équation du plan passant par \(A\) orthogonal à \(\vn = \vi + \vj\) dans \(\mathcal{R}'\).
    \end{enumerate}
\end{exercice}
\begin{exercice}
    Soient \(\va, \vb\) deux vecteur de l'espace. Résoudre l'équation \(\va \wedge \vect{x} = \vb\) d'inconnue un vecteur \(\vect{x}\) de l'espace.
\end{exercice}
\begin{exercice}
    Calculer le volume du parallélépipède construit sur les vecteurs \(\vu(-1,1,1), \vv(1,2,-1), \vw(2,0,3)\).
\end{exercice}
\begin{exercice}
    Déterminer les coordonnées du projeté orthogonal du point \(A(1,1,4)\) sur le plan \(\P\) d'équation cartésienne \(x-2y+2z = 2\) puis calculer \(d(A,\P)\).
\end{exercice}
\begin{exercice}
    Soient \(\Dr\) et \(\Dr'\) deux droites passant respectivement par \(A(1,2,3)\) et \(A'(0,4,-1)\) et dirigées respectivement par \(\vu(4,2,1)\) et \(\vu'(1,1,0)\).
    \begin{enumerate}
        \item Calculer \(d(\Dr,\Dr')\).
        \item Déterminer une équation cartésienne de la perpendiculaire commune à \(\Dr\) et à \(\Dr'\).
    \end{enumerate}
\end{exercice}
\begin{exercice}
    On considère un cube \(ABCDEFGH\) tel que l'arête \(AB\) est égale à \(1\) et tel que les points \(E, F, G, H\) sont les images dans cet ordre des points \(D, A, B, C\) par la translation de vecteur \(\vect{DE}\). On se propose de déterminer la perpendiculaire commune \(\Delta\) des droites \((AC)\) et \((DF)\). On se place dans le repère \((A, \vect{AF}, \vect{AD}, \vect{AB})\).
    \begin{enumerate}
        \item Faire un dessin et préciser les coordonnées des sommets \(B, C, D\) et \(F\)
        \item Soient \(P\) et \(Q\) les points d'intersection respectifs de \(\Delta\) avec \((CA)\), et \((DF)\). On définit \(u\) et \(v\) tels que \(\vect{AP} = u \vect{AC}\) et \(\vect{DQ} = v \vect{DF}\). Démontrer les deux égalités suivantes~: \[u \norme{\vect{AC}}^2 -v \vect{DF}\cdot\vect{AC} = \vect{AD}\cdot\vect{AC}\] et \[u \vect{AC}\cdot\vect{DF} - v\norme{\vect{DF}}^2= \vect{AD}\cdot\vect{DF}\]
        \item Déterminer les coordonnées de \(P\) et \(Q\).
        \item Soient \(I\) le point d'intersection des droites \((DP)\) et \((AB)\), \(J\) le point d'intersections des droites \((AQ)\) et \((ED)\). Déterminer les coordonnées de \(I\) et \(J\) et donner une construction simple de \(\Delta\).
    \end{enumerate}
\end{exercice}
