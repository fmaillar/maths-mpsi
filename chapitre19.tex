\chapter{Fractions rationnelles}
\label{chap:fractionrationnelles}
\minitoc
\minilof
\minilot

Dans tout le chapitre \(\K\) désigne un corps.

\section{Corps des fractions rationnelles à une indéterminée}

\subsection{Notion de fraction rationnelle}

On dispose de l'anneau intègre \(\K[X]\) des polynômes à une indéterminée dans le corps \(\K\). On admet qu'on peut définir sons corps des fractions, noté \(\K(X)\), comme le plus petit corps dont \(\K[X]\) est un sous-anneau (définition avec unicité à isomorphisme près).

Les éléments de \(\K(X)\) sont appelés fractions rationnelles à une indéterminée à coefficients dans \(\K\). Ils sont de la forme \(F=\frac{A}{B}\) avec \((A,B) \in \K[X]^2\) et \(B\) non nul.

Si \(F \in \K(X)\), un couple  \((A,B) \in \K[X]\times \K[X]\setminus\{0\}\) tel que \(F=\frac{A}{B}\) est appelé un représentant de \(F\). Il n'y a pas unicité de \((A,B)\).

Pour tous \((A_1,B_1,A_2,B_2) \in \K[X]^2\) avec \(B_1\) et \(B_2\) non nuls on a
\begin{equation}
  \frac{A_1}{B_1} = \frac{A_2}{B_2}   \iff A_1B_2 -A_2B_1 =0.
\end{equation}

\subsubsection{Lois de compositions internes}
\begin{itemize}
\item On définit l'addition par
  \begin{equation}
    \forall \left(\frac{A}{B}, \frac{C}{D} \right) \in \K(X)^2 \quad \frac{A}{B}+\frac{C}{D} = \frac{AD+BC}{BD};
  \end{equation}
\item on définit le produit par
\begin{equation}
    \forall \left(\frac{A}{B}, \frac{C}{D} \right) \in \K(X)^2 \quad \frac{A}{B} \times \frac{C}{D} = \frac{AC}{BD}.
  \end{equation}
\end{itemize}
Ces définitions sont légitimes puisque \(AC\) et \(BD\) sont des polynômes (donc \(AC+BD\) en est un aussi). De plus \(BD \neq 0\). Vérifions que ces définitions ne dépendent pas du représentant choisi. Si \(\frac{A}{B}=\frac{A^{*}}{B^{*}}\) et \(\frac{C}{D}=\frac{C^{*}}{D^{*}}\) alors
\begin{align}
  & (A^{*}D^{*}+B^{*}C^{*})BD - (AD+BC)B^{*}D^{*} \\
  &= A^{*}BDD^{*}+BB^{*}C^{*}D-AB^{*}DD^{*}-BB^{*}CD^{*} \\
  &=(A^{*}B-AB^{*})DD^{*} - BB^{*}(C^{*}D-CD^{*}) =0.
\end{align}
Alors la définition de l'addition ne dépend pas du représentant. Vérifions le aussi pour le produit. Si \(\frac{A}{B}=\frac{A^{*}}{B^{*}}\) et \(\frac{C}{D}=\frac{C^{*}}{D^{*}}\) alors
\begin{align}
  & A^{*}C^{*}BD-ACB^{*}D^{*} \\
  &= AB^{*}(C^{*}D-CD^{*}) \quad (A^{*}B=B^{*}A) \\
  &=0 \quad (C^{*}D=CD^{*}).
\end{align}
Alors la définition du produit ne dépend pas du représentant.

\emph{Éléments neutres}~: Pour l'addition, l'élément neutre est \(0_{\K(X)}=\frac{0_{\K[X]}}{1_{\K[X]}}\) avec
\begin{equation}
  \forall \frac{A}{B} \in \K(X) \quad \frac{A}{B} = 0_{\K(X)} \iff A = 0_{\K[X]}.
\end{equation}
Pour le produit, l'élément neutre est \(1_{\K(X)}=\frac{1_{\K[X]}}{1_{\K[X]}}\) avec
\begin{equation}
  \forall \frac{A}{B} \in \K(X) \quad \frac{A}{B} = 1_{\K(X)} \iff A = B.
\end{equation}

\emph{Symétriques}~: Pour l'addition, le symétrique de \(\frac{A}{B}\) et \(-\frac{A}{B}=\frac{-A}{B}=\frac{A}{-B}\). Pour la multiplication, si \(F=\frac{A}{B}\) est non nul alors c'est \(F^{-1}=\frac{B}{A}\).

\subsubsection{Immersion de \(\K[X]\) dans \(\K(X)\)} 
L'application \(\fonction{\varphi}{\K[X]}{\K(X)}{P}{\frac{P}{1_{\K[X]}}}\) est un morphisme injectif d'anneaux. L'image de \(\varphi\), \(\Image(\varphi)\), est un sous anneau de \(\K(X)\). Ainsi, \(\varphi\) est un isomorphisme de \(\K[X]\) sur \(\Image(\varphi)\), ce qui permet d'identifier \(P\) et \(\varphi(P)\). On peut alors considérer \(\K[X]\) comme un sous anneau de \(\K(X)\). En identifiant également les polynômes constants à des éléments de \(\K\), on a aussi~: \(\K\) est un sous corps de \(\K(X)\).

\emph{Remarque}~: La fraction rationnelle nulle, \(0_{\K(X)}\), est identifiée au scalaire nul, \(0_\K\). Idem pour l'unité.

Soit \((A,B) \in \K[X]^2\) avec \(B\) non nul. Si \(B\) divise \(A\) dans \(\K[X]\), il existe un polynôme \(Q \in \K[X]\) tel que \(A=BQ\) et alors \(A\cdot 1-BQ=0\) donc \(\frac{A}{B}=\frac{Q}{1}=Q\).

``Le quotient'' de deux polynômes au sens de la divisibilité des polynômes, lorsqu'il existe, est égal à la fraction rationnelle \(\frac{A}{B}\).

\subsubsection{Conjuguée d'une fraction rationnelle à coefficients complexes}

\begin{defdef}
  Soit \(F=\frac{A}{B} \in \C(X)\). On définit la fraction rationnelle \(\bar{F}\), la fraction rationnelle conjuguée, par \(\bar{F}=\frac{\bar{A}}{\bar{B}}\).
\end{defdef}

C'est légitime puisque~:
\begin{enumerate}
\item Les polynômes \(\bar{A}\) et \(\bar{B}\) sont dans \(\C[X]\) et comme \(B \neq 0\) alors \(\bar{B} \neq 0\).
\item Soient deux représentants \(F=\frac{A}{B}=\frac{A^*}{B^*}\) alors \(AB^*=A^*B\) d'où dans \(\K[X]\) on a \(\bar{A}\bar{B^*}=\bar{B}\bar{A^*}\) et ainsi \(\bar{F}=\frac{\bar{A}}{\bar{B}}=\frac{\bar{A^*}}{\bar{B^*}}\). La définition du conjugué ne dépend pas du représentant choisi. En utilisant les propriétés de conjugaison sur \(\C[X]\) on montre que~:
  \begin{equation}
    \forall (F,G) \in \C(X)^2 \qquad \overline{F+G}=\overline{F}+\overline{G} \quad \overline{FG}=\overline{F} \cdot \overline{G}.
  \end{equation}
\end{enumerate}

\subsection{Fraction rationnelle dérivée}

\begin{defdef}
  Soit \(F=\frac{A}{B}\in\K(X)\). On définit la fraction rationnelle dérivée de \(F\), notée \(F'\) par~:
  \begin{equation}
    F' = \frac{A'B-B'A}{B^2}.
  \end{equation}
\end{defdef}
C'est légitime puisque~:
\begin{enumerate}
\item \(A'B-B'A \in \K[X]\), \(B^2 \in \K[X] \setminus\{0\}\) donc c'est bien une fraction rationnelle.
\item Soient deux représentants \(F=\frac{A}{B}=\frac{A_1}{B_1}\) alors on sait que
  \begin{align}
    AB_1-A_1B = 0 \label{eq:bidulezz1}\\
    A'B_1 +AB_1' - A_1'B-A_1B'=0\label{eq:bidulezz2}
  \end{align}
  on a dérivé la première égalité. Ensuite
  \begin{align}
    &(A'B-AB')B_1^{2}-(A_1'B_1-A_1B_1')B^2\\
    &=A'BB_1^{2} - AB'B_1^{2} - A_1'B_1B^{2} -A_1B_1'B^{2} \\
    &=(A'B-A_1'B)BB_1 - (AB_1)B'B_1 +(A_1B)B_1'B\\
    &=(A_1B'-AB_1')BB_1 - (A_1B)B'B_1 +(AB_1)B_1'B\\
    &=0.
  \end{align}
  d'après les équations \eqref{eq:bidulezz1} et \eqref{eq:bidulezz2}. Donc
  \begin{equation}
    \frac{A'B-AB'}{B^2}=\frac{A_1'B_1-A_1B_1'}{B_1^{2}}.
  \end{equation}
  La définition de la dérivée ne dépend pas du représentant choisi.
\item Vérifions que c'est compatible avec la définition de la dérivée sur \(\K[X]\). Si \(P\in\K[X]\), alors on identifie \(P\) à la fraction rationnelle \(F=\frac{P}{1}\) et \(F'=\frac{P'1-P1'}{1^2}=\frac{P'}{1}\) identifié à \(P\). La dérivée de \(P\) en tant que fraction rationnelle est égale (ou identifiée) à sa dérivé en tant que polynôme.
\end{enumerate}

\begin{theo}
  Pour toutes fractions rationnelle \(F\) et \(G\) de \(\K(X)\) on a
  \begin{align}
    (F+G)'&=F'+G' \\
    (FG)&=F'G+FG'.
  \end{align}
\end{theo}
\begin{proof}
  On note \(F=\frac{A}{B}\) et \(G=\frac{C}{D}\) avec \(A,B,C\) et \(D\) quatre polynômes et \(B\) et \(D\) tous les deux non nuls. Alors \(F+G=\frac{AD+BC}{BD}\) et \(FG=\frac{AC}{BD}\). Lorsqu'on dérive on a~:
  \begin{align}
    (F+G)' &= \frac{(AD+BC)'BD-(AD+BC)(BD)'}{(BD)^2}\\
    &=\frac{A'BD^2+ABDD'+BB'CD+B^2C'D}{B^2D^2}\\
    &\phantom{=}-\frac{AD^2B'+ABDD'+BB'CD+B^2CD'}{B^2D^2}\\
    &=\frac{(A'B-AB')D^2}{B^2D^2} +\frac{B^2(C'D-CD')}{B^2D^2} \\
    &=\frac{A'B-AB'}{B^2} +\frac{C'D-CD'}{D^2}\\
    &=F'+G'.
  \end{align}
  Pour le produit
  \begin{align}
    (FG)' &= \frac{(AC)'BD-(AC)(BD)'}{(BD)^2}\\
    &=\frac{A'BCD+ABC'D-AB'CD-ABCD'}{B^2D^2}\\
    &=\frac{(A'B-AB')CD}{B^2D^2} + \frac{AB(C'D-CD')}{B^2D^2}\\
    &=F'G +FG'.
  \end{align}
\end{proof}

\subsection{Degré d'une fonction rationnelle}

\begin{defdef}
  Soit \(F=\frac{A}{B} \in \K(X)\). On définit le degré de \(F\) par
  \begin{equation}
    \deg(F)=\deg(A)-\deg(B).
  \end{equation}
  Déjà \(\deg(B) \in \N\) puisqu'il est non nul et \(\deg(A) \in \N\cup\{-\infty\}\). On peut montrer que \(\deg(F)=-\infty\) si et seulement si \(F=0\).
\end{defdef}

C'est légitime puisque~:
\begin{enumerate}
\item Si \(F=\frac{A}{B}=\frac{A_1}{B_1}\) alors \(AB_1=A_1B\) et \(\deg(AB_1)=\deg(A_1B)\) et donc \(\deg(A)-\deg(B)=\deg(A_1)-\deg(B_1)\). La définition de \(\deg(F)\) ne dépend pas du représentant choisi pour \(F\).
\item Si \(P\in\K[X]\), on identifie \(P\) à la fraction rationnelle \(\frac{P}{1}\). Son degré en tant que fraction rationnelle vaut \(\deg(P)-\deg(1)=\deg(P)\) est égal à son degré en tant que polynôme.
\end{enumerate}

\begin{prop}
  Soient \(F\) et \(G\) dans \(\K(X)\), alors
  \begin{align}
    \deg(F+G) \leqslant \max(\deg(F),\deg(G)) \\
    \deg(FG) = \deg(F)+\deg(G).
  \end{align}
\end{prop}
\begin{proof}
  Soient \((A,B)\) et \((C,D)\) deux représentants respectifs de \(F\) et \(G\). Alors \(F+G=\frac{AD+BC}{BD}\) et \(FG=\frac{AC}{BD}\). Leurs degrés valent~:
  \begin{align}
    \deg(F+G) &= \deg(AD+BC)-\deg(BD) \\ 
    & \leqslant \max(\deg(AB),\deg(BC))-\deg(B)-\deg(D) \\
    & \leqslant \max(\deg(A)+\deg(D),\deg(B)+\deg(C))-\deg(B)-\deg(D)\\
    & \leqslant \max(\deg(A)-\deg(B),\deg(C)-\deg(D))\\
    & \leqslant \max(\deg(F),\deg(G));
  \end{align}
  et pour le produit
  \begin{align}
    \deg(FG) &= \deg(AC)-\deg(BD) \\
    &= \deg(A)+\deg(C)-\deg(B)-\deg(D)\\
    &=\deg(F)+\deg(G).
  \end{align}
\end{proof}

\subsection{Représentants  irréductibles d'une fraction rationnelle}

\begin{defdef}
  Soit \(F=\frac{A}{B} \in \K(X)\). On appelle représentant irréductible de \(F\) tout couple \((A_1,B_1) \in \K[X]^2\) tel que \(B_1\neq 0\) et \(A_1 \wedge B_1=1\) et \(F=\frac{A_1}{B_1}\).

  On dira qu'un représentant est irréductible est unitaire si, de plus, \(B_1\) est unitaire.
\end{defdef}

\begin{theo}
  Toute fraction rationnelle admet au moins un représentant irréductible.
\end{theo}
\begin{proof}
  Soit \(F=\frac{A}{B} \in \K(X)\). Soit \(D=A \wedge B \neq 0\) (car \(B\neq 0\)). Il existe un couple \((A_1,B_1) \in \K[X]^2\) tel que \(A=DA_1\), \(B=DB_1\) et \(A_1 \wedge B_1=1\) avec \(B_1 \neq 0\) (car \(B_1\mid{}B\) et \(B\neq 0\)). Alors \(AB_1=DA_1B_1=A_1B\) donc \(\frac{A}{B}=\frac{A_1}{B_1}\).
\end{proof}

\begin{theo}
  Soient quatre polynômes \(A,B,A_1\) et \(B_1\) tels que \(B\) et \(B_1\) tous les deux non nuls et \(A_1\) et \(B_1\) sont premiers entre eux. Alors
  \begin{equation}
    \frac{A}{B}=\frac{A_1}{B_1} \iff \exists D \in \K[X]\setminus\{0\} \begin{cases} A=DA_1 \\ B=DB_1 \end{cases}.
  \end{equation}
  Auquel cas \(D\) est associé à \(A\wedge B\).
\end{theo}
\begin{proof}
  \(\impliedby\) S'il existe \(D \in \K[X]\setminus\{0\}\) tel que \( \begin{cases} A=DA_1 \\ B=DB_1 \end{cases}\) alors on a \(AB_1=DA_1B_1=A_1B\) et donc \(\frac{A}{B}=\frac{A_1}{B_1}\).

  \(\implies\) Si \(AB_1=A_1B\) alors \(B_1 \mid{}A_1B\) et \(A_1 \wedge B_1=1\) donc le théorème de Gar\ss{} affirme que \(B_1\mid{}B\). Il existe alors un polynôme \(D \in \K[X]\) non nul (puisque \(B\) est non nul) tel que \(B=B_1D\). Alors
  \begin{align}
    AB_1=A_1B=A_1B_1D \\
    B_1(A-A_1D)=0
  \end{align}
  comme \(B_1\) et non nul et que l'anneau \(\K[X]\) est intègre on a \(A=A_1D\). Puisque \(D\) est non nul soit son coefficient dominant \(\alpha \neq 0\). On pose \(\widetilde{D}=\frac{D}{\alpha}\) (\(\widetilde{D}\) est unitaire). Alors
  \begin{align}
    A \wedge B &= (A_1D) \wedge (B_1D) \\
    & = (\alpha A_1 \widetilde{D}) \wedge (\alpha B_1 \widetilde{D}) \\
    & = \widetilde{D} (\alpha A_1) \wedge (\alpha B_1) && \widetilde{D} \text{~unitaire}\\
    & = \widetilde{D} = \frac{D}{\alpha}.
  \end{align}
  Ainsi \(A \wedge B\) et \(D\) sont associés.
\end{proof}

\begin{theo}\label{theo:thm3}
  Soient quatre polynômes \(A_1,B_1,A_2\) et \(B_2\) tels que \(B_1\) et \(B_2\) tous les deux non nuls, \(A_1\) et \(B_1\) premiers entre eux et \(A_2\) et \(B_2\) premiers entre eux. Alors
  \begin{equation}
    \frac{A_1}{B_1}=\frac{A_2}{B_2} \iff \exists \lambda \in \K\setminus\{0\} \begin{cases} A_2=\lambda A_1 \\ B_2 = \lambda B_1 \end{cases}.
  \end{equation}
\end{theo}
\begin{proof}
  \(\implies\) D'après le théorème précédent, il existe un polynôme \(D\) non nul tel que \(A_2=D A_1\) et \(B_2=DB_1\) et \(D\) est associé à  \(A_2 \wedge B_2=1\). Donc \(D\) est un polynôme constant non nul. Autrement dit, il exist un scalaire \(\lambda\) non nul tel que \(D=\lambda\).

  \(\impliedby\) S'il existe un scalaire \(\lambda\) non nul tel que \(\begin{cases} A_2=\lambda A_1 \\ B_2 = \lambda B_1 \end{cases}\) alors \(A_1B_2=\lambda A_1B_1=A_2B_1\) et donc \(\frac{A_1}{B_1}=\frac{A_2}{B_2}\).
\end{proof}

\begin{prop}
  \begin{equation}
    \forall F \in \C(X) \quad F \in \R(X) \iff \bar{F}=F.
  \end{equation}
\end{prop}
\begin{proof}
  \(\implies\) C'est clair.

  \(\impliedby\) Soit \(F=\frac{P}{Q}\) avec \((P,Q)\) un représentant irréductible unitaire. Alors \(\bar{F}=\frac{\bar{P}}{\bar{Q}}=\frac{P}{Q}=F\) avec \(\bar{P}\wedge \bar{Q} =1\). Il existe donc un complexe \(\lambda\) non nul tel que \(\bar{Q}=\lambda Q\) et \(\bar{P}=\lambda P\). Comme \(Q\) et \(\bar{Q}\) sont unitaires on a \(\lambda=1\) et donc \(\bar{Q}=Q\) et \(\bar{P}=P\) et ainsi \(F \in \R(X)\).
\end{proof}

\subsection{Racines et pôles d'une fraction rationnelle}

\begin{defdef}
  Soit \(F=\frac{A}{B} \in \K(X)\). On suppose que \((A,B)\) est un représentant irréductible de \(F\). Soit un scalaire \(\alpha\) et un naturel non nul \(r\). On dit que~:
  \begin{itemize}
  \item \(\alpha\) est racine de \(F\) d'ordre \(r\) si et seulement si \(\alpha\) est racine de \(A\) d'ordre \(r\);
  \item \(\alpha\) est pôle de \(F\) d'ordre \(r\) si et seulement si \(\alpha\) est racine de \(B\) d'ordre \(r\).
  \end{itemize}
\end{defdef}

C'est légitime, puisque
\begin{enumerate}
\item si \((A,B)\) et \((A_1,B_1)\) sont deux représentants irréductibles de \(F\), d'après le théorème~
\ref{theo:thm3}, il existe un scalaire \(\lambda\) non nul tel que \(\begin{cases} A=\lambda A_1 \\ B = \lambda B_1 \end{cases}\).

  Pour tout scalaire \(\alpha\) et tout naturel non nul \(r\), \(\alpha\) est racine de \(A\) (resp. \(B\)) d'ordre \(r\) si et seulement si \(\alpha\) est racine de \(A_1\) (resp. \(B_1\)) d'ordre \(r\).

  La définition ne dépend pas du représentant irréductible choisi, mais il doit être irréductible.
\item Si \(P \in \K[X]\), pour tout couple \((A_1,B_1) \in \K[X] \times \K[X]\setminus\{0\}\), \((A_1,B_1)\) est un représentant irréductible de \(\frac{P}{1}\) si et seulement s'il existe un scalaire \(\lambda\) non nul tel que \(A_1=\lambda P\) et \(B_1=\lambda\).

  Les racines de \(P\) en tant que polynôme sont les mêmes que les racines de \(P\) en tant que fraction rationnelle avec les mêmes ordres.
\end{enumerate}

Il faut faire attention au corps, les racines et les pôles en dépendent. Par exemple \(F=\frac{X^2+1}{X^2+2}\).

\emph{Remarque}~: Soit \(F=\frac{A}{B} \in \K(X)\setminus\{0\}\), on note \(\ZZ_F\) l'ensemble des racines de \(F\) et \(\P_F\) l'ensemble de ses pôles. Alors~: \(\ZZ_F\) est fini, \(\P_F\) est fini et \(\ZZ_F \cap \P_F =\emptyset\).

\begin{proof}
  On suppose que \(\frac{A}{B}\) est un représentant irréductible de \(F\). \(F\) est non nul donc \(A\) est non nul. Ainsi \(\ZZ_F\) est l'ensemble des racines de \(A\) et comme \(A\) est non nul alors \(\ZZ_F\) est fini. De la même manière, le polynôme \(B\) est non nul donc l'ensemble de ses racines, c.-à.-d. \(\P_F\), est fini.
  
  S'il existe \(\alpha \in \ZZ_F \cap \P_F\) alors \(X-\alpha\mid{}A\) et \(X-\alpha\mid{}B\) mais comme \(A\) et \(B\) sont premiers entre eux, c'est absurde. Donc il n'existe aucun élément dans \(\ZZ_F \cap \P_F\).
\end{proof}

\subsection{Fonction rationnelle associée à une fraction rationnelle}

\begin{defdef}
  Soit \(F \in \K(X)\), \((A_1,B_1)\) un représentant irréductible de \(F\) et \(\P_F\) l'ensemble des pôles de \(F\). On définit l'application \(\widetilde{F}\), fonction rationnelle associée à la fraction rationnelle \(F\), par
  \begin{equation}
    \fonction{\widetilde{F}}{\K\setminus\P_F}{\K}{x}{\frac{\widetilde{A_1}(x)}{\widetilde{B_1}(x)}}.
  \end{equation}
\end{defdef}

C'est légitime, puisque~:
\begin{itemize}
\item Pour tout scalaire \(x \in \K\setminus\P_F\), \(\widetilde{B_1}(x) \neq 0\), donc \(\frac{\widetilde{A_1}(x)}{\widetilde{B_1}(x)}\) est bien défini dans \(\K\).
\item si \(\frac{A_2}{b_2}\) est un autre représentant irréductible de \(F\), il existe un scalaire non nul \(\lambda\) tel que \(\begin{cases}A_2 = \lambda A_1 \\ B_2=\lambda B_1\end{cases}\), donc pour tout \(x \in \K\setminus\P_F\) on a \(\frac{\widetilde{A_2}(x)}{\widetilde{B_2}(x)}=\frac{\lambda \widetilde{A_1}(x)}{\lambda \widetilde{B_1}(x)}=\frac{\widetilde{A_1}(x)}{\widetilde{B_1}(x)}\).
\item Si \(P \in \K[X]\), on dispose de la fonction polynomiale associée \(\widetilde{P}\). Le polynôme \(P\) est identifié à \(F=\frac{P}{1}\). Vérifions que \(\widetilde{F}=\widetilde{P}\). On sait que \(\P_F=\emptyset\) donc \(\fonction{\widetilde{F}}{\K}{\K}{x}{\frac{\widetilde{P}(x)}{\widetilde{1}(x)}=\widetilde{P}(x)}\). Donc  \(\widetilde{F}=\widetilde{P}\). 
\end{itemize}

\begin{prop}
  Soient \(F\) et \(G\) dans \(\K(X)\), \(\P_F\) (resp \(\P_G\)) l'ensemble des pôles de \(F\) (resp de \(G\)). Pour tout scalaire \(\lambda\) et tout \(x \in \K\setminus(\P_F \cup \P_G)\) on a
  \begin{align}
    \widetilde{F+G}(x) &= \widetilde{F}(x) + \widetilde{G}(x) \\
    (\widetilde{FG})(x)&= \widetilde{F}(x) \cdot \widetilde{G}(x) \\
    (\widetilde{\lambda F})(x)&=\lambda \widetilde{F}(x)
  \end{align}
\end{prop}
\begin{proof}
  Évident d'après la définition.
\end{proof}

\begin{prop}
  Soient \((F,G) \in \K[X]^2\). On suppose que les fractions rationnelles associées \(\widetilde{F}\) et \(\widetilde{G}\) coïncident en un nombre infini de valeurs de \(\K\). Alors \(F=G\).
\end{prop}
\begin{proof}
  Soir \(F=\frac{A}{B}\) et \(G=\frac{C}{D}\) deux représentants irréductibles. Alors
  \begin{align}
    \courbe &=\{x \in \K, \widetilde{F}(x)=\widetilde{G}(x)\} \\
    &=\{x \in \K, \frac{\widetilde{A}(x)}{\widetilde{B}(x)}=\frac{\widetilde{C}(x)}{\widetilde{D}(x)}\} \\
    &=\{x \in \K, \widetilde{AD-BC}(x)=0 \}\\
  \end{align}
  Comme \(\courbe\) est infini, on en déduit par la dernière égalité que le polynôme \(AD-BC\) est nul et par conséquent \(F=G\).
\end{proof}

\subsection{Composition}

\begin{defdef}
  Soit \(F=\frac{P}{Q} \in \K(X)\). Soit \(R\) un polynôme non nul de \(\K[X]\). On définit la fraction rationnelle de \(F\) et de \(R\) par \(F \circ F = F(R) = \frac{P \circ R}{Q \circ R}\).
\end{defdef}

C'est légitime, puisque~:
\begin{enumerate}
\item \(P \circ R\) et \(Q \circ R\) sont dans \(\K[X]\) et \(Q \circ R\) est non nul car \(Q\) et \(P\) sont non nuls;
\item Si \(F=\frac{P}{Q}=\frac{P_1}{Q_1}\) alors
  \begin{align}
    PQ_1&=P_1Q \\
    (PQ_1) \circ R &= (P_1Q) \circ R\\
    (P\circ R)(Q_1\circ R) &= (P_1\circ R)(Q\circ R)
  \end{align}
  donc \(\frac{P \circ R}{Q \circ R}=\frac{P_1 \circ R}{Q_1 \circ R}\). La définition ne dépend pas du représentant choisi;
\item Si \(P \in \K[X]\), \(F=\frac{P}{1}\) est identifiée à \(P\) et
  \begin{equation}
    \forall R \in \K[X]\setminus\{0\} \quad F \circ R = \frac{P \circ R}{1 \circ R}=\frac{P \circ R}{1}  
  \end{equation}
  identifiée à \(P\circ R\). La définition est compatible avec celle sur \(\K[X]\).
\end{enumerate}

\begin{prop}
  \begin{equation}
    \forall (F,R) \in \K(X)\times\K[X]\setminus\{0\} \quad \deg(F \circ R)=\deg(F) \cdot \deg(R).
  \end{equation}
\end{prop}
\begin{proof}
  Soit \(F=\frac{P}{Q}\), alors
  \begin{align}
    \deg(F \circ R) = \deg\left(\frac{P \circ R}{Q \circ R}\right) &= \deg(P\circ R) - \deg(Q\circ R) \\
    &=\deg(P)\deg(R)-\deg(Q)\deg(R)\\
    &=\deg(R)(\deg(P)-\deg(Q))\\
    &=\deg(R)\deg(F).
  \end{align}
\end{proof}

\emph{Remarque}~: De la même manière que pour les polynômes, on note indifféremment pour toute fraction rationnelle \(F \in \K(X)\) \(F\) ou \(F(X)\).

\begin{defdef}
  Soit \(F \in \K(X)\). \(F\) est paire si et seulement si \(F(X)=F(-X)\) et \(F\) est impaire si et seulement si \(F(X)=-F(-X)\).
\end{defdef}

\section{Étude locale d'une fraction rationnelle}

\subsection{Partie entière d'une fraction rationnelle}

\begin{theo}
  Soit \(F \in \K(X)\). Il existe un unique couple \((E,G) \in \K[X]\times\K(X)\) tel que \(F=E+G\) et \(\deg(G)<0\).

  Autrement dit, il existe un unique polynôme \(E \in \K[X]\) tel que \(\deg(F-E)<0\). Ce polynôme \(E\) est appelé la partie entière de la fraction rationnelle \(F\).
\end{theo}
\begin{proof}[Unicité]
  Soient deux couples \((E,G) \in \K[X]\times\K(X)\) et \((E_1,G_1) \in \K[X]\times\K(X)\) tels que
  \begin{equation}
    F=E+G=E_1+G_1 \qquad \deg(G)<0 \ \deg(G_1)<0.
  \end{equation}
  Alors \(E-E_1=G_1-G \in \K(X)\) et \(\deg(G_1-G)\leqslant \max(\deg(G_1),\deg(G))<0\), donc \(\deg(E-E_1)<0\). De plus \(E-E_1 \in \K[X]\) alors \(\deg(E-E_1)=-\infty\) et donc \(E-E_1\) est nul. Finalement \(E=E_1\) et \(G=G_1\).
\end{proof}
\begin{proof}[Existence]
  Soit \((A,B) \in \K[X]^2\) avec \(B\) non nul, un représentant quelconque de la fraction rationnelle \(F\). Le polynôme \(B\) est non nul, on peut donc effectuer la division euclidienne de \(A\) par \(B\)~:
  \begin{equation}
    \exists! (Q,R) \in \K[X]^2 \quad \begin{cases} A=BQ+R \\ \deg(R) < \deg(B) \end{cases}
  \end{equation}
  Alors \(F=\frac{A}{B}=Q+\frac{R}{B}\), avec \(Q \in \K[X]\), \(\frac{R}{B} \in \K(X)\) et \(\deg\left(\frac{R}{B}\right)=\deg(R)-\deg(B)<0\). Le couple \((Q,\frac{R}{B})\) convient.
\end{proof}

\emph{Retenir que la partie entière de la fraction rationnelle \(\frac{A}{B}\) est égale au quotient de la division euclidienne de \(A\) par \(B\).}

En particulier, si \(E\) désigne la partie entière de la fonction rationnelle \(\frac{A}{B}=F\).
\begin{equation}
  E=0 \iff \deg(F) <0.
\end{equation}

\emph{Remarque}~: \(\K(X)\) est un \(\K\)-espace vectoriel car \(\K\) est un sous corps de \(\K(X)\).

\begin{prop}
  L'application
  \begin{equation}
    \fonction{\psi}{\K(X)}{\K[X]}{F}{\text{partie entière de } F}
  \end{equation}
  est une application linéaire de \(\K\)-espaces vectoriels.
\end{prop}
\begin{proof}
  Soit \((F,G) \in \K(X)\) et \(\lambda \in \K\). On note \(H=\lambda F+G\). Soient \(A;B,C\) et \(D\) dans \(\K[X]\) tels que \(B\) et \(D\) sont tous les deux non nuls et tels que \(F=\frac{A}{B}\) et \(G=\frac{C}{D}\).

  Effectuons les divisions euclidiennes de \(A\) par \(B\) et de \(C\) par \(D\)~:
  \begin{align}
    \exists! (Q_1,R_1) \in \K[X]^2 \quad \begin{cases} A = BQ_1+R_1 \\ \deg(R_1) < \deg(B) \end{cases} \\ 
    \exists! (Q_2,R_2) \in \K[X]^2 \quad \begin{cases} C = DQ_2+R_2 \\ \deg(R_2) < \deg(D) \end{cases}
  \end{align}
  alors \(\psi(F)=Q_1\) et \(\psi(G)=Q_2\). On calcule \(H\)~:
  \begin{align}
    H = \lambda F+G &= \lambda\left(Q_1+\frac{R_1}{B}\right)+\left(Q_2+\frac{R_2}{D}\right) \\
    &=(\lambda Q_1 +Q_2) +\left(\lambda \frac{R_1}{B} +\frac{R_2}{D} \right).
  \end{align}
  De plus
  \begin{equation}
    \deg\left(\lambda \frac{R_1}{B} +\frac{R_2}{D} \right) \leqslant \max\left(\deg\left(\lambda \frac{R_1}{B}\right),\deg\left(\lambda \frac{R_2}{D}\right)\right) <0.
  \end{equation}
  Ainsi,
  \begin{equation}
    \psi(H)= \lambda Q_1+Q_2 = \lambda \psi(F)+\psi(G).
  \end{equation}
  alors \(\psi\) est linéaire.
\end{proof}

\subsection{Partie polaire d'une fraction rationnelle}

\subsubsection{Définition de la partie polaire}

\begin{lemme}
  Soient \(F \in \K(X)\), \(a \in \K\) et \(n \in \N^*\). On suppose que \(a\) est un pôle de \(F\) d'ordre \(n\). Alors il existe un unique couple \((R,G) \in \K[X] \times \K(X)\) tel que
  \begin{equation}
    \begin{cases} F = \frac{R}{(X-a)^n} +G \\ a\notin \P_G \\ \deg(R)<n \end{cases}.
  \end{equation}
\end{lemme}
\begin{proof}[Unicité]
  Soient \((R,G) \in \K[X] \times \K(X)\) et \((R_1,G_1) \in \K[X] \times \K(X)\) tels que
\begin{equation}
    \begin{cases} F = \frac{R}{(X-a)^n} +G=\frac{R_1}{(X-a)^n} +G_1 \\ a\notin \P_G\cup\P_{G_1} \\ \deg(R)<n \\ \deg(R_1)<n \end{cases}.
  \end{equation}
  Alors
  \begin{equation}
    \frac{R-R_1}{(X-a)^n}=G_1-G
  \end{equation}
  Comme \(a\) n'est pas un pole de \(G\) ni de \(G_1\), alors ce n'est pas un pôle de \(G_1-G\). Par l'absurde, supposons que \(R-R_1\neq 0\), alors \(a\) est une racine de \(R-R_1\) d'ordre \(p\) (\(p \in \intervalleentier{0}{n-1}\), \(a\) n'est pas forcément racine, on peut prendre \(p=0\), \(p\leqslant n-1\) car \(\deg(R-R_1)\leqslant n-1\)). Alors il existe un polynôme \(Q\) tel que
  \begin{equation}
    R-R_1=(X-a)^p \quad \text{et} \quad Q(a)\neq 0.
  \end{equation}
Ainsi
\begin{equation}
  \frac{R-R_1}{(X-a)^n} = \frac{(X-a)^pQ}{(X-a)^n} = \frac{Q}{(X-a)^{n-p}}.
\end{equation}
Comme \(Q(a)\neq 0\), \(a\) est un pôle d'ordre \(n-p\geqslant n-(n-1)=1\). Alors \(a\) est ``vraiment'' un pôle. Cependant \(\frac{R-R_1}{(X-a)^n}=G_1-G\) et on avait pris comme hypothèse que \(a\) n'est pas un pôle de \(G_1-G\). (Contradiction).
\end{proof}
\begin{proof}[Existence]
  Soit \((A,B)\) un représentant irréductible de \(F\). Comme \(a\) est un pôle de \(F\) d'ordre \(n\), il existe un polynôme \(C \in \K[X]\) tel que
  \begin{equation}
    \begin{cases} B=(X-a)^nC \\ C(a) \neq 0 \end{cases}.
  \end{equation}
  alors \(F=\frac{A}{(X-a)^nC}\). Comme \(C(a)\neq 0\), \(C\) et \(X-a\) sont premiers entre eux. On en déduit que \((X-a)^n\) et \(C\) sont premiers entre eux. Le théorème de Bezout affirme qu'il existe un couple \((U,V) \in \K[X]^2\) tel que
  \begin{equation}
    (X-a)^nU+CV=1,
  \end{equation}
  alors
  \begin{equation}
    F = \frac{A(X-a)^nU+ACV}{(X-a)^nC} = \frac{AU}{C} + \frac{AV}{(X-a)^n}.
  \end{equation}
  Comme \((X-a)^n\) est non nul, on peut effectuer la division euclidienne de \(AV\) par \((X-a)^n\)~:
  \begin{equation}
    \exists! (Q,R) \in \K[X]^2 \quad \begin{cases} AV=(X-a)^nQ+R \\ \deg(R)<n \end{cases}.
  \end{equation}
  On reporte cette équation dans l'expression de \(F\)~:
  \begin{align}
    F&=\frac{AU}{C} + \frac{(X-a)^nQ+R}{(X-a)^n}\\
    &=\frac{R}{(X-a)^n} + \left(\frac{AU}{C}+Q\right)
  \end{align}
  avec \(R \in \K[X]\) (\(\deg(R)<n\)) et \(\frac{AU}{C}+Q \in \K(X)\). Comme \(C(a)\neq 0\), le couple \((R,\frac{AU}{C}+Q)\) convient.
\end{proof}

\begin{theo}
  Soient \(F \in \K(X)\), \(a \in \K\) et \(n \in \N^*\). On suppose que \(a\) est un pôle de \(F\) d'ordre \(n\). Alors il existe un unique \(n\)-unlet \((\alpha_1, \ldots, \alpha_n) \in \K^n\) et une unique \(G \in \K(X)\) tels que
  \begin{equation}
    \begin{cases} F = \sum_{i=1}^n \frac{\alpha_i}{(X-a)^i} + G \\ a \notin \P_G \end{cases}
  \end{equation}
\end{theo}

\begin{defdef}
  La quantité \(\sum_{i=1}^n \frac{\alpha_i}{(X-a)^i}\) du théorème est appelée partie polaire de la fraction rationnelle \(F\) relative au pôle \(a\).
\end{defdef}

\begin{proof}[Unicité]
  S'il existe \((\alpha_1, \ldots, \alpha_n) \in \K^n\) et \(G_1 \in \K(X)\) tels que \(F=\sum_{i=1}^n \frac{\alpha_i}{(X-a)^i} + G_1\) et \(a\) n'est pas un pôle de \(G_1\) alors
  \begin{align}
    F &= \sum_{i=1}^n \frac{\alpha_i(X-a)^{n-i}}{(X-a)^n} + G_1\\
    &= \frac{\sum_{i=1}^n\alpha_i(X-a)^{n-i}}{(X-a)^n} + G_1
  \end{align}
  avec \(\deg\left(\sum_{i=1}^n\alpha_i(X-a)^{n-i}\right)\leqslant n-1 <n\). D'après l'unicité du lemme, on a forcément
  \begin{equation}
    \left(\sum_{i=1}^n\alpha_i(X-a)^{n-i},G_1\right) = (R,G)_{\text{lemme}}
  \end{equation}
  De plus \((1,X-a,\ldots,(X-a)^{n-1})\) est une base de \(\K_{n-1}[X]\). L'unicité de \(R\) entraîne l'unicité de ses coordonnées \((\alpha_1, \ldots, \alpha_n)\) dans cette base.
\end{proof}
\begin{proof}[Existence]
  D'après le lemme, il existe \((R,G) \in \K[X]\times\K(X)\) tel que
  \begin{equation}
    \begin{cases}
      F = \frac{R}{(X-a)^n}+G \\
      \deg(R)<n\\
      a \notin \P_G
    \end{cases}
  \end{equation}
  Comme \(R \in \K_{n-1}[X]\) et que \((1,X-a,\ldots,(X-a)^{n-1})\) en ait une base, il existe \((\beta_0, \ldots, \beta_{n-1}) \in \K^n\) tel que
  \begin{equation}
    R = \sum_{k=0}^{n-1} \beta_k (X-a)^k.
  \end{equation}
  Ainsi, la fraction rationnelle s'écrit
  \begin{align}
    F &= \frac{\sum_{k=0}^{n-1} \beta_k (X-a)^k}{(X-a)^n}+G \\
    &=\sum_{k=0}^{n-1} \frac{\beta_k}{(X-a)^{n-k}} +G \\
    &=\sum_{i=1}^{n} \frac{\beta_{n-i}}{(X-a)^{i}} +G.
  \end{align}
  Posons pour tout naturel \(i \in \intervalleentier{1}{n}\) \(\alpha_i=\beta_{n-i}\) donc
  \begin{equation}
    F = \sum_{i=1}^n \frac{\alpha_i}{(X-a)^i} + G
  \end{equation}
\end{proof}

\subsubsection{Détermination pratique de la partie polaire}

\paragraph{Cas d'un pôle simple (d'ordre 1)}

Soit \(F\in\K(X)\), \(a \in \P_F\) d'ordre \(1\). Le théorème affirme que
\begin{equation}
  \exists! (\alpha,G) \in \K \times \K(X) \quad \begin{cases} F=\frac{\alpha}{X-a}+G \\ a \notin \P_G \end{cases}.
\end{equation}
Alors
\begin{equation}
  (X-a)F=\alpha+(X-a)G
\end{equation}
et comme \(a\) n'est pas un pôle de \(G\) alors ce n'est pas un pôle de \((X-a)G\). \(a\) est un pôle de \(F\) d'ordre \(1\) alors ce n'est pas un pôle de \((X-a)F\). On peut prendre la valeur en \(a\) de la dernière équation~:
\begin{equation}
  \widetilde{(X-a)F}(a)=\alpha \label{eq:poleordre1}
\end{equation}

\emph{Autres expressions de \(\alpha\)}~: Posons \(F=\frac{A}{B}\) irréductible. Alors il existe un polynôme \(C \in \K[X]\) tel que \(B=(X-a)C\) avec \(C(a)\neq 0\). D'après l'équation \eqref{eq:poleordre1}, on a \(\alpha = \widetilde{(X-a)F}(a) = \widetilde{\left(\frac{A}{C}\right)}(a)\). C'est-à-dire
\begin{equation}
  \alpha = \frac{A(a)}{C(a)}.
\end{equation}
On peut dériver l'expression de \(B\)~: \(B'=C+(X-a)C'\) et trouver que \(B'(a)=C(a)\), alors~:
\begin{equation}
  \alpha = \frac{A(a)}{B'(a)}.
\end{equation}

\paragraph{Calcul de \(\alpha_n\) pour un pôle d'ordre \(n\)}

Soient \(F \in \K(X)\), \(a\in\K\) un pôle de \(F\) d'ordre \(n \in \N^*\).

\begin{theo}
  Il existe un unique \(n\)-unlet de scalaires \((\alpha_1, \ldots, \alpha_n) \in \K^n\) et une unique fraction rationnelle \(G \in \K(X)\) tels que
  \begin{equation}
    \begin{cases}
      F = \sum_{i=1}^n \frac{\alpha_i}{(X-a)^i} +G \\
      a \notin \P_G
    \end{cases}
  \end{equation}
\end{theo}

Auquel cas,
\begin{equation}
  (X-a)^nF = \sum_{i=1}^n \alpha_i(X-a)^{n-i} +(X-a)^nG.
\end{equation}
Comme \(a\) est un pôle de \(F\) d'ordre \(n\), alors ce n'est pas un pôle de \((X-a)^nF\). Ce n'est pas non plus un pôle de \(G\), alors ce n'est pas un pôle de \((X-a)^nG\). On prend la valeur en \(a\) de la dernière équation~:
\begin{equation}
  \widetilde{(X-a)^nF}(a)=\alpha_n.
\end{equation}

\emph{Autres expressions de \(\alpha_n\)}~: Notons \(F=\frac{A}{B}\) irréductible. Il existe alors un polynôme \(C \in \K[X]\) tel que \(B=(X-a)^nC\) avec \(C(a)\neq 0\). Alors \(F=\frac{A}{B}=\frac{A}{(X-a)^nC}\), \(\alpha_n = \widetilde{(X-a)^nF}(a)=\widetilde{\left(\frac{A}{C}\right)}(a)\). C'est-à-dire
\begin{equation}
  \alpha_n=\frac{A(a)}{C(a)}
\end{equation}
On peut dériver \(B\) \(n\) fois et d'après la formule de Leibniz on a
\begin{equation}
  B^{(n)} = \sum_{k=0}^n \binom{n}{k} \frac{n!}{(n-k)!}(X-a)^{n-k}C^{(n-k)}
\end{equation}
La valeur en \(a\) vaut donc \(B^{(n)}(a)=n! C(a)\), et ainsi
\begin{equation}
  \alpha_n = \frac{n! A(a)}{B^{(n)}(a)}
\end{equation}

\paragraph{Cas d'un pôle double}

Soient \(F \in \K(X)\), \(a\in\K\) un pôle de \(F\) d'ordre \(2\). Il existe un unique couple \((\alpha,\beta) \in \K^2\) et une unique fraction rationnelle \(G \in \K(X)\) tels que
\begin{equation}
  \begin{cases}
    F = \frac{\alpha}{(X-a)^2}+\frac{\beta}{X-a}+G \\
    a \notin \P_G
  \end{cases}.
\end{equation}
D'après le paragraphe précédent, on connaît \(\alpha=\widetilde{(X-a)^2F}(a)\). Déterminons \(\beta\).

\emph{Première méthode}~: Posons \(H=F-\frac{\alpha}{(X-a)^2}=\frac{\beta}{X-a}+G\). Alors d'après le premier paragraphe, comme \(a\) est un pôle simple de \(H\), \(\beta=\widetilde{(X-a)H}(a)\).

\emph{Deuxième méthode}~: On sait que
\begin{equation}
  (X-a)^2F=\alpha+\beta(X-a)+(X-a)^2G
\end{equation}
En dérivant, on obtient
\begin{equation}
  [(X-a)^2F]'=\beta+2(X-a)G+(X-a)^2G'.
\end{equation}
Le scalaire \(a\) n'est pas un pôle de \(G\), donc se n'est pas un pôle de \(G'\). Alors en appliquant cette formule en \(a\), on obtient
\begin{equation}
  \beta = \widetilde{[(X-a)^2F]'}(a)
\end{equation}
Alors en synthétisant
\begin{equation}
  \alpha=\widetilde{(X-a)^2F}(a) \quad  \beta = \widetilde{[(X-a)^2F]'}(a)
\end{equation}

\emph{Autres expressions de \(\beta\)}~:

Posons \(F = \frac{A}{B}=\frac{A}{C(X-a)^2}\) irréductible et tel que \(C(a)\neq 0\). On a déjà vu que \(\alpha=\frac{A(a)}{C(a)}=\frac{2A(a)}{B''(a)}\). On a
\begin{align}
  (X-a)^2F&=\frac{A}{C}\\
  [(X-a)^2F]' &=\frac{A'C-C'A}{C^2}
\end{align}
d'où \(\beta = \frac{A'(a)C(a)-C'(a)A(a)}{C(a)^2}\) et en dérivant trois fois~:
\begin{align}
  B&=(X-a)^2C \\
  B'&=2(X-a)C+(X-a)^2C'\\
  B''&=2C+4(X-a)C'+(X-a)^2C''\\
  B^{(3)}&=6C'+6(X-a)C''+(X-a)^2C^{(3)}
\end{align}
et en prenant la valeur en \(a\)~: \(C(a)=\frac{B''(a)}{2}\) et \(C'(a)=\frac{B^{(3)}}{6}\). Ainsi
\begin{align}
  \beta &= \frac{A'(a)\frac{B''(a)}{2}-A(a)\frac{B^{(3)}(a)}{6}}{\frac{B''(a)^2}{4}}\\
  &=\frac{2A'(a)B''(a)-\frac{2}{3}A(a)B^{(3)}(a)}{B''(a)^2}
\end{align}

\subsection[Décomposition en éléments simples]{Décomposition en éléments simples d'une fraction rationnelle}

\begin{defdef}
  On appelle élément simple sur le corps \(\K\) toute fonction fraction rationnelle de la forme~:
  \begin{equation}
    F=\frac{R}{P^j} \quad (R,P,j) \in \K[X]^2\times \N^*
  \end{equation}
  tels que \(P\) est irréductible sur le corps \(\K\), \(\deg(R) < \deg(P)\). Si \(\deg(P)=m\) alors \(F\) est de de \(m\)\ieme espèce.
\end{defdef}

\emph{Exemples}~:
\begin{enumerate}
\item Sur \(\C\), les polynômes irréductibles sont les polynôme de degré \(1\). Les éléments simples sont tous de première espèce. Ce sont les
  \begin{equation}
    \frac{\alpha}{(X-a)^j} \quad (\alpha,a,j)\in \C^2\times \N^*
  \end{equation}
\item Sur \(\R\), les polynômes irréductibles sont les polynômes de degré \(1\) et les polynômes de degré \(2\) dont le discriminant est strictement négatif. On dispose donc~:
  \begin{itemize}
  \item des éléments simples de 1\iere espèce \(\frac{\alpha}{(X-a)^j}\) où \((\alpha,a,j)\in \R^2\times \N^*\);
  \item des éléments simples de 2\ieme espèce \(\frac{\alpha X+\beta}{(X^2+pX+q)^j}\) où \((\alpha,\beta,p,q,j)\in \R^4\times \N^*\) et \(p^2<4q\).
  \end{itemize}
\end{enumerate}

\begin{theo}[Décomposition en éléments simples sur \(\C(X)\)]
  Toute fraction rationnelle \(F\) de \(\C(X)\) se décompose de façon unique en la somme d'un polynôme et d'éléments simples.

  Plus précisément, si \(F=\frac{A}{\prod_{i=1}^n(X-a_i)^{k_i}}\) avec \(A\in\K[X]\), \((a_1,\ldots,a_n) \in \C^n\) deux à deux distincts et \((k_1,\ldots,k_n)\in (\N^*)^n\) alors il existe de façon unique des complexes \(\alpha_{i,j}\) et un polynôme \(E\in\K[X]\) tels que
  \begin{equation}
    F=E+\sum_{i=1}^n\sum_{j=1}^{k_i} \frac{\alpha_{i,j}}{(X-a_i)^{j}}.
  \end{equation}
\end{theo}

\emph{Remarque}~: On ne suppose pas que \(F\) est sous forme irréductible. Si \(a_i\) n'est pas un pôle, alors les \(a_{i,j}\) correspondants sont nuls.

\begin{proof}[Unicité]
  \(E\) est nécessairement la partie entière de la fraction rationnelle \(F\). Si on note, pour tout \(i \in \intervalleentier{1}{n}\)
  \begin{equation}
    P_i = \sum_{j=1}^{k_i} \frac{\alpha_{i,j}}{(X-a_i)^{j}}
  \end{equation}
  alors \(F-P_i\) n'admet pas \(a_i\) pour pôle. Deux cas se présentent~:
  \begin{enumerate}
  \item Si \(a_i\) est un pôle de \(F\), \(F=P_i+(F-P_i)\). Alors \(P_i\) est forcément la partie polaire de \(F\) relative au pôle \(a_i\). D'où l'unicité des \(\alpha_{i,j}\);
  \item Si \(a_i\) n'est pas un pôle de \(F\), alors tous les \(\alpha_{i,j}\) doivent être nuls car sinon \(a_i\) serait un pôle.
  \end{enumerate}
  Dans les deux cas, les \(\alpha_{i,j}\) sont uniquement déterminés.
\end{proof}
\begin{proof}[Existence]
  Soit \(i \in \intervalleentier{1}{n}\). Si \(a_i\) est un pôle de \(F\), on pose \(P_i = \sum_{j=1}^{k_i} \frac{\alpha_{i,j}}{(X-a_i)^{j}}\) la partie polaire de \(F\) relative à \(a_i\), (\(k_i\) est l'ordre du pôle \(a_i\)). Si \(a_i\) n'est pas un pôle on pose \(P_i=0=\sum_{j=1}^{k_i} \frac{0}{(X-a_i)^{j}}\).

  \(E=F-\sum_{i=1}^n P_i\) est une fraction rationnelle et elle n'admet pour pôle aucun des \(a_i\). Donc \(E\) n'admet aucun pôle. Soit \((P,Q)\) une forme irréductible de \(E\), alors \(Q\) n'admet pas de racines. C'est donc un polynôme constant non nul (\(Q \in \C[X]\)). Alors \(E\) est un polynôme.
\end{proof}

Cette décomposition s'appelle la Décomposition en Éléments Simples (DES) de la fraction rationnelle \(F\).

\subsection{Pratique de la DES dans \(\C[X]\)}

Soit \(F \in \C(X)\) tel que \(F=\frac{A}{B}\) avec \((A,B)\in\C[X]^2\) et \(B\) non nul.
\begin{enumerate}
\item Trouver la partie entière de \(F\)
\item Factoriser \(B\) sous la forme
  \begin{equation}
    B=\lambda \prod_{i=1}^n(X-a_i)^{k_i} \quad \lambda \in \C^*
  \end{equation}
  \(a_1,\ldots,a_n \in \C\) deux à deux distincts et \(k_1, \ldots, k_n\) des naturels tous non nuls
\item Écrire la DES a priori, il existe de façon unique des complexes \(\alpha_{i,j}\) tels que
  \begin{equation}
    F=E+\sum_{i=1}^n\sum_{j=1}^{k_i} \frac{\alpha_{i,j}}{(X-a_i)^{j}}
  \end{equation}
\item Il reste à calculer tous les coefficients \(\alpha_{i,j}\)~:
  \begin{itemize}
  \item On a vu des formules pour les pôles simples, pôles doubles, \ldots
  \item On peut aussi utiliser les particularités de \(F\). Par exemple si \(F\) est paire ou impaire, on obtient des relations entre les coefficients en utilisant \(F(-X)=\pm F(X)\) et l'unicité de la DES.
  \end{itemize}

\emph{Exemple}~:
\begin{equation}
  F=\frac{1}{X^2+1}=\frac{1}{(X-\ii)(X+\ii)}
\end{equation}
Il existe un unique couple \((\alpha,\beta) \in \C^2\) tel que \(F=\frac{\alpha}{X-\ii}+\frac{\beta}{X+\ii}\).

\(F\) est paire : \(F(-X)=F(X)=\frac{-\beta}{X-\ii}+ \frac{-\alpha}{X+\ii}\). Alors par unicité de la DES : \(\beta=-\alpha\). On a qu'un seul coefficient à trouver~:
\begin{equation}
  \alpha = \widetilde{(X-\ii)F}(\ii) = \frac{1}{2\ii} = -\frac{\ii}{2}
\end{equation}
donc \(\beta=\frac{i}{2}\). Finalement
\begin{equation}
  F = \frac{-\ii/2}{X-i}+\frac{\ii/2}{X+\ii}
\end{equation}

Cette méthode fonctionne aussi avec des relations du genre \(F(\ii X)=F(X)\) ou autre.
\item S'il reste peu de coefficients à calculer on peut prendre des valeurs de la fonction rationnelle associée à \(F\) pour obtenir des équations
\item On peut également utiliser des limites de la fonction rationnelle associée à \(F\) en \(\pm \infty\), lorsqu'elles existent.
\end{enumerate}

\emph{Fraction rationnelle à coefficients réels}

Si \(F \in \R(X)\), on utilise que \(\bar{F}=F\) pour obtenir des relations entre les coefficients. Si \(a \in \C\) est un pôle de \(F\), alors \(\bar{A}\) est aussi un pôle du même ordre.

\emph{Exemple}~: \(F=\frac{1}{(X-3)(X^2+1)^2}\), alors la partie entière est nulle et
\begin{equation}
  F = \frac{1}{(X-3)(X-\ii)^2(X+\ii)^2}
\end{equation}
Il existe alors de façon unique cinq complexes \(\alpha,\beta,\gamma,\delta\) et \(\epsilon\) tels que
\begin{equation}
  F = \frac{\alpha}{X-3} + \frac{\beta}{X-\ii} +\frac{\gamma}{(X-\ii)^2} +\frac{\delta}{X+\ii} +\frac{\epsilon}{(X+\ii)^2}
\end{equation}
Alors
\begin{align}
  \alpha &= \widetilde{(X-3)F}(3)=\frac{1}{(1+3^2)^2}=\frac{1}{100} \\
  \gamma &= \widetilde{(X-\ii)^2F}(\ii) = \frac{1}{(\ii-3)(\ii+\ii)^2}=\frac{3+\ii}{40}.
\end{align}
Comme \(F \in \R[X]\) alors \(\bar{F}=F\) et donc
\begin{equation}
  \bar{F} = \frac{\bar{\alpha}}{X-3} + \frac{\bar{\beta}}{X+\ii} + +\frac{\bar{\gamma}}{(X+\ii)^2} +\frac{\bar{\delta}}{X-\ii} +\frac{\bar{\epsilon}}{(X-\ii)^2}
\end{equation}
et par unicité
\begin{align}
  \alpha=\bar{\alpha}\\
  \bar{\beta}=\delta\\
  \bar{\gamma}=\epsilon
\end{align}
d'où \(\epsilon = \frac{3-\ii}{40}\). Il ne reste qu'un coefficient à trouver.

\emph{Valeur en zéro}
\begin{align}
  F(0)=-\frac{1}{3}\\
  -\frac{99}{300} = \ii(\beta-\bar{\beta})-\frac{3}{20}\\
  \beta-\bar{\beta}=\frac{9}{50}\ii
\end{align}
\emph{Limite}
\begin{equation}
  0 = \lim\limits_{+\infty}\widetilde{XF(X)}=\alpha+\beta+\delta
\end{equation}
d'où \(\beta+\bar{\beta}=-\frac{1}{100}\). Alors \(\beta = \frac{-1}{200}+\frac{9}{100}\ii\). Finalement~:
\begin{align}
  F = &\frac{-1}{100(X-3)} + \left(\frac{-1}{200}+\frac{9}{100}\ii\right) \frac{1}{X-\ii} + \left(\frac{3+\ii}{40}\right)\frac{1}{(X-\ii)^2} \\ & + \left(\frac{-1}{200}-\frac{9}{100}\ii\right)\frac{1}{X+\ii} + \left(\frac{3-\ii}{40}\right)\frac{1}{(X+\ii)^2}
\end{align}

\emph{DES sur \(\R[X]\)}

\begin{equation}
  F = \frac{1}{(X-3)(X-\ii)^2(X+\ii)^2}
\end{equation}
Il existe alors de façon unique cinq réels \(a,b,c,d\) et \(e\) tels que
\begin{equation}
  F = \frac{a}{X-3} + \frac{bX+c}{X^2+1} + \frac{cX+e}{(X^2+1)^2}
\end{equation}
De la même manière, on trouve \(a=\frac{1}{100}\). On a
\begin{equation}
  \frac{1}{X-3} = (X^2+1)^2F = \frac{a(X^2+1)^2}{X-3} + (bX+c)(X^2+1)+dX+e
\end{equation}
en prenant les valeurs en \(\ii\) et en \(-\ii\) on a
\begin{align}
  \frac{1}{\ii-3}&=di+e \\
  \frac{1}{-\ii-3}&=-di+e
\end{align}
alors \(2e=-\frac{6}{10}\) et donc \(e=-\frac{3}{10}\). Ensuite
\begin{align}
  d =&\frac{1}{\ii}\left(\frac{1}{\ii-3}-e\right)\\
  =&-\frac{1}{10}
\end{align}
en zéro on a
\begin{equation}
  F(0)=-\frac{1}{3} = -\frac{1}{300}+c+e
\end{equation}
alors \(c=-\frac{3}{100}\). En l'infini on a
\begin{equation}
  \lim\limits_{+\infty} \widetilde{XF(X)}=0=a+b
\end{equation}
d'où \(b=-\frac{1}{100}\).

\subsection{Exemple classique DES de \(\frac{P'}{P}\) lorsque \(P \in \C[X]\setminus\C\)}

Soit \(P \in \C[X]\setminus\C\), \(\deg(P)\geqslant 1\). Le polynôme \(P\) est scindé donc il existe \(n \in \N^*\), \(\alpha_1, \ldots, \alpha_n\) des complexes deux à deux distincts, \(k_1,\ldots, k_n\) des entiers naturels non nuls et un complexe non nul \(\lambda\) tels que
\begin{equation}
  P = \lambda \prod_{i=1}^n (X-a_i)^{k_i}
\end{equation}
 Comme \(\deg(P)\geqslant 1\), on a \(\deg(P')=\deg(P)-1\). Alors \(\deg\left(\frac{P'}{P}\right)=-1<0\). La partie entière de \(frac{P'}{P}\) est nulle. Il existe de façon unique des complexes \(\alpha_{i,j}\) tels que
 \begin{equation}
   \frac{P'}{P} = \sum_{i=1}^n \sum_{j=1}^{k_i} \frac{\alpha_{i,j}}{(X-a_i)^j}.
 \end{equation}
Dérivons \(P\)~:
\begin{equation}
  P' = \lambda \sum_{i=1}^n \prod_{j \in \intervalleentier{1}{n} \setminus \{i\}} (X-a_j)^{k_j}(X-a_i)^{k_i-1} k_i
\end{equation}
Alors
\begin{align}
 \frac{P'}{P} &= \frac{\lambda \sum_{i=1}^n \prod_{j \in \intervalleentier{1}{n} \setminus \{i\}} (X-a_j)^{k_j}(X-a_i)^{k_i-1} k_i}{\lambda \prod_{i=1}^n (X-a_i)^{k_i}} \\
 &=\sum_{i=1}^n \left(\frac{\prod_{j \in \intervalleentier{1}{n} \setminus \{i\}} (X-a_j)^{k_j}(X-a_i)^{k_i-1} k_i}{\prod_{i=1}^n (X-a_i)^{k_i}} \right) \\
 &=\sum_{i=1}^n \frac{k_i(X-a_i)^{k_i-1}}{(X-a_i)^{k_i}} \\
 \frac{P'}{P} &= \sum_{i=1}^n \frac{k_i}{X-a_i}
\end{align}
