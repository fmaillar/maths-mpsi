\chapter{Nombres entiers naturels, récurrence et ensembles finis}
\minitoc
\minilof
\minilot
\label{chap:naturels}
\section{Nombres entiers naturels}

\subsection{Ensemble des naturels $\N$}

L'ensemble des entiers naturels $\N=\{0,1,2,\ldots\}$ est muni de deux lois de compositions internes~: une addition notée $+$ et d'une multiplication notée $\times$ ou $\cdot$ ou rien du tout qui vérifient les propriétés suivantes. On notera $\N^*=\N\setminus\{0\}$.

\begin{prop}[Loi $+$]
  Elle est associative 
  \begin{equation}
    \forall (a,b,c) \in \N^3 \quad (a+b)+c=a+(b+c),
  \end{equation}
  commutative
  \begin{equation}
    \forall (a,b) \in \N^2 \quad a+b=b+a,
  \end{equation}
  l'entier $0$ est son élément neutre
  \begin{equation}
    \forall a \in \N \quad a+0=0+a=a,
  \end{equation}
  et tout entier naturel est régulier pour cette loi
  \begin{equation}
    \forall (a,b,c) \in \N^3 \quad a+b=a+c \implies b=c.
  \end{equation}
\end{prop}
\begin{prop}[Loi $\times$]
  Elle est associative 
  \begin{equation}
    \forall (a,b,c) \in \N^3 \quad (a\times b)\times c=a\times (b\times c),
  \end{equation}
  commutative
  \begin{equation}
    \forall (a,b) \in \N^2 \quad a\times b=b\times a,
  \end{equation}
  distributive par rapport à la loi $+$
  \begin{equation}
    \forall (a,b,c) \in \N^3 \quad (a+ b)\times c=a\times c + b\times c,
  \end{equation}
  l'entier $1$ est son élément neutre
  \begin{equation}
    \forall a \in \N \quad a\times 1=1\times a=a,
  \end{equation}
  et tout entier naturel non nul est régulier pour cette loi
  \begin{equation}
    \forall a \in \N^* \ \forall (b,c) \in \N^3 \quad a\times b=a\times c \implies b=c.
  \end{equation}
\end{prop}
On munit l'ensemble $\N$ d'une relation d'ordre total notée $\leq$ qui vérifie les propriétés suivantes
\begin{prop}
  Toute partie non vide de $\N$ admet un plus petit élément et toute partie non vide et majorée de $\N$ admet un plus grand élément.
\end{prop}
\begin{prop}
  La relation d'ordre $\leq$ est compatible avec l'addition et la multiplication. C'est-à-dire que
  \begin{equation}
    \forall (a,b,c,d)\in \N^4 \quad
    \begin{cases}
      a\leq b \\ c \leq d
    \end{cases}
    \iff
    \begin{cases}
      a+c\leq b+d \\ ac\leq bd
    \end{cases}.
  \end{equation}
\end{prop}

 Si $p$ et $q$ sont des entiers naturels tels que $p\leq q$ alors $\intervalleentier{p}{q}= \N\cap\intervalleff{p}{q}$ 

\subsection{Théorème de récurrence}

\begin{theo}[théorème de récurrence]
  \label{theo:rec}
  Soient A une partie de $\N$, et $n_0\in\N$ tel que
  \begin{itemize}
  \item Initialisation~: $n_0\in A$;
  \item Hérédité~: $\forall n\geq n_0 \quad n\in A \implies n+1\in A$;
  \end{itemize}
  Conclusion~: Alors $\forall n \in N \quad n\geq n_0 \implies n \in A$.
\end{theo}
\begin{proof}
  Soit $E=\N\cap \intervallefo{n_0}{+\infty}$, l'ensemble des entiers plus grands ou égaux à $n_0$ et $B=\complement_E A$. Montrons par l'absurde que $B$ est vide. 

Supposons que $B$ soit non vide, alors il admet un plus petit élément noté $\alpha$. Alors par définition de $E$, $\alpha\geq n_0$ ; et par hypothèse d'initialisation $n_0\in A$, donc $\alpha > n_0$, c'est-à-dire donc $\alpha\geq n_0+1\geq 1$. Par suite $\alpha-1\geq n_0$. 

Puisque $\alpha$ est le plus petit élément de $B$, $\alpha-1 \notin B$ donc $\alpha-1 \in A$. Alors par hypothèse d'hérédité, on obtient que $\alpha-1+1=\alpha\in A$. 

L'hypothèse de départ qu'on avait fait nous disait que $\alpha \in B$. L'élément $\alpha$ ne peut être à la fois dans $A$ et dans $B$, on arrive donc à une absurdité, alors B est vide, $E=A$.
\end{proof}
\begin{cor}[Récurrence simple]
  \label{cor:recsimple}
  Soit $\P$ une propriété définie sur $\N\cap \intervallefo{n_0}{+\infty}$ telle que
  \begin{itemize}
  \item Initialisation $\P(n_0)$ est vraie;
  \item Hérédité $\forall n\geq n_0 \quad \P(n) \implies \P(n+1)$;
  \end{itemize}
  Conclusion alors $\forall n\geq n_0 \quad \P(n)$.
\end{cor}
\begin{proof}
  On applique le théorème de récurrence à $A=\enstq{n\in\N}{\P(n)}$ puisque~:
  \begin{itemize}
  \item Initialisation $\P(n_0) \iff n_0 \in A$
  \item Hérédité
    \begin{equation}
      \left(\forall n\geq n_0 \quad \P(n) \implies \P(n+1)\right) \iff \left( \forall n \geq n_0 \quad n\in A \implies n+1\in A\right)
    \end{equation}
  \end{itemize}
  Conclusion alors d'après le théorème $A=\N\cap \intervallefo{n_0}{+\infty}$
\end{proof}
\begin{cor}[Récurrence double]
  \label{cor:recdouble}
  Soit P une propriété définie sur $\N\cap \intervallefo{n_0}{+\infty}$ telle que
 \begin{itemize}
  \item Initialisation $\P(n_0)$ et $\P(n_1)$ sont vraie;
  \item Hérédité $\forall n\geq n_0 \ \P(n)$ et $\P(n+1) \implies \P(n+2)$;
  \end{itemize}
  Conclusion alors $\forall n \in \N\cap \intervallefo{n_0}{+\infty} \ \P(n)$ 
\end{cor}
\begin{proof}
  On définit sur $\N\cap \intervallefo{n_0}{+\infty}$ la propriété $Q(n)=(\P(n) \text{ et } \P(n+1))$ et on applique le corollaire~\ref{cor:recsimple} à $Q$.
\end{proof}
Il ne faut pas oublier d'initialiser deux fois. Il existe aussi des récurrences triples ou multiples, qu'il ne faut pas oublier d'initialiser autant de fois.
\begin{cor}[Récurrence forte]
  \label{cor:recforte}
  Soit P une propriété définie sur $\N\cap \intervallefo{n_0}{+\infty}$ telle que
  \begin{itemize}
  \item Initialisation $\P(n_0)$ est vraie;
  \item Hérédité $\forall n \in \N\cap \intervallefo{n_0}{+\infty} \ (\P(n_0), \ldots, \P(n)) \implies \P(n+1)$
  \end{itemize}
  Conclusion alors $\forall n \in \N\cap \intervallefo{n_0}{+\infty}\ \ \P(n)$
\end{cor}
\begin{proof}
  On définit sur $\N\cap \intervallefo{n_0}{+\infty}$ la propriété $R(n)=(\P(n_0), \ldots, \P(n))$ et on applique le corollaire~\ref{cor:recsimple} à $R$.
\end{proof}

\subsection{Suites définies par une relation de récurrence}

Soit $E$ un ensemble. On rappelle qu'une suite à valeur dans $E$ est une famille d'éléments de $E$ indexée par $\N$. Étant donné $a\in E$ et $f\in E^E$, il existe une seule suite $(u_n)_{n\in\N}$ telle que (I) $u_0=a$ et (H) $u_{n+1}=f(u_n)$. C'est une conséquence du théorème de récurrence (théorème~\ref{theo:rec}).

De même on peut définir une unique suite $(u_n)_{n\in\N}$ par la donnée de~:
\begin{enumerate}
\item $u_0, u_1$ deux éléments de $E$ et une relation de récurrence de la forme $u_{n+2}=f(u_{n+1},u_{n})$ avec $f:E\times E\longmapsto E$ d'après le corollaire~\ref{cor:recdouble}.
\item $u_0$ élément de $E$ et une relation de récurrence de la forme $u_{n+1}=f(n,u_{0},\ldots,u_{n})$ avec $f:\N\times E^{n+1}\longmapsto E$ d'après le corollaire~\ref{cor:recforte}.
\end{enumerate}

\subsection{Exemples}
\begin{defdef}[Suite arithmétique]
  Une suite arithmétique à valeurs dans le corps des complexes $\C$ est une suite définie par la donnée de $u_0\in\C$ et d'une relation de récurrence 
  \begin{equation}
    \forall n \in \N \quad u_{n+1}=u_n+r
  \end{equation}
  où le complexe $r$ est la raison et ne dépend pas de $n$.
\end{defdef}
\begin{prop}
  Soit $u$ une suite arithmétique de raison $r$, alors
  \begin{equation}
    \forall n \in \N \quad u_n=u_0+nr.
  \end{equation}
\end{prop}
\begin{proof}
  Pour tout entier $n$ on définit la propriété $\P(n) \ u_n=u_0+nr$. On initialise et on voit que $\P(0)$ est vraie. Ensuite on vérifie l'hérédité : soit un entier $n$ et on suppose que $\P(n)$ est vraie, alors $u_{n+1}=u_n+r$ par définition et $u_{n+1}=u_0+(n+1)r$ par hypothèse de récurrence, donc $\P(n+1)$ est vraie. Alors d'après le théorème de récurrence la proposition est vraie.
\end{proof}
\begin{prop}
  Soit $u$ une suite arithmétique de premier terme $a\in\C$ et de raison $r\in\C$, alors pour tout entier $n$
  \begin{equation}
    S_n=\sum_{k=0}^n u_k=(n+1)a+r\frac{n(n+1)}{2}.
  \end{equation}
\end{prop}
\begin{proof}
  Soit un entier $n$, alors
  \begin{equation}
    S_n=\sum_{k=0}^n u_0+kr= \sum_{k=0}^n u_0 + r\sum_{k=0}^n k.
  \end{equation}
  On montre par récurrence que $\sum_{k=0}^n k=\frac{n(n+1)}{2}$, en effet pour $n=0$ $\sum_{k=0}^0 k=0=\frac{0(0+1)}{2}$ et pour l'hérédité si on considère que la somme est vraie pour un entier $n \geq 1$ alors $\sum_{k=0}^{n+1} k= n+1 \sum_{k=0}^n k=n+1+\frac{n(n+1)}{2}=\frac{(n+2)(n+1)}{2}$.
\end{proof}
\begin{defdef}[Suite géométrique]
  Une suite géométrique à valeur dans $\C$ est une suite définie par la donnée de $u_0\in\C$ et d'une relation de récurrence
  \begin{equation}
    \forall n \in \N \quad u_{n+1}=ru_n,
  \end{equation}
  où $r\in\C$ est la raison de la suite et ne dépend pas de $n$.
\end{defdef}
\begin{prop}
  Soit $u$ une suite géométrique de raison $r$, alors
  \begin{equation}
    \forall n \in \N \quad u_n=u_0r^n.
  \end{equation}
\end{prop}
\begin{proof}
  Soit pour tout $n\in\N$ la propriété $\P(n)\ u_n=u_0r^n$. On remarque que $\P(0)$ est vraie puisque $r^0=1$. Soit un naturel $n\geq 1$ et supposons que $\P(n)$ soit vraie. Alors $u_{n+1}=r u_n$ par définition et par hypothèse de récurrence $u_{n+1}=r u_n = r u_0 r^n=u_0 r^{n+1}$ donc $\P(n+1)$ est vraie. Par théorème de récurrence pour tout entier $n\in\N$ $\P(n)$ est vraie.
\end{proof}
\begin{prop}
  Soit $u$ une suite géométrique de premier terme $u_0\in\C$ et de raison $r\in\C$. Soient $p$ et $q$ deux entiers naturels tels que $p\leq q$, alors
  \begin{equation}
    S_{p,q}=\sum_{k=p}^q u_k=
    \begin{cases}
      (q-p+1) & r=1 \\
      \frac{u_p-u_{q+1}}{1-r}=u_0r^p\left(\frac{1-r^{q-p+1}}{1-r}\right) & r \neq 1
    \end{cases}.
  \end{equation}
  C'est ``le premier terme écrit'' moins ``le premier terme négligé'' sur ``un moins la raison''.
\end{prop}
\begin{proof}
  Si $r=1$ alors $S_{p,q}=\sum_{k=p}^qu_0=(q-p+1)u_0$ et sinon on a
  \begin{align}
    (1-r)S_{p,q}=(1-r)\sum_{k=p}^q u_0r^k &= \sum_{k=p}^q u_0r^k - \sum_{k=p}^q u_0r^{k+1}\\
&=\sum_{k=p}^q u_0r^k - \sum_{j=p+1}^{q+1} u_0r^{j}\\
&=u_0 r^p -u_0 r^{q+1} = u_p-u_{q+1},
  \end{align}
et puisque $r\neq 1$ on a bien  $S_{p,q}=\frac{u_p-u_{q+1}}{1-r}$.
\end{proof}
\begin{prop}
  Soient $a$ et $b$ deux complexes et $n$ un entier naturel, alors
  \begin{equation}
    a^{n+1}-b^{n+1}=(a-b)\sum_{k=0}^n a^k b^{n-k}.
  \end{equation}
\end{prop}
\begin{proof}
  \begin{itemize}
  \item si $a=b$ la proposition est vraie puisque $0=0$;
  \item si $a=0$ alors $-b^{n+1}=-b b^n$;
  \item si $a\neq 0$ et $a\neq b$, on définit la suite géométrique $u$ de premier terme $1$ et de raison $\frac{b}{a}$ qui existe puisque $a\neq 0$ et qui est différente de $1$ puisque $a\neq b$. En appliquant la proposition précédente on a~: 
\begin{equation}
  S_{0,n}=\sum_{k=0}^n\left(\frac{b}{a}\right)^k=\frac{1-\left(\frac{b}{a}\right)^{n+1}}{1-\frac{b}{a}},
\end{equation}
donc
\begin{equation}
  a^n \sum_{k=0}^n\left(\frac{b}{a}\right)^k = \frac{a^{n+1}-b^{n+1}}{a-b},
\end{equation}
finalement
\begin{equation}
  (a-b)\sum_{k=0}^n b^ka^{n-k}=(a-b)\sum_{k=0}^n b^{n-k}a^{k}=a^{n+1}-b^{n+1}.
\end{equation}
\end{itemize}
On peut aussi le démontrer directement en utilisant les sommes télescopiques comme dans le calcul de $S_{p,q}$.
\end{proof}

\section{Entiers naturels $n!$ et $\binom{n}{k}$}

\subsection{Factorielle}

\begin{defdef}
  Pour tout entier naturel $n$ non nul, on définit $n!=\prod_{k=1}^n k$ et par convention $0!=1$.
\end{defdef}
\begin{prop}
  La suite $(n!)_{n\in\N}$ est l'unique suite $u$ vérifiant
  \begin{equation}
    \begin{cases}
      u_0=1 \\
      \forall n\geq 1 \quad u_{n+1}=(n+1) u_n
    \end{cases}.
  \end{equation}
\end{prop}
\begin{proof}
  L'unicité est assurée par le théorème de récurrence.
\end{proof}

\subsection{Coefficients binomiaux -- formule de Pascal}

\begin{defdef}[Coefficients binomiaux]
  Pour tout entier naturels $n$ et $p$ on définit un nombre noté $\binom{n}{p}$ par~:
  \begin{equation}
    \begin{cases}
      \binom{n}{p}=\frac{n!}{p!(n-p)!} & 0\leq p \leq n \\
      \binom{n}{p}=0 & p > n
    \end{cases}.
  \end{equation}
\end{defdef}
\begin{prop}
  Soit un entier naturel $n$, alors $\binom{n}{0}=1$, $\binom{n}{1}=n$ et $\binom{n}{2}=\frac{n(n-1)}{2}$.
\end{prop}
\begin{proof}
  \begin{gather}
    \forall n \in \N \quad \binom{n}{0}=\frac{n!}{0!(n-0)!}=1; \\
    \forall n \in \N \ \forall m\in \N^* \quad \binom{m}{1}=\frac{m!}{1!(m-1)!}=m \ \binom{0}{1}=0;\\
    \forall n \geq 2 \quad \binom{n}{2}=\frac{n!}{2!(n-2)!}=\frac{n(n-1)}{2} \ \binom{0}{2}=\binom{1}{2}=0.
  \end{gather}
\end{proof}
\begin{prop}
  \begin{gather}
    \forall (n,p)\in\N^2 \quad 0\leq p \leq n \quad \binom{n}{p}=\binom{n}{n-p};\\
    \forall (n,p)\in\N\times\N^* \quad \binom{n}{p}=\frac{n}{p}\binom{n-1}{p-1} \quad \binom{n}{p}=\frac{n-p+1}{p}\binom{n}{p-1};\\
    \forall (n,p)\in{\N^*}^2 \ \binom{n}{p}=\binom{n-1}{p-1}+\binom{n-1}{p}.
  \end{gather}
\end{prop}
\begin{proof}
  \begin{enumerate}
  \item Il faut et il suffit de changer $n$ par $n-p$ dans la définition;
  \item soient $(p,n)\in\N^2\ p\neq 0$ alors~:
    \begin{itemize} 
    \item si $p>n$ alors $\binom{n}{p}=0$ et $p-1>n-1$ alors $\binom{n-1}{p-1}=0$ d'où l'égalité, si $0<p\leq n$ alors
      \begin{align}
        \binom{n}{p}=\frac{n!}{p!(n-p)!}&=\frac{n}{p}\frac{(n-1)!}{(p-1)!((n-1)-(p-1))!}\\
        &=\frac{n}{p}\binom{n-1}{p-1}
      \end{align}
    \item si $p>n$ $\binom{n}{p}=0$ alors $p-1\geq n$ et si $p-1=n$ alors $n-p+1=0$ et les deux membres de l'égalité sont nuls; si $p-1>n$  $\binom{n}{p-1}=0$ et les deux membres de l'égalité sont nuls.
    \item si $0<p<n$ alors
      \begin{align}
        \binom{n}{p}&=\frac{n-p+1}{p}\frac{n!}{(p-1)!(n-p+1)(n-p)!}\\
        &=\frac{n-p+1}{p}\frac{n!}{(p-1)!(n-p+1)!}
      \end{align}
      donc $\binom{n}{p}=\frac{n-p+1}{p}\binom{n}{p-1}$
    \end{itemize}
  \item Soient $n$ et $p$ deux entiers tout deux non nuls, alors
    \begin{itemize}
    \item si $p>n$ alors $p>p-1>n-1$ alors $\binom{n}{p}=0=0+0=\binom{n-1}{p-1}+\binom{n-1}{p}$
    \item si $0<p\leq n-1$ alors $p<n$ et $p-1<n-1$ donc
      \begin{align}
        \binom{n-1}{p}+\binom{n-1}{p-1}&=\frac{(n-1)!}{p!(n-1-p)}+\frac{(n-1)!}{(p-1)!((n-1)-(p-1))!}\\
        &=\frac{(n-1)!}{(p-1)!(n-1-p)!}\left(\frac{1}{p}+\frac{1}{n-p}\right)\\
        &=\frac{(n-1)!}{(p-1)!(n-1-p)!}\frac{n}{p(n-p)}\\
        &=\frac{n!}{p!(n-p)!}=\binom{n}{p}
      \end{align}
    \item si $p=n$ alors
      \begin{equation}
        \binom{n-1}{p}+\binom{n-1}{p-1}=\binom{n-1}{n}+\binom{n-1}{n-1}=0+1=\binom{n}{n}.
      \end{equation}
    \end{itemize}
  \end{enumerate}
\end{proof}
\begin{prop}
  Soient deux entiers naturels $n$ et $p$, alors $\binom{n}{p}$ est un entier naturel.
\end{prop}
\begin{proof}
  On démontre ce résultat par récurrence sur $n\in\N$. On définit la propriété $\P(n)$ ``$\forall p \in \N \ \binom{n}{p}\in\N$''. On vérifie l'étape d'initialisation $n=0$ : $\binom{0}{p}=\begin{cases} 0 & p>0 \\ 1 & p=1\end{cases}$ donc $\P(0)$ est vraie. 
  
Vérifions l'hérédité, soit $n \in \N$ et supposons que $\P(n)$ soient vraie, alors $\binom{n+1}{0}=1\in \N$. Pour tout $p$ non nul $\binom{n+1}{p}=\binom{n}{p}+\binom{n}{p-1}$ d'après la relation de Pascal. L'hypothèse de récurrence donne $\binom{n}{p}\in\N$ et $\binom{n}{p-1}\in\N$ et puisque l'addition est une loi de composition interne sur $\N$ alors $\binom{n+1}{p}\in\N$. Donc $\P(n+1)$ est vraie. 

Le théorème de récurrence nous permet de conclure et de dire que la proposition $\P(n)$ est vraie pour n'importe quel entier naturel $n$.
\end{proof}

\subsection{Formules du binôme de Newton}

\begin{prop}
  Pour tous complexes $a$ et $b$ et tout entier $n$,
  \begin{equation}
    (a+b)^n=\sum_{k=0}^n \binom{n}{k}a^kb^{n-k}.
  \end{equation}
\end{prop}
\begin{proof}
  Soient $a$,$b$ deux complexes et $n$ un entier naturel, alors on définit la propriété $\P(n)$ ``$(a+b)^n=\sum_{k=0}^n \binom{n}{k}a^kb^{n-k}$''. Vérifions l'étape initiale~:
\begin{equation}
  (a+b)^0=1=\binom{0}{0}a^0b^0=\sum_{k=0}^0 \binom{0}{k}a^kb^{0-k},
\end{equation}
$\P(0)$ est vraie. Vérifions ensuite l'hérédité en supposant que $\P(n)$ soit vraie alors
\begin{align}
  (a+b)^n&=(a+b)(a+b)^n=(a+b)\sum_{k=0}^n \binom{n}{k}a^kb^{n-k}\\
  &=\sum_{k=0}^n \binom{n}{k}a^{k+1}b^{n-k}+\sum_{k=0}^n \binom{n}{k}a^kb^{n-k+1}\\
  &=\sum_{j=1}^{n+1} \binom{n}{j-1}a^{j}b^{n-j+1}+\sum_{k=0}^n \binom{n}{k}a^kb^{n-k+1}\\
  &=\binom{n}{n}a^{n+1}b^0 +\binom{n}{0}a^0b^{n+1} + \sum_{k=1}^{n}\left[\binom{n}{k-1}+\binom{n}{k}\right]a^kb^{n+1-k}\\
  &=\binom{n+1}{n+1}a^{n+1}b^0+\binom{n+1}{0}a^0b^{n+1}+ \sum_{k=1}^{n}\binom{n+1}{k}a^kb^{n+1-k}\\
  &=\sum_{k=0}^{n+1}\binom{n+1}{k}a^kb^{n+1-k}.
\end{align}
Donc $\P(n+1)$ est vraie.

Le théorème de récurrence nous permet donc de conclure et d'affirmer que pour tout entier $n$ la proposition $\P(n)$ est vraie.
\end{proof}
Cas particulier :
\begin{equation}
  \sum_{k=0}^n \binom{n}{k}=2^n.
\end{equation}

\section{Ensembles finis -- dénombrement}

\subsection{Notion d'ensemble fini}

\begin{defdef}
  Soit $E$ un ensemble. $E$ est fini s'il existe un entier $n$ non nul et une bijection $\varphi:\intervalleentier{1}{n} \longmapsto E$, sinon $E$ est infini.
\end{defdef}
\begin{theo}\label{theo:ensemblefini}
  Soient $p$ et $q$ deux entiers naturels non nuls. S'il existe une bijection $\varphi:\intervalleentier{1}{p} \longmapsto \intervalleentier{1}{q}$ alors $p=q$.
\end{theo}
\begin{proof}
  Hors programme, cependant la démonstration est écrite en annexe~\ref{chap:ensemblesFinis}.
\end{proof}
\begin{prop}[Définition]
  Soit $E$ un ensemble fini. Alors l'entier naturel $n$ de la définition est unique et on l'appelle le cardinale de $E$ et on le note $\Card E$.
\end{prop}
Par convention, l'ensemble vide est fini et son cardinal vaut $0$.
\begin{proof}
  Soient deux bijections $\varphi:\intervalleentier{1}{n} \longmapsto E$ et $\psi:\intervalleentier{1}{p} \longmapsto E$ (on note $\psi^-1:E \longmapsto \intervalleentier{1}{p}$). Par composé de deux bijections, l'application $\psi^-1\circ \varphi: \intervalleentier{1}{n} \longmapsto \intervalleentier{1}{p}$ est une bijection. Si on applique le théorème~\ref{theo:ensemblefini} on en déduit donc que $n=p$.
\end{proof}
\begin{prop}
  Soit $E$ un ensemble fini et $F$ un ensemble tel qu'il existe une bijection $\varphi:E\longmapsto F$. alors $F$ est un ensemble fini de même cardinal que $E$.
\end{prop}
\begin{proof}
  $E$ est fini, donc il existe un entier $n$ et une bijection $\psi:\intervalleentier{1}{n} \longmapsto E$. Ainsi l'application $\varphi\circ\psi:\intervalleentier{1}{n} \longmapsto F$ est bijective, puisque c'est la composée de deux applications bijectives. Donc $F$ est un ensemble fini et $\Card F=n=\Card E$.
\end{proof}

Si $E$ est un ensemble fini de cardinal $n\neq 0$, l'existence d'une bijection $\varphi:\intervalleentier{1}{n} \longmapsto E$ permet de ``numéroter'' les éléments de $E$ et donc d'écrire l'ensemble $E$ en extension $E=\{\varphi(1), \ldots, \varphi(n)\}$.
Réciproquement, si on dispose d'une écriture en extension on peut en déduire que $E$ est un ensemble fini mais pas forcément que son cardinal est égal à $p$, car pour cela il faut que les éléments soient distincts.

La notion de cardinal correspond à la notion intuitive du nombre d'éléments d'un ensemble.

\subsection{Parties finies}

\subsubsection{Parties finies de $\N$}

\begin{theo}\label{theo:partfinN}
  Soit $\P$ une partie non vide et majorée de $\N$. Alors il existe un entier non nul $n$ et une bijection croissante $\varphi:\intervalleentier{1}{n} \longmapsto \P$. $\P$ est donc fini et de plus le couple $(n,\varphi)$ est unique.
\end{theo}
\begin{proof}
  La démonstration est hors programme, elle se fait par récurrence sur $M\in\N^*$ et la propriété est $\P(M)$ Pour toute partie non vide $\P$ de $\N$ et majorée par M, il existe un unique couple $(n,\varphi)$ avec $n\neq 0$ et $\varphi:\intervalleentier{1}{n} \longmapsto \P$ bijective et croissante. 
\end{proof}

\begin{enumerate}
\item La définition de la fonction $\varphi$ correspond à l'idée de ranger les éléments par ordre croissant;
\item L'application $\varphi:\intervalleentier{1}{n} \longmapsto \P$ $\P \subset \N$ est telle que $\varphi(1)\geq 0$ $\varphi(2)\geq 1$ \ldots;
\item On montre par récurrence que pour tout $k\in\intervalleentier{1}{n} \ 
\varphi(k)\geq k-1$, en particulier $\varphi(n)\leq n-1$. 

Or $M$ est un majorant de $\P$ donc $\varphi(n)\leq M$ donc $n-1\leq M$ soit alors $n\geq M+1$.
\item M est un majorant de $\P$ donc $\varphi(n)\geq M$ et $\varphi$ est strictement croissante donc $\varphi(n-1)<\varphi(n)$ donc $\varphi(n-1)\leq\varphi(n)-1$ donc $\varphi(n-1)\leq M-1$. 

On montre par récurrence descendante sur $k\in\intervalleentier{0}{n}$ la propriété $\P(k) \ \varphi(k)\leq M-(n-k)$

Initialisation à n : $\P(n)$ est vraie puisque $\varphi(n)\leq M$.

Hérédité : Supposons $\P(k)$ vraie pour tout entier $k\in\{2, \ldots,n\}$ et montrons alors que $\P(k-1)$ est vraie. Puisque $\varphi$ est strictement croissante, on a $\varphi(k-1)\leq \varphi(k)-1$ et par hypothèse de récurrence $\P(k)$ est vraie donc $\varphi(k)\leq M-(n-k)$ et $\varphi(k-1)\leq M-(n-k)-1=M-(n-k+1)=M-(n-(k-1))$ donc $\P(k-1)$ est vraie. 

On en conclue donc , grâce au théorème de récurrence, que la proposition est vraie. $\forall k \in \{1, \ldots, n\} \varphi(k)\leq M-(n-k)$. Si $n=M+1$ alors 
$k-1\leq \varphi(k)\leq M - (M+1-k)=k-1$ donc $\forall k \in \{1, \ldots,n\} \varphi(k)=k-1$ et ainsi $\P=\varphi(\intervalleentier{1}{n})=\intervalleentier{0}{n-1}= \intervalleentier{0}{M}$.
\end{enumerate}
\begin{prop}
  Soit $\P$ une partie non vide de $\N$. Alors $\P$ est finie si et seulement si$\P$ est majorée.
\end{prop}
\begin{proof}
  \begin{itemize}
  \item[$\impliedby$] Déjà fait grâce au théorème~\ref{theo:partfinN}.
  \item[$\implies$] On démontre par récurrence sur $p=\Card(\P)\in \N^*$ la propriété $\P(p)$ pour toute partie finie $\P$ de $\N$, si $\P$ est de cardinal $p$ alors $\P$ est majorée. 

Regardons l'initialisation à $p=1$, si $\P$ est de cardinal $1$, il existe un entier $a\in\N$ tel que $\P=\{a\}$ et $a$ est un majorant de $\P$ donc $\P$ est majorée.

Ensuite, passons à l'hérédité. Soit $p \in \N$ et supposons que $\P(p)$ soit vérifiée. Soit une partie $\P$ de $\N$ à $p+1$ éléments, alors soit $a\in \P$ et $\P'=\P\setminus\{a\}$. La partie $\P'$ est une partie de $\N$ à $p$ éléments, donc d'après l'hypothèse de récurrence elle et majorée. Notons $M$ le majorant de $\P'$ et notons $N=\max(M,a)$. Alors $N$ est un majorant de la partie $\P$. La partie $\P$ est donc majorée. $\P(p+1)$ est donc vérifiée.

Grâce au théorème de récurrence, on peut affirmer que la propriété $\P(p)$ est vraie sur $N^*$. Donc une partie finie de $\N$ est majorée.
  \end{itemize}
\end{proof}

\subsubsection{Parties d'un ensemble fini}

\begin{theo}\label{theo:partiesfinies}
  Soient $E$ et F deux ensembles tel que $E$ soit fini et $F\subset E$. Alors F est un ensemble fini et $\Card(F)\leq\Card(E)$. De plus $F=E \iff \Card(F)=\Card(E)$
\end{theo}
\begin{proof}
  \begin{enumerate}
  \item Si $F=\emptyset$, $F$ est fini et son cardinal est nul, $0\leq\Card(E)$. De plus $E=F=\emptyset \iff \Card(E)=0=\Card(F)$;
  \item sinon il existe un entier $n$ non nul tel qu'il existe une bijection $\varphi:\intervalleentier{1}{n} \longmapsto E$. Soit $A=\varphi^{-1}(F) \subset\intervalleentier{1}{n}$. A est une partie non vide et majorée de $\N$. A est donc une partie finie de $\N$. Il existe donc un entier naturel $p$ non nul et une bijection $\psi:\intervalleentier{1}{p} \longmapsto A$. On peut définir la restriction $\fonction{\widetilde{\varphi}}{A}{F}{x}{\varphi(x)}$. $\widetilde{\varphi}$ est bijective. En effet~:
    \begin{itemize}
    \item $\widetilde{\varphi}$ est injective, puisque si pour tout $x,y$ de A $\widetilde{\varphi}(x)=\widetilde{\varphi}(y)$ alors $\varphi(x)=\varphi(y)$ et comme $\varphi$ est injective (puisque bijective) on a $x=y$.
    \item $\widetilde{\varphi}$ est surjective, pour tout $y\in F$ il existe $x\in E$ tel $y\varphi(x)$. Mais $A=\varphi^{-1}(F)$ donc $x\in A$ et $y=\varphi(x)=\widetilde{\varphi}(x)$.
    \end{itemize}
    Soit $f=\widetilde{\varphi}\circ \psi$, alors $f$ est bijective comme la composée de deux applications bijectives. On a prouvé que F est un ensemble fini de cardinal p. Soit $A'=\{x-1, x\in A\}\subset \N$. $A'$ est une partie finie de $\N$ de même cardinal que $A$ et donc que $F$. Puisque $A\subset\intervalleentier{1}{n}$ alors $A'\subset\intervalleentier{0}{n-1}$. $M=n-1$ est un majorant de $A'$. Alors $p=\Card(A')\leq M+1=n-1+1=n$. De plus, 
    \begin{itemize}
    \item si $E=F$ alors $\Card(E)=\Card(F)$
    \item si $\Card(E)=\Card(F)$, alors $n=p=M+1$ on en déduit que $A'=\intervalleentier{0}{M} = \intervalleentier{0}{n-1}$ et donc que $A=\intervalleentier{1}{M+1} = \intervalleentier{1}{n}$. $F=\varphi(A)=\varphi(\intervalleentier{1}{n})=E$
    \end{itemize}
  \end{enumerate}
\end{proof}

Si $E$ est un ensemble infini, $E$ peut par exemple être en bijection avec une de ces parties strictes. Comme par exemple $\N$ et l'ensemble des entiers pairs sont en bijection.

\subsection{Opérations sur les ensembles finis}

\subsubsection{Réunion d'ensembles finis}

\begin{prop}\label{prop:reunionfindis}
  Soient $E$ et F deux ensembles finis \emph{disjoints}. Alors $E\cup F$ est fini et 
  \begin{equation}
    \Card(E\cup F)=\Card(E)+\Card(F).
  \end{equation}
\end{prop}

\begin{proof}
  Soit $n=\Card(E)\geq 1$ et $m=\Card(F)\geq 1$. Il existe deux bijections $\varphi$ et $\psi$ telles que $\varphi:\intervalleentier{1}{n} \longmapsto E$ et $\psi:\intervalleentier{1}{m} \longmapsto F$. Soit maintenant l'application
  \begin{equation}
    \fonction{f}{\intervalleentier{1}{m+n}}{E\cup F}{k}{\begin{cases} \varphi(k) & 1\leq k\leq n \\ \psi(k-n) & n+1\leq k\leq m+n\end{cases}}.
  \end{equation}
  Alors~:
    \begin{itemize}
    \item la fonction $f$ est bien définie puisque si $1\leq k \leq n$ alors $\varphi(k)$ existe et $\varphi(k)\in E\subset E\cup F$. si $n+1\leq k \leq n+m$ alors $1\leq k-n \leq m$ et donc $\psi(k-n)$ existe et $\psi(k)\in F\subset E\cup F$.
    \item la fonction $f$ est injective. En effet, soient deux entiers $k$ et $k'$ différents  dans $\intervalleentier{1}{m+n}$. Alors trois cas se proposent à nous~:
      \begin{itemize}
      \item si ces deux entiers sont dans $\intervalleentier{1}{n}$, alors puisque $\varphi$ est injective alors $f$ l'est aussi.
      \item si ces deux entiers sont dans $\intervalleentier{n}{n+m}$, alors puisque $\psi$ est injective alors $f$ l'est aussi.
      \item par contre si par exemple $k\in\intervalleentier{1}{n}$ et $k'\in\intervalleentier{n+1}{n+m}$ (on peut inverser les positions) alors $f(k)=\varphi(k')\in E$ et $f(k')=\psi(k'-n)\in F$ et $E\cap F = \emptyset$ donc $f(k)\neq f(k')$ donc $f$ est aussi injective.
      \end{itemize}
    \item La fonction $f$ est surjective. En effet, soit un élément $y$ de $E\cup F$. Si $y\in E$, alors il existe un entier $k$ dans $\intervalleentier{1}{n}$ tel que $y=\varphi(k)=f(k)$ puisque $\varphi$ est surjective. Si $y\in F$, alors il existe un entier $k$ dans $\intervalleentier{1}{n}$ tel que $y=\psi(k)=f(k+n)$ puisque $\psi$ est surjective. Donc $f$ est surjective.
    \end{itemize}
    Au final, on a démontré que l'application $f$ est bijective, donc $E\cup F$ est un ensemble fini avec $\Card(E\cup F)=n+m=\Card(E)+\Card(F)$.
  \end{proof}
  \begin{prop}
    Soient $E$ et $F$ deux ensembles finis, alors $E\cup F$ est un ensemble fini et
    \begin{equation}
    \Card(E\cup F)=\Card(E)+\Card(F)-\Card(E\cap F).
  \end{equation}
\end{prop}
\begin{proof}
  On sait que 
  \begin{equation}
    E=(E\cap F)\cup (E\cap \bar{F}) \quad \emptyset=(E\cap F)\cap (E\cap \bar{F}),
  \end{equation} 
  alors en appliquant la proposition précédente on a
  \begin{equation}
    \Card(E)=\Card(E\cap F)+\Card(E\cap \bar{F}).
  \end{equation}
  Montrons que $E\cup F =(E\cap \bar{F}) \cup F$. D'une part $E \cap \bar{F} \subset E$ donc $(E\cap \bar{F}) \cup F \subset E\cup F$. D'autre part, soit un élément $x$, alors si $x \in $E$ \cup F$ trois cas de figure se présentent~:
  \begin{itemize}
  \item si $x\in F$ alors $x \in (E\cap \bar{F}) \cup F$;
  \item si $x \in E$ et $x \in F$ alors $x \in (E\cap \bar{F}) \cup F$;
  \item si $x \in E$ et $x \notin F$ alors $x \in E\cap \bar{F}$, donc $x \in (E\cap \bar{F}) \cup F$.
  \end{itemize}
  Alors $E\cup F \subset (E \cap \bar{F})\cup F$, donc avec les deux inclusions $E\cup F =(E\cap \bar{F}) \cup F$. On a aussi $(E\cap \bar{F})\cap F=\emptyset$, donc en appliquant la proposition~\ref{prop:reunionfindis}, $E\cup F$ est un ensemble fini et
  \begin{align}
    \Card(E\cup F)&=\Card(E\cap \bar{F})+\Card(F)\\ &=\Card(E) - \Card(E\cap F)+\Card(F).
  \end{align}
\end{proof}
\begin{prop}
  Si $F$ est une partie d'un ensemble fini $E$, alors $F$ est fini, son complémentaire $\complement_E F$ est aussi fini et $\Card(\complement_E F)=\Card(E)-\Card(F)$.
\end{prop}
\begin{proof}
  En appliquant le théorème~\ref{theo:partiesfinies}, les ensembles $F$ et $\complement_E F$ sont finis et comme ils sont disjoints de réunion égale à $E$, la proposition~\ref{prop:reunionfindis}, nous permet d'écrire que
  \begin{equation}
    \Card(E)=\Card(F)+\Card(\complement_E F).
  \end{equation}
\end{proof}
\begin{prop}
  Soient $p$ un entier naturel et $(A_k)_{k\in\intervalleentier{1}{p}}$ $p$ ensembles deux à deux disjoints. Alors $\bigcup_{k=1}^{p}A_k$ est un ensemble fini et
  \begin{equation}
    \Card\left(\bigcup_{k=1}^{p}A_k\right)=\sum_{k=1}^{p}\Card(A_k).
  \end{equation}
\end{prop}
\begin{proof}
  On démontre par récurrence sur $p\in \N^*$ la propriété $\P(p)$ ``pour toute suite d'ensemble $(A_k)_{k\in\intervalleentier{1}{p}}$ $p$ ensembles deux à deux disjoints. Alors $\bigcup_{k=1}^{p}A_k$ est un ensemble fini et $\Card\left(\bigcup_{k=1}^{p}A_k\right)=\sum_{k=1}^{p}\Card(A_k)$''. L'étape initiale pour $p=1$ est évident. L'hérédité se démontre en supposant que pour un naturel $p$, $\P(p)$ vraie. On note $B_p=\bigcup_{k=1}^{p}A_k$ et donc $\bigcup_{k=1}^{p+1}A_k=A_{p+1}\cup B_p$. Puisque $\P(p)$ est vraie, alors $B_p$ est fini de cardinal $\Card(B_p)=\sum_{k=1}^{p}\Card(A_k)$. Puisque tous les ensembles $(A_k)_{k\in\intervalleentier{1}{p+1}}$ sont deux à deux disjoints on a $A_{p+1}\cap B_p=\emptyset$. D'après la proposition~\ref{prop:reunionfindis}, $A_{p+1}\cup B_p$ est fini et
\begin{equation}
  \Card(A_{p+1}\cup B_p) = \Card(A_{p+1})+\Card(B_p).
\end{equation}
Donc si on remplace $B_p$ par l'union on obtient
\begin{equation}
  \Card\left(\bigcup_{k=1}^{p+1}A_k\right) = \sum_{k=1}^{p+1}\Card(A_k).
\end{equation}
Donc $\P(p+1)$ est vraie. Alors le théorème de récurrence nous permet de conclure en affirmant que la propriété P est vraie pour tout naturel $p$ non nul.
\end{proof}

\subsubsection{Produit d'ensembles finis}
\begin{prop}\label{prop:produitfini}
  Soient $E$ et $F$ deux ensembles finis, alors $E\times F$ est un ensemble fini et
  \begin{equation}
    \Card(E\times F)=\Card(E)\times \Card(F).
  \end{equation}
\end{prop}
\begin{proof}
Si $E=\emptyset$ alors $E\times F=\emptyset$ est finie et $\Card(E\times F)=0=\Card(E)\times \Card(F)$.
  
Si $E\neq\emptyset$, alors il existe un entier naturel non nul $n$ et une application bijective $\varphi:\intervalleentier{1}{n}\longmapsto E$. Donc on peut noter $E=\{\varphi(1),\ldots ,\varphi(n)\}$ et $A_k=\{\varphi(k)\}\times F$. Alors $E\times F=\bigcup_{k=1}^n A_k$. Puisque $\varphi$ est bijective, les $A_k$ sont deux à deux disjoints. Soit pour tout $k\in\intervalleentier{1}{n}$, $\fonction{\psi_k}{F}{A_k}{x}{(\varphi(k),x)}$. Cette fonction est bien définie, elle est bijective. Donc $A_k$ est un ensemble fini et $\Card(A_k)=\Card(F)$. Puisque les $A_k$ sont finis et deux à deux disjoints, le produit $E\times F$ est fini et d'après la proposition~\ref{prop:reunionfindis} on a
    \begin{equation}
      \Card(E\times F)=\sum_{k=1}^{n} A_k=\sum_{k=1}^{n}\Card(F)=n\Card(F)=\Card(E)\times \Card(F).
    \end{equation}
\end{proof}
\begin{prop}
  Soit $E$ un ensemble fini, alors pour tout entier naturel $p$ non nul $E^p$ est un ensemble fini et
  \begin{equation}
    \Card(E^p)=\Card(E)^p.
  \end{equation}
\end{prop}
\begin{proof}
  La démonstration se fait par récurrence sur $p\in\N^*$, $\P(p)$ ``$E^p$ est fini et $\Card(E^p)=\Card(E)^p$''. L'étape d'initialisation, pour $p=1$ est triviale. L'hérédité se démontre en considérant que pour tout naturel $p\geq 1$, $\P(p)$ vraie pour montrer $\P(p+1)$. En effet puisque $E^{p+1}=E^p \times E$ d'après l'hypothèse de récurrence $E^p$ est fini et $\Card(E^p)=\Card(E)^p$. La proposition~\ref{prop:produitfini} affirme que $E^{p+1}$ est fini et que
  \begin{equation}
    \Card(E^{p+1})=\Card(E^p)\times \Card(E)=\Card(E)^{p+1}.
  \end{equation}
  Donc $\P(p+1)$ est vraie. Ainsi par théorème de récurrence la proposition $\P(p)$ est vraie pour tout entier $p$ non nul.
\end{proof}

\subsection{Applications d'un ensemble fini vers un ensemble fini}
Soient $E$ et $F$ deux ensembles finis. On s'intéresse à l'ensemble des applications de $E$ dans $F$ noté $F^E$.
\begin{theo}\label{theo:bijinjsurj}
  Soient $E$ et F deux ensembles finis non vides \emph{de même cardinal} et $f\in F^E$, alors
  \begin{equation}
    f \text{~est bijective}\iff f \text{~est injective}\iff f \text{~est surjective}.
  \end{equation}
\end{theo}
\begin{proof}
  On sait déjà que la bijectivité entraîne l'injectivité et la surjectivité.
  \begin{itemize}
  \item Supposons que $f$ est injective. Soit $A=f(E)$, l'application $f^{|A}$ est une application bijective, donc $\Card(A)=\Card(E)$ et puisque $E$ et F ont le même cardinal, $\Card(A)=\Card(F)$ mais on sait aussi que $A\subset F$ donc au final $A=f(E)=F$. Cela signifie que $f$ est surjective. Elle est donc bijective ($f^{|A}=f$).
  \item Supposons que $f$ est surjective. Pour tout $y\in F$ on peut choisir un élément $x\in E$ tel que $y=f(x)$. Définissons l'ensemble $A$ constitué par ces éléments $x$ de E. L'application $f_{|A}:A\longmapsto E$ est donc bijective. Alors $\Card(A)=\Card(F)$ et puisque $E$ et F ont le même cardinal, $\Card(A)=\Card(E)$, or $A\subset E$ donc $A=E$ et ainsi $f$ est bijective ($f_{|A}=f$).
  \end{itemize}
\end{proof}
\begin{itemize}
\item Si $E$ et F sont finis de même cardinal, il existe une bijection $\varphi:E\longmapsto F$. En effet il existe un entier naturel non nul $n$ et une fonction $f_1:\intervalleentier{1}{n} \longmapsto E$ bijective (puisque $E$ est fini) et une fonction $f_2:\intervalleentier{1}{n} \longmapsto F$ bijective (puisque $F$ est fini), donc il faut et il suffit de prendre $\varphi=f_2\circ f_1^{-1}$.
\item Si $E$ et $F$ sont finis et une application $f:E\longmapsto F$, alors
  \begin{gather}
    f \text{~est injective} \implies \Card(E)\leq\Card(F); \\
    f \text{~est surjective}\implies \Card(E)\geq\Card(F);\\
    f \text{~est bijective} \implies \Card(E)=\Card(F). 
  \end{gather}
\end{itemize}
\begin{theo}
  Soient $E$ et $F$ des ensembles finis non vides, alors $F^E$ est un ensemble fini tel que
  \begin{equation}
    \Card\left(F^E\right)=\Card(F)^{\Card(E)}.
  \end{equation}
\end{theo}
\begin{proof}
  Puisque $E$ est fini, il existe un entier naturel $p$ non nul et une bijection $\varphi:\intervalleentier{1}{p} \longmapsto E$, c'est à dire qu'on peut écrire $E=\{\varphi(k)\}_{k\in\intervalleentier{1}{p}}$. Soit l'application $f$ telle que
  \begin{equation}
    \fonction{f}{F^E}{F^p}{u}{(u(\varphi(1)), \ldots, u(\varphi(p)))}.
  \end{equation}
  Alors $f$ est bien définie : puisque pour toute fonction $u$ de $F^E$ et tout entier $i$ de $\intervalleentier{1}{p}$, $\varphi(i) \in E$ et $u(\varphi(i))\in F$. La fonction $f$ est bijective, puisque pour tout $(b_1, \ldots, b_p) \in F^P$ il existe une unique application $u:E\longmapsto F$ telle que $u(\varphi(1))=b_1$, \ldots, $u(\varphi(p))=b_p$. Alors $F^E$ est fini et $\Card\left(F^E\right)=\Card(F^p)=\Card(F)^p=\Card(F)^{\Card(E)}$.
\end{proof}
\begin{prop}\label{prop:nbinj}
  Soient $E$ et F des ensembles finis non vide de cardinaux respectifs $p$ et $n$. Soit $I(E,F)$ l'ensemble des injections de $E$ dans F. Alors $I(E,F)$ est fini et
  \begin{equation}
    \Card(I(E,F))=\prod_{k=0}^{p-1}(n-k)=n(n-1) \dotsm (n-p+1).
  \end{equation}
\end{prop}
\begin{proof}
  Notons $A_n^p$ le nombre d'injections d'un ensemble à $p$ éléments sur un ensemble à $n$ éléments. On fait une récurrence sur le cardinal de l'ensemble de départ. Soit $p$ un entier naturel non nul, alors $\P(p)$ ``$\forall n \in \N^* \ A_n^p=\prod_{k=0}^{p-1}(n-k)$''. 
  \begin{itemize}
  \item[I] Si $\Card(E)=1$ alors $E=\{a\}$ donc $I(E,F)=F^E$ puisque toutes les applications de $E$ dans F sont injectives. Donc pour tout naturel non nul $n$ $A_n^1=\Card(F)^{\Card(E)}=n^1=n=\prod_{k=0}^{1-1}(n-k)$. Donc $\P(1)$ est vérifiée.
  \item[H] Supposons que $\P(p)$ soit vraie pour tout $p$ non nul, alors soit $E$ un ensemble fini de cardinal $p+1$. Soit un entier naturel non nul $n$ et F un ensemble de cardinal $n$. Soit un élément $a\in E$ et $E'=E\setminus\{a\}$. Se donner une injection $i$ de $E$ dans F revient à se donner
    \begin{itemize}
    \item $i(a)$ un élément de $F$ ($n=\Card(F)$ choix possibles),
    \item une application injective de $E'$ sur $F\setminus\{i(a)\}$ ($A_{n-1}^p$ choix possibles d'après l'hypothèse de récurrence).
    \end{itemize}
    Donc $A_{n}^{p+1}=n \times A_{n-1}^p$. Alors 
    \begin{itemize}
    \item si $n\geq 2$, alors  $A_{n-1}^p=\prod_{k=0}^{p-1}(n-1-k)$ et on a
      \begin{equation}
        A_{n}^{p+1}=n\prod_{k=0}^{p-1}(n-1-k)=\prod_{k=0}^{p}(n-k);
      \end{equation}
    \item si $n=1$, alors $A_{1}^{p+1}=0$ et $\prod_{k=0}^{p}(1-k)=0$ car $p\geq 1$.
    \end{itemize}
    On a montré que pour tout entier naturel $n$ non nul, $A_{n}^{p+1}=\prod_{k=0}^{p}(n-k)$. Donc $\P(p+1)$ est vraie.
  \item[C] Par théorème de récurrence, $\P(p)$ est vraie pour tout entier naturel $p$ non nul.
  \end{itemize}
\end{proof}
\begin{defdef}
  Une permutation d'un ensemble $E$ est une bijection de $E$ dans E. On note $\sigma(E)$ leur ensemble.
\end{defdef}
\begin{prop}
  Si $E$ est un ensemble fini, $\sigma(E)$ est un ensemble fini et
  \begin{equation}
    \Card(\sigma(E))=\prod_{k=0}^{n-1}(n-k)=n!.
  \end{equation}
\end{prop}
\begin{proof}
  En appliquant le théorème~\ref{theo:bijinjsurj} puisque $E$ est fini et en notant que $\sigma(E)=I(E,E)$ puis en appliquant la proposition~\ref{prop:nbinj} et on obtient
  \begin{equation}
    \Card(\sigma(E))=\prod_{k=0}^{n-1}(n-k)=n!.
  \end{equation}
\end{proof}

\subsection{Parties à $p$ éléments d'un ensemble fini}

Soit $E$ un ensemble fini. On note pour tout entier naturel $p$, $\Part_p(E)$ l'ensemble des parties de $E$ à $p$ éléments.
\begin{prop}\label{prop:ppartiesfinies}
  Soit $E$ un ensemble fini de cardinal $n\in \N$ et $p \in \N$ alors $\Part_p(E)$ est un ensemble fini et
  \begin{equation}
    \Card(\Part_p(E))=\binom{n}{p}
  \end{equation}
\end{prop}
\begin{proof}
  \begin{itemize}
  \item si $p=0$ alors $\Part_p(E)=\{\emptyset\}$ et $\Card(\Part_p(E))=1=\binom{n}{0}$;
  \item si $p>n$ alors $\Part_p(E)=\emptyset$ et $\Card(\Part_p(E))=0=\binom{n}{0}$;
  \item si $1\leq p \leq n$ alors le nombre d'injection de $\intervalleentier{1}{p}$ dans $E$ est $A_{n}^p=\frac{n!}{(n-p)!}$. Se donner une injection de  $\intervalleentier{1}{p}$ dans $E$ revient à
    \begin{itemize}
    \item choisir $p$ éléments distincts dans E, c'est à dire une partie à $p$ éléments de $E$ ($\Card(\Part_p(E))$ choix possibles);
    \item ordonner ces éléments, c'est à dire définir une bijection de $\intervalleentier{1}{p}$ dans $E$ ($p!$ choix possibles).
    \end{itemize}
    Ainsi,
    \begin{equation}
      \frac{n!}{(n-p)!}=\Card(\Part_p(E)) p!.
    \end{equation}
    Du coup
    \begin{equation}
      \Card(\Part_p(E))=\binom{n}{p}.
    \end{equation}
  \end{itemize}
\end{proof}
\begin{defdef}
  Soit $E$ un ensemble fini. On appelle combinaison de $E$ à $p$ éléments tout élément A de $\Part_p(E)$.
\end{defdef}
\begin{prop}
  Soit $E$ un ensemble fini, alors l'ensemble des parties de $E$ $\Part(E)$ est un ensemble fini et
  \begin{equation}
    \Card(\Part(E))=2^{\Card(E)}.
  \end{equation}
\end{prop}
\begin{proof}
  Soit $n=\Card(E)\in \N$. Alors $\Part(E)=\bigcup_{k=0}^{n}\Part_k(E)$, les $\Part_k(E)$ sont tous deux à deux disjoints, donc d'après le théorème~\ref{theo:partiesfinies} et la proposition~\ref{prop:ppartiesfinies} on a
  \begin{equation}
    \Card(\Part(E))=\sum_{k=0}^n \Card(\Part_k(E))=\sum_{k=0}^n \binom{n}{k}=2^n.
  \end{equation}
\end{proof}
