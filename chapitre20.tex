\chapter{Intégration sur un segment des fonctions à valeur réelles}\label{chap:integration}
\minitoc%
\minilof%
\minilot%
%
\section{Fonctions en escalier, continues par morceaux}

Soient deux réels \(a\) et \(b\) tels que \(a<b\) et deux naturels \(n\) et 
\(m\) tous les deux non nuls. On travaille dans cette section sur le segment 
\(\intervalleff{a}{b}\).

\subsection{Subdivision d'un segment}

\begin{defdef}
  On appelle subdivision du segment \(\intervalleff{a}{b}\) toute famille finie 
  \((a_i)_{i \in \intervalleentier{0}{n}}\) de \(n+1\) points de 
  \(\intervalleff{a}{b}\) telle que~:
  \begin{equation}
    \begin{cases}
      a_0=a \\
      a_n=b \\
      \forall i \in \intervalleentier{0}{n} \quad a_i<a_{i+1}
    \end{cases}.
  \end{equation}
  Le réels \(a_i\) sont appelés les points de la subdivision. Les intervalles 
  ouverts \(\intervalleoo{a_i}{a_{i+1}}\) sont appelés les intervalles ouverts 
  de la subdivision.
\end{defdef}

\begin{defdef}
  Soit \(\sigma=(a_i)_{i \in \intervalleentier{0}{n}}\) une subdivision du 
  segment \(\intervalleff{a}{b}\). on définit un réel appelé le pas de la 
  subdivision, noté \(\delta(\sigma)\) par~:
  \begin{equation}
    \delta(\sigma)=\max_{0 \leqslant i \leqslant n-1}(a_{i+1}-a_i)
  \end{equation}
\end{defdef}

\emph{Exemple fondamental}~: Soit une subdivison \(\sigma=(a_i)_{i \in 
\intervalleentier{0}{n}}\) de \(\intervalleff{a}{b}\) à \(n+1\) points telle 
que~:
\begin{equation}
  \forall k \in \intervalleentier{0}{n} \quad a_k=a+k \frac{b-a}{n}.
\end{equation}
\(\sigma\) est une subdivion régulière de \(\intervalleff{a}{b}\). Tous les 
intervalles de la subdivision ont la même longueur. Le pas de cette subdivision 
vaut \(\delta(\sigma)=\frac{b-a}{n}\).

\begin{defdef}
  Soient \(\sigma=(a_i)_{i \in \intervalleentier{0}{n}}\) et \(\sigma'=(b_i)_{i 
  \in \intervalleentier{0}{m}}\) deux subdivisions de \(\intervalleff{a}{b}\). 
  On dit que \(\sigma'\) est plus fine que \(\sigma\) si et seulement si
  \begin{equation}
    \intervalleentier{a_0}{a_n} \subset \intervalleentier{b_0}{b_m}.
  \end{equation}
\end{defdef}

On peut montrer que la relation ``être plus fine que'' est une relation d'ordre 
sur l'ensemble des subdivisions de \(\intervalleff{a}{b}\). Ce n'est pas un 
ordre total.

La subdivision \((a,b)\) est la moins fine de toutes les subdivisions. Mais il 
n'y a pas pas de subdivision plus fine que toutes les autres.

Si \(\sigma'\) est plus fine que \(\sigma\), alors le pas \(\delta(\sigma')\) 
est inférieur ou égal au pas \(\delta(\sigma)\). La réciproque est fausse.

Supposons qu'on ait \(\sigma=(0,1/2,1)\) et \(\sigma'=(0,1/3,2/3,1)\) alors 
\(\delta(\sigma')<\delta(\sigma)\) mais comme \(1/2 \notin \sigma'\) on ne peut 
pas dire que \(\sigma'\) est plus fine que \(\sigma\).

\subsection{Fonctions en escalier}

\subsubsection{Définitions}

\begin{defdef}
  Soit \(f \in \R^{\intervalleff{a}{b}}\). On dit que \(f\) est en escalier sur 
  \(\intervalleff{a}{b}\) s'il existe une subdivision \(\sigma=(a_i)_{i \in 
  \intervalleentier{0}{n}}\) telle que \(f\) soit constante sur tous les 
  intervalles ouverts de la subdivision. On note \(\E(\intervalleff{a}{b})\) 
  l'ensemble des fonction en escalier sur \(\intervalleff{a}{b}\).

  Il est à noter que la fonction \(f\) est définie en tout point \(a_i\) de la 
  subdivision, mais qu'elle n'est pas forcèment continue en ces points.
\end{defdef}

\begin{defdef}
  Soit \(f \in \E(\intervalleff{a}{b})\). Toute subdivision du segment 
  \(\sigma=(a_i)_{i \in \intervalleentier{0}{n}}\), telle que \(f\) soit 
  constante sur tous les intervalles ouverts de \(\sigma\), est appelée une 
  subdivision adaptée à la fonction \(f\).
\end{defdef}

\emph{Remarque}~:
\begin{enumerate}
  \item Si \(f \in \E(\intervalleff{a}{b})\) et si \(\sigma\) est une 
    subdivision adaptée à \(f\), alors toute subdivision \(\sigma'\) de 
    \(\intervalleff{a}{b}\) plus fine que \(\sigma\) est encore adaptée à \(f\);
  \item les fonctions constantes sur \(\intervalleff{a}{b}\) sont en escalier et 
    toutes les subdivisions de \(\intervalleff{a}{b}\) leurs sont adaptées.
\end{enumerate}

\subsubsection{Premières propriétés}

On peut avoir besoin de construire une subdivision adaptée à la fois à deux 
fonctions \(f\) et \(g\).

Soient \(\sigma=(a_i)_{i \in \intervalleentier{0}{n}}\) et \(\sigma'=(b_i)_{i 
\in \intervalleentier{0}{m}}\) deux subdivisions de \(\intervalleff{a}{b}\). On 
note \(A(\sigma)\) (resp. \(A(\sigma')\)) l'ensemble des points de \(\sigma\) 
(resp. \(\sigma'\)). Il existe une unique subdivision \(s\) de 
\(\intervalleff{a}{b}\) telle que \(A(s)=A(\sigma) \cup A(\sigma')\). La 
subdivision \(s\) est appelée la réunion de \(\sigma\) et \(\sigma'\). Elle est 
plus fine que \(\sigma\) et \(\sigma'\). Si \((f,g) \in 
\E(\intervalleff{a}{b})^2\) sont telles que~:
\begin{itemize}
  \item \(\sigma\) est adpatée à \(f\);
  \item \(\sigma'\) est adpatée à \(g\);
\end{itemize}
alors \(s\) est adaptée à \(f\) et à \(g\).

\begin{prop}
  L'ensemble \(\E(\intervalleff{a}{b})\) est un sous espace vectoriel du 
  \(\R\)-espace vectoriel \(\R^{\intervalleff{a}{b}}\).
\end{prop}
\begin{proof}
  Par définition, \(\E(\intervalleff{a}{b}) \subset \R^{\intervalleff{a}{b}}\). 
  Ensuite il est non vide puisqu'il contient les fonctions constantes. Soient 
  \(f\) et \(g\) dans \(\E(\intervalleff{a}{b})\) et un réel \(\lambda\). Soient 
  \(\sigma\) une subdivision de \(\intervalleff{a}{b}\) adaptée à \(f\) et 
  \(\sigma'\) une subdivision de \(\intervalleff{a}{b}\) adaptée à \(g\). Notons 
  \(s\) la réunion de ces deux subdivisions. On pose \(s=(a_i)_{i \in 
  \intervalleentier{0}{n}}\), elle est adaptée à \(f\) et à \(g\) donc \(f\) et 
  \(g\) sont constantes sur tous les intervalles 
  \(\intervalleoo{a_i}{a_{i+1}}\). Par conséquent, \(\lambda f +g\) est 
  constante sur tous les intervalles \(\intervalleoo{a_i}{a_{i+1}}\). Donc 
  \(\lambda f +g\) est une fonction en escalier et \(s\) est une subdivision 
  adaptée à \(\lambda f+g\).
\end{proof}

\begin{prop}
  Si \(f \in \E(\intervalleff{a}{b})\) alors \(\abs{f} \in 
  \E(\intervalleff{a}{b})\).
\end{prop}
\begin{proof}
  Soit \(\sigma=(a_i)_{i \in \intervalleentier{0}{n}}\) une subdivision adpaptée 
  à \(f\). Alors pour tout \(i \in \intervalleentier{0}{n}\) \(f\) est constante 
  sur \(\intervalleoo{a_i}{a_{i+1}}\). Donc \(\abs{f}\) est constante sur sur 
  \(\intervalleoo{a_i}{a_{i+1}}\). Alors \(\abs{f}\) est une fonction en 
  escalier.
\end{proof}

\subsection{Fonctions continues par morceaux}

\subsubsection{Définitions}

\begin{defdef}
  Soit \(f \in \R^{\intervalleff{a}{b}}\), on dit que \(f\) est continue par 
  morceaux sur \(\intervalleff{a}{b}\) s'il existe une subdivision 
  \(\sigma=(a_i)_{i \in \intervalleentier{0}{n}}\) telle que~:
  \begin{itemize}
    \item \(\forall i \in \intervalleentier{0}{n-1}\) \(f\) est continue sur 
      l'intervalle ouvert \(\intervalleoo{a_i}{a_{i+1}}\)
    \item \(\forall i \in \intervalleentier{0}{n-1}\) la restriction de \(f\) à 
      l'intervalle ouvert \(\intervalleoo{a_i}{a_{i+1}}\) admet une limite finie 
      à gauche en \(a_{i+1}\) et à droite en \(a_i\).
  \end{itemize}
  Une telle subdivision sera dite adaptée à \(f\).
\end{defdef}

Il est à noter que \(f\) est définie en tous les \(a_i\) mais n'est pas a priori 
continue en les \(a_i\). On note \(\CM(\intervalleff{a}{b})\) l'ensemble des 
fonctions continues par morceaux sur \(\intervalleff{a}{b}\). 

Les fonctions continues sur \(\intervalleff{a}{b}\) sont continues par morceaux 
sur \(\intervalleff{a}{b}\) et toute subdivision du segment 
\(\intervalleff{a}{b}\) leur est adaptée. Toute subdivision plus fine est encore 
adaptée.

\subsubsection{Premières propriétés}

\begin{prop}
  Les fonctions en escalier sur \(\intervalleff{a}{b}\) sont continues par 
  morceaux sur \(\intervalleff{a}{b}\). C'est-à-dire \(\E(\intervalleff{a}{b}) 
  \subset \CM(\intervalleff{a}{b})\).
\end{prop}
\begin{proof}
  C'est clair. Soit \(f\) une fonction en escalier. Alors il existe une 
  subdivison \((a_i)_{i \in \intervalleentier{0}{n}}\) telle que \(f\) soit 
  constante donc continue sur tous les intervalles ouverts 
  \(\intervalleoo{a_i}{a_{i+1}}\).
\end{proof}

\begin{prop}
  Les fonctions continues par morceaux sur \(\intervalleff{a}{b}\) sont bornées 
  sur \(\intervalleff{a}{b}\). C'est-à-dire \(\CM(\intervalleff{a}{b}) \subset 
  \Born(\intervalleff{a}{b})\).
\end{prop}
\begin{proof}
  Soit une fonction \(f \in \CM(\intervalleff{a}{b})\) et \(\sigma=(a_i)_{0 
  \leqslant i \leqslant n}\) une subdivision de \(\intervalleff{a}{b}\) adaptée 
  à \(f\). Alors pour tout \(i \in \intervalleentier{1}{n}\), 
  \(f_{|\intervalleoo{a_i}{a_{i+1}}}\) est continue et admet des limites finies 
  à droite en \(a_{i}\) et à gauche en \(a_{i+1}\). On définit le prolongement 
  par continuité \(g_i\) de \(f_{|\intervalleoo{a_i}{a_{i+1}}}\) à l'intervalle 
  fermé \(\intervalleff{a_i}{a_{i+1}}\). Alors \(g_i\) est continue sur le 
  segment \(\intervalleff{a_i}{a_{i+1}}\) donc bornée (d'après le théorème des 
  bornes, Théorème~\ref{theo:bornes}). Il existe alors \(M_i \in \R_+\) tel que pour tout \(x \in 
  \intervalleff{a_i}{a_{i+1}} \) on ait \(\abs{g_i(x)} \leqslant M_i\). D'où
  \begin{equation}
    \forall x \in \intervalleoo{a_i}{a_{i+1}} \quad \abs{f(x)}=\abs{g_i(x)} 
    \leqslant M_i.
  \end{equation}
  Soit \(M=\max\limits_{0 \leqslant i \leqslant n-1} M_i\) et \(N=\max\limits_{0 
  \leqslant i \leqslant n-1} \abs{f(a_i)}\). Alors \(\max(M,N)\) est un majorant 
  de \(\abs{f}\).

  La fonction \(f\) est alors bornée.
\end{proof}

\begin{prop}
  %  L'ensemble des fonctions continues par morceaux 
  \(\CM(\intervalleff{a}{b})\) est un sous espace du \(\R\)-espace vectoriel 
  \(\R^{\intervalleff{a}{b}}\).
\end{prop}
\begin{proof}
  Par définition, \(\CM(\intervalleff{a}{b})\subset \R^{\intervalleff{a}{b}}\). 
  Ensuite \(\CM(\intervalleff{a}{b})\) est non vide puisqu'il contient 
  \(\courbe(\intervalleff{a}{b})\). Soient \(f\) et \(g\) deux fonctions 
  continues par morceaux et un réel \(\lambda\). Soient \(\sigma\) une 
  subdivision de \(\intervalleff{a}{b}\) adaptée à \(f\) et \(\sigma'\) une 
  subdivision de \(\intervalleff{a}{b}\) adaptée à \(g\). Soit \(s\) la réunion 
  de \(\sigma\) et de \(\sigma'\). Elle est plus fine que \(\sigma\) et que 
  \(\sigma'\) donc elle est adaptée à \(f\) et à \(g\). On note \(s=(a_i)_{0 
  \leqslant i \leqslant n}\). Pour tout \(i \in \intervalleentier{0}{n-1}\) 
  \(f\) et \(g\) sont continues sur \(\intervalleoo{a_i}{a_{i+1}}\) donc 
  \(\lambda f+g\) l'est aussi. De plus pour tout \(i \in 
  \intervalleentier{0}{n-1}\) \(f_{|\intervalleoo{a_i}{a_{i+1}}}\) et 
  \(g_{|\intervalleoo{a_i}{a_{i+1}}}\) admettent des limites finies à droite en 
  \(a_i\) et à gauche en \(a_{i+1}\) alors \((\lambda f+ 
  g)_{|\intervalleoo{a_i}{a_{i+1}}}\) admet des limites finies à droite en 
  \(a_i\) et à gauche en \(a_{i+1}\).

  La fonction \(\lambda f +g\) est donc continue par morceaux et \(s\) est une 
  subdivision adpatée. Par caractérisation, \(\CM(\intervalleff{a}{b})\) est un 
  sous espace vectoriel de \(\R^{\intervalleff{a}{b}}\).
\end{proof}

\begin{prop}
  \begin{equation}
    \forall f \in \R^{\intervalleff{a}{b}} \quad f \in \CM(\intervalleff{a}{b}) 
    \implies \abs{f} \in \CM(\intervalleff{a}{b})
  \end{equation}
\end{prop}

\subsection{Approximation d'une fonction continue par morceaux par une fonction 
en escalier}

\begin{theo}
  Soit \(f \in \CM(\intervalleff{a}{b})\), alors
  \begin{equation}
    \forall \epsilon \geqslant 0 \ \exists (\varphi, \psi) \in 
    \E(\intervalleff{a}{b})^2 \quad \begin{cases} \varphi \leqslant f \leqslant 
    \psi \\ \psi-\varphi \leqslant 
    \fonction{\tilde{\epsilon}}{\intervalleff{a}{b}}{\R}{x}{\epsilon} 
    \end{cases}.
  \end{equation}
\end{theo}
\begin{proof}[Démonstration~: Première étape]
  On démontre le théorème dans le cas où \(f\) est continue. D'après le théorème 
  de Heine, \(f\) est uniformément continue sur \(\intervalleff{a}{b}\). 
  C'est-à-dire
  \begin{equation}
    \forall \epsilon >0 \ \exists \eta>0 \ \forall (x,x') \in 
    \intervalleff{a}{b}^2 \quad \abs{x-x'} \leqslant \eta \implies 
    \abs{f(x)-f(x')} \leqslant \frac{\epsilon}{2}.
  \end{equation}
  Soit un naturel \(n\) non nul tel que \(\frac{b-a}{n} \leqslant \eta\) et pour 
  tout \(k \in \intervalleentier{0}{n}\) on pose \(a_k=a_0+k \frac{b-a}{n}\). La 
  subdivision \(\sigma=(a_k)_{0 \leqslant k \leqslant n}\) est régulière de 
  \(\intervalleff{a}{b}\) à \(n+1\) points. On définit \(g\) en escalier sur 
  \(\intervalleff{a}{b}\) par~:
  \begin{equation}
    \forall k \in \intervalleentier{0}{n} \ \forall x \in 
    \intervallefo{a_k}{a_{k+1}} \quad g(x)=f(a_k),
  \end{equation}
  et \(g(b)=f(a_n)=f(b)\). La fonction \(g\) est bien en escalier. De plus pour 
  tout \(x \in \intervallefo{a}{b}\) il existe un unique \(k \in 
  \intervalleentier{0}{n-1}\) tel que \(x \in \intervallefo{a_k}{a_{k+1}}\).  
  Alors
  \begin{equation}
    \abs{g(x)-f(x)} = \abs{f(a_k)-f(x)},
  \end{equation}
  or \(\abs{x-a_k} \leqslant \abs{a_{k+1}-a_k} \leqslant \frac{b-a}{n}\leqslant 
  \eta\). D'où
  \begin{equation}
    \abs{g(x)-f(x)} = \abs{f(a_k)-f(x)} \leqslant \frac{\epsilon}{2},
  \end{equation}
  et \(\abs{g(b)-f(b)}=0 \leqslant \frac{\epsilon}{2}\). Alors
  \begin{equation}
    \forall x \in \intervalleff{a}{b} \quad \abs{g(x)-f(x)} \leqslant 
    \frac{\epsilon}{2}
  \end{equation}
  Soient les fonctions \(\psi=g + \frac{\tilde{\epsilon}}{2}\) et \(\varphi=g - 
  \frac{\tilde{\epsilon}}{2}\). Comme \(g\) est en escalier et les fonctions 
  constantes aussi alors \(\psi\) et \(\varphi\) sont en escalier 
  (\(\E(\intervalleff{a}{b})\) est un espace vectoriel). On a bien
  \begin{equation}
    \begin{cases}
      \varphi \leqslant f \leqslant \psi \\
      \psi-\varphi=\tilde{\epsilon}
    \end{cases}.
  \end{equation}

  On a démontré le théorème dans le cas où \(f\) est continue.
\end{proof}
\begin{proof}[Démonstration~: Deuxième étape]
  On revient au cas général. On suppose que \(f\) est continue par morceaux sur 
  \(\intervalleff{a}{b}\). Soit \(\sigma=(a_i)_{0 \leqslant i \leqslant m}\) une 
  subdivision de \(\intervalleff{a}{b}\) adaptée à \(f\). Soit \(i \in 
  \intervalleentier{0}{m-1}\), alors~:
  \begin{itemize}
    \item \(f_{|\intervalleoo{\alpha_i}{\alpha_{i+1}}}\) est continue
    \item \(f\) admet une limite finie à droite en chacun des \(a_i\)
    \item \(f\) admet une limite finie à gauche en chacun des \(a_{i+1}\)
  \end{itemize}
  On peut alors définir le prolongement par continuité de 
  \(f_{|\intervalleoo{a_i}{a_{i+1}}}\) à \(\intervalleff{a_i}{a_{i+1}}\) noté 
  \(f_i\).

  La fonction \(f_i\) est continue sur \(\intervalleff{a_i}{a_{i+1}}\) donc il 
  existe une fonction en escalier \(g_i\) sur \(\intervalleff{a_i}{a_{i+1}}\) 
  telle que~:
  \begin{equation}
    \forall x \in \intervalleff{a_i}{a_{i+1}} \quad \abs{f_i(x)-g_i(x)} < 
    \frac{\epsilon}{2}
  \end{equation}
  On définit la fonction \(g\) par~: \(g(a_i)=f(a_i)\) et
  \begin{equation}
    \forall x \in \intervalleoo{a_i}{a_{i+1}} \quad g(x)=g_i(x).
  \end{equation}
  Alors \(g\) est en escalier. De plus
  \begin{equation}
    \forall x \in \intervalleff{a}{b} \quad \abs{g(x)-f(x)} \leqslant 
    \frac{\epsilon}{2}
  \end{equation}
  Soient les fonctions \(\psi=g + \frac{\tilde{\epsilon}}{2}\) et \(\varphi=g - 
  \frac{\tilde{\epsilon}}{2}\). Comme \(g\) est en escalier et les fonctions 
  constantes aussi alors \(\psi\) et \(\varphi\) sont en escalier 
  (\(\E(\intervalleff{a}{b})\) est un espace vectoriel). On a bien
  \begin{equation}
    \begin{cases}
      \varphi \leqslant f \leqslant \psi \\
      \psi-\varphi=\tilde{\epsilon}
    \end{cases}.
  \end{equation}
\end{proof}

\section{Intégrale d'une fonction en escalier}

\subsection{Définition de l'intégrale d'une fonction en escalier}

\begin{theo}
  Soit \(f \in \E(\intervalleff{a}{b})\) et soit \(\sigma=(a_i)_{0 \leqslant i 
  \leqslant n}\) une subdivision de \(\intervalleff{a}{b}\) adaptée à \(f\). 
  Soit \((\lambda_i)_{0 \leqslant i \leqslant n}\) une famille de fonctions 
  constantes de \(f\) sur les intervalles ouverts 
  \(\intervalleoo{a_{i-1}}{a_i}\). Alors le réel
  \begin{equation}
    I(f,\sigma) = \sum_{i=1}^n (a_i-a_{i-1})\lambda_i
  \end{equation}
  est indépendant du choix de la subdivision \(\sigma\) adaptée à \(f\).
\end{theo}

\begin{defdef}
  Soit \(f \in \E(\intervalleff{a}{b})\). On définit un réel appelé intégrale de 
  \(f\) sur \(\intervalleff{a}{b}\) et noté \(\int_{\intervalleff{a}{b}}f\) ou 
  \(\int_a^b f\) ou encore \(\int_a^b f(x) \diff x\) par
  \begin{equation}
    \int_a^b f = I(f,\sigma)
  \end{equation}
  pour toute subdivision \(\sigma\) adaptée à \(f\).
\end{defdef}

\begin{proof}
  Soient \(\sigma\) une subdivision de \(\intervalleff{a}{b}\) adaptée à \(f\) 
  et \(c \in \intervalleff{a}{b}\). Soit \(\sigma' = \sigma \cup \{c\}\). Il 
  existe \(i_0 \in \intervalleentier{0}{n}\) tel que \(c \in 
  \intervalleff{a_{i_0}}{a_{i_0+1}}\).
  \begin{itemize}
    \item Si \(c=a_{i_0}\) ou \(c=a_{i_0+1}\) alors \(\sigma=\sigma'\) et donc 
      \(I(f,\sigma')=I(f,\sigma)\);
    \item sinon alors \(c \in \intervalleoo{a_{i_0}}{a_{i_0+1}}\). la 
      subdivision \(\sigma'\) est plus fine que \(\sigma\) et alors
      \begin{align*}
        \sigma &= (a_0, \ldots, a_n) \\
        \sigma' &= (a_0, \ldots, a_{i_0},c,a_{i_0+1}, \ldots, a_n),
      \end{align*}
      donc
      \begin{equation}
        I(f,\sigma')= \sum_{i=1}^{i_0}(a_i-a_{i-1})\lambda_i + 
        (c-a_{i_0})\lambda_{i_0+1} + (a_{i_0+1}-c)\lambda_{i_0+1} + 
        \sum_{i=i_0+1}^{n}(a_i-a_{i-1})\lambda_i,
      \end{equation}
      et finalement en retranchant
      \begin{align*}
        I(f,\sigma')-I(f,\sigma)&= (a_{i_0+1}-a_{i_0})\lambda_{i_0+1} - 
        (c-a_{i_0})\lambda_{i_0+1} - (a_{i_0+1}-c)\lambda_{i_0+1} \\
        &=0.
      \end{align*}
  \end{itemize}
  On en déduit alors par récurrence ``immédiate''  que si on rajoute un nombre 
  fini de points à \(\sigma\) pour former une subdivision \(\sigma'\) alors 
  \(I(f,\sigma')=I(f,\sigma)\). Toute subdivision \(\sigma'\) plus fine que 
  \(\sigma\) vérifie \(I(f,\sigma')=I(f,\sigma)\).

  Soient \(\sigma\) et \(\sigma'\) deux subdivisions quelconque adaptées à 
  \(f\). Soit \(s\) la réunion de \(\sigma\) et de \(\sigma'\). Alors \(s\) est 
  adaptée à \(f\) et est plus fine que \(\sigma\)~: \(I(f,s)=I(f,\sigma)\). La 
  subdivision \(s\) est adaptée à \(f\) et est plus fine que \(\sigma'\)~: 
  \(I(f,s)=I(f,\sigma')\). Par conséquent \(I(f,\sigma')=I(f,\sigma)\).
\end{proof}

\emph{Remarque}~: La valeur de \(I(f,\sigma)\) ne dépend pas de la valeur de 
\(f\) en les points de la subdivision. En particulier si \(f \in 
\R^{\intervalleff{a}{b}}\) est nulle sauf en un nombre fini de points alors 
\(f\) est en escalier sur \(\intervalleff{a}{b}\) et \(\int_a^b f=0\).

\begin{prop}
  Soit \(f \in \E(\intervalleff{a}{b})\) et \(g \in \R^{\intervalleff{a}{b}}\) 
  telles que \(f\) et \(g\) ne différent qu'un un nombre fini de points. Alors 
  \(g\) est également en escalier sur \(\intervalleff{a}{b}\) et
  \begin{equation}
    \int_a^b f = \int_a^b g.
  \end{equation}
\end{prop}
En particulier, une fonction constante égale à \(k \in \R\) sur 
\(\intervalleff{a}{b}\) sauf en un nombre fini de points est en escalier sur 
\(\intervalleff{a}{b}\) et
\begin{equation}
  \int_a^b f =k(b-a).
\end{equation}

\begin{proof}
  Soit \(\sigma\) une subdivision adaptée à \(f\). On rajoute à \(\sigma\) les 
  points de \(\intervalleff{a}{b}\) où \(f\) et \(g\) différents. On obtient 
  alors une subdivision \(\sigma'\) plus fine que \(\sigma\) et donc adaptée à 
  \(f\).

  On note \(\sigma'=(a_i)_{i \in \intervalleentier{0}{n}}\) et sur chaque 
  intervalle \(\intervalleoo{a_i}{a_{i+1}}\) \(g\) est égale à \(f\), donc elle 
  est constante. Ainsi \(g\) est en escalier et \(\sigma'\) est adaptée à \(g\).
  \begin{equation}
    \int_a^bf = I(f,\sigma')=I(g,\sigma')=\int_a^bg
  \end{equation}
\end{proof}

\subsection{Propriétés de l'intégrale d'une fonction en escalier}

\subsubsection{Additivité par rapport aux intervalles}

\begin{prop}
  Soit \(f \in \R^{\intervalleff{a}{b}}\) et \(c \in \intervalleff{a}{b}\), 
  alors
  \begin{equation}
    f \in \E(\intervalleff{a}{b}) \iff f_{|\intervalleff{a}{c}} \in 
    \E(\intervalleff{a}{c}) \text{~et~} f_{|\intervalleff{c}{b}} \in 
    \E(\intervalleff{c}{b}).
  \end{equation}
  Auquel cas
  \begin{equation}
    \int_a^b f = \int_a^c f +\int_c^b f.
  \end{equation}
\end{prop}
\begin{proof}
  Supposons que \(f \in \E(\intervalleff{a}{b})\). Soit 
  \(\sigma=(a_i)_{0\leqslant i \leqslant n}\) une subdivision de 
  \(\intervalleff{a}{b}\) adaptée à \(f\). Soit \(\sigma'\) la subdivision 
  obtenue en ajoutant \(c\) à \(\sigma\). Soit \((\lambda_i)_{0 \leqslant i 
  \leqslant n}\in\R^{n+1}\) la famille des valeurs constantes de \(f\) sur 
  \(\sigma'\). On note \(\sigma_1'\) (resp. \(\sigma_2'\)) les restrictions de 
  \(\sigma'\) à \(\intervalleff{a}{c}\) (resp. \(\intervalleff{c}{b}\)). La 
  fonction \(f_{|\intervalleff{a}{c}}\) (resp. \(f_{|\intervalleff{c}{b}}\)) est 
  constante sur les intervalles ouverts de \(\sigma_1'\) (resp. \(\sigma_2'\)). 
  Alors ces deux fonctions sont en escaliers.
  \begin{align*}
    \int_a^c f_{|\intervalleff{a}{c}} +\int_c^b f_{|\intervalleff{c}{b}} &= 
    I(f_{|\intervalleff{a}{c}},\sigma_1')+I(f_{|\intervalleff{c}{b}},\sigma_2') 
    \\
    &=\sum_{i=1}^{i_0-1} (a_i-a_{i-1})\lambda_i + (c-a_{i_0-1})\lambda_{i_0} + 
    (a_{i_0}-c)\lambda_{i_0} + \sum_{i=i_0+1}^{n} (a_i-a_{i-1})\lambda_i \\
    &= \sum_{i=1}^{i_0-1} (a_i-a_{i-1})\lambda_i + 
    \lambda_{i_0}(a_{i_0}-a_{i_0-1}) +\sum_{i=i_0+1}^{n} 
    (a_i-a_{i-1})\lambda_i\\
    &=\sum_{i=1}^{n} (a_i-a_{i-1})\lambda_i\\
    &=I(f,\sigma)=\int_a^bf
  \end{align*}

  Supposons désormais que \(f_{|\intervalleff{a}{c}} \in 
  \E(\intervalleff{a}{c})\) et \( f_{|\intervalleff{c}{b}} \in 
  \E(\intervalleff{c}{b})\). Soient \(\sigma=(a_i)_{i \in 
  \intervalleentier{0}{n}}\) et \(\sigma'=(a_i)_{i \in 
  \intervalleentier{m}{n}}\) deux subdivisions respectives de 
  \(\intervalleff{a}{c}\) et \(\intervalleff{c}{b}\) respectivement adaptées à  
  \(f_{|\intervalleff{a}{c}}\) et \( f_{|\intervalleff{c}{b}}\). Soit \(s\) la 
  réunion de \(\sigma\) et \(\sigma'\). La fonction \(f\) est constante sur tous 
  les intervalles ouverts de \(s\) donc \(f\) est en escalier sur 
  \(\intervalleff{a}{b}\). De la même manière on montre que \(\int_a^c 
  f_{|\intervalleff{a}{c}} +\int_c^b f_{|\intervalleff{c}{b}} = \int_a^b f\).
\end{proof}

\subsubsection{Linéarité par rapport aux fonctions}

On a vu que \(\E(\intervalleff{a}{b})\) est un \(\R\)-espace vectoriel.

\begin{theo}
  L'application 
  \(\fonction{\varphi}{\E(\intervalleff{a}{b})}{\R}{f}{\int_a^bf}\) est une 
  forme linéaire. Autrement dit
  \begin{equation}
    \forall \lambda \in \R \ \forall (f,g) \in \E(\intervalleff{a}{b})^2 \quad 
    \int_a^b (\lambda f +g) = \lambda \int_a^b f +\int_a^b g
  \end{equation}
\end{theo}
\begin{proof}
  Soient \((f,g) \in \E(\intervalleff{a}{b})^2\) et un réel \(\lambda\). Soient 
  \(\sigma\) et \(\sigma'\) deux subdivisions respectives de 
  \(\intervalleff{a}{b}\) respectivement adaptées à \(f\) et à \(g\). On 
  considère la réunion \(s=(a_i)_{0 \leqslant i \leqslant n}\) de \(\sigma\) et 
  \(\sigma'\). Alors \(s\) est adaptée à \(f\) et à \(g\). Soient 
  \((\alpha_i)_{0 \leqslant i \leqslant n}\) (resp. \((\beta_i)_{0 \leqslant i 
  \leqslant n}\)) la famille des valeurs de \(f\) (resp. \(g\)) sur les 
  intervalles ouverts \(\intervalleoo{a_{i-1}}{a_i}\).

  Alors \(\lambda f+g\) est constante sur chaque intervalle ouvert 
  \(\intervalleoo{a_{i-1}}{a_i}\) égale à \(\lambda \alpha_i +\beta_i\). Ainsi
  \begin{align*}
    I(\lambda f+g,s) &= \sum_{i=1}^n (a_i-a_{i-1})(\lambda \alpha_i+\beta_i) \\
    &=\lambda \sum_{i=1}^n (a_i-a_{i-1}) \alpha_i + \sum_{i=1}^n 
    (a_i-a_{i-1})\beta_i \\
    &=\lambda I(f,s)+ I(g,s)
  \end{align*}
  d'où la linéarité de \(\varphi\).
\end{proof}

\subsubsection{Croissance}

\begin{prop}
  Soient \(f\) et \(g\) en escalier sur \(\intervalleff{a}{b}\) telles que \(f 
  \leqslant g\). Alors
  \begin{equation}
    \int_a^b f \leqslant \int_a^b g.
  \end{equation}
  En particulier si \(f\geqslant \tilde{0}\) alors \(\int_a^b f \geqslant 0\).
\end{prop}
Cependant on peut avoir \(\int_a^b f \geqslant 0\) sans que \(f \geqslant 
\tilde{0}\).
\begin{proof}
  Soit \(f \in \E(\intervalleff{a}{b})\) telle que \(f \geqslant \tilde{0}\). 
  Soit aussi \(\sigma=(a_i)_{0 \leqslant i \leqslant n}\) une subdivision de 
  \(\intervalleff{a}{b}\) adaptée à \(f\). Soit \((\lambda_i)_{0 \leqslant i 
  \leqslant n}\) la la famille des valeurs de \(f\) sur les intervalles ouverts 
  \(\intervalleoo{a_{i-1}}{a_i}\). Puisque \(f \geqslant 0\) alors pour tout \(i 
  \in \intervalleentier{0}{n}\) on a \(\lambda_i \geqslant 0\). Ainsi
  \begin{equation}
    \int_a^b f = I(f,\sigma)=\sum_{i=1}^n(a_i-a_{i-1})\lambda_i \geqslant 0
  \end{equation}

  Soit \((f,g) \in \E(\intervalleff{a}{b})^2\) telles que \(f \leqslant g\). 
  Comme \(g-f\) est en escalaier et que \(g-f \geqslant \tilde{0}\) d'après la 
  première partie on a \(\int_a^b g-f \geqslant 0\). Par linéarité de 
  l'intégrale on a \(\int_a^b f \leqslant \int_a^b g\).
\end{proof}

\subsubsection{Majoration}

\begin{prop}
  Soit \(f\) en escalier sur \(\intervalleff{a}{b}\), alors \(\abs{f}\) est 
  aussi en escalier sur \(\intervalleff{a}{b}\) et
  \begin{equation}
    \abs{\int_a^b f} \leqslant \int_a^b \abs{f}
  \end{equation}
\end{prop}
\begin{proof}
  Soit \(\sigma=(a_i)_{0 \leqslant i \leqslant n}\) une subdivision de 
  \(\intervalleff{a}{b}\) adaptée à \(f\). Soit \((\lambda_i)_{0 \leqslant i 
  \leqslant n}\) la la famille des valeurs de \(f\) sur les intervalles ouverts 
  \(\intervalleoo{a_{i-1}}{a_i}\). Pour tout pour tout \(i \in 
  \intervalleentier{0}{n}\) \(\abs{f}\) est constante égale à 
  \(\abs{\lambda_i}\) sur \(\intervalleoo{a_{i-1}}{a_i}\). Donc \(\sigma\) est 
  adpatée à \(\abs{f}\) et
  \begin{equation}
    \int_a^b \abs{f} = \sum_{i=1}^n (a_i-a_{i-1})\abs{\lambda_i}.
  \end{equation}
  De plus
  \begin{align*}
    \abs{\int_a^b f} &= \abs{\sum_{i=1}^n (a_i-a_{i-1})\lambda_i}\\
    &\leqslant \sum_{i=1}^n (a_i-a_{i-1})\abs{\lambda_i}
  \end{align*}
  puisque \(a_{i-1} < a_i\). Donc on a bien la majoration
  \begin{equation}
    \abs{\int_a^b f} \leqslant \int_a^b \abs{f}.
  \end{equation}
\end{proof}

\begin{corth}
  \begin{equation}
    \forall f \in \E(\intervalleff{a}{b}) \quad \abs{\int_a^b f} \leqslant (b-a) 
    \sup\limits_{\intervalleff{a}{b}} \abs{f}.
  \end{equation}
\end{corth}
\begin{proof}
  On a montré que les fonctions en escalier sur \(\intervalleff{a}{b}\), sont 
  bornée. Alors la borne supérieure de \(\abs{f}\) (notée \(M\)) existe et grâce 
  à la majoration
  \begin{equation}
    \abs{\int_a^b f} \leqslant \int_a^b \abs{f} \leqslant \int_a^b \tilde{M} = 
    M(b-a).
  \end{equation}
\end{proof}

\section{Intégrale d'une fonction continue par morceaux}

\subsection{Définition de l'intégrale}

Soit une fonction \(f\) continue par morceaux sur \(\intervalleff{a}{b}\). On 
définit les ensembles \(\E^{-}(f)=\{\varphi \in \E(\intervalleff{a}{b}), \varphi 
\leqslant f\}\) et \(\E^{+}(f)=\{\psi \in \E(\intervalleff{a}{b}), \psi 
\geqslant f\}\). La fonction \(f\) est bornée, donc il existe deux réels \(m\) 
et \(M\) tels que \begin{equation}
  \tilde{m} \leqslant f \leqslant \tilde{M},
\end{equation}
et comme ces fonctions sont en escalier, on a bien \(\tilde{m} \in \E^{-}(f)\) 
et \(\tilde{M} \in \E^{+}(f)\). Les ensembles \(\E^{-}(f)\) et \(\E^{+}(f)\) 
sont non vides. On définit \begin{align*}
  I^{-}(f)=\enstq{\int_a^b \varphi}{\varphi \in \E^{-}(f)}, \\
  I^{+}(f)=\enstq{\int_a^b \varphi}{\varphi \in \E^{+}(f)}.
\end{align*}
Ces deux ensembles sont non vides puisque \(\E^{-}(f)\) et \(\E^{+}(f)\) sont 
non vides. Comme la fonction \(\tilde{M}\) majore l'ensemble \(\E^{-}(f)\), 
\(I^{-}(f)\) est majoré par \(M(b-a)\). De la même manière comme la fonction 
\(\tilde{m}\) minore l'ensemble \(\E^{-}(f)\), \(I^{+}(f)\) est minoré par 
\(m(b-a)\). Finalement, \(I^{-}(f)\) admet une borne supérieure notée \(S\) et 
\(I^{+}(f)\) admet une borne inférieure notée \(I\). Pour toutes fonctions 
\(\varphi \in \E^{-}(f)\) et \(\psi \in \E^{+}(f)\) on a
\begin{equation}
  \varphi \leqslant f \leqslant \psi,
\end{equation}
donc
\begin{equation}
  \int_a^b \varphi \leqslant \int_a^b \psi.
\end{equation}
Alors \(\int_a^b \psi\) est un majorant de \(I^{-}(f)\) alors\(\int_a^b \psi 
\geqslant S\). De plus \(\int_a^b \psi\) est un minorant de \(I^{-}(f)\) d'où
\begin{equation}
  S \leqslant I
\end{equation}

\begin{theo}
  En reprenant les notations, pour toute fonction \(f\) continue par morceaux 
  sur \(\intervalleff{a}{b}\), on a \(S=I\).
\end{theo}
\begin{proof}
  Supposons (par l'absurde) le contraire, c'est à dire que \(S < I\). Soit alors 
  \(\epsilon=\frac{I-S}{2(b-a)}>0\). Il existe un couple \((\varphi,\psi) \in 
  \E(\intervalleff{a}{b})^2\) tel que
  \begin{equation}
    \begin{cases}
      \varphi \leqslant f \leqslant \psi \\
      \psi-\varphi \leqslant \tilde{\epsilon}.
    \end{cases}
  \end{equation}
  Alors \(\varphi \in \E^{-}(f)\) et \(\psi \in \E^{+}(f)\). D'où
  \begin{equation}
    \int_a^b \varphi \leqslant S \quad \int_a^b \psi \geqslant I.
  \end{equation}
  D'une part on a
  \begin{equation}
    \int_a^b \psi -\varphi \leqslant \int_a^b 
    \tilde{\epsilon}=\epsilon(a-b)=\frac{I-S}{2}
  \end{equation}
  et d'autre part
  \begin{equation}
    \int_a^b \psi-\varphi = \int_a^b \psi - \int_a^b \varphi \geqslant I-S
  \end{equation}
  Finalement on a \(I-S \geqslant \frac{I-S}{2}\) et comme \(I>S\) on a \(1 
  \leqslant \frac{1}{2}\). Ce qui est absurde. Donc \(I=S\).
\end{proof}

\begin{defdef}
  Soit une fonction \(f\) continue par morceaux sur \(\intervalleff{a}{b}\). 
  Alors avec les notations précédentes, le réel \(I=S\) est appelé intégrale de 
  \(f\) sur \(\intervalleff{a}{b}\) et il est noté \(\int_a^b f\), 
  \(\int_{\intervalleff{a}{b}} f\) ou encore \(\int_a^b f(x)\,\diff x\).
\end{defdef}

\emph{Remarque}~: Cette définition est légitime car elle prolonge la définition 
de l'intégrale des fonctions en escalier.

Si \(f \in \E(\intervalleff{a}{b})\), alors \(f \in 
\E^{-}(\intervalleff{a}{b})\) et \(f \in \E^{+}(\intervalleff{a}{b})\) donc 
\(\int_a^b f\) appartient (au sens des fonctions en escalier) à \(I^{+}(f)\) et 
\(I^{-}(f)\). De plus \(\int_a^bf\) est le minimum de \(I^{+}(f)\) et le maximum 
de \(I^{-}(f)\). Donc \(S=I=\int_a^b f\) au sens de la fonction en escalier.

Ce qu'on a appelé \(\int_a^b f\) au sens des fonctions continues par morceaux 
est donc la même intégrale.

\emph{Interprétation graphique}~: Si \(f\) est continue par morceaux sur 
\(\intervalleff{a}{b}\), et si elle est positive, alors \(\int_a^b f\) 
représente l'aire entre l'axe des abscisses et la courbe représentative de 
\(f\).

\subsection{Linéarité de l'intégrale}

\begin{theo}
  L'application \(\fonction{\int_a^b}{\CM(\intervalleff{a}{b})}{\R}{f}{\int_a^b 
  f}\) est une forme linéaire. C'est-à-dire que
  \begin{equation}
    \forall (f,g) \in \CM(\intervalleff{a}{b})^2 \ \forall \lambda \in \R^2 
    \quad \int_a^b (\lambda f+g) = \lambda \int_a^b f + \int_a^b g.
  \end{equation}
\end{theo}
\begin{proof}
  La démonstration se déroule en trois étapes. Soient d'abord deux fonctions 
  \(f\) et \(g\) continues par morceaux sur \(\intervalleff{a}{b}\) et un réel 
  \(\lambda\). On montre d'abord que \(\int_a^b (-f) = -\int_a^b f\), ensuite 
  que \(\int_a^b (f+g) = \int_a^b f + \int_a^b g\) et enfin que \(\int_a^b 
  (\lambda f) = \lambda \int_a^b f\).

  Montrons le premier point. Par définition,
  \begin{align*}
    \int_a^b (-f) &= \sup\limits_{\varphi \leqslant -f \in 
    \E(\intervalleff{a}{b})} \int_a^b \varphi ,\\
    &= \sup\limits_{\psi \geqslant f \in \E(\intervalleff{a}{b})} \int_a^b -\psi 
    ,\\
  &= \sup\limits_{\psi \geqslant f \in \E(\intervalleff{a}{b})} \int_a^b -\psi 
  \end{align*}
  car \(\psi\) est une fonction en escalier et on a montré la linéarité. Ensuite
  \begin{equation}
    \int_a^b (-f) = -\inf\limits_{\psi \geqslant f \in \E(\intervalleff{a}{b})} 
    \int_a^b \psi ,\\
  \end{equation}
  d'après les propriétés de la borne inférieure et supérieures. Alors
  \begin{equation}
    \int_a^b (-f) = -\int_a^b f,
  \end{equation}
  par définition.

  Soit ensuites \((\varphi_1, \varphi_2) \in \E^{-}(f) \times \E^{-}(g)\) alors 
  \(\varphi+\varphi_2\) est en escalier et \(\varphi_1+\varphi_2 \leqslant 
  f+g\), donc \(\varphi_1+\varphi_2 \in \E^{-}(f+g)\). Ainsi
  \begin{equation}
    \int_a^b (\varphi_1+\varphi_2) \leqslant \int_a^b (f+g),
  \end{equation}
  par linéarité de l'intégrale pour les fonction en escalier, \(\int_a^b 
  (\varphi_1+\varphi_2) = \int_a^b \varphi_1 + \int_a^b \varphi_2\). D'où
  \begin{equation}
    \int_a^b \varphi_1 \leqslant \int_a^b (f+g) - \int_a^b \varphi_2,
  \end{equation}
  et cette égalité est vraie pour toute fonction \(\varphi_1 \in \E^{-}(f)\) 
  donc \(\int_a^b (f+g) - \int_a^b \varphi_2\) est un majorant de \(I^{-}(f)\). 
  Alors
  \begin{equation}
    \int_a^b (f+g) - \int_a^b \varphi_2 \geqslant \sup I^{-}(f) = \int_a^b f.
  \end{equation}
  De la même manière
  \begin{equation}
    \int_a^b \varphi_2 \leqslant \int_a^b (f+g) - \int_a^b f,
  \end{equation}
  est vraie pour toute fonction \(\varphi_2 \in \E^{-}(g)\), alors \(\int_a^b 
  (f+g) - \int_a^b f\) est un majorant de \(I^{-}(g)\). D'où
  \begin{equation}
    \int_a^b g = \sup I^{-}(g) \leqslant \int_a^b (f+g) - \int_a^b f.
  \end{equation}

  On vient de montrer que
  \begin{equation}
    \int_a^b f + \int_a^b g \leqslant \int_a^b (f+g),
  \end{equation}
  on a aussi
  \begin{equation}
    \int_a^b (-f) + \int_a^b (-g) \leqslant \int_a^b (-f-g).
  \end{equation}
  Soit alors
  \begin{equation}
    -\int_a^b f - \int_a^b g \leqslant -\int_a^b (f+g),
  \end{equation}
  et en passant à l'opposé on change le sens de l'inégalité
  \begin{equation}
    \int_a^b f + \int_a^b g \geqslant \int_a^b (f+g).
  \end{equation}
  Finalement, il y a égalité \begin{equation}
    \int_a^b f + \int_a^b g = \int_a^b (f+g).
  \end{equation}

  Si \(\lambda>0\) on a
  \begin{align*}
    \int_a^b (\lambda f) &= \sup\limits_{\varphi \leqslant \lambda f \in 
    \E(\intervalleff{a}{b})} \int_a^b \varphi \\
    &= \sup\limits_{\psi \leqslant f \in \E(\intervalleff{a}{b})} \int_a^b 
    \lambda \psi \\
    &= \sup\limits_{\psi \leqslant f \in \E(\intervalleff{a}{b})} \lambda 
    \int_a^b \psi \\
    &= \lambda \sup\limits_{\psi \leqslant f \in \E(\intervalleff{a}{b})} 
    \int_a^b \psi \\
    &=\lambda \int_a^b f.
  \end{align*}
  Si \(\lambda=0\), alors l'égalité est vérifié. Si \(\lambda<0\) alors
  \begin{align*}
    \int_a^b (\lambda f) = \int_a^b (-\lambda)(-f) = -\lambda \int_a^b(-f) = 
    \lambda \int_a^b f.
  \end{align*}
\end{proof}

\subsection{Positivité et croissance -- Majoration de la valeur absolue de 
l'intégrale}

\subsubsection{Fonctions continues par morceaux}

\begin{theo}
  Pour toute fonction \(f\) continue par morceaux sur \(\intervalleff{a}{b}\), 
  on a
  \begin{equation}
    f \geqslant \tilde{0} \implies \int_a^b f \geqslant 0.
  \end{equation}
  \danger Ce sont des inégalités larges.
\end{theo}
\begin{proof}
  La fonction nulle est en escalier. Si en plus \(f \geqslant \tilde{0}\) alors 
  \(\tilde{0} \in \E^{-}(f)\) et alors par définition \(\int_a^b f \geqslant 
  \int_a^b \tilde{0}=0\).
\end{proof}

\begin{theo}
  Pour toute fonctions \(f\) et \(g\) continues par morceaux sur 
  \(\intervalleff{a}{b}\), on a
  \begin{equation}
    f \geqslant g \implies \int_a^b f \geqslant \int_a^b g.
  \end{equation}
  \danger Ce sont des inégalités larges.
\end{theo}
\begin{proof}
  Il suffit d'appliquer le théorème précédent à \(f-g\).
\end{proof}

\begin{theo}
  Pour toute fonction \(f\) continue par morceaux sur \(\intervalleff{a}{b}\), 
  on a
  \begin{equation}
    \abs{\int_a^b f~} \leqslant \int_a^b \abs{f},
  \end{equation}
  et en particulier
  \begin{equation}
    \abs{\int_a^b f~} \leqslant (b-a) \sup\limits_{\intervalleff{a}{b}} \abs{f}.
  \end{equation}
\end{theo}
\begin{proof}
  Pour toute fonction \(f\) continue par morceaux sur \(\intervalleff{a}{b}\), 
  on a
  \begin{equation}
    -\abs{f} \leqslant f \leqslant \abs{f},
  \end{equation}
  alors
  \begin{equation}
    \int_a^b -\abs{f} \leqslant \int_a^bf \leqslant \int_a^b\abs{f},
  \end{equation}
  par linéarité on obtient
  \begin{equation}
    -\int_a^b \abs{f} \leqslant \int_a^bf \leqslant \int_a^b\abs{f},
  \end{equation}
  D'où
  \begin{equation}
    \abs{\int_a^b f~} \leqslant \int_a^b \abs{f}.
  \end{equation}

  Les fonctions continues par morceaux sont bornées. Soit donc \(g\) la fonction 
  constante égale à \(\sup\limits_{\intervalleff{a}{b}} \abs{f}\) sur 
  \(\intervalleff{a}{b}\), alors
  \begin{equation}
    \abs{f} \leqslant g
  \end{equation}
  et ainsi
  \begin{equation}
    \int_a^b \abs{f} \leqslant \int_a^b g = (b-a) 
    \sup\limits_{\intervalleff{a}{b}} \abs{f}
  \end{equation}
\end{proof}

\subsubsection{Cas des fonctions continues}

\danger \emph{Ce qui suit ne s'applique pas aux fonctions continues par 
morceaux}.

\begin{theo}
  Pour toute fonction continue \(f\) sur \(\intervalleff{a}{b}\), si \(f\) est 
  non nulle et \(f \geqslant 0\) alors \(\int_a^b f >0\).
\end{theo}
\begin{proof}
  Comme \(f\) est positive et non nulle, il existe un \(x_0 \in 
  \intervalleff{a}{b}\) tel que \(f(x_0)>0\). Comme \(f\) est continue, il 
  existe un couple \((\alpha,\beta) \in \intervalleff{a}{b}^2\) 
  (\(\alpha<\beta\)) tel que
  \begin{equation}
    \forall x \in \intervalleff{a}{b} \quad \frac{f(x_0)}{2}.
  \end{equation}
  Soit \(\varphi \in \R^{\intervalleff{a}{b}}\) définie par
  \begin{equation}
    \begin{cases}
      \varphi(x)=\frac{f(x_0)}{2} & x \in \intervalleff{\alpha}{\beta} \\
      \varphi(x)=0 & x \in \intervalleff{a}{b} \setminus 
      \intervalleff{\alpha}{\beta}
    \end{cases}.
  \end{equation}
  La fonction \(\varphi\) est donc en escalier sur \(\intervalleff{a}{b}\) et 
  \(\varphi \leqslant f\). Par conséquent
  \begin{equation}
    \int_a^b f \geqslant \int_a^b \varphi,
  \end{equation}
  et par définition
  \begin{equation}
    \int_a^b \varphi = (\beta-\alpha) \frac{f(x_0)}{2} >0.
  \end{equation}
  donc \(\int_a^b f >0\).
\end{proof}

\begin{corth}
  Soit une fonction \(f\) continue sur \(\intervalleff{a}{b}\). Si \(f\) est 
  positive et si son intégrale est nulle alors \(f\) est nulle.
\end{corth}

\begin{theo}
  Pour toutes fonctions \(f\) et \(g\) continues sur \(\intervalleff{a}{b}\), on 
  a
  \begin{equation}
    f \geqslant g \text{~et~} f \neq g \implies \int_a^b f > \int_a^b g.
  \end{equation}
\end{theo}
\begin{proof}
  Il suffit d'appliquer le théorème précédent à \(f-g\).
\end{proof}

\begin{corth}
  Pour toutes fonctions \(f\) et \(g\) continues sur \(\intervalleff{a}{b}\), on 
  a
  \begin{equation}
    f \geqslant g \text{~et~} \int_a^b f = \int_a^b g  \implies f = g.
  \end{equation}
\end{corth}

\begin{theo}
  Pour toute fonctions \(f\) continue sur \(\intervalleff{a}{b}\), on a
  \begin{equation}
    \abs{\int_a^b f~} = \int_a^b \abs{f} \iff f \text{~est de signe constant 
    sur~} \intervalleff{a}{b}.
  \end{equation}
\end{theo}
\begin{proof}
  \(\impliedby\) Si \(f\) est de signe constant, alors si \(f \geqslant 
  \tilde{0}\) alors \(\int_a^b f \geqslant 0\) et donc
  \begin{equation}
    \abs{\int_a^b f~} = \int_a^b f = \int_a^b \abs{f};
  \end{equation}
  sinon \(f \leqslant \tilde{0}\) et alors \(\int_a^b f \leqslant 0\) et donc
  \begin{equation}
    \abs{\int_a^b f~} = -\int_a^b f = \int_a^b (-f) = \int_a^b \abs{f}.
  \end{equation}

  \(\implies\) Deux cas se présentent selon le signe de l'intégrale.
  \begin{itemize}
    \item Si \(\int_a^b f \geqslant 0\) alors l'hypothèse s'écrit~:
      \begin{equation}
        \int_a^b f = \int_a^b \abs{f}
      \end{equation}
      avec \(f \leqslant \abs{f}\). Donc \(f=\abs{f}\) et alors \(f \geqslant 
      \tilde{0}\).
    \item Si \(\int_a^b f \leqslant 0\) alors \(-\int_a^b f = \int_a^b 
      \abs{f}\) et par linéarité
      \begin{equation}
        \int_a^b -f = \int_a^b \abs{f}
      \end{equation}
      avec \(-f \leqslant \abs{f}\) donc \(-f=\abs{f}\) et alors \(f \leqslant 
      \tilde{0}\).
  \end{itemize}
\end{proof}

\subsection{Additivité par rapport au segment d'intégration}

\subsubsection{Relation de Chasles}

\begin{theo}
  Soient trois réels \(a\), \(b\) et \(c\) tels que \(a \leqslant b<c\) et 
  soit une fonction \(f \in \R^{\intervalleff{a}{b}}\). La fonction \(f\) est 
  continue par morceaux si et seulement si sa restriction à 
  \(\intervalleff{a}{c}\) et sa restriction à \(\intervalleff{c}{b}\) sont 
  continues par morceaux. Auquel cas,
  \begin{equation}
    \int_a^b f = \int_a^c f +\int_c^b f.
  \end{equation}
\end{theo}
\begin{proof}
  L'équivalence se prouve comme pour les fonctions en escaliers, c'est-à-dire 
  en travaillant sur les subdivisions.

  Soit une fonction \(f \in \CM(\intervalleff{a}{b})\), alors 
  \(f_{|\intervalleff{a}{c}} \in \CM(\intervalleff{a}{c})\) et 
  \(f_{|\intervalleff{c}{b}} \in \CM(\intervalleff{c}{b})\). Soit 
  \(\epsilon>0\), alors
  \begin{align*}
    \exists (\varphi_1, \psi_1) \in \E(\intervalleff{a}{c})^2 \ \begin{cases} 
      \varphi_1 \leqslant f_{|\intervalleff{a}{c}} \\ \psi_1 - \varphi_1 
    \leqslant \tilde{\epsilon} \end{cases} \\
    \exists (\varphi_2, \psi_2) \in \E(\intervalleff{c}{b})^2 \ \begin{cases} 
      \varphi_2 \leqslant f_{|\intervalleff{c}{b}} \\ \psi_2- \varphi_2 
    \leqslant \tilde{\epsilon} \end{cases}
  \end{align*}
  Soient deux fonctions \(\psi\) et \(\varphi\) de \(\intervalleff{a}{b}\) 
  vers \(\R\) définies par
  \begin{align*}
    \forall x \in \intervallefo{a}{c} \quad \varphi(x)=\varphi_1(x) 
    \text{~et~} \psi(x)=\psi_1(x) \\
    \varphi(c)=\psi(c)=f(c) \\
    \forall x \in \intervalleof{c}{b} \quad \varphi(x)=\varphi_2(x) 
    \text{~et~} \psi(x)=\psi_2(x).
  \end{align*}
  Les fonction \(\psi\) et \(\varphi\) sont en escalier sur 
  \(\intervalleff{a}{b}\) et vérifient \(\varphi \leqslant f \leqslant \psi\). 
  De plus \(\psi-\varphi \leqslant \tilde{\epsilon}\).

  On peut appliquer la relations de Chasles pour les fonctions en escalier~:
  \begin{equation}
    \int_a^b f \leqslant \int_a^b \psi = \int_a^c \psi + \int_c^b \psi= 
    \int_a^c \psi_1 + \int_c^b \psi_2,
  \end{equation}
  par croissance de l'intégrale on a
  \begin{align*}
    \int_a^b f & \leqslant \int_a^c (\psi_1 + \tilde{\epsilon})  + \int_c^b 
    (\psi_2 + \tilde{\epsilon}) \\
    \leqslant \int_a^c f + \int_c^b f + \epsilon(b-a)
  \end{align*}
  et comme c'est vrai pour tout \(\epsilon >0\), lorsqu'on fait tendre 
  \(\epsilon\) vers zéro, on obtient~:
  \begin{equation}
    \int_a^b f \leqslant \int_a^c f + \int_c^b f.
  \end{equation}
  Si on applique ce résultat à \(-f\) on obtient l'inégalité inverse.
\end{proof}

\begin{prop}
  Soient un intervalle réel \(I\) et une fonction \(f \in \CM(I)\). pour tout 
  réels \(a\) et \(b\) de \(I\) on a
  \begin{equation}
    \int_a^b f = - \int_b^a f
  \end{equation}
\end{prop}

\begin{theo}
  Soient un intervalle réel \(I\) et une fonction \(f \in \CM(I)\). pour tout 
  réels \(a\) et \(b\) de \(I\) on a~:
  \begin{enumerate}
    \item Si \(f \geqslant 0\), si \(a \leqslant b\) alors \(\int_a^bf 
      \geqslant 0\) et si \(a \geqslant b\) \(\int_a^b f \leqslant 0\);
    \item \(\abs{\int_a^b f} \leqslant \int_a^b \abs{f}\) si \(a<b\) et 
      \(\abs{\int_a^b f} \geqslant \int_b^a \abs{f}\) si \(a \geqslant b\);
    \item \(\abs{\int_a^b f~} \leqslant \abs{b-a} 
      \sup\limits_{\intervalleff{a}{b}} \abs{f}\)
  \end{enumerate}
\end{theo}

\subsection{Invariance par translation}

\begin{theo}
  Soient deux réels \(a\) et \(b\) (\(a<b\)), \(f \in 
  \CM(\intervalleff{a}{b})\) et un autre réel \(T \in \R\). On dispose alors 
  de l'application translatée
  \begin{equation}
    \fonction{f_T}{\intervalleff{a+T}{b+T}}{\R}{x}{f(x-T)}
  \end{equation}
  alors \(f_T \in \CM(\intervalleff{a+T}{b+T})\) et
  \begin{equation}
    \int_{a+T}^{b+T} f_T = \int_a^b f
  \end{equation}
\end{theo}
\begin{proof}
  Si \(f\) est en escalier sur \(\intervalleff{a}{b}\), soit \(\sigma=(a_i)_{0 
  \leqslant i \leqslant n}\) une subdivision de \(\intervalleff{a}{b}\) 
  adaptée à \(f\). Soit \((\lambda_i)_{0 \leqslant i \leqslant n}\) la famille 
  des valeurs constantes de \(f\) sur \(\intervalleff{a}{b}\). Soit \(\sigma_T 
  = (a_i+T)_{0 \leqslant i \leqslant n}\) et c'est une subdivision de 
  \(\intervalleff{a+T}{b+T}\) adaptée à \(f_T\). La fonction \(f_T\) est en 
  escalier et
  \begin{equation}
    \int_{a+T}^{b+T} f_T = \sum_{i=1}^n ((a_i+T)-(a_{i-1}+T))\lambda_i = 
    \sum_{i=1}^n (a_i-a_{i-1})\lambda_i=\int_a^b f
  \end{equation}


  On suppose maintenant que \(f\) est continue par morceaux sur 
  \(\intervalleff{a}{b}\). Soit \(\sigma=(a_i)_{0 \leqslant i \leqslant n}\) 
  une subdivision de \(\intervalleff{a}{b}\) adaptée à \(f\). Soit \(\varphi 
  \in \E^{-}(f)\), on dispose alors de la fonction \(\varphi_T \in 
  \E(\intervalleff{a+T}{b+T})\) telle que \(\int_{a+T}^{b+T} \varphi_T = 
  \int_a^b\varphi\). Alors
  \begin{equation}
    \varphi_T \leqslant f_T
  \end{equation}
  donc \(\varphi_T \in \E^{-}(f)\) et alors
  \begin{equation}
    \int_a^b\varphi = \int_{a+T}^{b+T} \varphi_T \leqslant \int_{a+T}^{b+T} 
    f_T.
  \end{equation}
  C'est vrai pour toute fonction \(\varphi \in \E^{-}(f)\) donc 
  \(\int_{a+T}^{b+T} f_T\) est un majorant de \(I^{-}(f)\) d'où
  \begin{equation}
    \int_{a+T}^{b+T} f_T \geqslant \sup I^{-}(f) = \int_a^b f.
  \end{equation}
  On applique ce résultat à \(f_T\) et \(-T\) pour obtenir l'autre inégalité, 
  ou on refait le même raisonnement avec \(\psi \in \E^{+}(f)\).
\end{proof}

\subsubsection{Applications aux fonctions périodiques}

Soient \(f \in \R^\R\) et \(T>0\) tel que \(f\) soit \(T\)-périodique. 
Supposons que \(f\) est continue par morceaux sur tous les segments de \(\R\). 
Alors
\begin{align*}
  \forall a,b \in \R^2 \quad \int_{a+T}^{b+T} f = \int_a^b f \\
  \forall a,b \in \R^2 \quad \int_a^{a+T} f =\int_b^{b+T} f
\end{align*}

\begin{proof}
  Pour la première égalité, on applique directement le résultat de la 
  sous-sous-section précédente, \(f_T=f\) puisque \(f\) est \(T\)-périodique.

  Pour la deuxième égalité, on applique la relation de Chasles et la première 
  égalité.
\end{proof}

\subsection{Inégalité de la moyenne}

\begin{theo}
  Soient \((f,g) \in \CM(\intervalleff{a}{b})^2\) et \((m,M)\in \R^2\) tels 
  que \(\tilde{m} \leqslant f \leqslant \tilde{M}\). Si \(g \geqslant 
  \tilde{0}\) alors
  \begin{equation}
    m \int_a^b g \leqslant \int_a^b fg \leqslant M \int_a^b g
  \end{equation}
\end{theo}

\emph{Remarque}~: Puisque \(f\) est continue par morceaux, elle est bornée 
alors on pourra toujours trouver \(m\) et \(M\).

\begin{proof}
  Pour tout réel \(x \in \intervalleff{a}{b}\), on a\( \begin{cases} m 
  \leqslant f(x) \leqslant M \\ 0 \leqslant g(x) \end{cases}\). Donc
  \begin{equation}
    mg(x) \leqslant f(x)g(x) \leqslant Mg(x),
  \end{equation}
  par croissance de l'intégrale on a
  \begin{equation}
    \int_a^b mg \leqslant \int_a^b fg \leqslant \int_a^b Mg,
  \end{equation}
  et par linéarité
  \begin{equation}
    m \int_a^b g \leqslant \int_a^b fg \leqslant M \int_a^b g.
  \end{equation}
\end{proof}

\begin{corth}
  Soient une fonction \(f\)  continue par morceaux sur \(\intervalleff{a}{b}\) 
  et \((m,M)\in \R^2\) tels que \(\tilde{m} \leqslant f \leqslant \tilde{M}\), 
  alors
  \begin{equation}
    m \leqslant \frac{1}{b-a} \int_a^b f \leqslant M.
  \end{equation}
\end{corth}
\begin{proof}
  On applique le théorème avec \(g=\tilde{1}\).
\end{proof}

\begin{defdef}
  Pour toute fonction \(f\) continue par morceaux, le réel \(\mu = 
  \frac{1}{b-a} \int_a^b f\) est appelée la valeur moyenne de \(f\) sur le 
  segment \(\intervalleff{a}{b}\).
\end{defdef}

\emph{Remarque}~: Supposons que \(f\) est continue sur 
\(\intervalleff{a}{b}\). Le théorème des bornes permet de choisir \(m = 
\min\limits_{\intervalleff{a}{b}} f\) et \(M = 
\max\limits_{\intervalleff{a}{b}} f\). Alors la valeur moyenne \(\mu \in 
\intervalleff{m}{M}\). La fonction \(f\) est continue donc d'après le théorème 
des valeurs intermédiaires il existe un réel \(c \in \intervalleff{a}{b}\) tel 
que \(f(c)=\mu\).

\begin{theo}
  Pour toutes fonction \(f\) et \(g\) continues par morceaux sur 
  \(\intervalleff{a}{b}\),
  \begin{equation}
    \abs{\int_a^b fg~} \leqslant \sup_{\intervalleff{a}{b}} \abs{f} \int_a^b 
    \abs{g}
  \end{equation}
\end{theo}
\begin{proof}
  La fonction \(f\) est continue par morceaux docn bornée, alors \(M = 
  \sup_{\intervalleff{a}{b}} \abs{f} \geqslant 0\) existe. Alors
  \begin{equation}
    \abs{\int_a^b fg~} \leqslant \int_a^b \abs{fg}
  \end{equation}
  et comme \(\abs{fg} \leqslant M\abs{g}\) alors par croissance et par 
  linéarité on a
  \begin{equation}
    \abs{\int_a^b fg~} \leqslant M \int_a^b \abs{g}.
  \end{equation}
\end{proof}

\emph{Remarque}~: Avec \(g=\tilde{1}\) on retrouve \(\abs{\int_a^b fg~} 
\leqslant \abs{b-a} \sup_{\intervalleff{a}{b}} \abs{f}\).

\subsection{Inégalité de Cauchy-Schwarz et inégalité de Minkowski}

\subsubsection{Inégalité de Cauchy-Schwarz}

\begin{theo}
  Pour toutes fonctions \(f\) et \(g\) continues par morceaux sur 
  \(\intervalleff{a}{b}\), on a
  \begin{equation}
  \abs{\int_a^b fg} \leqslant \sqrt{\int_a^b f^2 \int_a^b g^2} \end{equation}
\end{theo}

\emph{Remarque}~: Si les fonctions sont continues, on note \(\langle f,g 
\rangle = \int_a^b fg\) , c'est un produit scalaire et on note \(\norme{f}_2 = 
\sqrt{\int_a^b f^2}\) c'est la norme euclidienne associée au produit scalaire. 
L'inégalité devient \(\abs{\prodscal{f}{g}} \leqslant \norme{f}_2 
\norme{g}_2\).

\begin{proof}
  Soit la fonction
  \begin{equation}
    \fonction{P}{\R}{\R}{\lambda}{\int_a^b (f+\lambda g)^2}
  \end{equation}
  Alors pour tout réel \(\lambda\), \(P(\lambda)>0\) et
  \begin{equation}
    P(\lambda)= \lambda^2 \int_a^b g^2 + 2\lambda \int_a^b fg + \int_a^b f^2.
  \end{equation}
  Deux cas se présentent~:
  \begin{enumerate}
    \item Si \(\int_a^b g^2=0\) alors on a \(2\lambda \int_a^b fg + \int_a^b 
      f^2 \geqslant 0\) pour tout \(\lambda\) alors \(\int_a^b fg=0\) et 
      l'inégalité est vraie \(0 \leqslant 0\);
    \item Sinon alors \(P\) est polynomiale de degré \(2\) et \(P \geqslant 
      \tilde{0}\). Alors le dsicriminant est négatif~:
      \begin{equation}
        4 \left(\int_a^b fg \right)^2 - 4 \int_a^bf^2 \int_a^b g^2 \leqslant 0
      \end{equation}
      et ainsi on a bien l'inégalité
      \begin{equation}
        \left(\int_a^b fg \right)^2 \leqslant \int_a^bf^2 \int_a^b g^2
      \end{equation}
      et comme \(\int_a^bf^2 \int_a^b g^2  \geqslant 0\), par positivité de 
      l'intégrale on a bien l'inégalité de Cauchy-Schwarz~:
      \begin{equation}
        \abs{\int_a^b fg~} \leqslant \sqrt{\int_a^b f^2 \int_a^b g^2}.
      \end{equation}
  \end{enumerate}
\end{proof}

\begin{prop}
  Pour toutes fonctions \(f\) et \(g\) continues sur \(\intervalleff{a}{b}\), 
  il y a égalité dans l'inégalité de Cauchy-Schwarz si et seulement si la 
  famille \((f,g)\) est liée.
\end{prop}
\begin{proof}
  Deux cas se présentent~:
  \begin{enumerate}
    \item Si \(\int_a^b g^2=0\) alors \(g^2=\tilde{0}\) donc \(g=\tilde{0}\). 
      Alors
      \begin{equation}
      \abs{\int_a^b fg} =0 = \sqrt{\int_a^b f^2 \int_a^b g^2} \end{equation}
    \item Sinon, alors le discriminant du polynôme \(P\) (cf demonstration 
      précédente) est nul et il existe une racine double \(\lambda_0\) tel que
      \begin{equation}
        \begin{cases}
          \int_a^b (f+\lambda_0 g)^2=0\\
          (f+\lambda_0 g)^2 \geqslant 0
        \end{cases}
      \end{equation}
      comme \(f+\lambda_0 g\) est continue, on a alors d'après l'intégrale 
      nulle \(f=-\lambda_0 g\).

      Réciproquement s'il existe \(\mu \in \R\) tel que \(f=\mu g\) alors
      \begin{equation}
        \abs{\int_a^b fg~} = \abs{\int_a^b \mu g^2~} = \abs{\mu} \int_a^b g^2
      \end{equation}
      et alors
      \begin{equation}
        \sqrt{\int_a^b f^2 \int_a^b g^2}  = \sqrt{\mu^2 \left(\int_a^b g^2 
        \right)^2} = \abs{\mu} \int_a^b g^2.
      \end{equation}
  \end{enumerate}
\end{proof}

\subsubsection{Inégalité de Minkowski}

\begin{theo}
  Pour toutes fonctions \(f\) et \(g\) continues par morceaux sur 
  \(\intervalleff{a}{b}\), on a
  \begin{equation}
    \sqrt{\int_a^b (f+g)^2} \leqslant \sqrt{\int_a^b f^2}+\sqrt{\int_a^b g^2} 
  \end{equation}
  C'est l'inégalité triangulaire \(||f+g||_2 \leqslant ||f||_2 + ||g||_2\).
\end{theo}
\begin{proof}
  Pour toutes fonctions \(f\) et \(g\) continues par morceaux sur 
  \(\intervalleff{a}{b}\), on a
  \begin{align*}
    \int_a^b (f+g)^2 & = \int_a^b f^2 + \int_a^b g^2 +2 \int_a^b fg \\
    &\leqslant \int_a^b f^2 + \int_a^b g^2 +2 \sqrt{\int_a^b f^2 \int_a^b g^2} 
    \\
    &\leqslant \left(\sqrt{\int_a^b f^2}+\sqrt{\int_a^b g^2}\right)^2.
  \end{align*}
  La première inégalité est issue de l'inégalité de Cauchy-Schwarz. Cependant 
  \(\sqrt{\int_a^b f^2}+\sqrt{\int_a^b g^2} \geqslant 0\) et \(\int_a^b 
  (f+g)^2 \geqslant 0\) alors en passant à la racine on a bien l'inégalité de 
  Minkowski~:
  \begin{equation}
    \sqrt{\int_a^b (f+g)^2} \leqslant \sqrt{\int_a^b f^2}+\sqrt{\int_a^b g^2}.
  \end{equation}
\end{proof}

\begin{prop}
  Pour toutes fonctions \(f\) et \(g\) continues par morceaux sur 
  \(\intervalleff{a}{b}\), il y a égalité dans l'inégalité de Minkowski si et 
  seulement si \((f,g)\) est liée.
\end{prop}
\begin{proof}
  Commece cas d'égalité dépend du cas d'égalité dans l'inégalité de 
  Cauchy-Schwarz, c'est évident.
\end{proof}

\section{Sommes de Riemann}

\subsection{Définition}

\begin{defdef}
  Soient deux réels \(a, b\) (\(a<b\)), \(\sigma=(a_i)_{0 \leqslant i 
  \leqslant n}\) une subdivision de \(\intervalleff{a}{b}\) ed \(n+1\) points, 
  une fonction \(f \in \R^{\intervalleff{a}{b}}\) et \(\alpha=(\alpha_i)_{0 
  \leqslant i \leqslant n-1}\) une famille de \(n\) points de 
  \(\intervalleff{a}{b}\) telle que pour tout \(i \in 
  \intervalleentier{0}{n-1}\), \(\alpha_i \in \intervalleff{a_i}{a_{i+1}}\). 
  On définit
  \begin{equation}
    R(f,\sigma,\alpha)=\sum_{i=0}^{n-1}(a_{i+1}-a_i)f(\alpha_i)
  \end{equation}
  la somme de Riemann associée à \(f\), à \(\sigma\) et à la famille de points 
  \(\alpha\).
\end{defdef}

\emph{Cas particulier important}~: Si la subdivision \(\sigma\) est régulière, 
c'est-à-dire que pour tout \(i \in \intervalleentier{0}{n-1}\) \(a_i=a+i 
\frac{b-a}{n}\) alors la somme de Riemann s'écrit
\begin{equation}
  R(f,\sigma,\alpha) = \frac{b-a}{n} \sum_{i=0}^{n-1}f(\alpha_i).
\end{equation}

\subsection{Théorème de convergence des sommes de Riemman pour les fonctions 
continues}

\begin{theo}
  Soient deux réels \(a\) et \(b\) tels que \(a<b\) et une fonction \(f\) 
  continue sur \(\intervalleff{a}{b}\). Pour tout \(\epsilon >0\), il existe 
  \(\eta >0\) tel que si pour toute subdivision \(\sigma=(a_i)_{0 \leqslant i 
  \leqslant n-1}\) de \(\intervalleff{a}{b}\) de pas \(\delta(\sigma) 
  \leqslant \eta\) alors pour toute famille de points \(\alpha=(\alpha_i)_{0 
  \leqslant i \leqslant n-1}\) de \(\intervalleff{a}{b}\) (telle que pour tout 
  \(i \in \intervalleentier{0}{n-1}\) on a \(\alpha_i \in 
  \intervalleff{a_i}{a_{i+1}}\)) on a
  \begin{equation}
    \abs{R(f,\sigma,\alpha)- \int_a^b f~} \leqslant \epsilon
  \end{equation}
\end{theo}
\begin{proof}
  Pour toute subdivision \(\sigma=(a_i)_{0 \leqslant i \leqslant n-1}\) de 
  \(\intervalleff{a}{b}\) et pour toute famille de points 
  \(\alpha=(\alpha_i)_{0 \leqslant i \leqslant n-1}\) de 
  \(\intervalleff{a}{b}\) telle que pour tout \(i \in 
  \intervalleentier{0}{n-1}\) on a \(\alpha_i \in 
  \intervalleff{a_i}{a_{i+1}}\), on a
  \begin{align*}
    R(f,\sigma,\alpha)- \int_a^b f &= \sum_{i=0}^{n-1}(a_{i+1}-a_i)f(\alpha_i) 
    - \sum_{i=0}^{n-1} \int_{a_i}^{a_{i+1}} f \\
    &=\sum_{i=0}^{n-1} \int_{a_i}^{a_{i+1}} (f(\alpha_i)-f).
  \end{align*}
  Comme \(f\) est continue sur \(\intervalleff{a}{b}\), grâce au théorème de 
  Heine, elle est uniformément continue~:
  \begin{equation}
    \forall \epsilon >0 \ \exists \eta >0 \ \forall x,x' \in 
    \intervalleff{a}{b} \quad \abs{x-x'} \leqslant \eta \implies 
    \abs{f(x)-f(x')} \leqslant \frac{\epsilon}{b-a}
  \end{equation}
  Si \(\sigma\) vérifie \(\delta(\sigma) \leqslant \eta\) alors pour tout \(i 
  \in \intervalleentier{0}{n-1}\) et tout \(x \in 
  \intervalleff{a_i}{a_{i+1}}\) on a
  \begin{equation}
    \abs{x-\alpha_i} \leqslant \abs{a_{i+1}-a_i} \leqslant \delta(\sigma) 
    \leqslant \eta
  \end{equation}
  d'où \(\abs{f(x)-f(\alpha_i)} \leqslant \frac{\epsilon}{b-a}\). Par 
  croissance de l'intégrale et pour tout \(i \in \intervalleentier{0}{n-1}\) 
  on a
  \begin{equation}
    \int_{a_i}^{a_{i+1}} \abs{f(\alpha_i)-f} \leqslant 
    \int_{a_i}^{a_{i+1}}\frac{\epsilon}{b-a}.
  \end{equation}
  Alors
  \begin{align*}
    \abs{R(f,\sigma,\alpha)- \int_a^b f~} &\leqslant \sum_{i=0}^{n-1} 
    \int_{a_i}^{a_{i+1}} \abs{f(\alpha_i)-f} \\
    &\leqslant \sum_{i=0}^{n-1} \int_{a_i}^{a_{i+1}}\frac{\epsilon}{b-a} = 
    \int_a^b \frac{\epsilon}{b-a}=\epsilon.
  \end{align*}
\end{proof}

\begin{cor}
  Soient deux réels \(a\) et \(b\) tels que \(a<b\) et une fonction \(f\) 
  continue sur \(\intervalleff{a}{b}\). Pour tout \(\epsilon >0\), il existe 
  \(n_0 \in \N\) tel que pour tout \(n \geqslant n_0\) et pour toute famille  
  \(\alpha=(\alpha_i)_{0 \leqslant i \leqslant n-1}\) une famille de \(n\) 
  points de \(\intervalleff{a}{b}\) telle que pour tout \(i \in 
  \intervalleentier{0}{n-1}\), \(\alpha_i \in \intervalleff{a+i 
  \frac{b-a}{n}}{a+(i+1)\frac{b-a}{n}}\) alors
  \begin{equation}
    \abs{\frac{b-a}{n} \sum_{i=0}^{n-1}f(\alpha_i) - \int_a^b f} \leqslant 
    \epsilon
  \end{equation}
  Cas particulier pour le choix de la famille \(\alpha\)~:
  \begin{enumerate}
    \item Si pour tout \(i \in \intervalleentier{0}{n-1}\) 
      \(\alpha_i=a_i=a+i\frac{b-a}{n}\) alors
      \begin{equation}
        \lim\limits_{n \to \infty} \frac{b-a}{n} 
        \sum_{i=0}^{n-1}f\left(a+i\frac{b-a}{n}\right) = \int_a^b f
      \end{equation}
    \item Si pour tout \(i \in \intervalleentier{0}{n-1}\) 
      \(\alpha_i=a_{i+1}=a+(i+1)\frac{b-a}{n}\) alors
      \begin{equation}
        \lim\limits_{n \to \infty} \frac{b-a}{n} 
        \sum_{j=1}^{n}f\left(a+j\frac{b-a}{n}\right) = \int_a^b f.
      \end{equation}
  \end{enumerate}
\end{cor}

\subsection{Cas des fonctions lipschitziennes}

Dans le cas où \(f\) est lipschitzienne, on peut majorer la différence entre 
les sommes de Riemann et \(\int_a^b f\).

Soient deux réels \(a\) et \(b\) tels que \(a<b\) et \(k>0\) tel que \(f\) 
soit \(k\)-lipschitzienne sur \(\intervalleff{a}{b}\). Soit \(\sigma=(a_i)_{0 
\leqslant i \leqslant n}\) une subdivision de \(\intervalleff{a}{b}\) de pas 
\(\delta(\sigma)\). Soit \((\alpha_i)_{0 \leqslant i \leqslant n-1}\) une 
famille de points de \(\intervalleff{a}{b}\) telle que pour tout \(i \in 
\intervalleentier{0}{n-1}\) on a \(\alpha_i \in \intervalleff{a_i}{a_{i+1}}\). 
Alors
\begin{align*}
  \abs{R(f,\sigma,\alpha)- \int_a^b f~} &\leqslant \sum_{i=0}^{n-1} 
  \int_{a_i}^{a_{i+1}} \abs{f(\alpha_i)-f(x)}\,\diff x \\
  &\leqslant k \sum_{i=0}^{n-1} \int_{a_i}^{a_{i+1}} \abs{\alpha_i-x}\,\diff x
\end{align*}

Soit \(i \in \intervalleentier{0}{n-1}\), alors
\begin{align*}
  \int_{a_i}^{a_{i+1}} \abs{\alpha_i-x}\,\diff x &= \int_{a_i}^{\alpha_i} 
  \abs{\alpha_i-x}\,\diff x + \int_{\alpha_i}^{a_{i+1}} 
  \abs{\alpha_i-x}\,\diff x \\
  &= \left[ -\frac{(\alpha_i-x)^2}{2}\right]_{a_i}^{\alpha_i} + 
  \left[\frac{(x-\alpha_i)^2}{2}\right]_{\alpha_i}^{a_{i+1}} \\
  &= \frac{(\alpha_i-a_i)^2}{2} + \frac{(a_{i+1}-\alpha_i)^2}{2} \\
  & \leqslant \frac{(a_{i+1}-a_i)^2}{2} + \frac{(a_{i+1}-\alpha_i)^2}{2} \\
  & \leqslant \delta(\sigma)^2
\end{align*}
Alors
\begin{equation}
  \abs{R(f,\sigma,\alpha)- \int_a^b f~} \leqslant kn \delta(\sigma)^2.
\end{equation}

En particulier si \(\sigma = \left(a+i \frac{b-a}{n}\right)_{0 \leqslant i 
\leqslant n}\) alors \(\delta(\sigma)=\frac{b-a}{n}\) et donc

\begin{equation}
  \abs{R(f,\sigma,\alpha)- \int_a^b f~} \leqslant k \frac{(b-a)^2}{n}.
\end{equation}

\section{Brève extension aux fonctions à valeurs complexes}

Soit \((a,b) \in \R^2\), \(a<b\). On s'intéresse aux fonction de 
\(\intervalleff{a}{b}\) vers \(\C\).

\subsection{Intégrale d'une fonction continue par morceaux}

\begin{defdef}
  Soit \(f \in \C^{\intervalleff{a}{b}}\). On dit que \(f\) est continue par 
  morceaux sur \(\intervalleff{a}{b}\) s'il existe une subdivision 
  \(\sigma=(a_i)_{0 \leqslant i \leqslant n}\) de \(\intervalleff{a}{b}\) 
  telle que~:
  \begin{enumerate}
    \item pour tout \(i \in \intervalleentier{1}{n}\) \(f\) est continue sur 
      \(\intervalleoo{a_{i-1}}{a_i}\)
    \item pour tout \(i \in \intervalleentier{1}{n}\) 
      \(f_{|\intervalleoo{a_{i-1}}{a_i}}\) admet des limites finies à gauche 
      en \(a_i\) et à droite en \(a_{i-1}\).
  \end{enumerate}
\end{defdef}

D'après les résultats sur les limites et la continuité des fonctions à valeur 
complexes, on a~:
\begin{prop}
  Pour toute fonction \(f \in \C^{\intervalleff{a}{b}}\), \(f\) est continue 
  par morceaux si et seulement si \(\Re(f)\) et \(\Im(f)\) sont continues par 
  morceaux.
\end{prop}

On note \(\CM(\intervalleff{a}{b},\C)\) l'ensemble des fonctions de 
\(\C^{\intervalleff{a}{b}}\) qui sont continues par morceaux.

\begin{defdef}
  Pour toute fonction \(f \in \CM(\intervalleff{a}{b},\C)\), on appelle 
  intégrale de \(f\) sur \(\intervalleff{a}{b}\) et on note 
  \(\int_{\intervalleff{a}{b}}\), \(\int_a^b f\) ou encore \(\int_a^b 
  f(x)\,\diff x\) le complexe
  \begin{equation}
    \int_a^b f = \int_a^b \Re(f) +\ii \int_a^b \Im(f)
  \end{equation}
\end{defdef}

\begin{prop}
  \begin{equation}
    \forall f \in \CM(\intervalleff{a}{b},\C) \quad \int_a^b \bar{f} = 
    \overline{\int_a^b f}
  \end{equation}
\end{prop}

\subsection{Propriétés de l'intégrale}

\subsubsection{Linéarité}

L'application
\begin{equation}
  \fonction{\int_a^b}{\CM(\intervalleff{a}{b},\C)}{\C}{f}{\int_a^bf}
\end{equation}
est une forme linéaire sur le \(\C\)-espace vectoriel 
\(\CM(\intervalleff{a}{b},\C)\).

\begin{proof}
  Soient \((f,g) \in \CM(\intervalleff{a}{b},\C)^2\) et un complexe 
  \(\lambda\). On note
  \begin{align*}
    f_1 = \Re(f) \quad & f_2 = \Im(f) \\
    g_1 = \Re(g) \quad & g_2 = \Im(g) \\
  u = \Re(\lambda) \quad & v = \Im(\lambda) \end{align*}
  alors~:
  \begin{align*}
    \int_a^b (f+g) &= \int_a^b (f_1+g_1) + \ii \int_a^b (f_2+g_2) \\
    &=\int_a^b f_1 + \ii \int_a^b f_2 + \int_a^b g_1 + \ii \int_a^b g_2 \\
    &=\int_a^b f+ \int_a^b g
  \end{align*}
  et comme
  \begin{equation}
    \lambda f = (uf_1-vf_2)+ \ii(vf_1+uf_2)
  \end{equation}
  alors on a
  \begin{align*}
    \int_a^b(\lambda f) &= \int_a^b (uf_1-vf_2)+ \ii \int_a^b (cf_1+uf_2) \\
    &= (u+\ii v) \int_a^b f_1 + (\ii u-v) \int_a f_2\\
    &=\lambda \int_a^b f_1 + \ii \lambda \int_a^b f_2 = \lambda \int_a^b f
  \end{align*}
\end{proof}

\subsubsection{Relation de Chasles}

\begin{theo}
  Soient trois réels \(a\leqslant c \leqslant b\) et \(f \in 
  \C^{\intervalleff{a}{b}}\). Alors
  \begin{equation}
    f \in \CM(\intervalleff{a}{b},\C) \iff f_{|\intervalleff{a}{c}} \in 
    \CM(\intervalleff{a}{c},\C) \text{~et~} f_{|\intervalleff{c}{b}} \in 
    \CM(\intervalleff{c}{b},\C).
  \end{equation}
  Auquel cas
  \begin{equation}
    \int_a^b f = \int_a^c f + \int_c^b f.
  \end{equation}
\end{theo}

\begin{proof}
  L'équivalence est la conséquence immédiate de la propriété analogue avec les 
  fonctions à valeur réelles. La relation de Chasles est aussi la conséquence 
  immédiate de la relation de Chasles avec les fonctions à valeur réelles.
\end{proof}

\subsubsection{Inégalité de la moyenne}

\begin{theo}
  Pour toute fonction \(f \in \CM(\intervalleff{a}{b},\C)\), \(\abs{f} \in 
  \CM(\intervalleff{a}{b},\R)\) et
  \begin{equation}
    \abs{\int_a^b f~} \leqslant \int_a^b \abs{f}
  \end{equation}
\end{theo}
\begin{proof}
  On distingue deux cas, selon que \(\int_a^b f \in \Rpluss\) ou non.
  \begin{enumerate}
    \item Supposons que \(\int_a^b f \in \Rpluss\). Alors \(\abs{\int_a^b 
      f~}=\int_a^b f\) et donc \(\int_a^b \Im(f) =0\) et alors \(\int_a^b f = 
      \int_a^b \Re(f)\). On a
      \begin{equation}
        \Re(f) \leqslant \abs{f}
      \end{equation}
      et par croissance de l'intégrale
      \begin{equation}
        \abs{\int_a^b f~} = \int_a^b \Re(f) \leqslant \int_a^b \abs{f}
      \end{equation}
    \item Supposons maintenant que \(\int_a^b f \notin \Rpluss\) et en 
      particulier \(\int_a^b f \neq 0\). Alors il existe un réel \(\rho >0\) 
      et un réel \(\theta\) tels que
      \begin{equation}
        \int_a^b f = \rho \e^{\ii \theta}
      \end{equation}
      et alors par linéarité
      \begin{equation}
        \rho = \int_a^b \e^{-\ii \theta}f
      \end{equation}
      Soit \(g=\int_a^b \e^{-\ii \theta}f\) alors \(\int_a^b g=\rho \in 
      \Rpluss\). Donc on peut lui ppliquer le premier point~:
      \begin{align*}
        \abs{\int_a^b g} &\leqslant \int_a^b \abs{g} \\
        \abs{\int_a^b \e^{-\ii \theta} f~} &\leqslant \int_a^b \abs{\e^{-\ii 
        \theta} f} \\
        \abs{\int_a^b f} &\leqslant \int_a^b \abs{f} \\
      \end{align*}
  \end{enumerate}
\end{proof}

\danger \emph{La positivité et la croissance n'ont pas de sens pour \(f \in 
\C^{\intervalleff{a}{b}}\)}.
