\chapter{Opérations élémentaires sur les matrices}
\label{chap:operationssurlesmatrices}
\minitoc
\minilof
\minilot

Dans tout le chapitre, \(\K\) est un corps, sous-corps de \(\C\).

\section{Opérations élémentaires sur les matrices}

\subsection{Opérations élémentaires sur les lignes}

Soient deux naturels non nuls \(n\) et \(p\), et \(A \in \Mnp{n}{p}{\K}\). On note \(a_{ij}\) les coefficients de \(A\). Pour tout \(i \in \intervalleentier{1}{n}\), \(\L_i\) est la \(i\)\ieme{} ligne de \(A\), \(\L_i =(a_{i1}, \ldots, a_{ip}) \in \K^p\). On va décrire trois types d'opérations élémentaires sur leslignes et on va voir que chacune d'elle est équivalente à la multiplication à gauche de \(A\) par une matrice inversible.
%
\subsubsection{Préliminaire : multiplication de \(A\) par les matrices de la base canonique}
Soit \((j_0,h_0) \in \intervalleentier{1}{n}^2\). Notons \(E_{i_0h_0} \in \Mn{n}{\K}\) la matrice dont tous les coefficients sont nuls sauf celui de la ligne \(i_0\) et la colonne \(h_0\) qui vaut \(1\).

Notons \((E'_{ij})_{1\leqslant i \leqslant n, 1\leqslant j \leqslant p}\) la base canonique de \(\Mnp{n}{p}{\K}\). Alors
\begin{align}
  E_{i_0h_0} A &= E_{i_0h_0} \sum_{i=1}^n \sum_{j=1}^p a_{ij} E'_{ij} \\
  &= \sum_{i=1}^n \sum_{j=1}^p a_{ij} (E_{i_0h_0}E'_{ij})\\
  &= \sum_{i=1}^n \sum_{j=1}^p a_{ij} \delta_{h_0 i} E'_{i_0j} \\
  &= \sum_{i=1}^na_{h_0j} E'_{i_0j}.
\end{align}
\( E_{i_0h_0} A\) est la matrice à \(n\) lignes et \(p\) colonnes dans laquelle la ligne \(i_0\) est égale à la ligne \(h_0\) de \(A\). Toutes les autres lignes sont nulle.
\begin{equation}
  E_{i_0h_0} A =
  \begin{pmatrix}
    0 & \ldots & 0 \\
    a_{h_0 1}& \ldots & a_{h_0 p} \\
    0 & \ldots & 0
  \end{pmatrix}.
\end{equation}
%
\subsubsection{Addition d'un multiple d'une ligne à une autre}
Soient \(i\) et \(h\) deux entiers différents de \(\intervalleentier{1}{n}\) et un scalaire \(\lambda\). Il s'agit de remplacer la ligne \(L_i\) par la ligne \(L_i+\lambda L_h\). Cette opération élémentaire est notée \(L_i \leftarrow L_i+\lambda L_h\).
\begin{defdef}
  Pour tout scalaire \(\lambda\) et tous entiers \(i\) et \(h\) différents de \(\intervalleentier{1}{n}\), on définit la matrice de transvection \(T_{i,h}(\lambda)\) par
  \begin{align}
    T_{i,h}(\lambda) &= I_n +\lambda E_{ih}\\
    &= \begin{pmatrix}
      1 &        &        &         &   \\
        & \ddots &        & \lambda &   \\
        &        & \ddots &         &   \\
        &        &        &\ddots   &   \\
        &        &        &         & 1
      \end{pmatrix}.
  \end{align}
\end{defdef}
%
\begin{prop}
  Pour tout scalaire \(\lambda\) et tous entiers \(i\) et \(h\) différents de \(\intervalleentier{1}{n}\), la matrice \(T_{i,h}(\lambda)\) est inversible et \(T_{i,h}(\lambda)^{-1} = T_{i,h}(-\lambda)\).
\end{prop}
\begin{proof}
  \begin{align}
     T_{i,h}(-\lambda) \cdot T_{i,h}(\lambda) &= (I_n + \lambda E_{ih}) (I_n - \lambda E_{ih})\\
     &= I_n -\lambda E_{ih} + \lambda E_{ih} - \lambda^2 E_{ih}E_{ih}.
   \end{align}
   Comme \(i \neq h\) alors \(E_{ih}E_{ih}=0\). Finalement \(T_{i,h}(-\lambda) \cdot T_{i,h}(\lambda) = I_n\).
\end{proof}
%
\begin{prop}
  Pour tout scalaire \(\lambda\) et tous entiers \(i\) et \(h\) différents de \(\intervalleentier{1}{n}\),
  \begin{equation}
    L_i \leftarrow L_i+\lambda L_h \iff A \leftarrow T_{i,h}(\lambda) A.
  \end{equation}
\end{prop}
\begin{proof}
  On sait que
  \begin{equation}
    T_{i,h}(\lambda) A = (I_n + \lambda E_{ih})A= A + \lambda E_{ih}A.
  \end{equation}
  La matrice \(\lambda E_{ih}A \in \Mnp{n}{p}{\K}\) est la matrice dont toutes les lignes sont nulles sauf éventuellement la ligne qui contient \(\lambda\) fois la ligne \(h\) de \(A\). Ajouter cette matrice à \(A\) revient à ajouter \(\lambda L_h\) à la ligne \(L_i\) de \(A\).
\end{proof}
%
\subsubsection{Multiplication d'une ligne par un scalaire non nul}
Soient un réel \(\mu\) non nul et un entier \(i\) de \(\intervalleentier{1}{n}\). Il s'agit de remplacer la ligne \(L_i\) par la ligne \(\mu L_i\). Cette opération élémentaire est notée \(L_i \leftarrow \mu L_i\).
%
\begin{defdef}
  Pour tout réel non nul \(\mu\) et tout naturel \(i\) de \(\intervalleentier{1}{n}\), on définit la matrice de dilatation \(D_i(\mu)\) par
  \begin{align}
    D_i(\mu) &= I_n +(\mu-1)E_{ii} \\
    &=
    \begin{pmatrix}
      1 & & & & & & \\
      & \ddots & & & & & \\
      & & 1 & & & & \\
      & & & \mu & & & \\
      & & & & 1& & \\
      & & & & & \ddots & \\
      & & & & & &1
    \end{pmatrix}.
  \end{align}
\end{defdef}
%
\begin{prop}
 Pour tout réel non nul \(\mu\) et tout naturel \(i\) de \(\intervalleentier{1}{n}\), la matrice de dilatation \(D_i(\mu)\) est inversible et
 \begin{equation}
   D_i(\mu)^{-1} =  D_i(\mu^{-1}).
 \end{equation}
\end{prop}
\begin{proof}
  Pour tout réel non nul \(\mu\) et tout naturel \(i\) de \(\intervalleentier{1}{n}\), la matrice de dilatation \(D_i(\mu)\) est diagonale à coefficients tous non nuls. Alors elle s'inverse facilement.
\end{proof}
%
\begin{prop}
  Pour tout réel non nul \(\mu\) et tout naturel \(i\) de \(\intervalleentier{1}{n}\), on a
  \begin{equation}
    L_i \leftarrow \mu L_i \iff A \leftarrow D_i(\mu)A.
  \end{equation}
\end{prop}
\begin{proof}
  On sait que
  \begin{equation}
    D_i(\mu) A = (I_n + (\mu-1) E_{ii})A= A + (\mu-1)E_{ii}A.
  \end{equation}
  La matrice \((\mu-1)E_{ii}A \in \Mnp{n}{p}{\K}\) est la matrice dont toutes les lignes sont nulles sauf éventuellement la \(i\)\ieme{} ligne qui contient \((\mu-1)L_i\) (où \(L_i\) est la \(i\)\ieme{} ligne de \(A\)). Quand on l'ajoute à \(A\), on remplace la ligne \(L_i\) de \(A\) par \(L_i +(\mu-1) L_i=\mu L_i\).
\end{proof}
%
\subsubsection{Échange de deux lignes}
Soient deux entiers \(i\) et \(h\) différents dans \(\intervalleentier{1}{n}\). On veut échanger, dans la matrice \(A\), les lignes \(L_i\) et \(L_h\). Cette opération est notée \(L_i \leftrightarrow L_h\).
\begin{defdef}
  Pour tout entiers \(i\) et \(h\) différents dans \(\intervalleentier{1}{n}\), on définit la matrice de permutation \(P_{ih}\) telle que
  \begin{align}
    P_{ih} &= I_n - E_{ii}-E_{hh} +E_{ih}+E_{hi}\\
    &=\begin{pmatrix}
      1&        &  &        &   &        &        &   &   &        & \\
       & \ddots &  &        &   &        &        &   &   &        & \\
       &        &1 &        &   &        &        &   &   &        & \\
       &        &  & 0      &   & \ldots &        & 1 &   &        & \\
       &        &  &        & 1 &        &        &   &   &        & \\
       &        &  & \vdots &   & \ddots & \vdots &   &   &        & \\
       &        &  &        &   &        & 1      &   &   &        & \\
       &        &  & 1      &   & \ldots &        & 0 &   &        & \\
       &        &  &        &   &        &        &   & 1 &        & \\
       &        &  &        &   &        &        &   &   & \ddots & \\
       &        &  &        &   &        &        &   &   &        & 1
   \end{pmatrix}.
  \end{align}
\end{defdef}
%
\begin{prop}
  Pour tout entiers \(i\) et \(h\) différents dans \(\intervalleentier{1}{n}\), la matrice \(P_{ih}\) est inversible et
  \begin{equation}
    (P_{ih})^{-1}=P_{ih}.
  \end{equation}
\end{prop}
\begin{proof}
  \begin{align}
    P_{ih} \cdot P_{ih} &= (I_n - E_{ii}-E_{hh} +E_{ih}+E_{hi})\cdot (I_n - E_{ii}-E_{hh} +E_{ih}+E_{hi})\\
    &=I_n.
  \end{align}
\end{proof}
%
\begin{prop}
  Pour tout entiers \(i\) et \(h\) différents dans \(\intervalleentier{1}{n}\), on a
  \begin{equation}
    L_i \leftrightarrow L_h \iff A \leftarrow P_{ih}A.
  \end{equation}
\end{prop}
\begin{proof}
  En effet, le calcul donne
  \begin{equation}
    P_{ih}A = A - E_{ii}A-E_{hh}A+E_{ih}A+E_{hi}A.
  \end{equation}
  Les matrices \(E_{ii}A\) et \(E_{hh}A\) de \(\Mnp{n}{p}{\K}\) contiennent respectivement la ligne \(L_i\) en \(i\)\ieme{} ligne et la ligne \(L_h\) en \(h\)\ieme{} ligne et des zéros partout ailleurs. La matrice \(A - E_{ii}A-E_{hh}A\) est la matrice obtenue à partir de \(A\) en remplaçant \(L_i\) et \(L_h\) par des lignes nulles. On lui ajoute ensuite \(E_{ih}A+E_{hi}A\), c'est-à-dire les matrices qui contiennent \(L_h\) et \(L_i\) en position \(i\) et \(h\). On  a donc mis \(L_i\) à la place de \(L_h\) et vice-versa.
\end{proof}
%
\subsection{Manipulations élémentaires sur les colonnes}
Voir la fiche complémentaire. Les propositions et définitions sont équivalentes à celles de la première sous-section.
%
\section{Système d'équations linéaires}
\subsection{Vocabualires et interprétations}
\subsubsection{Définition}
\begin{defdef}
  Soient deux naturels non nuls \(n\) et \(p\). Le nombre \(n\) est le nombre d'équations du système (L) et le nombre \(p\) est le nombre d'inconnues de (L). Soit \((a_{ij})\) une famille d'éléments de \(\K\), appelée famille de coefficients du système (L). Soit \((b_i) \in \K^n\) la famille des seconds membres de (L). On appelle solution du système
  \begin{equation}
    (L)
    \begin{cases}
      a_{11}x_1 + \dotsb +a_{1p}x_p = b_1 \\
      \vdots \\
      a_{n1}x_1 + \dotsb +a_{np}x_p = b_n
    \end{cases},
  \end{equation}
  tout \(p\)-uplet \((x_i)_{1\leqslant i\leqslant p} \in \K^p\) qui vérifi les \(n\) équations de (L). On note \(E(L)\) l'ensemble des solutions de \((L)\). C'est un sous-ensemble éventuellement vide de \(\K^p\).
\end{defdef}
%
\begin{defdef}
  Si la famille des seconds membres est la famille nulle, on dit que le système linéaire est homogéne.
\end{defdef}
%
À tout système linéaire \((L)\), on associe le système homogène \((H)\) appelé système homogène associé à \((L)\).
\begin{equation}
  (H)
  \begin{cases}
    a_{11}x_1 + \dotsb +a_{1p}x_p = 0 \\
    \vdots \\
    a_{n1}x_1 + \dotsb +a_{np}x_p = 0
  \end{cases}.
\end{equation}
%
\subsubsection{Traduction matricielle}
Soient une matrice \(A=(a_{ij}) \in \Mnp{n}{p}{\K}\) et un vecteur \(B=(b_{i}) \in \Mnp{n}{1}{\K}\). Pour tout vecteur \(X=(x_i)_{1\leqslant i \leqslant p} \in \K^p\), on a
\begin{align}
  X \in E(L) &\iff AX=B, \\
  X \in E(H) &\iff AX=0.
\end{align}
%
\subsubsection{Traduction fonctionelle}
Soit \(u\) l'application linéaire canoniquement associée à \(A\), \(u \in \Lin{\K^p}{\K^n}\). On note \(b \in \K^p\) le vecteur de coordonnées \((b_1, \ldots, b_n)\) dans la base canonique. Pour tout vecteur \(x=(x_i)_{1\leqslant i \leqslant p} \in \K^p\), on a
\begin{align}
  x \in E(L) &\iff u(x)=B, \\
  x \in E(H) &\iff x\in\Ker(u).
\end{align}

\subsubsection{Traduction vectorielle}

Soit pour tout \(j \in \intervalleentier{1}{p}\) \(C_j\) la \(j\)\ieme{} colonne de \(A\). Pour tout vecteur \((x_1,\ldots,x_p) \in \K^p\), on a
\begin{align}
  (x_1,\ldots,x_p) \in E(L) &\iff x_1C_1+\dotsb+x_pC_p=B, \\
  (x_1,\ldots,x_p) \in E(H) &\iff x_1C_1+\dotsb+x_pC_p=0.
\end{align}
Cela signifie que \((x_1,\ldots,x_p)\) est une relation de dépendance linéaire de la famille des colonnes.

\subsubsection{Traduction duale}

Pour tout entier \(i\) de \(\intervalleentier{1}{n}\), on définit la forme linéaire 
\begin{equation}
\fonction{\varphi_i}{\K^p}{\K}{(x_1,\ldots,x_p)}{a_{i1}x_1+ \dotsb +a_{ip}x_p}.
\end{equation} 
 Pour tout vecteur \(x \in \K^p\), on a
\begin{align}
  x \in E(L) &\iff \forall i \in \intervalleentier{1}{n} \quad \varphi_i(x)=b_i, \\
  x \in E(H) &\iff x\in\bigcap_{i=1}^n\ker \varphi_i.
\end{align}

\subsection{Rang d'un système linéaire}

\begin{defdef}
  On appelle rang du système linéaire \((L)\) le rang de la matrice \(A\) associée au système linéaire \((L)\).
\end{defdef}
%
\begin{prop}
  Avec les notations de la sous-section précédente,
  \begin{equation}
    \rg(L)=\rg(A)=\rg(u)=\rg(C_1,\ldots,C_p)=\rg(H).
  \end{equation}
  Le rang d'un système linéaire \((L)\) est à la fois inférieur ou égal au nombre d'inconnues \(p\) et au nombre d'équations \(n\).
\end{prop}

\subsection{Description de l'ensemble des solutions}

Soit \((L)\) un système linéaire à \(n\) équations, \(p\) inconnues et de rang \(r\).

\subsubsection{Description de \(E(H)\)}

\begin{theo}
  \(E(H)\) est un sous-espace vectoriel de dimension \(p-r\).
\end{theo}
\begin{proof}
  Avec les notations ci-avant et par définition, \(E(H)=\Ker u\), alors c'est un sous-espace vectoriel. L'application \(u\) est linéaire et \(\rg u=r\) alors le théorème du rang donne \(\rg u +\dim \ker u = \dim \K^p\). C'est-à-dire \(\dim \ker u = p-r\).
\end{proof}
\begin{corth}
  Le système \((H)\) admet une solution non nulle si et seulement si \(p>r\).
\end{corth}

\subsubsection{Description de \(E(L)\)}

\begin{theo}
  Soit \(E(L)\) est vide, soit il est non vide et dans ce cas c'est un sous-espace affine de direction \(E(H)=\ker u\), de dimension \(p-r\).

  Si \(a \in E(L)\) alors \(E(L)=a+E(H)\).
\end{theo}
%
\begin{defdef}
  Lorsque \(E(L)\) est non vide, on dit que le système \((L)\) est compatible.
\end{defdef}
%
Un système homogène est toujours compatible.
%
\subsection{Cas particulier d'un  système de Cramer}
%
\begin{defdef}
  Le système linéaire \((L)\) est dit de Cramer si et seulement si
  \begin{enumerate}
  \item \(n=p\), il y a autant d'inconnues que d'équations;
  \item \(\rg(L)=n=p\).
  \end{enumerate}
\end{defdef}
Il en résulte
\begin{theo}
  \((L)\) est un système de Cramer si et seulement si la matrice associée à \((L)\) est inversible.
\end{theo}
%
\begin{theo}
  Tout système de Cramer admet une et une seule solution. En pratique, pour un système de Cramer homogéne, la solution unique est nulle.
\end{theo}
\begin{proof}
 La matrice \(A\) associée à \((L)\) est inversible, donc l'application \(u\) associée est bijective alors
 \begin{align}
   u(x)=b &\iff x=u^{-1}(b) \\
   u(x)=0 &\iff x=0.
 \end{align}
\end{proof}
%
\section{Applications des opérations élémentaires}
%

\subsection{Recherche du rang}
%
\begin{lemme}
  Soit une matrice \(A \in \Mnp{n}{p}{\K}\). On suppose que \(B\in \Mnp{n}{p}{\K}\) est obtenue à partir de \(A\) par une suite finie d'opérations élémentaires sur les lignes et les colonnes de \(A\). Alors \(\rg(A)=\rg(B)\).
\end{lemme}
\begin{proof}
  %Il existe des matrices inversibles \(P \in \GLn{p}{\K}\) et \(Q \in \GLn{n}{\K}\) telles que \(B=QAP\). 
Les matrices \(A\) et \(B\) représentent la même application linéaire \(u \in \Lin{\K^p}{\K^n}\) mais dans deux bases différentes. Alors \(\rg(A)=\rg(u)=\rg(B)\).
\end{proof}
\emph{Application}~: Lorsqu'on veut déterminer le rang d'une matrice \(A\), on peut effectuer des opérations élémentaires sur cette matrice jusqu'à obtenir une matrice dont le rang est facilement calculable (comme par exemple la matrice \(J_{n,p,r}\)).
%
\subsection{Pivot de Gauss}
%
\begin{lemme}
  Soit une matrice \(A \in \Mnp{n}{p}{\K}\) telle que
  \begin{equation}
    A =
    \begin{pmatrix}
      \alpha & * & \ldots & * \\
      0 & & & \\
      \vdots & & A' & \\
      0 & & & 
    \end{pmatrix}
  \end{equation}
  où \(\alpha \neq 0\) et \(A'\in \Mnp{n-1}{p-1}{\K}\). Alors \(\rg(A)=\rg(A')+1\).
\end{lemme}
\begin{proof}
  Soit \((L'_2, \ldots, L'_n)\) les lignes de \(A'\). Alors les lignes de \(A\) sont~:
  \begin{itemize}
  \item \(L_1=(\alpha, *, \ldots, *)\)
  \item pour tout \(i \in \intervalleentier{2}{n}\), \(L_i=(0,L'_i)\).
  \end{itemize}
  Ainsi
  \begin{equation}
    \VectEngendre(L_1) \cap \VectEngendre(L_2, \ldots, L_n)=\{0\},
  \end{equation}
  d'où
  \begin{equation}
    \dim \VectEngendre(L_2, \ldots, L_n) = \dim \VectEngendre(L_1) + \dim \VectEngendre(L_2, \ldots, L_n).
  \end{equation}
  La ligne \(L_1\) est non nulle donc \(\dim \VectEngendre(L_1)=1\). De plus \(\dim \VectEngendre(L_2, \ldots, L_n) = \rg(A')\). Alors finalement \(\rg(A)=\rg(A')+1\).
\end{proof}
%
\begin{prop}[Méthode du pivot de Gauss]
  Soit une matrice \(A \in \GLn{n}{\K}\). Il existe une suite d'opérations élémentaires sur les lignes qui permet de transformer \(A\) en une matrice triangualire supérieure \(T\) à coefficients diagonaux tous non nuls.
\end{prop}
\begin{proof}
  On démontre par réccurence sur \(n \in \N^*\) l'assertion \(\P(n)\) ``Pour toute matrice \(A \in \GLn{n}{\K}\), il existe une matrice \(T\) triangualire supérieure obtenue à partir de \(A\) à l'aide d'une suite d'opérations élémentaires sur les lignes de \(A\).''

\emph{Initialisation}~: \(n=1\) \(A\) est un scalaire alors elle est triangulaire supérieure.

\emph{Hérédité}~: Soit un naturel non nul \(n\). On suppose \(\P(n)\). Soit une matrice \(A \in \GLn{n+1}{\K}\). La première colonne de \(A\) est donc non nulle. Quitte à échanger deux lignes, on suppose que \(a_{11}\neq 0\). En effectuant, pour tout \(i \in \intervalleentier{2}{n}\), l'opération \(L_i \leftarrow L_i -\frac{a_{i1}}{a_{11}}L_1\) on obtient
\begin{equation}
  B = \begin{pmatrix}
      a_{11} & * & \ldots & * \\
      0 & & & \\
      \vdots & & B' & \\
      0 & & & 
    \end{pmatrix},
\end{equation}
avec \(a_{11}\neq 0\). D'après le lemme, \(\rg(B')=\rg(B)-1=n\). Alors \(B'\) est inversible et en lui appliquant \(\P(n)\) on écrit qu'il existe une suite finie d'opérations élémentaires sur les lignes de \(B'\) (par consquent su les lignes de \(B\)) qui permet de passer de \(B'\) à une matrice triangulaire \(T'\). Autrement dit de \(B\) à \(T\). Alors \(\P(n+1)\) est vraie.

\emph{Conclusion}~: On a montré que \(\P(1)\) est vraie et que pour tout naturel \(n\) non nul \(\P(n) \implies \P(n+1)\). Alors par théorème de récurrence, \(\P(n)\) est vraie pour tout naturel \(n\) non nul.
\end{proof}

\emph{Remarque}~: Dans l'hérédité, on utilise \(a_{11}\neq 0\) comme ``pivot'' pour annuler tous les autres coefficients de la première colonne. On peut appliquer cette méthode que la matrice soit inversible ou non. Dans tous les cas, on arrive à  une matrice triangulaire supérieure \(T\) de même rang que la matrice de départ. Si \(A\) est non invesible, il y aura des coefficients nuls dans la diagonale de \(T\).

\subsection{Inverser une matrice}
Soit \(A \in \GLn{n}{\K}\). D'après la sous-section précédente, il existe une matrice \(P \in \GLn{n}{\K}\) et une matrice \(T \in \Mn{n}{\K}\) triangulaire supérieure à coefficients diagonaux tous non nuls telles que \(T=PA\).

Comme \(t_{nn}\) est non nul, on peut l'uiliser comme pivot pour annuler tous les autres coefficients de la dernière colonne~:
\begin{equation}
  \forall i \in \intervalleentier{1}{n-1} \quad L_i \leftarrow L_i - \frac{t_{in}}{t_{nn}}L_n.
\end{equation}
On prend comme pivots successifs \(t_{n-1 n-1}\), \ldots, \(t_{22}\) pour arriver à la matrice diagonale des coefficients \(t_{ii}\). Il suffit ensuite de faire l'opération \(L_i \leftarrow L_i \frac{1}{t_{ii}}\) pour tout \(i \in \intervalleentier{1}{n}\) pour arriver à \(I_n\).

On a donc montré l'existence de \(P \in \GLn{n}{\K}\) telle que \(I_n=PA\). Ce n'est pas surprenant puisque \(A\) es inversible. Mais \(P\) est le produit des matrices correspondants à chacune des opérations élémentaires effectuées.

En pratique, on écrit côte à côte la matrice \(A\) (à inverser) et la matrice identité \(I_n\). On effectue en parallèle les opérations élémentaires sur les lignes qui permettent de passer de \(A\) à \(I_n\).

\emph{Exemple}~: Il s'agit d'inverser \(A =\begin{pmatrix}  1 & 0 & -1 \\ 0 & -1 & 1\\ -1 & 2 & 0\end{pmatrix}\).
\begin{align}
  \begin{pmatrix}  1 & 0 & -1 \\ 0 & -1 & 1\\ -1 & 2 & 0\end{pmatrix} & \begin{pmatrix}  1 & 0 & 0 \\ 0 & 1 & 0\\ 0 & 0 & 1\end{pmatrix} \\
  \begin{pmatrix}  1 & 0 & -1 \\ 0 & -1 & 1\\ 0 & 2 & -1\end{pmatrix} & \begin{pmatrix}  1 & 0 & 0 \\ 0 & 1 & 0\\ 1 & 0 & 1\end{pmatrix} && L_3 \leftarrow L_3+L_1\\
  \begin{pmatrix}  1 & 0 & -1 \\ 0 & -1 & 1\\ 0 & 0 & 1\end{pmatrix} & \begin{pmatrix}  1 & 0 & 0 \\ 0 & 1 & 0\\ 1 & 2 & 1\end{pmatrix} && L_3 \leftarrow L_3+2L_2 \\
  \begin{pmatrix}  1 & 0 & -1 \\ 0 & -1 & 1\\ 0 & 0 & 1\end{pmatrix} & \begin{pmatrix}  1 & 0 & 0 \\ -1 & -1 & -1\\ 1 & 2 & 1\end{pmatrix} && L_2 \leftarrow L_2-L_3\\
%\end{align}
%\begin{align}
  \begin{pmatrix}  1 & 0 & 0 \\ 0 & -1 & 0\\ 0 & 0 & 1\end{pmatrix} & \begin{pmatrix}2 & 2 & 1 \\ -1 & -1 & 1\\ 1 & 2 & 1 \end{pmatrix} && L_1 \leftarrow L_1+L_3\\
  \begin{pmatrix}  1 & 0 & 0 \\ 0 & 1 & 0\\ 0 & 0 & 1\end{pmatrix} & \begin{pmatrix} 2 & 2 & 1 \\ 1 & 1 & 1\\ 1 & 2 & 1 \end{pmatrix} && L_2 \leftarrow -L_2
\end{align}

On vient de montrer que \(A\) est inversible et que \(A^{-1}=\begin{pmatrix} 2 & 2 & 1 \\ 1 & 1 & 1\\ 1 & 2 & 1 \end{pmatrix}\).

\subsection{Résoudre un système d'équations linéaire}

La méthode du pivot de Gauss permet de résoudre un système de Cramer d'écriture matricielle \(AX=B\). C'est un système de Cramer donc \(A\) est inversible. Par des opérations élémentaires sur les lignes, on peut se ramener à un système équivalent dont la matrice est triangulaire supérieure à coefficients diagonaux tous non nuls que l'on sait résoudre de proche en proche~:
\begin{equation}
  \exists P \in \GLn{n}{\K} \ \exists T \in T_n^s(\K) \cap \GLn{n}{\K} \ T=PA \quad AX=B \iff TX=PB.
\end{equation}

Si le système n'est pas de Cramer, deux cas se présentent~:

\emph{Premier cas}~: Si \(r=n < p\). En appliquant la méthode du pivot de Gauss, on arrive au système de la forme~:
\begin{equation}
  \begin{cases}
    \alpha_{11}x_{i_1} + \dotsb + \alpha_{1p}x_{i_p} &=\beta_1 \\
    \alpha_{22}x_{i_2} + \dotsb + \alpha_{1p}x_{i_p} &=\beta_2 \\
    \vdots & \vdots \\
    \alpha_{rr}x_{i_r} + \dotsb  &=\beta_r
  \end{cases}.
\end{equation}
Ce système admet des solutions (il n'est pas compatible). On peut choir librement les \(p-r\) inconnues \(x_{i_{r+1}}, \ldots, x_{i_p}\) et exprimer toutes les autres en fonction de celles-ci en remontant le système.

\emph{Deuxième cas}~: Si \(r<n\), la méthode du pivot de Gauss conduit à un système équivalent de la forme~:
\begin{equation}
  \begin{cases}
    \alpha_{11}x_{i_1} + \dotsb + \alpha_{1p}x_{i_p} &=\beta_1 \\
    \vdots & \vdots \\    
    \alpha_{rr}x_{i_r} + \dotsb  &=\beta_r\\
    0 &=\beta_{r+1} \\
    \vdots & \vdots \\
    0 &=\beta_{n}
  \end{cases}.
\end{equation}
Soit \(\beta_{r+1}=\ldots=\beta_{n}=0\) est vraie et on est ramené au cas précédent, soit ce n'est pas vérifié et le système est incompatible.

\emph{Exemple}~: On veut résoudre le système linéaire suivant
\begin{equation}
  \begin{cases}
    2x-y+3z-t&=1\\
    x+y+z+t&=0\\
    x-4y-z-4t&=3
  \end{cases},
\end{equation}
en inversant \(L_1\) et \(L_2\) on a
\begin{equation}
  \begin{cases} 
    x+y+z+t&=0\\
    2x-y+3z-t&=1\\
    x-4y-z-4t&=3
  \end{cases},
\end{equation}
ensuite en effectuant \(L_2 \leftarrow L_2-2L_1\) et \(L_3 \leftarrow L_3-L_1\) on obtient
\begin{equation}
  \begin{cases} 
    x+y+z+t&=0\\
    -3y+z-3t&=1\\
    -5y-2z-5t&=3
  \end{cases},
\end{equation}
en effectuant \(L_3 \leftarrow L_3+2L_2\) on obtient
\begin{equation}
  \begin{cases} 
    x+y+z+t&=0\\
    -3y+z-3t&=1\\
    -11y-11t&=5
  \end{cases}.
\end{equation}
En ``remontant'' le système on arrive à
\begin{equation}
  \begin{cases}
    x &= \frac{9}{11} \\
    z &= -\frac{4}{11}\\
    y &= -t -\frac{5}{11}
  \end{cases},
\end{equation}
et l'ensemble des solutions de ce système est
\begin{equation}
  S = \left(\frac{9}{11}, -\frac{5}{11}, -\frac{4}{11}, 0\right) + \VectEngendre(0,-1,0,1).
\end{equation}
