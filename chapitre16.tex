\chapter{Espaces vectoriels}
\label{chap:espaceVectoriel}
\minitoc
\minilof
\minilot

Dans tout ce chapitre, \((\K,+,\cdot)\) est un corps, en pratique \(\K=\R\) ou \(\C\), et en général \(E\) désigne un \(\K\)-espace vectoriel.

\section{Espaces vectoriels, sous-espaces vectoriels}

\subsection{Structure d'espace vectoriel}

\begin{defdef}
  Soit \(E\) un ensemble. On appelle loi de composition externe sur \(E\) à opérateurs dans le corps \(\K\) toute application \(\fonction{\perp}{\K \times E}{E}{(\lambda,x)}{\lambda x}\).
\end{defdef}
\begin{defdef}
  On appelle espace vectoriel sur le corps \(\K\) tout triplet \((E,+,\perp)\) où
  \begin{enumerate}
  \item \((E,+)\) est un groupe abélien;
  \item \(\perp\) est une loi de composition externe sur \(E\) à opérateurs dans \(\K\);
  \item vérifiant les propriétés suivantes~:
    \begin{gather}
      \forall (\lambda,\lambda') \in \K^2 \ \forall x \in E \quad (\lambda+\lambda')x=\lambda x + \lambda' x; \\
      \forall \lambda \in \K \ \forall (x,x)' \in E^2 \quad \lambda(x+x')=\lambda x + \lambda x';\\
      \forall (\lambda, \mu) \in \K^2 \ \forall x \in E \quad \lambda(\mu x)=(\lambda \mu) x;\\
      \forall x \in E \quad 1x=x.
    \end{gather}
  \end{enumerate}
\end{defdef}

Les éléments du corps \(\K\) sont les scalaires et les éléments de \(E\) sont les vecteurs. Comme \((E,+)\) est un groupe abélien, on dispose d'un neutre pour la loi \(+\), noté \(0\), appelé le vecteur nul. Attention à ne pas confondre l'addition vectorielle et l'addition scalaire. Il ne faut pas confondre non plus la multiplication interne dans le corps \(\K\) et la multiplication externe \(\perp\). Il n'y a pas de multiplication interne dans un espace vectoriel. Cela n'a aucun sens, dans un espace vectoriel, de multiplier les vecteurs entre eux. Il ne faut pas non plus ajouter un scalaire et un vecteur. Comme \((E,+)\) est un groupe, il est non vide. Alors un espace vectoriel n'est jamais vide.

\subsection{Premiers exemples}
\subsubsection{Propositions préliminaires}
\begin{prop}
Soit \(E\) un \(\K\)-espace vectoriel, soit \(\L\) un sous-corps de \(\K\). Alors \(E\) est un \(\L\)-espace vectoriel. Plus particuliérement, tout \(\C\)-espace vectoriel est un \(\R\)-espace vectoriel.
\end{prop}
\begin{proof}
  \((E,+)\) est un groupe abélien. \(\L\) est un sous-corps de \(\K\). On peut donc considérer comme loi externe à opérateurs dans \(\L\) la loi induite par celle à opérateurs dans \(\K\). Les propriétés de la définition sont respectées dans \(\K\) alors elles le sont dans \(\L\).
\end{proof}
\begin{prop}
  Tout corps est un espace vectoriel sur lui-même.
\end{prop}
\begin{proof}
  \((\K,+)\) est un groupe abélien. On prend comme loi externe à opérateurs dans \(\K\) la multiplication de \(\K\). Comme c'est un corps, la loi \(\cdot\) est distributive sur la loi \(+\). la loi \(\cdot\) est associative. L'élément \(1\) est neutre pour la multiplication dans \(\K\).
\end{proof}
\begin{cor}
  Tout corps est un espace vectoriel sur chacun de ses sous-corps.
\end{cor}

\subsubsection{Espace vectoriel produit}

Soient deux \(\K\)-espace vectoriels \((E_1,+_1,\perp_1)\) et \((E_2,+_2,\perp_2)\). On munit le produit cartésien \(E_1 \times E_2\)
\begin{itemize}
\item d'une loi de composition interne noté \(+\) définie par
\begin{equation}
  \forall (x_1, y_1)\in E_1^2 \ \forall (x_2, y_2) \in E_2^2 \quad (x_1,x_2) + (y_1,y_2) = (x_1+y_1,x_2+y_2);
\end{equation}
\item d'une loi de composition externe \(\perp\) à opérateurs dans \(\K\) définie par
  \begin{equation}
    \forall \lambda \in \K \ \forall (x_1,x_2) \in E^2 \quad \lambda(x_1,x_2)=(\lambda x_1, \lambda x_2).
  \end{equation}
\end{itemize}
Ainsi \((E_1 \times E_2,+,\perp)\) est un espace vectoriel appelé espace vectoriel produit.

\subsubsection[Espace vectoriel des applications]{Espace vectoriel des applications d'un ensemble \(X\) vers un espace vectoriel \(E\)}

Soient \(X\) un ensemble non vide et \((E,+_E,\perp_E)\) un \(\K\)-espace vectoriel. On s'intéresse à l'ensemble des applications de \(X\) vers \(E\), c'est-à-dire \(E^X\). Soient deux fonctions \(f\) et \(g\) de \(E^X\). On définit la fonction \(f+g\) par
\begin{equation}
  \forall x \in E \quad (f+g)(x)=f(x)+_Eg(x).
\end{equation}
Soit l'application `` multiplication par un scalaire'' \(\fonction{\perp}{\K\times E^X}{E^X}{(\lambda,f)}{\lambda f}\) telle que
\begin{equation}
  \forall x \in E \ \forall \lambda \in \K \quad (\lambda f)(x)=\lambda f(x).
\end{equation}
Ainsi \((E^X,+,\perp)\) obtient une structure de \(\K\)-espace vectoriel.

\subsection{Règles de calcul dans un espace vectoriel}
Soit \((E,+,\perp)\) un espace vectoriel sur \(\K\).
\begin{prop}
  Soient un naturel \(n\), \(\lambda, \lambda', \lambda_1, \ldots, \lambda_n\) des scalaires de \(\K\) et \(x,x',x_1, \ldots, x_n\) des vecteurs de \(E\). Alors on a
\begin{table}[!h]
  \centering
  \begin{tabular}{|c|c|} \hline
    A & B \\ \hline
    \(0 x = 0\) & \(\lambda 0 = 0\) \\
    \((-\lambda)x=-(\lambda x)\) & \(\lambda(-x)=-(\lambda x)\)\\
    \((\lambda-\lambda')x=\lambda x- \lambda' x\) & \(\lambda(x-x')=\lambda x- \lambda x'\)\\
    \(\left(\sum_{i=1}^n \lambda_i \right)x=\sum_{i=1}^n \lambda_i x\) & \(\lambda\left(\sum_{i=1}^n x_i\right)=\sum_{i=1}^n \lambda x_i\)\\
    \((k\lambda)x=k(\lambda x)\) & \(\lambda (kx)=k(\lambda x)\) \\
  \hline\end{tabular}
\end{table}
\end{prop}
\begin{proof}
   On définit pour tout vecteur \(x\) de \(E\) l'application \(\fonction{f_x}{\K}{E}{\lambda}{\lambda x}\). C'est un morphisme de groupes de \((\K,+)\) dans \((E,+)\). C'est-à-dire que
  \begin{equation}
    \forall (\lambda,\lambda') \in \K^2 \quad f_x(\lambda+\lambda')=f_x(\lambda)+f_x(\lambda').
  \end{equation}
  Alors les propriétés de la colonne A découlent des propriétés des morphismes de groupes.

  De la même manière on définit pour tout scalaire \(\lambda\) de \(\K\) l'application \(\fonction{h_\lambda}{E}{E}{x}{\lambda x}\). C'est un endomorphisme du groupe \((E,+)\). C'est-à-dire que
  \begin{equation}
    \forall (x,x') \in E^2 \quad h_\lambda(x+x')=h_\lambda(x)+h_\lambda(x').
  \end{equation}
  Alors les propriétés de la colonne B découlent des propriétés des morphismes de groupes.
\end{proof}

\begin{prop}
  Pour tout scalaire \(\lambda\) et tout vecteur \(x\) on a
  \begin{equation}
    \lambda x = 0 \iff \lambda = 0 \text{~ou~} x=0.
  \end{equation}
\end{prop}
\begin{proof}
  Si \(x=0\) ou \(\lambda=0\) alors d'après ce qui précéde on a \(\lambda x=0\).

  Si maintenant \(\lambda \neq 0\), alors il est inversible dans \(\K\) et \(\lambda^{-1}\) existe. Alors d'une part
  \begin{equation}
    \lambda^{-1}(\lambda x)=\lambda 0 =0
  \end{equation}
  et d'autre part
  \begin{equation}
    \lambda^{-1}(\lambda x)=1 x=x.
  \end{equation}
  Alors \(x=0\). On a montré que si \(\lambda x=0\) alors soit \(\lambda=0\) ou soit \(x=0\).
\end{proof}

Soit \(\lambda \in \K^*\). Comme la fonction qu'on avait définit \(h_\lambda\) est un endomorphisme du groupe \((E,+)\), on dispose de son noyau \(\Ker(h_\lambda)\) et de son image \(\Image(h_\lambda)\). Alors
\begin{equation}
  \Ker(h_\lambda)=\enstq{x \in E}{h_\lambda(x)=0}=\{0\},
\end{equation}
alors \(h_\lambda\) est injective. L'image vaut
\begin{equation}
  \Image(h_\lambda)=\enstq{y \in E}{\exists x \in E \quad y=h_\lambda(x)=\lambda x} = E,
\end{equation}
donc \(h_\lambda\) est surjective.

Au final c'est un endomorphisme bijectif de \(E\), donc un endomorphisme et un isomorphisme de \(E\). Ainsi \(h_\lambda\) est un automorphisme de \(E\).

\begin{prop}
  Pour tous scalaires \(\lambda\) et \(\lambda'\) de \(\K\) et tous vecteurs \(x\) et \(x'\) de \(E\) on a
  \begin{align}
    \lambda x = \lambda x' &\iff \lambda=0 \text{~ou~} x=x'; \\
    \lambda x = \lambda' x &\iff x=0 \text{~ou~} \lambda=\lambda'.
  \end{align}
\end{prop}
\begin{proof}
  Supposons que \(\lambda x=\lambda x'\) on a les équivalences suivantes
  \begin{align}
    \lambda x=\lambda x' &\iff \lambda x-\lambda x' = 0\\
    &\iff \lambda(x-x')=0\\
    &\iff \lambda = 0 \text{~ou~} x-x'=0\\
    &\iff \lambda = 0 \text{~ou~} x=x'.
  \end{align}
  Supposons que \(\lambda x=\lambda' x\) on a les équivalences suivantes
  \begin{align}
    \lambda x=\lambda' x &\iff \lambda x-\lambda' x= 0\\
    &\iff (\lambda-\lambda')x=0\\
    &\iff x = 0 \text{~ou~} \lambda-\lambda'=0\\
    &\iff x = 0 \text{~ou~} \lambda=\lambda'.
  \end{align}
\end{proof}

\subsection{Sous-espaces vectoriels}

\subsubsection{Définition et caractérisation}

\begin{defdef}
  Soit un \(\K\)-espace vectoriel \((E,+_E,\perp_E)\) et \(E'\) une partie de \(E\). On dit que \(E'\) est un sous-espace vectoriel de \(E\) si \(E'\) est stable par \(+_E\) et \(\perp_E\) et si, muni des lois induites, \((E',+_{E'}, \perp_{E'})\) est un \(\K\)-espace vectoriel.
\end{defdef}
\begin{theo}[Caractérisation des sous-espaces vectoriels]
  Il y a équivalence entre les assertions suivantes
  \begin{enumerate}
  \item \(E'\) est un sous-espace vectoriel de \(E\);
  \item \(E'\) est une partie de \(E\) non vide stable pour \(+\) et \(\perp\);
  \item \(E'\) est une partie de \(E\) non vide telle que
    \begin{equation}
      \label{eq:bidul11}
      \forall (x,y) \in E'^2 \ \forall (\lambda, \mu) \in \K^2 \quad \lambda x+\mu y \in E' ;
    \end{equation}
  \item \(E'\) est une partie de \(E\) non vide telle que
    \begin{equation}
      \label{eq:bidul12}
      \forall (x,y) \in E'^2 \ \forall \lambda \in \K \quad \lambda x+ y \in E' ;
    \end{equation}
  \item \((E',+,\perp)\) est un sous-groupe de \((E,+)\) et \(E'\) est stable par \(\perp\).
  \end{enumerate}
\end{theo}
\begin{proof}
  On montre ces équivalences de manière circulaire~:
  \begin{itemize}
  \item[\(1 \implies 2\)] C'est la définition d'un sous-espace vectoriel;
  \item[\(2 \implies 3\)] on a déjà \(E' \neq \emptyset\) et \(E' \subset E\) et l'équation~\eqref{eq:bidul11} est la découle de la définition de la stabilité par \(\perp\) et de la stabilité par \(+\);
  \item[\(3 \implies 4\)] on a déjà \(E' \neq \emptyset\) et \(E' \subset E\) et l'équation~\eqref{eq:bidul12} découle de l'équation~\eqref{eq:bidul11} en prenant \(\mu=1\);
  \item[\(4 \implies 5\)] on a déjà \(E' \neq \emptyset\) et \(E' \subset E\) et on prend dans~\eqref{eq:bidul12} \(\lambda=-1\) et on obtient
    \begin{equation}
      \forall (x,y) \in E'^2 \quad y-x \in E',
    \end{equation}
    alors par caractérisation \(E'\) est un sous-groupe de \(E\); comme \(E'\) est un sous-groupe, \(0 \in E'\) et
    \begin{equation}
      \forall x \in E' \ \forall \lambda \in \K \quad \lambda x +0=\lambda x \in E',
    \end{equation}
    et donc \(E'\) est stable par \(\perp\).
  \item[\(5 \implies 1\)] on a déjà \(E' \subset E\) et \(E'\) stable par \(\perp\) et par \(+\) (puisque c'est un sous-groupe de \((E,+)\)). On peut considérer \(E'\) muni des lois induites et on a
    \begin{itemize}
    \item \((E',+_{E'})\) est un groupe abélien puisque c'est un sous-groupe du groupe abélien \((E,+_E)\);
    \item la loi de composition externe est la loi induite par celle de \(E\);
    \item les propriétés du troisième point de la définition d'un espace vectoriel sont vérifiées sur \(E'\) puisqu'elles le sont sur \(E\) et on considère les lois induites;
    \end{itemize}
    donc \((E',+_E,\perp_E)\) est un sous-espace vectoriel de \(E\).
  \end{itemize}
\end{proof}

\subsubsection{Exemples}
\label{subsec:exemples}

Si \(E\) est un \(\K\)-espace vectoriel, \(\{0\}\) et \(E\) sont les sous-espaces vectoriels triviaux de \(E\). Si \(E_1\) est un sous-espace vectoriel d'un \(\K\)-espace vectoriel \(E\) et si \(E_2\) est un sous-espace vectoriel de \(E_1\) alors \(E_2\) est un sous-espace vectoriel de \(E\). L'ensemble des suites réelles bornées \(B(\R)\) est un sous-espace vectoriel de l'espace vectoriel des suites réelles \(\R^\N\). Soit un intervalle réel \(I\), l'ensemble des fonction dérivables sur \(I\), \(\deriv{I}{\R}\), est un sous-espace vectoriel du \(\R\)-espace vectoriel des fonctions de \(I\) vers \(\R\), \(\R^I\). Idem pour \(\cont{I}{\R}\), pour \(\forall n \in \N^* \ \classe{n}(I,\R)\), et pour \(\classe{\infty}(I,\R)\). \(\R\) est un sous-espace vectoriel du \(\R\)-espace vectoriel \(\C\)~:
\begin{itemize}
\item \(\R \subset \C\);
\item \(\R \neq \emptyset\);
\item \(\forall (x,y) \in \R^2 \ \forall \lambda \in \R \quad \lambda x +y \in \R\).
\end{itemize}

Mais \(\R\) n'est pas un sous-espace vectoriel du \(\C\)-espace vectoriel \(\C\).

\section{Applications linéaires}

\subsection{Notion d'application linéaire}

\subsubsection{Définition}

\begin{defdef}
  Soient deux \(\K\)-espaces vectoriel \((E,+_E,\perp_E)\) et \((F,+_F,\perp_F)\). On appelle application linéaire de \(E\) vers \(F\) tout morphisme \(u\) du groupe \((E,+_E)\) sur le groupe \((F,+_F)\) tel que
  \begin{equation}
    \forall x \in E \ \forall \lambda \in \K \quad u(\lambda x)=\lambda u(x).
  \end{equation}
Autrement dit
\begin{equation}
  \forall (x,x') \in E^2 \ \forall \lambda \in \K \quad u(\lambda x +x')=\lambda u(x)+u(x').
\end{equation}
\end{defdef}

\subsubsection{Vocabulaire}

Soient \(E\) et \(F\) deux \(K\)-espaces vectoriels et \(u\) une application linéaire de \(E\) vers \(F\). On note \(\Lin{E}{F}\) l'ensemble des applications linéaires de \(E\) dans \(F\). Si \(E=F\), \(+_E=+_F\) et \(\perp_E=\perp_F\) on dira que \(u\) est un endomorphisme linéaire de \(E\) ou encore opérateur linéaire de \(E\). On note \(\Endo{E}\) leur ensemble. Si \(u\) est bijective, on dira que \(u\) est un isomorphisme linéaire de \(E\) dans \(F\). On dira alors que \(E\) et \(F\) sont isomorphes si et seulement s'il existe un isomorphisme de l'un vers l'autre. On notera \(\Isom{E}{F}\) leur ensemble. Si \(u\) est un endomorphisme linéaire et un isomorphisme linéaire, on dira que c'est un automorphisme linéaire de \(E\) et on note \(\GL{E}\) leur ensemble.

\subsubsection{Propriétés}

Soient \(E\) et \(F\) deux \(K\)-espaces vectoriels et \(u \in \Lin{E}{F}\).

\begin{prop}
  \(u\) est un morphisme du groupe \((E,+_E)\) vers \((F,+_F)\), c'est-à-dire que
  \begin{equation}
    \forall (x,x') \in E^2 \quad u(x+_E x')=u(x)+_F u(x').
  \end{equation}
  Alors \(u\) vérifie les égalités suivantes
  \begin{enumerate}
  \item \(u(0)=0\);
  \item \(\forall x \in E \quad u(-x)=-u(x)\);
  \item \(\forall (x,x') \in E^2 \quad u(x-x')=u(x)-u(x')\);
  \item \(\forall n \in \N^* \forall (x_1, \ldots, x_n) \in E^n \quad u\left(\sum_{i=1}^n x_i \right) = \sum_{i=1}^n u(x_i)\).
  \end{enumerate}
  De plus on dispose de l'image et du noyau de \(u\)~:
  \begin{gather}
    \Image(u) = \enstq{y \in  F}{\exists x \in E \quad y=f(x)};\\
    \Ker(u)=\enstq{x \in E}{u(x)=0}.
  \end{gather}
\end{prop}
%
\begin{prop}
  \begin{align}
    u \text{~est injective} &\iff \Ker(u)=\{0\}; \\
    u \text{~est surjective} &\iff \Image(u)=F.
  \end{align}
\end{prop}
Si \(E_1\) est un sous-espace vectoriel de \(E\), on peut considérer la restriction \(\fonction{u_1}{E_1}{F}{x}{u(x)}\). Alors \(u \in \Lin{E_1}{F}\), \(\Ker(u_1)=E_1 \cap \Ker(u)\) et \(\Image(u_1)=u(E_1)\).
\begin{prop}
  Soient un naturel \(n\) non nul, \(x_1, \ldots, x_n\) des vecteurs de \(E\) et \(\lambda_1, \ldots, \lambda_n\) des scalaires de \(\K\). On dispose de \(\sum_{i=1}^n \lambda_i x_i \in E\). Alors
  \begin{equation}
    u\left(\sum_{i=1}^n \lambda_i x_i \right)= \sum_{i=1}^n \lambda_i u(x_i).
  \end{equation}
\end{prop}

\subsubsection{Exemples}

Soient un \(\K\)-espace vectoriel \((E,+_E,\perp_E)\) et \(E_1\) un sous-espace vectoriel de \(E\). On définit \(\fonction{j}{E_1}{E}{x}{x}\) appelé injection canonique de \(E_1\) dans \(E\) et \(j \in \Lin{E_1}{E}\). Soient \(E\) et \(F\) deux \(\K\)-espaces vectoriels. On dispose de l'espace vectoriel produit \(E \times F\) muni des lois produits. On définit \(\fonction{\Pi}{E\times F}{E}{(x,y)}{x}\), et alors \(\Pi \in \Lin{E}{F}\) et \(\Pi\) est surjective \(\forall x \in E \quad x=\Pi(x,0)\).

\paragraph{Homothétie}
Soit \(E\) un \(\K\)-espace vectoriel. Pour tout scalaire \(\lambda\), on définit l'application \(\fonction{h_\lambda}{E}{E}{x}{\lambda x}\). Ces fonctions vérifient les propositions suivantes
\begin{prop}
  Pour tout scalaire \(\lambda\), \(h_\lambda \in \Endo{E}\).
\end{prop}
\begin{proof}
  On a déjà vu que c'est un endomorphisme du groupe \((E,+)\) (cf.\ la sous-section sur les régles de calculs dans un espace vectoriel). Il reste à vérifier que
  \begin{equation}
    \forall x \in E \ \forall \alpha \in \K \quad h_\lambda(\alpha x)=\alpha h_\lambda(x)
  \end{equation}
  En effet, grâce à la pseudo-associativité et à la commutativité de la multiplication dans le corps \(\K\), on a pour tout scalaire \(\alpha\) et tout vecteur \(x\)
  \begin{equation}
    h_\lambda(\alpha x) = \lambda (\alpha x)= \alpha (\lambda x)=\alpha h_\lambda(x).
  \end{equation}
  Alors \(h_\lambda \in \Endo{E}\).
\end{proof}
\begin{prop}[Commutativité des homothéties]
  Pour tous scalaire \(\lambda\) et \(\mu\), on a
  \begin{equation}
    h_\lambda \circ h_\mu = h_{\lambda\mu}=h_\mu \circ h_\lambda.
  \end{equation}
\end{prop}
\begin{proof}
  Soit un vecteur \(x\), alors
  \begin{align}
    h_\lambda \circ h_\mu(x)&= \lambda(\mu x)) \\
    &=(\lambda\mu)(x)\\
    &= h_{\lambda\mu}(x).
  \end{align}
  Alors \(h_\lambda \circ h_\mu = h_{\lambda\mu}\). Comme la multiplication est commutative sur \(\K\), on peut faire commuter les deux applications \(h_\mu \circ h_\lambda=h_\lambda \circ h_\mu \).
\end{proof}
\begin{prop}
  Pour tout scalaire \(\lambda\) non nul, \(h_\lambda\) est bijective et \((h_\lambda)^{-1}=h_{\lambda^{-1}}\).
\end{prop}
\begin{proof}
  Le scalaire \(\lambda\) est non nul donc son inverse existe et d'après la proposition précédente,
  \begin{equation}
    h_{\lambda^{-1}} \circ h_\lambda = h_\lambda \circ h_{\lambda^{-1}} = h_1=\Id.
  \end{equation}
\end{proof}

\paragraph{Dérivation}
Soit un intervalle réel \(I\). On appelle application de dérivation l'application telle que \(\fonction{\Dr}{\Dr(I,\R)}{\R^I}{f}{f'}\). Cette application est linéaire,
\begin{equation}
  \forall (f,g) \in \deriv{I}{\R}^2 \ \forall \alpha \in \R \quad \Dr(\alpha f+g)= \alpha \Dr(f)+\Dr(g).
\end{equation}

\subsection{Images directe et réciproque d'un sous-espace vectoriel par une application linéaire}

\subsubsection{Image directe}

\begin{theo}
  Soient \(E\) et \(F\) deux espaces vectoriels, \(u \in \Lin{E}{F}\), \(E_1\) un sous-espace vectoriel de \(E\). Alors l'image directe \(u(E_1)\) est un sous-espace vectoriel de \(F\).
\end{theo}

\begin{proof}
  Déjà \(u(E_1)\) est inclus dans \(F\) par définition et il n'est pas vide, car \(0 \in E_1\) et \(u\) est linéaire donc \(u(0_E)=0_F \in u(E_1)\). Soient deux vecteurs \(x,y\) de \(u(E_1)\) et un scalaire \(\lambda\). Alors il existe \(z\) et \(t\) dans \(E_1\) tels que \(x=u(z)\) et \(y=u(t)\). Comme \(u\) est linéaire on a
  \begin{equation}
    \lambda x +y = \lambda u(z)+u(t)=u(\lambda z+t).
  \end{equation}
  Comme \(z\) et \(t\) sont dans \(E_1\) et que c'est un sous-espace vectoriel on a \(\lambda z +t \in E_1\). Ainsi \(\lambda x + y \in u(E_1)\). Par caractérisation \(u(E_1)\) est un sous-espace vectoriel de \(F\).
\end{proof}
%
\begin{corth}
  Soient deux \(K\)-espaces vectoriel \(E\) et \(F\) et \(u \in \Lin{E}{F}\). Alors \(\Image(u)=u(E)\) est un sous-espace vectoriel de \(F\).
\end{corth}

\subsubsection{Image réciproque}

\begin{theo}
  Soient \(E\) et \(F\) deux espaces vectoriels, \(u \in \Lin{E}{F}\), \(F_1\) un sous-espace vectoriel de \(F\). Alors l'image réciproque \(u^{-1}(F_1)\) est un sous-espace vectoriel de \(E\).
\end{theo}
\begin{proof}
  Par définition, l'image réciproque \(u^{-1}(F_1)\) est incluse dans \(E\). Elle n'est pas vide puisque \(F_1\) est un sous-espace vectoriel, alors \(0_F \in F_1\) et comme \(u\) est linéaire \(u(0_E)=0_F\) donc \(0_E \in u^{-1}(F_1)\). Soient deux vecteurs \(x,y\) de \(u^{-1}(F_1)\) et un scalaire \(\lambda\). Comme \(u\) est linéaire \(u(\lambda x+y)=\lambda u(x)+u(y)\). Or comme \(F_1\) est un sous-espace vectoriel et que \(u(x) \in F_1\) et \(u(y) \in F_1\) on a \(u(\lambda x+y) \in F_1\) alors \(\lambda x+y \in u^{-1}(F_1)\). Par caractérisation \(u^{-1}(F_1)\) est un sous-espace vectoriel de \(E\).
\end{proof}

\begin{corth}
  Soient \(E\) et \(F\) deux espaces vectoriels et \(u \in \Lin{E}{F}\). Alors \(\Ker(u)\) est un sous-espace vectoriel de \(E\).
\end{corth}

\subsection{Équations linéaires}
Soient \(E\) et \(F\) deux espaces vectoriels, \(u \in \Lin{E}{F}\) et \(b \in F\). On veut résoudre l'équation
\begin{equation}
  u(x)=b,
\end{equation}
c'est-à-dire trouver l'ensemble
\begin{equation}
  \mathcal{S}=\enstq{x \in E}{u(x)=b}.
\end{equation}
Deux cas se présentent~:
\begin{itemize}
\item soit \(b \not\in \Image(u)\) et alors \(\mathcal{S}\) est vide;
\item soit \(b \in \Image(u)\) alors il existe \(x_0 \in E\) tel que \(u(x_0)=b\), \(x_0\) est la solution particulière et on dit que l'équation est compatible. Alors pour tout \(x \in E\)
\begin{align}
  x \in \mathcal{S} &\iff u(x)=b \\
&\iff u(x)=u(x_0) \\
&\iff u(x-x_0)=0_F \\
&\iff x-x_0 \in \Ker(u) \\
&\iff \exists y \in \Ker(u) \ x=x_0+y.
\end{align}
Donc \(\mathcal{S}=\enstq{x_0 + y}{y \in \Ker(u)} = "x_0+\Ker(u)"\). On dit que c'est le sous-espace affine passant par \(x_0\) de direction le sous-espace vectoriel \(\Ker(u)\).
\end{itemize}

\emph{Applications}~: On peut par exemple étendre ce résultat aux suites récurrentes linéaires doubles, ou encore aux équations différentielles. Soient par exemples \(a\), \(b\) et \(c\) trois applications au moins continues de \(\R\) dans \(\R\). Soit aussi l'équation différentielle \(y''+ay'+by=c\) et l'application linéaire \(\fonction{u}{\Dr^{2}(\R,\R)}{\R^{\R}}{f}{f''+af'+bf}\). Alors l'équation différentielle peut s'écrire \(u(f)=c\) et résoudre l'équation homogéne associée \(u(f)=0\) consiste à trouver le noyau de \(u\). Si \(c \in \Image{u}\) alors il existe \(f_0 \in \Dr^{2}(\R,\R)\) tel que \(u(f_0)=c\) et \(f_0\) est la solution dite particulière. L'ensemble des solution de l'équation différentielle est donc le sous-espace affine \(\mathcal{S}=f_0 + \Ker(u)\).

\subsection{\(\K\)-espace vectoriel \((\Lin{E}{F}, +, \perp)\)}

\begin{theo}
  Soient deux \(\K\)-espaces vectoriels \(E\) et \(F\), alors l'ensemble des applications linéaires de \(E\) dans \(F\) \(\Lin{E}{F}\) est un sous-espace vectoriel du \(\K\)-espace vectoriel des applications de \(F\) dans \(E\) \((F^E,+,\perp)\).
\end{theo}

\begin{proof}
  \(F\) étant un \(\K\)-espace vectoriel, on sait que \((F^E,+,\perp)\) est un \(\K\)-espace vectoriel, de plus
  \begin{itemize}
  \item \(\Lin{E}{F}\) est inclus dans \(F^E\);
  \item l'application nulle qui va de \(E\) dans \(F\) est linéaire, donc \(\Lin{E}{F}\) n'est pas vide;
  \item Soient deux applications \(f,g \in \Lin{E}{F}\) et un scalaire \(\alpha\). Montrons que \(h=\alpha f +g\) est linéaire. Soit alors deux vecteurs \(x,y\) de \(E\) et un scalaire \(\lambda\) et on a
    \begin{align}
      h(\lambda x+y)&=(\alpha f+g)(\lambda x+y) \\
      &=\alpha f(\lambda x+y)  +g(\lambda x+y) \\
      &=\alpha(\lambda f(x)+f(y)) +\lambda g(x)+g(y) \\
      &=\lambda(\alpha f(x)) + \lambda g(x) + \alpha f(y) +g(y)\\
      &=\lambda(\alpha f(x)+g(x))+\alpha f(y) +g(y) \\
      &=\lambda h(x)+h(y),
    \end{align}
    donc \(h=\alpha f+g\) est linéaire, \(h \in \Lin{E}{F}\).
  \end{itemize}
  Alors par caractérisation, \(\Lin{E}{F}\) est un sous-espace vectoriel de \(F^E\).
\end{proof}

\subsection{Formes linéaires}
On sait que \(\K\) est un espace vectoriel sur lui-même.
\begin{defdef}
  Soit \((E,+,\perp)\) un \(\K\)-espace vectoriel, on appelle forme linéaire sur \(E\) toute application linéaire de \(E\) vers \(\K\).
\end{defdef}

Soient, par exemple, deux reéls \(a\) et \(b\) tels que \(a<b\) alors l'application
\begin{equation}
 \fonction{I}{\mathit{C}(\intervalleff{a}{b},\R)}{\R}{f}{\int_{a}^b f(t)\D t}
\end{equation}
est une forme linéaire. Puisque pour toutes fonctions \(f\) et \(g\) continues de \(\intervalleff{a}{b}\) vers \(\R\) et tout sclaire \(\alpha\) on a
\begin{align}
  \int_{a}^b (\alpha f(t)+g(t))\D t &= \alpha \int_{a}^b f(t)\D t + \int_{a}^b g(t)\D t \\
  I(\alpha f+g)&=\alpha I(f)+I(g).
\end{align}
Soit \(X\) un ensemble quelconque. \(\K\) est un espace vectoriel sur lui-même, donc on dispose du \(\K\)-espace vectoriel \(\K^X\) des applications de \(X\) vers \(\K\). Alors pour tout élément \(a \in X\), on peut définir \(\fonction{\hat{a}}{\K^X}{\K}{a}{f(a)}\). La fonction \(\hat{a}\) est une forme linéaire sur \(\K^X\).
\begin{theo}
  Toute forme linéaire non nulle est surjective.
\end{theo}
\begin{proof}
  Soit \((E,+,\perp)\) un \(\K\)-espace vectoriel et soit \(f \in \K^E\) une forme linéaire non nulle. Comme elle est non nulle, il existe \(x_0 \in E\) tel que \(f(x_0)\neq 0\). On pose \(\lambda =f(x_0)\). Comme il est non nul, on dispose de son inverse \(\lambda^{-1}\). Pour tout scalaire \(\mu\), on a
  \begin{align}
    \mu & =1 \mu \\
    &= \mu (\lambda \lambda^{-1})\\
    &=(\mu\lambda^{-1})f(x_0)\\
    &=f(\mu\lambda^{-1}x_0).
  \end{align}
  Donc \(\mu \in \Image(f)\). Alors \(\K \subset \Image(f)\). L'autre inclusion est évidente donc \(\K=\Image(f)\). Donc l'application \(f\) est surjective.
\end{proof}

Si \(E\) est un espace vectoriel, on note \(E^*=\Lin{E}{\K}\) l'ensemble des formes linéaires sur \(E\). Il est appelé le dual de \(E\).

\subsection{Ensemble des endomorphismes \(\Endo{E}\)}
\subsubsection{Composition d'applications linéaires}
\begin{theo}
  Soient trois \(\K\)-espaces vectoriels \(E\), \(F\) et \(G\). Soient \(u \in \Lin{E}{F}\) et \(v\in \Lin{F}{G}\). Alors \(v\circ u \in \Lin{E}{G}\).
\end{theo}
\begin{proof}
  Déjà, \(v\circ u\) est bien définie puisque \(\Image(u) \subset F\). Soient \(x\) et \(x'\) deux vecteurs de \(E\) et un scalaire \(\lambda\). Alors
  \begin{align}
    v\circ u(\lambda x+x') &=v(u(\lambda x+x')) \\
    &=v(\lambda u(x)+u(x'))\\
    &=\lambda v\circ u(x) + v\circ u(x'),
  \end{align}
  donc \(v\circ u\) est linéaire.
\end{proof}

\begin{prop}
  Soient trois \(\K\)-espaces vectoriels \(E\), \(F\) et \(G\) et \(u \in \Lin{E}{F}\). On définit 
  \begin{equation}
      \fonction{\Phi_u}{\Lin{F}{G}}{\Lin{E}{G}}{v}{v \circ u}. 
  \end{equation}
  Alors \(\Phi_u\) est linéaire.
\end{prop}
\begin{proof}
  Soient deux applications linéaires \(v,v'\) de \(\Lin{F}{G}\) et un scalaire \(\lambda\). On a
  \begin{equation}
    \Phi_u(\lambda v+v')=(\lambda v+v') \circ u \qquad \lambda \Phi_u(v)+\Phi_u(v')=\lambda v \circ u + v' \circ u.
  \end{equation}
  Alors en appliquant cette fonction à tout vecteur \(x\) de \(E\), on obtient
  \begin{align}
    \Phi_u(\lambda v+v')(x) &= ((\lambda v+v') \circ u)(x) \\
    &= \lambda v \circ u(x) + v' \circ u (x) \\
    &=(\lambda v \circ u + v'\circ u)(x) \\
    &=(\Phi_u(v)+\Phi_u(v'))(x).
  \end{align}
  Comme ceci est vrai pour tout \(x \in E\), on a alors \(\Phi_u(\lambda v+v') = \lambda \Phi_u(v)+\Phi_u(v')\). \(\Phi_u\) est donc linéaire.
\end{proof}

\begin{cor}
  La loi de composition \(\circ\) est distributive à droite par rapport à la loi \(+\) dans \(\Endo{E}\).
  \begin{equation}
    \forall (u,v,v') \in \Endo{E}^3 \quad (v+v') \circ u = v \circ u + v' \circ u.
  \end{equation}
  Ce qui est valable même si les applications ne sont pas linéaires.
\end{cor}

\begin{prop}
  Soient trois \(\K\)-espaces vectoriels \(E,F,G\) et \(u \in \Lin{F}{G}\). On définit
  \begin{equation}
  	\fonction{\Psi_v}{\Lin{E}{F}}{\Lin{E}{G}}{v}{v \circ u}.
  \end{equation} 
  Alors \(\Psi_v\) est linéaire.
\end{prop}
\begin{proof}
  Soient deux applications linéaires \(u,u'\) de \(\Lin{E}{F}\) et un scalaire \(\lambda\). On a
  \begin{equation}
    \Psi_v(\lambda u+u')=v \circ (\lambda u+u') \qquad \lambda \Psi_v(u)+\Psi_v(u')=\lambda v \circ u + v \circ u'.
  \end{equation}
  Alors en appliquant cette fonction à tout vecteur \(x\) de \(E\), on obtient
  \begin{align}
    \Psi_v(\lambda u+u')(x)&=v \circ (\lambda u+u')(x) \\
    &=\lambda v(u(x)) + v(u'(x)) \label{eg:eg2}\\
    &=\lambda v\circ u(x) +v\circ u'(x) \\
    &=(\lambda \Psi_v(u)+\Psi_v(u'))(x).
  \end{align}
  Attention, l'égalité~\eqref{eg:eg2} est valable parce que \(v\) est linéaire. Comme ceci est vrai pour tout \(x \in E\), on a alors \(\Psi_v(\lambda u+u')=\lambda \Psi_v(u)+\Psi_v(u')\), et donc \(\Psi_v\) est linéaire.
\end{proof}
\begin{cor}
  La loi de composition \(\circ\) est distributive à gauche par rapport à la loi \(+\) dans \(\Endo{E}\).
  \begin{equation}
    \forall (u,v,v') \in \Endo{E}^3 \quad v \circ (u+u') = v \circ u + v \circ u'.
  \end{equation}
  Ce qui est valable uniquement si \(v\) est linéaire.
\end{cor}

\subsubsection{Structure de \(\Endo{E}\)}

Soit \((E,+,\perp)\) un \(\K\)-espace vectoriel et \(\Endo{E}\) l'ensemble des endomorphismes linéaires de \(E\). Alors
\begin{itemize}
\item \((\Endo{E},+,\perp)\) est un \(\K\)-espace vectoriel et son élément nul est l'application nulle;
\item \((\Endo{E}, +, \circ)\) est un anneau~:
  \begin{itemize}
  \item \((\Endo{E},+)\) est un groupe abélien (puisque \((\Endo{E},+,\perp)\) est un \(\K\)-espace vectoriel),
  \item on a vu que la loi \(\circ\) était associative et distributive par rapport à la loi \(+\),
  \item l'élément neutre pour \(\circ\) est l'identité \(\Id_{\Endo{E}}\);
  \end{itemize}
\item la linéarité des fonctions construites \(\Phi_u\) et \(\Psi_v\) entraînent que
  \begin{equation}
    \forall (u,v) \in \Endo{E}^2 \ \forall \lambda \in \K \quad \lambda u \circ v = (\lambda u) \circ v = u \circ (\lambda v).
  \end{equation}
\end{itemize}

Ces trois points principaux signifient que \((\Endo{E},+,\circ, \perp)\) est une \(\K\)-algébre.

\subsubsection{Groupe linéaire \(\GL{E}\)}

\begin{theo}
  Soient deux \(\K\)-espaces vectoriels \(E\) et \(F\). Alors
  \begin{equation}
    u \in \Isom{E}{F} \implies u^{-1} \in \Isom{F}{E}.
  \end{equation}
\end{theo}
\begin{proof}
  Soit un isomorphisme \(u\) de \(E\) dans \(F\). Comme \(u\) est bijective, son inverse \(u^{-1}\) est bien définie de \(F\) dans \(E\) et elle est aussi bijective. Montrons qu'elle est linéaire. Soient deux vecteurs \(x\) et \(x'\) de \(E\) et un scalaire \(\lambda\). Alors
  \begin{align}
    u^{-1}(\lambda x+x') &= u^{-1}(\lambda u\circ u^{-1} (x) + u \circ u^{-1}(x')) \\
    &=u^{-1}(u(\lambda u^{-1} x +u^{-1}(x'))) \\
    &=\lambda u^{-1} x +u^{-1}(x').
  \end{align}
  Donc \(u^{-1}\) est linéaire et comme elle est aussi bijective on a \(u^{-1} \in \Isom{F}{E}\).
\end{proof}

\begin{corth}
  Soit \(E\) un \(\K\)-espace vectoriel, alors
  \begin{equation}
    u \in \GL{E} \implies u^{-1} \in \GL{E}.
  \end{equation}
\end{corth}
Étant donné un \(\K\)-espace vectoriel \(E\), on dispose de
\begin{itemize}
\item \(\GL{E} = \enstq{u \in \Endo{E}}{u\text{~est bijective}}\) ;
\item \(\uinv{\Endo{E}}=\enstq{u \in \Endo{E}}{\exists v \in \Endo{E} \quad v \circ u = u \circ v = Id_E}\) est le groupe des inversible de l'anneau \((\Endo{E}, +, \circ)\).
\end{itemize}
%
\begin{prop}
  Soit \(E\) un \(\K\)-espace vectoriel, alors
  \begin{equation}
    \GL{E} = \uinv{\Endo{E}}.
  \end{equation}
\end{prop}
\begin{proof}
  Soit \(u \in E^E\), alors
  \begin{align}
    u \in \GL{E} &\iff u \in \Endo{E} \text{~et u bijective} \\
    &\iff u \in \Endo{E} \quad \exists v \in E^E \ v \circ u=u \circ v=\Id_E \\
    &\iff u \in \Endo{E} \quad \exists v \in \Endo{E} \ v \circ u=u \circ v=\Id_E \\
    &\iff u \in \uinv{\Endo{E}}.
  \end{align}
  La deuxième équivalence est due à la caractérisation des bijections et la troisième au corollaire précédent. Du coup \(\GL{E} = \uinv{\Endo{E}}\).
\end{proof}
\begin{theo}
  Soit \(E\) un \(\K\)-espace vectoriel. Alors \((\GL{E}, \circ)\) est un groupe appelé groupe des automorphismes linéaires de \(E\), ou groupe linéaire de \(E\).
\end{theo}
\begin{proof}
  On sait que \(\GL{E} = \uinv{\Endo{E}}\) et on a vu dans le chapitre~
\ref{chap:groupes} au théorème~
\ref{theo:uinversible} à la page~\pageref{theo:uinversible} que c'est un groupe.
\end{proof}

\section{Intersection et somme de sous-espaces vectoriels}

\subsection{Intersection d'une famille de sous-espaces vectoriels}

\begin{theo}
  Soit \(E\) un \(\K\)-espace vectoriel, \(I\) un ensemble quelconque et \((E_i)_{i \in I}\) une famille quelconque de sous-espaces vectoriels de \(E\). Alors \(\bigcap_{i \in I} E_i\) est un sous-espace vectoriel de \(E\).
\end{theo}
\begin{proof}
  On note \(F\) cette intersection. Alors
  \begin{itemize}
  \item par définition \(F\) est inclus dans \(E\);
  \item pour tout \(i \in I\), \(E_i\) est un sous-espace vectoriel donc \(0 \in E_i\). Alors \(0 \in F\) et \(F\) est non vide;
  \item soient un scalaire \(\lambda\) et deux vecteurs \((x,y) \in F^2\). Comme ils sont dans \(F\), ils sont dans tous les \(E_i\) pour \(i \in I\); et comme chaque \(E_i\) est un sous-espace vectoriel, on a \(\lambda x +y \in E_i\) et comme c'est vrai pour chaque \(i \in I\) alors \(\lambda x +y \in F\).
  \end{itemize}
  Par caractérisation, \(F\) est un sous-espace vectoriel de \(E\).
\end{proof}
%
En général, l'union de deux sous-espace vectoriel n'est pas un sous-espace vectoriel. Dans le \(\R\)-espace vectoriel \(\C\) :
\begin{itemize}
\item \(\R\) est un sous-espace vectoriel;
\item \(\ii\R\) aussi.
\end{itemize}
Pourtant \(\R \cup \ii\R\) n'est pas un sous-espace vectoriel. En effet \(1 \in \R\) donc \(1 \in \R \cup \ii\R\) et \(\ii \in \ii\R\) donc \(\ii \in \R\cup \ii\R\), mais \(1+i \not\in \R\cup\ii\R\). Donc \(\R\cup\ii\R\) n'est pas stable par addition. Ce n'est donc pas un sous-espace vectoriel.

\subsection{Sous-espace vectoriel engendré par une partie}

Soit un \(\K\)-espace vectoriel \(E\) et \(X\) une partie quelconque de \(E\).

\subsubsection{Définition de \(\VectEngendre{X}\)}

Soit \(\Xi=\enstq{E' \in \Part(E)}{E' \text{~est un sous-espace vectoriel de E et } X \subset E'}\). Alors
\begin{itemize}
\item \(\Xi\) est une partie non vide car \(E \in \Xi\);
\item d'après la dernière sous section, \(\bigcap_{E' \in \Xi} E'\) est un sous-espace vectoriel de \(E\);
\item pour tout \(E' \in \Xi\), on a \(X \in E'\) donc il est dans l'intersection \(X \subset \bigcap_{E' \in \Xi} E'\);
\item soit \(E_0'\) un sous-espace vectoriel de \(E\) qui contient \(X\), alors par définition \(E_0' \in \Xi\).
\end{itemize}

Donc \(\bigcap_{E' \in \Xi} E' \subset E_0'\).

\begin{defdef}
  Soient un \(\K\)-espace vectoriel \(E\) et \(X\) une partie de \(E\). L'ensemble \(\Xi\) des sous-espaces vectoriels de \(E\) qui contiennent \(X\) admet un plus petit élément pour l'inclusion, c'est \(\bigcap_{E' \in \Xi} E'\). Ce plus petit élément est appelé sous-espace vectoriel engendré par la partie \(X\) et est noté \(\VectEngendre(X)\).

  En ``clair'' : \(\VectEngendre(X)\) est le plus petit sous-espace vectoriel de \(E\) contenant \(X\).
\end{defdef}

\begin{prop}[Caractérisation pratique]
  Soient un \(\K\)-espace vectoriel \(E\) et \(X\) une partie de \(E\). Pour tout \(E_1 \in \Part(E)\),
  \begin{equation}
    E_1 = \VectEngendre(X) \iff
    \begin{cases}
      E_1 \text{~est un sous-espace vectoriel de } E \\
      X \subset E_1 \\
      \forall E_2 \text{~sous-espace vectoriel de E } X \subset E_2 \implies E_1 \subset E_2
    \end{cases}.
  \end{equation}
\end{prop}

\subsubsection{Description de \(\VectEngendre(X)\)}

\begin{prop}
  Soit un \(\K\)-espace vectoriel \(E\). Pour toute partie \(X\) de \(E\), \(X=\VectEngendre(X)\) si et seulement si \(X\) est un sous-espace vectoriel de \(E\).
\end{prop}
\begin{proof}
  Par définition \(\VectEngendre(X)\) est un sous-espace vectoriel de \(E\) donc si \(X=\VectEngendre(X)\) alors \(X\) est un sous-espace vectoriel de \(E\). Si maintenant on vérifie les hypothèses pour la caractérisation
  \begin{itemize}
  \item \(E_1=X\) est un sous-espace vectoriel de \(E\);
  \item \(X \subset E_1=X\);
  \item pour tout sous-espace vectoriel \(E_2\) de \(E\) si \(X \subset E_2\) alors \(X=E_1 \subset E_2\).
  \end{itemize}
  Comme elles sont validées, on peut écrire que \(E_1=X=\VectEngendre(X)\).
\end{proof}
\begin{prop}
  Si \(X=\emptyset\) alors \(\VectEngendre(X)=\{0_E\}\).
\end{prop}
\begin{proof} On démontre ce résultat grâce à une caractérisation. Vérifions les hypothèses~:
  \begin{itemize}
  \item \(\{0_E\}\) est un sous-espace vectoriel de \(E\);
  \item \(\emptyset \subset \{0_E\}\)
  \item pour tout sous-espace vectoriel \(E_2\) de \(E\), \(\{0\} \subset E_2\).
  \end{itemize}
  Alors par caractérisation, \(\VectEngendre(\emptyset)=\{0_E\}\).
\end{proof}
\begin{prop}
  Si \(X \in \Part(E) \setminus\{\emptyset\}\) alors
  \begin{equation}
    \VectEngendre(X)=\enstq{y \in E}{\exists n \in \N^* \ \exists (x_1, \ldots, x_n) \in X^n \ \exists (\lambda_1, \ldots, \lambda_n) \in \K^n \quad y=\sum_{i=1}^n \lambda_i x_i}.
  \end{equation}
  C'est l'ensemble des combinaisons linéaires finies d'éléments de \(X\).
\end{prop}
\begin{proof}
  Soit
  \begin{equation}
    V=\enstq{y \in E}{\exists n \in \N^* \ \exists (x_1, \ldots, x_n) \in X^n \ \exists (\lambda_1, \ldots, \lambda_n) \in \K^n \quad y=\sum_{i=1}^n \lambda_i x_i}.
  \end{equation}
  On va montrer que \(V=\VectEngendre(X)\) par caractérisation~:
  \begin{itemize}
  \item \(V\) est un sous-espace vectoriel de \(E\)~:
    \begin{itemize}
    \item \(V \subset E\);
    \item \(V \neq \emptyset\) car \(X \neq \emptyset\) donc il existe \(x \in X\) et \(x=1 x \in V\);
    \item Soit deux vecteurs \(y,z \in V\) et un scalaire \(\alpha\)~:
      \begin{align}
        y \in V &\quad \exists n \in \N^* \ \exists x_1, \ldots, x_n \in X \ \exists \lambda_1, \ldots, \lambda_n \in \K \ y=\sum_{i=1}^n \lambda_i x_i \\
        z \in V &\quad \exists p \in \N^* \ \exists x'_1, \ldots, x'_p \in X \ \exists \lambda'_1, \ldots, \lambda' \in \K \ z=\sum_{i=1}^p \lambda'_i x'_i,
      \end{align}
      alors
      \begin{align}
        \alpha y +z &= \alpha \sum_{i=1}^n \lambda_i x_i + \sum_{i=1}^p \lambda'_i x'_i \\
        &= \sum_{i=1}^n (\alpha\lambda_i) x_i + \sum_{i=1}^p \lambda'_i x'_i,
      \end{align}
      et donc \(\alpha y +z \in V\).
    \end{itemize}
    Alors par caractérisation, \(V\) est un sous-espace vectoriel de \(E\).
  \item \(X \subset V\) : en effet pour tout \(x \in X\), \(x = 1x \in V\);
  \item soit \(F\) un sous-espace vectoriel de \(E\) qui contient \(X\), alors \(F\) est stable par combinaison linéaire et donc il contient toutes les combinaisons linéaires d'éléments de \(X\), c'est-à-dire \(V\).
  \end{itemize}
  Alors par caractérisation, \(V=\VectEngendre(X)\).
\end{proof}
\begin{defdef}
  Si \(E_1=\VectEngendre(X)\), on dira que \(X\) est une partie génératrice du sous-espace vectoriel \(E_1\).
\end{defdef}
\begin{defdef}
  Si \(X\) est une partie finie de \(E\) de cardinal \(q \in \N^*\) tel que \(X=\{x_1, \ldots, x_q\}\) et tel que \(\VectEngendre(X)=\{\sum_{i=1}^q \lambda_i x_i \ \lambda_1,\ldots,\lambda_q \in \K\}\) alors \(X\) est appelé une famille génératrice de \(\VectEngendre(X)\).
\end{defdef}
Il n'y a pas d'unicité de la famille génératrice, en effet \(\{0\}=\VectEngendre(\emptyset)=\VectEngendre(\{0\})\).

\emph{Exemples}~:  Dans le \(\R\)-espace vectoriel \(\C\) on a \(\R=\VectEngendre(\{1\})=\VectEngendre(\{\pi\})\), \(\ii\R=\VectEngendre(\{\ii\})\) et \(\C=\VectEngendre(\{1,\ii\})\) et alors \(\{1,\ii\}\) est une famille génératrice du \(\R\)-espace vectoriel \(\C\). Dans le \(\C\)-espace vectoriel \(\C\) \(\{1\}\) est une famille génératrice du \(\C\)-espace vectoriel \(\C\).

\subsection{Somme de deux sous-espaces vectoriels}

\begin{defdef}
  Soient un \(\K\)-espace vectoriel \(E\) puis \(E_1\) et \(E_2\) deux de ses sous-espaces vectoriels. On définit l'ensemble
  \begin{equation}
    E_1+E_2=\enstq{y \in E}{\exists (x_1,x_2)\in E_1\times E_2 \quad y=x_1+x_2},
  \end{equation}
  et il est appelé la somme de \(E_1\) et de \(E_2\).
\end{defdef}

\begin{theo}
  Soient un \(\K\)-espace vectoriel \(E\) puis \(E_1\) et \(E_2\) deux de ses sous-espaces vectoriels. Alors
  \begin{enumerate}
  \item \(E_1+E_2\) est un sous-espace vectoriel de \(E\);
  \item \(E_1+E_2=\VectEngendre(E_1 \cup E_2)\).
  \end{enumerate}
\end{theo}

\begin{proof}
  \begin{enumerate}
  \item On démontre que c'est un sous-espace vectoriel par caractérisation~:
    \begin{itemize}
    \item par définition \(E_1 + E_2 \subset E\);
    \item comme \(E_1\) et \(E_2\) sont des sous esapces vectoriels de \(E\), \(0\in E_1\) et \(0 \in E_2\) donc par somme \(0=0+0 \in E_1+E_2\), donc il n'est pas vide;
    \item pour tout scalaire \(\lambda\) et tous vecteusr \(y\), \(y'\) de \(E_1+E_2\) on peut dire qu'il existe \(x_1\), \(x'_1\) dans \(E_1\) et \(x_2\), \(x'_2\) dans \(E_2\) tels que \(y=x_1+x_2\) et \(y'=x'_1+x'_2\), alors
      \begin{align}
        \lambda y + y' &= \lambda(x_1+x_2)+x'_1+x'_2 \\
        &=(\lambda x_1+x'_1)+(\lambda x_2+x'_2),
      \end{align}
      comme \(E_1\) et \(E_2\) sont des sous-espaces vectoriel de \(E\) on a \((\lambda x_1+x'_1) \in E_1\) et \((\lambda x_2+x'_2) \in E_2\), ainsi \(\lambda y + y' \in E_1+E_2\).
    \end{itemize}
    Alors \(E_1+E_2\) est un sous-espace vectoriel de \(E\).
  \item
    \begin{itemize}
    \item on sait que \(E_1+E_2\) est un sous-espace vectoriel de \(E\);
    \item quelque soit \(x_1 \in E_1\) on a aussi \(0 \in E_2\) donc \(x_1=x_1+0 \in E_1+E_2\), alors \(E_1 \subset E_1+E_2\), de la même manière \(E_2 \subset E_1+E_2\), ainsi en réunissant les deux ensembles : \(E_1 \cup E_2 \subset E_1+E_2\);
    \item Soit un sous-espace vectoriel \(F\) de \(E\) qui contient \(E_1 \cup E_2\), montrons que \(E_1+E_2 \subset F\)~: Soit \(y \in E_1+E_2\) alors il existe \((x_1,x_2) \in E_1 \times E_2\) tel que \(y=x_1+x_2\); comme \(x_1 \in E_1 \subset E_1 \cup E_2  \subset F\) idem \(x_2 \in E_2 \subset E_1 \cup E_2  \subset F\); \(F\) est un sous-espace vectoriel alors \(y=x_1+x_2 \in F\) et finalement \(E_1+E_2 \subset F\).
    \end{itemize}
    Alors par caractérisation \(E_1+E_2=\VectEngendre(E_1 \cup E_2)\). Il est appelé l'espace somme de \(E_1 \cup E_2\).
  \end{enumerate}
\end{proof}

\begin{defdef}
  Soient un \(\K\)-espace vectoriel \(E\) puis \(E_1\) et \(E_2\) deux de ses sous-espaces vectoriels. On dit que la somme \(E_1+E_2\) est directe, on notera \(E_1 \oplus E_2\), si et seulement si \(E_1 \cap E_2 =\{0\}\).
\end{defdef}

Par exemple, \(\C=\R \oplus \ii\R\). Puisque tout complexe \(z\) admet une partie réelle \(a\) et une partie imaginaire \(b\) tels que \(z=a+\ii b\), de plus \(\R\cap \ii \R=\{0\}\).

\begin{theo}\label{theo:caracSommeDirecte}
  Soient un \(\K\)-espace vectoriel \(E\) puis \(E_1\) et \(E_2\) deux de ses sous-espaces vectoriels. Il y a équivalence entre les assertions suivantes~:
  \begin{enumerate}
  \item \(\forall x \in E_1+E_2 \ \exists! (x_1,x_2) \in E_1 \times E_2 \quad x=x_1+x_2\);
  \item \(\exists! (x_1,x_2) \in E_1\times E_2 \quad x_1+x_2=0\);
  \item \(\forall (x_1,x_2) \in E_1\times E_2 \quad x_1+x_2=0 \implies x_1=x_2=0\);
  \item \(E_1 \cap E_2 = \{0\}\).
  \end{enumerate}
\end{theo}
\begin{proof}
  \(1 \implies 2\) : On applique l'égalité à \(0\) puisque \(E_1+E_2\) est un sous esapce vectoriel de \(E\).

  \(2 \implies 3\) : On sait que \(0_{E_1}+0_{E_2}=0_E\) et si \(x_1+x_2=0\) avec \(x_1\in E_1\) et \(x_2\in E_2\) alors par unicité on en déduit que \(x_1=x_2=0\).

  \(3 \implies 4\) : Soit \(x \in E_1 \cap E_2\) alors \(-x \in E_1 \cap E_2\) puisque c'est un sous-espace vectoriel. Alors \(0=x-x\) donc d'après \(3\), \(x=-x=0\) donc \(E_1 \cap E_2 \subset \{0\}\) et l'autre inclusion est vraie puisque c'est un sous-espace vectoriel.

  \(4 \implies 1\) Supposons que \(E_1 \cap E_2 = \{0\}\). Par définition de l'espace somme on sait que pour tout \(x \in E_1+E_2\) il existe \((x_1,x_2) \in E_1 \times E_2\) tel que \(x=x_1+x_2\). Montrons que ce couple est unique~: supposons qu'il existe un deuxième couple \((x'_1,x'_2) \in E_1 \times E_2\) tel que \(x=x'_1+x'_2\). Alors \(x_1+x_2=x'_1+x'_2\). Alors \(x_1-x'_1=x_2-x'_2\) or \(x_1 -x'_1 \in E_1\) et \(x_2-x'_2 \in E_2\)  et comme ils sont égaux ils sont dans l'intersection \(E_1 \cap E_2\). Or comme elle est nulle, par hypothèse, donc \(x_1=x'_1\) et \(x'_2=x_2\).
\end{proof}

\subsection{Sous-espaces vectoriels supplémentaires}

\subsubsection{Définitions}

\begin{defdef}
  Soient \(E_1\) et \(E_2\) des sous-espace vectoriels d'un même \(\K\)-espace vectoriel \(E\). On dit que \(E_1\) et \(E_2\) sont supplémentaire dans \(E\) si et seulement si \(E_1 \oplus E_2=E\), c'est-à-dire si et seulement si \(E_1 \cap E_2 = \{0\}\) et \(E=E_1+E_2\).
\end{defdef}
\begin{theo}
  Soient \(E_1\) et \(E_2\) deux sous-espace vectoriels d'un même \(\K\)-espace vectoriel \(E\). Alors ils sont supplémentaires si et seulement si
  \begin{equation}
    \forall x \in E \ \exists! (x_1,x_2) \in E_1 \times E_2 \quad x=x_1+x_2.
  \end{equation}
\end{theo}
\begin{proof}
  S'ils sont supplémentaires alors on a \(E_1 \cap E_2 = \{0\}\) (qui donne l'unicité du couple)  et \(E=E_1+E_2\) (qui donne l'existence du couple). Si maintenant on suppose que
\begin{equation}
  \forall x \in E \ \exists! (x_1,x_2) \in E_1 \times E_2 \quad x=x_1+x_2.
\end{equation}
alors elle nous dit que \(E \subset E_1+E_2\) et comme l'autre inclusion est triviale on a \(E = E_1+E_2\). Par le théorème~
\ref{theo:caracSommeDirecte} c'est équivalent à \(E_1 \cap E_2 =\{0\}\).
\end{proof}

\emph{Exemples}~: \(\R\) et \(\ii\R\) sont supplémentaires dans le \(\R\)-espace vectoriel \(C\). Pour n'importe quel \(\K\)-espace vectoriel \(E\) \(E=E \oplus \{0\}\). Si on note \(E=\R^\R\) le \(\R\)-espace vectoriel des fonctions et si on note \(P\) l'ensemble des fonctions paires et \(I\) l'ensemble des fonctions impaires alors \(\R^R=P \oplus I\).
En effet, ce sont des sous-espace vectoriel de \(\R^\R\) et on a montré au chapitre~
\ref{chap:fonctionsusuelles} que
\begin{equation}
  \forall f \in \R^R \ \exists! (\varphi, \psi) \in P\times I \quad f=\varphi+\psi,
\end{equation}
puisque pour toute fonction \(f\) on a~:
\begin{gather}
  \forall x \in \R \quad \varphi(x)=\frac{f(x)+f(-x)}{2}, \\
  \forall x \in \R \quad  \psi(x)=\frac{f(x)-f(-x)}{2}.
\end{gather}

Ne pas confondre supplémentaire et complémentaire. Le complémentaire d'un sous-espace vectoriel n'est pas un sous-espace vectoriel puisqu'il ne contient pas \(0\).

\subsubsection{Supplémentaires d'un sous-espace vectoriel}

\begin{defdef}
  Soit un \(\K\)-espace vectoriel \(E\) et \(E_1\) un sous-espace vectoriel de \(E\). On appelle supplémentaire de \(E_1\) dans \(E\) tout sous-espace vectoriel \(E_2\) de \(E\) tel que \(E= E_1 \oplus E_2\).
\end{defdef}

Un sous-espace vectoriel donné admet-il un supplémentaire? Si oui, est-ce-que ce supplémentiare est unique?

Pour l'existence, on la démontrera en dimension finie au chapitre~
\ref{chap:dimensionfinie} et en dimension infinie c'est une conséquence de l'axiome du choix. Cependant, il n'y a pas unicité du supplémentaire.

\begin{prop}\label{prop:deuxsuppiso}
  Soit un \(\K\)-espace vectoriel \(E\) et \(E_1\) un sous-espace vectoriel de \(E\). Supposons que \(E_1\) admettent deux supplémentaires notés \(E_2\) et \(E'_2\). Alors
  \begin{itemize}
  \item Ils sont isomorphes;
  \item si \(E_2 \subset E'_2\) ou l'inverse alors \(E_2=E'_2\).
  \end{itemize}
\end{prop}
\begin{proof}
  \begin{itemize}
  \item voir la sous-sous-section~
\ref{sec:projections}
  \item On sait que \(E=E_1 \oplus E_2=E_1 \oplus E'_2\) et que \(E_2 \subset E'_2\). Soit \(x \in E'_2 \subset E\) et comme \(E=E_1 \oplus E_2\) il existe un unique couple \((x_1,x_2 ) \in E_1 \times E_2\) tel que \(x=x_1+x_2\). Donc \(x-x_2=x_1\) et alors comme \(x_2 \in E_2 \subset E'_2\) et comme c'est un sous-espace vectoriel on a \(x-x_2\in E'_2\)  et \(x_1 \in E_1\) comme ils sont égaux et que \(E_1 \cap E'_2=0\)  alors \(x=x_2 \in E_2\). On a montré \(E'_2 \subset E_2\) donc l'égalité.
  \end{itemize}
\end{proof}

\subsubsection{Préparation du théorème du rang}
Comme son nom l'indique, ce théorème sert à préparer la démonstration du théorème du rang, Théorème~
\ref{theo:rang}.
\begin{theo}\label{theo:preptheorang}
  Soient \(E\) et \(F\) deux \(\K\)-espaces vectoriels et \(u \in \Lin{E}{F}\). Ainsi tout supplémentaire à \(\Ker(u)\) est isomorphe à \(\Image(u)\).
\end{theo}
\begin{proof}
  Soit un supplémentaire \(S\) de \(\Ker(u)\) dans \(E\). Construisons un isomorphisme de \(S\) sur \(\Image(u)\). Soit l'application \(v=u_{|S}^{|\Image(u)}\). Alors~:
  \begin{enumerate}
  \item \(v\) est bien définie puisque pour tout \(x \in S\), \(u(x) \in \Image(u)\);
  \item \(v\) est linéaire puisque c'est une restriction de l'application linéaire \(u\);
  \item \(\Ker(v)=\Ker(u) \cap S =\{0\}\) puisque par hypothèse ils sont supplémentaires, alors \(v\) est injective;
  \item Montrons que \(\Image(u)=\Image(v)\)~:
    \begin{enumerate}
    \item Déjà \(\Image(v) \subset \Image(u)\) par définition de \(v\)
    \item et ensuite soit \(y \in \Image(u)\), alors il existe \(x \in E\) tel que \(y=u(x)\). Comme \(E=\Ker(u) \oplus S\) il existe un unique couple \((z,t) \in \Ker(u) \times S\) tel que \(x=z+t\). Alors \(y=u(z+t)=u(z)+u(t)=u(t)=v(t)\) parce que \(z \in \Ker(u)\) et parce que \(t \in S\). Alors \(y \in \Image(v)\). Ainsi \(\Image(u) \subset \Image(v)\)
    \end{enumerate}
    Par double inclusion \(\Image(u)=\Image(v)\), donc \(v\) est surjective.
  \end{enumerate}
  Finalement l'application \(v\) est bijective et linéaire. C'est donc un isomorphisme de \(S\) sur \(\Image(u)\).
\end{proof}

\section{Éléments remarquables de \(\Endo{E}\) et \(\GL{E}\)}
Dans toute cette section, \(E\) désignera un \(\K\)-espace vectoriel muni de son addition et de sa multiplication par un scalaire.
\subsection{Projecteurs}

\subsubsection{Projections}
\label{sec:projections}

\begin{defdef}
  Soient \(E_1, E_2\) deux sous-espaces vectoriels supplémentaires de \(E\). Pour tout vecteur \(x \in E\) il existe un unique couple \((x_1,x_2) \in E_1 \times E_2\) tel que \(x=x_1+x_2\) (d'après le théorème~
\ref{theo:caracSommeDirecte}). Alors on peut définir les applications \(\fonction{p_1}{E}{E_1}{x}{x_1}\) et \(\fonction{p_2}{E}{E_2}{x}{x_2}\). L'application \(p_1\) est appelée la projection sur \(E_1\) parallèlement à \(E_2\). De manière analogue, l'application \(p_2\) est appelée la projection sur \(E_2\) parallèlement à \(E_1\).
\end{defdef}

\begin{theo}\label{theo:propprojection}
  Soient \(E_1, E_2\) deux sous-espaces vectoriels supplémentaires de \(E\). Soit aussi \(p_1\) la projection sur \(E_1\) parallèlement à \(E_2\) et l'application \(p_2\) la projection sur \(E_2\) parallèlement à \(E_1\). Alors ces projections vérifient les propriétés suivantes
  \begin{enumerate}
  \item \(p_1\) et \(p_2\) sont des endomorphismes de \(E\);
  \item \(\Image(p_1)=E_1\) et de manière analogue \(\Image(p_2)=E_2\);
  \item \(\Ker(p_1)=E_2\) et de manière analogue \(\Ker(p_2)=E_1\);
  \item \(p_1 \circ p_1=p_1\) et de manière analogue \(p_2 \circ p_2=p_2\);
  \item \(p_1 \circ p_2 = p_2 \circ p_1=0\);
  \item \(p_1+p_2=\Id\).
  \end{enumerate}
\end{theo}

\begin{proof}
  \begin{enumerate}
  \item
    Soient \(x\) et \(y\) deux vecteurs de \(E\) et un scalaire \(\lambda\). Puisque \(E_1\) et \(E_2\) sont deux sous-espaces vectoriels supplémentaires de \(E\), il existe deux uniques couples \((x_1,x_2) \in E_1 \times E_2\) et \((y_1,y_2) \in E_1 \times E_2\) tels que \(x=x_1+x_2\) et \(y=y_1+y_2\). Alors
    \begin{equation}
      \lambda x +y= \lambda (x_1+x_2) + (y_1+y_2) = (\lambda x_1+y_1) + (\lambda x_2 +y_2),
    \end{equation}
    et comme \(E_1\) et \(E_2\) sont stables par combinaison linéaire \(\lambda x_1+y_1 \in E_1\) et \(\lambda x_2 +y_2 \in E_2\) et comme cette décomposition est unique on a bien~:
    \begin{align}
      \lambda p_1(x)+p_1(y) &= p_1(\lambda x+y); \\
      \lambda p_2(x)+p_2(y) &= p_2(\lambda x+y).
    \end{align}
    Ce qui montre bien que \(p_1\) et \(p_2\) sont linéaires, de plus par définition \(\Image(p_1) \subset E_1 \subset E\) et \(\Image(p_2) \subset E_2 \subset E\) donc \(p_1,p_2 \in \Endo{E}\).
  \item On démontre le résultat pour \(p_1\). Par définition \(\Image(p_1) \subset E_1\), alors montrons l'autre inclusion. Soit \(x_1 \in E_1\) et on a \(x_1 = x_1+0\) donc \(p(x_1)=x_1\). Ainsi \(x_1 \in \Image(p_1)\) et alors \(E_1 \subset \Image(p_1)\). Par double inclusion \(\Image(p_1)=E_1\).
  \item On démontre le résultat pour \(p_1\). Soit \(x_2 \in E_2\) et comme \(x_2=0+x_2\) alors \(p_1(x_2)=0\), on a montré l'inclusion \(E_2 \subset \Ker(p_1)\). Soit maintenant \(x \in \Ker(p_1) \subset E\) alors \(p_1(x)=0\). Comme \(x \in E\), il existe un unique couple \((x_1,x_2) \in E_1 \times E_2\) tel que \(x=x_1+x_2\). Comme \(x_1=p_1(x)=0\) on a \(x=x_2 \in E_2\). Alors on vient de montrer \(\Ker(p_1) \in E_2\). Par double inclusion on obtient \(\Ker(p_1)=E_2\).
  \item On démontre le résultat pour \(p_1\). Soit \(x \in E\), alors \(p_1(x) \in E_1\) donc \(p_1(p_1(x))=p_1(x)\) (puisque \(\Image(p_1)=E_1\)). donc \(p_1 \circ p_1=p_1\).
  \item On démontre le résultat pour \(p_1\). Soit \(x \in E\), comme \(p_2(x) \in E_2 = \Ker(p_1)\) alors \(p_1(p_2(x))=0\) alors \(p_1 \circ p_2=0\).
  \item Soit un vecteur \(x \in E\), il existe un unique couple \((x_1,x_2) \in E_1 \times E_2\) tel que \(x=x_1+x_2\). De plus \(x_1=p_1(x)\) et \(x_2=p_2(x)\) alors \(x=(p_1+p_2)(x)\). Finalament \(p_1+p_2 = \Id\).
\end{enumerate}
\end{proof}

\begin{proof}[Démonstration de la proposition~
\ref{prop:deuxsuppiso}]
  Soient \(E_1, E_2, E'_2\) trois sous-espaces vectoriels de \(E\) tels que \(E_1\) et \(E_2\) soient supplémentaires et tels que \(E_1\) et \(E'_2\) soient aussi supplémentaires.
  \begin{equation}
    E=E_1 \oplus E_2 = E_1 \oplus E'_2
  \end{equation}
Soit \(p_1\) la projection sur \(E_1\) parallèlement à \(E'_2\) et \(p'_2\) la projection sur \(E'_2\) parallélement à \(E_1\). Soit l'application \(\varphi={p'_2}_{|E_2}^{|E'_2}\), qui est légitime puisque \(E'_2=\Image(p'_2)\). L'application \(\varphi\) est linéaire puisque c'est une restriction de \(p'_2\) qui est linéaire. Comme \(E_2\) est un supplémentaire de \(E_1\) on a \(E_1 = \Ker(p'_2)\) et par définition \(E'_2 = \Image(p'_2)\). D'après le théorème~
\ref{theo:preptheorang}, comme \(p'_2\) est linéaire tout supplémentaire de \(\Ker(p'_2)\) est isomorphe à \(\Image(p'_2)\). C'est-à-dire en remplaçant que \(\varphi\) est un isomorphisme de \(E_2\) sur \(E'_2\).
\end{proof}

\subsubsection{Projecteurs}

\begin{defdef}
  On appelle projecteur tout endomorphisme \(p \in \Endo{E}\) tel que \(p \circ p=p\).
\end{defdef}

\begin{prop}[Caractérisation de l'image d'un projecteur]\label{prop:caracimageproj}
  Soit \(p\) un projecteur de \(E\) alors pour tout vecteur \(x \in E\),
  \begin{equation}
    x \in \Image(p) \iff p(x)=x.
  \end{equation}
\end{prop}
\begin{proof}
  Si \(p(x)=x\) alors \(x \in \Image(p)\), c'est trivial. Soit maintenant \(x \in \Image(p)\), alors il existe \(y \in E\) tel que \(x=p(y)\) et \(p(x)=p(p(y))=p(y)=x\).
\end{proof}

\begin{prop}[Caractérisation des projecteurs]
  Soit une application \(p \in E^E\), alors
  \begin{equation}
    p \text{~est un projecteur} \iff p \in \Endo{E} \text{~et~} \forall x \in \Image(p) \quad p(x)=x.
  \end{equation}
\end{prop}
\begin{proof}
  Si \(p\) est un projecteur alors par définition, \(p\) est linéaire et d'après la proposition~
\ref{prop:caracimageproj} pour tout \(x \in \Image(p)\) \(p(x)=x\).

  Si maintenant on suppose que \(p \in \Endo{E}\) et \(\forall x \in \Image(p)\) \(p(x)=x\). On veut montrer que c'est un projecteur. Déjà \(p\) est linéaire et ensuite \(p(x)=x \in \Image(p)\) donc \(p(x)=p(p(x))\) alors \(p \circ p=p\).
\end{proof}

\begin{prop}\label{prop:projecteursupplementaire}
  Soit \(p\) un projecteur de \(E\), alors
  \begin{equation}
    E = \Ker(p) \oplus \Image(p).
  \end{equation}
  Cependant, la réciproque n'est pas vraie. Ce n'est pas une caractérisation.
\end{prop}
\begin{proof}
  Soit \(x \in \Ker(p) \cap \Image(p)\). Alors comme \(x \in \Ker(p)\) on a \(p(x)=0\) et comme on a aussi \(x \in \Image(p)\) alors \(p(x)=x\). Au final \(x = 0\) donc \(\Ker(p) \cap \Image(p) \subset \{0\}\). Ce sont aussi des sous-espaces vectoriels donc leur intersection aussi est un sous-espace vectoriel, alors l'autre inclusion est triviale. Ainsi \(\Ker(p) \cap \Image(p)=\{0\}\).

  Soit un vecteur \(x \in E\). On a \(x = p(x) +(x-p(x))\) et \(p(x) \in \Image(p)\) et \(p(x-p(x))=p(x)-p(p(x))=0\) (\(p\) est un projecteur) alors \(x-p(x) \in \Ker(p)\). On vient de montrer que \(E \subset \Image(p) + \Ker(p)\). L'autre inclusion est triviale puisque ce sont des sous-espaces vectoriels et donc \(E = \Ker(p) +\Image(p)\).

  Finalement le noyau et l'image d'un projecteur sont supplémentaires dans l'espace vectoriel \(E\).
\end{proof}

\begin{prop}[Équivalence entre projecteur et projection]
Les projections et les projecteurs de \(E\) vérifient~:
  \begin{enumerate}
  \item Toute projection est un projecteur;
  \item Tout projecteur \(p\) est la projection sur \(\Image(p)\) parallèlement à \(\Ker(p)\).
  \end{enumerate}
  On pourra utiliser indifféremment les deux termes.
\end{prop}
\begin{proof}
  \begin{enumerate}
  \item Déjà vu au théorème~
\ref{theo:propprojection} au point 4.
  \item Si \(p\) est un projecteur, on a montré que \(E = \Image(p) \oplus \Ker(p)\) (proposition~
\ref{prop:projecteursupplementaire}). Alors \(p\) est la projection sur \(\Image(p)\) parallèlement à \(\Ker(p)\).
  \end{enumerate}
\end{proof}

\begin{prop}[Projecteur associé]
  Soit un projecteur \(p \in \Endo{E}\), alors \(q=\Id-p\) est aussi un projecteur appelé projecteur associée de \(p\) et
  \begin{gather}
    \Image(q) = \Ker(p);\\
    \Ker(q)=\Image(p);\\
    p \circ q = q \circ p =0.
  \end{gather}
\end{prop}
\begin{proof}
  On note \(E_1 = \Image(p)\) et \(E_2 = \Ker(p)\). D'après la sous-section~
\ref{sec:projections} on sait que \(q=\Id-p\) est la projection sur \(E_2\) parallèlement à \(E_1\) et les égalités en découlent.
\end{proof}

\begin{defdef}
  Pour tout endomorphisme \(u \in \Endo{E}\) on définit l'ensemble des invariants
  \begin{equation}
    \Inv(u)=\enstq{x \in E}{u(x)=x},
  \end{equation}
  et l'ensemble des opposés
  \begin{equation}
    \Opp(u)=\enstq{x \in E}{u(x)=-x}.
  \end{equation}
\end{defdef}

\begin{prop}
  Pour tout endomorphisme \(u \in \Endo{E}\), \(\Inv(u)\) et \(\Opp(u)\) sont des sous-espaces vectoriels de \(E\).
\end{prop}
\begin{proof}
  Soit un vecteur \(x \in E\)
  \begin{align}
    x \in \Inv(u) &\iff u(x) = x \\
    &\iff (u-\Id)(x)=0 \\
    &\iff x \in \Ker(u-\Id)
  \end{align}
  et alors \(\Inv(u) = \Ker(u-\Id)\), de la même manière \(\Opp(u) = \Ker(u+\Id)\). Comme ce sont des noyaux d'applications linéaires, ce sont des sous-espaces vectoriels de \(E\).
\end{proof}

\begin{prop}
  Pour tout endomorphisme \(u \in \Endo{E}\), \(\Inv(u) \cap \Opp(u) = \{0\}\).
\end{prop}
\begin{proof}
  L'inclusion \(\{0\} \subset \Inv(u) \cap \Opp(u)\) est triviale puisque ce sont deux sous-espaces vectoriels donc l'intersection l'est aussi. Soit \(x \in \Inv(u) \cap \Opp(u)\) alors \(-x=u(x)=x\) donc \(2x=0\) et comme dans \(\K\), \(2\neq 0\) on a bien \(x=0\), alors \(\Inv(u) \cap \Opp(u) \subset \{0\}\).

  Au final \(\Inv(u) \cap \Opp(u) = \{0\}\), cependant en général on n'a pas \(E = \Inv(u)+\Opp(u)\).
\end{proof}

\begin{prop}
  Soit un projecteur \(p \in \Endo{E}\), alors \(\Inv(p)=\Image(p)\) et \(\Opp(p)=\{0\}\).
\end{prop}
\begin{proof}
  On a \(\Inv(p)=\enstq{x \in E}{p(x)=x}=\Image(p)\), d'après la caractérisation de l'image d'un projecteur.

  Ensuite soit \(x \in \Opp(p)\) alors \(p(x)=-x\) et comme \(p\) est linéaire on a \(x=p(-x)\) et ainsi \(x \in \Image(p)=\Inv(p)\). Donc \(x \in \Opp(p) \cap \Inv(p)= \{0\}\).
  Alors \(\Opp(p) \cap \Inv(p) \subset \{0\}\) et donc \(\Opp(p) \subset {0}\) et c'est un sous-espace vectoriel l'autre inclusion est triviale. Au final  \(\Opp(p)=\{0\}\).
\end{proof}

\subsection{Homothéties vectorielles}

\begin{defdef}
  Soit un scalaire \(\lambda\), on appelle homothétie vectorielle de rapport \(\lambda\) l'application \(\fonction{h_\lambda}{E}{E}{x}{\lambda x}\)
\end{defdef}

\begin{prop}
  Soient \(\lambda, \mu\) des scalaires et \(\lambda_0\) un scalaire non nul, alors
  \begin{enumerate}
  \item \(h_\lambda \in \Endo{E}\);
  \item \(h_\lambda \circ h_\mu = h_{\lambda \mu} = h_{\mu \lambda} = h_\mu \circ h_\lambda\);
  \item \(h_{\lambda_0} \in \GL{E} \ (h_{\lambda_0})^{-1}=h_{\lambda_0^{-1}}\);
  \item l'application \(\fonction{\Phi}{\K\setminus\{0\}}{\GL{E}}{\lambda}{h_\lambda}\) est un morphisme de groupes du groupe \((\K\setminus\{0\}, \times)\) sur le groupe \((\GL{E}, \circ)\). Soit \(\H=\Image(\Phi)\), alors \(\H\) est un sous-groupe de \((\GL{E}, \circ)\) isomorphe à \((\K\setminus\{0\}, \times)\).
  \end{enumerate}

\end{prop}

\begin{proof}
  Les trois premiers points ont déjà vu à la sous-sous section~
\ref{subsec:exemples}.

  Pour le quatrième point, l'application \(\Phi\) est bien définie d'après le troisième point. C'est un morphisme grâce au deuxième point. Déjà \(\Phi^{|\H}\) est surjectif par définition. Montrons qu'il est injectif.

Déjà \(\Ker(\Phi)=\enstq{\lambda \in \K}{\Phi(\lambda)=\Id}\) et soit un scalaire non nul \(\lambda\), alors
\begin{align}
  \Phi(\lambda) = 1 &\iff \forall x \in E \quad \lambda x=x\\
  &\iff \forall x \in E \quad (\lambda-1)x=0\\
  &\iff \lambda=1.
\end{align}
Alors  \(\Ker(\Phi)=\{1\}\) donc \(\Phi\) est injectif.

Finalement \(\Phi\) induit une bijection de \(\K\setminus\{0\}\) sur \(\H=\Image(\Phi)\).
\end{proof}

\subsection{Symétries vectorielles}

\begin{defdef}
  On appelle symétrie vectorielle de \(E\) tout endomorphisme involutif \(s\) de \(E\), c'est-à-dire que \(s \in \Endo{E}\) et \(s \circ s = \Id\).
\end{defdef}

\begin{prop}
  Soit une symétrie de \(s\) de \(E\), alors \(s \in \GL{E}\) et \(s^{-1}=s\).
\end{prop}
\begin{proof}
  Trivial
\end{proof}

\subsubsection{Des projecteurs aux symétries}

Soit un projecteur \(p\) de \(E\), on dispose de \(E_1=\Image(p)\) et \(E_2=\Ker(p)\). On sait que \(E=E_1 \oplus E_2\) et que \(p\) est la projection sur \(E_1\) parallèlement à \(E_2\). On lui associe le projecteur \(q=\Id-p\).

\begin{prop}
  L'application \(s=p-q=2p-\Id\) est une symétrie vectorielle de \(E\), appelée la symétrie par rapport à \(E_1\) selon \(E_2\).
\end{prop}
\begin{proof}
  Comme \((p,q) \in \Endo{E}^2\) et que \(s=p-q\) alors \(s \in \Endo{E}\). De plus
  \begin{align}
    s \circ s &=(p-q) \circ (p-q)\\
    &=p \circ p -p \circ q -q \circ p +q \circ q\\
    &=p+q\\
    &=\Id.
  \end{align}
  Puisque \(\circ\) est distributive à droite et à gauche et puisque \(p\) et \(q\) sont des projecteurs.
\end{proof}

\emph{Remarque}~: L'application \(\sigma=q-p\) est la symétrie associée au projecteur \(q\), ou la symétrie par rapport à \(E_2\) selon \(E_1\).

\subsubsection{Des symétries aux projecteurs}

Soit une symétrie vectorielle \(s \in \Endo{E}\). Comme \(\K\) est \(\R\) ou \(\C\), le nombre \(\frac{1}{2}\) existe et on définit une application \(p = \frac{1}{2}(s+\Id)\).

\begin{prop}
  L'application \(p\) est un projecteur de \(E\) et \(s\) est la symétrie associée à \(p\).
\end{prop}
\begin{proof}
  Comme \(s, \Id \in \Endo{E}\) alors \(p \in \Endo{E}\). De plus
  \begin{align}
    p \circ p &= \frac{1}{2}(s+\Id) \circ \frac{1}{2}(s+\Id) \\
    &=\frac{1}{4}(s \circ s + s\circ \Id +\Id \circ s +\Id \circ \Id)\\
    &=\frac{1}{4}(\Id +s +s+\Id)\\
    &=\frac{1}{2}(s+\Id).
  \end{align}
\(p\) est donc un projecteur, \(s=2p-\Id\) donc \(s\) est la symétrie associée à \(p\).
\end{proof}

\subsubsection{\(\Inv(s)\) et \(\Opp(s)\) pour une symétrie \(s\)}

Si on garde les notations de la sous-sous section précédente, on a \(p,q\) des projecteurs et \(s\) une symétrie tels que \(p=\frac{1}{2}(s+\Id)\), \(q=p-\Id\) et \(E_1=\Image(p)\) et \(E_2=\Ker(p)\). Alors pour tout \(x \in E\)
\begin{align}
  x \in \Inv(s) &\iff s(x)=x \\
  &\iff p(x)-q(x)=p(x)+q(x) \\
  &\iff 2q(x)=0\\
  &\iff x \in \Ker(q).
\end{align}
Alors comme \(\Ker(q)=\Image(p)=E_1\) on a \(\Inv(s)=E_1\).
\begin{align}
   x \in \Opp(s) &\iff s(x)=-x \\
  &\iff p(x)-q(x)=-p(x)-q(x) \\
  &\iff 2p(x)=0\\
  &\iff x \in \Ker(p).
\end{align}
Alors comme \(\Ker(p)=E_2\) on a \(\Opp(s)=E_2\).
%
\begin{prop}
  Soit une symétrie vectorielle \(s\) de \(E\), alors
\begin{equation}
  E=\Inv(s) \oplus \Opp(s).
\end{equation}
\end{prop}
\begin{proof}
  Si on note \(p\) le projecteur associé à \(s\), on a montré juste avant que \(\Opp(s)=\Ker(p)\) et \(\Inv(s)=\Image(p)\) et on sait que l'image et le noyau d'un projecteur sont supplémentaires (proposition~
\ref{prop:projecteursupplementaire}).
\end{proof}

\emph{Exemple}~: Soit l'application \(\fonction{s}{E}{E}{x}{-x}\), donc \(s=-\Id \in \Endo{E}\). \(s \circ s = (-\Id) \circ (-\Id) = \Id\) donc \(s\) est une symétrie. Le projecteur auquel \(s\) est associée est nul : \(p=\frac{1}{2}(s+\Id)=0\). Alors \(E_1=\Image(p)=\{0\}\) et \(E_2=\Ker(p)=E\) et alors \(\Opp(s)=\Ker(p)=E\) et \(\Inv(s)=\Image(p)=\{0\}\).

\subsection{Affinités vectorielles}
\subsubsection{Des projecteurs aux affinités}
\begin{lemme}\label{lemme:affinites}
  Soient \(E\) un \(\K\)-espace vectoriel et \(E_1,E_2\) deux sous-espaces vectoriels supplémentaires de \(E\), alors une application linéaire de \(E\) est parfaitement définie par son action sur \(E_1\) et sur \(E_2\). C'est-à-dire que
  \begin{equation}
    \forall (u_1,u_2) \in \Endo{E}^2 \ \exists! u \in \Endo{E} \quad u_{|E_1}={u_1}_{|E_1} \ u_{|E_2}={u_2}_{|E_2}.
  \end{equation}
  On note \(p_1\) la projection sur \(E_1\) parallélement à \(E_2\) et \(p_2=\Id-p_1\) la projection sur \(E_2\) parallélement à \(E_1\).
\end{lemme}
\begin{proof}[Unicité]
  Soit \(u \in \Endo{E}\) tel que \(u_{|E_1}={u_1}_{|E_1}\) et  \(u_{|E_2}={u_2}_{|E_2}\). Pour tout \(x \in E\), on a~:
  \begin{align}
    x&=p_1(x)+p_2(x)\\
    u(x)&=u(p_1(x))+u(p_2(x))\\
    &=u_1(p_1(x)) +u_2(p_2(x))\\
    &=[u_1\circ p_1 +u_2 \circ p_2](x).
  \end{align}
  Donc nécessairement \(u=u_1\circ p_1 +u_2 \circ p_2\) et il y a unicité.
\end{proof}
\begin{proof}[Existence]
  Soit \(u=u_1\circ p_1 +u_2 \circ p_2\). Puisque \(u_1,p_1,u_2,p_2 \in \Endo{E}\) on a \(u \in \Endo{E}\).
  \begin{itemize}
  \item Si \(x \in E_1\) alors \(p_1(x)=x\) et \(p_2(x)=0\) donc \(u(x)=u_1(x)+u_2(0)=u_1(x)\);
  \item Si \(x \in E_2\) alors \(p_1(x)=0\) et \(p_2(x)=x\) donc \(u(x)=u_1(0)+u_2(x)=u_2(x)\).
  \end{itemize}
  Donc \(u_{|E_1}={u_1}_{|E_1} \ u_{|E_2}={u_2}_{|E_2}\).
\end{proof}
%
\begin{defdef}\label{def:affinites}
  Soient \(E_1\) et \(E_2\) deux supplémentaires dans \(E\) et un scalaire \(\alpha\). On appelle affinité de rapport \(\alpha\), de base \(E_1\) selon \(E_2\) l'unique application linéaire \(a_\alpha \in \Endo{E}\) telle que \({a_\alpha}_{|E_1} = \Id_{|E_1}\) et \({a_\alpha}_{|E_2} = {h_\alpha}_{|E_2}\).
\end{defdef}
%
\begin{prop}
  Avec les notations du lemme~
\ref{lemme:affinites} et de la définition~
\ref{def:affinites}, \(a_{\alpha}=p_1+\alpha p_2\).
\end{prop}
\begin{proof}
  Puisque \(E_1\) et \(E_2\) sont supplémentaires, alors \(a_\alpha= {a_\alpha}_{|E_1} + {a_\alpha}_{|E_2}\), c'est à dire avec les notations de la définition \(a_\alpha= \Id \circ p_1 + h_\alpha \circ p_2\) et donc pour tout \(x \in E\) \(a_\alpha(x)=p_1(x)+\alpha p_2(x)\).
\end{proof}

\emph{Cas particuliers}~: Si \(E_1=E\) et \(E_2=\{0\}\) alors \(a_\alpha=\Id\) et si c'est le contraire, \(E_1=\{0\}\) et \(E_2=E\) alors \(a_\alpha=h_\alpha\). \(\alpha_{-1}=p_1-p_2\) est une symétrie.

\begin{prop}[Propriétés des affinités vectorielles]
  On considère toujours le même couple de sous-espaces vectoriels supplémentaire, \(E=E_1 \oplus E_2\) et on considère les affinités de base \(E_1\) selon \(E_2\). Alors~:
  \begin{gather}
    \forall (\alpha,\beta) \in \K^2 \quad a_{\alpha} \circ a_{\beta} = a_{\alpha \beta} = a_{\beta} \circ a_{\alpha} ;\\
    \forall \alpha \in \K\setminus\{0\} \quad a_\alpha \in \GL{E} \ (a_\alpha)^{-1}=a_{\alpha^{-1}} ; \\
    \forall \alpha \in \K \quad (a_\alpha-\Id) \circ (a_\alpha -\alpha \Id)=0 ; \\
    \forall \alpha \in \K\setminus\{1\} \quad E=\Ker(a_\alpha - \Id) \oplus \Ker(a_\alpha - \alpha \Id).
  \end{gather}
\end{prop}
\begin{proof}
    \begin{align}
      a_{\alpha} \circ a_{\beta} &= (p_1+\alpha p_2) \circ (p_1+\beta p_2) \\
      &= p_1 \circ p_1 + \beta p_1 \circ p_2 + \alpha p_2 \circ p_1 +\alpha \beta p_2 \circ p_2 \\
      &=p_1+ \alpha\beta p_2\\
      &=a_{\alpha\beta}
    \end{align}
    puisque \(p_1\) et \(p_2\) sont des projecteurs.

    Si \(\alpha\) est non nul, alors il est inversible d'invers \(\alpha^{-1}\) et on a
    \begin{equation}
      a_\alpha \circ a_{\alpha^{-1}} = a_{\alpha \alpha^{-1}} = \alpha_1 = p_1+p_2=\Id
    \end{equation}
    donc \(a_\alpha \in \GL{E}\) et \((a_\alpha)^{-1}=a_{\alpha^{-1}}\).

    On a \(a_\alpha - \Id=p_1+\alpha p_2 -(p_1+p_2) = (\alpha-1)p_2\) et \(a_\alpha -\alpha \Id = p_1+\alpha p_2 -\alpha(p_1+p_2) = (1-\alpha)p_1\). Alors
    \begin{equation}
      (a_\alpha-\Id) \circ (a_\alpha -\alpha \Id) = -(\alpha-1)^2 p_2 \circ p_1=0.
    \end{equation}

    On a \(a_\alpha-\Id = (\alpha-1)p_2\) et comme \(\alpha \neq 1\) on a \(\Ker(a_\alpha-\Id)=\Ker(p_2)=E_1\). De la même manière \(a_{\alpha}-\alpha \Id = (1-\alpha)p_1\) et comme \(\alpha \neq 1\) on a \(\Ker(a_\alpha-\alpha\Id)=\Ker(p_1)=E_2\). Comme \(E=E_1 \oplus E_2\) alors \(E=\Ker(a_\alpha - \Id) \oplus \Ker(a_\alpha - \alpha \Id)\).
\end{proof}

\emph{Remarques}~: La troisième propriété fait appel à des notions de polynôme annulateur. Si \(p\) est est projecteur alors \(p \circ p=p\) et on dit que \(X^2-X\) est annulateur de \(p\). Si \(s\) est une symétrie, comme \(s \circ s =\Id\) alors son polynôme annulateur est \(X^2-1\). Si \(a_\alpha\) est une affinité de rapport \(\alpha\) alors \((X-1)(X-\alpha\) est son polynôme annulmateur. La quatrième propriété est en fait une conséquence de la troisième et du théorème de décomposition des noyaux, car les polynômes sont premiers entre eux cf.\ chapitre~
\ref{chap:polynomes}.

\begin{prop}
  Soit un scalaire \(\alpha \in \K\setminus\{1\}\), si une application \(a \in \Endo{E}\) vérifie
  \begin{equation}
    (a-\Id)\circ (a-\alpha \Id)=0,
  \end{equation}
  alors \(a\) est une affinité vectorielle de rapport \(\alpha\).
\end{prop}
\begin{proof}
  Comme \(1-\alpha\neq 0\), on peut poser \(p_1=\frac{1}{1-\alpha} (a-\alpha\Id)\). Cette application est un endomorphisme de \(E\) et de plus
  \begin{align}
    p_1 \circ p_1 &= \frac{1}{1-\alpha} (a-\alpha\Id) \circ \frac{1}{1-\alpha} (a-\alpha\Id) \\
    &=\frac{1}{(1-\alpha)^2} (a-\alpha\Id) \circ (a-\alpha\Id) \\
    &=\frac{1}{(1-\alpha)^2} (a-\Id +(1-\alpha)\Id) \circ (a-\alpha\Id) \\
    &=\frac{1}{(1-\alpha)^2} (a-\Id) \circ (a-\alpha\Id) + \frac{1-\alpha}{(1-\alpha)^2}(a-\alpha\Id) \\
    &= 0 + \frac{1}{(1-\alpha)}(a-\alpha\Id)\\
    &=p_1.
  \end{align}
  Alors \(p_1\) est un projecteur. Ainsi \(E_1=\Image(p_1)\) et \(E_2=\Ker(p_1)\) et ils sont supplémentaires dans \(E\). Alors
  \begin{equation}
    a=(1-\alpha)p_1 +\alpha \Id = p_1 +\alpha(\Id-p_1) = p_1+\alpha p_2,
  \end{equation}
  avec \(p_2\) le projecteur associé à \(p_1\). L'endomorphisme \(a\) est donc une affinité vectorielle.
\end{proof}

\subsubsection{\(\Inv(a)\) et \(\Opp(a)\) pour une affinité \(a\)}

Soient, comme d'habitude, \(E_1\) et \(E_2\) deux sous-espace vectoriels supplémentaires de \(E\). On note \(p_1\) la projection sur \(E_1\) parallélement à \(E_2\) et \(p_2\) la projection sur \(E_2\) parallélement à \(E_1\). Pour tout scalaire \(\alpha\), on note \(a_\alpha = p_1+\alpha p_2\) l'affinité vectorielle de rapport \(\alpha\) de base \(E_1\) selon \(E_2\). Alors
\begin{align}
  \forall x \in E \quad x \in \Inv(a_\alpha) &\iff a_\alpha(x)=x \\
  &\iff p_1(x)+\alpha p_2(x) = p_1(x)+p_2(x) \\
  &\iff (\alpha-1) p_2(x)=0.
\end{align}
Si \(\alpha=1\) alors \(a_1=\Id\) et \(\Inv(a_1)=E\). Sinon, alors \(\Inv(a_\alpha)=\Ker(p_2)=\Image(p_1)=E_1\), et
\begin{align}
  \forall x \in E \quad x \in \Opp(a_\alpha) &\iff a_\alpha(x)=-x \\
  &\iff p_1(x)+\alpha p_2(x) = -p_1(x)-p_2(x) \\
  &\iff 2p_1(x)=-(\alpha+1)p_2(x).
\end{align}
Or \(2p_1(x) \in E_1\) et \(-(\alpha+1)p_2(x) \in E_2\). Puisque ces sous-espaces vectoriels sont supplémentaires alors leur intersection est nulle donc chacun des termes est nul. Donc
\begin{equation}
  \forall x \in E \quad x \in \Opp(a_\alpha) \iff \begin{cases} 2p_1(x) &=0 \\-(\alpha+1)p_2(x)&=0 \end{cases}
\end{equation}
Deux cas se présentent~:
\begin{itemize}
\item si \(\alpha=-1\) alors \(x \in \Opp(a_\alpha) \iff 2p_1(x)=0 \iff x \in \Ker(p_1)=E_2\);
\item sinon alors \(x \in \Opp(a_\alpha) \iff \begin{cases} x \in \Ker(p_1) \\ x \in \Ker(p_2) \end{cases} \iff x=0\).
\end{itemize}
Finalement
\begin{itemize}
\item si \(\alpha \neq 1\) et \(\alpha \neq -1\) alors \(\Inv(a_\alpha)=E_1\) et \(\Opp(a_\alpha)=\{0\}\);
\item si \(\alpha =1\) alors \(\Inv(a_{-1})=E_1\) et \(\Opp(a_{-1})=E_2\) et on a bien les propriétés d'une symétrie \(E=\Inv(a_{-1}) \oplus \Opp(a_{-1})\);
\item si \(\alpha=1\) alors  \(\Inv(a_{1})=E\) et \(\Opp(a_{1})=\{0\}\) et c'est normal parce que \(a_1=\Id\).
\end{itemize}

\section{Translation, sous-espace affine}

\subsection{Notion d'espace affine}

\begin{defdef}
  On appelle espace affine tout couple \((\A,E)\) où~:
  \begin{enumerate}
  \item \(\A\) est un ensemble non vide ;
  \item \((E,+,\perp)\) est un \(\R\)-espace vectoriel ;
  \item Il existe une application \(\fonction{+}{\A\times E}{E}{(A,x)}{A+x}\) telle que~:
    \begin{enumerate}
    \item \(\forall A \in \A \quad \fonction{+_A}{E}{E}{x}{A+x}\) est bijective;
    \item \(\forall A \in \A \ \forall (x,y) \in E^2 \quad (A+x)+y = A+(x+y)\).
    \end{enumerate}
  \end{enumerate}
\end{defdef}

\begin{prop}[Exemple fondamental d'espace affine]
  Soit un \(\R\)-espace vectoriel \((E,+,\perp)\). Alors le couple \((E,E)\) muni de la loi \(+\) de \(E\) est un espace affine. On dit que le \(\R\)-espace vectoriel \(E\) est muni de sa structure d'espace affine canonique.
\end{prop}
\begin{proof}
  \begin{enumerate}
  \item \(E\) est non vide, puisque c'est un espace vectoriel ;
  \item \(E\) est un \(\R\)-espace vectoriel ;
  \item Soit l'application \(\fonction{+}{E\times E}{E}{(A,x)}{A+x}\), soit \(A \in E\) alors
    \begin{enumerate}
    \item l'application \(x \longmapsto A+x\) est injective
      \begin{equation}
        \forall (x,y) \in E^2 \quad A+x=A+y \implies x=y,
      \end{equation}
      puisque \((E,+)\) est un groupe et donc tous les éléments sont simplifiables pour \(+\).
    \item l'application \(x \longmapsto A+x\) est surjective
      \begin{equation}
        \forall B \in E \ \exists x \in E \quad A+x=B,
      \end{equation}
      en prenant \(x=B-A \in E\) puisque \(E\) est un espace vectoriel.
    \item associativité de l'addition dans \(E\).
    \end{enumerate}
  \end{enumerate}
  Alors le couple \((E,E)\) muni de la loi \(+\) de \(E\) est un espace affine.
\end{proof}

Dans toute la suite, \((E,+,\perp)\) est un \(\R\)-espace vectoriel muni de sa structure canonique d'espace affine. Si \((A,x) \in E\times E\), \(A\) est appelé un point et \(x\) est appelé un vecteur. On représentera en général les points par une majuscule et les vecteurs par une minuscule.

\subsection{Translations}

\subsubsection{Equipollence -- Relation de Chasles}

On dispose de l'application \(\fonction{+}{E\times E}{E}{(A,x)}{A+x}\).

\begin{defdef}
  Soit un point \(A \in E\). Pour tout point \(B \in E\), il existe un unique vecteur \(x \in E\) tel que \(A+x=B\), on note \(x=\vect{AB}\). Ainsi \(B=A+\vect{AB}\).
\end{defdef}
\begin{defdef}
  Soients quatre points \(A,B,C\) et \(D\) de \(E\). Alors les bipoints \((A,B)\) et \((C,D)\) sont équipollents si et seulement si \(\vect{AB}=\vect{CD}\).
\end{defdef}

\emph{Remarque}~:la relation d'équipollence est en fait une relation d'équivalence (réflexive, symetrique, transitive).

\begin{prop}
  Soient trois points \(A,B\) et \(C\) de \(E\), alors
  \begin{enumerate}
  \item \(\vect{AB}=0 \iff A=B\);
  \item \(\vect{BA}=-\vect{AB}\);
  \item \(\vect{AB}+\vect{BC}=\vect{AC}\), c'est la relation de Chasles.
  \end{enumerate}
\end{prop}

\emph{Remarque}~:Si on fixe une origine \(A\) de l'espace \(\E\) on pourra identifier un point \(B\) avec le vecteur \(\vect{AB}\). En général \(A=0\) et on identifiera le vecteur \(\vect{0B}\) au point \(B\).

\subsubsection{Translations}

\begin{defdef}
  Soit \(x_0 \in E\). On appelle translation de vecteur \(x_0\) l'application \(\fonction{t_{x_0}}{E}{E}{A}{A+x_0}\). On note \(\tau(E)\) l'ensemble des translations de l'espace affine \(E\), \(\tau(E)=\{t_{x_0}, x_0 \in E\}\).
\end{defdef}

\begin{prop}\label{prop:associativitevec}
  Pour tous vecteurs \(x_0, y_0\) de \(E\)
  \begin{equation}
    t_{x_0} \circ t_{y_0} = t_{x_0 +y_0} = t_{y_0} \circ t_{x_0}.
  \end{equation}
\end{prop}
\begin{proof}
  Soit un point \(A\) de \(E\), alors
  \begin{equation}
    t_{x_0} \circ t_{y_0} (A)=t_{x_0}(A+y_0) = (A+y_0)+x_0 = A+(x_0+y_0) =t_{x_0+y_0}(A),
  \end{equation}
  puisque la loi \(+\) est associative. Ensuite de la même manière
  \begin{equation}
    t_{y_0} \circ t_{x_0} (A)=t_{y_0}(A+x_0) = (A+x_0)+y_0 = A+(x_0+y_0)=t_{x_0+y_0}(A).
  \end{equation}
\end{proof}

\begin{prop}
  Pour tout vecteur \(x_0 \in E\), la translation \(t_{x_0}\) est bijective et \((t_{x_0})^{-1}=t_{-x_0}\).
\end{prop}
\begin{proof}
  Il suffit d'appliquer la proposition~
\ref{prop:associativitevec} pour \(y_0=-x_0\).
\end{proof}

On peut considérer l'application \(\fonction{T}{E}{\sigma(E)}{x_0}{t_{x_0}}\) avec \(\sigma(E)\) l'ensemble des permutations de \(E\). On rappelle que \((\sigma(E),\circ)\) est un groupe. La proposition~
\ref{prop:associativitevec} peut se reformuler ainsi~: \(T\) est un morphisme du groupe \((E,+)\) sur le groupe \((\sigma(E),\circ)\). On note \(T(E)=\Image(T)\).

\begin{prop}
  \((T(E),\circ)\) est un sous-groupe abélien du groupe \((\sigma(E),\circ)\).
\end{prop}

\emph{Noyau du morphisme \(T\)}~:

\begin{equation}
  \Ker(T)= \enstq{x_0 \in E}{t_{x_0} = \Id},
\end{equation}
alors pour tout \(x_0 \in E\)
\begin{align}
  x_0 \in \Ker(T) &\iff \forall A \in E \quad t_{x_0}(A)=A\\
  &\iff \forall A \in E \quad A+x_0=A \\
  &\iff x_0=0.
\end{align}
Alors \(\Ker(T)=\{0\}\), alors \(T\) est injectif. Ainsi \(T\) induit un isomorphisme de \((E,+)\) sur \((T(E),\circ)\). On en déduit que
\begin{prop}
  Le groupe \((T(E),\circ)\) est isomorphe au groupe \((E,+)\).
\end{prop}
%
\begin{theo}
  Soit \(f \in E^E\), alors \(f\) est une translation si et seulement si pour tous points \(M\) et \(N\) de \(E\) \(\vect{f(M)f(N)}=\vect{MN}\)
\end{theo}
\begin{proof}
  Si \(f\) est une translation de \(E\), il existe un vecteur \(x_0 \in E\) tel que \(f=t_{x_0}\) et
  \begin{equation}
    \forall (M, N) \in E^2 \quad f(M)=M+x_0 \text{~et~} f(N)=N+x_0.
  \end{equation}
  donc \(\vect{f(M)f(N)}=f(N)-f(M)=(N+x_0)-(M+x_0)=N-M=\vect{MN}\).

  Soit \(A \in E\) et on pose \(B=f(A)\) et \(x_0=\vect{AB}\). Montrons que \(f=t_{x_0}\). Pour tout point \(M \in E\) on a par hypothèse
  \begin{equation}
    \vect{f(A)f(M)}=\vect{AM},
  \end{equation}
  alors
  \begin{align}
   f(M)&=f(A)+\vect{AM} \\
   f(M)&=B+(\vect{AB}+\vect{BM}) \\
   f(M)&=(B+\vect{BM})+\vect{AB} \\
   f(M)&=M+x_0\\
   f(M)&=t_{x_0}(M).
  \end{align}
  Alors \(f=t_{x_0}\) est une translation.
\end{proof}

\subsection{Sous-espaces affines}

\subsubsection{Définition}

\begin{defdef}
  On appelle sous-espace affine de \(E\) toute partie \(W_1\) de \(E\) telle que~:
  \begin{enumerate}
  \item il existe un point \(A_1\) de \(E\);
  \item il existe un sous-espace vectoriel \(E_1\) de \(E\);
  \end{enumerate}
  vérifiant \(W_1=\enstq{A_1+x}{x \in E_1} = \{A_1\}+E_1\). On notera \(W_1=A_1+E_1\).
\end{defdef}
%
\begin{prop}
  Si \(W_1=A_1+E_1\) est un sous-espace affine de \(E\), alors le point \(A_1\) appartient à \(W_1\).
\end{prop}
\begin{proof}
  \(E_1\) est un sous-espace vectoriel de \(E\) donc \(0 \in E_1\) et \(A_1=A_1 +0 \in W_1\).
\end{proof}
%
\begin{prop}
  Pour tout point \(A \in E\), le singleton \(\{A\}\) est un sous-espace affine.
\end{prop}
\begin{proof}
  \(\{0\}=E_1\) est un sous-espace vectoriel de \(E\) donc \(\{A_1\} = A_1 +\{0\}\) est un sous-espace affine de \(E\).
\end{proof}
%
\begin{prop}
  \begin{itemize}
  \item Les sous-espace vectoriels sont des sous-espaces affines;
  \item si \(W_1=A_1+E_1\) est un sous-espace affine, alors c'est un sous-espace vectoriel si et seulement s'il contient \(0\).
  \end{itemize}
\end{prop}
\begin{proof}
  \begin{itemize}
  \item Soit \(E_1\) un sous-espace vectoriel de \(E\), alors \(E_1=\{0\}+E_1\), c'est donc un sous-espace affine;
  \item soit \(W_1=A_1+E_1\) un sous-espace affine~: \(\implies\) Si \(W_1\) est un sous-espace vectoriel, alors il contient \(0\);

    \(\impliedby\) Si \(0 \in W_1\), il existe \(x \in E_1\) tel que \(A_1+x=0\) et donc \(A_1 = -x \in E_1\) car c'est un sous-espace vectoriel.

    Montrons que \(W_1=E1\). Si \(A \in W_1\), il existe alors \(x \in E_1\) tel que \(A=A_1+x \in E_1\) donc \(W_1 \subset E_1\). Si \(A \in E_1\), alors \(A=A_1 +\vect{A_1A}\) et comme \(A_1\) et \(A\) sont dans \(E\), le vecteur \(\vect{A_1A} \in E\) donc \(A \in W_1\). Ainsi \(E_1 \subset W_1\). Par double inclusion \(W_1=E1\)
  \end{itemize}
\end{proof}

\paragraph{Choix du point \(A_1\)}
\begin{prop}

Soit \(W_1=A_1+E_1\) un sous-espace affine. Pour tout point \(A' \in E\), \(A' \in W_1\) si et seulement s'il existe un sous-espace vectoriel \(E'\) de \(E\) tel que \(W_1=A'+E'\).
\end{prop}
\begin{proof}
  Considérons que \(A' \in W_1\) alors \(A'=A'+0 \in W_1\).

  Soit maintenant \(A' \in W_1\), il existe un vecteur \(x \in E_1\) tel que \(A'=A+x\). Montrons que \(W_1=A'+E_1\). Soit \(M \in W_1\), alors il existe \(y \in E_1\) tel que \(M=A_1+y=(A'-x)+y=A'+(-x+y)\). Comme \(x,y \in E_1\) et puisque c'est un sous-espace vectoriel on a \(-x+y \in E_1\) donc \(M \in A'+E_1\). Alors \(W_1 \subset A'+E_1\). Soit \(M \in A'+E_1\), alors il existe \(y \in E_1\) tel que \(M=A'+y=(A_1+x)+y=A_1+(x+y)\). Comme \(x,y \in E_1\) et puisque c'est un sous-espace vectoriel on a \(x+y \in E_1\) donc \(M \in A_1+E_1=W_1\). Donc \(A'+E_1 \subset W_1\).

  Finalement par double inclusion \(W_1=A'+E_1\).
\end{proof}

\paragraph{Choix du sous-espace vectoriel}
\begin{prop}
  Soit \(W_1=A_1+E_1\) un sous-espace affine. Alors
  \begin{align}
    E_1 &=\enstq{\vect{A_1M_1}}{M_1 \in W_1}\\
    &=\enstq{\vect{M_1N_1}}{M_1,N_1 \in \W_1}.
  \end{align}
\end{prop}

\begin{proof}
  Soit \(x \in E_1\) et \(M_1=A_1+x \in W_1\) avec \(x=\vect{A_1M_1}\). Donc \(E_1 \subset \enstq{\vect{A_1M_1}}{M_1 \in W_1}\). Soit \(x \in \enstq{\vect{A_1M_1}}{M_1 \in W_1}\) alors \(M_1=A_1+x\), or \(M_1 \in W_1\) donc il existe \(y \in E_1\) tel que \(M_1=A_1+y\). Ainsi \(x=y \in E_1\). Donc \(\enstq{\vect{A_1M_1}}{M_1 \in W_1} \subset E_1\). Finalement par double inclusion \(E_1 =\enstq{\vect{A_1M_1}}{M_1 \in W_1}\).

Comme \(A_1 \in W_1\), on a \(\enstq{\vect{A_1M_1}}{M_1 \in W_1} \subset \enstq{\vect{M_1N_1}}{(M_1,N_1) \in \W_1^2}\). Soit maintenant \(x \in \enstq{\vect{M_1N_1}}{(M_1,N_1) \in \W_1^2}\), alors il existe \((M_1, N_1) \in W_1^2\) tel que \(x=\vect{M_1N_1}\) et il existe \((y,z) \in E_1^2\) tels que \(M_1=A_1+y\) et \(N_1=A_1+z\) donc \(x=z-y \in E_1\) puisque c'est un sous-espace vectoriel. Alors \(\enstq{\vect{M_1N_1}}{(M_1,N_1) \in \W_1^2} \subset E_1\). Par double inclusion, on a bien \(E_1=\enstq{\vect{M_1N_1}}{(M_1,N_1) \in \W_1^2}\).
\end{proof}
%
\begin{cor}
  Pour tout sous-espace affine \(W_1\), il existe un unique sous-espace vectoriel \(E_1\) tel que \(W_1\) puisse s'écrire
  \begin{equation}
    W_1=A_1+E_1.
  \end{equation}
  Ce sous-espace vectoriel est la direction de \(W_1\) et on dit que c'est le sous-espace affine passant par \(A\) et dirigé par \(E_1\).
\end{cor}

\subsubsection{Parallélisme de sous-espaces affines}

\begin{defdef}
  Soient \(W_1\) et \(W_2\) deux sous-espaces affines de directions respectives \(E_1\) et \(E_2\). Alors
  \begin{enumerate}
  \item \(W_1\) est parallèle à \(W_2\) si et seulement si \(E_1 \subset E_2\) ;
  \item \(W_1\) et \(W_2\) sont parallèles si et seulement si \(E_1=E_2\).
  \end{enumerate}
\end{defdef}
%
\begin{prop}
  Dans l'ensemble des sous-espaces affines de \(E\)
  \begin{enumerate}
  \item la relation ``est parallèle à'' est réflexive et transitive;
  \item la relation \(\RelBin{}\) définie par \(W_1\RelBin{}W_2\) par ``\(W_1\) et \(W_2\) sont parallèles'' est une relation d'équivalence.
  \end{enumerate}
\end{prop}
\begin{proof}
  \begin{enumerate}
  \item Si \(W_1\) est un sous-espace affine, \(W_1\) est parallèle à lui-même car \(E_1 \subset E_1\), donc elle est symétrique. Si \(W_1\) est parallèle à \(W_2\) et si \(W_2\) est parallèle à \(W_3\) alors \(E_1 \subset E_2 \subset E_3\) donc \(W_1\) est parallèle à \(W_3\), donc elle est transitive.
  \item Puisque \(E_1=E_1\) alors \(W_1\RelBin{}W_1\), symétrique. Si \(W_1 \RelBin{}W_2\) alors \(W_1=W_2\) donc \(W_2\RelBin{}W_1\), symétrique. Si \(W_1 \RelBin{}W_2\) et \(W_2\RelBin{}W_3\) alors \(E_1=E_2=E_3\) donc \(W_1\RelBin{}W_3\).
  \end{enumerate}
\end{proof}
%
\begin{prop}
  Soient deux sous-espaces affines \(W_1\) et \(W_2\) alors
  \begin{equation}
    W_1 \subset W_2 \implies W_1 \text{~est parallèle à } W_2.
  \end{equation}
\end{prop}
\begin{proof}
  Notons \(E_1\) et \(E_2\) les directions respectives de \(W_1\) et \(W_2\). Soit \(I \in W_1\), alors \(W_1 = I+E_1\). Puisque \(W_1 \subset W_2\) on a\(I \in W_2\) donc \(W_2 = I+E_2\). Soit \(x \in E_1\) alors \(I+x \in W_1 \subset W_2\) donc il existe \(y \in E_2\) tel que \(I+x=I+y\) donc \(x=y \in E_2\). Finalement \(E_1 \subset E_2\). Donc \(W_1\) est parallèle à \(W_2\).
\end{proof}

\begin{prop}
  Soient deux sous-espaces affines \(W_1\) et \(W_2\) alors
  \begin{equation}
    W_1 \subset W_2 \iff \begin{cases} W_1 \text{~est parallèle à } W_2 \\ W_1 \cap W_2 \neq \emptyset \end{cases}
  \end{equation}
\end{prop}
\begin{proof}
  \(\implies\)~: Déjà vu dans la proposition précédente et \(W_1 \cap W_2=W_1 \neq \emptyset\).

  \(\impliedby\)~: Puisque \(W_1\) est parallèle à \(W_2\), on a \(E_1 \subset E_2\). Comme \(W_1 \cap W_2 \neq \emptyset\), il existe \(I \in W_1 \cap W_2\) tel que \(W_1=I+E_1\) et \(W_2=I+E_2\) et comme \(E_1 \subset E_2\) alors \(W_1 \subset W_2\).
\end{proof}

\subsubsection{Intersection de sous-espaces affines}

Soient \(W_1\) et \(W_2\) deux sous-espaces affines de direction respectives \(E_1\) et \(E_2\).

\begin{prop}\label{prop:interssaffines}
  Si \(W_1 \cap W_2 \neq \emptyset\), alors \(W_1 \cap W_2\) est un sous-espace affine. De plus si pour tout \(I \in W_1 \cap W_2\) \(W_1 \cap W_2 = I+E_1 \cap E_2\).
\end{prop}

Si \(W_1\cap W_2=\emptyset\), alors ce n'est pas un sous-espace effine puisqu'un sous-espace affine n'est jamais vide.

\begin{proof}
  Soit \(I \in W_1 \cap W_2\). On sait que \(I+\E_1 \cap E_2\) est un sous-espace affine car \(E_1\cap E_2\) est un sous-espace vectoriel. Montrons que \(W_1 \cap W_2 = I +E_1 \cap E_2\).

Soit \(M \in W_1 \cap W_2\), alors \(M \in I+E_1\) donc il existe \(x \in E_1\) tel que \(M=I+x\) et \(M \in I+E_2\) aussi donc il existe \(y \in E_2\) tel que \(M=I+y\). D'où \(x=y=\vect{IM}\) alors \(x=y \in E_1 \cap E_2\), alors \(M=I+x \in I+E_1 \cap E_2\).  On a montré l'inclusion \(W_1 \cap W_2 \subset E_1 \cap E_2\).

Soit \(M \in I+E_1 \cap E_2\). Il existe \(x \in E_1 \cap E_2\) tel que \(M=I+x\). Comme \(x \in E_1\) alors \(M \in I+E_1=W_1\) et aussi \(x \in E_2\) alors \(M \in I+E_2=W_2\). Ainsi \(M \in W_1 \cap W_2\). On a montré l'inclusion  \(E_1 \cap E_2 \subset W_1 \cap W_2 \).

Finalement par double inclusion \(W_1 \cap W_2 = I +E_1 \cap E_2\).
\end{proof}

\begin{prop}[Cas particulier]
  Si \(E_1\) et \(E_2\) sont supplémentaires dans \(E\), alors \(W_1 \cap W_2\) est un singleton.
\end{prop}
\begin{proof}
  \(W_1\) et \(W_2\) sont tous deux non vides car ce sont des sous-espaces affines. Soit \((A_1,A_2) \in W_1 \times W_2\) et \(\vect{A_1A_2} \in E\).

Or \(E=E_1 \oplus E_2\) donc il existe un unique couple \((x_1,x_2) \in E_1 \times E_2\) tel que \(\vect{A_1A_2}=x_1+x_2\) alors \(A_2-x_2=x_1+A_1\). Comme \(A_2\) et \(x_2\) sont dans \(W_2\) alors \(A_2-x_2 \in W_2\) et comme \(A_1\) et \(x_1\) sont dans \(W_1\) alors \(A_1+x_1 \in W_1\). Si on pose \(I=A_2-x_2=x_1+A_1\) alors \(I \in W_1\cap W_2\). D'après la propostion~
\ref{prop:interssaffines}, \(W_1 \cap W_2\) est un sous-espace affine et \(W_1 \cap W_2 = I+E_1 \cap E_2 = I+\{0\}=\{I\}\).
\end{proof}

L'union de sous-espaces affines n'est pas, en général, un sous-espace affine.

\begin{prop}
  Si \(W_1 \cap W_2 \neq \emptyset\), pour tout \(I \in W_1 \cup W_2\), \(I+(E_1+E_2)\) est le plus petit sous-espace affine contenant \(W_1 \cup W_2\).
\end{prop}
\begin{proof}
  Soit \(A \in W_1 \cap W_2\). Alors~:
  \begin{itemize}
  \item \(W=A+(E_2+E_2)\) est un sous-espace affine puisque \(E_1+E_2\) est un sous-espace vectoriel;
  \item \(W_1=A+E_1\) et \(W_2=A+E_2\), et comme \(E_1 \subset E_1+_2\) et \(E_2 \subset E_1+E_2\) alors \(W_1 \subset W\) et \(W_2 \subset W\), donc \(W_1 \cup W_2 \subset W\); et alors \(W\) est un sous-espace affine qui contient \(W_1 \cup W_2\);
  \item soit \(W'\) un sous-espace affine qui contient \(W_1 \cup W_2\) et soit \(E'\) sa direction; Puisque \(A \in W_1 \cap W_2 \subset W_1\cup W_2 \subset W'\) on peut écrire que \(W'=A+E'\); soit \(M \in W\) il existe alors \(x \in E_1+E_2\) tel que \(M=A+x\); il existe \(x_1 \in E_1\) et \(x_2 \in E_2\) tels que \(x=x_1+x_2\), donc \(M=A+x_1+x_2\); comme \(A+x_1 \in W_1 \subset W'=A+E'\) on a \(x_1 \in E'\); comme aussi \(A+x_2 \in W_2 \subset W'=A+E'\) on a \(x_2 \in E'\); enfin \(M=A+(x_1+x_2) \in A+E'=W'\) alors \(W \subset W'\).
  \end{itemize}
  On vient de montrer que pour tout \(I \in W_1 \cup W_2\), \(I+(E_1+E_2)\) est le plus petit sous-espace affine contenant \(W_1 \cup W_2\).
\end{proof}

\subsection{Barycentres et convexité}

\subsubsection{Barycentre}

\begin{defdef}
  On appelle un point pondéré dans \(E\) tout couple \((A,\lambda)\) où \(A\) est un point de \(E\) et \(\lambda\) un réel.
\end{defdef}

\begin{theo}
  Soient un entier naturel \(n\) non nul, \(\{(A_i,\lambda_i)\}_{i \in \llbracket 1,n \rrbracket}\) une famille de points pondérés de \(E\) tel que \(\sum_{i=1}^n \lambda_i \neq 0\) (masse totale non nulle). Alors il existe un unique point \(G\), appelé barycentre de la famille noté \(G=\Bary(A_i,\lambda_i)_{i=1, \ldots, n}\), vérifiant les propriétés équivalentes suivantes~:
  \begin{enumerate}
  \item \(\sum_{i=1}^n \vect{GA_i}=0\);
  \item \(\forall A \in E \sum_{i=1}^n \lambda_i \vect{AG}=\sum_{i=1}^n \lambda_i \vect{AA_i}\);
  \item \(\exists A \in E \sum_{i=1}^n \lambda_i \vect{AG}=\sum_{i=1}^n \lambda_i \vect{AA_i}\).
  \end{enumerate}
\end{theo}
\begin{proof}
  L'égalité \(1\) est équivalente à \(\sum_{i=1}^n \lambda_i A_i = \left(\sum_{i=1}^n \lambda_i \right) G\) alors pour tout point \(A \in E\) on a
  \begin{equation}
    \sum_{i=1}^n \lambda_i A_i = \left(\sum_{i=1}^n \lambda_i \right) G \iff \sum_{i=1}^n \lambda_i \vect{AA_i} = \sum_{i=1}^n \lambda_i \vect{AG}
  \end{equation}
Les égalités \(2\) et \(3\) ne dépendent pas du point \(A\), donc \(1 \iff 2 \iff 3 \iff \left(\sum_{i=1}^n \lambda_i \right) G = \sum_{i=1}^n \lambda_i A_i\) et comme la masse totale est non nulle, le point \(G\) existe et est unique et vérifie \(G = \frac{\sum_{i=1}^n \lambda_i A_i}{\sum_{i=1}^n \lambda_i}\).
\end{proof}

\begin{prop}[Homogénéité]
  Soit une famille de points pondérés de \(E\) \((A_i,\lambda_i)_{i=1, \ldots, n}\) de masse totale non nulle. Soit un réel \(\lambda\) non nul.  Alors
  \begin{equation}
    \Bary(A_i,\lambda_i)_{i=1, \ldots, n} = \Bary(A_i,\lambda \lambda_i)_{i=1, \ldots, n}
  \end{equation}
\end{prop}

\begin{proof}
  Par hypothèse \(\sum_{i=1}^n \lambda_i \neq 0\) et comme \(\lambda \neq 0\) on a \(\sum_{i=1}^n \lambda \lambda_i \neq 0\). On dispose de
  \begin{align}
    G &= \Bary(A_i,\lambda_i)_{i=1, \ldots, n};\\
    G' &= \Bary(A_i,\lambda\lambda_i)_{i=1, \ldots, n}.
  \end{align}
  alors \(G'= \frac{\sum_{i=1}^n \lambda\lambda_i A_i}{\sum_{i=1}^n \lambda\lambda_i}=\frac{\sum_{i=1}^n \lambda_i A_i}{\sum_{i=1}^n \lambda_i}=G\).
\end{proof}

\emph{Conséquence Importante}~; Si \((A_i,\lambda_i)_{i=1, \ldots, n}\) est une famille de points pondérés de masse totale non nulle, on peut se ramener à \(G = \Bary(A_i,\lambda_i)_{i=1, \ldots, n}=\Bary(A_i,\alpha_i)_{i=1, \ldots, n}\) avec \(\sum_{i=1}^n \alpha_i = 1\) en prenant pour tout \(i \in \llbracket 1,n \rrbracket\) \(\alpha_i=\frac{\lambda_i}{\sum_{i=1}^n \lambda_i}\).

\begin{prop}[Associtivité]
  Soit une famille de points pondérés de \(E\) \((A_i,\lambda_i)_{i=1, \ldots, n}\) de masse totale non nulle. Soit \((I_j)_{j=1, \ldots, p}\) une partition de \(\llbracket 1,n \rrbracket\), ce qui signifie que
  \begin{align}
    \bigcup_{j=1}^p I_j &= \intervalleentier{1}{n}; \\
    \forall (j,k) \in &\intervalleentier{1}{n}^2 \quad j \neq k \implies I_j \cap I_k= \emptyset.
  \end{align}
On suppose de plus que chaque ``sous-masse'' est non nulle, c'est à dire que pour tout \(j \in \llbracket 1,p\rrbracket \), \(\mu_j=\sum_{i \in I_j} \lambda_i \neq 0\). On définit pour tout \(j \in \llbracket 1,p\) le barycentre \(G_j=\Bary(A_i,\lambda_i)_{i \in I_j}\). Alors
\begin{equation}
  \Bary(A_i,\lambda_i)_{i=1, \ldots, n} = \Bary(G_j, \mu_j)_{j=1, \ldots, p}.
\end{equation}
\end{prop}
\begin{proof}
  On pose
  \begin{align}
    G' &=\Bary(G_j,\mu_j )_{j=1, \ldots, p};\\
    G  &=\Bary(A_i,\lambda_i)_{i=1, \ldots, n}.\\
  \end{align}
  Alors pour tout \(j \in \intervalleentier{1}{p}\)~:
  \begin{align}
    G_j &= \frac{\sum_{i\in I_j} \lambda_i A_i}{\sum_{i \in I_j} \lambda_i} = \frac{1}{\mu_j} \sum_{i\in I_j} \lambda_i A_i;\\
    G' &= \frac{\sum_{j=1}^p \mu_j G_j}{\sum_{j=1}^p \mu_j} = \frac{\sum_{j=1}^p \sum_{i \in I_j}\lambda_i A_i}{\sum_{j=1}^p \sum_{i \in I_j}\lambda_i}.
  \end{align}
  Comme \((I_j)_{j=1, \ldots, p}\) est une partition de \(\intervalleentier{1}{n}\) alors
    \begin{equation}
      G' = \frac{1}{\sum_{i=1}^n \lambda_i} \sum_{i=1}^n \lambda_i A_i =G.
    \end{equation}
\end{proof}
%
\begin{prop}
  Soit une famille de points pondérés de \(E\) \((A_i,\lambda_i)_{i=1, \ldots, n}\) de masse totale non nulle. Soit \(x_0 \in E\), on condidére la translation de vecteur \(x_0\), \(t_{x_0}\). Alors
  \begin{equation}
    \Bary(t_{x_0}(A_i), \lambda_i)_{i=1, \ldots, n} = t_{x_0}(\Bary(A_i,\lambda_i)_{i=1,\ldots, n}).
  \end{equation}
\end{prop}
\begin{proof}
  Soien les points
  \begin{align}
    G &= \Bary(A_i,\lambda_i)_{i=1,\ldots, n}; \\
    G' &=\Bary(t_{x_0}(A_i), \lambda_i)_{i=1, \ldots, n}.
  \end{align}
  Alors
  \begin{equation}
    G' = \frac{1}{\sum_{i=1}^n \lambda_i}{\sum_{i=1}^n \lambda_i t_{x_0}(A_i)} = \frac{1}{\sum_{i=1}^n \lambda_i} \left(\sum_{i=1}^n \lambda_i A_i + \sum_{i=1}^n \lambda_i x_0 \right) = G+x_0 = t_{x_0}(G).
\end{equation}
\end{proof}

\subsubsection{Stabilité d'un sous-espace affine par barycentration}

\begin{prop}
  Soit \(W_1\) un sous-espace affine et soit une famille de points pondérés de \(E\) \((A_i,\lambda_i)_{i=1, \ldots, n}\) de masse totale non nulle de barycentre \(G\). Alors
  \begin{equation}
    \forall i \in \intervalleentier{1}{n} \quad A_i \in W_1 \implies G \in W_1.
  \end{equation}
\end{prop}
\begin{proof}
  Soient \(A_1 \in W_1\) et \(E_1\) la direction de \(E_1\), \(W_1=A_1+E_1\). Supposons que pour tout \(i \in \intervalleentier{1}{n}\), \(A_i \in W_1\) alors \(AA_i \in E_1\). Or \(E_1\) est un sous-espace vectoriel de \(E\) donc \(\sum_{i=1}^n \lambda_i \vect{AA_i} \in E_1\). Par définition du barycentre on a
  \begin{equation}
    \vect{AG}=\frac{1}{\sum_{i=1}^n \lambda_i}\sum_{i=1}^n \lambda_i \vect{AA_i}.
  \end{equation}
  donc \(\vect{AG} \in E_1\) et comme \(G=A+\vect{AG}\) on a \(G \in W_1\).
\end{proof}

\subsubsection{Segments}

\begin{defdef}
  Soient \(A\) et \(B\) deux points de l'espace affine \(E\). On définit le sgment \([A;B]\) par
  \begin{equation}
    [A;B] = \enstq{M \in E}{\exists \lambda \in \intervalleff{0}{1} \quad  M = \lambda A+(1-\lambda)B}.
  \end{equation}
  C'est l'ensemble des barycentres de \(A\) et \(B\) à coefficients dans \(\intervalleff{0}{1}\).
\end{defdef}

\emph{Remarques}~:
\begin{enumerate}
\item \([A;B]=[B;A]\);
\item si \(A \neq B\) on définit la droite \((AB)\) par
  \begin{align}
    \forall M \in E \quad M \in (AB) &\iff \exists \mu \in \R \vect{AM}=\mu \vect{AB} \\
    &\iff \exists \mu \in \R M=\mu B +(1-\mu)A.\\
  \end{align}
  C'est l'ensemble des barycentres de \(A\) et de \(B\). Soit l'application suivante \(\fonction{\varphi}{\R}{E}{\lambda}{\lambda A +(1-\lambda)B}\). Elle est injective et induit une bijection de \(\R\) sur \((AB)\). \(\varphi\) est un paramétrage de \((AB)\). La restriction de \(\varphi\) à \([0,1]\) permet de réaliser un paramétrage du segment \([A;B]\).
\end{enumerate}

\subsubsection{Parties convexes}

\begin{defdef}
  Soit \(\courbe{}\) une partie de \(E\). \(\courbe{}\) est dite conxe si et seulement si \(\forall (A,B) \in \courbe{}^2 \quad [A;B] \in \courbe{}\).
\end{defdef}

\begin{prop}
  Pour toute partie \(\courbe{}\) de \(E\), \(\courbe{}\) est convexe si et seulement si elle est stable par barycentration à coefficients positifs non nuls.
\end{prop}
\begin{proof}
  \(\impliedby\), Supposons que  \(\courbe{}\) est stable par barycentration à coefficients positifs non nuls. Alors si \(A\) et \(B\) sont dans \(\courbe{}\), tous leurs barycentres à coefficients positifs non nuls le sont aussi et donc \(\courbe{}\) est une partie convexe.

  \(\implies\), On va montrer par récurrence sur \(\N\setminus\{0,1\}\) l'assertion \(\P(n)\) ``pour toute famille \((A_i,\lambda_i)_{i =1, \ldots, n}\) de points pondérés de \(\courbe{}\) telle que tous les \(\lambda_i \geqslant 0\) et \(\sum_{i=1}^n \lambda_i\neq 0\) alors \(\Bary(A_i,\lambda_i)_{i=1, \ldots, n} \in \courbe{}\)''.

  \emph{Initialisation}~: \(\P(2)\) est vraie par définition d'une partie convexe.

  \emph{Hérédité}~: Soit \(n \geqslant 2\) et supposons \(\P(n)\). Soit une famille, \((A_{i}, \lambda_i)_{i=1, \ldots, n+}\) telle que pour tout \(i\) \(A_{i} \in \courbe{}\), \(\lambda_{i} \geqslant 0\) et \(\sum_{i=1}^{n+1}\lambda_i \neq 0\). Deux cas se présentent~:
  \begin{enumerate}
  \item si \(\sum_{i=1}^{n}\lambda_i =0\) alors pour tout \(i \in \llbracket 1,n \rrbracket \lambda_i =0\) et donc
    \begin{equation}
      G=\Bary(A_i,\lambda_i)_{i=1, \ldots, n+1}=A_{n+1} \in \courbe{}
    \end{equation}
  \item Sinon, on peut appliquer l'hypothèse de récurrence à la famille \((A_i,\lambda_i)_{i=1, \ldots, n}\) alors \(G_n=\Bary(A_i,\lambda_i)_{i=1,\ldots, n} \in \courbe{}\).
    Alors par associativite
    \begin{equation}
      G=\Bary\left(\left(G_n,\sum_{i=1}^n \lambda_i\right), (A_{n+1}, \lambda_{n+1})\right),
    \end{equation}
    et alors \(G \in \courbe{}\).
  \end{enumerate}
  Dans les deux cas, \(G \in \courbe{}\) et alors par définiton \(\courbe{}\) est une partie convexe. Alors \(\P(n+1)\) est vraie.

  \emph{Conclusion}~: on a montré que \(\P(2)\) est vraie et que \(\forall n \geqslant 2 \P(n) \implies \P(n+1)\) alors par théorème de récurrence \(\P(n)\) est vraie pour tout \(n \geqslant 2\).
\end{proof}

\begin{cor}
  Les sous-espace affines sont des parties convexes.
\end{cor}
%
\begin{prop}
  Soit \((\courbe{}_i)_{i \in I}\) une famille de parties convexes de \(E\). Alors \(\bigcap_{i \in I} \courbe{}_i\) est aussi une partie convexe.
\end{prop}
\begin{proof}
  Soient deux points \(A,B \in \bigcap_{i\in I} \courbe{}_i\). Alors pour tout \(i \in I\) \(A,B \in C_i\) et comme \(C_i\) est convexe on a \([A;B] \subset C_i\) et donc \([A;B] \subset \bigcap_{i \in I} C_i\).
\end{proof}
%
\emph{Remarques}~:
\begin{enumerate}
\item Si \(\courbe{}_1\) et \(\courbe{}_2\)  sont deux parties convexes d'intersection vide, alors comme l'ensemble vide est convexe, la proposition précédente est encore vraie.
\item Par contre l'union de deux paries convexes n'est pas convexe en général. Les parties convexes de \(\R\) sont les intervalles cf.\ chapitre~
\ref{chap:reels}.
\end{enumerate}
