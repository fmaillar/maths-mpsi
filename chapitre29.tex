\chapter{Fonctions de deux variables réelles}
\label{chap:fonctions2variables}
\minitoc
\minilof
\minilot

\section{Espace \(\R^2\) -- Normes}

\subsection{Normes}

\subsubsection{Définition}

\begin{defdef}
  Soit \(E\) un espace vectoriel réel. On appelle norme sur \(E\) toute application \(N : E \to \R\) qui vérifie les quatre points suivants~:
  \begin{enumerate}
  \item Positivité~: pour tout \(x \in E\), \(N(x)\geqslant 0\);
  \item Séparation~: pour tout \(x \in E\), \(N(x)=0\) si et seulement \(x=0\);
  \item Homogénéité~: pour tout \(x \in E\) et tout \(\lambda \in \R\), \(N(\lambda x)=\abs{\lambda}N(x)\);
  \item Inégalité triangulaire~: pour tout \((x, y) \in E^2\), \(N(x+y) \leqslant N(x) + N(y)\).
  \end{enumerate}
\end{defdef}

\emph{Exemples}~: La valeur absolue est une norme sur \(\R\). L'application \(N : f \longmapsto \sup\limits_{\intervalleff{a}{b}} \abs{f}\) est une norme sur le \(\R\)-espace vectoriel \(\cont{\intervalle{a}{b}, \R}\).

\begin{defdef}
  Si \(N\) est une norme sur l'espace vectoriel réel \(E\), \((E, N)\) est appelé espace vectoriel normé.
\end{defdef}

\subsubsection{Norme usuelle sur \(\R^2\)}

\begin{prop}
  Si pour tout \(x =(x_1, x_2) \in \R^2\) on pose \(\norme{x}_2 = \sqrt{x_1^2+ x_2^2}\) et \(\norme{x}_{\infty}(x)=\max(\abs{x_1}, \abs{x_2})\), alors \(\norme{}_2\) et \(\norme{}_\infty\) sont des normes sur \(\R^2\).
\end{prop}
\begin{proof}
  On sait déjà que \(\norme{}_2\) est une norme, puisque c'est la norme euclidienne associée au produit scalaire sur \(\R^2\). Pour \(\norme{}_\infty\), la positivité, la séparation, l'homogénéité sont faciles à démontrer. Pour montrer l'inégalité triangulaire il suffit de remarquer~:
  \begin{align}
    \abs{x_1+x_2} \leqslant \abs{x_1}+\abs{x_2} \leqslant \norme{x}_\infty + \norme{y}_\infty, \\
    \abs{y_1+y_2} \leqslant \abs{y_1}+\abs{y_2} \leqslant \norme{x}_\infty + \norme{y}_\infty,
  \end{align}
  alors \(\norme{x+y}_\infty = \max(\abs{x_1+x_2}, \abs{y_1+y_2}) \leqslant \norme{x}_\infty + \norme{y}_\infty\).
\end{proof}

\subsubsection{Normes équivalentes}

\begin{defdef}
  Soient \(N_1\) et \(N_2\) deux normes sur le même \(\R\)-espace vectoriel \(E\). On dit que \(N_1\) et \(N_2\) sont équivalentes s'il existe deux réels \(\alpha\) et \(\beta\) strictement positifs tels que pour tout \(x \in E\) on a
  \begin{equation}
    \alpha N_1(x) \leqslant N_2(x) \neq \beta N_1(x).
  \end{equation}  
\end{defdef}
La relation \og{} être équivalente à\fg{} sur l'ensemble des normes de \(E\) est une relation réflexive, symétrique et transitive~: c'est donc une véritable relation d'équivalence.
%
\begin{prop}
  Les normes \(\norme{}_2\) et \(\norme{}_\infty\) sur le \(\R\)-espace vectoriel \(\R^2\) sont équivalentes~:
  \begin{equation}
    \forall x \in \R^2 \quad \norme{x}_\infty \leqslant \norme{x}_2 \leqslant \sqrt{2}\norme{x}_\infty.
  \end{equation}
\end{prop}
\begin{proof}
  Soit \(x=(x_1, x_2) \in \R^2\), on a
  \begin{equation}
    0 \leqslant \norme{x}_2^2 = x_1^2+x_2^2 \leqslant 2\norme{x}_\infty.
  \end{equation}
  Alors
  \begin{equation}
    \norme{x}_2 \leqslant \sqrt{2} \norme{x}_\infty,
  \end{equation}
  et comme \(\abs{x_1} \leqslant \sqrt{x_1^2 + x_2^2}\), \(\abs{x_2} \leqslant \sqrt{x_1^2 + x_2^2}\) on a bien \(\norme{x}_\infty \leqslant \norme{x}_2\).
\end{proof}

\emph{Remarque~:} On peut montrer que toutes les normes d'un espace vectoriel \emph{de dimension finie} sont équivalentes.

\subsection{Parties bornées}

\begin{defdef}
  Soit \((E, N)\) un espace vectoriel normé. Soit \(a \in E\) et \(r \in \Rplusetoile\). On appelle
  \begin{enumerate}
  \item la boule fermée de centre \(a\) et de rayon \(r\) l'ensemble
    \begin{equation}
      \Bbar{a}{r}{} = \enstq{x\in E}{N(x-a) \leqslant r};
    \end{equation}
  \item la boule ouverte de centre \(a\) et de rayon \(r\) l'ensemble
    \begin{equation}
      B(a, r) = \enstq{x\in E}{N(x-a) < r};
    \end{equation}
  \item la sphère de centre \(a\) et de rayon \(r\) l'ensemble
    \begin{equation}
      S(a, r) = \enstq{x\in E}{N(x-a) = r}.
    \end{equation}    
  \end{enumerate}
\end{defdef}
%
\begin{theo}
  Soient \(N_1\) et \(N_2\) deux normes équivalentes sur un même \(\R\)-espace vectoriel \(E\). Il existe alors des réels strictement positif \(\lambda\) et \(\mu\) tels que pour tout \(x \in E\) on ait
  \begin{equation}
    \lambda N_1(x) \leqslant \N_2(x) \leqslant \mu N_1(x).
  \end{equation}
  Alors
  \begin{equation}
    \forall (a, r) \in E \times \Rplusetoile \quad \Bbar{a}{\frac{r}{\mu}}{1} \leqslant \Bbar{a}{r}{2} \leqslant \Bbar{a}{\frac{r}{\lambda}}{1},
  \end{equation}
  où \(\bar{B_1}\) (\(\bar{B_2}\)) désigne la boule fermé au sens de \(N_1\) (\(N_2\)).
\end{theo}
\begin{proof}
  Soit \(x \in E\). Si \(x \in \Bbar{a}{\frac{r}{\mu}}{1}\), alors \(N_1(x-a) \leqslant \frac{r}{\mu}\), d'où
  \begin{equation}
    N_2(x-a) \leqslant \mu N_1(x-a) \leqslant \mu \frac{r}{\mu}=r.
  \end{equation}
  Donc \(x \in \Bbar{a}{r}{2}\).

  Si \(x \in \Bbar{a}{r}{2}\) alors \(N_2(x-a) \leqslant r\) et alors \(N_1(x-a) \leqslant \frac{1}{\lambda} N_2(x-a) \leqslant \frac{r}{\lambda}\). Alors \(x \in \Bbar{a}{\frac{r}{\lambda}}{1}\).
\end{proof}

Cela signifie que si \(N_1\) et \(N_2\) sont équivalentes, alors toute boule pour \(N_2\) est incluse dans une boule pour \(N_1\), et contient une boule pour \(N_1\).

\begin{defdef}
  Soit \((E, N)\) un espace vectoriel normé. Soit \(A \in \P(E)\), on dit que \(A\) est une partie bornée de \(E\) (pour la norme \(N\)) lorsqu'il existe \(M \in \Rplus\) tel que pour tout \(x \in A\) on a \(N(x) \leqslant M\).
\end{defdef}
\begin{defdef}[Définition équivalente]
  Soit \(A \in \P(E)\), alors
  \begin{equation}
    A \text{~est une partie bornée de~} E \iff \exists a \in E \ \exists r \in \Rplusetoile \quad A \subset \Bbar{a}{r}{}.
  \end{equation}
\end{defdef}
\begin{proof}
  \(\implies\) Pour tout \(x \in A\), \(N(x) \leqslant M\) donc \(N(x-0) \leqslant M+1\). Ainsi \(x \in \Bbar{0}{M+1}{}\).

  \(\impliedby\) Pour tout \(x \in A\), \(N(x-a) \leqslant r\), alors \(N(x) \leqslant N(x-a)+N(a) \leqslant r+N(a)=M\).
\end{proof}
\begin{prop}
  Si \(N_1\) et \(N_2\) sont deux normes équivalentes sur le \(\R\)-espace vectoriel normée \(E\), alors les parties bornées de \((E, N_1)\) sont les parties bornées de \((E, N_2)\).
\end{prop}
\begin{proof}
  Soit \(A \in \P(E)\) une partie bornée au sens de \(N_1\). Alors il existe une boule fermée pour \(N_1\) qui contient \(A\). Les normes \(N_1\) et \(N_2\) sont équivalentes, alors cette boule fermé pour \(N_1\) est incluse dans une autre boule fermée pour \(N_2\). Donc \(A\) est contenue dans une boule fermée pour \(N_2\). Ainsi \(A\) est une partie bornée pour \(N_2\). On peut interchanger les rôles de \(N_1\) et \(N_2\).
\end{proof}

\emph{Propriétés}~:
\begin{itemize}
\item L'intersection de deux parties bornées de \(E\) est une partie partie bornée de \(E\);
\item l'union de deux parties bornées de \(E\) est une partie partie bornée de \(E\).
\end{itemize}

\subsection{Voisinages et parties ouvertes}

Soit \((E, \norme{.})\) un espace vectoriel normé.

\subsubsection{Voisinages}

\begin{defdef}
  Soit \(a \in E\) et \(\Vvois \in \P(E)\). On dit que \(\Vvois\) est un voisinage de \(a\) de \(E\) s'il existe un \(r \in \Rplusetoile\) tel que \(B(a, r) \subset \Vvois\).
\end{defdef}

\emph{Propriétés}~:
\begin{itemize}
\item Toute partie de \(E\) contenant un voisinage de \(a \in E\) est encore un voisinage de \(a\);
\item l'ensemble des voisinages d'un point \(a \in E\) est stable par intersection \emph{finie} et réunion quelconque.
\end{itemize}

Soit pour tout naturel \(n\), \(I_n=\intervalleoo{\frac{-1}{n+1}}{\frac{1}{n+1}}\), alors \(\bigcap_{n \in \N} I_n =\{0\}\). Pour tout naturel \(n\), \(I_n\) est un voisinage de zéro mais pas l'intersection.

\emph{Remarque}~: On peut montrer que si \(N_1\) et \(N_2\) sont deux normes équivalentes de \(E\), alors les voisinages de \(a\) de \((E, N_1)\) et les voisinages de \(a\) de \((E, N_2)\) sont les mêmes.

\subsubsection{Parties ouvertes}

\begin{defdef}
 Soit \(\Oouvert \in \P(E)\). On dit que \(\theta\) est une partie ouverte de \(E\) si et seulement s'il est voisinage de chacun de ses points. 

  Autrement dit~: Soit \(\Oouvert \in \P(E)\),
  \begin{equation}
    \Oouvert \text{~est ouvert} \iff \forall a \in \Oouvert \ \exists r \in \Rplusetoile \quad B(a, r) \subset \Oouvert.
  \end{equation}
  On note \(\Oouvert(E)\) l'ensemble des ouverts de \(E\).
\end{defdef}
\begin{prop}
  \begin{enumerate}
  \item \(E \in \Oouvert(E)\);
  \item \(\emptyset \in \Oouvert(E)\);
  \item \(\Oouvert(E)\) est stable par intersection finie;
  \item \(\Oouvert(E)\) est stable par union quelconque.
  \end{enumerate}
\end{prop}
%
\begin{prop}[Admise]
  Si \(N_1\) et \(N_2\) sont des normes équivalentes sur \(E\), les ouverts de \((E, N_1)\) sont les mêmes que \((E, N_2)\).
\end{prop}

\emph{Exemples usuels~:}
\begin{enumerate}
\item Dans \(\R\), les intervalles \(\intervalleoo{a}{b}\) sont des ouverts;
\item dans \(\R^2\), les rectangles \(\intervalleoo{a}{b}\times\intervalleoo{c}{d}\) sont des ouverts;
\item toute boule ouverte d'un espace vectoriel normé est un ouvert.
\end{enumerate}

\subsection{Parties fermées -- Points adhérents}

\begin{defdef}
  On appelle fermé ou partie fermée de l'espace vectoriel normé \((E, N)\) toute partie \(F\) de \(E\) dont le complémentaire dans \(E\) est un ouvert.

  On note \(\F(E)\) l'ensemble des fermés de \(E\). On a donc
  \begin{equation}
    \forall F \in \P(E) \quad F \in \F(E) \iff E\setminus F \in \Oouvert(E).
  \end{equation}
\end{defdef}
\begin{prop}
  \begin{enumerate}
  \item \(\emptyset \in \F(E)\);
  \item \(E \in \F(E)\);
  \item \(\F(E)\) est stable par intersection quelconque et par union finie.
  \end{enumerate}
\end{prop}
%
\emph{Exemples}~: Les singletons et même toutes les parties finies de \(E\) sont des fermés.
%
\begin{defdef}
  Soit \(A \in \P(E)\) et \(x \in E\). On dit que le point \(x\) est adhérent à la partie \(A\) si et seulement si
  \begin{equation}
    \forall \epsilon>0 \ \exists a \in A \quad N(x-a) \leqslant \epsilon.
  \end{equation}
\end{defdef}
\emph{Remarque~:} Soit \(A \in \P(E)\), alors tous les points de \(A\) sont adhérents à \(A\).
\begin{defdef}[Définition équivalente]
  Soit \(x \in E\). \(x\) est adhérant à \(A\) si et seulement si pour tout \(\epsilon>0\), \(A \cap B(x, \epsilon) \neq \emptyset\).
\end{defdef}

On note \(\bar{A}\) l'ensemble des point adhérents à la partie \(A\). On a toujours \(A \subset \bar{A}\).

\begin{prop}
  Soit \(F \in \P(E)\), \(F\) est fermé si et seulement si \(F=\bar{F}\).
\end{prop}
\begin{proof}
  \begin{align}
    F \notin \F(E) &\iff E\setminus F \notin \Oouvert(E) \\
    &\iff \exists a \in E\setminus F \ \forall r \in \Rplusetoile \quad B(a, r) \not\subset E\setminus F \\
    &\iff \exists a \in E\setminus F \ \forall r \in \Rplusetoile \quad B(a, r) \cap F \neq \emptyset \\
    &\iff \exists a \in E\setminus F \ a \text{~est adhérent à~} F.
  \end{align}
  On a montré que \(F\) est non fermé si et seulement s'il existe un point adhérent à \(F\) qui n'est pas dans \(F\), c'est-à-dire \(\bar{F} \not\subset F\). Par contraposée,
  \begin{align}
    F \text{~est fermé} &\iff \bar{F} \subset F \\
    &\iff \bar{F}=F.
  \end{align}
\end{proof}

\section{Limites et continuité des fonctions de deux variables réelles}

Soient \(E=\R^2\) muni de la norme euclidienne \(\norme{.}=\norme{.}_2\) et \(A\) une partie non vide de \(E\). On s'intéresse à l'ensemble des applications de \(A\) vers \(\R\), \(\R^A=\F(A, \R)\). On sait que \((\F(A, \R), +, \perp)\) est un \(\R\)-espace vectoriel et que \((\F(A, \R), +, \times)\) est un anneau commutatif non intègre. On note \(\uinv(A)=\enstq{f \in \F(A, \R)}{0 \in f(A)}\) l'ensemble des inversibles de \(A\).

\subsection{Applications partielles}

Soit \(a=(a_1, a_2) \in A\). Notons \(A_{a, 1}=\enstq{x_1 \in \R}{(x_1, a_2) \in A}\) et \(A_{a, 2}=\enstq{x_2 \in \R}{(a_1, x_2) \in A}\). Soit \(f \in \R^A\), on définit~:
\begin{gather}
  \fonction{\varphi_{a, 1}}{A_{a, 1}}{\R}{x_1}{f(x_1, a_2)}, \\ \fonction{\varphi_{a, 2}}{A_{a, 2}}{\R}{x_2}{f(a_1, x_2)}.
\end{gather}
\(\varphi_{a, 1}\) (\(\varphi_{a, 2}\)) est la première (deuxième) application partielle de \(f\) au point \(a\).

\subsection{Limites et continuité en un point}

Soient \(A\) une partie no vide de \(E\), \(a\) un point de adhérent à \(A\) et \(f \in \R^A\).

\subsubsection{Limites en un point}

\begin{defdef}
  Soit \(\ell \in \R\). On dit que \(f\) admet \(\ell\) pour limite au point \(a\) si et seulement si
  \begin{equation}
    \forall \epsilon>0 \ \exists \eta>0 \ \forall x \in A \quad \norme{x-a} \leqslant \eta \implies \abs{f(x)-\ell} \leqslant \epsilon.
  \end{equation}
  On note \(\ell = \lim\limits_{x \to a} f(x)\) ou \(\ell=\lim\limits_{a}f\).
\end{defdef}
\begin{defdef}[Définition équivalente avec les voisinages]
  \begin{equation}
    \ell=\lim\limits_{a}f \iff \forall V \in \Vvois(\ell) \ \exists U \in \Vvois(a) \ \forall x \in A \quad x \in U \implies f(x) \in V.
  \end{equation}
\end{defdef}

On montre l'unicité de la limite exactement comme pour les fonction d'une seule variable réelle~: Par l'absurde supposons que \(f\) admet en \(a\) deux limites distinctes \(\ell_1\) et \(\ell_2\). Soit \(\epsilon = \frac{\abs{\ell_1 - \ell_2}}{3}\). Alors il existe \(\eta_1\) et \(\eta_2\) strictement positif tels que pour tout \(x \in A\)~:
\begin{align}
  \norme{x-a} \leqslant \eta_1 &\implies \abs{f(x)-\ell_1}\leqslant \epsilon, \\
  \norme{x-a} \leqslant \eta_2 &\implies \abs{f(x)-\ell_2}\leqslant \epsilon.
\end{align}
On définit \(\eta_0=\min(\eta_1, \eta_2)\) et soit \(x \in A \cap B(a,\eta_0)\), alors
\begin{align}
  \abs{\ell_1-\ell_2} &\leqslant \abs{\ell_1-f(x)} + \abs{f(x)-\ell_2} \\
  &\leqslant 2\epsilon = \frac{2}{3}\abs{\ell_1-\ell_2}.
\end{align}
Ce qui est absurde. Donc \(\ell_1=\ell_2\).

\emph{Remarque}~: On peut montrer que si \(N_1\) et \(N_2\) sont deux normes équivalentes, alors \(f\) admet une limite en \(a\) dans \((E, N_1)\) si et seulement si elle admet une limite en \(a\) dans \((E, N_2)\), et ces limits sont égales.

\begin{prop}
  On suppose que \(f\) admet une limite \(\ell\) en \(a=(a_1, a_2)\). Soient \((x_n)_{n \in \N}\in A^\N\) et \((y_n)_{n \in \N}\in A^\N\) qui vérifient \(\lim x_n=a_1\) et \(\lim y_n=a_2\). Alors \(\lim f(x_n, y_n)=\ell\).
\end{prop}
\begin{proof}
  La démonstration est identique à celle d'une fonction de la variable réelle.
\end{proof}

Ce résultat sert à nier ``\(f(x) \to \ell\)'' en exhibant une droite \((x_n, y_n)_{n \in \N}\) telle que \(f(x_n, y_n) \not\to \ell\). Cela servira à montrer qu'une fonction n'est pas continue en un point.

\subsubsection{Continuité en un point}

\begin{defdef}
  Soient \(A \in \P(\R^2)\), \(f \in \R^A\) et \(a \in A\). La fonction \(f\) est continue en \(a\) si et seulement si \(\lim\limits_a f=f(a)\) existe.
\end{defdef}
\emph{Exemples~:}
\begin{enumerate}
\item Soient \(\fonction{p_1}{\R^2}{\R}{(x_1, x_2)}{x_1}\) et \(\fonction{p_2}{\R^2}{\R}{(x_1, x_2)}{x_2}\) les projecteurs canoniques de \(\R^2\). Ces applications sont continues en tout point. En effet,
  \begin{equation}
    \forall (x, y) \in \R^2 \times \R^2 \quad \abs{p_1(x)-p_1(y)} = \abs{x_1-y_1} \leqslant \norme{x-y}.
  \end{equation}
  Il suffit de prendre \(\eta=\epsilon\) dans la définition.
\item Les applications polynomiales de deux variables sont continues.
\end{enumerate}
\begin{prop}
  Soient \(A \in \P(\R^2)\), \(a \in A\) et \(f \in \R^A\). Si \(f\) est continue au voisinage de \(A\), alors elle est bornée au voisinage de \(a\). Autrement dit
  \begin{equation}
    \exists M \in \Rplus \ \exists r \in \Rplus \ \forall a \in A \quad x \in B(a, r) \implies \abs{f(x)} \leqslant M.
  \end{equation}
\end{prop}
\begin{proof}
  Écrivons la définition de la continuité en \(a\) avec \(\epsilon=1\)~:
  \begin{align}
    \exists \eta \in \Rplusetoile \ \forall x \in A &\quad \norme{x-a}\leqslant \eta \implies \abs{f(x)-f(a)}\leqslant 1 \\
    \exists \eta \in \Rplusetoile \ \forall x \in A &\quad x \in B(a, \eta) \implies \abs{f(x)} \leqslant 1+\abs{f(a)}.
  \end{align}
  En posant \(M=f(a)+1\), alors \(f\) est bornée par \(M\).
\end{proof}

\subsubsection{Continuité de \(f\) et continuité des applications partielles}

\begin{prop}
  Soient \(A \subset \R^2\), \(a \in A\) et \(f \in \R^A\). Si \(f\) est continue en \(a=(a_1, a_2)\) et si les applications partielles \(\varphi_{a, 1}\) et \(\varphi_{a, 2}\) sont respectivement définies au voisinage de \(a_1\) et \(a_2\), alors elles sont respectivement continues en \(a_1\) et \(a_2\).
\end{prop}
\begin{proof}
  On démontre le résultat pour \(\varphi_{a, 1}\). Soit \(A_{a, 1} = \enstq{x_1 \in \R}{(x_1, a_2) \in A}\). Soit
  \begin{equation}
    \fonction{\varphi_{a, 1}}{A_{a, 1}}{\R}{x_1}{f(x_1, a_2)}.
  \end{equation}
  La fonction \(f\) est continue en \(a\)~:
  \begin{equation}
    \forall \epsilon >0 \ \exists \eta>0 \ \forall x \in A \quad \norme{x-a} \leqslant \eta \implies \abs{f(x)-f(a)} \leqslant \epsilon.
  \end{equation}
  Soit \(x_1 \in A_{a, 1}\), \(\abs{x_1 - a_1}\leqslant \eta\). Alors
  \begin{equation}
    \norme{(x_1, a_2) - (a_1, a_2)} = \sqrt{(x_1-a_1)^2 +(a_2-a_2)^2}=\abs{x_1-a_1}\leqslant \eta.
  \end{equation}
  Donc \(\abs{f(x_1, a_2) - f(a_1, a_2)} \leqslant \epsilon\). C'est-à-dire \(\abs{\varphi_{a, 1}(x_1)-\varphi_{a, 1}(a_1)} \leqslant \epsilon\). Alors \(\varphi_{a, 1}\) est continue en \(a_1\). De même pour \(\varphi_{a, 2}\).
\end{proof}

La réciproque est fausse. En effet, soit \(\fonction{f}{\R^2}{\R}{(x, y)}{\begin{cases} \frac{xy}{x^2+y^2} & (x, y) \neq (0, 0) \\ 0 & (x, y)=(0, 0) \end{cases}}\). Les application \(\varphi_{0, 1}\) et \(\varphi_{0, 2}\) sont constantes nulles donc continues. Cependant pour tout naturel \(n\) non nul \(f\left(\frac{1}{n}, \frac{1}{n}\right)=\frac{\frac{1}{n} \frac{1}{n}}{\frac{1}{n^2} + \frac{1}{n^2}}=\frac{1}{2}\) qui ne tend pas vers zéro. Alors \(f\) n'est pas continue en zéro.

\emph{Retenir que la continuité de \(f\) implique la continuité des applications partielles et que la réciproque est \emph{fausse}.}

\subsubsection{Espace vectoriel et anneau \(\cont{A}{\R}\)}

\begin{defdef}
  Soient \(A \subset \R^2\) et \(f \in \R^A\). On dit que \(f\) est continue sur \(A\) si elle est continue en tout point de \(A\). On note \(\cont{A}{\R}\) l'ensemble des fonctions de \(A\) vers \(\R\) continues sur \(A\).
\end{defdef}
\begin{prop}
  \((\cont{A}{\R}, +, \perp)\) est un \(\R\)-espace vectoriel. \((\cont{A}{\R}, +, \times)\) est un anneau commutatif non intègre. L'ensemble des inversibles de cet anneau est \(\uinv{\cont{A}{\R}} = \enstq{f \in \cont{A}{\R}}{0 \notin f(A)}\).
\end{prop}
\begin{proof}
  La démonstration est identique à celle pour les fonctions d'une variable réelle.
\end{proof}
%

\subsection{Généralisation aux fonctions à valeurs dans \(\R^2\)}

\begin{defdef}
  Soit \(A \subset \R^2\), \(f : A \longmapsto \R^2\). On appelle applications coordonnées de \(f\) les applications \(f_1 : A \longmapsto \R\) et \(f_2 : A \longmapsto \R\) telles que
  \begin{equation}
    \forall x \in A \quad f(x) = (f_1(x), f_2(x)).
  \end{equation}
\end{defdef}
\begin{defdef}
  Soit \(A \subset \R^2\), \(f : A \longmapsto \R^2\) et \(a \in A\). On dit que \(f\) est continue en \(a\) si et seulement si
  \begin{equation}
    \forall \epsilon>0 \ \exists \eta>0 \ \forall x \in A \quad \norme{x-a}\leqslant \eta \implies \norme{f(x)-f(a)}\leqslant \epsilon.
  \end{equation}
\end{defdef}
\begin{prop}
  Soit \(A \subset \R^2\), \(f : A \longmapsto \R^2\) et \(a \in A\). La fonction \(f\) est continue en \(a\) si et seulement si les applications coordonnées de \(f\) sont continues en \(a\).
\end{prop}

\emph{Ne pas confondre applications partielles et applications coordonnées.} La continuité de \(f : A \rightarrow \R^2\) équivaut à la continuité de ses applications coordonnées tandis que \(g : A \rightarrow \R\) peut avoir des applications partielles continues sans être continue.

\subsection{Composée de fonctions continues}

\begin{theo}
  Soient \(A \subset \R^2\), \(\B\subset \R\) ou \(\R^2\) et \(p \in \{1, 2\}\). Soient \(f : A \rightarrow B\) et \(g: B \rightarrow \R^p\). Soient \(a \in A\) et \(b=f(a) \in B\). Si \(f\) est continue en \(a\) et \(g\) continue en \(b\) alors \(g \circ f\) est continue en \(a\).

Par conséquent, si \(f\) est continue sur \(A\) et \(g\) continue sur \(B\) alors \(g \circ f\) est continue sur \(A\).
\end{theo}

\section{Calcul différentiel}

\subsection{Dérivée en un point selon un vecteur}

\subsubsection{Définition de la dérivée en un point selon un vecteur}

Soient \(U\) un ouvert de \(\R^2\), \(a \in U\) et \(f \in \R^U\). Pour tout \(h>0\), on s'intéresse à l'application \(t \longmapsto f(a+th)\). \(U\) est un ouvert, \(a \in U\) et \(f\) est définie sur \(U\) donc \(f\) est définie au voisinage de \(a\).

Pour tout \(t\) au voisinage de \(0\), \(a+th\) est au voisinage de \(a\) donc \(t \longmapsto f(a+th)\) est définie au voisinage de zéro. Plus concrètement, il existe \(r>0\) tel que \(B(a, r) \subset U\).
\begin{align}
  \forall t \in \R \quad a+th \in B(a, r) &\iff \norme{(a+th)-a} \leqslant r \\
  &\iff \abs{t} \norme{h} \leqslant r.
\end{align}
Si \(h=(0, 0)\), alors \(t \longmapsto f(a+th)\) est définie sur \(\R\). Sinon, \(a+th \in B(a, r) \iff \abs{t} \leqslant \frac{r}{\norme{h}}\). L'application \(t \longmapsto f(a+th)\) est définie au moins sur \(\intervalleoo{\frac{-r}{\norme{h}}}{\frac{r}{\norme{h}}}\). Dans tous les cas, il existe \(\delta>0\) tel que \(\Psi_\fonctionR{h}{t}{f}(a+th)\) soit définie au moins sur \(\intervalleoo{-\delta}{\delta}\). 

Il est légitime de se demander si \(\fonction{\Psi_h}{\intervalleoo{-\delta}{\delta}}{\R}{t}{f(a+th)}\) est dérivable en zéro.

\begin{defdef}
  Si \(\Psi_h\) est dérivable, on dit que \(f\) admet une dérivée en \(a\) selon le vecteur \(h\) et on note
  \begin{equation}
    D_h f(a) = \Psi_h'(0) = \lim\limits_{t \to 0}\frac{f(a+th)-f(a)}{t}.
  \end{equation}
\end{defdef}


\subsubsection{Dérivée partielle première}

Soient \(U\) un ouvert de \(\R^2\), \(a \in U\) et \(f \in \R^U\). Notons \((e_1, e_2)\) la base canonique de \(\R^2\).

\begin{defdef}
  Sous réserve d'existence de \(D_h f(a)\) pou \(h \in \{e_1, e_2\}\), on pose~:
  \begin{align}
    \derivep{f}{x_1}(a) &= D_{e_1}f(a); \\
    \derivep{f}{x_2}(a) &= D_{e_2}f(a). \\
  \end{align}
\end{defdef}

\emph{Interprétation~:}
\begin{align}
  D_{e_1}f(a) &= \lim\limits_{t \to 0}\frac{f(a+te_1)-f(a)}{t} \\
  &= \lim\limits_{t \to 0}\frac{f(a_1+t, a_2)-f(a_1, a_2)}{t}\\
  &= \lim\limits_{t \to 0}\frac{\varphi_{a, 1}(a_1+t)-\varphi_{a, 1}(a_1)}{t}\\
  &= \varphi_{a, 1}'(a_1).
\end{align}
Alors \(\derivep{f}{x_1}(a)\) existe si et seulement si l'application partielle \(\varphi_{1, 1}\) est dérivable en \(a_1\). Auquel cas~:
\begin{equation}
  \derivep{f}{x_1}(a) = \varphi_{a, 1}'(a_1).
\end{equation}

\emph{Notation~:} Les dérivées partielles peuvent être notées \(\derivep{f}{x_1}(a) = D_1 f(a)\).

\emph{Attention~: \(f\) peut admettre des dérivées partielles en un point, et même être dérivable en un point selon tout vecteur, sans pour autant être continue en ce point.}

On peut exhiber le cas de la fonction \(\fonction{f}{\R^2}{\R}{(x, y)}{\begin{cases} \frac{y^2}{x} & x\neq 0 \\ 0 & x=0 \end{cases}}\). En notant \(a=(0, 0)\) on a
\begin{equation}
  \forall t>0 \ h_1\neq 0 \quad \frac{f(a+th)-f(a)}{t} = \frac{f(th)}{t} = \frac{h_2^2}{h_1}.
\end{equation}
Donc \(\lim\limits_{t\to 0} \frac{f(a+th)-f(a)}{t}\) existe. Si \(h_1=0\) alors pour tout \(t>0\) on a \(\frac{f(a+th)-f(a)}{t}=0\).

La fonction \(f\) admet une dérivée en \((0, 0)\) selon tout vecteur \(h\). Par contre, pour tout naturel \(n\) non nul
\begin{equation}
  f\left(\frac{1}{n^2}, \frac{1}{n}\right)=1,
\end{equation}
et ne tend pas vers zéro, donc \(f\) n'est pas continue en zéro.

\subsection{Fonction de classe \(\classe{1}\) sur un ouvert}

Soient \(U\) un ouvert de \(\R^2\) et \(f \in \R^U\).

\begin{defdef}
  On dit que \(f\) est de classe \(\classe{1}\) sur l'ouvert \(U\) si et seulement si pour tout vecteur \(h\) l'application \(\fonction{g_h}{U}{\R}{a}{D_h f(a)}\) est définie et continue sur \(U\).

  On note \(\classe{1}(U, \R)\) l'ensemble des fonctions de \(U\) vers \(\R\) qui sont de classe \(\classe{1}\).
\end{defdef}
%
\begin{prop}
  Si \(f \in \classe{1}(U, \R)\) alors \(f\) admet des dérivées partielles premières et elles sont continues sur \(U\).
\end{prop}
\begin{proof}
  C'est une conséquence immédiate de la définition en prenant \(h=e_1\) et \(h=e_2\).
\end{proof}
%
\begin{theo}[Théorème fondamental]
  Soient \(U\) un ouvert de \(\R^2\) et \(f \in \R^U\). Si les applications dérivées partielles d'ordre \(1\), \(\derivep{f}{x_1}\) et \(\derivep{f}{x_2}\) sont définies et continues sur l'ouvert \(U\), alors~:
  \begin{enumerate}
  \item pour tout \(a \in U\), \(f\) admet un \(DL_1(a)\)~: pour tout \(h \in \R^2\) au voisinage de \((0,0)\) on a
    \begin{equation}
      f(a+h) = f(a) + h_1\derivep{f}{x_1} + h_2\derivep{f}{x_2} + \petito{\norme{h}};
    \end{equation}
  \item pour tout \(a \in U\) et pour tout \(h \in \R^2\), \(f\) admet une dérivée en \(a\) selon \(h\) et
    \begin{equation}
      D_h f(a) = h_1\derivep{f}{x_1} + h_2\derivep{f}{x_2},
    \end{equation}
    d'où pour \(h\) au voisinage de \((0, 0)\)
    \begin{equation}
      f(a+h) = f(a) + D_h f(a) + \petito{\norme{h}};
    \end{equation}
  \item \(f \in \classe{1}(U, \R)\);
  \item pour tout \(a \in U\), l'application \(\fonction{L}{\R^2}{\R}{h}{D_h f(a)}\) est une forme linéaire sur \(\R^2\) appelée différentielle de \(f\) en \(a\) et notée \(\D f_a\).
  \end{enumerate}
\end{theo}

\emph{Notation différentielle~:} Soient \(U\) un ouvert de \(\R^2\) et \(f \in \classe{1}(U, \R^2)\). Notons \(\fonction{p_1}{\R^2}{\R}{(x_1, x_2)}{x_1}\) et \(\fonction{p_2}{\R^2}{\R}{(x_1, x_2)}{x_2}\). \(p_1\) et \(p_2\) sont de classe \(\classe{1}\). Pour tout \(a\) et \(h\) de \(\R^2\) on a
\begin{align}
  D_h p_1(a) &= h_1 \derivep{p_1}{x_1}(a) + h_2 \derivep{p_1}{x_2}(a) =h_1 \\
  D_h p_2(a) &= h_1 \derivep{p_2}{x_1}(a) + h_2 \derivep{p_2}{x_2}(a) =h_2.
\end{align}
Ainsi
\begin{align}
  D_h f(a) &= h_1 \derivep{f}{x_1}(a) + h_2 \derivep{f}{x_2}(a) \\
  &= D_h p_1(a) \derivep{f}{x_1}(a) + D_h p_2(a) \derivep{f}{x_2}(a),  
\end{align}
et comme c'est vrai pour tout vecteur \(h\) on a
\begin{equation}
  D f(a) = D p_1(a) \derivep{f}{x_1}(a) + D p_2(a) \derivep{f}{x_2}(a),
\end{equation}
et en passant en notation différentielle
\begin{equation}
  \D f(a) = \derivep{f}{x_1}(a)\D x_1 + \derivep{f}{x_2}(a) \D x_2.
\end{equation}

\subsection{Gradient}

Soient \(U\) un ouvert de \(\R^2\) et \(f \in \classe{1}(U, \R)\). Pour tout \(a \in U\), l'application \(\fonction{\D f_a}{\R^2}{\R}{h}{D_h f(a)}\) est une forme linéaire.


\begin{defdef}
  Pour tout \(a \in U\), le vecteur normal de cette forme linéaire est appelé gradient de \(f\) en \(a\), et est noté \(\grad f(a)\). C'est l'unique vecteur \(\vn\) tel que pour tout vecteur \(\vu \in \R^2\) on a
  \begin{equation}
    \D f_a(\vu) = \prodscal{\vu}{\vn}.
  \end{equation}  
\end{defdef}

D'après l'expression de \(\D f_a\) on a
\begin{align}
  \grad f(a) &= \left(\derivep{f}{x_1}(a), \derivep{f}{x_2}(a)\right) \\
  &= \derivep{f}{x_1}(a) \vect{e_1} + \derivep{f}{x_2}(a) \vect{e_2}.
\end{align}

\emph{Remarque}~: L'application \(\grad : a \longmapsto \grad f(a)\) est appelée champ de vecteurs.

\subsection{Espace vectoriel et anneau \(\classe{1}(U, \R)\)}

\begin{lemme}
  Soient \(U\) un ouvert de \(\R^2\). Si \(f\) et \(g\) sont deux fonctions de classe \(\classe{1}\), alors \(fg\) est de classe \(\classe{1}\). De plus pour tout \(a \in U\),
  \begin{equation}
    \grad (fg)(a) = f(a) \grad g(a) + g(a) \grad f(a).
  \end{equation}
  De plus pour tout vecteur \(h \in \R^2\) on a aussi
  \begin{equation}
    D_h (fg)(a) = f(a) D_h g(a) + g(a) D_h f(a).
  \end{equation}
\end{lemme}
\begin{proof}
  On fixe \(a \in U\). Notons \(\alpha\), \(\beta\) et \(\gamma\) les premières applications partielles respectives de \(f\), \(g\) et \(fg\) en \(a\). Alors \(\alpha\) et \(\beta\) sont dérivable en \(a_1\)~:
  \begin{align}
    \alpha'(a_1) &= \derivep{f}{x_1}(a) \\
    \beta'(a_1) &= \derivep{g}{x_1}(a).
  \end{align}
  On a \(\gamma = \alpha \beta\), donc \(\gamma\) est dérivable en \(a_1\) et
  \begin{align}
    \gamma'(a_1) &= \alpha(a_1) \beta'(a_1) + \beta(a_1) \alpha'(a_1) \\
    &=\alpha(a_1) \derivep{g}{x_1}(a) + \beta(a_1) \derivep{f}{x_1}(a_1) \\
    &=f(a) \derivep{g}{x_1}(a) + g(a)\derivep{f}{x_1}(a).
  \end{align}
  Alors \(\derivep{fg}{x_1}(a)\) existe et vaut \(f(a) \derivep{g}{x_1}(a) + g(a)\derivep{f}{x_1}(a)\). Ce qui est vrai pour tout \(a \in U\), donc \(\derivep{fg}{x_1}\) est défini sur \(U\) et son expression montre qu'elle est continue. De la même manière on montre que \(\derivep{fg}{x_2}\) est définie et continue sur \(U\). Alors, d'après le théorème fondamental, \(fg\) est de classe \(\classe{1}\).
\end{proof}

\begin{lemme}
  Soient \(U\) un ouvert de \(\R^2\) et \(f \in classe{1}(U, \R)\) telle que \(0 \not\in f(U)\). Alors \(\frac{1}{f}\) est de classe \(\classe{1}\). De plus pour tout \(a \in U\),
  \begin{equation}
    \grad \left(\frac{1}{f}\right)(a) = \frac{-1}{f(a)^2} \grad f(a).
  \end{equation}
  De plus pour tout vecteur \(h \in \R^2\) on a aussi
  \begin{equation}
    D_h \left(\frac{1}{f}\right)(a) = \frac{-1}{f(a)^2} D_h f(a).
  \end{equation}
\end{lemme}
\begin{proof}
Soit \(\varphi\) la première application partielle de \(f\) en \(a\). Comme \(0 \notin f(U)\), \(\frac{1}{f}\) est définie. Soit \(\psi\) la première application partielle de \(\frac{1}{f}\) en \(a\). Ainsi \(\psi= \frac{1}{\varphi}\). L'application \(f\) est de classe \(\classe{1}\) donc \(\varphi\) est dérivable en \(a_1\) et \(\psi'(a_1)=-\frac{\varphi'(a_1)}{\varphi(a_1)^2}\). Cela signifie que \(\derivep{\frac{1}{f}}{x_1}(a)\) existe et vaut \(-\frac{\varphi'(a_1)}{\varphi(a_1)^2}=-\frac{1}{\varphi(a_1)^2}\derivep{f}{x_1}(a)\).

C'est vrai pour tout \(a \in U\) donc \(\derivep{\frac{1}{f}}{x_1}\) est définie sur \(U\). L'application \(f\) étant de classe \(\classe{1}\), l'expression \(\derivep{\frac{1}{f}}{x_1}\) montre qu'elle est continue. De même, \(\derivep{\frac{1}{f}}{x_2}\) est définie et continue sur \(U\).

Le théorème fondamental permet alors de conclure que \(\frac{1}{f}\) est de classe \(\classe{1}\) sur \(U\).
\end{proof}
\begin{lemme}
  Soit \(U\) un ouvert de \(\R^2\). Pour toutes fonctions \(f\) et \(g\) de classe \(\classe{1}\) sur \(U\) et pour tout réel \(\lambda\), la fonction \(\lambda f +g\) est de classe \(\classe{1}\) sur \(U\). 

De plus, pour tout \(a \in U\) et pour tout \(h \in \R^2\),
\begin{gather}
  \grad{\lambda f+g}(a) = \lambda \grad{f}(a) + \grad{g}(a), \\
  D_h(\lambda f+g)(a) = \lambda D_h(f)(a) + D_h(g)(a).
\end{gather}
\end{lemme}
\begin{theo}
  Soit \(U\) un ouvert de \(\R^2\). Alors \(\classe{1}(U, \R)\) est un sous-espace vectoriel et un sous-anneau de \(\cont{U}{\R}\). De plus
    \begin{equation}
      \uinv{\classe{1}(U, \R)}=\enstq{f \in \classe{1}(U, \R)}{0 \notin f(U)}.
    \end{equation}
\end{theo}

\subsection{Dérivée d'une fonction composée}

\begin{theo}
  Soit \(U\) un ouvert de \(\R^2\) et \(f \in \classe{1}(U, \R)\). Soit \(I\) un intervalle réel et \(\varphi \in \classe{1}(I, \R^2)\) telle que \(\varphi(I) \subset U\). On note \(\varphi_1\) et \(\varphi_2\) les applications coordonnées de \(\varphi\). On dispose de \(f \circ \varphi\). 
  
  Alors \(f \circ \varphi \in \classe{1}(I, \R)\) et pour tout \(t \in I\) on a
  \begin{equation}
    (f \circ \varphi)'(t) = \derivep{f}{x_1}(\varphi(t))\varphi_1'(t) + \derivep{f}{x_2}(\varphi(t))\varphi_2'(t).
  \end{equation}
\end{theo}
\begin{proof}
  Soit \(t_0 \in I\). Montrons que \(f \circ \varphi\) est dérivable en \(t_0\) en prouvant qu'elle admet un développement limité à l'ordre \(1\) en \(t_0\).

  D'abord \(f\) est de classe \(\classe{1}\) donc pour tout \(a \in U\), on a
  \begin{equation}
    f(a+h) = f(a) + D_h f(a) + \petito{\norme{h}}.
  \end{equation}
  Pour \(t\) au voisinage de \(t_0\), \(\varphi(t)\) est au voisinage de \(\varphi(t_0)\). On l'applique avec \(a = \varphi(t_0)\) et \(h=\varphi(t)-\varphi(t_0)\)~:
  \begin{align}
    f\circ \varphi(t) = f\circ \varphi(t_0)  + \derivep{f}{x_1}(\varphi(t_0))(\varphi_1(t)-\varphi_1(t_0)) \notag \\ + \derivep{f}{x_2}(\varphi(t_0))(\varphi_2(t)-\varphi_2(t_0)) + \norme{\varphi(t)-\varphi(t_0)}\epsilon(\norme{\varphi(t)-\varphi(t_0)}),
  \end{align}
  avec \(\lim\limits_{0}\epsilon=0\).

  Les fonctions \(\varphi_1\) et \(\varphi_2\) sont donc de classe \(\classe{1}\) au voisinage de \(t_0\).
  \begin{gather}
    \varphi_1(t)=\varphi_1(t_0)+\varphi_1'(t_0)(t-t_0) + (t-t_0)\petito{1}\\
    \varphi_2(t)=\varphi_2(t_0)+\varphi_2'(t_0)(t-t_0) + (t-t_0)\petito{1}.
  \end{gather}
  Au voisinage de \(t_0\),
  \begin{align}
    f \circ \varphi(t) &= f \circ \varphi(t_0) + \derivep{f}{x_1}(\varphi(t_0))\varphi_1'(t_0)(t-t_0) + \derivep{f}{x_1}(\varphi(t_0))(t-t_0)\petito{1} \notag \\
    &+ \derivep{f}{x_2}(\varphi(t_0))\varphi_2'(t_0)(t-t_0) + \derivep{f}{x_2}(\varphi(t_0))(t-t_0)\petito{1} \notag \\
    &+ \norme{(\varphi_1'(t_0)(t-t_0)+ (t-t_0)\petito{1} + \varphi_2'(t_0)(t-t_0)+ (t-t_0)\petito{1})}\petito{1} \\
    &= f \circ \varphi(t_0) + (t-t_0)\left[ \derivep{f}{x_1}(\varphi(t_0))\varphi_1'(t_0) \derivep{f}{x_2}(\varphi(t_0))\varphi_2'(t_0)\right] +\petito{t-t_0}. 
  \end{align}
  Alors \(f \circ \varphi\) admet un développement limité à l'ordre \(1\) en \(t_0\).
  \begin{equation}
    (f \circ \varphi)'(t_0) = \derivep{f}{x_1}(\varphi(t_0))\varphi_1'(t_0) + \derivep{f}{x_2}(\varphi(t_0))\varphi_2'(t_0).
  \end{equation}
\end{proof}
%
On peut généraliser au cas où \(\varphi\) est définie sur un ouvert de \(\R^2\).
%
\begin{theo}
  Soient \(U\) et \(V\) deux ouverts de \(\R^2\). Soient \(f \in \classe{1}(U, \R)\) et \(\varphi \in \classe{1}(V, \R^2)\) telles que \(\varphi(V) \subset U\). On dispose donc de l'application \(f \circ \varphi : V \rightarrow \R\). 

Alors l'application \(f\circ \varphi\) est de classe \(\classe{1}\), pour tout \(a \in V\) et pour \(i \in \{1, 2\}\) on a
\begin{equation}
  \derivep{f\circ \varphi}{x_i}(a) = \derivep{f}{x_1}(\varphi(a))\derivep{\varphi_1}{x_i}(a) + \derivep{f}{x_2}(\varphi(a))\derivep{\varphi_2}{x_i}(a).
\end{equation}
\end{theo}

\paragraph{Interprétation géométrique du gradient}

\begin{prop}
  Soient \(U\) un ouvert de \(\R^2\) et \(f \in \classe{1}(U, \R)\). On appelle lignes de niveaux de \(f\) les ensembles suivants~:
  \begin{equation}
    \Gamma_\alpha = \enstq{(x,y) \in U}{f(x,y)=\alpha \in \R}.
  \end{equation}
  Soit \(\alpha \in \R\) et supposons que \(\Gamma_\alpha\) est paramétrée par \((I, g)\) (\(I\) un intervalle réel et \(g \in \classe{1}(U, \R^2)\)) et que la courbe est régulière. 

  Alors pour tout \(t \in I\), la courbe \(\Gamma_\alpha\) admet au point \(M(t)\) une tangente orthogonale à \(\grad f(M(t))\).
\end{prop}
\begin{proof}
  Pour tout \(t \in I\), \(f(g(t))=\alpha\) et \(f\) et \(g\) sont de classe \(\classe{1}\). On applique le théorème de dérivation d'une fonction composée~:
  \begin{equation}
    (f \circ g)'(t) = \derivep{f}{x_1}(g(t))g_1'(t) + \derivep{f}{x_2}(g(t))g_2'(t) = 0.
  \end{equation}
  Ce qui est équivalent à écrire que le produit scalaire entre \(\grad f(M(t))\) et le vecteur directeur de la tangente en \(M\) est nul.
\end{proof}

\subsection{Extension locale d'une fonction de classe \(\classe{1}\) sur un ouvert}

\subsubsection{Définition}

\begin{defdef}
  Soient \(A \subset \R^2\), \(a \in A\) et \(f \in \R^A\).
  \begin{enumerate}
  \item On dit que \(f\) présente un maximum (minimum) local en \(a\) si et seulement s'il existe un voisinage \(V\) de \(a\) tel que pour tout \(x \in A \cap V\) on a \(f(x) \leqslant f(a)\) (\(f(x) \geqslant f(a)\));
  \item on dit que \(f\) présente un extremum local en \(a\) si elle admet un maximum local ou un minimum local;
  \item on dit que l'extremum est global lorsque l'inégalité est vraie sur \(A\) tout entier;
  \item on dit que l'extremum est strict lorsque l'inégalité est stricte pour \(x \neq a\).
  \end{enumerate}
\end{defdef}

\subsubsection{Condition nécessaire d'existence}

\begin{defdef}
  Soit \(U\) un ouvert de \(\R^2\). Soient \(f \in \classe{1}(U, \R)\) et \(a \in U\). On dit que \(a\) est un point critique pour \(f\) lorsque \(\grad f(a)=\vect{0}\).
\end{defdef}

\begin{theo}
  Soient \(U\) un ouvert de \(\R^2\), \(a \in U\) et \(f \in \classe{1}(U, \R)\). Si \(f\) admet un extremum local en \(a\), alors \(a\) est un point critique de \(f\). La réciproque est fausse.
\end{theo}
\begin{proof}
  Supposons que \(f\) admet un maximum local en \(a\). Il existe \(r>0\) tel que \(B(a, r) \subset U\) et pour tout \(x \in B(a, r)\) on a \(f(x)\leqslant f(a)\). 

  Soit, pour tout \(h\in \R^2\), l'application \(\psi_h: t \rightarrow f(a+th)\). On sait qu'il existe \(\delta>0\) tel que \(\Psi_h\) soit au moins définie sur \(\intervalleoo{-\delta}{\delta}\). Si \(\norme{th}\leqslant r\) et \(\abs{t}\leqslant \delta\) alors \(f(a+th)\) est définie et \(f(a+th) \leqslant f(a)\). 

Or \(\psi_h\) est dérivable en \(0\), \(\psi_h\) admet un maximum local en zéro, zéro est à l'intérieur de son intervalle de définition donc \(\psi_h'(0)=0\). On a montré que pour tout \(h \in \R^2\) \(D_h f(a)=0\). L'application \(\D f_a\) est une forme linéaire nulle donc son vecteur normal (\(\grad f(a)\)) est nul.
\end{proof}
%
La réciproque est fausse. En effet soit \(\fonction{f}{\R^2}{\R}{(x,y)}{xy}\), de classe \(\classe{1}\). Pour tout \((x,y) \in \R^2\) on a
\begin{equation}
  \derivep{f}{x} = y \qquad \derivep{f}{y} = x,
\end{equation}
donc \((0, 0)\) est un point critique. Cependant, est-ce un extremum?

\(f\) n'admet pas de maximum en \((0,0)\)~:
\begin{equation}
  \forall r>0 \ \exists (x,y) \in B((0,0), r) \quad f(x, y)>0
\end{equation}
en prenant par exemple \(x=y=\frac{r}{2}\) alors \(f(x,y)=\frac{r^2}{4}>0\)

\(f\) n'admet pas de minimum en \((0,0)\)~:
\begin{equation}
  \forall r>0 \ \exists (x,y) \in B((0,0), r) \quad f(x, y)<0
\end{equation}
en prenant par exemple \(x=\frac{r}{2}\) et \(y=-x\) alors \(f(x,y)=-\frac{r^2}{4}<0\).

Un tel point est appelé point col ou point selle.

\emph{Il faut absolument se placer sur un ouvert.} Sinon on ne peut pas appliquer le théorème. C'est comme pour les fonctions d'une variable, on se place à l'intérieur d'un intervalle.

En pratique pour rechercher des extremums~:
\begin{itemize}
\item si le domaine de départ n'est pas ouvert, on le sépare en ``plusieurs morceaux'' et on traite la frontière à part;
\item sur le ou les ``morceaux'' ouverts, on commence par déterminer les points critiques, et en chaque point critique trouvé on regarde s'il y a vraiment un extremum ou pas.
\end{itemize}

\subsection{Fonctions de classe \(\classe{1}\), Théorème de Schwarz}

Soient \(U\) un ouvert de \(\R^2\), \(a \in U\) et \(f \in \R^U\). Supposons que \(f\) est de classe \(\classe{1}\) sur \(U\). On dispose des applications \(\derivep{f}{x_1}\) et \(\derivep{f}{x_2}\) qui sont continues. 

Si \(\derivep{f}{x_1}\) admet une dérivée partielle première par rapport à \(x_1\) (\(x_2\)), on la note \(\deriveps{f}{x_1}\) \(\left( \derivepc{f}{x_2}{x_1}\right)\). Si \(\derivep{f}{x_2}\) admet une dérivée partielle première par rapport à \(x_1\) (\(x_2\)), on la note \(\derivepc{f}{x_1}{x_2}\) \(\left(\deriveps{f}{x_2}\right)\). \(f\) admet au plus quatre dérivées partielles secondes.

\begin{defdef}
  Soient \(U\) un ouvert de \(\R^2\) et \(f \in \R^U\). Si les quatre dérivées partielles secondes de \(f\) sont définies et \emph{continues} sur \(U\), on dit que \(f\) est de classe \(\classe{2}\) sur \(U\).
\end{defdef}

\begin{theo}[Théorème de Schwarz (admis)]
  Soient \(U\) un ouvert de \(\R^2\), \(f\) une application de classe \(\classe{2}\) sur \(U\). Alors pour tout \(a \in U\)
  \begin{equation}
    \derivepc{f}{x_2}{x_1} (a) = \derivepc{f}{x_1}{x_2} (a).
  \end{equation}
\end{theo}

\subsection{Exemple d'équation au dérivées partielles}

Les équations aux dérivée partielles sont la généralisation des équations différentielles aux fonctions à plusieurs variables.



Déterminez l'ensemble des fonctions \(f \in \classe{2}(\R^2, \R)\) telles que
\begin{equation}
  \forall (x,y) \in \R^2 \qquad \deriveps{f}{x}(x, y) = 0
\end{equation}


Il existe une fonction \(g \in \classe{2}(\R, \R)\) telle que pour tout \((x, y) \in \R^2\) on a \(\derivep{f}{x}(x, y)=g(y)\). Puis il existe \(h \in  \classe{2}(\R, \R)\) telle que pour tout \((x, y) \in \R^2\) on a \(f(x, y) = g(y)x+h(y)\).

Réciproquement, s'il existe \((g, h) \in \classe{2}(\R, \R)^2\) tel que pour tout \((x, y) \in \R^2\) on a \(f(x, y) = g(y)x+h(y)\) alors \(f \in \classe{2}(\R^2, \R)\) et pour tout \((x, y) \in \R^2\)
\begin{equation}
  \derivep{f}{x}(x,y)=y \quad   \deriveps{f}{x}(x,y)=0.
\end{equation}
