\chapter{Trois théorèmes généraux d'analyse pour aborder l'étude des fonctions 
usuelles}
\label{chap:theogen}
Les résultats énoncés ici seront démontrés plus tard dans l'année, mais servent 
pour l'étude des fonctions usuelles. Soient \(I\) un intervalle réel contenant 
au moins deux points et une application \(f \in \R^I\).

\section{Existence de primitives}

\begin{defdef}
  On appelle primitive de l'application \(f\) sur \(I\) une application \(F \in 
  \R^I\) dérivable sur \(I\) telle que
  \begin{equation}
    \forall x \in I \quad F'(x)=f(x).
  \end{equation}
\end{defdef}
\begin{theo}
  Toute application continue sur l'intervalle \(I\) admet des primitives sur I. 
  Plus précisément, quelque soit \(a \in I\), si \(f\) est continue sur \(I\) 
  alors elle admet une unique primitive \(F\) qui s'annule en \(a\) et \(F\) est 
  définie par
  \begin{equation}
    \forall x \in I \quad F(x)=\int_{a}^x f(t) \D t.
  \end{equation}
\end{theo}
Les autres primitives de \(f\) sur \(I\) sont les applications \(F+c\) avec 
\(c\) une application constante de \(I\) vers \(\R\).

\section{Monotonie et bijectivité}

\begin{defdef}
  L'application \(f\) est dite strictement monotone sur \(I\) si elle y est 
  strictement croissante ou strictement décroissante
\end{defdef}

\begin{defdef}
  Soit J une partie de \(\R\). On dit que l'application \(f\) induit une 
  bijection de \(I\) sur J si tout élément de J admet un unique antécédent par 
  \(f\) dans I. On définit alors l'application réciproque \(f^{-1} \in I^J\) qui 
  a tout élément \(y \in J\) associe l'unique élément \(x \in I\) tel que 
  \(y=f(x)\).
\end{defdef}

\begin{theo}
  Si \(f\) est une application continue est strictement monotone de \(I\) vers 
  \(\R\), alors
  \begin{itemize}
    \item \(f(I)=\enstq{f(x)}{x \in I}\) est un intervalle de \(\R\);
    \item \(f\) induit une bijection de \(I\) vers \(f(I)\);
    \item l'application réciproque \(f^{-1}\) est une application continue et 
      strictement monotone de \(f(I)\) vers \(\R\) et de plus elle est de même 
      monotonie que \(f\).
  \end{itemize}
\end{theo}

\section{Dérivation de l'application réciproque}

\begin{theo}
  Si l'application \(f\) est continue et strictement monotone sur l'intervalle 
  \(I\) et si de plus elle est dérivable en tout point de I, alors pour tout \(y 
  \in f(I)\)
  \begin{itemize}
    \item si \(f'(f^{-1}(y)) \neq 0\), alors \(f^{-1}\) est dérivable en \(y\) 
      et
      \begin{equation}
        \left(f^{-1} \right)'(y)=\frac{1}{f'\left(f^{-1}(y)\right)};
      \end{equation}
    \item si \(f'(f^{-1}(y)) = 0\), alors \(f^{-1}\) n'est pas dérivable en 
      \(y\). Graphiquement, la courbe représentative admet une tangente 
      verticale au point de coordonnées \((y,f^{-1}(y))\).
  \end{itemize}
\end{theo}
