\chapter{Groupe symétrique}
\label{chap:groupesymetrique}
\minitoc
\minilof
\minilot

\section{Groupe \((\sigma_n, \circ)\) où \(n \in \N^*\)}

\subsection{Groupes \((\S(E),\circ)\)}

\emph{Rappel}~: Soit un ensemble fini \(E\). On note \(\S(E)\) l'ensemble des 
permutations de \(E\). Nous avons vu que \((\S(E),\circ)\) était un groupe.

Soient deux ensembles finis \(E\) et \(F\) de même cardinal. Il existe alors une 
bijection \(f: E \rightarrow F\). Montrons que \((\S(E),\circ)\) et 
\((\S(F),\circ)\) sont isomorphes. Soit 
\(\fonction{\varphi}{\S(E)}{\S(F)}{\sigma}{f \circ \sigma \circ f^{-1}}\). Pour 
toute permutation \(\sigma\) de \(\S(E)\), comme \(f\), \(f^{-1}\) sont 
bijective alors \(\varphi(\sigma)\) est bijective. L'application \(\varphi\) est 
bien définie.

Pour toutes permutations \(\sigma \in \S(E)\) et \(\sigma' \in \S(F)\) on a
\begin{align*}
  \varphi(\sigma) = \sigma' &\iff f \circ \sigma \circ f^{-1} = \sigma' \\
  & \iff \sigma =f^{-1} \circ \sigma' \circ f.
\end{align*}
Tout élément de \(\S(F)\) admet un et un seul antécédent par \(\varphi\)~: 
\(\varphi\) est bijective.

Pour toutes permutations \(\sigma, \zeta\) de \(E\), on a
\begin{align*}
  \varphi(\sigma) \circ \varphi(\zeta) &= (f \circ \sigma \circ f^{-1}) \circ (f 
  \circ \zeta \circ f^{-1}) \\
  & = f \circ \sigma \circ (f^{-1} \circ f) \circ \zeta \circ f^{-1} \\
  & = \varphi(\sigma \circ \zeta).
\end{align*}
Alors \(\varphi\) est un isomorphisme du groupe \((\S(E),\circ)\) sur 
\((\S(F),\circ)\).

\emph{Conséquence}~: Pour tout naturel non nul \(n\), si \(E\) est fini de 
cardinal \(n\) alors \((\S(E),\circ)\) est isomorphe à  
\((\S(\intervalleentier{1}{n}),\circ)\).

\subsection{Définition de \(\sigma_n\)}

\begin{defdef}
  Pour tout naturel non nul \(n\), on appelle groupe symétrique de type \(n\), 
  noté \(\sigma_n\), le groupe des permutations de l'ensemble 
  \(\intervalleentier{1}{n}\). Son cardinal vaut \(n\)!.
\end{defdef}

\emph{Vocabulaire}~: Si \(\sigma\) et \(\zeta\) sont deux éléments de 
\(\sigma_n\), on parlera du ``produit'' de \(\sigma\) par \(\zeta\) pour 
désigner la composée \(\sigma \circ \zeta\).

\danger \(\sigma_n\) n'est pas commutatif.

\subsection{Exemples fondamentaux de permutations}

\subsubsection{Transpositions (\(n\geqslant 2\))}

\begin{defdef}
  On appelle transposition de \(\intervalleentier{1}{n}\) toute permutation \(t 
  \in \sigma_n\) telle qu'il existe un couple \((i,j) \in 
  \intervalleentier{1}{n}^2\) avec \(i\neq j\) vérifiant
  \begin{equation}
    t(i) = j \quad t(j) = i \quad \forall k \in \intervalleentier{1}{n} 
    \setminus\{i,j\} \ t(k) = k.
  \end{equation}
\end{defdef}

\emph{Notation}~: Une telle transposition \(t\) est notée \(t = (i,j) \ i<j\).

\subsubsection{Cycles}

\begin{defdef}
  Soit un entier \(p\) tel que \(2 \leqslant p \leqslant n\). On appelle cycle 
  de longueur \(p\) tout \(c \in \sigma_n\) vérifiant~: il existe \(a_1, \ldots, 
  a_p\) dans \(\intervalleentier{1}{n}\), deux à deux distincts, tels que
  \begin{equation}
    \begin{cases}
      \forall i \in \intervalleentier{1}{p-1} \quad c(a_i) = a_{i+1} \\
      c(a_p) = a_1\\
      \forall k \in \intervalleentier{1}{n} \setminus \{a_1, \ldots, a_p\} \quad 
      c(k) = k
    \end{cases}.
  \end{equation}
\end{defdef}

\emph{Notation}~: Un tel cycle \(c\) est noté \(c = (a_1, \ldots, a_p)\).

\emph{Remarque}~: Un cycle de longueur 2 est une transposition et un cycle de 
longueur \(n\) est une permutation circulaire.

\subsubsection{Conséquences}

Si \(n\geqslant 3\), le groupe \(\sigma_n\) n'est pas commutatif. Soit \(t_1 = 
(1,2)\) et \(t_2 = (1,3)\). Alors \(
t_1 \circ t_2 = \begin{pmatrix} 1 & 2 & 3 \\ 3 & 1 & 2
\end{pmatrix}
\) (c'est une permutation circulaire) et \(t_2 \circ t_1 = \begin{pmatrix} 1 & 2 
  & 3 \\ 2 & 3 & 1
\end{pmatrix}
\) (c'est une autre permutation circulaire) et \(t_1 \circ t_2 \neq t_2 \circ 
t_1\).

\emph{Exemples}~:
\begin{enumerate}
  \item si \(n = 1\) alors \(\sigma_1 = \{\Id\}\);
  \item si \(n = 2\) alors \(\sigma_2 = \{\Id, (1,2)\}\);
  \item si \(n = 3\) alors \(\sigma_2 = \{\Id, (1,2), (1,3), (2,3), (1,2,3), 
    (1,3,2)\}\).
\end{enumerate}

\section{Signature d'une permutation}

\subsection{Première définition}

\begin{defdef}
  Soit \(n\geqslant 2\) et \(\sigma \in \sigma_n\). Soit \((i,j) \in 
  \intervalleentier{1}{n}^2\) tel que \(i<j\). On dit que le couple \((i,j)\) 
  présente une \emph{inversion} pour \(\sigma\) si et seulement si 
  \(\sigma(i)>\sigma(j)\).

  On note \(I(\sigma)\) le nombre d'inversion de la permutation \(\sigma\) et on 
  pose que la signature de \(\sigma\) vaut
  \begin{equation}
    \epsilon(\sigma) = (-1)^{I(\sigma)}.
  \end{equation}
\end{defdef}

\emph{Exemple}~: Soit la permutation \(\sigma = \begin{pmatrix} 1 & 2 & 3 \\ 3 & 
  1 & 2
\end{pmatrix}
\). Alors~:
\begin{itemize}
  \item \((1,2)\) est une inversion puisque \(3>1\);
  \item \((1,3)\) est une inversion puisque \(3>2\);
  \item \((2,3)\) n'est pas une inversion puisque \(1<2\).
\end{itemize}
Ainsi la signature de \(\sigma\) vaut \(1\).

\begin{defdef}
  Soit \(\sigma \in \sigma_n\). On dit que \(\sigma\) est paire si sa signature 
  vaut \(1\) et impaire si elle vaut \(-1\).
\end{defdef}
\danger Ne pas confondre la parité d'une permutation avec la parité d'une 
application. D'ailleurs ça n'a aucun sens puisque \(\intervalleentier{1}{n}\) 
n'est pas centré en zéro.
%
\begin{theo}
  Les transpositions sont des permutations impaires.
\end{theo}
\begin{proof}
  Soit \((i,j) \in \intervalleentier{1}{n} ^2\) tel que \(i<j\), alors
  \begin{equation}
    t = \begin{pmatrix} 1 & \ldots & i-1 & i & i+1& \ldots& j-1& j& j+1 &\ldots& 
    n \\ 1 & \ldots & i-1 & j & i+1& \ldots& j-1& i& j+1 &\ldots& n\end{pmatrix}
  \end{equation}
  Inversions~:
  \begin{itemize}
    \item \((i,i+1)\) \ldots \((i,j)\);
    \item \((i+1,j)\) \ldots \((j-1,j)\).
  \end{itemize}
  Alors au final on a \(I(\sigma)=[j-(i+1)+1]+[(j-1)-(i+1)+1] = 2(j-i)-1\) qui 
  est un nombre impair donc la signature vaut \(\epsilon(\sigma) = -1\) et donc 
  \(t\) est une permutation impaire.
\end{proof}

\subsection{Deuxième définition}

La signature d'une permutation peut également être définie de la façon 
suivante~:
\begin{defdef}
  Soit \(\sigma \in \sigma_n\), la signature de \(\sigma\) est
  \begin{equation}
    \epsilon(\sigma) = \prod_{i<j} \frac{\sigma(j)-\sigma(i)}{j-i}.
  \end{equation}
\end{defdef}
Montrons que cette définition est équivalente à la première.
\begin{proof}
  Notons \(A = \enstq{(i,j) \in \intervalleentier{1}{n}^2}{i<j}\), \(N = 
  \prod_{i<j}\sigma(j)-\sigma(i)\) et \(D = \prod_{i<j} j-i\). Soit \((i,j) \in 
  A\) et posons \(h = \min(\sigma(i),\sigma(j)) \in A\) et \(k = 
  \max(\sigma(i),\sigma(j))\in A\). L'application 
  \(\fonction{\psi}{A}{A}{(i,j)}{(h,k)}\) est une bijection. On a
  \begin{equation}
    \sigma(j)-\sigma(i) = \epsilon_{ij} (k-h),
  \end{equation}
  avec \(\epsilon_{ij} = 1\) si \((i,j)\) n'est pas une inversion pour 
  \(\sigma\) et \(-1\) si c'est une inversion pour \(\sigma\). Ainsi
  \begin{align*}
    \frac{N}{D} &= \frac{\prod_{(i,j) \in A}\epsilon_{ij}\prod_{(i,j) \in 
    A}(k-h)}{\prod_{(i,j) \in A}(j-i)}\\
    & = \prod_{(i,j) \in A}\epsilon_{ij}\\
  & = (-1)^{I(\sigma)}.  \end{align*}
\end{proof}

\subsection{Morphisme \(\epsilon\)}

\begin{theo}
  Soit \(n\geqslant 2\). L'application 
  \(\fonction{\epsilon}{(\sigma_n,\circ)}{(\{-1,1\},\times)}{\sigma}{\epsilon(\sigma)}\) 
  est un morphisme de groupes surjectif.
\end{theo}
\begin{proof}
  L'application \(\epsilon\) est bien définie. Ensuite~:
  \begin{itemize}
    \item \(\epsilon(\Id) = 1\) (l'identité n'a aucune inversion);
    \item \(\epsilon((1,2)) = -1\) (possible puisque \(n\geqslant 2\)).
  \end{itemize}
  Donc \(\epsilon\) est surjective.

  Soient \(\sigma\) et \(s\) deux éléments de \(\sigma_n\). Alors
  \begin{align*}
    \epsilon(\sigma \circ s) &= \prod_{(i,j)\in A} \frac{\sigma\circ 
    s(j)-\sigma\circ s(i)}{j-i} \\
    & = \prod_{(i,j)\in A} \frac{\sigma\circ s(j)-\sigma\circ s(i)}{s(j)-s(i)} 
    \prod_{(i,j)\in A} \frac{s(j)-s(i)}{j-i}\\
    &= \prod_{(i,j)\in A} \frac{\sigma\circ s(j)-\sigma\circ 
    s(i)}{s(j)-s(i)}\epsilon(s).
  \end{align*}
  On note \(h = \min(s(i),s(j))\) et \(k = \max(s(i),s(j))\), l'application
  \begin{equation}
    \fonction{\psi}{A}{A}{(i,j)}{(h,k)}
  \end{equation}
  est une bijection. D'où
  \begin{equation}
    \epsilon(\sigma \circ s) = \prod_{(i,j)\in A} 
    \frac{\sigma(k)-\sigma(h)}{k-h}\epsilon(s) = \epsilon(\sigma)\epsilon(s).
  \end{equation}
  L'application \(\epsilon\) est bien un morphisme.
\end{proof}

\emph{Remarque}~: L'hypothèse \(n \geqslant 2\) ne sert à démontrer que la 
surjectivité.

\begin{corth}
  Pour tout \(\sigma \in \sigma_n\), 
  \(\epsilon(\sigma^{-1})=\epsilon(\sigma)^{-1} = \epsilon(\sigma)\).
\end{corth}
Le produit de deux permutations de même parité est paire et le produit de deux 
permutations de parités différentes est impaire.

\subsection{Groupe alterné \(\A_n\)}

\begin{defdef}
  On appelle groupe alterné de type \(n\) l'ensemble des permutations paires de 
  \(\intervalleentier{1}{n}\),
  \begin{equation}
    \A_n = \enstq{\sigma \in \sigma_n}{\epsilon(\sigma) = 1}.
  \end{equation}
\end{defdef}
\begin{prop}
  \((\A_n,\circ)\) est un sous-groupe de \((\sigma_n,\circ)\).
\end{prop}
\begin{proof}
  Il suffit de voir que \(\A_n = \ker \sigma\). Le noyau d'un morphisme est un 
  sous-groupe.
\end{proof}

Par contre l'ensemble des permutations impaires n'est pas stable. On ne peut 
donc pas définir de groupe des permutations impaires.

\begin{prop}
  Soit \(n \geqslant 2\). Soit \(\tau_0\) une permutation impaire fixée. Alors
  \begin{equation}
    \sigma_n \setminus \A_n = \enstq{\sigma \circ \tau_0}{\sigma \in \A_n}.
  \end{equation}
  Par conséquent \(\A_n\) et \(\sigma_n \setminus \A_n\) ont le même cardinal
  \begin{equation}
    \Card \A_n = \Card \sigma_n \setminus \A_n = \frac{n!}{2}.
  \end{equation}
\end{prop}
\begin{proof}
  Soit l'application \(\fonction{\varphi}{\A_n}{\sigma_n \setminus 
  \A_n}{\sigma}{\sigma \circ \tau_0}\). Elle est bien définie puisque pour toute 
  \(\sigma \in \A_n\) on a\(\epsilon(\sigma \circ \tau_0) = -1\). Soit aussi 
  l'application \(\fonction{\varphi}{\sigma_n \setminus 
  \A_n}{\A_n}{\sigma}{\sigma \circ \tau_0^{-1}}\) (qui est aussi bien défini 
  puisque le produit de deux permutations impaires est paire).

  Pour toute \(\sigma \in \A_n\), on a
  \begin{equation}
    \psi \circ \varphi(\sigma) = \psi(\sigma \circ \tau_0) = \sigma.
  \end{equation}
  De même pour toute \(\sigma \in \sigma_n \setminus \A_n\), on a
  \begin{equation}
    \varphi \circ \psi(\varphi) = \varphi(\sigma \circ \tau_0^{-1}) = \sigma.
  \end{equation}
  Par caractérisation des bijections, l'application \(\varphi\) est bijective.

  \(\A_n\) et \(\sigma_n \setminus \A_n\) sont des ensembles finis et il existe 
  une bijection de \(\A_n\) sur \(\sigma_n \setminus \A_n\). Donc ils ont le 
  même cardinal et comme ils forment une partition de \(\sigma_n\) on a bien
  \begin{equation}
    \Card \A_n = \Card \sigma_n \setminus \A_n = \frac{n!}{2}.
  \end{equation}
\end{proof}

\section{Décomposition d'une permutation en produits de transpositions}

\begin{theo}
  Soit \(n \geqslant 2\). Toute permutation \(\sigma \in \sigma_n\) peut 
  s'écrire comme un produit fini de transpositions. Cette décomposition n'est 
  pas unique
\end{theo}
\begin{proof}
  On montre par réccurence sur \(n\geqslant 2\) l'assertion \(\P(n)\) ``Toute 
  \(\sigma \in \sigma_n\) peut s'écrire comme un produit fini de 
  transpositions''.

  \emph{Initialisation}~: \(n = 2\). Déja \(\sigma_n = \{\Id, (1,2)\}\) et \(\Id 
  = (1,2) \circ (1,2)\). \(\P(2)\) est vraie.

  \emph{Hérédité}~: Soit \(n\geqslant 2\) et on suppose \(\P(n)\). Soit \(\sigma 
  \in \sigma_{n+1}\). On distingue deux cas.

  \emph{Premier cas}~: Si \(\sigma(n+1) = n+1\), alors 
  \(\sigma(\intervalleentier{1}{n}) = \intervalleentier{1}{n}\). On peut définir 
  \(\tau\) comme la restriction au départ et à l'arrivée à 
  \(\intervalleentier{1}{n}\), \(\tau \in \sigma_n\). En lui appliquant 
  \(\P(n)\), on trouve qu'il existe \(p \in \N^*\), \(\tau_1, \ldots, \tau_p\) 
  des transpositions de \(\sigma_n\) telles que \(\tau = \tau_1 \circ \ldots 
  \circ \tau_p\). On prolonge les \(\tau_i\) en \(n+1\) en posant \(\tau_i(n+1) 
  = n+1\). Alors \(\tau(n+1) = n+1\). Finalement \(\sigma = \tau_1 \circ \ldots 
  \circ \tau_p\).

  \emph{Deuxième cas}~: Si \(\sigma(n+1)\neq n+1\) on pose \(t_0 = 
  (\sigma(n+1),n+1)\) et \(s = t_0 \circ \sigma\) et on applique le premier cas 
  à \(s\).

  Finalement \(\P(n+1)\) est vraie.

  \emph{Conclusion}~: On a montré que \(\P(2)\) est vraie et que pour tout 
  \(n\geqslant 2\) \(\P(n) \implies \P(n+1)\). Alors le théorème de récurrence 
  nous dit que \(\P\) est vraie pour tout \(n\geqslant 2\).
\end{proof}

\emph{Exemple}~: On considère un cycle \(c = (a_1, \ldots, a_p)\) alors \(c = 
(a_1,a_2) \circ (a_2,a_3) \circ \ldots \circ (a_{p-1},a_p)\). Ainsi
\begin{equation}
  \epsilon(c) = \epsilon(t_1) \ldots \epsilon(t_{p-1}) = (-1)^{p-1}.
\end{equation}
