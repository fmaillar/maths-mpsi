\chapter{Matrices}
\label{chap:matrices}
\minitoc
\minilof
\minilot

\section{Opérations sur les matrices}

\subsection{\(\K\)-espace vectoriel \(\Mnp{n}{p}{\K}\)}

\subsubsection{Définition}

\begin{defdef}
  On appelle matrice à \(n\) lignes et \(p\) colonnes à coefficients dans \(\K\) toute application \(\fonction{A}{\intervalleentier{1}{n} \times \intervalleentier{1}{p}}{\K}{(i,j)}{a_{ij}}\).
\end{defdef}

\emph{Notation}~: Au lieu d'écrire \(\fonction{A}{\intervalleentier{1}{n} \times \intervalleentier{1}{p}}{\K}{(i,j)}{a_{ij}}\), on note \(A=(a_{ij})_{(i,j)\in \intervalleentier{1}{n} \times \intervalleentier{1}{p}}\). On note aussi la matrice \(A\) par un ``tableau'' rectangulaire de \(n\) lignes et \(p\) colonnes
\begin{equation}
  \begin{pmatrix}
    a_{11} & \ldots & a_{1p} \\
    \vdots & a_{ij} & \vdots \\
    a_{n1} & \ldots & a_{np}
  \end{pmatrix}.
\end{equation}

\emph{Vocabulaire}~:
\begin{itemize}
\item Si \(n=p\), on parle de matrice carrée. La diagonale de \(A\) est le vecteur \((a_{ii})_{i \in \intervalleentier{1}{n}}\);
\item si \(n=1\), on parle de matrice ligne;
\item si \(p=1\), on parle de matrice colonne;
\item pour tout \(i \in \intervalleentier{1}{n}\), le vecteur \((a_{ij})_{j \in \intervalleentier{1}{p}}\) est le le \(i\)\ieme{} vecteur ligne;
\item pour tout \(j \in \intervalleentier{1}{p}\), le vecteur \((a_{ij})_{i \in \intervalleentier{1}{n}}\) est le le \(j\)\ieme{} vecteur colonne.
\end{itemize}

On appelle sous matrice de la matrice \(\fonction{A}{\intervalleentier{1}{n} \times \intervalleentier{1}{p}}{\K}{(i,j)}{a_{ij}}\) toute restriction de l'application \(A\) à \(I_1 \times J_1\) avec \(I_1\neq \emptyset \subset \intervalleentier{1}{n}\) et \(J_1 \neq \emptyset \subset \intervalleentier{1}{p}\).

L'ensemble des matrices à \(n\) lignes et \(p\) colonnes à coefficients dans \(\K\) est noté \(\Mnp{n}{p}{\K}\)
\begin{equation}
  \Mnp{n}{p}{\K}=\{A=(a_{ij})_{(i,j)\in \intervalleentier{1}{n} \times \intervalleentier{1}{p}}, \forall (i,j) \in \intervalleentier{1}{n} \times \intervalleentier{1}{p} \quad a_{ij} \in \K\}
\end{equation}

Pour tout couple \((n,p) \in \N^*\times \N^*\), on définit la matrice nulle à \(n\) lignes et \(p\) colonnes par \(0_{n,p}=(0)_{(i,j)\in \intervalleentier{1}{n} \times \intervalleentier{1}{p}}\). 

Si \(n=p\), on définit la matrice identité, notée \(I_n\) par \(I_n=(\delta_{ij})_{(i,j)\in \intervalleentier{1}{n}^2}\).

\begin{defdef}[Égalité de deux matrices]
  Soit quatre entiers naturels non nuls \(n,p,n'\) et \(p'\). Soient deux matrices \(M \in  \Mnp{n}{p}{\K}\) et \(N \in Mnp{n'}{p'}{\K}\). Alors
  \begin{equation}
    M=N \iff n=n',\ p=p' \text{~et~} \forall (i,j) \in \intervalleentier{1}{n} \times \intervalleentier{1}{p} \ \ a_{ij}=b_{ij}.
  \end{equation}
\end{defdef}

\subsubsection{Structure de \(\K\)-espace vectoriel}

On munit l'ensemble \(\Mnp{n}{p}{\K}\) de deux lois~:

\begin{itemize}
\item une addition, notée \(+\), définie par~:
  \begin{equation}
    \forall (A,B) \in \Mnp{n}{p}{\K}^2 \quad S=A+B \in \Mnp{n}{p}{\K}
  \end{equation}
  avec
  \begin{equation}
    \forall (i,j) \in \intervalleentier{1}{n} \times \intervalleentier{1}{p} \quad s_{ij}=a_{ij}+b_{ij}.
  \end{equation}
\item une multiplication externe, notée \(\cdot\), définie par~:
  \begin{equation}
    \forall A \in \Mnp{n}{p}{\K} \ \forall \lambda \in \K \quad B=\lambda A \in \Mnp{n}{p}{\K}
  \end{equation}
  avec
  \begin{equation}
    \forall (i,j) \in \intervalleentier{1}{n} \times \intervalleentier{1}{p} \quad b_{ij}=\lambda a_{ij}
  \end{equation}
\end{itemize}

\begin{theo}
  L'ensemble \((\Mnp{n}{p}{\K}, +, \perp)\) est un \(\K\)-espace vectoriel de dimension finie égale à \(np\).

  La base canonique de \(\Mnp{n}{p}{\K}\) est la famille \((E_{ij})_{(i,j)\in \intervalleentier{1}{n} \times \intervalleentier{1}{p}}\) où pour tout \((i_0,j_0)\in \intervalleentier{1}{n} \times \intervalleentier{1}{p}\) on a \(E_{i_0j_0}=(e_{ij})_{(i,j)\in \intervalleentier{1}{n} \times \intervalleentier{1}{p}}\) et
  \begin{equation}
    \forall (i,j) \in \intervalleentier{1}{n} \times \intervalleentier{1}{p} \quad e_{ij} = \delta_{ii_0} \delta_{jj_0}.
  \end{equation}
\end{theo}
\begin{proof}
  Toute matrice \(A=(a_{ij})_{(i,j)\in \intervalleentier{1}{n} \times \intervalleentier{1}{p}} \in \Mnp{n}{p}{\K}\) s'écrit comme
  \begin{equation}
    A=\sum_{i=1}^n\sum_{j=1}^p a_{ij} E_{ij}.
  \end{equation}
  Tout élément de \(\Mnp{n}{p}{\K}\) s'écrit comme une combinaison linéaire de la famille \((E_{ij})_{(i,j)\in \intervalleentier{1}{n} \times \intervalleentier{1}{p}}\). Le cardinal de \((E_{i,j})\) vaut \(np\) donc cette famille est une base de \(\Mnp{n}{p}{\K}\).
\end{proof}

\subsubsection{Matrice d'application linéaire}

Soit \(E\) un \(\K\)-espace vectoriel de dimension finie non nulle \(p\) et \(\E=(e_1, \ldots, e_p)\) une base de \(E\). Soit \(F\) un autre \(\K\)-espace vectoriel de dimension finie non nulle \(n\) et \(\F=(f_1, \ldots, f_n)\) une base de \(F\). Soit \(u \in \Lin{E}{F}\).
\begin{defdef}
  On appelle matrice de l'application linéaire \(u\) dans les bases \(\E\) et \(\F\) et on note \(\Mat_{\E,\F}(u)\) la matrice \(M \in \Mnp{n}{p}{\K}\) telle que
  \begin{equation}
    \forall j \in \intervalleentier{1}{p} \quad u(e_j) = \sum_{i=1}^n a_{ij} f_i.
  \end{equation}

Si \(E=F\), \(\E=\F\) et on parle de matrice dans la base \(\E\) de l'endomorphisme linéaire \(u\) de \(E\). La matrice \(\Mat_\E(u)\) est alors carrée.
\end{defdef}

\emph{Exemples}~: Soit un \(\K\)-espace vectoriel \(E\) de dimension finie non nulle \(p\).
\begin{itemize}
\item Homothéties~: Soit un scalaire \(\alpha\) et l'homothétie \(\fonction{h_\alpha}{E}{E}{x}{\alpha x}\). Soit \(\E=(e_1,\ldots, e_p)\) une base de \(E\). Pour tout \(j \in \intervalleentier{1}{p}\) on a \(h(e_j)=\alpha e_j\). Alors \(\Mat_\E(h_\alpha)=\alpha I_p\). Si \(\alpha=1\), on voit que \(I_p\) est la matrice de l'identité.
\item Projecteurs et symétries~: Soient \(E_1\) et \(E_2\) deux sous espaces vectoriels supplémentaires de \(E\) de dimension non nulles.
  \begin{equation}
    E=E_1 \oplus E_2 \quad \dim(E_1)=r \in \intervalleentier{1}{p-1}.
  \end{equation}

Soit \(p\) le projecteur sur \(E_1\) parallèlement à \(E_2\) et \(s\) la symétrie associée, \(s=2p -\Id\). Soit \(\E=(e_1, \ldots, e_r, e_{r+1}, \ldots, e_p)\) une base adpatée à la décomposition \(E=E_1 \oplus E_2\), puisqu'ils sont supplémentaires dans \(E\). Soient
\begin{equation}
  P=\Mat_\E(p) \quad S=\Mat_\E(s)
\end{equation}
et
\begin{equation}
  \begin{cases}
    p(e_i)=e_i, s(e_i)=e_i & i \in \intervalleentier{1}{r} \\
    p(e_i)=0, s(e_i)=-e_i & i \in \intervalleentier{r+1}{p}.
  \end{cases}
\end{equation}
\end{itemize}

\subsubsection{Isomorphisme canonique entre \(\Lin{\K^p}{\K^n}\) et \(\Mnp{n}{p}{\K}\)}


Soient \(\E_c\) et \(\F_c\) les bases canoniques respectives de \(\K^p\) et \(\K^n\). Soit l'application \[\fonction{\Phi}{\Lin{\K^p}{\K^n}}{\Mnp{n}{p}{\K}}{u}{\Mat_{\E_c,\F_c}(u)}.\]

\begin{prop}
  L'application \(\Phi\) est linéaire.
\end{prop}
\begin{proof}
  Soient \(u\) et \(v\) deux applications de \(\Lin{\K^p}{\K^n}\) et un scalaire \(\lambda\). 

Pour tout \(j \in \intervalleentier{1}{p}\), le \(j\)\ieme{} vecteur colonne de la matrice \(\Mat_{\E_c,\F_c}(\lambda u+v)\) est le vecteur des coordonnées de \((\lambda u+v)(e_j)\) dans la base \(\F_c\). De plus
\begin{equation}
  (\lambda u+v)(e_j) = \lambda u(e_j) +v(e_j).
\end{equation}

Ensuite, on applique les définitions de la somme de matrice et de la multiplication par un scalaire pour obtenir le résultat.
\end{proof}

\begin{prop}
  L'application \(\Phi\) est bijective.
\end{prop}
\begin{proof}
  Pour toute matrice \(M \in \Mnp{n}{p}{\K}\), il existe une unique application linéaire \(u \in \Lin{\K^p}{\K^n}\) telle que \(\Phi(u)=M\) car~: une application linéaire est entiérement déternminée par l'image d'une base. Ainsi, \(u\) est entiérement déterminée par la donnée de \((u(e_1), \ldots, u(e_p))\). Les vecteurs \(u(e_1), \ldots, u(e_p)\) sont complétement déterminés par leurs coordonnées dans la base \(\F_c\).
\end{proof}

Grâce à ces propositions, il découle le théorème suivant.
\begin{theo}
  \(\Phi\) est un isomorphisme de \(\K\)-espaces vectoriels de \(\Lin{\K^p}{\K^n}\) sur \(\Mnp{n}{p}{\K}\). Nous disposons alors de \(\Phi^{-1}\). Pour toute matrice \(A \in \Mnp{n}{p}{\K}\), l'application \(u=\Phi^{-1}(A)\) est appelée application linéaire canoniquement associée à la matrice \(A\).
\end{theo}

\emph{Exemple}~: Soit la matrice \(A=\begin{pmatrix} 1 & 2 & 3 \\ 4 & 5 & 6 \end{pmatrix}\). Soit \(u\) l'application linéaire canoniquement associée à \(A\). Alors
\begin{equation}
  u((1,0,0))=(1,4) \quad u((0,1,0))=(2,5) \quad u((0,0,1))=(3,6)
\end{equation}
Pour tout vecteur \(x=(x_1,x_2,x_3) \in \R^3\) on a
\begin{equation}
  u(x)=x_1(1,4) + x_2(2,5) +x_3(3,6)=(x_1+2x_2+3x_3, 4x_1+5x_2+6x_3)
\end{equation}

\subsubsection{Identification d'un vecteur de \(\K^n\) et une matrice colonne}

Soit l'application \(\fonction{\Psi}{\K^n}{\Mnp{n}{1}{\K}}{(x_1,\ldots,x_n)}{\begin{pmatrix}x_1 \\ \vdots \\ x_n \end{pmatrix}}\).

\begin{theo}
  L'application \(\Psi\) est un ismorphisme de \(\K\)-espaces vectoriels de \(\K^n\) sur \(\Mnp{n}{1}{\K}\) qui permet d'identifier les vecteurs de \(\K^n\) avec les matrices colonnes de \(\Mnp{n}{1}{\K}\).
\end{theo}

\subsubsection{Identification d'un vecteur de \(\K^p\) et une matrice ligne}

Soit l'application \(\fonction{\gamma}{(\K^p)^{*}}{\Mnp{1}{p}{\K}}{f}{\begin{pmatrix}f(e_1) & \ldots & f(e_p) \end{pmatrix}}\).

\begin{theo}
  L'application \(\gamma\) est un ismorphisme de \(\K\)-espaces vectoriels de \((\K^p)^{*}=\Lin{\K^p}{\K}\) sur \(\Mnp{1}{p}{\K}\) qui permet d'identifier les formes linéaires de \((\K^p)^{*}\) avec les matrices lignes de \(\Mnp{n}{1}{\K}\).
\end{theo}

\subsection{Produits de matrices}

\subsubsection{Définition}

Soient trois naturels non nuls \(n\), \(p\) et \(m\).

\begin{defdef}
  Soient deux matrices \(A \in \Mnp{n}{p}{\K}\) et \(B \in \Mnp{m}{n}{\K}\). On définit le produit \(BA\) par 
\begin{equation}
  BA=(c_{ij})_{(i,j)\in \intervalleentier{1}{m} \times \intervalleentier{1}{p}} \in \Mnp{m}{p}{\K}
\end{equation}
avec
\begin{equation}
  \forall i \in \intervalleentier{1}{m} \ \forall j \in \intervalleentier{1}{p} \quad c_{ij}=\sum_{k=1}^n b_{ik}a_{kj}.
\end{equation}

En pratique,
\begin{equation}
  \begin{pmatrix} b_{i1} & \ldots & b_{in} \end{pmatrix} \begin{pmatrix} a_{1j} \\ \vdots \\ a_{nj} \end{pmatrix} = \begin{pmatrix} b_{i1}a_{1j} & \ldots & b_{in}a_{nj}\end{pmatrix}.
\end{equation}
\end{defdef}

\danger Pour donner un sens à \(BA\), il faut que le nombre de colonne de \(B\) soit égal au nombre de lignes de \(A\).

\danger Le produit \(BA\) peut être défini sans que \(AB\) existe.

\danger Même si les deux produits sont définis, ils ne sont en général pas de même dimension.

\danger Même si les deux produits sont définis et s'ils sont de même dimension, ils ne sont pas égaux en général. En effet si on note \(A=E_{12}=\begin{pmatrix} 0 & 1 \\ 0 & 0 \end{pmatrix}\) et \(B=E_{21}=\begin{pmatrix} 0 & 0 \\ 1 & 0 \end{pmatrix}\) alors
\begin{equation}
  BA = E_{22} = \begin{pmatrix} 0 & 0 \\ 0 & 1 \end{pmatrix} \quad AB = E_{11} = \begin{pmatrix} 1 & 0 \\ 0 & 0 \end{pmatrix}
\end{equation}


\emph{Exemple}~: Produit de matrices de bases canoniques. Soient \(n\), \(m\) et \(p\) trois naturels non nuls. Notons \((E_{ij})\) la base canonique de \(\Mnp{n}{p}{\K}\), \((E'_{ij})\) la base canonique de \(\Mnp{m}{n}{\K}\) et \((E''_{ij})\) la base canonique de \(\Mnp{m}{p}{\K}\). pour tout \(i \in \intervalleentier{1}{m}\), \(j \in \intervalleentier{1}{n}\), \(k \in \intervalleentier{1}{n}\) et \(l \in \intervalleentier{1}{p}\) on a
\begin{equation}
  E'_{ij} E_{kl} =(e_{\alpha \beta}) \in \Mnp{m}{p}{\K}.
\end{equation}
Alors
\begin{align}
  e_{\alpha \beta} &=\sum_{\gamma=1}^n (E'_{ij})_{\alpha \gamma} (E_{kl})_{\gamma \beta}\\
  &=\sum_{\gamma=1}^n \delta_{i\alpha} \delta_{j\gamma} \delta_{k\gamma} \delta_{l\beta} \\
  &= \delta_{i\alpha} \delta_{jk} \delta_{l\beta}\\
  &= \delta_{jk} (\delta_{i\alpha} \delta_{l\beta})\\
  &= \delta_{jk} (E''_{il})_{\alpha\beta}.
\end{align}
Donc
\begin{equation}
  E'_{ij} E_{kl} = \delta_{jk} E''_{il}.
\end{equation}

\subsubsection{Produit matriciel et composé d'applications linéaires}

Soient trois naturels non nuls \(m\), \(n\) et \(p\). Soient \(\E_c\) la base canonique de \(\K^n\), \(\F_c\) la base canonique de \(\K^m\),  \(\Gb_c\) la base canonique de \(\K^p\).

\begin{prop}
  Soient \(u \in \Lin{\K^p}{\K^n}\) et \(v \in \Lin{\K^n}{\K^m}\). Alors \(v \circ u \in \Lin{\K^p}{\K^m}\). Alors
  \begin{equation}
    \Mat_{\G_c,\F_c}(v \circ u) = \Mat_{\E_c,\F_c}(v) \Mat_{\G_c,\E_c}(u)
  \end{equation}
\end{prop}
\begin{proof}
Soient \(A= (a_{ij})=\Mat_{\G_c,\E_c}(u)\) et \(B=(b_{ij})= \Mat_{\E_c,\F_c}(v)\). On note \(C=BA\). Soit un naturel \(j \in \intervalleentier{1}{p}\) et alors
\begin{align}
  v \circ u (g_j) &= v\left( \sum_{i=1}^n a_{ij} e_i\right) \\
  &=\sum_{i=1}^n a_{ij} v(e_i) \\
  &=\sum_{i=1}^n a_{ij} \sum_{k=1}^m b_{ki} f_k \\
  &=\sum_{k=1}^m c_{kj} f_k
\end{align}
D'où \(BA=C=\Mat_{\G_c,\F_c}(v \circ u)\).
\end{proof}

\subsubsection{Écriture matricielle de l'effet d'une application linéaire sur un vecteur}

\begin{prop}
  Soient \(u \in \Lin{\K^p}{\K^n}\), \(A\) la matrice de \(u\) dans les bases canoniques \(\E_c\) et \(\F_c\) respectives de \(\K^p\) et \(\K^n\). Soit \(x \in \K^p\) et \(y=u(x)\). 
  \(X\) est le vecteur colonne des coordonnées de \(x\) dans \(\E_c\). \(Y\) est le vecteur colonne des coordonnées de \(u(x)\) dans \(\F_c\). Alors
  \begin{equation}
    Y=AX.
  \end{equation}
\end{prop}

\begin{proof}
  Notons \(X=\begin{pmatrix} x_1 \\ \vdots \\ x_p \end{pmatrix}\), \(A=(a_{ij})\) et \(Y=\begin{pmatrix} y_1 \\ \vdots \\ y_n \end{pmatrix}\). Pour tout \(i \in \intervalleentier{1}{n}\) on a \((AX)_i=\sum_{k=1}^p a_{ik}x_k\). Alors
  \begin{align}
    y&=u\left(\sum_{j=1}^p x_je_j \right)\\
    &=\sum_{j=1}^p x_j u(e_j)\\
    &=\sum_{j=1}^p x_j \sum_{i=1}^n a_{ij}f_i \\
    &=\sum_{i=1}^n \left( \sum_{j=1}^p x_j a_{ij}\right)f_i \\
    &=\sum_{j=1}^p (AX)_i f_i
  \end{align}
  Donc \(Y=AX\).
\end{proof}

\subsubsection{Propriétés du produit matriciel}

\begin{theo}
  Soient quatres naturels non nuls \(m\), \(n\), \(p\) et \(k\). Soit un scalaire \(\lambda \in \K\). Soient cinq matrices \((B,B') \in \Mnp{m}{n}{\K}^2\), \((C,C') \in \Mnp{n}{p}{\K}^2\) et \(C \in \Mnp{k}{m}{\K}\). Alors
  \begin{enumerate}
  \item pseudo associativité \(C(BA)=(CB)A\);
  \item pseudo distributivité \(B(A+A')=BA+BA'\);
  \item pseudo distributivité \((B+B')A=BA+B'A\);
  \item pseudo associativité \((\lambda B)A=\lambda (BA)=B(\lambda A)\);
  \item élément neutre \(I_n A=A\);
  \item élément neutre \(A I_p = A\).
  \end{enumerate}
\end{theo}
\begin{proof}
  Démontrons par exemple le deuxième point. Soient \(u\), \(u'\) et \(v\) les applications linéaires canoniquement respectiviements associées à \(A\), \(A'\) et \(B\). Alors
  \begin{align}
    B(A+A') =& \Mat(v) \cdot (\Mat(u)+\Mat(u')) \\
    &=\Mat(v \circ (u+u')) \\
    &=\Mat(v \circ u) + \Mat(v \circ u') \\
    &=BA +BA'.
  \end{align}

Les autres se démontrent de manière analogue.
\end{proof}

\emph{Remarque}~: L'application \(\fonction{F}{\Mnp{m}{n}{\K}\times \Mnp{n}{p}{\K}}{\Mnp{m}{p}{\K}}{(B,A)}{BA}\) est linéaire par rapport à chacun de ses arguments mais elle n'est pas linéaire. On dira que \(F\) est bilinéaire.

\subsection{Anneau \(\Mn{n}{\K}\)}

Soit un naturel non nul \(n\). On note \(\Mn{n}{\K}=\Mnp{n}{n}{\K}\).

\subsubsection{Structure}

On a déjà vu que \((\Mn{n}{\K},+,\perp)\) était un espace vectoriel. Alors \((\Mn{n}{\K},+)\) est un groupe commutatif. De plus la multiplication des matrices est une loi de composition interne sur \(\Mn{n}{\K}\). Elle est associative et distributive par rapport à l'addition. De plus \(I_n\) est l'élément neutre de la multiplication.

Ainsi, \((\Mn{n}{\K},+,\cdot)\) est un anneau.

\emph{Remarque}~: \(\Mn{n}{\K}\) est une \(\K\)-algèbre. C'est un anneau non-commutatif et non intègre dès que \(n\geqslant 2\).
\begin{equation}
  E_{11}E_{1n}=\delta_{11}E_{1n} \qquad E_{1n}E_{11}=\delta_{n1}E_{11}=0 \, (n\geqslant 2)
\end{equation}

\subsubsection{Isomorphisme canonique entre \(\Mn{n}{\K}\) et \(\Endo{\K^p}\)}

Soit \(\E_c\) la base canonique de \(\K^n\). Soit l'application
\begin{equation}
  \fonction{\Phi}{\Endo{\K^p}}{\Mn{n}{\K}}{u}{\Mat_{\E_c}(u)}.
\end{equation}

On sait que \(\Phi\) est un isomorphisme d'espaces vectoriels et de plus
\begin{equation}
  \forall (u,v) \in \Endo{\K^p}^2 \qquad \Phi(v \circ u) =\Phi(v) \Phi(u) \quad \Phi(\Id)=I_n
\end{equation}

Alors c'est un isomorphisme d'anneaux.

\subsection{Matrices carrées inversibles, groupe linéaire \(\GLn{n}{\K}\)}

On définit le groupe linéaire \(\GLn{n}{\K}\) comme étant l'ensemble des matrices carrées inversibles. C'est-à-dire
\begin{equation}
  \GLn{n}{\K}=\{A \in \Mn{n}{\K}, \exists B \in \Mn{n}{\K} \ AB=BA=I_n\}=\uinv{\Mn{n}{\K}}
\end{equation}

On en déduit
\begin{theo}
  \((\GLn{n}{\K},\cdot)\)  est un groupe, le groupe linéaire des matrices carrées.
\end{theo}


\begin{theo}
  Soit une matrice \(A \in \Mn{n}{\K}\) et \(u\) l'endomorphisme canoniquement associé à \(A\). Il y a équivalence entre les assertions suivantes~:
  \begin{enumerate}
  \item \(A \in \GLn{n}{\K}\);
  \item \(u \in \GLn{n}{\K^n}\);
  \item \(\exists B \in \Mn{n}{\K} \ AB=I_n\);
  \item \(\exists B \in \Mn{n}{\K} \ BA=I_n\).
  \end{enumerate}
\end{theo}
\begin{proof}
  Par définition \(1 \implies 3\) et \(1 \implies 4\). 

  \(1 \implies 2\)~: \(A\) est inversible alors par définition il existe une matrice \(B \in \Mn{n}{\K}\) telle que \(AB=BA=I_n\). Soit \(v\) l'endomorphisme canoniquement associée à \(B\). Ainsi \(\Mat(u \circ v) = \Mat(v \circ u) =AB=BA=I_n\). Donc \(v \circ u=u \circ v=\Id\). Alors \(u \in \GL{\K^n}\).

  \(2 \implies 1\)~: Comme \(u\) est inversible, il existe un endomorphisme \(v \in \Endo{\K^n}\) tel que \(u \circ v=v \circ u=\Id\). Soit alors \(B\) la matrice de \(v\) dans la base canonique. Alors
  \begin{equation}
    AB=\Mat(u) \Mat(v)=\Mat(u \circ v)=\Mat(\Id)
  \end{equation}
  alors \(AB=I_n\) et \(A \in \GLn{n}{\K}\).

  \(3 \implies 2\)~: Soit \(v\) l'endomorphisme canoniquement associé à \(B\) alors \(u\circ v=\Id\) est bijectif donc surjectif. Alors \(u\) est surjectif. Or \(u\) est un endormorphisme en dimension finie. Donc \(u\) est bijectif.

  \(4 \implies 2\)~: Soit \(v\) l'endomorphisme canoniquement associé à \(B\) alors \(v\circ u=\Id\) est bijectif donc injectif. Alors \(u\) est injectif. Or \(u\) est un endormorphisme en dimension finie. Donc \(u\) est bijectif.
\end{proof}

\begin{theo}
  L'application
  \begin{equation}
    \fonction{\tilde{\Phi}}{\GL{\K^n}}{\GLn{n}{\K}}{u}{\Mat_{\E_c}(u)}
  \end{equation}
  est un isomorphisme de groupes. Particulièrement pour toute isomorphisme \(u \in \GL{\K^n}\) on a \(\tilde{\Phi}(u^{-1})=\tilde{\Phi}(u)^{-1}\).
\end{theo}
\begin{proof}
  L'application \(\tilde{\Phi}\) est bien définie. Elle est injective puisque c'est une restriction de \(\Phi\). Elle est surjective car pour tout \(A \in \GLn{n}{\K}\) l'endomorphisme canoniquement associée à \(A\) est dans \(\GL{\K^n}\). De plus pour tout isomorphismes \(u\) et \(v\) de \(\GL{\K^n}\) on a
\begin{equation}
  \tilde{\Phi}(v \circ u)= \tilde{\Phi}(v) \cdot \tilde{\Phi}(u).
\end{equation}
 \(\tilde{\Phi}\) est un isomorphisme de groupes.
\end{proof}

\emph{Moyen pratique pour déterminer si une matrice est inversible}

Soit \(A \in \Mnp{n}{k}{\K}\) et \(u\) l'endomorphisme canoniquement associé. D'après ce qui précéde,
\begin{align}
  A \in \GLn{n}{\K} & \iff u \text{~est surjectif}\\
  & \iff \forall y \in \K^n \ \exists x \in \K^n \quad y=u(x)\\
  & \iff \forall Y \in \Mnp{n}{1}{\K} \ \exists X \in \Mnp{n}{1}{\K} \quad Y=AX.
\end{align}
C'est un système d'équations linéaires à \(n\) équations et \(n\) inconnues.
\begin{equation}
  Y=AX \iff \begin{cases} y_1 = \sum_{k=1}^n a_{1k} x_k \\ \ldots \\ y_n = \sum_{k=1}^n a_{nk} x_k \end{cases}.
\end{equation}

\subsection{Sous anneau des matrices diagonales et triangulaires}

\subsubsection{Définition}

Soient un naturel \(n\) et \(A=(a_{ij}) \in \Mn{n}{\K}\). On dit que~:
\begin{enumerate}
\item \(A\) est diagonale si et seulement si pour tout \((i,j) \in \intervalleentier{1}{n} ^2\) on a \(i \neq j \implies a_{ij}=0\). On note \(A=\text{diag}(a_{11},\ldots,a_{nn})\). L'ensemble des matrices diagonales de \(\Mn{n}{\K}\) est \(\Dr_n(\K)\).
\item \(A\) est triangulaire supérieure si et seulement si  pour tout \((i,j) \in \intervalleentier{1}{n} ^2\) on a \(i > j \implies a_{ij}=0\). L'ensemble des matrices diagonales de \(\Mn{n}{\K}\) est \(T_n^s(\K)\).
\item \(A\) est triangulaire inférieure si et seulement si  pour tout \((i,j) \in \intervalleentier{1}{n} ^2\) on a \(i < j \implies a_{ij}=0\). L'ensemble des matrices diagonales de \(\Mn{n}{\K}\) est \(T_n^i(\K)\).
\end{enumerate}

\subsubsection{Propriétés}

\begin{prop}
  \begin{equation}
    T_n^s(\K) \bigcap T_n^i(\K) = \Dr_n(\K)
  \end{equation}
\end{prop}
\begin{prop}
  \(\Dr_n(\K)\), \(T_n^s(\K)\) et \(T_n^i(\K)\) sont des sous espaces vectoriels de \(\Mn{n}{\K}\). De plus
  \begin{equation}
    \dim(\Dr_n(\K)) = n \quad \dim(T_n^s(\K))=\dim(T_n^i(\K))=\frac{n(n+1)}{2}
  \end{equation}
\end{prop}
\begin{proof}
  Soit une matrice diagonale \(A=\text{diag}(a_{11},\ldots,a_{nn})\), alors \(A=\sum_{i=1}^n a_{ii}E_{ii}\). Ainsi \(\Dr_n(\K)=\VectEngendre{(E_{ii})_{1\leqslant i\leqslant n}}\) et la famille \((E_{ii})_{1\leqslant i\leqslant n}\) est libre (puisque c'est une sous famille de la base canonique). Cette famille est donc une base de \(\Dr_n(\K)\) et alors on a bien la dimension \(\dim(\Dr_n(\K)) = \Card((E_{ii})_{1\leqslant i\leqslant n})=n\).

  De la même manière, on a  \(T_n^s(\K) = \VectEngendre{(E_{ij})_{1\leqslant i \leqslant j \leqslant n}}\). De la même manière la famille génératrice est une sous famille de la base canonique. Donc elle est libre. Finalement c'est une base. Le cardinal de cette famille vaut \(\frac{n(n-1)}{2}+n=\frac{n(n+1)}{2}\).

  Idem pour \(T_n^s(\K)\). 
\end{proof}

\begin{prop}
  \(\Dr_n(\K)\), \(T_n^s(\K)\) et \(T_n^i(\K)\) sont des sous anneaux de \(\Mn{n}{\K}\). De plus \(\Dr_n(\K)\) est commutatif, \(T_n^s(\K)\) et \(T_n^i(\K)\) ne sont ni commutatifs ni intègre (si \(n \geqslant 2\)).
\end{prop}
\begin{proof}
  Ce sont déjà des sous espaces vectoriels, alors il s'agit de vérifier s'ils sont stables par la multiplication et si l'élémentneutre \(I_n\) leurs appartient.

  Clairement, \(I_n\) est diagonale donc triangulaire supérieure et inférieure. 

  Soient ensuite deux matrices \(A\) et \(B\) diagonales et on note \(C=AB\). pour tout \(i\) et \(j\) dans \(\intervalleentier{1}{n}\) on a
  \begin{equation}
    \begin{cases}
      i \neq j & c_{ij} = \sum_{k=1}^n a_{ik}b_{kj}=a_{ii}b_{ij}=0 \\
      i=j & c_{ii}=a_{ii}b_{ii}
    \end{cases}
  \end{equation}
  Alors \(C\) est diagonale. De plus pour tout \(i \in \intervalleentier{1}{n}\) on a \(c_{ii}=b_{ii}a_{ii}\) donc \(AB=C=BA\). 
  
  Soient ensuite deux matrices \(A\) et \(B\) triangulaire supérieures et on note \(C=AB\). pour tout \(i\) et \(j\) dans \(\intervalleentier{1}{n}\) on a si \(i > j\)
  \begin{equation}
    c_{ij}=\sum_{k=1}^n a_{ik} b_{kj} = \sum_{k=1}^{i-1} a_{ik} b_{kj} + \sum_{k=i}^{n} a_{ik} b_{kj}
  \end{equation}
  Si \(k \in \intervalleentier{1}{i-1}\) alors \(a_{ik}=0\) et si \(k \in \intervalleentier{i}{n}\) alors \(b_{ik}=0\). D'où \(c_{ij}=0\). Alors \(C\) est triangulaire supérieure.

  La preuve est analogue pour la stabilité des matrices triangulaires inférieures.
\end{proof}

\begin{prop}
  Soit une matrice \(A \in \mathcal{A}\) où \(\mathcal{A}=\Dr_n(\K)\) ou \(T_n^s(\K)\) ou \(T_n^i(\K)\). Ainsi, \(A\) est inversible dans \(\mathcal{A}\) si et seulement si pour tout \(i \in \intervalleentier{1}{n}\) \(a_{ii} \neq 0\).
\end{prop}
\begin{proof}
  Dans le premier cas, on suppose que \(\mathcal{A}=\Dr_n(\K)\). Alors
  \begin{align}
    A \text{~est inversible dans } \Dr_n(\K) &\iff \exists B \in \Dr_n(\K) \ AB=BA=I_n \\
    &\iff \exists B = \text{diag}(b_{11}, \ldots, b_{nn}) \ \forall i \in \intervalleentier{1}{n} \ a_{ii}b_{ii}=1\\
    &\iff \forall i \in \intervalleentier{1}{n} \ a_{ii} \in \uinv{\K}\\
    &\iff \forall i \in \intervalleentier{1}{n} \ a_{ii} \neq 0.
  \end{align}

  Dans un deuxième cas, on suppose que \(\mathcal{A}=T_n^s(\K)\). Alors si on suppose que \(A\) est inversible dans \(T_n^s(\K)\) alors il existe une matrice triangulaire supérieure \(B\) telle que \(AB=BA=I_n\). Alors pour tout \(i \in \intervalleentier{1}{n}\)
  \begin{align}
    1=\sum_{k=1}^n a_{ik}b_{ki} &= \sum_{k=1}^{i-1}a_{ik}b_{ki} + a_{ii}b_{ii} + \sum_{k=i+1}^{n}a_{ik}b_{ki} \\
    &=a_{ii}b_{ii}
  \end{align}
   Alors \(a_{ii} \in \uinv{\K}\) et donc \(a_{ii} \neq 0\).

   Supposons maintenant que pour tout \(i \in \intervalleentier{1}{n}\), \(a_{ii}\) est non nul. Il s'agit de montrer que \(A \in \GLn{n}{\K}\) et que \(A^{-1}\) est triangulaire supérieure. Soient \(\E_c\) la base canonique de \(\K^n\) et \(u\) l'endomorphisme canoniquement associé à \(A\). 

   Alors pour tout \(j \in \intervalleentier{1}{n}\), on a \(u(e_j) = \sum_{i=1}^n a_{ij}e_i\). C'est-à-dire
   \begin{align}
     u(e_1) &=a_{11} e_{1} \\
     u(e_2) &=a_{11} e_{1} + a_{22} e_2\\     
     \ldots \\
     u(e_n) &= a_{1n}e_1 + \dotsb + a_{nn}e_n.
   \end{align}
   Alors comme les \(a_{ii}\) sont non nuls, on peut inverser le système d'équation et trouver les \(e_i\) en fonction des \(u(e_i)\). Alors pour tout \(j \in \intervalleentier{1}{n}\) on a \(e_j \in \VectEngendre(u(e_1), \ldots, u(e_j))\). La famille \(u(\E_c)\) est génératrice de \(\K^n\) et elle est de cardinal \(n\), c'est donc une base de \(\K^n\). Ainsi \(u\) est une application bijective. Alors au final \(A\) est inversible.
   \begin{equation}
     A^{-1} = \Mat_{\E_c}(u^{-1}) \quad \forall j \in \intervalleentier{1}{n} \ u^{-1}(e_j) \in \VectEngendre(e_1, \ldots, e_j).
   \end{equation}
   Alors \(A^{-1}\) est triangulaire supérieure.

   La démonstration est similaire pour la stabilité des matrices triangualires inférieures.
\end{proof}

\subsection{Transposition}

Soient deux naturel non nuls \(n\) et \(p\).

\begin{defdef}
  Soit une matrice \(A=(a_{ij}) \in \Mnp{n}{p}{\K}\). On définit une matrice appelée transposée de \(A\), notée \(A^\top \in \Mnp{p}{n}{\K}\), par \(A^\top=(\alpha_{ij})\in \Mnp{p}{n}{\K}\) telle que pour tout couple \((i,j) \in \intervalleentier{1}{p} \times \intervalleentier{1}{n}\) on ait \(\alpha_{ij}=a_{ji}\).
\end{defdef}

\emph{Exemple}~: \(\begin{pmatrix}1 & 3 \\ 2 & 6\end{pmatrix}^\top = \begin{pmatrix}1 & 2 \\ 3 &6\end{pmatrix}\)

\begin{prop}
  Pour toutes matrices \(A\) et \(B\) de \(\Mnp{n}{p}{\K}\) et tout scalaire \(\lambda\) on a~:
  \begin{enumerate}
  \item \((A^\top)^\top=A\);
  \item \((\lambda A)^\top=\lambda A^\top\);
  \item \((A+B)^\top=A^\top+B^\top\);
  \item pour toute matrice \(A\) de \(\Mnp{n}{p}{\K}\) et \(B\) de \(\Mnp{m}{n}{\K}\) \((BA)^\top=A^\top B^\top\).
  \end{enumerate}
\end{prop}
\begin{proof}
  On démontre le dernier point. On note \(A=(a_{ij})\), \(A^\top =(\alpha_{ij})\), \(B=(b_{ij})\) et \(B^\top =(\beta_{ij})\). On définit \(C=A^\top B^\top = (c_{ij})\) et \(D=BA=(d_{ij})\). Alors pour tout \(i \in \intervalleentier{1}{p}\) et tout \(j \in \intervalleentier{1}{m}\) on a
  \begin{equation}
    c_{ij}=\sum_{k=1}^m \alpha_{ik}\beta_{kj}=\sum_{k=1}^m a_{ki} b_{jk}=d_{ji}
  \end{equation}
  Alors \(C=D^\top\).
\end{proof}

\begin{prop}
  Soit \(A \in \GLn{n}{\K}\), alors \(A^\top \in \GLn{n}{\K}\) et
  \begin{equation}
    (A^\top)^{-1} = (A^{-1})^\top.
  \end{equation}
\end{prop}
\begin{proof}
  Si \(A \in \GLn{n}{\K}\), alors \(A^{-1}\) existe et \(AA^{-1}=I_n\) et en prenant la transposée, on a \((A^{-1})^\top A^\top=I_n\). Alors \(A^\top\) est inversible et \((A^\top)^{-1} = (A^{-1})^\top\).
\end{proof}

\subsection{Matrices symétriques et antisymétriques}

Soit un naturel non nul \(n\) et \(A \in \Mn{n}{\K}\). La matrice \(A\) est dite
\begin{itemize}
\item symétrique si et seulement si \(A^\top=A\);
\item antisymétrique si et seulement si \(A^\top=-A\).
\end{itemize}

L'ensemble des matrices de \(\Mn{n}{\K}\) symétriques est noté \(\SN{n}{\K}\) et celui des matrices antisymétriques \(\AN{n}{\K}\).

\begin{theo}
  Les ensembles \(\Mn{n}{\K}\) et \(\SN{n}{\K}\) sont des sous espaces vectoriels supplémentaires de \(\Mn{n}{\K}\).
  \begin{equation}
    \Mn{n}{\K}=\SN{n}{\K} \oplus \AN{n}{\K}.
  \end{equation}
  De plus toute matrice \(A \in \Mn{n}{\K}\) se décompose comme
  \begin{equation}
    A = \frac{1}{2} (A+A^\top) + \frac{1}{2} (A-A^\top)
  \end{equation}
  avec \(\frac{1}{2} (A+A^\top) \in \SN{n}{\K}\) et \(\frac{1}{2} (A-A^\top) \in \AN{n}{\K}\).
\end{theo}
\begin{proof}
  L'application \(\fonction{s}{\Mn{n}{\K}}{\Mn{n}{\K}}{A}{A^\top}\) est un endomrophisme involutif. C'est donc une symétrie. On voit \(\SN{n}{\K}=\Inv{s}\) et \(\AN{n}{\K}=\Opp{s}\). Alors ce sont des sous espaces vectoriels supplémentaires. Soit \(p\) le projecteur associé à \(s\). On a \(p=\frac{1}{2}(s+\Id)\) et \(q=\Id-p\). Alors \(\SN{n}{\K}=\Image(p)\) et \(\AN{n}{\K}=\Ker(p)=Image(q)\).

  Soit une matrice \(A \in \Mn{n}{\K}\). Alors
  \begin{align}
    A&= p(A)+ (A-p(A))\\
    &= p(A)+q(A)
  \end{align}
  La famille \(p(E_{ij})_{1\leqslant i,j \leqslant n}\) est génératrice de \(\Image(p)\) et pour tout \((i,j) \in \intervalleentier{1}{n}^2\) on a \(p(E_{ij})=p(E_{ji})\). Alors \(p(E_{ij})_{1\leqslant i \leqslant j \leqslant n}\) est génératrice de \(\Image(p)=\SN{n}{\K}\). Donc \(\dim(\SN{n}{\K}) \leqslant \frac{n(n+1)}{2}\). De même \(q(E_{ij})_{1\leqslant i,j \leqslant n}\) est génératrice de \(\Image(q)\) et pour tout \((i,j) \in \intervalleentier{1}{n}^2\) on a \(q(E_{ij})=-q(E_{ji})\), en particulier \(Q(E_{ii})=0\). Alors \(q(E_{ij})_{1\leqslant i \leqslant j \leqslant n}\) est génératrice de \(\Image(q)=\AN{n}{\K}\). Donc \(\dim(\AN{n}{\K}) \leqslant \frac{n(n-1)}{2}\). Or puisqu'ils sont supplémentaires, on a l'égalité dans les dimensions.
\end{proof}

\emph{Exemple pertinent}~: Pour toute matrice \(M \in \Mn{n}{\K}\), les matrices \(MM^\top\) et \(M^\top M\) sont symétriques. C'est la base de la décomposition en valeurs singulières.

\section{Matrice d'une application linéaire}

Soient deux entiers non nuls \(n\) et \(p\). Soit un \(\K\)-espace vectoriel \(E\) de dimension finie égale à \(p\) et \(\E=(e_1,\ldots,e_p)\) une base de \(E\). Soit un \(\K\)-espace vectoriel \(F\) de dimension finie égale à \(n\) et \(\F=(f_1,\ldots,f_n)\) une base de \(F\).

\subsection{Matrice d'une application linéaire, étant données deux bases}

\subsubsection{Définitions (Rappel)}

Soit \(u \in \Lin{E}{F}\). On appelle matrice de \(u\) dans les bases \(\E\) et \(\F\) (de \(E\) et \(F\)). On note \(\Mat_{\E,\F}(u)=(a_{ij})\in\Mn{n}{p}{\K}\) la matrice dont la \(j\)\ieme{} colonne représente les coordonnées de \(u(e_j)\) dans \(\F\). C'est-à-dire
\begin{equation}
  \forall j \in \intervalleentier{1}{p} \quad u(e_j) = \sum_{i=1}^n a_{ij} f_i.
\end{equation}

\begin{prop}
  Soient \(u \in \Lin{E}{F}\), \(x \in E\) et \(y=u(x)\). Soient \(A=\Mat_{\E,\F}(u)\), \(X\) le vecteur colonne des coordonnées de \(x\) dans \(\E\) et \(Y\) le vecteur colonne des coordonnées de \(y\) dans \(\F\). Alors \(Y=AX\).
\end{prop}

\emph{Remarque}~: Dans le cas où \(E=F\) et \(\E=\F\), on note \(\Mat_{\E}(u)=\Mat_{\E,\E}(u) \in \Mn{p}{\K}\).

\subsubsection{Propriétés}

Ces propriétés se démontrent de manière similaire que dans le cas ``base canonique''.
\begin{enumerate}
\item L'application \(\fonction{\Mat_{\E,\F}}{\Lin{E}{F}}{\Mnp{n}{p}{\K}}{u}{\Mat_{\E,\F}(u)}\) est un isomorphisme de \(\K\)-espaces vectoriels. Pour toute matrice \(A \in \Mnp{n}{p}{\K}\) il existe une unique application \(u \in \Lin{E}{F}\) telle que \(A=\Mat_{\E,\F}(u)\).
\item Soit un troisième espace vectoriel \(G\) de dimension finie non nulle \(m\). Soit \(\Gb=(g_a,\ldots,g_m)\) une base de \(\Gb\). Soient \(u \in \Lin{E}{F}\), \(v \in \Lin{F}{G}\) et \(v \circ u \in \Lin{E}{G}\). Alors
  \begin{equation}
    \Mat_{\E,\G}(v \circ u) = \Mat_{\F,\G}(v) \Mat_{\E,\F}(u).
  \end{equation}
\item L'application \(\fonction{\Mat_\E}{\Endo{E}}{\Mn{n}{\K}}{u}{\Mat_\E(u)}\) est un isomorphismes d'espaces vectoriels et d'anneaux.
\item L'application \(\fonction{\GL\Mat_\E}{\GL{E}}{\GLn{n}{\K}}{u}{\Mat_\E(u)}\) est un isomorphismes de groupes et en particulier
  \begin{equation}
    \forall u \in \GL{E} \quad \GL\Mat_\E(u^{-1})=\GL\Mat_\E(u)^{-1}.
  \end{equation}
\end{enumerate}

\subsubsection{Théorème de caractérisation des endomorphismes d'espaces vectoriels par les matrices}

\begin{theo}
  On suppose que \(E\) et \(F\) ont la même dimension \(n\) non nulle. Soit \(u \Lin{E}{F}\). Il y a équivalence entre les assertions suivantes~:
  \begin{enumerate}
  \item \(u\) est un isomorphisme de \(E\) dans \(F\);
  \item pour toute base \(\E\) de \(E\) et toute base \(\F\) de \(F\),
    \begin{equation}
      \Mat_{\E,\F}(u) \in \GLn{n}{\K};
    \end{equation}
  \item il existe une base \(\E\) de \(E\) et une base \(\F\) de \(F\),
    \begin{equation}
      \Mat_{\E,\F}(u) \in \GLn{n}{\K}.
    \end{equation}
  \end{enumerate}
\end{theo}
\begin{proof}
  \(1 \implies 2\) On dispose de l'application \(u^{-1} \in \GL(F)\). On sait que \(u^{-1} \circ u = \Id\). Pour toutes bases de \(\E\) de \(E\) et \(\F\) de \(F\), on a
  \begin{equation}
    \Mat_{\E,\E}(u^{-1} \circ u)=\Mat_{\F,\E}(u^{-1}) \Mat_{\E,\F}(u)=I_n.
  \end{equation}
  Alors \(\Mat_{\E,\F}(u) \in \GLn{n}{\K}\).

\(2 \implies 3\) Comme \(E\) et \(F\) sont de dimension finie, ils admettent des bases.

\(3 \implies 1\) Soit \(B=(\Mat_{\E,\F}(u))^{-1}\) et soit \(v\) l'unique application linéaire de \(F\) dans \(E\) telle que \(B=\Mat_{\F,\E}(v)\). Alors
\begin{equation}
  I_n = \Mat_{\E,\F}(u) \cdot B = B \cdot \Mat_{\E,\F}(u),
\end{equation}
et comme \(B=\Mat_{\F,\E}(v)\), on a
\begin{equation}
  I_n = \Mat_{\F,\F}(u \circ v) = \Mat_{\E,\E}(v \circ u)
\end{equation}
Alors \(u \circ v = \Id_F\) et \(v \circ u = \Id_E\). L'application \(u\) est bijective.
\end{proof}

\subsection{Matrice dans une base des coordonnées d'une famille finie de vecteurs}

\subsubsection{Définitions}

\begin{defdef}
  Soit un naturel \(q\) non nul et \(\X=(x_i)_{1 \leqslant i \leqslant q} \in E^q\). On définit la matrice de \(\X\) dans la base \(\E\) de \(E\), notée \(\Mat_\E(\X)\) par
  \begin{equation}
    \Mat_{\E}(\X) =
    \begin{pmatrix}
      a_{11} & \ldots & a_{1q} \\
      \vdots & a_{ij} & \vdots \\
      a_{p1} & \ldots & a_{pq}
    \end{pmatrix}
  \end{equation}

  La \(j\)\ieme{} colonne de \(\Mat_{\E}(\X)\) est composée de coordonnées de \(x_j\) dans la base \(\E=(e_1, \ldots, e_p)\). Pour tout \(j \in \intervalleentier{1}{q}\) on a
  \begin{equation}
    x_j = \sum_{i=1}^p a_{ij}e_i.
  \end{equation}
  \(\Mat_{\E}(\X) \in \Mnp{p}{q}{\K}\). Particulièrement qi \(q=1\), \(\X=(x)\) avec \(x \in E\), et alors \(\Mat_{\E}(\X)\) est la matrice colonne des coordonnées du vecteur \(x\) dans la base \(\E\).
\end{defdef}

\subsubsection{Caractérisation des bases parmi les matrice}

\begin{theo}
  Soit \(\X=(x_i)_{1 \leqslant i\leqslant p}\) une famille de \(p\) vecteurs de \(E\) (\(p=\dim E\)). Il y a équivalence entre les assertions suivantes~:
  \begin{enumerate}
  \item \(\X\) est une base de \(E\);
  \item pour toute base \(\E\) de \(E\), \(\Mat_\E(\X) \in \GLn{p}{\K}\);
  \item Il existe une base \(\E\) de \(E\) telle que \(\Mat_\E(\X) \in \GLn{p}{\K}\).
  \end{enumerate}
\end{theo}
\begin{proof}
  \(1 \implies 2\)~: Soit \(E\) une base de \(E\). On définit \(u\) l'unique application linéaire de \(E\) dans \(E\) telle que \(u(\E)=\X\). Comme \(\E\) et \(\X\) sont deux bases de \(E\), \(u\) est bijective et \(\Mat_\E(\X)=\Mat_\E(u) \in \GLn{p}{\K}\).

  \(2 \implies 3\)~: L'espace vect\(E\) est de dimension finie, donc il admet au moins une base.

  \(3 \implies 1\)~: Supposons qu'il existe une base \(\E\) de \(E\) telle que \(\Mat_\E(\X) \in \GLn{p}{\K}\). Soit \(u\) l'unique endomorphisme de \(E\) tel que \(\Mat_\E(u)=\Mat_\E(\X)\). Cette matrice est inversible donc \(u\) est aussi inversible. Puisque \(u(\E)=\X\), \(u \in \GL{E}\), et \(\E\) est une base de \(E\) alors \(\X\) est une base de \(E\).
\end{proof}

\subsection{Matrice dans une base d'une famille finie de formes linéaires}

Soient un naturel \(q\) non nul et \(\F=(f_1, \ldots f_q) \in (E^*)^q\). On définit la matrice de la famille des formes linéaires \(\F\) dans la base \(\E='e_1, \ldots, e_p)\) par~:
\begin{equation}
  \Mat_\E(\F) = (f_i(e_j)) \in M_{q,p}(\K).
\end{equation}

\subsection{Matrices de passage, formule de changement de base}

Soient \(E\) un \(\K\)-espace vectoriel de dimension \(n \in \N^*\) et deux bases de \(E\) \(\E=(e_1, \ldots, e_n)\) \(\E'=(e'_1, \ldots, e'_n)\). 

\subsubsection{Matrices de passages}

\begin{defdef} 
  On appelle matrice de passage de \(\E\) à \(\E'\) et on note \(\P_{\E,\E'}\), la matrice
  \begin{equation}
    \P_{\E,\E'} = \Mat_\E(\E') \in \Mn{n}{\K}.
  \end{equation}
  C'est la matrice carrée dont les colonnes représentent les coordonnées des vecteurs de \(\E'\) dans la base \(\E\).
\end{defdef}

\begin{prop}
  \begin{enumerate}
  \item \(\P_{\E,\E'}\) est inversible car \(\E\) et \(\E'\) sont inversibles.
  \item \(\P_{\E,\E'} = \Mat_{\E',\E}(\Id_E)\) puisque pour tout \(j\), \(j\)\ieme{} vecteur colonne de la matrice est le vecteur des coordonnées de \(e'_j=\Id_E(e'_j)\) dans la base \(E\).
  \item Si \(\E''\) est une troisième base de \(E\),
    \begin{equation}
      \P_{\E,\E''}=\P_{\E,\E'} \cdot \P_{\E',\E''}.
    \end{equation}
    \begin{proof}
      On sait que \(\P_{\E,\E''} = \Mat_{\E'',\E}(\Id_E)\), alors
      \begin{align}
        \P_{\E,\E'} \cdot \P_{\E',\E''} &= \Mat_{\E',\E}(\Id_E) \Mat_{\E'',\E'}(\Id_E) \\
        &= \Mat_{\E'',\E}(\Id_E \circ \Id_E) \\
        &=\P_{\E,\E''}
      \end{align}
    \end{proof}
  \item \((\P_{\E,\E'})^{-1} = \P_{\E',\E}\).
    \begin{proof}
      D'après le troisième point, on a \(\P_{\E,\E'} \P_{\E',\E}=\P_{\E,\E}=I_n\).
    \end{proof}
  \end{enumerate}
\end{prop}

\subsubsection{Effet d'un changement de base sur la matrice colonne d'un vecteur}

\begin{theo}
  Soient \(\E\) une base de \(E\) (``l'ancienne'') et \(\E'\) une autre base de \(E\) (``la nouvelle''). 

Soit \(P=\P_{\E,\E'}\) la matrice de passage de l'ancienne à la nouvelle base. Soit un vecteur \(x \in E\), \(X=\Mat_\E(x)\), \(X'=\Mat_{\E'}(x)\). 

Alors \(X=PX'\), \danger les anciennes coordonnées en fonction des nouvelles.
\end{theo}
\begin{proof}
  Pour tout vecteur \(x \in E\), on a \(\Id_E(x)=x\). On a \(\P_{\E,\E'} = \Mat_{\E,\E'}(\Id_E)\) alors \(X = \Mat_{\E,\E'}(\Id_E) X'\) c'est-à-dire \(X=PX'\).
\end{proof}

\subsubsection{Effet d'un changement de base sur la matrice d'une application linéaire}

\begin{theo}
  Soient un \(\K\)-espace vectoriel \(E\) de dimension finie \(p\) non nulle et un autre \(\K\)-espace vectoriel \(F\) de dimension finie \(n\) non nulle. Soient \(\E\) et \(\E'\) des bases de \(E\), \(\F\) et \(\F'\) des bases de \(F\). Soient les matrices de passages \(P=\P_{\E,\E'} \in \GLn{p}{\K}\) et \(Q=\P_{\F,\F'} \in \GLn{n}{\K}\). Soit une application linéaire \(u \in \Lin{E}{F}\) telle que \(A=\Mat_{\E,\F}(u)\) et \(A'=\Mat_{\E',\F'}(u)\). Alors
  \begin{equation}
    A'=Q^{-1}AP.
  \end{equation}
\end{theo}
\begin{proof}
  Soit  \(u \in \Lin{E}{F}\), alors \(u \circ \Id_E = u = \Id_F \circ u\). Donc \(\Mat_{\E',\F}(u \circ \Id_E) = \Mat_{\E',\F}(\Id_F \circ u)\) Alors en développant \(\Mat_{\E,\F}(u) \Mat_{\E',\E}(\Id_E) =\Mat_{\F',\F}(\Id_F) \Mat_{\E',\F'}(u) \). Donc au final \(AP=QA'\) et comme \(Q\) est inversible on a \(A'=Q^{-1}AP\).
\end{proof}

\begin{theo}
   Soient un \(\K\)-espace vectoriel \(E\) de dimension finie \(p\) non nulle, \(\E\) et \(\E'\) des bases de \(E\), et \(u\) un endomorphisme de \(E\) tel que \(A=\Mat_{\E}(u)\) et \(A'=\Mat_{\E'}(u)\) alors \(A'=P^{-1}AP\).
\end{theo}
\begin{proof}
  C'est un cas particulier du théorème précédent.
\end{proof}

\section{Rang d'une matrice}

Soient deux naturel non nuls \(n\) et \(p\), et une matrice \(A \in \Mnp{n}{p}{\K}\).

\subsection{Rang du système des vecteurs colonnes}

On note \(A=(a_{ij})_{1\leqslant i\leqslant n, 1\leqslant j\leqslant p}\) et pour tout \(j \in \intervalleentier{1}{p}\) le vecteur colonne \(j\) est \(c_j=(a_{ij})_{1 \leqslant i \leqslant n} \in \K^n\). La famille des vecteurs colonnes de la matrice \(A\) est \(\courbe_A=(c_j)_{1 \leqslant j \leqslant p} \in (\K^n)^p\).

\begin{defdef}
  On appelle rang de la matrice \(A\) et on note \(\rg(A)\)  le rang de la famille de vecteur \(\courbe_A\) des colonnes de \(A\).
\end{defdef}

On en déduit que \(\rg(A) \leqslant \min(n,p)\) car \(\courbe_A\) est une famille de \(p\) vecteurs de \(\K^n\). De plus,
\begin{align}
  \rg(A) = 0 &\iff \dim \VectEngendre(\courbe_A)=0 \\
  &\iff \forall j \in \intervalleentier{1}{p} c_j = 0_{\K^n} \\
  \iff A=0
\end{align}

\begin{prop}
  Soit un \(\K\)-espace vectoriel \(E\) de dimension finie \(p\) non nulle, \(\E\) une base de \(E\) et \(\X=(x_i)_{1\leqslant i\leqslant n}\) une famille de vecteur de \(E\). Alors
  \begin{equation}
    \rg(\X)=\rg(\Mat_\E(\X)).
  \end{equation}
\end{prop}
\begin{proof}
  Soit \(u: E \rightarrow \K^p\) qui à \(x \in E\) associe ses coordonnées dans la base \(\E\). C'est donc un isomorphisme de \(E\) sur \(\K^p\). Pour tout \(j \in \intervalleentier{1}{n}\) on a \(u(x_j)=c_j\) où les \(c_j\) sont les colonnes de \(\Mat_\E(\X)\). On a
  \begin{equation}
    \rg(\Mat_\E(\X)) = \rg(c_1,\ldots,c_n) = \rg(u(x_1),\ldots,u(x_n))=\rg(X),
  \end{equation}
  puisque \(u\) est bijective.
\end{proof}

\begin{corth}[Première interprétation]
  Le rang d'une matrice est égal au rang de toute famille de vecteurs représentée dans une base par cette matrice.
\end{corth}

\subsection{Rang de l'application linéaire associée}

\begin{theo}
  Soient un \(\K\)-espace vectoriel \(E\) de dimension finie \(p\) non nulle et un autre \(\K\)-espace vectoriel \(F\) de dimension finie \(n\) non nulle. Soient \(\E\) et \(\F\) et \(\F'\) deux bases respectives de \(E\) et \(F\). Soit \(u \in \Lin{E}{F}\). Alors
  \begin{equation}
    \rg(u) = \rg(\Mat_{\E,\F}(u))
  \end{equation}
\end{theo}
\begin{proof}
  On a
  \begin{align}
    \rg(u) = \dim\Image(u) &= \dim \VectEngendre u(\E) \\
    &= \rg(u(\E))\\
    &=\rg(\Mat_{\F}(u(\E))\\
    &=\rg(\Mat_{\E,\F}(u)).
  \end{align}
\end{proof}

\begin{corth}[Deuxième interprétation]
  Le rang d'une matrice est le rang de toute application linéaire représentés par cette matrice dans des bases.
\end{corth}

\begin{corth}
  Soit \(A \in \Mn{n}{\K}\), alors \(A\) est inversible si et seulement si \(\rg(A)=n\).
\end{corth}
\begin{proof}
  Soit \(u\) l'endomorphisme canoniquement associé à \(A\). Alors
  \begin{align}
    A \in \GLn{n}{\K} &\iff u \text{~bijectif}\\
    &\iff u \text{~surjectif}\\
    &\iff \Image(u)=E\\
    &\iff \rg(u) = n\\
    &\iff \rg(A) = n.
  \end{align}
\end{proof}

\subsection{Matrice \(J_r\)}

Soient deux naturel non nuls \(n\) et \(p\). Soit un troisième naturel non nul \(r\) tel que \(r \leqslant \min(n,p)\). Posons
\begin{equation}
  J_{n,p,r}=
  \begin{pmatrix}
    I_r & 0_{r,p-r} \\ 0_{n-r,r} & 0_{n-p,p-r}
  \end{pmatrix}.
\end{equation}

\begin{prop}
  \begin{equation}
    \rg(J_{n,p,r})=r.
  \end{equation}
\end{prop}

\begin{theo}
  Soit une matrice \(A \in \Mnp{n}{p}{\K}\) et un un naturel \(r\) non nul tel que \(r \leqslant \min(n,p)\). Alors \(r=\rg(A)\) si et seulement s'il existe deux matrices \(U \in \GLn{n}{\K}\) et \(V \in \GLn{p}{\K}\) telles que \(A=UJ_{n,p,r}V\).
\end{theo}
\begin{proof}
  \(\implies\) Soit l'application linéaire \(u\) de \(\K^p\) vers \(\K^n\) canoniquement associée à \(A\). Nous allons construire des bases de \(\K^p\) et \(\K^n\) dans lesquelles la matrice de \(u\) sera \(J_{n,p,r}\). 
  
  Comme \(\rg(u)=\rg(A)=r\), d'après le théorème du rang, on a
  \begin{equation}
    \dim \K^p = \rg(u)+\dim \Ker(u),
  \end{equation}
  alors \(\dim \Ker(u) = p-r\). Soit \((e_{r+1}, \ldots, e_p)\) une base de \(\Ker(u)\). Par théorème de la base incomplète, on peut compléter cette famille libre de \(\K^p\) en base de \(\K^p\) : \((e_1, \ldots, e_r, e_{r+1}, \ldots, e_p)\). Pour tout \(i \in \intervalleentier{1}{r}\), on pose \(f_i = u(e_i)\). Alors
  \begin{align}
    \rg(f_i)_{1 \leqslant i \leqslant r} &= \rg(u(e_i))_{1 \leqslant i \leqslant r}\\
    &= \rg(u(e_i))_{1 \leqslant i \leqslant p}\\
    &= \rg(u(\E))\\
    &=\rg(u)=r.
  \end{align}
  Donc \((f_i)_{1\leqslant i\leqslant r}\) est une famille libre de \(\K^p\) (voir proposition~
\ref{prop:caracrangbase}). On peut donc compléter cette famille en une base \(\F=(f_1, \ldots, f_r,f_{r+1}, \ldots, f_n)\). Alors pour tout \(i \in \intervalleentier{1}{r}\) \(u(e_i)=f_i\), et pour tout \(i \in \intervalleentier{r+1}{p}\) \(u(e_i)=0\). Donc \(\Mat_{\E,\F}(u) = J_{n,p,r}\).

  On a \(A=\Mat_{\E_c,\F_c}(u)\). Soient \(P=\P_{\E_c,\E} \in \GLn{p}{\K}\) et \(Q=\P_{\F_c,\F} \in \GLn{n}{\K}\). Alors \(J_{n,p,r}=Q^{-1}AP\), c'est-à-dire \(A=QJ_{n,p,r}P^{-1}\). En posant \(U=Q\), \(V=P^{-1}\) on a bien \(A=U J_{n,p,r} V\).
  
  \(\impliedby\)  On définit les famille \(\E\) et \(\F\) de vecteurs respectives de \(\K^p\) et \(\K^n\) par~:
  \begin{equation}
    U=\Mat_{\F_c}(\F)=\P_{\F_c,\F} \quad V^{-1}=\Mat_{\E_c}(\E)=\P_{\E_c,\E}.
  \end{equation}
  Comme \(U\) et \(V^{-1}\) sont inversibles, \(\E\) et \(\F\) sont des bases respectives de \(E\) et \(F\). 

  Soit \(u\) l'unique application linéaire de \(E\) vers \(F\) telle que \(\Mat_{\E,\F}(u)=J_{n,p,r}\). On voit déjà que \(\rg(u)=r\). Ensuite on a
  \begin{align}
    A = U J_{n,p,r} V &= \P_{\F_c,\F} \Mat_{\E,\F}(u) (\P_{\E_c,\E})^{-1} \\
    &=(\P_{\F,\F_c})^{-1} \Mat_{\E,\F}(u) \P_{\E,\E_c}.
  \end{align}
  La matrice \(A\) est donc représentative de l'application \(u\). Donc d'après la deuxième interprétation du rang on a \(\rg(A)=\rg(u)=r\).
\end{proof}

 On a montré que toute application linéaire \(u: E \rightarrow F\) de rang \(r\) avec \(\dim E=p\) et \(\dim F=n\) peut être représentée dans des bases de \(E\) et \(F\) par la matrice \(J_{n,p,r}\).

\subsection{Rang de la transposée}

\begin{theo}
  Pour toute matrice \(A \in \Mnp{n}{p}{\K}\), on a
  \begin{equation}
    \rg(A^\top)=\rg(A).
  \end{equation}
\end{theo}
\begin{proof}
  Soit \(r=\rg(A)\), il existe deux matrices \(U \in \GLn{n}{\K}\) et \(V \in \GLn{p}{\K}\) telles que \(A=U J_{n,p,r} V\). Alors 
  \begin{equation}
    A^\top = V^\top J_{n,p,r}^\top U^\top = V^\top J_{p,n,r} U^\top.
  \end{equation}
  Alors \(\rg(A^\top)=\rg(A)\).
\end{proof}

\subsection{Rang d'un système de vecteurs lignes}

On note pour tout toute matrice \(A \in \Mnp{n}{p}{\K}\) et pour tout \(i \in \intervalleentier{1}{n}\), \(L_i=(a_{i1}, \ldots, a_{ip}) \in \K^p\) ety \(\L_A=(L_i)_{1\leqslant i \leqslant n}\).
\begin{prop}
  Pour toute matrice \(A \in \Mnp{n}{p}{\K}\), on a
  \begin{equation}
    \rg(\L_A)=\rg(A).
  \end{equation}
\end{prop}
\begin{proof}
  \begin{equation}
    \rg(A)=\rg(A^\top)=\rg(\courbe_{A^\top})=\rg(\L_A).
  \end{equation}
\end{proof}
