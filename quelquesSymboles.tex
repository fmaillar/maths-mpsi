\chapter{Quelques symboles}\label{chap:qqsymboles}

\section{Alphabet grec}

\begin{tabularx}{\textwidth}{XXX}
  \(\alpha, A \): alpha & \(\iota, I\): iota & \(\rho, R\): rho \\
  \(\beta, B\): beta & \(\kappa, K\): kappa & \(\sigma, \Sigma\): sigma \\
  \(\gamma, \Gamma\): gamma & \(\lambda, \Lambda\): lambda & \(\tau, T\): tau 
  \\
  \(\delta, \Delta\): delta & \(\mu, M\): mu & \(\upsilon, \Upsilon\): 
  upsilon \\
  \(\epsilon, E\): epsilon & \(\nu, N\): nu & \(\varphi, \Phi\): phi \\
  \(\zeta, Z\): zêta & \(\xi, \Xi\): xi & \(\chi, X\): chi \\
  \(\eta, E\): êta & \(o, O\): omicron & \(\psi, \Psi\): psi\\
  \(\theta, \Theta\): theta & \(\pi, \Pi\): pi & \(\omega, \Omega\): omega
\end{tabularx}

\section{Quantificateurs}

Le quantificateur universel, noté \(\forall\), veut dire ``pour tout'' ou 
``quelque soit''. Le quantificateur existentiel, noté \(\exists\), signifie ``il 
existe \ldots{} tel que''. Le quantificateur existentiel, noté \(\exists!\), 
signifie ``il existe un unique \ldots{} tel que''.

\section{Symboles ensemblistes}

Le symbole \(\in\) signifie ``appartient à'' et exprime l'appartenance d'un 
élément à un ensemble. Sa négation est notée \(\notin\). Par exemple \(1 \in 
\N\), \(2+\ii \notin \R\). Le symbole \(\subset\) signifie ``est contenu dans'' 
et exprime l'inclusion d'un ensemble dans un autre. Sa négation est notée 
\(\not\subset\). Par exemple \(N \subset \R\) mais \(\C \not\subset \R\). Le 
symbole \(\subsetneq\) signifie ``est contenu dans et est différent''. Par 
exemple l'ensemble des fonctions à valeurs réelles dérivables sur un intervalle 
\(I\) sont continues sur \(I\), mais les fonctions continues ne sont pas toutes 
dérivables: \(\Dr(I,\R) \subsetneq \classe{0}(I,\R)\). Le symbole \(\emptyset\) 
est ``l'ensemble vide'', il ne possède aucun élément.

\(\cup\) ``union'', c'est l'union de deux ensembles. Par exemple, soit deux 
ensembles \(A\) et \(B\) alors \(A \cup B\) est l'ensemble des éléments 
appartenant à \(A\) \emph{ou} à \(B\).

\(\cap\) ``inter'', c'est l'intersection de deux ensembles. Par exemple, soit 
deux ensembles \(A\) et \(B\) alors \(A \cap B\) est l'ensemble des éléments 
appartenant à \(A\) \emph{et} à \(B\).

\(\setminus\) ``privé de'', c'est la différence de deux ensembles. Par exemple, 
soit deux ensembles \(A\) et \(B\) alors \(A \setminus B\) est l'ensemble des 
éléments appartenant à \(A\) \emph{mais pas} à \(B\).

\(\Delta\) ``différence symétrique'', c'est la différence symétrique.
