\chapter{Logique et quantification}

\begin{Exercise}
  Soit $(a, b, c) \in \R^3$. Écrire les négations des propositions suivantes~:
  \begin{enumerate}
  \item $a=b=c$;
  \item $ab=0$;
  \item $-3 \leq a \leq 3 \leq c$.
  \end{enumerate}
\end{Exercise}
%
\begin{Exercise}
  Soit $x$ un réel quelconque. Que signifient les propositions suivantes et que peut-on en conclure ?
  \begin{enumerate}
  \item $\forall \epsilon>0 \quad x<epsilon$;
  \item $\forall \epsilon>0 \quad \abs{x}<\epsilon$;
  \item $\exists \epsilon>0 \quad x<\epsilon$;
  \end{enumerate}

\end{Exercise}

%%% Local Variables: 
%%% mode: latex
%%% TeX-master: "mathematiques"
%%% End: 
