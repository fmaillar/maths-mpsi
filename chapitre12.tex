\chapter{Limites et continuité des fonctions réelles de la variable réelle}

\minitoc%
\minilof%
\minilot%
\section[Ensemble des fonctions de \(X\) vers \(\R\)]{Ensemble des fonctions de
\(X\) vers \(\R\) \((\R^X, +, \cdot, \perp,\leqslant)\)}

\subsection{\(\R\)-espace vectoriel \((\R^X, +, \perp)\) \& anneau \((\R^X, +,
\cdot)\) }

On note \(\R^X\) l'ensemble des applications de \(X\) vers \(\R\). On munit
\(\R^X\) de deux lois de composition internes, l'une est  appelée addition notée
\(+\) et l'autre est appelée multiplication notée \(\cdot\) ou rien. On le munit
aussi d'une loi de composition externe appelée multiplication par un réel notée
\(\perp\). Ces opérations sont telles que pour toutes fonctions \(f\) et \(g\)
de \(\R^X\) et tout réel \(\lambda\)
\begin{align}
  \fonction{f+g}{X}{\R}{x}{f(x)+g(x)}, \\
  \fonction{fg}{X}{\R}{x}{f(x)g(x)}, \\
  \fonction{\lambda f}{X}{\R}{x}{\lambda f(x)}.
\end{align}
\begin{prop}
  L'ensemble \((\R^X, +, \perp)\) est un \(\R\)-espace vectoriel. L'élément
  neutre pour l'addition est la fonction nulle \(\tilde{0}\).
\end{prop}
\begin{prop}
  L'ensemble \((\R^X, +, \cdot)\) est un anneau commutatif non intègre (dès que
  \(X\) admet au moins deux éléments). De plus pour toutes fonctions \(f\) et
  \(g\) de \(\R^X\) et tout réel \(\lambda\) on a
  \begin{equation}
    \lambda(fg)=(\lambda f)g=f(\lambda g).
  \end{equation}
\end{prop}
Dans ce cas, on dit que \((\R^X,+,\cdot,\perp)\) est une \(\R\)-algèbre. Le
neutre pour la multiplication est la fonction constante égale à un.
\begin{defdef}
  Soit \(f \in \R^X\). On suppose que pour tout réel \(x\), \(f(x)\neq 0\),
  alors on peut définir la fonction inverse de \(f\) notée \(\dfrac{1}{f}\)
  telle que \(\fonction{\dfrac{1}{f}}{X}{\R}{x}{\dfrac{1}{f(x)}}\).
\end{defdef}

\subsection{Ensemble \((\R^X, \leqslant)\) partiellement ordonné}

On définit une relation d'ordre sur \(\R^X\) à partir de la relation d'ordre
usuel \(\leqslant\) sur \(\R\) de la manière suivante
\begin{equation}
  \forall f,g \in \R^X \quad f \leqslant g \iff \forall x \in X \ f(x) \leqslant
  g(x).
\end{equation}
Cet ordre n'est pas total. En effet, on peut trouver des fonctions qui ne sont
pas comparables par \(\leqslant\). Attention à la notation \(f < g\), car elle
est ambiguë puisqu'elle peut signifier
\begin{itemize}
  \item soit \(\forall x \in X \ f(x) < g(x)\);
  \item soit \(f \leqslant g\) et \(f \neq g\) c'est-à-dire que pour tout \(x
    \in \X\), \(f(x) \leqslant g(x)\) et \(\exists x_0 \in X \ f(x_0) <
    g(x_0)\).
\end{itemize}
Ne pas utiliser cette notation si possible. En particulier se méfier de la
notation \(f > \tilde{0}\) qui peut signifier
\begin{itemize}
  \item soit \(f\) est à valeur strictement positives;
  \item soit \(f\) est à valeurs positives ou nulles et \(f\) est différente de
    \(\tilde{0}\).
\end{itemize}

Les propriétés de compatibilité de l'ordre sur \(\R^X\) avec l'addition, la
multiplication par un réel et la multiplication interne se déduisent des
propriétés de compatibilité de l'ordre sur \(\R\) avec l'addition et la
multiplication.

\subsection{Application valeur absolue, borne supérieure et borne inférieure}

\subsubsection{Valeur absolue}

\begin{defdef}
  Pour toute fonction \(f \in \R^X\), on définit une fonction appelée valeur
  absolue notée \(\abs{f}\) et définie par
  \begin{equation}
    \forall x \in X \quad \abs{f}(x)=\abs{f(x)}.
  \end{equation}
\end{defdef}

\subsubsection{Borne supérieure et borne inférieure}
\begin{defdef}
  Soient \(f\) et \(g\) deux fonctions de \(X\) dans \(\R\). On définit deux
  applications \(S\) et \(I\) de \(X\) vers \(\R\) par
  \begin{gather}
    \fonction{S}{X}{\R}{x}{\max(f(x),g(x))},\\
    \fonction{I}{X}{\R}{x}{\min(f(x),g(x))}.
  \end{gather}
\end{defdef}

\begin{prop}
  La fonction \(S\) (resp.\ \(I\)) est la borne supérieure (resp.\ inférieure)
  de \(\{f,g\}\) dans l'ensemble ordonné \((\R^X, \leqslant)\). On notera
  \begin{equation}
    S=\sup(f,g), \quad I=\inf(f,g).
  \end{equation}
\end{prop}
\begin{proof}
  Pour la fonction \(S\). Montrons que \(S\) est le plus petit des majorants de
  \(\{f,g\}\) dans \((\R^X, \leqslant)\). Soit un réel \(x \in X\)
  \(S(x)=\max(f(x),g(x))\) alors \(S(x) \geqslant f(x)\) et \(S(x) \geqslant
  g(x)\) donc \(S \geqslant f\) et \(S \geqslant g\) donc \(S\) est un majorant
  de \(\{f,g\}\). Soit \(h\) un majorant de \(\{f,g\}\) dans \(\R^X\),
  c'est-à-dire que \(h \geqslant f\) et \(h \geqslant g\) et donc que pour tout
  réel \(x\) de la partie \(X\) on a \(h(x) \geqslant f(x)\) et \(h(x) \geqslant
  g(x)\) donc que \(h(x) \geqslant \max(f(x),g(x))=S(x)\) donc \(h \geqslant
  S\). Alors \(S\) est le plus petit majorant. Finalement \(S=\sup\{f,g\}\).
\end{proof}
\begin{prop}
  On peut exprimer différemment \(\sup\) et \(\inf\), en effet pour toutes
  fonctions \(f\) et \(g\) de \(X\) dans \(\R\) on a
  \begin{equation}
    \sup(f,g)=\frac{1}{2}(f+g+\abs{f-g}) \quad
    \inf(f,g)=\frac{1}{2}(f+g-\abs{f-g}).
  \end{equation}
\end{prop}
\begin{proof}
  Voir celle du chapitre~\ref{chap:reels}.
\end{proof}
\begin{cor}
  Pour toute fonction \(f \in \R^X\),
  \begin{equation}
    \abs{f}=\sup(f,-f).
  \end{equation}
\end{cor}
\begin{proof}
  En effet, \(\sup(f,-f)=\frac{1}{2}(f-f+\abs{2f})=\abs{f}\).
\end{proof}

\subsubsection{Applications \(f^+\) et \(f^-\)}

\begin{defdef}
  Soit une fonction \(f \in \R^X\), on lui associe deux applications de \(X\)
  dans \(\R\), notées \(f^+\) et \(f^-\) définies par
  \begin{equation}
    f^+=\sup(f,0)=\frac{\abs{f}+f}{2}, \quad f^-=\sup(-f,0)=\frac{\abs{f}-f}{2}.
  \end{equation}
\end{defdef}
\begin{prop}[Propriétés]
  \begin{itemize}
    \item \(f^+\) et \(f^-\) sont à valeurs positives ou nulles;
    \item \(\abs{f}=f^++f^-\);
    \item \(f=f^+-f^-\);
    \item \(-f^-=\inf(f,0)\);
    \item \(-f^+=\inf(-f,0)\).
  \end{itemize}
\end{prop}

\subsection{Fonctions majorées, minorées et bornées}

\subsubsection{Définitions}
\begin{defdef}
  Soit une fonction \(f \in \R^X\), on dit que
  \begin{itemize}
    \item \(f\) est majorée, s'il existe un réel \(M\) tel que pour tout \(x \in
      X\) \(f(x) \leqslant M\);
    \item \(f\) est minorée, s'il existe un réel \(m\) tel que pour tout \(x \in
      X\) \(f(x) \geqslant m\);
    \item \(f\) est bornée si elle est majorée et minorée.
  \end{itemize}
\end{defdef}
\begin{prop}
  Soit une fonction \(f\) de \(X\) dans \(\R\), alors \(f\) est bornée si et
  seulement s'il existe un réel \(M\) positif tel que pour tout réel \(x \in
  X\), \(\abs{f(x)} \leqslant M\).
\end{prop}
\begin{proof}
  Voir celle du chapitre~\ref{chap:suites}.
\end{proof}
\begin{defdef}
  On appelle \(\Born(X,\R)\) l'ensemble des applications bornées de \(X\) dans
  \(\R\).
\end{defdef}

\subsubsection{Borne supérieure d'une fonction majorée}
Soit une fonction \(f \in \R^X\), alors \(f\) est majorée si et seulement si la
partie \(f(X)\) est majorée dans \(\R\). Puisque \(X\) est non vide alors
\(f(X)\) est non vide et par conséquent \(f(X)\) admet une borne supérieure dans
\(\R\) et on la note \(\sup\limits_{X}f\) ou \(\sup\limits_{x \in X}f(x)\).

\begin{defdef}
  Soit une fonction \(f \in \R^X\) et \(a\) un réel de \(X\).
  \begin{itemize}
    \item On dit que \(f\) admet un maximum en \(a\) si pour tout \(x \in X\),
      \(f(x) \leqslant f(a)\);
    \item On dit que \(f\) admet un maximum local en \(a\) s'il existe un réel
      \(h >0\) tel que pour tout réel \(x \in X\), \(\abs{x-a} \leqslant h
      \implies f(x) \leqslant f(a)\);
    \item On dit que \(f\) admet un minimum en \(a\) si pour tout \(x \in X\),
      \(f(x) \geqslant f(a)\);
    \item On dit que \(f\) admet un minimum local en \(a\) s'il existe un réel
      \(h >0\) tel que pour tout réel \(x \in X\), \(\abs{x-a} \leqslant h
      \implies f(x) \geqslant f(a)\).
  \end{itemize}
\end{defdef}

Soit \(f \in \R^X\), si elle est majorée alors \(\sup\limits_{X}f\) existe mais
a priori \(\sup\limits_{X}f \notin f(X)\) (une borne supérieure n'est en général
pas atteinte). Cependant si \(\sup\limits_{X}f \in f(X)\) alors \(f\) admet un
maximum et c'est la borne supérieure. Mais si \(f\) admet une borne supérieure
ce n'est pas forcément le maximum.

\subsubsection{Propriétés de de la borne supérieure}
\begin{prop}
  Soient deux fonctions \(f,g\) de \(\R^X\). Si \(f\) et \(g\) sont majorées,
  alors \(f+g\) est majorée et de plus \begin{equation}
    \sup\limits_{X}(f+g) \leqslant \sup\limits_{X}f + \sup\limits_{X}g.
  \end{equation}
\end{prop}
En général, ce n'est pas une égalité. Il suffit de prendre
\(X=\intervalleff{0}{1}\) et \(f=\Id\) et \(g=-\Id\) pour voir que
\(\sup\limits_{X}(f+g) \neq \sup\limits_{X}f + \sup\limits_{X}g\)
(\(\sup\limits_{X}f=1\), \(\sup\limits_{X}g=0\) et \(\sup\limits_{X}(f+g)=0\)).
\begin{proof}
  Comme \(f\) et \(g\) sont majorées, alors leurs bornes supérieures existent.
  Soit \(x \in X\) alors
  \begin{equation}
    f(x)+g(x) \leqslant \sup\limits_{X}f + \sup\limits_{X}g.
  \end{equation}
  Donc \(\sup\limits_{X}f + \sup\limits_{X}g\) est un majorant de \(f+g\), donc
  \(f+g\) admet une borne supérieure. Par définition, comme c'est le plus petit
  des majorants, en passant à la borne supérieure dans l'inégalité on a bien le
  résultat
  \begin{equation}
    \sup\limits_{X}(f+g) \leqslant \sup\limits_{X}f + \sup\limits_{X}g.
  \end{equation}
\end{proof}
\begin{prop}
  Soit une fonction \(f \in \R^X\) et \(\lambda \in \Rplus\). On suppose que
  \(f\) est majorée, alors \(\lambda f\) est aussi majorée et
  \begin{equation}
    \sup\limits_{X}(\lambda f)= \lambda \sup\limits_{X}f.
  \end{equation}
\end{prop}
\begin{proof}
  \(f\) est majorée, donc elle admet une borne supérieure et puisque \(\lambda
  \geqslant 0\) alors, on a
  \begin{equation}
    \lambda f \leqslant \lambda \sup\limits_{X}f.
  \end{equation}
  Alors \(\sup\limits_{X}f\) est un majorant de la partie \(\enstq{\lambda
  f(x)}{x \in X}\). La fonction \(\lambda f\) est majorée donc
  \(\sup\limits_{X}\lambda f\) existe. Puisque c'est le plus petit des majorant
  de la partie \(\enstq{\lambda f(x)}{x \in X}\), on passe à la borne supérieure
  dans l'inégalité
  \begin{equation}
    \label{eq:inégalité2}
    \sup\limits_{X}\lambda f \leqslant \lambda \sup\limits_{X}f.
  \end{equation}
  Deux cas de figures se présentent~:
  \begin{itemize}
    \item si \(\lambda =0\) alors \(\lambda f\) est la fonction nulle donc il y
      a égalité.
    \item sinon, on a \(\lambda > 0\) et
      \begin{equation}
        \lambda \sup\limits_{X} f= \lambda \sup\limits_{X} \frac{1}{\lambda}
        \lambda f.
      \end{equation}
      Or d'après la première partie de la preuve, en notant \(g=\lambda f\) et
      en appliquant l'équation.~\eqref{eq:inégalité2} à la fonction \(g\) et au
      réel \(\frac{1}{\lambda}\), on a
      \begin{equation}
        \label{eq:inegalite1}
        \sup\limits_{X}\frac{1}{\lambda} g \leqslant \frac{1}{\lambda}
        \sup\limits_{X}g.
      \end{equation}
      %on sait que \(\sup\limits_{X}\frac{1}{\lambda} \lambda f \leqslant
      %\frac{1}{\lambda} \sup\limits_{X}\lambda f\).
      Or \(\dfrac{1}{\lambda}g=f\) et en multipliant par \(\lambda > 0\) de
      chaque coté de l'équation.~\eqref{eq:inegalite1}, on obtient
      \begin{equation}
        \lambda \sup\limits_{X} f \leqslant \sup\limits_{X}\lambda f.
      \end{equation}
      D'où l'égalité par double inégalité.
  \end{itemize}
\end{proof}
\begin{prop}
  Soient \(f\) et \(g\) deux applications de \(X\) dans \(\R\). On suppose que
  \(f \geqslant \tilde{0}\), \(g \geqslant \tilde{0}\) et que \(f\) et \(g\)
  sont majorées. Alors la fonction \(fg\) est majorée et
  \begin{equation}
    \sup\limits_{X}(fg) \leqslant (\sup\limits_{X}f)(\sup\limits_{X}g).
  \end{equation}
  Ce n'est en général pas une égalité.
\end{prop}
\begin{proof}
  Les fonctions \(f\) et \(g\) sont majorées donc \(\sup\limits_{X}f\) et
  \(\sup\limits_{X}g\) existent. Soit \(x \in X\) alors
  \begin{gather}
    0 \leqslant f(x) \leqslant \sup\limits_{X}f, \\
    0 \leqslant g(x) \leqslant \sup\limits_{X}g.
  \end{gather}
  Donc en multipliant les inégalités
  \begin{equation}
    0 \leqslant f(x)g(x) \leqslant (\sup\limits_{X}f)(\sup\limits_{X}g).
  \end{equation}
  Alors \((\sup\limits_{X}f)(\sup\limits_{X}g)\) est un majorant de \(fg\), donc
  la fonction \(fg\) est majorée, elle admet donc une borne supérieure et de
  plus
  \begin{equation}
    \sup\limits_{X} fg \leqslant (\sup\limits_{X}f)(\sup\limits_{X}g).
  \end{equation}
\end{proof}

\subsubsection{Définition et propriétés de la borne inférieure}
\begin{defdef}
  Soit une fonction \(f \in \R^X\) est minorée, alors la partie \(f(X)\) est
  minorée et non vide. Donc celle-ci admet une borne inférieure notée
  \(\inf\limits_{X} f\) ou \(\inf\limits_{x \in X} f(x)\).
\end{defdef}
\begin{lemme}
  Soit une fonction \(f \in \R^X\). Alors \(f\) est minorée si et seulement si
  \(-f\) est majorée. Auquel cas \(\inf\limits_{X} f = -\sup\limits_{X} (-f)\)
\end{lemme}
\begin{proof}
  En effet, puisque
  \begin{align}
    f \text{~est minorée} &\iff \exists m \in \R \ \forall x \in X \ m \leqslant
    f(x)\\
    &\iff \exists m \in \R \ \forall x \in X \ -f(x) \leqslant -m\\
    &\iff \exists M \in \R \ \forall x \in X \ -f(x) \leqslant M\\
    &\iff -f \text{~est majorée}.
  \end{align}
  La fonction \(-f\) est majorée, donc \(\sup\limits_{X} (-f)\) existe et
  \begin{equation}
    \forall x \in X \quad -f(x) \leqslant \sup\limits_{X}(-f),
  \end{equation}
  et donc
  \begin{equation}
    \forall x \in X \quad f(x) \geqslant -\sup\limits_{X}(-f).
  \end{equation}
  Du coup \(-\sup\limits_{X}(-f)\) est un minorant de la partie \(f(X)\). Soit
  \(m\) un minorant de \(f(X)\), alors \begin{equation}
    \forall x \in X \quad f(x) \geqslant m,
  \end{equation}
  donc \begin{equation}
    \forall x \in X \quad -f(x) \leqslant -m.
  \end{equation}
  Alors \(-m\) est majorant de la partie \(\enstq{-f(x)}{x \in X}\). Donc \(-m
  \geqslant \sup\limits_{X}(-f)\) (puisque \(\sup\limits_{X}(-f)\) est le plus
  petit des majorants de \(\enstq{-f(x)}{x \in X}\)). Ainsi \(m \leqslant
  -\sup\limits_{X}(-f)\) et \(-\sup\limits_{X}(-f)\) est donc le plus grand des
  minorants de \(\enstq{f(x)}{x \in X}\).
  \begin{equation}
    \inf\limits_{X} f = -\sup\limits_{X} (-f).
  \end{equation}
\end{proof}
Grâce à ce lemme, on peut déduire trois propositions~:
\begin{prop}
  Soient \(f\) et \(g\) deux applications de \(X\) dans \(\R\). Si \(f\) et
  \(g\) sont minorée, alors \(f+g\) l'est aussi et de plus
  \begin{equation}
    \inf\limits_{X}(f+g) \geqslant \inf\limits_{X} f + \inf\limits_{X} g.
  \end{equation}
\end{prop}
\begin{proof}
  Les applications \(f\) est \(g\) sont minorées, donc d'après le lemme \(-f\)
  et \(-g\) sont majorées. Alors d'après la propriété sur la borne supérieure
  \(-f-g\) est majorée. Encore d'après le lemme l'application \(f+g\) est
  minorée et
  \begin{equation}
    \inf\limits_{X}(f+g)=-\sup\limits_{X}(-f-g).
  \end{equation}
  D'après les propriétés de la borne supérieure
  \begin{equation}
    \sup\limits_{X}(-f-g) \leqslant \sup\limits_{X}(-f) + \sup\limits_{X}(-g).
  \end{equation}
  Alors en inversant l'inégalité on obtient
  \begin{equation}
    \inf\limits_{X}(f+g) = - \sup\limits_{X}(-f-g) \geqslant
    -\sup\limits_{X}(-f) - \sup\limits_{X}(-g) = \inf\limits_{X} f +
    \inf\limits_{X} g.
  \end{equation}
\end{proof}
\begin{prop}
  Soient \(f \in \R^X\) et un réel positif \(\lambda\). Si \(f\) est minorée,
  alors \(\lambda f\) est minoré et de pus
  \begin{equation}
    \inf\limits_{X} (\lambda f) = \lambda \inf\limits_{X} f.
  \end{equation}
\end{prop}
\begin{proof}
  Si \(f\) est minorée, alors \(-f\) est majorée, comme \(\lambda \geqslant 0\)
  alors \(-\lambda f\) est majorée et sa borne supérieure existe et de plus
  \(\sup\limits_{X}(\-\lambda f)=\lambda \sup\limits_{X}(-f)\). Alors \(\lambda
  f\) est minorée et \begin{equation}
    \inf\limits_{X}(\lambda f) = -\sup\limits_{X}(-\lambda
    f)=-\lambda\sup\limits_{X}(-f)= \lambda\inf\limits_{X} f
  \end{equation}
\end{proof}
\begin{prop}
  Soient \(f\) et \(g\) deux fonctions de \(X\) dans \(\R\). On suppose que
  \(f\) et \(g\) sont minorées et à valeurs négatives, alors \(fg\) est majorée
  et
  \begin{equation}
    \sup\limits_{X} fg \leqslant \inf\limits_{X} f \inf\limits_{X} g.
  \end{equation}
\end{prop}
\begin{proof}
  \(f\) et \(g\) sont minorées, donc leurs bornes inférieures existent. Alors
  \(-f\) et \(-g\) sont positives et majorées. D'après les propriétés sur la
  borne supérieure \(fg=(-f)(-g)\) est majorée et de plus
  \begin{equation}
    \sup\limits_{X}[(-f)(-g)]\leqslant \sup\limits_{X}(-f)\sup\limits_{X}(-g),
  \end{equation}
  donc \begin{equation}
    \sup\limits_{X} fg\leqslant (-\inf\limits_{X} f)(-\inf\limits_{X} g).
  \end{equation}
  Finalement
  \begin{equation}
    \sup\limits_{X} fg\leqslant \inf\limits_{X} f \inf\limits_{X} g.
  \end{equation}
\end{proof}

\subsubsection{Ensemble des fonctions bornées}
\begin{theo}
  L'ensemble \(\Born(X,\R)\) est un sous-espace vectoriel de \(\R^X\).
  c'est-à-dire qu'il est non vide et stable par combinaison linéaire.
\end{theo}
On peut aussi dire que c'est une sous-algèbre puisque \(\tilde{0}\) et
\(\tilde{1}\) sont bornées et qu'il est stable par produit.
\begin{defdef}[Norme infinie]
  Pour toute fonction \(f \in \Born(X,\R)\), \(\abs{f}\) est majorée donc sa
  borne supérieure existe. On appelle cette borne supérieure la norme infinie ou
  la norme de la convergence uniforme
  \begin{equation}
    \norme{f}_\infty=\sup\limits_{X}\abs{f}.
  \end{equation}
\end{defdef}
\begin{theo}
  L'application norme infinie
  \(\fonction{\norme{{.}}_\infty}{\Born(X,\R)}{\R}{f}{\norme{f}_\infty}\) est
  une vraie norme, c'est-à-dire qu'elle vérifie pour toutes fonctions bornées
  \(f\) et \(g\) et tout réel \(\lambda\)~:
  \begin{align}
    \norme{f}_\infty \geqslant 0; \\
    \norme{f}_\infty = 0 \iff f=\tilde{0};\\
    \norme{f+g}_\infty \leqslant \norme{f}_\infty + \norme{g}_\infty;\\
    \norme{\lambda f}_\infty = \abs{\lambda} \norme{f}_\infty.
  \end{align}
\end{theo}
\begin{proof}
  \begin{itemize}
    \item \(\abs{f} \geqslant \tilde{0}\) donc en passant à la borne supérieure
      \(\norme{f}_\infty \geqslant 0\);
    \item \(\norme{f}_\infty = 0 \iff \sup\limits_{X}\abs{f}=0\) c'est-à-dire si
      et seulement si \(\abs{f}=\tilde{0}\) donc si et seulement si
      \(f=\tilde{0}\);
    \item C'est une conséquence des propriétés de la borne supérieure.
  \end{itemize}
\end{proof}

\subsection{Fonctions paires et fonctions impaires}

\begin{defdef}
  Soit une fonction \(f\) de \(\R^X\). On suppose que pour tout \(x \in X\),
  \(-x \in X\). La fonction \(f\) est dite paire si et seulement si pour tout
  \(x \in X\) \(f(-x)=f(x)\) et elle est dite impaire si et seulement si  pour
  tout \(x \in X\) \(f(-x)=-f(x)\). On note \(\P(X,\R)\) l'ensemble des
  fonctions paires de \(X\) dans \(\R\) et \(\I(X,\R)\) l'ensemble des fonctions
  impaires de \(X\) dans \(\R\).
\end{defdef}

Si une fonction \(f \in \P(X,\R)\) alors son graphe est symétrique par rapport à
l'axe des ordonnées. Si une fonction \(f \in \I(X,\R)\) alors son graphe est
symétrique par rapport à l'origine. Si \(0 \in X\) est que \(f\) est impaire,
alors \(f(0)=0\). Cependant, zéro peut ne pas être dans \(X\). Néanmoins si
\(f\) est définie sur un intervalle I non vide symétrique par rapport à \(0\)
alors \(0 \in I\).
\begin{prop}
  Soit \(X\) une partie de \(\R\) symétrique par rapport à zéro. \(\P(X,\R)\) et
  \(\I(X,\R,)\) sont des sous-espaces vectoriels supplémentaires de \(\R^X\).
  C'est-à-dire qu'ils sont stables par combinaisons linéaires et que pour toute
  fonction \(f \in \R^X\) il existe un unique couple \((\varphi, \psi)\) de
  fonction de \(\R^X\) tel que : \(f= \varphi+\psi\), \(\varphi\) est paire et
  \(\psi\) est impaire.
\end{prop}
\begin{proof}[Analyse \& unicité]
  On suppose qu'il existe un tel couple \((\varphi, \psi)\) de \(\R^X\), alors
  pour tout réel \(x \in X\) on a
  \begin{equation}
    \forall x \in X \quad f(x)= \varphi(x) + \psi(x),
  \end{equation}
  et aussi
  \begin{equation}
    \forall x \in X \quad f(-x)= \varphi(x) - \psi(x).
  \end{equation}
  Alors
  \begin{equation}
    \varphi(x)=\frac{f(x)+f(-x)}{2} \quad \psi(x)=\frac{f(x)-f(-x)}{2}.
  \end{equation}
  On a prouvé l'unicité du couple. Si la solution existe elle est unique et le
  couple est donné comme étant les fonction ci-dessus.
\end{proof}
\begin{proof}[Synthèse \& existence]
  On définit les fonctions suivantes
  \begin{gather}
    \fonction{\varphi}{X}{\R}{x}{\frac{f(x)+f(-x)}{2}}, \\
    \fonction{\psi}{X}{\R}{x}{\frac{f(x)-f(-x)}{2}}.
  \end{gather}
  Montrons que le couple est solution du problème. Soit un réel \(x \in X\),
  alors
  \begin{gather}
    \varphi(x)+\psi(x)=\frac{f(x)+f(-x)}{2} + \frac{f(x)-f(-x)}{2} =f(x),\\
    \varphi(-x)=\frac{f(-x)+f(-(-x))}{2}=\varphi(x),\\
    \psi(-x)=\frac{f(-x)-f(-(-x))}{2}=-\psi(x).
  \end{gather}
  Alors on a montré qu'il existe une fonction \(\varphi\) paire et un fonction
  \(\psi\) impaire telle que \(f=\varphi+\psi\).
\end{proof}

\subsection{Fonctions périodiques}

\begin{defdef}
Soit \(T\) un réel non nul et \(f \in \R^\R\). On dit que \(T\) est une période
de \(f\), ou que \(f\) est \(T\)-périodique si pour tout réel \(x \in \R\),
\(f(x+T)=f(x)\). \end{defdef}
On n'a pas forcément besoin que \(f\) soit définie sur \(\R\) mais sinon il faut
préciser que pour tout réel \(x \in X\), \(x+T\in X\) comme par exemple la
fonction tangente.
%
\begin{prop}
  Soit un réel \(T\) non nul. L'ensemble des applications périodiques de \(\R\)
  dans \(\R\) est un sous-espace vectoriel de \(\R^\R\). C'est même une
  sous-algèbre.
\end{prop}
%
\section{Limites et continuité en un point}

\subsection{Fonctions définies sur un voisinage}

Soit un réel \(a\).

\begin{defdef}
  Soit une fonction \(f:\Dr_{f} \longrightarrow \R\) avec \(\Def{f} \subset
  \R\). On dit que \(f\) est définie au voisinage de \(a\) s'il existe un réel
  \(h > 0\) tel que
  \begin{itemize}
    \item soit \(\intervalleff{a-h}{a+h}\setminus\{a\} \subset \Def{f}\);
    \item soit \(\intervalleoo{a-h}{a} \subset \Def{f}\);  \item ou soit
      \(\intervalleoo{a}{a+h} \subset \Def{f}\).
  \end{itemize}
\end{defdef}
Par exemple la fonction \(x \longmapsto \frac{\ln x}{x}\) est définie au
voisinage de zéro.
\begin{defdef}
  Soit une fonction \(f:\Def{f} \longrightarrow \R\) avec \(\Def{f} \subset
  \R\). On dit que \(f\) est définie au voisinage de \(+\infty\) s'il existe un
  réel \(A\) tel que \(\intervallefo{A}{+\infty} \subset \Def{f}\). De la même
  manière, on dit que \(f\) est définie au voisinage de \(-\infty\) s'il existe
  un réel \(B\) tel que \(\intervalleof{-\infty}{B} \subset \Dr_{f}\).
\end{defdef}
\begin{defdef}
  Soit \(b \in \Rbar\) et \(f : \Def{f} \longrightarrow \R\). On dit qu'une
  propriété \(P\) portant sur \(f\) est vraie au voisinage de \(b\), si \(f\)
  est définie au voisinage de \(b\) et si
  \begin{itemize}
    \item \(b \in \R\), il existe un réel \(h\) tel que \(P\) soit vraie sur
      \(\intervalleff{a-h}{a+h} \cap \Def{f}\);
    \item \(b=+\infty\), il existe un réel \(A\) tel que \(P\) soit vraie sur
      \(\intervallefo{A}{+\infty} \cap \Def{f}\);
    \item \(b=+\infty\), il existe un réel \(A\) tel que \(P\) soit vraie sur
      \(\intervalleof{-\infty}{A} \cap \Def{f}\).
  \end{itemize}
\end{defdef}

\subsection{Limite en un point d'une fonction}

\subsubsection{Fonctions de limite nulle}

\begin{defdef}
  \begin{itemize}
    \item Soit un réel \(a\) et une fonction \(f:\Def{f} \longrightarrow \R\),
      on dit que \(f\) tend vers zéro en \(a\) si \(f\) est définie au voisinage
      de \(a\) et si
      \begin{equation}
        \forall \epsilon > 0 \ \exists \eta > 0 \ \forall x \in \Def{f} \quad
        \abs{x-a} \leqslant \eta \implies \abs{f(x)} \leqslant \epsilon ;
      \end{equation}
    \item on dit que \(f\) tend vers zéro en \(+\infty\) si \(f\) est définie au
      voisinage de \(+\infty\) et si
      \begin{equation}
        \forall \epsilon > 0 \ \exists A \in \R \ \forall x \in \Def{f} \quad x
        \geqslant A \implies \abs{f(x)} \leqslant \epsilon ;
      \end{equation}
    \item on dit que \(f\) tend vers zéro en \(-\infty\) si \(f\) est définie au
      voisinage de \(+\infty\) et si
      \begin{equation}
        \forall \epsilon > 0 \ \exists A \in \R \ \forall x \in \Def{f} \quad x
        \leqslant A \implies \abs{f(x)} \leqslant \epsilon.
      \end{equation}
  \end{itemize}
\end{defdef}

\subsubsection{Fonctions admettant une limite finie}

\begin{defdef}
  Soient \(a \in \Rbar\) et \(\ell \in \R\). On dit que \(f\) tend vers \(\ell\)
  en \(a\) si la fonction \(f-\ell\) tend vers zéro en \(a\).
\end{defdef}
\begin{theo}[Unicité de la limite]
  Soit \(a \in \Rbar\) et \(f\) une application définie au voisinage de \(a\).
  Il existe au plus un réel \(\ell\) tel que \(f\) tende vers \(\ell\) en \(a\).
  S'il existe on dit que \(\ell\) est la limite de la fonction \(f\) en \(a\).
  On note
  \begin{equation}
    \lim\limits_{x \to a}f(x)=\ell \qquad \lim\limits_{a}f=\ell \qquad f(x)
    \underset{x \to a}{\longrightarrow} \ell.
  \end{equation}
\end{theo}
\begin{proof}
  Pour \(a \in \R\). On suppose qu'il existe \(\ell\) et \(\ell'\) deux réels
  différents tels que \(f-\ell\) et \(f-\ell'\) tendent vers zéro en \(a\). On
  va chercher une absurdité pour montrer que \(\ell=\ell'\). Soit \(\epsilon =
  \frac{\abs{\ell-\ell'}}{3} > 0\) puisque \(\ell \neq \ell'\). Il existe donc
  deux réels \(\eta_1\) et \(\eta_2\) strictement positifs tels que pour tout
  \(x \in \Def{f}\) on ait
  \begin{gather}
    \abs{x-a} \leqslant \eta_1 \implies \abs{f(x)-\ell} \leqslant \epsilon, \\
    \abs{x-a} \leqslant \eta_2 \implies \abs{f(x)-\ell'} \leqslant \epsilon.
  \end{gather}
  Soit \(\eta=\min(\eta_1, \eta_2)\), alors pour tout \(x \in \Def{f}\) si
  \(\abs{x-a} \leqslant \eta\) alors par inégalité triangulaire
  \begin{equation}
    \abs{\ell-\ell'} \leqslant \abs{f(x)-\ell} + \abs{f(x)-\ell'} \leqslant
    \frac{2}{3}\abs{\ell-\ell'}.
  \end{equation}
  Car \(\abs{\ell-\ell'} >0\), on peut simplifier l'inégalité et \(1 \leqslant
  \frac{2}{3}\), ce qui est absurde. Donc \(\ell=\ell'\).
\end{proof}

\subsubsection{Fonction admettant une limite infinie}

\begin{defdef}
  Soit \(a \in \Rbar\) et \(f\) une application définie au voisinage de \(a\).
  On dit que \(f\) tend vers \(+\infty\) en \(a\) si
  \begin{itemize}
    \item si \(a \in \R\)
      \begin{equation}
        \forall A \in \R \ \exists \eta > 0 \ \forall x \in \Def{f} \quad
        \abs{x-a} \leqslant \eta \implies f(x) \geqslant A;
      \end{equation}
    \item si \(a= + \infty\)
      \begin{equation}
        \forall A \in \R \ \exists B \in \R \ \forall x \in \Def{f} \quad x
        \geqslant B \implies f(x) \geqslant A;
      \end{equation}
    \item si \(a= - \infty\)
      \begin{equation}
        \forall A \in \R \ \exists B \in \R \ \forall x \in \Def{f} \quad x
        \leqslant B \implies f(x) \geqslant A.
      \end{equation}
  \end{itemize}
  On dit aussi que \(f\) admet \(+\infty\) pour limite en \(a\) et on note
  \begin{equation}
    \lim\limits_{x \to a}f(x)=+\infty \qquad \lim\limits_{a}f=+\infty \qquad
    f(x) \underset{x \to a}{\longrightarrow} +\infty.
  \end{equation}
\end{defdef}
\begin{defdef}
  Soit \(a \in \Rbar\) et \(f\) une application définie au voisinage de \(a\).
  On dit que \(f\) tend vers \(-\infty\) en \(a\) si \(-f\) tend vers
  \(+\infty\) en \(a\). C'est-à-dire
  \begin{itemize}
    \item si \(a \in \R\)
      \begin{equation}
        \forall A \in \R \ \exists \eta > 0 \ \forall x \in \Def{f} \quad
        \abs{x-a} \leqslant \eta \implies f(x) \leqslant A;
      \end{equation}
    \item si \(a= + \infty\)
      \begin{equation}
        \forall A \in \R \ \exists B \in \R \ \forall x \in \Def{f} \quad x
        \geqslant B \implies f(x) \leqslant A;
      \end{equation}
    \item si \(a= - \infty\)
      \begin{equation}
        \forall A \in \R \ \exists B \in \R \ \forall x \in \Def{f} \quad x
        \leqslant B \implies f(x) \leqslant A.
      \end{equation}
  \end{itemize}
  On dit aussi que \(f\) admet \(-\infty\) pour limite en \(a\) et on note
  \begin{equation}
    \lim\limits_{x \to a}f(x)=-\infty \qquad \lim\limits_{a}f=-\infty \qquad
    f(x) \underset{x \to a}{\longrightarrow} -\infty.
  \end{equation}
\end{defdef}
\begin{theo}
  \begin{equation}
    \lim\limits_{x \to +\infty} \ln x = + \infty.
  \end{equation}
\end{theo}
\begin{proof}
  Soit \(A \in \R\). Comme \(\R\) est archimédien, il existe un entier naturel
  \(n\) tel que \(n \ln 2 \geqslant A\). Soit \(B=2^n\), alors
  \begin{equation}
    \forall x \geqslant B \quad \ln x \geqslant \ln B,
  \end{equation}
  car le logarithme népérien est croissant. Alors \(\ln x \geqslant \ln B
  \geqslant A\) donc \(\ln x \geqslant A\). On a montré
  \begin{equation}
    \forall A \in \R \ \exists B \in \R \ \forall x \geqslant B \quad \ln x
    \geqslant A,
  \end{equation}
  donc \(\lim\limits_{x \to +\infty} \ln x = + \infty\).
\end{proof}

\subsubsection{Restriction --- Limites à gauche --- Limites à droite}
Soit un intervalle réel \(I\).
\begin{prop}
  Soit un réel \(a\) et une fonction \(f \in \R^I\). Soit un intervalle non vide
  \(I' \subset I\). On suppose que \(f_{|I^{'}}\) est définie au voisinage de
  \(a\). Alors
  \begin{equation}
    \lim\limits_{\begin{array}{l} x \to a \\ x \in
      I\end{array}}f(x)=\lim\limits_{a}f=\ell \implies
      \lim\limits_{\begin{array}{l} x \to a \\ x \in
      I'\end{array}}f(x)=\lim\limits_{a}f_{|I^{'}}=\ell
  \end{equation}
  La réciproque est fausse. Par exemple, on prend \(I=[-1,1]\), \(I'=[0,1]\),
  \(a=0\) et
  \begin{equation}
    \fonction{f}{I}{\R}{x}{%
      \begin{cases}
          0 & -1 \leqslant x < 0 \\
          1 & 0 \leqslant x \leqslant 1
      \end{cases}
    },
  \end{equation}
  alors \(\lim\limits_{x \to 0}f_{|I^{'}}(x)=1\) cependant \(f\) n'admet pas de
  limite en zéro.
\end{prop}
\begin{proof}
  On a \( I' \subset I\), par exemple si \(\lim\limits_{a} f =\ell\) alors
  \begin{equation}
    \forall \epsilon > 0 \ \exists \eta > 0 \ \forall x \in I' \subset I \quad
    \abs{x-a} \leqslant \eta \implies \abs{f(x) - \ell} \leqslant \epsilon.
  \end{equation}
\end{proof}

\emph{Limites à droite et limite à gauche}~:

Soit \(f \in \R^I\), \(a\in \R\) et \(I'=I \cap ]a, +\infty[ \neq \emptyset\).
Alors \(f\) est définie au voisinage de \(a\). On considère la restriction de
\(f\) à \(I'\). On dit que \(f\) admet une limite à droite en \(a\) si
\(f_{|I^{'}}\) admet une limite en \(a\). On note sous réserve d'existence
\begin{equation}
  \lim\limits_{\begin{array}{l} x \to a \\ x \in
    I\end{array}}f(x)=\lim\limits_{\begin{array}{c} x \to a \\
  >\end{array}}f(x)=\lim\limits_{x \to a^{+}}f(x)=\lim\limits_{a^{+}}f
\end{equation}
\begin{defdef}[Écriture quantifiée de la limite à droite]
  Soit \(f \in \R^I\), \(a\in \R\) et \(l \in \R\). La fonction \(f\) admet pour
  limite à droite \(\ell\) en a si et seulement si
  \begin{equation}
    \forall \epsilon > 0 \ \exists \eta > 0 \ \forall x \in \Def{f} \quad
    a<x\leqslant a+\eta \implies \abs{f(x)-\ell} \leqslant \epsilon
  \end{equation}
\end{defdef}

\begin{prop}
  Si \(f\) tend vers \(\ell \in \Rbar\), en \(a\) alors
  \(\lim\limits_{a^{+}}f=\ell\). La réciproque est fausse.
\end{prop}
\begin{proof}
  C'est une conséquence du résultat sur les restrictions et de la définition de
  la limite à droite.
\end{proof}

On définit de la même manière la limite à gauche et les propriétés sont
identiques.

\begin{prop}
  Soient \(f \in \R^I\), \(a \in \R\) et \(\ell \in \Rbar\). La fonction \(f\)
  admet \(\ell\) pour limite à gauche et limite à droite en \(a\) si et
  seulement si \(\lim\limits_{\begin{array}{c} x \to a \\ x \in
  I\setminus\{a\}\end{array}}f(x)=\ell\).
\end{prop}
\begin{proof}
  \begin{itemize}
    \item[\(\impliedby\)] C'est la conséquence de la proposition sur les
      restrictions.
    \item[\(\implies\)] On peut par exemple le montrer avec \(\ell=+\infty\). On
      a la limite à droite \(\lim\limits_{a^{+}}f=\ell\), alors
      \begin{equation}
        \forall A \in \R \ \exists \eta > 0 \ \forall x \in I \quad a < x < a +
        \eta \implies f(x) \geqslant A
      \end{equation}
      et la limite à gauche \(\lim\limits_{a^{-}}f=\ell\)
      \begin{equation}
        \forall A \in \R \ \exists \alpha > 0 \ \forall x \in I \quad a-\alpha
        \leqslant x < a
      \end{equation}
      Soit \(\beta=\min(\alpha, \eta)\). Pour tout \(x \in I\setminus\{a\}\)
      \begin{equation}
        \abs{x-a} \leqslant \beta \implies a-\beta \leqslant x \leqslant
        a+\beta,
      \end{equation}
      alors deux cas de figure se présentent~:
      \begin{itemize}
        \item soit \(a < x \leqslant a+\beta \leqslant a + \eta\) et alors
          \(f(x) \geqslant A\);
        \item soit \(\alpha - \beta \leqslant x < a\) et comme \(\beta \leqslant
          \alpha\) alors on a \(a-\alpha \leqslant x < a\) donc \(f(x) \geqslant
          A\).
      \end{itemize}
      Pour tout \(x \in I\setminus\{a\}\) on a bien \(\abs{x-a}\leqslant \beta
      \implies f(x) \geqslant A\) donc \(\lim\limits_{\begin{array}{l} x \to a
      \\ x \in I\setminus\{a\}\end{array}}f(x)=+\infty\).
  \end{itemize}
\end{proof}

L'égalité des limites à gauche et à droite à \(\ell\) n'est pas équivalente à
\(\lim\limits_{a}f=\ell\). En effet, si on considère la fonction définie sur
\(\intervalleff{-1}{1}\) qui est nulle partout sauf en \(0\) où elle vaut \(1\).
Alors les limites à droite et à gauche sont nulles pourtant la limite de \(f\)
en zéro n'existe pas. Si \(f\) tend vers \(\ell\) en zéro, alors les limites à
gauche et à droite sont égales à \(\ell=0\). En effet, si \(f\) était de limite
nulle en zéro on aurait
\begin{equation}
  \forall \epsilon > 0 \ \exists \eta > 0 \ \forall x \in \intervalleff{-1}{1}
  \quad \abs{x-0} \leqslant \eta \implies \abs{f(x)-0} \leqslant \epsilon.
\end{equation}
Avec \(\epsilon=\frac{1}{2}\), on écrit
\begin{equation}
  \exists \eta > 0 \ \forall x \in \intervalleff{-1}{1} \quad \abs{x}\leqslant
  \eta \implies \abs{f(x)} \leqslant \frac{1}{2}.
\end{equation}
Or \(0 \in \intervalleff{-1}{1}\) et \(\abs{0}\leqslant \eta\) donc
\(\abs{f(0)}=1 \leqslant \frac{1}{2}\), ce qui est absurde. Donc \(f\) n'est pas
continue en zéro.

\subsubsection{Propriétés des limites}

Soient \(f\) une fonction définie au voisinage de \(a \in \Rbar\) et \(\ell \in
\R\).
\begin{prop}
  \begin{gather}
    \lim\limits_{a} f = \ell \iff \lim\limits_{a} \abs{f-\ell} = 0 \\
    \lim\limits_{a} f = \ell \iff \lim\limits_{a} \abs{f}=\abs{\ell}.
  \end{gather}
\end{prop}
\begin{proof}
  Voir la démonstration équivalente du chapitre~\ref{chap:suites} relative aux
  suites.
\end{proof}
\begin{prop}
  Si la fonction \(f\) admet une limite finie en \(a \in \Rbar\), alors \(f\)
  est bornée au voisinage de \(a\).
\end{prop}
\begin{proof}
  Par exemple, si on prend \(a=-\infty\) alors il existe \(l \in \R\) tel que
  \(\lim\limits_{-\infty} f = \ell\). C'est-à-dire
  \begin{equation}
    \forall \epsilon >0 \ \exists A \in \R \ \forall x \in \Def{f} \quad x
    \leqslant A \implies \abs{f(x)-\ell} \leqslant \epsilon.
  \end{equation}
  Avec \(\epsilon=1\), on écrit
  \begin{equation}
    \exists A \in \R \ \forall x \in \Def{f} \quad x \leqslant A \implies
    \abs{f(x)-\ell} \leqslant 1.
  \end{equation}
  On pose \(M=1+\abs{l}\), alors
  \begin{equation}
    \forall x \in \Def{f} \cap \intervalleof{-\infty}{A} \quad \abs{f(x)}
    \leqslant \abs{f(x)-\ell}+\abs{\ell} \leqslant 1+\abs{\ell}=M.
  \end{equation}
  La fonction \(f\) est donc bornée au voisinage de \(-\infty\).
\end{proof}
\begin{prop}
  Soit un réel \(m\). On suppose que \(\lim\limits_{a} f=\ell\) avec \(\ell>m\).
  Alors \(f\) est minorée par \(m\) au voisinage de \(a\).
\end{prop}
\begin{proof}
  Prenons le cas où \(a \in \R\). Alors deux cas de figures se présentent~: soit
  \(\ell \in \R\) ou soit \(\ell = + \infty\)
  \begin{itemize}
    \item Dans le cas où \(\ell \in \R\), on pose \(\epsilon = \ell-m >0\) et
      donc
      \begin{equation}
        \exists \eta > 0 \ \forall x \in \Def{f} \quad \abs{x-a} \leqslant \eta
        \implies \abs{f(x)-\ell} \leqslant \epsilon.
      \end{equation}
      Alors pour tout \(x \in \Def{f} \cap \intervalleff{a-\eta,}{a+\eta}\) on a
      \(-\epsilon \leqslant f(x)-l \leqslant \epsilon\) soit en particulier
      \(m-\ell \leqslant f(x)-\ell\), donc \(m \leqslant f(x)\). La fonction
      \(f\) est donc minorée au voisinage de \(a\).
    \item Dans le cas où \(\ell=\infty\), on écrit la définition
      \begin{equation}
        \forall m \in \R \ \exists \eta \in \R \ \forall x \in \Def{f} \quad
        \abs{x-a} \leqslant \eta \implies f(x) \geqslant m.
      \end{equation}
      Donc \(\forall x \in \Def{f} \cap \intervalleff{a-\eta}{a+\eta} \quad f(x)
      \geqslant m\). La fonction \(f\) est donc minorée au voisinage de \(a\).
  \end{itemize}
\end{proof}
\begin{cor}
  \begin{itemize}
    \item Si \(f\) admet une limite positive, finie ou infinie, en \(a \in
      \Rbar\), alors \(f\) est minorée par un réel strictement positif au
      voisinage de \(a\),
    \item si \(f\) admet une limite non nulle, alors \(\abs{f}\) est minorée par
      un réel strictement positif au voisinage de \(a\).
  \end{itemize}
\end{cor}
\begin{prop}
  Soient \(a \in \R\), \(f\in \R^X\), et \(l \in \Rbar\).
  \begin{equation}
    \lim\limits_{a}f=\ell \iff \lim\limits_{h \to 0} f(a+h)=\ell.
  \end{equation}
\end{prop}
\begin{proof}
  Soient l'ensemble \(Y=\enstq{h \in \R}{a+h \in X}\) et la fonction
  \begin{equation}
    \fonction{g}{Y}{\R}{h}{f(a+h)}.
  \end{equation}
  Alors
  \begin{equation}
    \lim\limits_{x \to a, x \in X}f(x)=\ell \iff \lim\limits_{x-a \to 0, x \in
    X}f(a+x-a)=\ell \iff \lim\limits_{h \to 0, h \in Y}g(h)=\ell.
  \end{equation}
\end{proof}

\subsection{Notion de continuité en un point}

\subsubsection{Définition de la continuité en un point}

\begin{prop}
  Soient \(f \in \R^X\) et \(a \in X\). La limite en \(a\) de \(f\) existe et
  est finie si et seulement si \(\lim\limits_{a} f = f(a)\).
\end{prop}
\begin{proof}
  \begin{itemize}
    \item[\(\impliedby\)] Évident
    \item[\(\implies\)] Soit \(l=\lim\limits_{a} f\). Supposons que \(l \neq
      f(a)\). Alors puisque \(\ell\) existe et est finie, on a
      \begin{equation}
        \forall \epsilon > 0 \ \exists \eta > 0 \ \forall x \in X \quad
        \abs{x-a} \leqslant \eta \implies \abs{f(x)-\ell} \leqslant \epsilon
      \end{equation}
      Avec \(\epsilon=\frac{\abs{f(a)-\ell}}{2}\) et avec \(a \in X\)
      (\(\abs{a-a}=0 \leqslant \eta\)) on a
      \begin{equation}
        \abs{f(a)-\ell} \leqslant \frac{\abs{f(a)-\ell}}{2},
      \end{equation}
      et comme \(\abs{f(a)-\ell} > 0\) on simplifie et on obtient \(1 \leqslant
      \frac{1}{2}\). Ce qui est absurde donc \(\ell=f(a)\).
  \end{itemize}
\end{proof}
\begin{defdef}
  Soient \(f \in \R^X\) et \(a \in X\). On dit que \(f\) est continue en \(a\)
  si et seulement si \(f\) admet une limite finie en \(a\). En vertu de la
  proposition précédente, \(f\) est continue en a si et seulement si
  \(\lim\limits_{a} f = f(a)\).
\end{defdef}

\subsubsection{Prolongement par continuité en un point}

Soient \(X\) une partie de \(\R\) et \(a\) un réel. On suppose que \(f\) est
définie sur X, qu'elle est définie au voisinage de \(a\) mais pas en \(a\), par
exemple \(X=]a,b]\), ou \(X=\R\setminus\{a\}\). On suppose que \(f\) admet une
limite finie \(\ell\) en \(a\). Le prolongement par continuité de \(f\) en \(a\)
est la fonction \(g:X\cup\{a\} \longrightarrow \R\) définie comme égale à \(f\)
sur \(X\) et égale à \(\ell\) en \(a\).

\begin{prop}
  Le prolongement par continuité de \(f\) en \(a\) est une application continue
  en \(a\) et c'est l'unique prolongement de \(f\) à \(X \cup \{a\}\) qui soit
  continu en \(a\).
\end{prop}
\begin{proof}[Preuve de la continuité]
  On sait que \(\ell=\lim\limits_{a}f\) donc
  \begin{equation}
    \forall \epsilon >0 \ \exists A \in \R \ \forall x \in X \quad x \leqslant A
    \implies \abs{f(x)-\ell} \leqslant \epsilon.
  \end{equation}
  Pour tout \(x \in X \cup \{a\}\) tel que \(\abs{x-a} \leqslant \eta\), deux
  cas de figure se présentent~:
  \begin{itemize}
    \item si \(x \in X\) alors \(\abs{g(x)-\ell}=\abs{f(x)-\ell} \leqslant
      \epsilon\),
    \item si \(x=a\) alors \(\abs{g(a)-\ell}=0 \leqslant \epsilon\).
  \end{itemize}
  Donc pour tout \(x \in X \cup \{a\}\), si \(\abs{x-a} \leqslant \eta\) alors
  \(\abs{g(x)-\ell} \leqslant \epsilon\). La fonction \(g\) admet une limite
  finie en \(a\), alors elle est continue en \(a\).
\end{proof}
\begin{proof}[Preuve de l'unicité]
  Soit \(h\) un prolongement de \(f\) à \(X \cup \{a\}\) continue en \(a\). Pour
  tout réel \(x \in X\), \(f(x)=g(x)=h(x)\).
  \begin{equation}
    h(a)= \lim\limits_{\begin{array}{c} x \to a \\ x \in
    X\cup\{a\}\end{array}}h(x).
  \end{equation}
  D'après les résultats sur les restrictions cela implique que
  \begin{equation}
    h(a)= \lim\limits_{\begin{array}{c} x \to a \\ x \in
      X\cup\{a\}\end{array}}h(x) \text{~et~} h(a)= \lim\limits_{\begin{array}{c}
    x \to a \\ x \in X\cup\{a\}\end{array}}f(x).
  \end{equation}
  On en déduit donc par unicité de la limite que \(h(a)=\ell=g(a)\). Ainsi
  \(h=g\) et \(g\) est l'unique prolongement de \(f\) en \(a\) continu.
\end{proof}

\emph{Exemples}~:
\begin{itemize}
  \item Soit la fonction \(\fonction{f}{\R^*}{\R}{x}{\frac{\sin x}{x}}\). On
    sait que \(\lim\limits_{0} f=1\). On peut prolonger \(f\) par continuité en
    zéro en prenant \(f(0)=1\) (en notant abusivement \(f\) la fonction
    prolongée alors que ces deux fonctions sont différentes). On obtient une
    fonction définie sur \(\R\) appelée sinus cardinal et notée \(\sinc\)
    couramment utilisée en physique.
  \item Soit la fonction \(\fonction{g}{\R^*}{\R}{x}{\e^{-\frac{1}{x^2}}}\). On
    sait que \(\lim\limits_{0} g=0\). On peut prolonger \(g\) par continuité en
    posant \(g(0)=0\).
  \item Soit la fonction \(\fonction{h}{\R^*}{\R}{x}{\e^{-\frac{1}{x}}}\). Cette
    fonction n'est pas prolongeable par continuité puisque
    \(\lim\limits_{0^{+}}h=0\) et \(\lim\limits_{0^{-}}h=+\infty\). Par contre
    la restriction de \(h\) à \(]0,+\infty[\) notée \(h_1\) est prolongeable par
    continuité en posant \(h_1(0)=0\).
\end{itemize}

\subsubsection{Continuité à gauche et continuité à droite}

\begin{defdef}
  Soient une fonction \(f \in \R^X\) et un réel \(a \in X\). Alors \(f\) est
  continue à droite en \(a\) si et seulement si la restriction de \(f\) à
  \(X\cap\intervallefo{a}{+\infty}\) est continue en \(a\) si et seulement si la
  restriction de \(f\) à \(X\cap \intervallefo{a}{+\infty}\) admet une limite
  finie en \(a\) si et seulement si \(\lim\limits_{x \to a^{+}}f(x) = f(a)\).
\end{defdef}
La définition est similaire pour la continuité à gauche.
\begin{prop}
  Soient \(f \in \R^X\) et un réel \(a \in X\). Alors \(f\) est continue en
  \(a\) si et seulement si \(f\) est continue à droite en \(a\) et si \(f\) est
  continue à gauche en \(a\).
\end{prop}
Les limites à gauche et à droite peuvent être différentes. Par exemple prenons
la fonction qui vaut 1 sur \(\Rpluss\) et est nulle ailleurs. Alors
\(\lim\limits_{0^{+}} f=1\) et \(\lim\limits_{0^{-}} f=0\). La fonction \(f\)
n'est pas continue à gauche en zéro.
\begin{proof}
  \begin{itemize}
    \item[\(\implies\)] C'est une conséquence du théorème sur les restrictions.
    \item[\(\impliedby\)] Soit \(\epsilon > 0\), il existe alors deux réels
      \(\eta_1\) et \(\eta_2\) strictement positifs tels que pour tout \(x \in
      X\) on a
      \begin{align}
        a < x \leqslant a +\eta_1 \implies \abs{f(x)-f(a)} \leqslant \epsilon;\\
        a-\eta_2 \leqslant x < a \implies \abs{f(x)-f(a)} \leqslant \epsilon.
      \end{align}
      On pose \(\eta=\min(\eta_1,\eta_2)\). Si \(\abs{x-a} \leqslant \eta\),
      trois cas de figures se présentent
      \begin{itemize}
        \item si \(a < x \leqslant a+\eta_0 \leqslant a+\eta_1\) alors
          \(\abs{f(x)-f(a)} \leqslant \epsilon\);
        \item si \(x=a\) alors \(\abs{f(x)-f(a)} = 0 \leqslant \epsilon\);
        \item si \(a-\eta_2 < a-\eta_0 \leqslant x < a\) alors \(\abs{f(x)-f(a)}
          \leqslant \epsilon\).
      \end{itemize}
      Finalement,
      \begin{equation}
        \forall x \in X \quad \abs{x-a} \leqslant \eta \implies \abs{f(x)-f(a)}
        \leqslant \epsilon.
      \end{equation}
      Alors \(\lim\limits_{a} f=f(a)\). La fonction \(f\) est continue en \(a\).
  \end{itemize}
\end{proof}

\subsection{\(\R\)-espace vectoriel des fonctions de limite nulle}
Soit \(a \in \Rbar\).
\begin{theo}
  L'ensemble des applications de \(X\) vers \(\R\) admettant une limite nulle en
  \(a\) est un sous-espace vectoriel de \(\R^X\). C'est-à-dire qu'il est non
  vide et stable par combinaison linéaire.
\end{theo}
\begin{proof}
  Avec \(a=+\infty\). Cette ensemble contient l'application nulle. Soit \(f\) et
  \(g\) deux fonctions de limite nulle en \(+\infty\) et un réel \(\lambda\).
  Alors Pour tout \(\epsilon >0\) il existe deux réels \(A\) et \(B\) tels que
  pour tout \(x \in X\) on a
  \begin{align}
    x \geqslant A &\implies \abs{f(x)} \leqslant
    \frac{\epsilon}{2(\abs{\lambda}+1)}; \\
    x \geqslant B &\implies \abs{g(x)} \leqslant \frac{\epsilon}{2}.
  \end{align}
  Soit \(C=\max(A,B)\), alors pour tout réel \(x \in X\), si \(x \geqslant C\)
  alors \begin{equation}
    \abs{\lambda f(x)+g(x)} \leqslant \abs{\lambda f(x)}+\abs{g(x)}\leqslant
    \abs{\lambda}\frac{\epsilon}{2(\abs{\lambda}+1)} + \frac{\epsilon}{2}
    \leqslant \epsilon.
  \end{equation}
  Donc \(\lambda f+g\) tend vers zéro en l'infini.
\end{proof}
\begin{prop}
Soient \(f\) et \(g\) deux applications de \(\R^X\) et \(a \in \Rbar\). On
suppose que \(f\) est de limite nulle en \(a\) et que \(g\) est bornée au
voisinage de \(a\), alors l'application \(fg\) est de limite nulle en \(a\).
\end{prop}
\begin{proof}
  Avec \(a \in \R\). La fonction \(g\) est bornée au voisinage de \(a\), il
  existe donc un réel \(M \geqslant 0\) et un réel \(h > 0\) tels que pour tout
  \(x \in X \cap [a-h,a+h]\) \(\abs{g(x)} \leqslant M\). On sait aussi que la
  limite de \(f\) en \(a\) est nulle, donc
  \begin{equation}
    \forall \epsilon > 0 \ \exists \eta > 0 \ \forall x \in X \quad \abs{x-a}
    \leqslant \eta \implies \abs{f(x)} \leqslant \frac{\epsilon}{M+1}.
  \end{equation}
  Soit \(\alpha = \min(\eta, h) >0\), alors pour tout \(x \in X\) si
  \(\abs{x-a}\leqslant \alpha\) alors \(\abs{f(x)g(x)} \leqslant
  \frac{\epsilon}{M+1}M \leqslant \epsilon\). Donc \(\lim\limits_{a} fg=0\).
\end{proof}

\subsection{Opérations algébriques}

les opérations 1,2,3 et 4 concernant les limites finies et infinies, la somme,
le produit et le quotient sont sur le polycopié.

\subsubsection{Composition}
\begin{theo}
  Soient \(I\) et \(J\) deux intervalles réels, \(f \in \R^I\) et \(g \in \R^J\)
  et \(a,b\) dans \(\Rbar\). On suppose que \(\lim\limits_{a}f=b\),
  \(\lim\limits_{b}g=\ell\) et que \(f(I) \subset J\) alors \(\lim\limits_{a} g
  \circ f=\ell\)
\end{theo}
\begin{proof}
  Soit par exemple \(a \in \R\), \(b=-\infty\) et \(\ell=+\infty\). Comme
  \(\lim\limits_{-\infty}g=+\infty\) on écrit
  \begin{equation}
    \forall A \in \R \ \exists B \in \R \ \forall y \in J \quad y \leqslant B
    \implies g(y) \geqslant A.
  \end{equation}
  Comme \(\lim\limits_{a}f=-\infty\) alors
  \begin{equation}
    \exists \eta > 0 \ \forall x \in I \quad \abs{x-a} \leqslant \eta \implies
    f(x) \leqslant B.
  \end{equation}
  Alors partant de là on a
  \begin{align}
    \forall x \in I \quad \abs{x-a} \leqslant \eta &\implies f(x) \leqslant B\\
    &\implies g \circ f(x)=g(f(x)) \geqslant A
  \end{align}
  donc \(\lim\limits_{x \to a} g \circ f(x)=+\infty=\ell\)
\end{proof}
\begin{cor}
  Soient \(I\) et \(J\) deux intervalles réels, \(f \in \R^I\) et \(g \in \R^J\)
  et \(a \in I\). On suppose que \(f\) est continue en \(a\), \(g\) est continue
  en \(f(a)\) et que \(f(I) \subset J\). Alors \(g \circ f\) est continue en
  \(a\).
\end{cor}
\begin{proof}
  D'après la définition, \(g \circ f\) est continue en \(a\) est équivalent à ce
  que la limite de \(g \circ f\) en \(a\) soit finie. On applique ensuite le
  théorème du dessus.
\end{proof}

\subsection[Caractérisation séquentielle]{Caractérisation séquentielle de la
limite et de la continuité}

\begin{theo}[Caractérisation séquentielle de la limite]
  Soient \(f \in \R^X\), \((a,\ell) \in \Rbar^2\). Alors \(\lim\limits_{a}
  f=\ell\) si et seulement si pour toute suite \(u \in X^\N\), \(\lim u =a
  \implies \lim f(u)=\ell\).
\end{theo}
\begin{proof}
  On prend par exemple \(a=-\infty\) et \(\ell \in \R\).
  \begin{itemize}
    \item[\(\implies\)] On suppose que la limite en \(a\) de \(f\) vaut
      \(\ell\). Alors
      \begin{equation}
        \forall \epsilon > 0 \ \exists A \in \R \ \forall x \in X \quad x
        \leqslant A \implies \abs{f(x)-\ell} \leqslant \epsilon.
      \end{equation}
      Soit une suite \(u \in X^\N\) telle que \(\lim u = a =-\infty\), alors
      \begin{equation}
        \exists n_0 \in \N \ \forall n \in \N \quad n \geqslant n_0 \implies u_n
        \leqslant A.
      \end{equation}
      Pour tout \(n \geqslant n_0\), \(u_n \in X\) et \(u_n \leqslant A\) donc
      \(\abs{f(u_n) -\ell} \leqslant \epsilon\). Par conséquent \(\lim
      f(u)=\ell\).
    \item[\(\impliedby\)] On montre cela par contraposée. Supposons que \(f\)
      n'admette pas \(\ell\) comme limite en \(a=-\infty\) alors
      \begin{equation}
        \exists \epsilon \ \forall A \in \R \ \exists x \in X \quad x \leqslant
        A \wedge \abs{f(x)-\ell} > \epsilon.
      \end{equation}
      En particulier avec \(A=-n\) pour tout naturel \(n\).
      \begin{equation}
        \forall n \in \N \exists x_n \in X \ x_n \leqslant -n \wedge
        \abs{f(x_n)-\ell} \geqslant \epsilon.
      \end{equation}
      La suite \((x_n)_{n \in \N}\) tend vers \(a=-\infty\) et la suite
      \(f(x_n)_{n \in \N}\) ne tend pas vers \(\ell\). On a donc montré que si
      \(f\) n'admet pas \(\ell\) pour limite en \(a\) alors il existe une suite
      \(x \in X^\N\) qui converge vers \(a\) telle que \(f(x_n)\) ne converge
      pas vers \(\ell\). Finalement, par contraposée, la limite de \(f\) en
      \(a\) vaut \(\ell\).
  \end{itemize}
\end{proof}
\begin{theo}[Caractérisation séquentielle de la continuité]
  Soient \(f \in \R^X\), \(a \in X\). Alors \(f\) est continue en \(a\) si et
  seulement si pour toute suite \(u \in X^\N\) si \(u\) tend vers \(a\) alors
  \(f(u)\) tend vers \(f(a)\).
\end{theo}
\begin{proof}
  La fonction \(f\) est continue en \(a\) si et seulement si \(\lim\limits_{a}f
  = f(a)\) et d'après le théorème précédent si et seulement si pour toute suite
  \(u \in X^\N\) \(\lim u =a \implies \lim f(u)=f(a)\).
\end{proof}

Les théorèmes précédents sont très utiles pour démontrer qu'une fonction n'a pas
de limite ou alors qu'elle n'est pas continue. Pour cela il suffit de trouver~:
\begin{itemize}
  \item une suite \(u\) qui tend vers \(a\) alors que \(f(u)\) diverge;
  \item deux suites \(u\) et \(v\) qui tendent vers \(a\) alors que \(f(u)\) et
    \(f(v)\) ont deux limites distinctes.
\end{itemize}

Par exemple, montrons que la fonction cosinus n'a pas de limite en \(+\infty\).
Soit pour tout naturel \(n\) \(u_n=2n\pi\), alors \(u\) tend à l'infini,
pourtant \(\cos(u)\) est constante égale à 1. Soit pour tout naturel \(n\),
\(v_n=\pi(2n+1)\). Alors \(v\) tend à l'infini et pourtant \(\cos(v)\) est
constante égale à \(-1\). Alors cosinus n'admet pas de limite en l'infini.

\subsection{Applications monotones}

\subsubsection{Définitions}

Soient \(X\) une partie de \(\R\) contenant au moins deux éléments et \(f \in
\R^X\). Soit \(x\) et \(x'\) deux éléments de \(X\). Alors
\begin{enumerate}
  \item \(f\) est croissante si \(x \leqslant x' \implies f(x) \leqslant
    f(x')\);
  \item \(f\) est décroissante si \(x \leqslant x' \implies f(x) \geqslant
    f(x')\);
  \item \(f\) est monotone si elle est croissante ou décroissante;
  \item \(f\) est strictement croissante si \(x < x' \implies f(x) < f(x')\);
  \item \(f\) est strictement décroissante si \(x < x' \implies f(x) > f(x')\);
  \item \(f\) est strictement monotone si elle est strictement croissante ou
    strictement décroissante.
\end{enumerate}

\subsubsection{Monotonie et opérations algébriques}

\begin{prop}
  Soient \(f\) et \(g\) deux applications de \(\R^X\) et un réel \(\lambda\).
  Alors
  \begin{enumerate}
    \item si \(f\) et \(g\) sont croissantes, alors \(f+g\) est croissante;
    \item si \(f\) est croissante et \(\lambda \geqslant 0\) alors \(\lambda f\)
      est croissante;
    \item si \(f\) est croissante et \(\lambda \leqslant 0\) alors \(\lambda f\)
      est décroissante;
    \item si \(f\) et \(g\) sont croissantes et positives, alors \(fg\) est
      croissante;
    \item si \(f\) est croissante et \(f(X)\subset ]0,+\infty[\) alors
      \(\frac{1}{f}\) est décroissante;
    \item si \(f\) est croissante et \(f(X)\subset ]-\infty, 0[\) alors
      \(\frac{1}{f}\) est décroissante.
  \end{enumerate}
\end{prop}
\begin{proof}
  Les quatre premiers points se démontrent facilement, intéressons nous au
  cinquième et au sixième. Si \(f\) est croissante et que \(f(X)\subset
  \intervalleoo{0}{+\infty}[\) ou \(f(X)\subset \intervalleoo{-\infty}{0}\).
  Alors soit deux éléments \(x\) et \(x'\) de \(X\). Si \(x \leqslant x'\)
  alors \(f(x) \leqslant f(x')\). Dans les deux cas \(f(x)f(x') > 0\). Donc en
  divisant on a \(\frac{f(x)}{f(x)f(x')} \leqslant \frac{f(x')}{f(x)f(x')}\) et
  en simplifiant \(\frac{1}{f(x')} \leqslant \frac{1}{f(x)}\). Alors
  \(\frac{1}{f}\) décroît.
\end{proof}
\begin{theo}
  Soient \(f \in \R^X\) et \(g \in \R^Y\) tels que \(f(X) \subset Y\). Si \(f\)
  et \(g\) sont monotones alors \(g \circ f\) est monotone. De plus
  \begin{itemize}
    \item \(g \circ f\) est croissante si \(f\) et \(g\) sont de même monotonie;
    \item \(g \circ f\) est décroissante sinon.
  \end{itemize}
\end{theo}

\subsubsection{Monotonie et injectivité}
\begin{theo}
  Soit \(g \in \R^X\), alors \(f\) est injective et monotone si et seulement si
  \(f\) est strictement monotone.
\end{theo}
\begin{proof}
  \begin{itemize}
    \item[\(\implies\)] Soit \((x,x') \in X^2\) tels que \(x < x'\). Alors
      \(f(x) \leqslant f(x')\) puisque \(f\) croît et ensuite comme \(x \neq
      x'\) et que \(f\) est injective alors \(f(x) < f(x')\). Donc \(f\) est
      strictement croissante.
    \item[\(\impliedby\)] \(f\) est strictement monotone donc \(f\) est
      monotone. Soit \(x\) et \(x'\) deux éléments de \(X\) différents. Deux cas
      de figure se présentent
      \begin{itemize}
        \item soit \(x < x'\) auquel cas \(f(x) < f(x')\) donc \(f(x) \neq
          f(x')\);
        \item soit \(x > x'\) auquel cas \(f(x) > f(x')\) donc \(f(x) \neq
          f(x')\).
      \end{itemize}
      Dans les deux cas, \(\forall (x,x') \in X^2 \quad x \neq x' \implies f(x)
      \neq f(x')\). La fonction \(f\) est donc injective.
  \end{itemize}
\end{proof}
\begin{cor}
  Soit \(f \in \R^X\) une application strictement monotone, alors \(f\) induit
  une bijection \(g\) de \(X\) sur \(f(X)\). De plus la bijection réciproque
  \(g^{-1}\) (notée abusivement \(f^{-1}\)) est strictement monotone, de même
  sens de monotonie que \(f\).
\end{cor}
\begin{proof}
  La fonction \(f\) est injective d'après le théorème précédent. La fonction
  \(g=f^{|f(X)}\) est surjective. Donc \(g\) est bijective. On définit donc sa
  bijection réciproque \(g^{-1}:f(X) \longrightarrow X\). On suppose que \(f\)
  est strictement décroissante. Montrons que \(g^{-1}\) est strictement
  décroissante.

  La fonction \(g\) est strictement décroissante si et seulement si
  \begin{equation}
    \forall x,x' \in f(X) \quad x < x' \implies g^{-1}(x) > g^{-1}(x') ,
  \end{equation}
  c'est-à-dire, par contraposée, si et seulement si \begin{equation}
    \forall x,x' \in f(X) \quad  g^{-1}(x) \leqslant g^{-1}(x') \implies x
    \geqslant x'.
  \end{equation}
  Montrons cette dernière inégalité pour conclure. Soient \((x, x') \in
  f(X)^2\), supposons que \(g^{-1}(x) \leqslant g^{-1}(x')\) alors comme \(f\)
  est décroissante on a \(f \circ g^{-1}(x) \geqslant f \circ g^{-1}(x')\) donc
  comme \(f=g\), on a bien \(x \geqslant x'\). La fonction \(g^{-1}\) est alors
  strictement décroissante.
\end{proof}

\subsubsection{Monotonie et limite}

\begin{theo}
  Soit \(a \in \R\) et \(b \in \R\cup\{+\infty\}\) et \(f : \intervallefo{a}{b}
  \longrightarrow \R\). On suppose que \(f\) croît. Alors
  \begin{itemize}
    \item Soit \(f\) est majorée et \(f\) admet une limite en \(b\) et
      \(\lim\limits_{b} = \sup\limits_{\intervalleff{a}{b}}f\);
    \item Soit \(f\) n'est pas majorée et \(\lim\limits_{b}f=+\infty\).
  \end{itemize}
\end{theo}
\begin{proof}
  Soit la partie \(E=\enstq{f(x)}{x \in \intervallefo{a}{b}}\).
  \begin{itemize}
    \item si \(f\) est majorée, alors \(E\) est une partie non vide et majorée
      de \(\R\). Elle admet donc une borne supérieure notée S. Par
      caractérisation de la borne supérieure
      \begin{equation}
        \forall \epsilon >0 \ \exists y_0 \in E \quad S-\epsilon <y_0 \leqslant
        S,
      \end{equation}
      ou alors
      \begin{equation}
        \forall \epsilon >0 \ \exists x_0 \in \intervallefo{a}{b} \quad
        S-\epsilon <f(x_0) \leqslant S.
      \end{equation}
      Pour tout \(x \geqslant x_0\) dans \(\intervallefo{a}{b}\) \(S \geqslant
      f(x) \geqslant f(x_0) > S-\epsilon\). Pour tout \(x \geqslant x_0\) dans
      \(\intervallefo{a}{b}\), \(\abs{f(x)-S}=S-f(x) \leqslant \epsilon\) donc
      \(\lim\limits_{b} f =S\).
    \item Si \(f\) n'est pas majorée. Alors
      \begin{equation}
        \forall A \in \R \ \exists y_0 \in E \quad y_0 \geqslant A,
      \end{equation}
      ou alors
      \begin{equation}
        \forall A \in \R \ \exists x_0 \in \intervallefo{a}{b} \quad f(x_0)
        \geqslant A.
      \end{equation}
      La fonction \(f\) est croissante, donc
      \begin{equation}
        \forall x \in \intervallefo{a}{b} \quad x \geqslant x_0 \implies f(x)
        \geqslant f(x_0) \geqslant A.
      \end{equation}
      Alors \(f\) tend vers \(+\infty\).
  \end{itemize}
\end{proof}

\begin{theo}
  Soient \(a\) et \(b\) deux réels tels que \(a < b\) et \(f:\intervalleff{a}{b}
  \longrightarrow \R\). On suppose que \(f\) croît. Alors \(f\) est majorée et
  \(\lim\limits_{b^{-}}f \leqslant f(b)\).
\end{theo}
\begin{proof}
  \(f\) est croissante donc \(f(b)\) est un majorant de
  \(E=f(\intervallefo{a}{b})\). Alors \(\sup E\) existe et \(\sup E \leqslant
  f(b)\), c'est-à-dire \(\lim\limits_{b^{-}}f \leqslant f(b)\).
\end{proof}
On peut énoncer de la même manière.
\begin{theo}
  Soient \(a \in \R\cup \{-\infty\}\) et \(b \in \R\) tels que \(b > a\) et
  \(f:\intervalleof{a}{b} \longrightarrow \R\) croissante. Alors
  \begin{itemize}
    \item soit \(f\) est minorée sur \(\intervalleof{a}{b}\) et alors \(f\)
      admet une limite en \(a\) telle que \(\lim\limits_{a} f =
      \inf\limits_{\intervalleof{a}{b}}f\);
    \item soit \(f\) n'est pas minorée et \(\lim\limits_{a} f = -\infty\).
  \end{itemize}
\end{theo}
\begin{proof}
  Soit la fonction \(\fonction{g}{\intervallefo{-b}{-a}}{\R}{x}{-f(-x)}\). La
  fonction \(g\) est croissante comme \(f\). Appliquons le théorème précédent à
  \(g\) puisque \(f\) est majorée si et seulement si \(g\) est minorée.
\end{proof}
\begin{theo}
  Soit deux réels \(a,b\) tels que \(a < b\) et soit une application croissante
  \(f:\intervalleff{a}{b} \longrightarrow \R\). Alors \(f\) est minorée,
  \(\lim\limits_{a^{+}} f\) existe et \(f(a) \leqslant \lim\limits_{a^{+}} f\).
\end{theo}
\begin{cor}
  Soient deux réels \(a\) et \(b\) tels que \(a < b\), un réel \(c \in
  \intervalleoo{a}{b}\) et une application \(f :\intervalleff{a}{b}
  \longrightarrow \R\) croissante. Alors \(\lim\limits_{c^{+}} f\) et
  \(\lim\limits_{c^{-}} f\) existe et de plus \(\lim\limits_{c^{-}} f \leqslant
  f(c) \leqslant \lim\limits_{c^{+}} f\).
\end{cor}
\begin{proof}
  C'est une conséquence des deux théorèmes précédents.
\end{proof}

Il existe des énoncés analogues pour les fonctions décroissantes.

\section{Continuité sur un intervalle}

Soit \(I\) un intervalle de \(\R\) qui contient au moins deux éléments
distincts.

\subsection{Retour sur la continuité en un point}

\begin{rappel}
  Soit \(f \in \R^I\) et \(a \in I\). Alors \(f\) est continue en \(a\) si et
  seulement si \(f\) admet une limite finie en \(a\), c'est-à-dire
  \begin{align}
    &\iff \lim\limits_{x \to a} f(x) = f(a)\\
    &\iff \forall \epsilon>0 \ \exists \eta>0 \ \forall x \in I \quad \abs{x-a}
    \leqslant \eta \implies \abs{f(x)-f(a)} \leqslant \epsilon\\
    &\iff \forall \epsilon>0 \ \exists \eta>0 \ \forall x \in I\setminus\{a\}
    \quad \abs{x-a} \leqslant \eta \implies \abs{f(x)-f(a)} \leqslant \epsilon\\
    &\iff \lim\limits_{x \in I\setminus\{a\} \to a } f = f(a).
  \end{align}
\end{rappel}
On notera \(\contpoint{a}{I}{\R}\) l'ensemble des applications de \(\R^I\) qui
sont continues en \(a\).

\begin{rappel}
  Soit \(f \in \R^I\) et \(a \in I\). Alors, \(f\) n'est pas continue en \(a\)
  s'écrit
  \begin{equation}
    \exists \epsilon >0 \ \forall \eta >0 \ \exists x \in I\setminus\{a\} \quad
    \abs{x-a} \leqslant \eta \text{~et~} \abs{f(x)-f(a)} > \epsilon.
  \end{equation}
\end{rappel}

\begin{prop}
  \begin{enumerate}
    \item Si \(f \in \contpoint{a}{I}{\R}\) alors \(f\) est bornée au voisinage
      de \(a\), ça ne signifie pas que \(f\) est bornée sur \(I\).
    \item L'ensemble \(\contpoint{a}{I}{\R}\) est un sous-espace vectoriel et un
      sous-anneau de \(\R^I\).
    \item Si \(f \in \contpoint{a}{I}{\R}\) et \(f(a) \neq 0\) alors
      \(\frac{1}{f}\) est définie au voisinage de \(a\) et \(\frac{1}{f}\) est
      continue en \(a\).
    \item Soit \(f \in \R^I\), \(J\) un intervalle réel et \(g \in \R^J\) telles
      que \(f(I) \subset J\). Alors si \(f \in \contpoint{a}{I}{\R}\) et \(f \in
      \contpoint{a}{J}{\R}\) alors \(g \circ f \in \contpoint{a}{I}{\R}\).
  \end{enumerate}
\end{prop}

\subsection{Continuité sur un intervalle}

\subsubsection{Définition}

\begin{defdef}
  Soit \(f \in \R^I\). On dit que \(f\) est continue, ou continue sur \(I\), si
  \(f\) est continue en tout point de I. On note \(\cont{I}{\R}\) l'ensemble des
  applications continues de \(I\) vers \(\R\). De plus
  \begin{equation}
    \cont{I}{\R}=\bigcap_{a \in I} \contpoint{a}{I}{\R}.
  \end{equation}

\end{defdef}

\subsubsection{Restrictions et prolongements}

\begin{prop}
  Soient \(f \in \R^I\), \(J\) un intervalle inclus dans \(I\) qui contient au
  moins deux points. Si la fonction \(f\) est continue, alors \(f_{|J}\) est
  continue. La réciproque est fausse.
\end{prop}
\begin{proof}
  Soit \(a \in J\) alors \(a \in I\) et puisque \(f\) est continue \(f(a) =
  \lim\limits_{x\in I, x\to a} f(x)\) et comme \(J \subset I\), d'après les
  résultats sur les restrictions de la continuité en un point, on peut écrire
  que \(f_{|J}(a) = \lim\limits_{x\in I, x\to a} f_{|J}(x)\), alors \(f_{|J}\)
  est continue en \(a\). Puisque c'est vrai pour tout \(a \in J\) alors
  \(f_{|J}\) est continue.
\end{proof}
\begin{prop}
  Soient \(a,b,c\) trois réels tels que \(a<b<c\) et \(f:\intervalleff{a}{c}
  \rightarrow \R\) alors, \(f\) est continue sur \(\intervalleff{a}{c}\) si et
  seulement si \(f\) est continue sue \(\intervalleff{a}{b}\) et sur
  \(\intervalleff{b}{c}\).
\end{prop}
\begin{proof}
  \begin{itemize}
    \item[\(\implies\)] C'est la conséquence de la proposition précédente.
    \item[\(\impliedby\)] La fonction \(f\) est continue en tout point de
      \(\intervalleff{a}{c}\) autre que \(b\). Pourtant \(\lim\limits_{b^{-}}
      f=f(b)\) et \(\lim\limits_{b^{+}} f=f(b)\) donc \(\lim\limits_{b} f\)
      existe et vaut \(f(b)\). Alors \(f\) est continues en \(b\).
  \end{itemize}
\end{proof}
\begin{prop}
  Soient \(a\) et \(b\) deux réels, \(f:\intervallefo{a}{b} \rightarrow \R\) une
  application continue. On suppose que \(f\) admet une limite finie \(\ell\) en
  \(b\). Alors l'application
  \(\fonction{g}{\intervalleff{a}{b}}{\R}{x}{\begin{cases}f(x) & x \in
  \intervallefo{a}{b} \\ \ell & x=b\end{cases}}\) est continue sur
  \(\intervalleff{a}{b}\).
\end{prop}
\begin{proof}
  On a déjà vu que \(g\) est continue en \(b\) et que c'est l'unique
  prolongement de \(f\) à \(\intervalleff{a}{b}\) continu en \(b\). Il reste
  simplement à voir que \(g\) est continue en tout point de
  \(\intervallefo{a}{b}\). Or \(g_{\intervallefo{a}{b}}=f\) et \(f\) est
  continue, donc \(g\) est continue sur \(\intervalleff{a}{b}\).
\end{proof}

\subsection{Opérations algébriques}

\begin{prop}
  Soient \(f\) et \(g\) deux applications continues de \(I\) dans \(\R\) et
  \(\lambda\) un réel, alors~:
  \begin{itemize}
    \item \(f+g\) est continue;
    \item \(\lambda f\) est continue;
    \item pour tout naturel \(n\), \(f^n\) est continue;
    \item \(fg\) est continue;
    \item si \(f\) ne s'annule pas, \(\frac{1}{f}\) est continue;
    \item si \(f\) ne s'annule pas, \(\frac{g}{f}\) est continue;
    \item si \(f\) ne s'annule pas, pour tout relatif \(n\), \(f^n\) est
      continue.
  \end{itemize}
  \(\cont{I}{\R}\) est alors un sous-espace vectoriel et un sous-anneau de
  \(\R^I\).
\end{prop}
\begin{proof}
  C'est la conséquence des résultats vus sur la continuité en un point.
\end{proof}
\begin{prop}
  Soient \(f\) et \(g\) deux application continues de I vers \(\R\). Alors
  \begin{enumerate}
    \item \(\abs{f}\) est continue;
    \item \(\sup(f,g)\) et \(\inf(f,g)\) sont continues;
    \item \(f^{+}\) et \(f^{-}\) sont continues.
  \end{enumerate}
\end{prop}
\begin{proof}
  \begin{enumerate}
    \item Soit \(a\) un élément de \(I\), alors \(f\) admet une limite finie
      \(\ell\) en \(a\), donc \(\abs{f}\) tend vers \(\abs{\ell}\);
      l'application \(\abs{f}\) est donc continue en tout point de \(I\);
    \item Puisque \(\sup(f,g)=\frac{f+g+\abs{f-g}}{2}\) et
      \(\inf(f,g)=\frac{f+g-\abs{f-g}}{2}\) alors d'après le point précédent
      elles sont continues;
    \item Puisque \(f^{+}=\sup(f,0)\) et \(f^{-}=\sup(-f,0)\) alors d'après le
      point précédent elles sont continues;
  \end{enumerate}
\end{proof}
\begin{prop}
  Soient \(I\) et \(J\) deux intervalles réels et \(f\) et \(g\) deux
  applications telles que \(f(I) \subset J\). Si \(f\) et \(g\) sont continues
  alors la composée \(g \circ f\) est continue.
\end{prop}
\begin{proof}
  On le démontre en prenant un point \(a\) quelconque de \(I\).
\end{proof}

\subsection{Image d'un intervalle par une application continue}

\begin{rappel}
  On a montré au chapitre~\ref{chap:reels} sur les réels que les intervalles de
  \(\R\) sont les parties
  convexes de \(\R\). C'est-à-dire les parties \(I\) telles que
  \begin{equation}
    \forall (x,y) \in I^2 \quad x \leqslant y \implies \intervalleff{x}{y}
    \subset I
  \end{equation}
\end{rappel}

\subsubsection[Image continue d'un intervalle quelconque]{Image d'un intervalle
quelconque \(I\) par une application continue \(f\)}

\begin{theo}
  Soient \((a,b) \in I^2\) avec \(a<b\) et une application \(f:I\rightarrow \R\)
  continue telle que \(f(a) f(b) < 0\) (Elle change de signe sur l'intervalle).
  Alors il existe un élément \(c \in \intervalleoo{a}{b}\) tel que \(f(c)=0\).
\end{theo}
\begin{proof}[Démonstration hors-programme]
  On utilise une méthode de dichotomie et on construit par récurrence deux
  suites \(a\) et \(b\). Initialement, on pose \(a_0=a\) et \(b_0=b\) on a bien
  \(f(a_0)f(b_0) < 0\) et quitte à changer \(f\) en \(-f\) on peut supposer que
  \(f(a_0)>0\) et \(f(b_0)<0\). Pour l'hérédité, on suppose avoir construit
  \(a\) et \(b\) jusqu'au rang \(n\) de telle sorte que
  \begin{equation}
    f(a_n) > 0 \geqslant f(b_n).
  \end{equation}
  Soit alors \(c_n=\frac{a_n+b_n}{2}\). Si \(f(c_n) >0\) alors \(a_{n+1}=c_n\)
  et \(b_{n+1}=b_n\), sinon \(a_{n+1}=a_n\) et \(b_{n+1}=c_n\). Dans les deux
  cas \(f(a_{n+1})> 0 \geqslant f(b_{n+1})\). Les deux suites \((a_n)\) et
  \((b_n)\) vérifient~:
  \begin{enumerate}
    \item \(\forall n \in \N \quad f(a_n) > 0 > f(b_n)\);
    \item La suite \((a_n)\) est croissante et la suite \((b_n)\) est
      décroissante;
  \item pour tout naturel \(n\), \(b_{n+1}-a_{n+1}=\frac{b_n -a_n}{2}\).
  \end{enumerate}
  La suite \((b_n-a_n)\) est géométrique de raison \(\frac{1}{2}\) donc elle
  tend vers zéro. Les suite \((a_n)\) et \((b_n)\) sont adjacentes. Alors elles
  tendent vers la même limite qu'on note \(c\). Alors pour tout naturel \(n\),
  \(a\leqslant a_n \leqslant c \leqslant b_n \leqslant b\). Donc \(c \in
  [a,b]\). On sait que \(f\) est  continue en \(c\). Alors les suites
  \((f(a_n))\) et \((f(b_n))\) tendent vers \(f(c)\). D'ailleurs pour tout
  naturel \(n\), \(f(a_n) \geqslant 0 \geqslant f(b_n)\) donc en passant à la
  limite on obtient que \(f(c) \geqslant 0 \geqslant f(c)\) donc \(f(c)=0\). On
  a trouvé un élément \(c \in \intervalleff{a}{b}\) tel que \(f(c)=0\). Or
  \(f(a)f(b) < 0\) donc \(c\) est dans\(\intervalleoo{a}{b}\).
\end{proof}
\begin{cor}[Théorème des valeurs intermédiaires]
  Soient \(f\) une application continue sur \(I \subset \R\) à valeurs réelles
  et \((a,b) \in I^2\) tels que \(a < b\). Alors pour tout \(\lambda\)
  strictement comprit entre \(f(a)\) et \(f(b)\), il existe un réel \(c \in
  \intervalleoo{a}{b}\) tel que \(\lambda =f(c)\).
\end{cor}
\begin{proof}
  Soit l'application continue \(g=f-\lambda\). Puisque \(\lambda\) est
  strictement compris entre \(f(a)\) et \(f(b)\) on a \(g(a)g(b) < 0\). Donc, en
  appliquant le théorème, il existe un réel \(c \in \intervalleoo{a}{b}\) tel
  que \(g(c)=0\), c'est-à-dire \(f(c)=\lambda\).
\end{proof}
\begin{cor}[Théorème fondamental]
  L'image directe par une application continue d'un intervalle est un
  intervalle.
\end{cor}
\begin{proof}
  Soient \(I\) un intervalle et \(f \in \R^I\) une application continue.
  Montrons que \(f(I)\) est un intervalle en montrant qu'il est convexe. Soit
  \((y,y') \in f(I)^2\) avec \(y \leqslant y'\). Si \(y=y'\) alors
  \(\intervalleff{y}{y'} \subset f(I)\), sinon il existe \((a,b) \in I^2\) tel
  que \(y=f(a)\) et \(y'=f(b)\). Pour tout \(\lambda\) strictement compris entre
  \(y\) et \(y'\) il existe un réel \(c \in I\) strictement compris entre \(a\)
  et \(b\) tel que \(\lambda=f(c) \in f(I)\). Alors \(\intervalleff{y}{y'}
  \subset f(I)\). Ainsi \(f(I)\) est convexe, c'est donc un intervalle.
\end{proof}
\begin{cor}
  Soient un intervalle \(I\) et \(f \in \cont{I}{\R}\), alors
  \begin{enumerate}
    \item si \(f\) admet pour limite \(+\infty\) (resp.\ \(-\infty\)) en une
      extrémité de \(I\) et prend par ailleurs une valeur strictement négative
      (resp.\ positive) en un point de \(I\), alors \(f\) s'annule au moins en
      un point de \(I\);
    \item si \(f\) admet pour limite \(+\infty\) en une extrémité de \(I\) et
      \(-\infty\) en l'autre alors \(f\) s'annule en au moins un point de \(I\).
  \end{enumerate}
\end{cor}
\begin{proof}
  \begin{enumerate}
    \item Soit par exemple \(I=\intervallefo{a}{b}\) avec \(\lim\limits_{b} f
      =+\infty\). Alors il existe \(d \in I\) tel que \(f(d)>0\). On sait qu'il
      existe \(e \in I\) tel que \(f(e) < 0\). Alors d'après le théorème il
      existe \(c \in I\) tel que \(f(c)=0\).
    \item Soit par exemple \(I=\intervalleoo{a}{b}\) avec \(\lim\limits_{b} f
      =+\infty\) et \(\lim\limits_{a} f =-\infty\) alors il existe \(d\) et
      \(e\) dans I tels que \(f(d)>0\) et \(f(e) <0\). D'après le théorème, il
      existe un point \(c\) de I tel que \(f(c)=0\).
  \end{enumerate}
\end{proof}

\subsubsection[Image continue d'un segment]{Image d'un segment
\(\intervalleff{a}{b}\) par une application continue}

Si \(A\) est une partie de \(\R\) non majorée, on définit \(\sup A = + \infty\)
par convention. Si on voit \(A\) comme une partie de \(\Rbar\), alors
\(+\infty\) est un majorant de \(A\) dans \(\Rbar\) et c'est même le seul donc
c'est le plus petit et donc \(+\infty=\sup_{\Rbar} A\).

\begin{lemme}
  Soient \(A\) et \(B\) deux parties non vide de \(\R\), alors
  \begin{itemize}
    \item si \(A\) et \(B\) sont majorées alors \(A \cup B\) est majorée et
      \(\sup(A\cup B)=\max(\sup A, \sup B)\);
    \item si \(A\) ou \(B\) n'est pas majoré alors \(A \cup B\) n'est pas
      majorée et \(\sup(A\cup B)=+\infty\).
  \end{itemize}
\end{lemme}
\begin{proof}
  \begin{itemize}
    \item Si \(A\) et \(B\) sont majorées, alors \(\sup A\) et\(\sup B\)
      existent. Soit \(c \in A \cup B\). Si \(c \in A\) alors \(c \leqslant \sup
      A \leqslant \max(\sup A, \sup B)\), sinon \(c \in B\) alors \(c \leqslant
      \sup B \leqslant \max(\sup A, \sup B)\). Dans tous les cas \(c \leqslant
      \max(\sup A, \sup B)\) donc \(A \cup B\) est majorée. Soit \(M\) un autre
      majorant de \(A \cup B\), alors \(M\) est un majorant de \(A\) et de \(B\)
      et donc \(M \geqslant \sup A\) et \(M \geqslant \sup B\). Par conséquent
      \(M \geqslant \max(\sup A, \sup B)\). C'est donc bien la borne supérieure
      \(\max(\sup A, \sup B)=\sup(A \cup B)\).
    \item Si \(A\) ou \(B\) n'est pas majorée, alors l'union non plus et donc
      \(\sup(A\cup B)=+\infty\).
  \end{itemize}
\end{proof}
Alors pour tout réel \(\alpha < \beta< \gamma\) et pour toute application \(f\)
\begin{equation}
  \sup\limits_{\intervalleff{\alpha}{\gamma}}f =
  \max(\sup\limits_{\intervalleff{\alpha}{\beta}}f,
  \sup\limits_{\intervalleff{\beta}{\gamma}}f).
\end{equation}
%
\begin{theo}[Théorème des bornes]
  \label{theo:bornes}
  Soient \(a<b\) deux réels et \(f:\intervalleff{a}{b} \longrightarrow \R\)
  continue. Alors \(f\) est bornée et atteint ses bornes, c'est-à-dire qu'il
  existe \(c\) et \(d\) dans \(\intervalleff{a}{b}\) tels que
  \(f(c)=\sup\limits_{\intervalleff{a}{b}}f\) et
  \(f(d)=\inf\limits_{\intervalleff{a}{b}}f\)
\end{theo}
\begin{proof}
  On construit deux suites \((a_n)\) et \((b_n)\) par récurrence. Initialement
  \(a_0=a\) et \(b_0=b\). Soit \(S=\sup\limits_{\intervalleff{a}{b}}f \in
  \R\cup\{+\infty\}\) alors \(S=\sup\limits_{\intervalleff{a_0}{b_0}}f \in
  \R\cup\{+\infty\}\). On suppose avoir construit \((a_n)\) et \((b_n)\) telles
  que \(S=\sup\limits_{\intervalleff{a_n}{b_n}}f \in \R\cup\{+\infty\}\) alors
  on pose \(c_n=\frac{a_n+b_n}{2}\). On sait d'après le lemme que
  \begin{equation}
    \sup\limits_{\intervalleff{a_n}{b_n}}f=\max(\sup\limits_{\intervalleff{a_n}{c_n}}f,
    \sup\limits_{\intervalleff{c_n}{b_n}}f).
  \end{equation}
  Deux cas se présentent~:
  \begin{itemize}
    \item si \(S=\sup\limits_{\intervalleff{a_n}{c_n}}f\) alors \(a_{n+1}=a_n\)
      et \(b_{n+1}=c_n\);
    \item si \(S=\sup\limits_{\intervalleff{c_n}{b_n}}f\) alors \(a_{n+1}=c_n\)
      et \(b_{n+1}=b_n\).
  \end{itemize}
  Dans les deux cas, la suite \((b_n-a_n)\) est géométrique de raison
  \(\frac{1}{2}\). Les suites \((a_n)\) et \((b_n)\) vérifient les propriétés
  suivantes
  \begin{enumerate}
    \item pour tout naturel \(n\), \(S=\sup\limits_{\intervalleff{a_n}{b_n}}f\);
    \item la suite \(a\) est croissante et la suite \(b\) est décroissante;
    \item la suite \((b-a)_n\) tend vers zéro.
  \end{enumerate}
  Alors les suite \((a_n)\) et \((b_n)\) sont adjacentes, elles tendent alors
  vers la même limite notée \(c\). Alors \(c \in \intervalleff{a}{b}\). Il reste
  à prouver que \(S\) est finie et que \(f(c)=S\). Supposons par l'absurde que
  \(S=+\infty\). Alors pour tout naturel \(n\), \(+\infty =
  \sup\limits_{\intervalleff{a_n}{b_n}}f\). Il existe alors \(x_n \in
  \intervalleff{a_n}{b_n}\) tel que \(f(x_n) \geqslant n\). On sait aussi que
  pour tout naturel \(n\), \(a_n \leqslant x_n \leqslant b_n\) et que par
  théorème des gendarmes \((x_n)_{n _in \N}\) tend vers \(c\). D'une part, pour
  tout naturel \(n\), \(f(x_n) \geqslant n\) donc en passant à la limite \(\lim
  f(x_n) = +\infty\). D'autre part comme \(f\) est continue sur
  \(\intervalleff{a}{b}\), notamment en \(c\), alors \((f(x_n))_{n\in \N}\) tend
  vers \(f(c)\). On arrive donc à une absurdité par unicité de la limite donc
  \(S\) est finie. Avec \(\epsilon=\frac{1}{n+1}\), par caractérisation de la
  borne supérieure, il existe \(x_n \in \intervalleff{a_n}{b_n}\) tel que
  \begin{equation}
    S-\frac{1}{n+1} < f(x_n) \leqslant S.
  \end{equation}
  Le théorème des gendarmes appliqué à cette inégalité nous montre que \(S=\lim
  f(x_n)\). De la même manière pour tout naturel \(n\), \(a_n \leqslant x_n
  \leqslant b_n\) donc par théorème des gendarmes \((x_n)_{n\in \N}\) tend vers
  \(c\) et comme \(f\) est continue \((f(x_n))_{n\in \N}\) tend vers \(f(c)\).
  Par unicité de la limite \(f(c)=S\). La fonction continue \(f\) est majorée
  sur \(\intervalleff{a}{b}\) et atteint sa borne supérieure.
\end{proof}

\begin{theo}[Théorème fondamental]
  L'image directe par une application continue d'un segment est un segment.
\end{theo}
\begin{proof}
  Soit \(f:\intervalleff{a}{b} \longrightarrow \R\) une application continue. On
  sait déjà que l'image directe \(f(\intervalleff{a}{b})\) est un intervalle. Le
  théorème des bornes nous dit que \(f\) est bornée et atteint ses bornes. Ainsi
  \(f(\intervalleff{a}{b})=\intervalleff{\inf\limits_{\intervalleff{a}{b}}f}{\sup\limits_{\intervalleff{a}{b}}f}\).
\end{proof}

\subsubsection[Image directe d'un intervalle]{Image directe d'un intervalle par
une application continue et monotone}

\begin{theo} \label{theo:imagesegment}
  Soient \(a<b\) des réels et \(f:\intervalleff{a}{b} \longrightarrow \R\) une
  application continue et croissante. Alors
  \(f(\intervalleff{a}{b})=\intervalleff{f(a)}{f(b)}\).
\end{theo}
\begin{proof}
  C'est une conséquence du théorème fondamental. Comme \(f\) est continue
  \(f(\intervalleff{a}{b})=\intervalleff{\inf\limits_{\intervalleff{a}{b}}f}{\sup\limits_{\intervalleff{a}{b}}f}\).
  Il existe \(c\) et \(d\) dans \(\intervalleff{a}{b}\) tels que
  \(\inf\limits_{\intervalleff{a}{b}}f=f(c)\) et
  \(\sup\limits_{\intervalleff{a}{b}}f=f(d)\) or \(f\) croît donc \(f(a)=f(c)\)
  et \(f(b)=f(d)\). Ainsi \(f(\intervalleff{a}{b})=\intervalleff{f(a)}{f(b)}\).
\end{proof}

\begin{theo}
  Soient \(a\) un réel et \(b\) dans \(\R\cup\{+\infty\}\) avec \(a < b\), une
  application continue croissante \(f:\intervallefo{a}{b} \longrightarrow \R\).
  Alors
  \begin{enumerate}
    \item si \(f\) n'est pas majorée,
      \(f(\intervallefo{a}{b})=\intervallefo{f(a),}{+\infty}\)
    \item si \(f\) est majorée, on pose \(S=\sup\limits_{\intervallefo{a}{b}}f\)
      et alors
      \begin{equation}
        \intervallefo{f(a)}{S} \subset f(\intervallefo{a}{b}) \subset
        \intervalleff{f(a)}{S}.
      \end{equation}
      Si on suppose de plus que \(f\) est strictement croissante alors
      \begin{equation}
        f(\intervallefo{a}{b}) = \intervallefo{f(a)}{S}.
      \end{equation}
  \end{enumerate}
\end{theo}
\begin{proof}
  \begin{enumerate}
    \item
      \begin{itemize}
        \item Soit \(y \in f(\intervallefo{a}{b})\). Il existe alors un réel \(x
          \in \intervallefo{a}{b}\) tel que \(y=f(x)\). Puisque \(f\) croît
          alors \(f(x) \geqslant f(a)\). Alors \(y \in
          \intervallefo{f(a)}{+\infty}\).
        \item Soit \(y \in \intervallefo{f(a)}{+\infty}\), il existe alors un
          réel \(c \in \intervallefo{a}{b}\) tel que \(f(c) \geqslant y\).
          L'application continue \(f\) croît sur \(\intervalleff{a}{c}\) donc
          d'après le théorème~\ref{theo:imagesegment} on écrit que
          \(f(\intervalleff{a}{c})=\intervalleff{f(a)}{f(c)}\). On sait que
          \(f(a) \le y \leqslant f(c)\) alors \(y \in f(\intervalleff{a}{c})\).
          Il existe donc un réel \(d \in \intervalleff{a}{c}\) tel que
          \(y=f(d)\).
      \end{itemize}
      Par double inclusion, on a montré l'égalité
      \(f(\intervallefo{a}{b})=\intervallefo{f(a)}{+\infty}\).
    \item Si \(f\) est majorés, alors on pose
      \(S=\sup\limits_{\intervallefo{a}{b}}f\)
      \begin{itemize}
        \item Si \(y \in f(\intervallefo{a}{b})\), alors il existe \(x \in
          \intervallefo{a}{b}\) tel que \(y=f(x)\). \(f\) est croissante donc
          \(y \geqslant f(a)\) mais \(S=\sup\limits_{\intervallefo{a}{b}}f\)
          alors \(y \leqslant S\). Alors \(y \in \intervalleff{f(a)}{S}\). D'où
          l'inclusion.
        \item Si \(y \in \intervallefo{f(a)}{S}\), alors \(y < S\) et soit
          \(\epsilon=S-y\). Par caractérisation de la borne supérieure, il
          existe un réel \(c \in \intervallefo{a}{b}\) tel que
          \begin{equation}
            y=S-\epsilon < f(c) \leqslant S
          \end{equation}
          la fonction \(f\) croît et est continue sur \(\intervalleff{a}{c}\)
          donc \(f(\intervalleff{a}{c})=\intervalleff{f(a)}{f(c)}\). Alors \(y
          \in f(\intervalleff{a}{c})\) et donc \(y \in f(\intervallefo{a}{b})\)
          (car \(f(\intervalleff{a}{c}) \subset f(\intervallefo{a}{b})\)).
      \end{itemize}
      On a montré que
      \begin{equation}
        \intervallefo{f(a)}{S} \subset f(\intervallefo{a}{b}) \subset
        \intervalleff{f(a)}{S},
      \end{equation}
      c'est-à-dire
      \begin{equation}
        f(\intervallefo{a}{b}) = \intervallefo{f(a)}{S} \text{~ou~}
        f(\intervallefo{a}{b}) = \intervalleff{f(a)}{S}.
      \end{equation}
      Dans le cas général, on ne peut rien dire de plus. Dans le cas où \(f\)
      est strictement croissante, on va montrer que \(f(\intervallefo{a}{b}) =
      \intervallefo{f(a)}{S}\). Il faut montrer que \(S\) n'admet pas
      d'antécédents. On suppose par l'absurde que \(S\) en admet un. Alors il
      existe un réel \(c \in \intervallefo{a}{b}\) tel que \(f(c)=S\). Puisque
      \(f\) croît strictement il existe un \(x \in \intervalleoo{c}{b}\) tel que
      \(f(x) \geqslant f(c)=S\). Or c'est absurde puisque \(S\) est la borne
      supérieure de \(f\). Alors \(S\) n'admet pas d'antécédent et
      \(f(\intervallefo{a}{b}) = \intervallefo{f(a)}{S}\).
  \end{enumerate}
\end{proof}

\subsection{Continuité d'une bijection réciproque}

\subsubsection[Pour qu'une application monotone soit continue]{Condition
suffisante pour qu'une application monotone soit continue}

\begin{theo}\label{theo:monotone-intervalle-donc-continue}
  Soit \(f\) une application monotone d'un intervalle \(I\) vers \(\R\). Si
  \(f(I)\) est un intervalle, alors \(f\) est continue. La réciproque est
  toujours vraie même sans l'hypothèse de monotonie.
\end{theo}
\begin{proof}
  Supposons par exemple que \(f\) soit croissante tel que \(f(I)\) soit un
  intervalle. Supposons par l'absurde qu'il existe \(c \in I\) tel que \(f\) ne
  soit pas continue en \(c\)
  \begin{enumerate}
    \item Dans le cas où \(c\) est à l'intérieur de \(I\). On a
      \begin{equation}
        \ell_1 = \lim\limits_{c^{-}}f \leqslant f(c) \leqslant \ell_2 =
        \lim\limits_{c^{+}}f.
      \end{equation}
      Dire que \(f\) n'est pas continue en \(c\) revient à dire que l'une des
      deux au moins de ces inégalités est stricte. Supposons que c'est la
      première \(\ell_1 < f(c)\) et \(\ell_1 = \lim\limits_{c^{-}}f =
      \sup\limits_{x < c} f(x)\). Soient \(b \in I\) tel que \(b < c\), alors
      \(f(b) \leqslant l_1\). Soit \(y \in \intervalleoo{\ell_1}{f(c)} \subset
      \intervalleoo{f(b)}{f(c)}\). Alors on a
      \begin{itemize}
        \item \(y \in \intervalleoo{f(b)}{f(c)}\);
        \item \(f(b)\) et \(f(c)\) sont dans \(f(I)\);
        \item \(I\) est convexe (car c'est un intervalle).
      \end{itemize}
      Donc \(y \in f(I)\). Il existe donc un \(x \in I\) tel que \(y=f(x)\), or
      \begin{itemize}
        \item si \(x < c\) alors \(f(x) < \ell_1\);
        \item si \(x \geqslant c\) alors \(f(x) \geqslant f(c)\) (car \(f\) est
          croissante).
      \end{itemize}
      C'est absurde puisqu'on avait posé \(f(x)=y \in
      \intervalleoo{\ell_1}{f(c)}\). Alors \(f\) est continue.
    \item Si \(c\) est une borne de \(I\) alors on fait le même raisonnement
      avec une inégalité.
  \end{enumerate}
\end{proof}

\subsubsection{Applications continues strictement monotones}

\begin{theo}
  Soit \(f\) une application continue et strictement monotone sur l'intervalle
  \(I\). Alors
  \begin{enumerate}
    \item \(f(I)\) est un intervalle;
    \item \(f\) induit une bijection de \(I\) sur \(f(I)\);
    \item la bijection réciproque \(g\), notée abusivement \(f^{-1}\), est
      continue et strictement monotone (de même monotonie que \(f\)) sur
      \(f(I)\).
  \end{enumerate}
\end{theo}
\begin{proof}
  Tout a déjà été vu sauf la continuité de \(f^{-1}\). On sait que \(f^{-1}\)
  est strictement monotone (déjà vu) et que \(I=f^{-1}(f(I))\) est un
  intervalle. Donc \(f^{-1}\) est continue.
\end{proof}

\subsubsection[Pour qu'une application continue soit monotone]{Condition
suffisante pour qu'une application continue soit monotone}

\begin{theo}
  Soit \(f\) une application continue et injective de \(I\) dans \(\R\). Alors
  \(f\) est monotone et le théorème précédent s'applique.
\end{theo}
\begin{proof}[Démonstration par l'aburde]
  Supposons que \(f\) n'est pas monotone, alors \(f\) n'est pas croissante donc
  il existe \((a,b) \in I^2\) tel que \(a < b\) et \(f(a) > f(b)\). Elle n'est
  pas non plus décroissante donc il existe \((c,d) \in I^2\) tel que \(c < d\)
  et \(f(c) > f(d)\). Soit la fonction
  \begin{equation}
    \fonction{\varphi}{\intervalleff{0}{1}}{\R}{t}{f(ta+(1-t)c)-f(tb+(1-t)d)}.
  \end{equation}
  La fonction \(\varphi\) est bien définie parce que pour tout \(t \in
  \intervalleff{0}{1}\) \(ta+(1-t)c \in \intervalleff{a}{c}\) et \(tb+(1-t)d \in
  \intervalleff{b}{d}\). \(I\) est un intervalle et \(a\), \(b\), \(c\) et \(d\)
  sont dans \(I\) donc \(\intervalleff{a}{c} \subset I\) et
  \(\intervalleff{b}{d} \subset I\). La fonction \(\varphi\) est continue
  puisque \(f\) est continue. On a \(\varphi(0)=f(c)-f(d) <0\) et
  \(\varphi(1)=f(a)-f(b) >0\). Par théorème des valeurs intermédiaires il existe
  un réel \(t_0 \in \intervalleoo{0}{1}\) tel que \(\varphi(t_0)=0\).
  C'est-à-dire
  \begin{equation}
    f(t_0a+(1-t_0)c)-f(t_0b+(1-t_0)d)=0,
  \end{equation}
  comme \(f\) est injective, cela signifie que \(t_0a+(1-t_0)c=t_0b+(1-t_0)d\)
  donc \(t_0(a-b)=(1-t_0)(b-c)\). Le premier membre est strictement négatif
  alors que le deuxième membre est strictement positif. C'est absurde, donc
  \(f\) est monotone.
\end{proof}
\begin{cor}
  Soient \(I\) et \(J\) deux intervalles réels, et soit \(\fonctionR{f}{I}{J}\).
  Il y a équivalence entre
  \begin{enumerate}
    \item \(f\) est une bijection et \(f\) est continue;
    \item \(f\) est une bijection et \(f\) est monotone.
  \end{enumerate}
  Une telle application qui vérifie l'une ou l'autre hypothèse est appelée un
  \emph{homéomorphisme}.
\end{cor}
\begin{proof}
  \begin{itemize}
    \item[\(1 \implies 2\)] C'est la conséquence du théorème précédent, \(f\)
      est continue et bijective donc \(f\) est continue et injective donc
      monotone.
    \item[\(2 \implies 1\)] La fonction \(f\) est monotone et \(f(I)=J\)
      (puisque \(f\) est bijective) donc \(f(I)\) est un intervalle alors par
      théorème~\ref{theo:monotone-intervalle-donc-continue}) l'application \(f\)
      est continue.
  \end{itemize}
\end{proof}

\subsection{Applications uniformément continues}

\subsubsection{Définition}

\begin{defdef}
  Soit \(I\) un intervalle réel et \(f\) une application continue de I dans
  \(\R\). On dit que \(f\) est uniformément continue sur \(I\), ou uniformément
  continue, si et seulement si
  \begin{equation}
    \forall \epsilon > 0 \ \exists \eta > 0 \ \forall (x,x') \in I^2 \quad
    \abs{x-x'} \leqslant \eta \implies \abs{f(x)-f(x')} \leqslant \epsilon
  \end{equation}
\end{defdef}
Si la fonction \(f\) est non uniformément continue cela équivaut à
\begin{equation}
  \exists \epsilon > 0 \ \forall \eta > 0 \ \exists (x,x') \in I^2 \quad
  \abs{x-x'} \leqslant \eta \text{~et~} \abs{f(x)-f(x')} > \epsilon
\end{equation}

\subsubsection{Propriétés}

\begin{theo}
  Soit \(f\) une application de I vers \(\R\). Si \(f\) est uniformément
  continue, alors elle est continue. La réciproque est fausse.
\end{theo}
\begin{proof}
  Pour tout \((x,x') \in I^2\) on note
  \begin{equation}
    C(x,x') \iff \abs{x-x'} \leqslant \eta \implies \abs{f(x)-f(x')} \leqslant
    \epsilon.
  \end{equation}
  Alors
  \begin{align}
    f \text{~est uniformément continue} &\iff \forall \epsilon > 0 \ \exists
    \eta_{\epsilon} > 0 \ \forall (x,x') \in I^2 \quad C(x,x')\\
    &\implies \forall \epsilon > 0 \ \forall x \in I \ \exists \eta_{\epsilon,
    x} > 0 \ \forall x' \in I \quad C(x,x')\\
    &\implies \forall x \in I \ \left[\forall \epsilon > 0 \ \exists
    \eta_{\epsilon, x} > 0 \ \forall x' \in I \quad C(x,x')\right]\\
  &\implies \forall x \in I \ f \text{~est continue en } x \end{align}
  Alors \(f\) est continue sur \(I\).
\end{proof}

\emph{Contre-exemple}~: Soit la fonction
\(\fonction{f}{\intervalleoo{0}{1}}{\R}{x}{\frac{1}{x}}\). Alors \(f\) est
continue, montrons qu'elle n'est pas uniformément continue. Soit \(\epsilon=1\),
montrons que
\begin{equation}
  \forall \eta > 0 \ \exists (x,x') \in \intervalleoo{0}{1}^2 \quad \abs{x-x'}
  \leqslant \eta \text{~et~} \abs{f(x')-f(x)} > 1.
\end{equation}
Soit \((x,x') \in \intervalleoo{0}{1}^2\) tel que \(x < x'\), alors
\begin{align}
  \abs{f(x)-f(x')} > 1 &\iff \abs{\frac{1}{x} - \frac{1}{x'}} > 1\\
  &\iff \abs{x'-x} > \abs{xx'} \\
  &\iff x'-x > xx'.
\end{align}
On a deux cas, soit \(1>\eta >0\) ou soit \(\eta>1\) alors
\begin{itemize}
  \item Pour tout \(1>\eta >0\), si on pose \(x'=\eta\) et \(x=\frac{\eta}{2}\)
    alors \(\abs{x'-x}= \frac{\eta}{2} \leqslant \eta\) et \(xx'= \eta
    \frac{\eta}{2} < \frac{\eta}{2}\). Avec ce \(x\) et ce \(x'\) on a \(x'-x >
    xx'\) ce qui est équivalent à \(\abs{f(x')-f(x)} > 1\);
  \item pour tout \(\eta \geqslant 1\), on pose \(\eta_0 \in
    \intervalleoo{0}{1}\) tel que \(\abs{x-x'} \leqslant \eta_0 \leqslant \eta\)
    et, comme dans le premier cas, \(\abs{f(x)-f(x')} > 1\).
\end{itemize}
La fonction \(f\) n'est pas uniformément continue. On dispose cependant du
théorème suivant.
\begin{theo}[Théorème de Heine]
  Soit \(f\) une application continue définie sur un \emph{segment}. Alors \(f\)
  est uniformément continue.
\end{theo}
\begin{proof}[Démonstration par l'absurde]
  Soient deux réels, \(a\) et \(b\) tels que \(a<b\). Supposons que \(f\) ne
  soit pas uniformément continue. Alors
  \begin{equation}
    \exists \epsilon > 0 \ \forall \eta > 0 \ \exists (x,y) \in
    \intervalleff{a}{b}^2 \quad \abs{x-y} \leqslant \eta \text{~et~}
    \abs{f(x)-f(y)} > \epsilon.
  \end{equation}
  En particulier
  \begin{equation}
    \exists \epsilon > 0 \ \forall n \in \N \ \exists (x_n,y_n) \in
    \intervalleff{a}{b}^2 \quad \abs{x_n-y_n} \leqslant \frac{1}{n+1}
    \text{~et~} \abs{f(x_n)-f(y_n)} > \epsilon.
  \end{equation}
  La suite \((x_n)\) est bornée (puisqu'elle est à valeurs dans
  \(\intervalleff{a}{b}\)) et on peut donc appliquer le théorème de
  Bolzano-Weiertrass. Il existe une application \(\varphi : \N \longrightarrow
  \N\) strictement croissante telle que la suite extraite \((x_{\varphi(n)})\)
  soit convergente de limite notée \(c\). On sait que
  \begin{equation}
    \forall n \in \N \quad a \leqslant x_{\varphi(n)} \leqslant b.
  \end{equation}
  Donc \(c \in \intervalleff{a}{b}\) et on sait que
  \begin{equation}
    \forall n \in \N \quad \abs{x_{\varphi(n)}-y_{\varphi(n)}} \leqslant
    \frac{1}{\varphi(n)+1} \leqslant \frac{1}{n+1}.
  \end{equation}
  Alors en passant à la limite \(\lim x_{\varphi(n)}-y_{\varphi(n)}\). La suite
  \((y_{\varphi(n)})\) tend aussi vers \(c\). On avait supposé que \(f\) était
  non uniformément continue donc
  \begin{equation}\label{eq:etoile}
    \forall n \in \N \quad \abs{f(x_{\varphi{n}})-f(y_{\varphi{n}})} > \epsilon.
  \end{equation}
  Les suites \((y_{\varphi(n)})\) et \((x_{\varphi(n)})\) convergent de même
  limite \(c\), \(f\) est continue sur \(\intervalleff{a}{b}\) par hypothèse
  donc particulièrement en \(c\). Par conséquent les suites
  \(f(x_{\varphi{n}})\) et \(f(y_{\varphi{n}})\) convergent vers \(f(c)\). On
  peut donc passer à la limite dans l'équation~\ref{eq:etoile} pour écrire que
  \(0 \geqslant \epsilon\).
  C'est absurde puisqu'on avait posé \(\epsilon >0\). Alors la fonction \(f\)
  est uniformément continue.
\end{proof}
\begin{prop}
  \begin{enumerate}
    \item Soient \(f\) et \(g\) deux fonctions uniformément continues de \(I\)
      vers \(\R\) et un réel \(\lambda\). Alors \(\lambda f +g\) est
      uniformément continue mais en général le produit \(fg\) n'est pas
      uniformément continu.
    \item Soient deux intervalles réels \(I\), \(J\) et deux fonctions \(f \in
      \R^I\) et \(g\in \R^J\) uniformément continues telles que \(f(I)\subset
      J\). Alors la fonction \(f \circ g\) est uniformément continue sur \(I\).
  \end{enumerate}
\end{prop}
\begin{proof}
  \begin{enumerate}
    \item Soit \(\epsilon >0\), il existe alors deux réels \(\eta_1\) et
      \(\eta_2\) tels que pour tout \((x,x') \in I^2\) on ait
      \begin{align}
        \abs{x-x'} \leqslant \eta_1 &\implies \abs{f(x)-f(x')} \leqslant
        \frac{\epsilon}{2(\abs{\lambda}+1)},\\
        \abs{x-x'} \leqslant \eta_2 &\implies \abs{g(x)-g(x')} \leqslant
        \frac{\epsilon}{2}.
      \end{align}
      Soit \(\eta=\min(\eta_1,\eta_2)\) et donc si \(\abs{x'-x} \leqslant \eta\)
      alors \begin{align}
        \abs{(\lambda f+g)(x)-(\lambda f+g)(x')} &\leqslant \abs{\lambda}
        \abs{f(x)-f(x')} + \abs{g(x)-g(x')} \\
        &\leqslant \frac{\abs{\lambda}}{\abs{\lambda} +1}
        \frac{\epsilon}{2}+\frac{\epsilon}{2} \leqslant \epsilon.
      \end{align}
      Alors \(\lambda f+g\) est uniformément continue.
    \item Puisque \(g\) est uniformément continue, pour tout \(\epsilon >0\) il
      existe \(\eta >0\) tels que pour tout \((y,y') \in J^2\) si \(\abs{y-y'}
      \leqslant \eta\) alors \(\abs{g(y)-g(y')} \leqslant \epsilon\). Comme
      \(f\) est aussi uniformément continue il existe \(\alpha >0\) tel que pour
      tout \((x,x') \in I^2\) si \(\abs{x-x'} \leqslant \alpha\) alors
      \(\abs{f(x)-f(x')} \leqslant \eta\). Comme \(f(x)\) et \(f(x')\) sont dans
      \(f(I)\), ils sont dans \(J\) et \(\abs{f(x)-f(x')} \leqslant \eta\).
      Alors en appliquant \(g\) on a \(\abs{g \circ f(x)-g \circ f(x')}
      \leqslant \epsilon\). Alors la composée \(g \circ f\) est uniformément
      continue.
  \end{enumerate}
\end{proof}
Attention, le produit \(fg\) n'est pas forcément uniformément continu.

\subsection{Applications lipschitziennes}

\subsubsection{Définitions}

\begin{defdef}
  Soit \(f \in \R^I\) et un réel \(k \geqslant 0\). On dit que \(f\) est
  \(k-\)lipschitzienne si et seulement si
  \begin{equation}
    \forall (x,x')^2 \in I \quad \abs{f(x)-f(x')} \leqslant k \abs{x-x'}.
  \end{equation}
  On dit que \(f\) est lipschitzienne s'il existe un réel \(k \geqslant 0\) tel
  que \(f\) soit \(k-\)lipschitzienne.
\end{defdef}
Si \(f\) est \(k-\)lipschitzienne, alors \(f\) est \(k'-\)lipschitzienne pour
tout \(k' \geqslant k\). Lorsque \(k \in \intervalleoo{0}{1}\), on dit que \(f\)
est \(k-\)contractante.
%
\begin{theo}
  Toute fonction lipschitzienne est uniformément continue. La réciproque est
  fausse.
\end{theo}
\begin{proof}
  Soient \(k >0\) et \(f\) une application \(k-\)lipschitzienne, alors pour tout
  \((x,x') \in I^2\), \(f(x)-f(x') \leqslant k\abs{x-x'}\). Soit \(\epsilon >0\)
  il existe alors \(\eta = \frac{\epsilon}{k+1}\) tel que si \(\abs{x-x'}
  \leqslant \eta\) alors \begin{equation}
    \abs{f(x)-f(x')} \leqslant k\abs{x-x'} \leqslant \frac{k}{k+1} \epsilon
    \leqslant \epsilon.
  \end{equation}
  Donc \(f\) est uniformément continue.
\end{proof}
La réciproque est fausse. On en donne un contre-exemple. Soit
\(I=\intervalleff{0}{1}\) et \(\fonction{f}{I}{\R}{x}{\sqrt{x}}\). Alors \(f\)
est continue sur le segment \(I\). D'après le théorème de Heine, elle est
uniformément continue. Montrons qu'elle n'est pas lipschitzienne. Supposons par
l'absurde qu'elle le soit, alors il existe un \(k >0\) tel que pour tout
\((x,x') \in I^2\), \(\abs{\sqrt{x}-\sqrt{x'}} \leqslant k\abs{x-x'}\). Soit
alors en prenant \(x'=0\) on a \(\sqrt{x} \leqslant k x\). Alors pour tout \(x
\in \intervalleof{0}{1}\) on aurait \(\frac{\sqrt{x}}{x}=\frac{1}{\sqrt{x}}
\leqslant k\). Par passage à la limite en \(0\) on aurait \(k \geqslant +
\infty\). Ce qui est complètement absurde. Alors \(f\) n'est pas lipschitzienne.

Si on note \(\Lip{I}{\R}\) l'ensemble des applications lipschitziennes de \(I\)
dans \(\R\) et \(\mu\cont{I}{\R}\) l'ensemble des applications uniformément
continues de I dans \(\R\). Alors on a montré que
\begin{equation}
  \Lip{I}{\R} \varsubsetneq \mu\cont{I}{\R} \varsubsetneq \cont{I}{\R}.
\end{equation}

\begin{prop}
  Soient \(I\) un intervalle réel, deux réels \(k\) et \(k'\) positifs, un autre
  réel \(\lambda\) et deux applications \(f\in \R^I\) \(k-\)lipschitzienne et
  \(g\in \R^I\) \(k'-\)lipschitzienne. Alors
  \begin{enumerate}
    \item \(\lambda f+g\) est \(\abs{\lambda}k+k'\)-lipschitzienne;
    \item si on suppose que \(f(I) \subset J\) alors \(g \circ f\) est
      \(kk'-\)lipschitzienne.
  \end{enumerate}
\end{prop}
\begin{proof}
  \begin{enumerate}
    \item Soient \((x,x') \in I^2\) alors
      \begin{align}
        \abs{(\lambda f+g)(x)-(\lambda f+g)(x')} &\leqslant \abs{\lambda}
        \abs{f(x)-f(x')}+\abs{g(x)-g(x')}\\
        &\leqslant (\abs{\lambda}k+k')\abs{x-x'},
      \end{align}
      alors \(\lambda f+g\) est \(\abs{\lambda}k+k'\)-lipschitzienne.
    \item Soient \((x,x') \in I^2\) alors
      \begin{equation}
        \abs{g \circ f(x)-g \circ f(x')} \leqslant k' \abs{f(x)-f(x')} \leqslant
        kk' \abs{x-x'},
      \end{equation}
      alors \(g \circ f\) est \(kk'-\)lipschitzienne
  \end{enumerate}
\end{proof}

\section[Brève extension aux fonctions complexes]{Brève extension aux fonctions
à valeurs complexes}

\subsection{Notion de fonction à valeurs complexes}

Soit \(X\) une partie de \(\R\) qui contient au moins deux éléments. On note
\(\C^X\) l'ensemble des fonctions définies sur \(X\) à valeurs dans \(\C\).

\begin{defdef}
  Soit \(f \in \C^X\), on lui associe les fonctions suivantes~:
  \begin{enumerate}
    \item La partie réelle de \(f\), \(\fonction{\Re(f)}{X}{\R}{x}{\Re(f(x))}\);
    \item La partie imaginaire de \(f\),
      \(\fonction{\Im(f)}{X}{\R}{x}{\Im(f(x))}\);
    \item Le module de \(f\), \(\fonction{\abs{f}}{X}{\R}{x}{\abs{f(x)}}\);
    \item Le conjugué de \(f\), \(\fonction{\bar{f}}{X}{\R}{x}{\bar{f(x)}}\).
  \end{enumerate}
\end{defdef}
\begin{defdef}
  Soit \(f \in \C^X\). On dit que \(f\) est bornée lorsque \(\abs{f}\) est
  bornée.
\end{defdef}
\begin{prop}
  Soit \(f \in \C^X\), alors \(f\) est bornée si et seulement si \(\Re(f)\) et
  \(\Im(f)\) sont bornées.
\end{prop}
\begin{prop}
  Soient \(f\) et \(g\) de \(\C^X\) et deux complexes \(\lambda\) et \(\mu\). Si
  les fonctions \(f\) et \(g\) sont bornées alors \(\lambda f+\mu g\) est bornée
  et \(fg\) aussi.
\end{prop}

\subsection{Limite et continuité en un point}

\subsubsection{Limites}

\begin{defdef}
  Soit \(f \in \C^X\) et \(a \in \Rbar\). On dit que la fonction \(f\) tend vers
  le complexe \(\ell\) en \(a\) lorsque la fonction réelle \(\abs{f-l}\) tend
  vers zéro en \(a\).
\end{defdef}
\begin{prop}
  Soient \(f \in \C^X\), \(a \in \Rbar\) et \(l \in \C\). Alors
  \begin{equation}
    \lim\limits_{a}f = l \iff \lim\limits_{a}\Re(f) = \Re(l) \wedge
    \lim\limits_{a}\Im(f) = \Im(l).
  \end{equation}
\end{prop}
\begin{proof}
  \begin{itemize}
    \item[\(\implies\)] Soit \(x \in X\), alors
      \(\abs{\Re(f)x-\Re(l)}=\abs{\Re(f(x)-l)} \leqslant \abs{f(x)-l}\) par
      passage à la limite, on a \(\lim\limits_{a}\Re(f) = \Re(l)\) et de la même
      manière \(\abs{\Im(f)(x)-\Im(l)}=\abs{\Im(f(x)-l)} \leqslant
      \abs{f(x)-l}\) par passage à la limite, on a \(\lim\limits_{a}\Im(f) =
      \Im(l)\).
    \item[\(\impliedby\)] Si \(\Re(f)\) tend vers \(\alpha\) en \(a\) et si
      \(\Im(f)\) tend vers \(\beta\) en \(a\) alors pour tout \(x \in X\) on a
      par inégalité triangulaire
      \begin{equation}
        \abs{f(x)-(\alpha+\ii \beta)} \leqslant \abs{\Re(f)(x)-\alpha} +
        \abs{\Im(f)(x)-\beta},
      \end{equation}
      alors en passant à la limite on montre que \(f\) tend vers \(\alpha+\ii
      \beta\) en \(a\).
  \end{itemize}
\end{proof}
\begin{cor}[Unicité de la limite]
  Soient \(f \in \C^X\) et \(a \in \Rbar\). Il existe au plus un complexe
  \(\ell\) tel que \(f\) tende vers \(\ell\) en \(a\). Si un tel \(\ell\)
  existe, il est la limite de \(f\) en \(a\) et on note \(\lim\limits_{x \to a}
  f(x)=\lim\limits_{a}f = \ell\).
\end{cor}
\begin{proof}
  D'après la proposition, \(\Re(\ell)\) et \(\Im(\ell)\) sont uniques, donc
  \(\ell\) est unique.
\end{proof}
\begin{prop}
  Soient \(f \in \C^X\), \(a \in \Rbar\) et \(\ell \in \C\), si
  \(\ell=\lim\limits_{a}f\) alors
  \begin{itemize}
    \item \(\lim\limits_{a} \abs{f}=\abs{\ell}\);
    \item \(\lim\limits_{a} \Re(f)=\Re(\ell)\);
    \item \(\lim\limits_{a} \Im(f)=\Im(\ell)\);
    \item \(\lim\limits_{a} \bar{f}=\bar{\ell}\).
  \end{itemize}
\end{prop}
\begin{proof}
  On a déjà vu la preuve pour la partie réelle et la partie imaginaire. Soit \(x
  \in X\), alors \(\abs{\bar{f}(x)-\bar{\ell}} = \abs{f(x)-\ell} \rightarrow 0\)
  et \(\abs{\abs{f}(x)-\abs{\ell}} \leqslant \abs{f(x)-\ell} \rightarrow 0\).
\end{proof}
\begin{prop}
  Soient \(f \in \C^X\) et \(a \in \Rbar\). Si \(f\) admet une limite en \(a\),
  alors \(f\) est bornée au voisinage de \(a\).
\end{prop}
\begin{proof}
  Si \(f\) tend vers \(\ell\) en \(a\), alors \(\Im(f)\) et \(\Re(f)\) admettent
  des limites finies en \(a\) et donc elles sont bornées au voisinage de \(a\).
  Ainsi \(f\) est bornée au voisinage de \(a\).
\end{proof}
\subsubsection{Opérations algébriques sur les limites}
cf polycopié
\subsubsection{Continuité en un point}
\begin{defdef}
  Soit \(f \in \C^X\) et \(a \in X\). On dit que \(f\) est continue en \(a\) si
  et seulement si \(f\) admet une limite finie en \(a\).
\end{defdef}
\begin{prop}
Soit \(f \in \C^X\) et \(a \in X\). Alors \(f\) est continue en \(a\) si et
seulement si \(\Im(f)\) et \(\Re(f)\) sont continues en \(a\). \end{prop}
\begin{proof}
  \(f\) est continue en \(a\) si et seulement si \(f\) admet une limite en
  \(a\). C'est équivalent à ce que \(\Im(f)\) et \(\Re(f)\) aient des limites
  finies en \(a\), c'est-à-dire si et seulement si \(\Im(f)\) et \(\Re(f)\) sont
  continues en \(a\).
\end{proof}
\subsection{Continuité sur un intervalle}
Soit \(I\) un intervalle réel qui contient au moins deux points.
\begin{defdef}
Soit \(f \in \C^I\). On dit que \(f\) est continue, ou continue sur I si elle
est continue en tout point de I. On note \(\cont{I}{\C}\) l'ensemble des
fonctions continues de \(I\) vers \(\C\). \end{defdef}
\begin{prop}
  Soit \(f \in \C^I\), alors \begin{equation}
    f \in \cont{I}{\C} \iff \Re(f) \in \cont{I}{\R} \text{~et~}\Im(f) \in
    \cont{I}{\R}
  \end{equation}
\end{prop}
\begin{prop}
  Soit \(f \in \cont{I}{\C}\), alors \(\bar{f} \in \cont{I}{\C}\) et \(\abs{f}
  \in \cont{I}{\R}\).
\end{prop}
\begin{prop}
  L'ensemble \(\cont{I}{\C}\) est un sous-espace vectoriel et un sous-anneau de
  \(\C^I\).
\end{prop}
\begin{theo}
  Soient \(I\) et \(J\) deux intervalles réels et \(f \in \R^I\) et \(g \in
  \C^J\) tel que \(f(I) \subset J\). Si \(f\) est continue et que \(g\) est
  continue alors \(f \circ g\) est continue. Attention, \(f\) doit être à
  valeurs réelles.
\end{theo}
\begin{theo}
  Soient deux réels \(a\) et \(b\) tels que \(a < b\) et \(f:\intervalleff{a}{b}
  \longrightarrow \C\). Si \(f\) est continue sur \(\intervalleff{a}{b}\) alors
  \(f\) est bornée et il existe \(x_0 \in \intervalleff{a}{b}\) tel que
  \(\abs{f(x_0)} = \sup\limits_{\intervalleff{a}{b}}\abs{f}\).
\end{theo}
\begin{proof}
  La fonction \(\abs{f}\) est continue sur \(\intervalleff{a}{b}\). On lui
  applique le théorème des bornes et on dit qu'elle est bornée et atteint ses
  bornes.
\end{proof}

