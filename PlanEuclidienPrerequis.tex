\chapter{Plan géométrique euclidien -- Prérequis}

L'objet de cette anexxe est de rappeler des résultats et des notions supposés connus à l'issue de la classe de terminale scientifique, mais avec une formalisation nouvelle. Le plan géométrique euclidien $\P$ est un ensemble dont les éléments son appelés points et notés en lettres majusucules. On lui associe une ensemble noté $\vect{\P}$ et appelé plan vectoriel euclidien. Ses éléments sont appelés vecteurs et notés en lettres minuscules coiffés d'une flèche. On appelle scalaire les éléments de l'ensemble $\R$ des nombres réels.

\section{Plan vectoriel $\protect\vect{\P}$}
%\section{Plan vectoriel $\P$}

Le plan vectoriel $\vect{\P}$ est muni de deux lois~:
\begin{itemize}
\item la loi d'addition vectorielle, qui aux vecteurs $\vu$ et $\vv$ fait correspondre le vecteur noté $\vu +\vv$;
\item la loi de multiplication par un scalaire, qui au scalaire $\lambda$ et au vecteur $\vu$ fait correspondre le vecteur $\lambda \vu$.
\end{itemize}

Il possède de plus un élément appelé vecteur nul et noté $\vec{0}$.

\begin{prop}
  $(\vect{\P}, +, \cdot)$ est un $\R$-espace vectoriel, ce qui signifie que pour tout $\vu,\vv,\vw \in \vect{\P}$ tout scalaires $\lambda,\mu$ on a~:
  \begin{enumerate}
  \item $(\vu+\vv)+\vw=\vu +(\vv+\vw)$, la loi $+$ est associative;
  \item $\vu+\vv=\vv+\vu$, la loi $+$ est commutative;
  \item $\vu+\vect{0}=\vu$, le vecteur nul est neutre pour l'addition;
  \item il existe un unique vecteur $\vu' \in \vect{\P}$ tel que $\vu+\vu'=\vect{0}$, c'est l'opposé $\vu'=-\vu$;
  \item $1 \vu=\vu$;
  \item $\lambda(\mu \vu)=(\lambda \mu) \vu$;
  \item $(\lambda+\mu) \vu = \lambda\vu +\mu\vu$;
  \item $\lambda(\vu+\vv)=\lambda \vu +\lambda\vv$, c'est la distributivité de $\cdot$ sur $+$;
  \item $0 \vu = \lambda \vect{0}=\vect{0}$.
  \end{enumerate}
\end{prop}

\begin{defdef}
  Soient deux vecteurs $\vu$ et $\vv$ de $\vect{\P}$.
  \begin{enumerate}
  \item On dit que $\vu$ et $\vv$ sont colinéaires si~:
    \begin{equation}
      \vu = \vect{0} \text{ ou } \exists \lambda \in \R \ \vv=\lambda\vu
    \end{equation}
  \item On dit que $\vu$ et $\vv$ sont colinéaires et de même sens si~:
    \begin{equation}
      \vu = \vect{0} \text{ ou } \exists \lambda \in \R_+ \ \vv=\lambda\vu
    \end{equation}
  \end{enumerate}
\end{defdef}

\begin{prop}
  Il existe dans $\vect{\P}$ des couples $(\vu,\vv)$ de vecteurs non colinéaires. De plus, si $(\vu,\vv)$ est un tel couple, alors c'est une base de $\vect{\P}$, c'est-à-dire que pour tout vecteur $\vw \in \vect{\P}$ il existe un unique couple de scalaires $(\lambda, \mu)$ tel que
  \begin{equation}
    \vw=\lambda \vu + \mu \vv
  \end{equation}
\end{prop}

\begin{defdef}
  Le couple $(\lambda,\mu)$ est alors appelé coordonnées du vecteur $\vw$ dans la base $(\vu,\vv)$.
\end{defdef}

Ce résultat caractérise le fait que $\vect{\P}$ est un espace vectoriel de dimension deux, c'est-à-dire un plan (cf. chapitre \ref{chap:espaceVectoriel}).

\section{Plan affine $\P$}

Le plan affine $\P$ est muni d'une loi qui aux points $A$ et $B$ fait correspondre le vecteur $\vect{AB}$.

\begin{prop}
  Soient trois points de $\P$, $A$, $B$ et $C$ et alors~:
  \begin{enumerate}
  \item $\vect{AB}+\vect{BC}=\vect{AC}$, c'est la relation de Chasles;
  \item $\vect{AA}=\vect{0}$;
  \item $\vect{AB}=-\vect{BA}$;
  \item pour tout vecteur $\vu \in \vect{\P}$, il existe un unique point $A' \in \P$ tel que $\vect{AA'}=\vu$. On note alors $A'=A+\vu$.
  \end{enumerate}
\end{prop}

\begin{defdef}
  \begin{enumerate}
  \item On dit que trois points $A, B$ et $C$ sont alignés si les vecteurs $\vect{AB}$ et $\vect{BC}$ sont colinéaires.
  \item On dit que trois points $A, B$ et $C$ sont alignés dans cet ordre si les vecteurs $\vect{AB}$ et $\vect{BC}$ sont colinéaires et de même sens.
  \end{enumerate}
\end{defdef}

\begin{defdef}
  On appelle point pondéré tout couple $(A,\lambda)$ où $A \in \P$ et $\lambda \in \R$.
\end{defdef}

\begin{prop}
  Soient un naturel $n \geq 2$ et $\{(A_k,\lambda_k)_{k \in \intervalleentier{1}{n}}\}$ une famille de $n$ points pondérés telle que $\sum_{i=1}^n \lambda_i \neq 0$. Alors il existe un unique point $G \in \P$ tel que
\begin{equation}
  \sum_{i=1}^n \lambda_i \vect{GA_i}=0
\end{equation}
Il est appelé barycentre du système de points pondérés $\{(A_k,\lambda_k)_{k \in \intervalleentier{1}{n}}\}$ et on note
\begin{equation}
  G = \Bary\{(A_k,\lambda_k)_{k \in \intervalleentier{1}{n}}\}
\end{equation}
On a alors pour tout point $M \in \P$
\begin{equation}
  \sum_{i=1}^n \lambda_i \vect{MA_i} = \sum_{i=1}^n \lambda_i \vect{MG}
\end{equation}
\end{prop}
\begin{prop}
  Soient $A,B, A_1, \ldots, A_p,B_1, \ldots, B_q$ des points de $\P$ et $\lambda$, $\mu$, $\lambda_1$, \ldots, $\lambda_p$, et $\mu_1$, \ldots, $\mu_q$ des réels. En notant $G = \Bary\{(A_k,\lambda_k)_{k \in \intervalleentier{1}{p}}\}$, $\alpha=\sum_{i=1}^p \lambda_i$, $H = \Bary\{(B_l,\mu_l)_{l \in \intervalleentier{1}{q}}\}$, et $\beta=\sum_{i=1}^q \mu_i$ on a
  \begin{enumerate}
  \item L'associativité du barycentre, c'est à dire que
    \begin{equation}
      \Bary\{(G,\alpha),(H,\beta)\} = \Bary\{(A_k,\lambda_k)_{k \in \intervalleentier{1}{p}}, (B_l,\mu_l)_{l \in \intervalleentier{1}{q}}\}
    \end{equation}
  \item La commutativité du barycentre
    \begin{equation}
      \Bary\{(A,\lambda),(B,\mu)\}=\Bary\{(B,\mu),(A,\lambda)\}
    \end{equation}
  \item
    \begin{equation}
      \Bary\{(A,\lambda),(A,\lambda)\}=A
    \end{equation}
  \end{enumerate}
\end{prop}

\section{Distances et normes}

Le plan vectoriel $\vect{\P}$ est muni d'une loi qui au vecteur $\vu$ associe le réel noté $||\vu||$ appelé sa norme ou norme euclidienne.
\begin{prop}
  Soient $\vu$ et $\vv$ deux vecteurs de $\vec{E}$ et $\lambda \in \R$, on a les propriétés suivantes
  \begin{enumerate}
  \item positivité
    \begin{equation}
      ||\vu|| \geq 0
    \end{equation}
  \item séparation
    \begin{equation}
      ||\vu|| = 0 \Rightarrow \vu = \vec{0}
    \end{equation}
  \item homogénéité
    \begin{equation}
      ||\lambda \vu||=|\lambda| \cdot ||\vu||
    \end{equation}
  \item inégalité triangulaire
    \begin{equation}
      ||\vu+\vv|| \leq ||\vu|| + ||\vv||
    \end{equation}
    égalité si et seulement s'ils sont colinéaires et de même sens
  \end{enumerate}
\end{prop}
Le plan affine $\P$ est muni d'une loi qui aux points $A$ et $B$ fait correspondre le réel $||\vect{AB}||$, noté $AB$ et appelé distance de $A$ à $B$.
\begin{prop}
  Soient $A$, $B$ et $C$ trois points de $\P$ on a les propriétés suivantes
  \begin{enumerate}
  \item positivité
    \begin{equation}
      AB \geq 0
    \end{equation}
  \item séparation
    \begin{equation}
      AB = 0 \Rightarrow A=B
    \end{equation}
  \item symétrie
    \begin{equation}
      AB=BA
    \end{equation}
  \item inégalité triangulaire
    \begin{equation}
      AC \leq AB + BC
    \end{equation}
    égalité si et seulement si $A$ $B$ et $C$ sont alignés dans cet ordre
  \end{enumerate}
\end{prop}

\section{Angles non orientés, orthogonalité}

Le plan vectoriel $\vect{\P}$ est muni d'une loi qui aux vecteurs non nuls $\vu$ et $\vv$ associe le réel noté $\widehat{(\vu,\vv)}$, appartenant à $[0, \pi]$, appelé mesure de l'angle non orienté entre les vecteurs $\vu$ et $\vv$.
\begin{prop}
  Soient $\vu$, $\vv$ et $\vw$ des vecteurs non nuls de $\vect{\P}$ et un réel non nul $\lambda$, on a
  \begin{enumerate}
  \item $\widehat{(\vu,\vw)} \leq \widehat{(\vu,\vv)} + \widehat{(\vv,\vw)}$
  \item $\widehat{(\vu,\vu)} = 0$
  \item $\widehat{(\vu,\vv)} = \widehat{(\vv,\vu)}$
  \item Si $\lambda>0$ $\widehat{(\vu,\lambda \vv)}= \widehat{(\vu,\vv)}$ et si $\lambda < 0$ alors $\widehat{(\vu,\lambda \vv)}= -\widehat{(\vu,\vv)}$
  \end{enumerate}
\end{prop}

Deux vecteurs $\vu$ et $\vv$ sont donc colinéaires si et seulement si $\vu$ ou $\vv$ est nul ou si $\widehat{(\vu,\vv)}=0$ ou $\pi$.
\begin{defdef}
  On dit que deux vecteurs $\vu$ et $\vv$ sont orthogonaux si et seulement si $\vu$ est nul ou si $\vv$ est nul ou encore si $\widehat{(\vu,\vv)} = \frac{\pi}{2}$.
\end{defdef}

Étant donné trois points $A$, $B$ et $C$ de $\E$ tels que $A \neq B$ et $C \neq B$, on appelle mesure de l'angle non orienté $\widehat{ABC}$ la mesure de l'angle non orienté $\widehat{(\vect{BA},\vect{BC})}$.

\begin{prop}
  Pour tout triangle $ABC$ rectangle en $B$, on a
  \begin{enumerate}
  \item Le théorème de Pythagore : $AB^2+BC^2=AC^2$;
  \item $AB=AC\cos(\widehat{ABC})$;
  \item $BC=AC\sin(\widehat{ABC})$;
  \item $BC=AB\tan(\widehat{ABC})$.
  \end{enumerate}
\end{prop}

\section{Bases et repères}

Un couple $(\vu,\vv)$ de vecteurs de $\vect{\P}$ est une base de $\vec{E}$ si et seulement si $\vu$, $\vv$ sont non coplinéaires. 

\begin{defdef}
  On appelle repère de l'espace $\P$ tout triplet $(O,\vu,\vv)$ où $O$ est un point quelconque de l'espace $\P$ et $(\vu,\vv)$ une base de $\vect{\P}$.

Si $M$ est un point de $\P$, on appelle coordonnées du point $M$ dans le repère $(O,\vu,\vv)$ l'unique couple de réels $(x,y)$ tel que
\begin{equation}
  \vect{OM}=x\vu+y\vv
\end{equation}
\end{defdef}

Soient $(\vu,\vv)$ une base de $\vect{\P}$ et $\theta$ la mesure principale de l'angle orineté $\widehat{(\vu,\vv)}$. Par définition, $\theta \in ]-\pi, \pi]$ et comme $\vu$ et $\vv$ ne sont pas colinéaires, $\theta \neq 0$. Si $\theta >0$ on dit que la base est directe sinon on dit qu'elle est indirecte.

\begin{defdef}
  On dit qu'un repère $(O,\vu,\vv)$ est direct (resp. indirect) si la base $(\vu,\vv)$ est directe (resp. indirecte).
\end{defdef}

\begin{defdef}
  On dit que $(\vu,\vv)$ est une base orthonormée si les vecteurs $\vu$ et $\vv$ sont des vecteurs orthogonaux et de norme 1. Le repère $(O,\vu,\vv)$ est orthonormé si la base $(\vu,\vv)$ est orthonormée.
\end{defdef}

\section{Figures usuelles}

Soient $A$ et $B$ deux points distincts du plan $\P$ et $r \in \R_+$.
\begin{itemize}
\item La droite $(AB)$ est l'ensemble des points $M \in \P$ tels que $A, B$ et $M$ sont alignés. Le vecteur $\vect{AB}$ est un vecteur directeur de la droite $(AB)$.
\item Le segment $[AB]$ est l'ensemble des points $M \in \P$ tels que $A, B$ et $M$ sont alignés dans cet ordre.
\item Le cercle de centre $A$ et de rayon $r$ est l'ensemble des points $M \in \P$ tels que $AM=r$.
\end{itemize}
Soient deux droites $\Dr$ et $\Dr'$ du plan $\P$.
\begin{itemize}
\item On dit que $\Dr$ et $\Dr'$ sont orthogonales si elles ont des vecteurs directeurs orthogonaux.
\item On dit que $\Dr$ et $\Dr'$ sont parallèles si elles ont des vecteurs directeurs colinéaires.
\end{itemize}
