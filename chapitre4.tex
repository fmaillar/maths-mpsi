\chapter{Équations différentielles linéaires}
\label{chap:equadiff}
\minitoc
\minilof
\minilot

Ce chapitre présente les équations différentielles linéaires du premier et du deuxième ordre. Les équations différentielles apparaissent dans toutes les branches de la physique. Elles sont en général la description d'un phénomène variant dans l'espace ou dans le temps. Les équations différentielles que nous étudieront là ne sont qu'une petite partie de celle qui existent et qui sont pour la plupart non linéaires, comme par exemple les équations de Navier Stokes.

On considère des fonctions à valeurs dans un corps \(\K\). Soit un entier naturel non nul \(n\), \(I\) un intervalle réel et \(a_0, \ldots a_n\) et \(b\) des fonctions de \(I\) dans \(\K\) au moins continues. On considère l'équation différentielle
\begin{equation}
\sum_{i=0}^n a_i(x) y^{(i)} = b(x) \label{eq:eqdiff}.
\end{equation}
On peut alors définir ce qu'on appelle des \emph{solutions} de cette équation différentielle.
\begin{defdef}
Soit \(f : I \longrightarrow \K\). On dira que \(f\) est une solution de l'équation différentielle~\eqref{eq:eqdiff} si et seulement si :
\begin{enumerate}
\item \(f\) est au moins \(n\) fois dérivable,
\item \(\forall x \in I \sum_{i=0}^n a_i(x) y^{(i)} = b(x)\).
\end{enumerate}
\end{defdef}
En pratique on traitera dans ce cours du cas où \(n=1\) de façon complète et du cas \(n=2\) dans le cas où \(a_0\), \(a_1\) et \(a_2\) sont des constantes. On recherchera des solutions qui vérifient des conditions initiales de la forme 
\begin{equation}
 \forall i \in \intervalleentier{0}{n-1} \quad y^{(i)}(t_0)=y_i,
\end{equation}
où \(t_0 \in I\)  et les \(y_i\) sont des scalaires fixes dans \(\K\).
\section{Équation différentielle linéaire homogène du premier ordre}
\label{sec:equadifflinhomog1}
\subsection[Solution équation homogène coefficients constants]{Solution d'une équation différentielle linéaire homogène du premier ordre à coefficients constants}
\label{subsec:solutioneqdifflinhomog1coefconstants}
Soit un scalaire \(a \in \K\). On s'intéresse à l'équation 
\begin{equation}
\label{eq:Hdiff}
y'+ay=0.
\end{equation}
%
\begin{theo}
\label{theo:1}
Soit une application \(f \in \K^{\R}\). \(f\) est solution de l'équation \eqref{eq:Hdiff} si et seulement s'il existe un scalaire \(\lambda \in \K\) tel que pour tous réel \(t\), \(f(t)=\lambda \e^{-at}\).
\end{theo}
On illustre ce théorème par la figure \ref{fig:eqdiff1}.
\begin{proof}
Soit \(\lambda \in \K\) tel que , \(\fonction{f}{\R}{\K}{t}{\lambda \e^{-at}}\). \(f\) est dérivable et
\begin{equation}
\forall t \in \R \quad f'(t)+af(t)=-a \lambda \e^{-at}+a \lambda \e^{-at}=0,
\end{equation}
donc la fonction \(f\) est solution de l'équation \eqref{eq:Hdiff} sur \(\R\). 

Soit maintenant une fonction \(f\) solution de l'équation \eqref{eq:Hdiff} sur \(\R\). Posons la fonction \(\fonction{g}{\R}{\K}{t}{f(t)\e^{at}}\) \(g\) est dérivable par produit de fonctions qui le sont et
\begin{equation}
\forall t \in \R \quad g'(t)=f'(t)\e^{at} + af(t)\e^{at}=\e^{at}(f'(t)+af(t))=0,
\end{equation}
la fonction \(g\) est donc constante sur \(\R\). Il existe donc un réel \(\lambda\) tel que \(\forall t \in \R \quad g(t)=\lambda\). Par conséquent \(f(t)=g(t)\e^{-at}=\lambda \e^{-at}\).
\end{proof}
\begin{figure}[h]
    \centering
    \includegraphics[scale=0.7]{Equa-diff.png}
    \caption{Exemple de représentation graphique de solution de l'équation différentielle \(y' = -y/4\)}
    \label{fig:eqdiff1}
\end{figure}
\subsection[Solution équation homogène coefficients non constants]{Solution d'une équation différentielle linéaire homogène du premier ordre à coefficients non constants}
Soient maintenant \(I\) un intervalle réel, \(a\) une fonction de \(I\) vers \(\K\) au moins continue sur \(I\). On considère l'équation
\begin{equation}
y'+a(t)y=0 \label{eq:Hdiff1}.
\end{equation}
Pour le cas où \(\K=\R\), on suppose qu'il existe une solution \(f:I \longrightarrow \R\) de l'équation \eqref{eq:Hdiff1} \emph{qui ne s'annule pas sur \(I\)} telle que
\begin{equation}
\forall t \in I \quad \frac{f'(t)}{f(t)}=-a(t).
\end{equation}
La fonction \(a\) est continue sur \(I\), donc elle admet une primitive \(A\) sur \(I\). Par intégration il existe une constante \(K \in \R\) telle que pour tout \(t \in I\)~:
\begin{equation}
\ln\abs{f(t)}=-A(t) +K \iff \abs{f(t)}=\e^K \e^{-A(t)}.
\end{equation}
La fonction \(f\) est continue, car dérivable, et ne s'annule pas sur l'intervalle \(I\) donc elle est de signe constant sur \(I\). Dans les deux cas on pose \(\lambda=\pm \e^K\) et on a \(f(t)=\lambda e^{-A(t)}\). 

Revenons maintenant au cas général où \(\K \in \{\R, \C\}\).
%
\begin{theo}\label{theo:2}
Soit \(A\) une primitive de la fonction \(a\) continue sur \(I\). Soit \(f:I \longrightarrow \K\), \(f\) est solution de l'équation \(\H\) si est seulement s'il existe un scalaire \(\lambda\) tel que \(\forall t \in I \quad f(t)=\lambda \e^{-A(t)}\)
\end{theo}
\begin{proof}
Soit \(\lambda \in \K\) et \(\fonction{f}{I}{\R}{t}{\lambda \e^{-A(t)}}\). \(A\) est dérivable sur \(I\) (puisque c'est une primitive de \(a\)) et l'exponentielle l'est aussi donc \(f\) est dérivable par composition.
\begin{equation}
\forall t \in I \quad f'(t)+a(t)f(t)=\lambda(-A'(t)\e^{-A(t)}) + \lambda a(t)\e^{-A(t)}=0,
\end{equation}
\(f\) est donc solution de l'équation \eqref{eq:Hdiff1}.

Soit maintenant une fonction \(f\) solution de l'équation \eqref{eq:Hdiff1} sur \(I\). On définit alors la fonction \(\fonction{g}{I}{\K}{t}{f(t)\e^{A(t)}}\). La fonction \(f\) est dérivable puisque c'est une solution de l'équation différentielle~\eqref{eq:Hdiff1}. La fonction  \(A\) est aussi dérivable, l'exponentielle est dérivable donc \(g\) est dérivable par composition et produit. Alors
\begin{equation}
\forall t \in I \quad g'(t)=f'(t) \e^{A(t)} + f(t)A'(t)\e^{A(t)}=\e^{A(t)}(f'(t) + a(t)f(t)) =0.
\end{equation}
Il existe donc un scalaire \(\lambda\) tel que \(\forall t \in I \quad g(t)=\lambda\). D'où 
\begin{equation} 
\forall t \in I \quad f(t)=g(t) \e^{-A(t)}=\lambda \e^{-A(t)}.
\end{equation}
\end{proof}

\emph{Remarques}~:
\begin{enumerate}
\item si on prend comme fonction \(a\) une fonction constante, une primitive de \(a\) est \(A\) une fonction linéaire et on retrouve le résultat du premier théorème;
\item on constate à posteriori que les solutions de l'équation \eqref{eq:Hdiff1} distinctes de l'application nulle ne s'annulent pas sur \(I\);
\item une combinaison linéaire de solutions de l'équation \eqref{eq:Hdiff1} est une solution de l'équation \eqref{eq:Hdiff1}, si on note \(S_{\H} \) l'ensemble des solutions de l'équation \eqref{eq:Hdiff1}, alors puisque \(S_\mathcal{H} \) est soit non vide ou soit stable par combinaison linéaire alors c'est un \(\K\)-espace vectoriel (c'est même une droite vectorielle).
\end{enumerate}

\section{Équation différentielle linéaire du premier ordre avec second membre}
\label{sec:equadifflinpremierordresecondmembre}
On considère l'équation différentielle
\begin{equation}
y'+a(t)y=b(t) \label{eq:Ediff}
\end{equation}
avec \(a\) et \(b\) des fonctions au moins continues sur l'intervalle réel \(I\) à valeurs dans \(\K\).
On considère aussi l'équation homogène à l'équation \eqref{eq:Ediff}, c'est l'équation \eqref{eq:Hdiff}.
\subsection{Solution générale et solution particulière}
\label{subsec:solutiongeneraleetpart}
\begin{theo}\label{theo:3}
Supposons connaître une solution \(f_0\) de l'équation \eqref{eq:Ediff}, qu'on appellera solution particulière, alors une fonction \(f:I \longrightarrow \K\) est solution de l'équation \eqref{eq:Ediff} si et seulement si la fonction \(f-f_0\) est solution de l'équation \eqref{eq:Hdiff}.
\begin{equation}
\mathcal{S}_{\E}=f_0 + \mathcal{S}_{\H}
\end{equation}
\end{theo}
\begin{proof}
  Soit \(f:I \rightarrow \K\) solution de l'équation \eqref{eq:Ediff}. Soit \(g=f-f_0\), alors \(f\) et \(f_0\) sont solutions de l'équation \eqref{eq:Ediff}, donc elles sont dérivables et donc \(g\) est dérivable.
  \begin{equation}
    \forall t \in I \quad g'(t)+a(t)g(t)=(f'(t)+a(t)f(t))-(f'_0(t)+a(t)f_0(t))=b(t)-b(t)=0
  \end{equation}
g est donc solution de l'équation \eqref{eq:Ediff}.

Supposons ensuite que \(g=f-f_0\) est solution de l'équation \eqref{eq:Hdiff}, alors \(f=f_0+g\) est dérivable et
\begin{align}
  \forall t \in I \quad f'(t)+a(t)f(t)&=(g'(t)+f_0'(t)) + a(t)(g(t)+f_0(t)) \\ &=\underbrace{g'(t)+a(t)g(t)}_{b(t)} + \underbrace{f_0'(t)+a(t)f_0(t)}_{0}
\end{align}
donc \(f\) est solution de l'équation \eqref{eq:Ediff}.
\end{proof}

\emph{Remarque}~: Toute la difficulté est bien sûr de trouver une solution particulière de l'équation \eqref{eq:Ediff} dont l'existence n'est pas assurée par le théorème~\ref{theo:3}.

\subsection{Recherche de solutions particulière}
\label{subsec:recherchesolutionpart}
\subsubsection{Lorsque l'équation \eqref{eq:Hdiff} est à coefficient constants}
\label{subsubsec:recherchesolutionpart-coefconstants}
On suppose que le second membre \(b\) est de la forme \og exponentielle polynôme \fg{}, c'est-à-dire \(\forall t \in I \quad b(t)=P(t) \e^{mt}\) où \(P\) est une fonction polynomiale, \(P(t)=\sum_{i=0}^d \alpha_i t^i\) et \(m \in \K\). Dans ce cas on cherche une solution particulière sous la forme \(t \longmapsto Q(t)\e^{mt}\) où \(Q\) est un polynôme de même degré que \(P\).

\emph{Remarques}:
\begin{itemize}
\item c'est une méthode assez lourde quand le degré est élevé;
\item on peut l'appliquer avec un second membre polynomial ou avec un second membre \(t \longmapsto \e^{mt}\);
\item on peut aussi l'appliquer avec un second membre en sinus ou en cosinus, cosinus hyperbolique ou sinus hyperbolique car ces fonctions s'écrivent comme somme d'exponentielle.
\end{itemize}
\subsubsection{Lorsque l'équation \eqref{eq:Hdiff} est à coefficients non constants}
\label{subsubsec:recherchesolutionpart-coefnnconstants}
Il n'y a pas de méthode générale. Si on ne voit pas de solution \og évidente \fg{} on utilise la méthode de la variation de la constante.
%
\subsection{Principe de superposition}
\label{subsec:principesuperposition}
\begin{prop}
  Soient trois applications \(a\), \(b_1\) et \(b_2\) de \(I\) vers \(\K\) au moins continues. Si \(f_1\) et \(f_2\) sont des solutions respectives des équations
  \begin{align}
    y'+a(t)y&=b_1(t)\\ y'+a(t)y&=b_2(t)
  \end{align}
Alors la fonction \(f=f_1+f_2\) est une solution de l'équation différentielle 
\begin{equation}
  y'(t)+a(t)=b_1(t)+b_2(t)
\end{equation}
\end{prop}
\begin{proof}
  Soient \(f_1\) et \(f_2\) des solutions respectives des équations précédentes, alors elles sont dérivables sur \(I\) et donc \(f\) est dérivable sur \(I\)
  \begin{equation}
    f'+a \cdot f=(f_1' + f_2')+a(f_1+f_2)=(f_1'+af_1)+(f_2'+af_2)=b_1+b_2
  \end{equation}
Donc \(f\) est solution de \(\E\).
\end{proof}
%
\subsection{Variation de la constante}
\label{subsec:variationdelaconstante}
Soit l'équation
\begin{equation}
  \label{eq:varcnst}
  y'+a(t)y=b(t)
\end{equation}
avec \(a\) et \(b\) des fonctions au moins continues sur l'intervalle réel \(I\). Soit \(A\) une primitive de la fonction \emph{continue} \(a\) sur \(I\). On a vu que l'ensemble des solutions de l'équation homogène associée est
\begin{equation}
  S_{\H}=\{I \to \K : t \longmapsto \lambda \e^{-A(t)}; \lambda \in \R \}.
\end{equation}
On cherche des fonctions solutions de \(\E\) de la forme \(t \longmapsto \lambda(t)\e^{-A(t)}\) avec \(\lambda\) une fonction. D'où le nom de variation de la constante. Procédons par un raisonnement d'analyse/synthèse.
\begin{proof}[Analyse] Supposons qu'il existe une solution \(f\) de \(\E\) sur \(I\), on définit alors la fonction
  \begin{equation}
    \fonction{g}{I}{\K}{t}{f(t)\e^{A(t)}}.
  \end{equation}
  La fonction \(f\) est dérivable puisque c'est une solution de \(\E\) et comme \(A\) est dérivable aussi on déduit que \(g\) est dérivable sur \(I\), et pour tout \(t \in I\)~:
\begin{equation}
  g'(t)=f'(t)\e^{A(t)}+f(t)A(t)\e^{A(t)}=b(t)\e^{A(t)},
\end{equation}
puisque \(f\) est solution de \(\E\). Les fonctions \(b\) et \(A\) sont continues donc \(t \longmapsto b(t)\e^{A(t)}\) est continue par théorème généraux, elle admet donc des primitives sur l'intervalle \(I\). Soit \(C\) l'une de ses primitives alors, pour tout \(t \in I\), 
\begin{equation}
  g(t) = C(t) + \mu \iff f(t)=C(t)\e^{-A(t)} + \mu \e^{-A(t)}.
\end{equation}
Alors on a montré que si \(f\) est solution alors \(f\) est de cette forme.
\end{proof}
\begin{proof}[Synthèse] Soit \(C\) une primitive de la fonction continue \(t \longmapsto b(t)\e^{-A(t)}\), soit un scalaire \(\mu \in \K\). On définit \(\fonction{f}{I}{K}{t}{C(t)\e^{-A(t)}+\mu \e^{-A(t)}}\), alors \(f\) est dérivable puisque \(A\), \(C\) et l'exponentielle le sont et, pour tout \(t \in I\),
  \begin{align}
    f'(t)+a(t)f(t)&=C'(t) \e^{-A(t)} -a(t)C(t)\e^{-A(t)}-\mu a(t) \e^{-A(t)} \notag \\ 
    &\phantom{=}+ a(t)C(t)\e^{-A(t)} + \mu a(t) \e^{-A(t)} \\ 
    & = b(t) \e^{A(t)} \e^{-A(t)}=b(t).
  \end{align}
Ceci démontre l'existence d'une solution pour l'équation avec second membre et fournit une méthode pour la trouver.
\end{proof}
%
\begin{theo} 
  \label{theo:4}
  Soient \(I\) un intervalle réel, \(a\) et \(b\) deux fonctions au moins dérivables sur \(I\). Soit \(t_0\) un réel de \(I\) et \(y\) un scalaire de \(\K\). Il existe alors une unique fonction \(f:I \longmapsto \R\) solution de l'équation \eqref{eq:Ediff} qui satisfait la condition initiale \(y(t_0)=y_0\)
\end{theo}
\begin{proof}
  Soit \(f_0\) une solution particulière ---~dont on connaît l'existence grâce à la variation de la constante. Soit \(A\) une primitive de \(a\). D'après les théorèmes~\ref{theo:2} et~\ref{theo:3}, on sait qu'une application \(f:I \longmapsto \R\) est solution de l'équation \eqref{eq:Ediff} si et seulement si il existe \(\mu \in \K\) tel que
  \begin{equation}
    \forall t \in I \quad f(t)=f_0(t) + \mu \e^{-A(t)}
  \end{equation}
Donc 
\begin{align}
  f \text{~satisfait les conditions initiales} & \iff f(t_0)=y_0 \\ & \iff f_0(t_0)+\mu\e^{-A(t_0)} = y_0 \\ & \iff \mu = \e^{A(t_0)}(y_0-f_0(t_0))
\end{align}
Il existe donc une et une seule valeur de \(\mu\) telle que \(f\) soit solution de \(\E\) et satisfait les conditions initiales.
\end{proof}

\subsection{Problème de raccordement des solutions}
\label{subsec:pbmraccordement}
Soient à présent \(a\), \(b\) et \(c\) trois applications au moins continues de \(I\) vers \(\K\). On considère l'équation différentielle suivante
\begin{equation}
  \label{eq:raccord}
  a(t) y'+b(t)y=c(t).
\end{equation}
On distingue deux cas :
\begin{itemize}
\item si \(a\) ne s'annule pas sur \(I\), alors l'équation \eqref{eq:raccord} est équivalent à
  \begin{equation}
    y' + \frac{b}{a}(t) y = \frac{c}{a}(t),
  \end{equation}
et on sait alors résoudre ce genre d'équations comme on vient de le faire auparavant;
\item si \(a\) s'annule, c'est en général en un nombre fini de points. On commence par résoudre l'équation sur les sous intervalles de \(I\) où \(a\) ne s'annule pas, ensuite on envisage l'existence de solutions globales sur \(I\) qui devront être dérivables ---~donc continues~--- en tout point.
\end{itemize}
%
\section{Équations différentielles homogènes du second ordre à coefficients constants}
\label{sec:eqdiffsecondordrecoefconstants}
Soient \(a\),\(b\) et \(c\) trois scalaires avec \(a \neq 0\). Considérons l'équation différentielle
\begin{equation}
  \label{eq:Hdiff2}
  ay''+by'+cy=0
\end{equation}

\emph{Remarque}~: L'ensemble des solutions de cette équation noté \(\mathcal{S}_{\H}\) est un \(\K\)-espace vectoriel.
%
\subsection{Résolution dans \(\C\)}
\label{subsec:resdansC}
Maintenant \(a\), \(b\) et \(c\) sont des complexes avec \(a \neq 0\). On considère l'équation différentielle~\eqref{eq:Hdiff2} et on cherche des solutions de la forme \(g :t \longmapsto \e^{rt}\) avec \(r \in \C\). Alors 
\begin{align}
  g\text{~est solution de } \eqref{eq:Hdiff2} & \iff ag'' + bg'+cg=0  \\ & \iff ar^2+br+c=0
\end{align}
On appelle cette dernière équation \emph{l'équation caractéristique} de l'équation \eqref{eq:Hdiff2}, et on note \(\Delta\) son discriminant. Soient \(r \in \C\) une racine de cette équation, un complexe \(\lambda\) et une fonction \(f:t \longmapsto \lambda \e^{rt}\). La fonction \(f\) est donc une solution de l'équation \eqref{eq:Hdiff2}.
%
\begin{theo}
  \label{theo:5} 
  Les solutions à valeurs complexes de l'équation \eqref{eq:Hdiff2} sont de la forme~:
  \begin{itemize}
  \item Si le discriminant de l'équation caractéristique est nul~: 
    \begin{equation} 
      t \longmapsto (\lambda t + \mu)\e^{rt},
    \end{equation} 
    avec \(r\) la racine double et \(\lambda \ \mu\) des complexes;
  \item Si le discriminant de l'équation caractéristique est non nul~: 
    \begin{equation} 
      t \longmapsto \lambda \e^{r_1 t} + \mu \e^{r_2 t},
    \end{equation} 
    avec \(r_1 \ r_2\) les racines et \(\lambda \ \mu\) des complexes.
  \end{itemize}
\end{theo}
\begin{proof}[Analyse] Soit \(f\) une solution de l'équation \eqref{eq:Hdiff2}. On définit
  \begin{equation}
    \fonction{g}{\R}{\C}{t}{f(t)\e^{-rt}}.
  \end{equation}
  Comme \(f\) est solution de l'équation \eqref{eq:Hdiff2}, elle est dérivable deux fois. Donc pour tout réel \(t\) \(f(t)=g(t) \e^{rt}\), et en injectant cette forme dans l'équation différentielle, on obtient~:
\begin{equation}
  \forall t \in \R \quad ag''(t)+(2ar+b)g'(t)=0
\end{equation}
C'est une équation du premier ordre en \(g'\) que l'on sait résoudre. On distingue deux cas~:
\begin{itemize}
\item \(2ar+b=0 \iff r=-\frac{b}{2a} \ \Delta = 0\) On obtient une racine double donc \(g''=0\) soit \(g\) est une fonction affine et donc 
  \begin{equation}
   \exists (\mu,\lambda) \in \C^2 \ \forall t \in \R \quad f(t)=(\lambda t+\mu)\e^{rt};
  \end{equation}
\item Le discriminant est non nul et on note \(\delta\) la racine carrée complexe de \(\Delta\), ainsi \(2ar+b=\delta\) donc 
  \begin{equation}
    g''+\frac{a}{\delta} g'=0.
  \end{equation}
D'après le théorème~\ref{theo:1}, il existe un complexe \(\lambda\) tel que
\begin{equation}
  \forall t \in \R \quad g'(t)=\lambda \e^{-\frac{\delta}{a}}.
\end{equation}
Ainsi il existe un autre complexe, la constante d'intégration, \(\mu\) tel que
\begin{equation}
 \forall t \in \R \quad g(t)=-\lambda \frac{a}{\delta} \e^{-\frac{\delta}{a} t} + \mu. 
\end{equation}
En revenant à la fonction \(f\), on a
\begin{equation}
  \forall t \in \R \quad f(t)=\lambda' \e^{r_2 t} + \mu \e^{r_1 t},
\end{equation}
où \(r_1\) et \(r_2\) sont les racines de l'équation caractéristique.
\end{itemize}
Dans chacun des deux cas, on est arrivé à trouver la solution \(f\) de l'équation différentielle.
\end{proof}
\begin{proof}[Synthèse]
  Ici aussi, on distingue deux cas
\begin{itemize}
\item Cas où le discriminant est nul. L'équation caractéristique admet alors une racine double \(r=-\frac{b}{2a}\). On considère alors la fonction \(\fonction{f_1}{\R}{\C}{t}{t\e^{rt}}\). Montrons que \(f_1\) est solution de l'équation \eqref{eq:Hdiff2}. La fonction \(f_1\) est deux fois dérivable et on a 
  \begin{equation}
    \forall t \in \R  \quad af_1''(t)+bf_1'(t)+cf_1(t)=\e^{rt} \left( (ar^2+br+c)t + 2ar+b \right),
  \end{equation}
\(r\) est une racine double donc \(ar^2+br+c=2ar+b=0\) donc \(f_1\) est solution de l'équation \eqref{eq:Hdiff2}. Soit la fonction \(\fonction{f_2}{\R}{\C}{t}{\e^{rt}}\). Elle est solution de l'équation \eqref{eq:Hdiff2} puisque \(r\) est une racine de l'équation caractéristique.
\item Cas où le discriminant n'est pas nul. L'équation caractéristique admet alors deux solutions distinctes \(r_1\) et \(r_2\). Si on pose les fonctions associées \(f_1\) et \(f_2\), elles sont solutions de l'équation \eqref{eq:Hdiff2}. Puisque l'ensemble des solutions de l'équation \eqref{eq:Hdiff2} est un espace vectoriel alors toute combinaison linéaire de \(f_1\) et \(f_2\) est une solution de l'équation \eqref{eq:Hdiff2}.
\end{itemize}
\end{proof}
%
\subsection{Résolution sur \(\R\)}
\label{subsec:resR}
Dans ce paragraphe, on considère que les scalaires \(a\), \(b\) et \(c\) sont des réels avec \(a \neq 0\). On note \(\Delta=b^2-4ac\) le discriminant de l'équation caractéristique.
\begin{theo}
  \label{theo:6} 
  Les solutions à valeurs complexes de l'équation \eqref{eq:Hdiff2} sont de la forme suivante
  \begin{itemize}
  \item si le discriminant est nul, \(t \longmapsto (\lambda t + \mu)\e^{rt}\), avec \(r\) la racine double et \(\lambda \ \mu\) des réels;
  \item si le discriminant est positif, \(t \longmapsto \lambda \e^{r_1 t} + \mu \e^{r_2 t}\), avec \(r_1 \ r_2\) les racines et \(\lambda \ \mu\) des réels;
  \item si le discriminant est négatif, \(t \longmapsto \e^{\alpha t} (A \cos \omega t + B \sin \omega t)\) avec \(A\) et \(B\) des réels et \(\alpha=-\frac{b}{2a} \ \omega = \frac{\sqrt{-\Delta}}{2a}\).
  \end{itemize}
  On dit alors que \(\mathcal{S}_{\H}\) est de dimension 2, c'est-à-dire que c'est un plan vectoriel.
\end{theo}
\begin{proof}
  On cherche des solutions de la première sous-section qui sont à valeurs réelles~:
  \begin{itemize}
  \item Cas 1, si le discriminant est nul alors si \(f\) est à valeurs réelles alors \(f(0)=\mu \in \R\) et \(f(1)=(\lambda+\mu)\e^r \iff \lambda=f(1)\e^{-r}-f(0) \in \R\). Réciproquement si \(\lambda\) et \(\mu\) sont des réels alors \(f\) est à valeurs réelles;
  \item Cas 2, si le discriminant est positif alors si \(f\) est à valeurs réelles alors on a \(f(0)=\lambda + \mu\) et \(f(1)=\lambda \e^{r_1} + \mu \e^{r_2}\) donc \(\mu=\frac{f(1)-f(0)\e^{r_1}}{\e^{r_2}-\e^{r_1}}\) et \(\lambda = f(0) - \mu\) et comme \(r_2 \neq r_1\) et que \(\exp\) est injective alors \(\e^{r_1} \neq \e^{r_2}\) ainsi \(\lambda, \mu \in \R\), réciproquement si \(\lambda, \mu \in \R\) alors \(f\) est à valeur réelles;
  \item Cas 3, si le discriminant est négatif alors l'équation caractéristique admet deux racines complexes conjuguées \(r_1=\alpha+\ii \omega\) et \(r_2=\alpha-\ii \omega\) avec \(\alpha=-\frac{b}{2a}\) et \(\omega=\frac{\sqrt{-\Delta}}{2a}\), il existe un doublet de complexe \((\lambda, \mu)\) tel que pour tous réel \(t\) on ait \(f(t)=\lambda \e^{r_1 t} + \mu \e^{r_2 t}=\lambda \e^{\alpha t} \e^{\ii \omega t} + \mu \e^{\alpha t} \e^{\ii \omega t}\). Alors
      \begin{align}
        f(t)&=(\lambda_1 + \ii \lambda_2) \e^{\alpha t} \e^{\ii \omega t} +(\mu_1 + \ii \mu_2) \e^{\alpha t} \e^{-\ii \omega t} \\
       &= \lambda_1\e^{\alpha t} \cos(\omega t) + \ii \lambda_1 \e^{\alpha t} \sin(\omega t) + \ii \lambda_2 \e^{\alpha t} \cos(\omega t) - \lambda_2 \e^{\alpha t} \sin(\omega t) \notag \\ 
       &\phantom{=}+ \mu_1\e^{\alpha t}\cos(\omega t) -\ii \mu_1 \e^{\alpha t} \sin(\omega t) + \ii\mu_2 \e^{\alpha t} \cos(\omega t) + \mu_2 \e^{\alpha t} \sin(\omega t) \\  
       &=\e^{\alpha t} \left[ \cos(\omega t) (\lambda_1+\mu_1)+ \sin(\omega t)(\mu_2-\lambda_2)\right]+\ii\e^{\alpha t} \cos(\omega t) (\lambda_2+\mu_2) \notag \\ 
       &\phantom{=}+ \ii\e^{\alpha t} \sin(\omega t)(\lambda_1-\mu_1).
      \end{align}
Si \(f\) est à valeurs réelles alors \(\Im(f)=\tilde{0}\), ce qui implique que pour tous réel \(t\) on ait
\begin{equation}
  \e^{\alpha t} \left[ \cos(\omega t) (\lambda_2+\mu_2)+ \sin(\omega t)(\lambda_1-\mu_1)\right] = 0,
\end{equation}
puisque pour tous les réels \(t\) \(\e^{\alpha t} \neq 0\) on a
\begin{equation}
  \cos(\omega t) (\lambda_2+\mu_2)+ \sin(\omega t)(\lambda_1-\mu_1) = 0,
\end{equation}
et en posant \(t=0\) et \(t=\frac{\pi}{2 \omega}\), on obtient un système d'équation qui nous donne \(\mu=\bar{\lambda}\).

Réciproquement si \(\mu=\bar{\lambda}\) alors pour tout réel \(t\), on a
\begin{equation}
 f(t)=\lambda \e^{\alpha t} \e^{\ii \omega t} + \bar{\lambda} \e^{\alpha t} \e^{-\ii\omega t}, 
\end{equation}
alors
\begin{equation}
 f(t)=(\lambda + \bar{\lambda}) \e^{\alpha t} \cos \omega t + \ii(\lambda - \bar{\lambda}) \e^{\alpha t}\sin \omega t. 
\end{equation}
On pose \(A=2 \Re(\lambda)\) et  \(B=2\Im(\lambda)\), ainsi
\begin{equation}
  \forall t \in \R \quad f(t)=\e^{\alpha t} (A \cos \omega t -B \sin \omega t).
\end{equation}
Si \((A,B)=(0,0)\) alors \(f(t)=0\). Sinon, on a \(A^2+B^2>0\) et 
\begin{equation}
\left(\dfrac{A}{\sqrt{A^2+B^2}} \right)^2+\left(\dfrac{B}{\sqrt{A^2+B^2}} \right)^2=1,
\end{equation}
donc il existe un réel \(\varphi\) tel que \(\cos \varphi=\dfrac{A}{\sqrt{A^2+B^2}}\) et \(\sin \varphi = \dfrac{B}{\sqrt{A^2+B^2}}\) avec \(C=\sqrt{A^2+B^2}\). On a donc pour tous réel \(t\) 
\begin{equation}
f(t)=\e^{\alpha t} (C \cos \varphi \cos \omega t - C \sin \varphi \sin \omega t) = C \e^{\alpha t} \cos(\omega t + \varphi)
\end{equation}
 Dans les deux cas, les solutions obtenues sont à valeurs réelles.
  \end{itemize}
Dans chacun de ses trois cas, on a trouvé une solution à cette équation différentielle.
\end{proof}
%
\section{Équation différentielles linéaires du second ordre à coefficients constants avec second membre}
\label{sec:eqdifflinsecondordrecoefconstantsecondmembre}
Soient \((a,b,c) \in \K^{*}\), \(d:I \longmapsto \K\) une fonction au moins continue. On considère l'équation différentielle
\begin{equation}
  \label{eq:Ediff2}
  ay'' + by' + cy=d(t)
\end{equation}
On note \(\mathcal{S}_\E\) l'ensemble des solutions de l'équation \eqref{eq:Ediff2}. On introduit l'équation homogène associée
\begin{equation}
  \label{eq:Hdiff3}
  ay''+by'+cy=0
\end{equation}
et de la même manière on note \(\mathcal{S}_{\H}\) l'ensemble des solutions de l'équation \eqref{eq:Hdiff3}.
\subsection{Solution générale et solution particulière}
\label{subsec:solutiongeneraleetsolutionpart}
\begin{theo}
Supposons connaître une solution \(f_0\) de l'équation \(\E\) ---~dite solution particulière~--- alors une application \(f:I \longmapsto \K\) est une solution de \(\E\) si et seulement si la fonction \(f-f_0\) est une solution de l'équation \eqref{eq:Hdiff3}. On a donc \(\mathcal{S}_\E=\mathcal{S}_{\H} + f_0\). On dit que la solution générale de \(\E\) est la somme d'une solution particulière et de la solution générale de l'équation \eqref{eq:Hdiff3}.
\end{theo}
\begin{proof}
  Supposons que \(f\) est une solution de \(\E\) et posons \(g=f-f_0\) puisque \(f\) et \(f_0\) sont des solutions de \(\E\), elles sont dérivables deux fois et par conséquent \(g\) est aussi dérivable deux fois, et on a
  \begin{align}
    \forall t \in I \quad ag''(t)+bg'(t)+cg(t) &= af''(t)+bf'(t)+cf(t) \\ &- (af_0''(t)+bf_0'(t)+cf_0(t))\\ &= d(t)-d(t)=0
  \end{align}
La fonction \(g\) est donc solution de l'équation \eqref{eq:Hdiff3}.

Soit une application \(f:I \longmapsto \K\). Supposons que \(g=f-f_0\) soit solution de l'équation \eqref{eq:Hdiff3}. Puisque \(g\) est solution de l'équation \eqref{eq:Hdiff3} elle est deux fois dérivable et puisque \(f\) est solution de l'équation \eqref{eq:Ediff2} alors elle est deux fois dérivable. Ainsi donc \(f=g+f_0\) est deux fois dérivable et on a pour tout \(t \in I\)~:
\begin{align}
  af''(t)+bf'(t)+cf(t)&=(ag''(t)+bg'(t)+cg(t))\notag\\ &\phantom{=}+(af_0''(t)+bf_0'(t)+cf_0(t))\\ &=0+d(t) 
\end{align}
donc \(f\) est solution de l'équation \eqref{eq:Ediff2}.
\end{proof}
\subsection{Principe de superposition}
\begin{prop}
  Soient \(f_1\) et \(f_2\) des solutions particulières respectives des équations
  \begin{align}
    ay''+by'+cy=&d_1(t) \label{eq:EdiffE1}\\ 
    ay''+by'+cy=&d_2(t) \label{eq:EdiffE2},
  \end{align}
où \(d_1\) et \(d_2\) sont des fonctions au moins continues de \(I\) vers \(\K\). Alors \(f=f_1+f_2\) est une solution de l'équation
\begin{equation}
  ay''+by'+cy=d_1(t)+d_2(t)
\end{equation}
\end{prop}
\begin{proof}
  Soient \(f_1\) et \(f_2\) des solutions respectives de l'équation \eqref{eq:EdiffE1} et de l'équation \eqref{eq:EdiffE2}, alors \(f_1\) et \(f_2\) sont dérivables sur \(I\) donc \(f_1+f_2\) aussi~:
  \begin{align}
    \forall t \in I \quad af''(t)+bf'(t)+cf(t)&=af_1''(t)+bf_1'(t)+cf_1(t) \notag \\ %je met \notag parce que ce n'est qu'unes seule equation avec celle du dessous
    &\phantom{=}+af_2''(t)+bf_2'(t)+cf_2(t)\\ &=d_1(t)+d_2(t)
  \end{align}
Car \(f_1\) et \(f_2\) sont solutions de l'équation \eqref{eq:EdiffE1} et de l'équation \eqref{eq:EdiffE2} respectivement. La fonction \(f\) est donc une solution de l'équation \eqref{eq:Ediff2}.
\end{proof}
\emph{Application :} si le second membre est de la forme \(d_1(t)+ \dotsb +d_n(t)\) on cherche indépendamment des solutions particulières \(f_i\) des équations différentielles correspondantes avec le second membre \(d_i\) et ensuite on somme toutes les solutions pour trouver la solution particulière de l'équation de départ.

\subsection{Recherche d'une solution particulière dans le cas où le second membre est une \og exponentielle polynôme \fg}
\label{subsec:recherchesolutionpartexppol}
Dans ce cours on s'intéresse uniquement à des seconds membre de cette forme.
\begin{defdef}
  On appelle exponentielle polynôme toute fonction \(d\) de la forme
  \begin{equation}
    d(t)=\sum_{k=1}^{n} \e^{m_k t} P_k (t),
  \end{equation}
où \(n \in \N^*\), \(m_1, \ldots, m_n \in \K\) et \( P_1, \ldots, P_n\) sont des fonctions polynomiales. 
\end{defdef}
On considère l'équation différentielle  suivante avec \((a,b,c) \in \K^{*}\times\K\times\K\)
\begin{equation}
  \label{eq:Ediff4}
  ay''+by'+cy=\sum_{k=1}^{n} \e^{kt} P_k(t)
\end{equation}
On cherche une solution particulière de l'équation \eqref{eq:Ediff4}. Grâce au principe de superposition, il suffit de trouver une solution particulière pour chacune des équation \(\E_k\) : 
\begin{equation}
  \label{eq:Ediffk}
  ay''+by'+cy= \e^{kt} P_k(t)
\end{equation}
On admet l'indice \(k\) dans une partie de la suite pour alléger la notation.
\begin{theo}
  Pour tout entier \(k \in \intervalleentier{1}{n}\) il existe une solution particulière \(f_k\) de l'équation \eqref{eq:Ediffk} de la forme suivante
  \begin{itemize}
  \item si \(m_k\) n'est pas racine de l'équation caractéristique alors
    \begin{equation}
      \forall t \in I \quad f_k(t)=e^{m_k t} Q_k(t)
    \end{equation}
    où \(\deg(Q_k)=\deg(P_k)\);
  \item si \(m_k\) est une racine simple de l'équation caractéristique alors 
    \begin{equation}
      \forall t \in I \quad f_k(t)=e^{m_k t} Q_k(t)
    \end{equation}
    où \(\deg(Q_k)=\deg(P_k)+1\);
  \item si \(m_k\) est une racine double de l'équation caractéristique alors 
    \begin{equation}
      \forall t \in I \quad f_k(t)=e^{m_k t} Q_k(t)
    \end{equation}
    où \(\deg(Q_k)=\deg(P_k)+2\).
  \end{itemize}
La fonction \(f=\sum_{k=1}^{n} f_k\) est alors une solution particulière de l'équation \eqref{eq:Ediff4}. Dans le deuxième cas, on pourra supposer le terme constant de \(Q_k\) nul et dans le troisième cas, on pourra supposer que le terme de degré \(0\) et \(1\) de \(Q_k\) sont nuls.
\end{theo}
\begin{proof}[Analyse]
Soit  \(f\) une solution de l'équation \eqref{eq:Ediffk}. On définit \(\fonction{g}{I}{\K}{t}{\e^{-mt}f(t)}\) alors puisque \(f\) est solution, elle est deux fois dérivable et donc \(g\) est dérivable deux fois. Alors 
\begin{gather}
  \forall t \in I \quad f(t)=g(t)\e^{mt}, \\ 
  \forall t \in I \quad f'(t)= (mg(t)+g'(t))\e^{mt}, \\
  \forall t \in I \quad f''(t)=(m^2g(t)+2mg'(t)+g''(t))\e^{mt}.
\end{gather}
La fonction \(f\) est solution de l'équation \eqref{eq:Ediffk} donc 
\begin{gather}
  \forall t \in I \quad af''(t)+bf'(t)+cf(t)=\e^{mt} P(t), \\
  \forall t \in I \quad  \e^{mt}(am^2g(t)+2amg'(t)+ag''(t))  \notag \\ +\e^{mt}(bmg(t)+bg'(t))+c\e^{mt}g(t)=\e^{mt}P(t), \\
  \forall t \in I \ \e^{mt} \neq 0  \quad  ag''(t)+(2am+b)g'(t) +(am^2+bm+c)g(t)=P(t). \label{eq:F}
\end{gather}
On a montré que \(g\) est solution d'une équation différentielle linéaire du 2\up{nd} ordre à coefficients constants dont le 2\up{nd} membre est un polynôme. On distingue trois cas :
\begin{enumerate}
\item si \(am^2+bm+c \neq 0\) alors il existe une solution particulière de l'équation \eqref{eq:F} sous la forme \(g=Q\) où \(Q\) est un polynôme de même degré que \(P\). Il en existe une seule de cette forme.
\item si \(am^2+bm+c=0\) et \(2am+b \neq 0\) alors il existe une solution particulière de l'équation \eqref{eq:F} sous la forme \(g=Q\) où \(Q\) est un polynôme de degré \(deg(P)+1\).
\item si  \(am^2+bm+c=0\) et \(2am+b = 0\) alors il existe une solution particulière de l'équation \eqref{eq:F} sous la forme \(g=Q\) où \(Q\) est un polynôme de degré \(deg(P)+2\).
\end{enumerate}
On propose de démontrer ces trois cas un par un.

Si \(am^2+bm+c \neq 0\) on pose \(P(t)=\sum_{j=0}^{n} \alpha_j t^j\) et on cherche \(g\) sous la forme \(g(t)=\sum_{j=0}^{n} \beta_j t^j\). La fonction \(g\) est deux fois dérivable et 
\begin{gather}
\forall t \in I \qquad g'(t)=\sum_{j=1}^n \beta_j j t^{j-1}=\sum_{j=0}^{n-1} \beta_{j+1} t^j; \\
\forall t \in I \qquad g''(t)=\sum_{k=0}^{n-2} \beta_{k+2} (k+2)(k+1) t^k.
\end{gather}
 Ainsi \(g\) est solution de l'équation \eqref{eq:F} si et seulement si 
  \begin{align}
    \forall t \in I \quad  a \sum_{j=0}^{n-2} \beta_{j+2} (j+2)(j+1)t^j+ (2am+b) \sum_{j=0}^{n-1} \beta_{j-1}(j+1)t^j \notag \\
      + (am^2+bm+c)\sum_{j=0}^{n} \beta_j t^j  =  \sum_{j=0}^{n} \alpha_j t^j,
\end{align}
c'est-à-dire en changeant les variables si et seulement si
\begin{align}
    \forall t \in I \quad \sum_{j=0}^{n-2} a\beta_{j+2} (j+2)(j+1)t^j +  \sum_{j=0}^{n-2}(2am+b)\beta_{j+1}(j+1)t^j  \notag \\ 
    + (2am+b)\beta_n nt^{n-1}+ \sum_{j=0}^{n-2} (am^2+bm+c) \beta_j t^j  + (am^2+bm+c)\beta_{n-1} t^{n-1} \notag \\ 
    + (am^2+bm+c) \beta_nt^n=\sum_{j=0}^{n} \alpha_j t^j.
  \end{align}
Deux polynômes sont égaux si et seulement s'ils ont les mêmes coefficients, donc si et seulement si pour tout naturel \(j\) de \(\intervalleentier{0}{n-2}\)
\begin{equation}
  \begin{cases}
    \alpha_j= a \beta_{j+2}(j+2)(j+1)+(2am+b)\beta_{j+1}(j+1)+(am^2+bm+c)\beta_j\\ 
    \alpha_{n-1}= (2am+b)\beta_n n +(am^2+bm+c)\beta_{n-1} \\ 
    \alpha_n=(am^2+bm+c) \beta_n
  \end{cases}.
\end{equation}

Comme \(am^2+bm+c \neq 0\), on peut calculer \(\beta_n\) en fonction de \(\alpha_n\). Ensuite on peut exprimer \(\beta_{n-1}\) en fonction de \(\beta_{n}\) et \(\alpha_{n-1}\) et donc tous les coefficients de proche en proche. Ce système se résout de proche en proche et admet une unique solution.

Si \(am^2+bm+c=0\) et \(2am+b \neq 0\), \(g\) est solution de \(ay''+(2am+b)y'=P\) et donc \(g'\) est solution de \(0z''+az'+(2am+b)z=P\) avec \(2am+b \neq 0\). D'après le premier cas, la dernière équation admet une solution polynomiale de degré \(\deg(P)\), et si on prend une primitive de cette solution, on obtient une fonction polynomiale \(g\) de degré \(\deg(P) + 1\) qui est solution de l'équation initiale. On peut choisir la constante d'intégration comme on veut, mais d'habitude on prend la constante nulle.

Si \(am^2+bm+c = 0\) et \(2am+b = 0\) alors \(ay''=P\) et puisque \(a \neq 0\), on peut trouver une solution en intégrant deux fois \(\dfrac{P}{a}\) et on obtient une solution polynomiale de degré \(\deg(P)+2\) et les constantes d'intégration peuvent être choisies de façon arbitraire.
\end{proof}
\begin{proof}[Synthèse]
Soit \(g_k\) une solution polynomiale de \((F_k)\), on définit alors l'application
\begin{equation}
  \fonction{f_k}{I}{\K}{t}{g_k(t)e^{m_k t}}.
\end{equation}
Comme \(g_k\) est solution alors elle est deux fois dérivable et donc \(f_k\) est deux fois dérivable. D'après le même calcul que dans l'analyse on a
  \begin{equation}
    \forall t \in I \ af_k''(t)+bf_k'(t)+cf_k(t)=\e^{m_k t} P_k(t)
  \end{equation}
La somme \(f=\sum_{k=1}^n f_k\) est une solution particulière de l'équation \eqref{eq:Ediff4} d'après le principe de superposition.
\end{proof}
%
Cette méthode permet également de trouver des solutions particulières pour les seconds membres de la forme
\begin{equation}
\sum_{k=1}^n \e^{m_k t} \left( P_k(t) \cos(\alpha_k t) +Q_k(t)\sin(\alpha_kt)\right),
\end{equation}
grâce au passage aux exponentielles complexes.
%
\begin{theo}
Soient \((t_0,y_0,y'_0) \in I \times \K^2\), \((a,b,c) \in \K^{*} \times \K^2\) et \(d: I \longmapsto \K\) une fonction exponentielle polynôme. Alors l'équation différentielle \(ay''+by'+cy=d(t)\) admet une unique solution qui satisfait les conditions initiales \( \left\{~\begin{array}{l} y(t_0)=y_0 \\ y'(t_0)=y'_0 \end{array}\right.\).
\end{theo}
\begin{proof}
On considère \(\K=\C\) pour simplifier les notations. Soit \(f_0\) une solution particulière de l'équation \eqref{eq:Ediff4}~\footnote{on peut le faire car le théorème précédent montre qu'il en existe} et on divise le problème en deux cas.

Le discriminant de l'équation caractéristique est non nul et soient \(r_1\) et \(r_2\) les racines de l'équation caractéristique. Alors une fonction \(f\) est solution de l'équation différentielle si et seulement s'il existe des complexes \(\mu\) et \(\lambda\) tels que \(\forall t \in I \ f(t)=\lambda \e^{r_1 t}+ \mu \e^{r_2 t} + f_0(t)\). Alors la suite d'équivalence est vraie~:
  \begin{align}
    f \text{~satisfait les CI} &\iff  \begin{cases}f(t_0)=y_0 \\ f'(t_0)=y'_0 \end{cases}\\
    &\iff  \begin{cases}\lambda \e^{r_1 t_0}+ \mu \e^{r_2 t_0} + f_0(t_0)=y_0 \\ \lambda r_1 \e^{r_1 t_0}+ \mu r_2 \e^{r_2 t_0} + f_0'(t_0)=y'_0 \end{cases}\\ 
    &\iff  \begin{cases}\mu\e^{r_2t_0}=y_0-f_0(t_0)-\lambda \e^{r_1t_0} \\ \lambda r_1 \e^{r_1t_0}+r_2(y_0-f_0(t_0)-\lambda\e^{r_1t_0})+f'_0(t_0)=y'_0 \end{cases}\\ 
    &\iff \begin{cases} \mu=\e^{-r_2t_0}(y_0-f_0(t_0)-\lambda \e^{r_1t_0}) \\ \lambda(r_1-r_2)=\e^{-r_1t_0}(y'_0-f'_0(t_0)+r_2f_0(t_0)-r_2y_0)\end{cases}.
  \end{align}
Puisque \(r_2 \neq r_1\) il existe un unique couple de complexe \((\lambda, \mu)\) tel que \(f\) satisfait les conditions initiales.

Dans le cas du discriminant nul, on pose \(r\) sa racine double. Si la fonction \(f\) est solution de l'équation différentielle alors il existe un couple de complexes \((\lambda, \mu)\) tel que \(\forall t \in I \ f(t)=(\lambda t+\mu)\e^{rt}+f_0(t)\). Alors la suite d'équivalence est vraie :
  \begin{align}
    f \text{~satisfait les CI} &\iff \begin{cases} f(t_0)=y_0 \\ f'(t_0)=y'_0 \end{cases}\\ 
    &\iff \begin{cases}(\lambda t_0+\mu)\e^{rt_0}+f_0(t_0)=y_0 \\ (\lambda+r(\lambda t_0 +\mu)\e^{rt_0}+f_0'(t_0)=y'_0\end{cases}\\ 
    &\iff \begin{cases}\mu=\e^{-rt_0}(y_0-f_0(t_0))-\lambda t_0 \\ (\lambda+\lambda r t_0)\e^{rt_0}+r(y_0-f_0(t_0))-\lambda t_0 r \e^{rt_0}+f'(t_0)=y'_0\end{cases}\\ 
    &\iff \begin{cases}\lambda \e^{rt_0}=y'_0 -f'_0(t_0)-r(y_0-f_0(t_0)) \\ \mu=\e^{-rt_0}(y_0-f_0(t_0))-\lambda t_0\end{cases}.
  \end{align}
\end{proof}
\cleardoublepage
\section{Exercices}
\begin{exercice}
    Résoudre l'équation différentielle~:
    \begin{equation}
        y'-xy = x\sin(x^2).
    \end{equation}
    On pourra rechercher une solution particulière de la forme \(x \longmapsto \alpha\cos(x^2)+\beta\sin(x^2)\)
\end{exercice}
\begin{exercice}
    Résoudre l'équation différentielle suivante
    \begin{equation}
        y'\cos x + y \sin x = x \sin 2x + x^2
    \end{equation}
\end{exercice}
\begin{exercice}
    Déterminer toutes les fonctions dérivables de \(\R\) dans \(\R\), \(x\) et \(y\) tels que \(x(0)=y(0)=0\) et pour tout réel \(t\)~:
    \begin{align}
        x'(t) & = x(t) + y(t) + \sin t \\
        y'(t) & = -x(t) + 3y(t)
    \end{align}
\end{exercice}
\begin{exercice}
    Résoudre sur \(\intervalleoo{-\frac{\pi}{2}}{\frac{\pi}{2}}\) l'équation différentielle~:
    \begin{equation}
        y'-y\tan x = -\cos^2 x
    \end{equation}
\end{exercice}
\begin{exercice}
    Déterminer les fonctions dérivables sur \(\intervalleoo{0}{+\infty}\) qui vérifie pour tout réel \(x\) strictement positif~:
    \begin{equation}
        f'(x) = f\left(\frac{1}{4x}\right)
    \end{equation}
    Si $f$ est une solution, on pourra montrer que la fonction \(\fonction{g}{\R}{\R}{t}{f(\e^t)}\) est une solution d'une équation différentielle linéaire à coefficients constants que l'on résoudra.
\end{exercice}
\begin{exercice}
    Résoudre l'équation différentielle~:
    \begin{equation}
        y' = \abs{y-x}
    \end{equation}
\end{exercice}
\begin{exercice}
    Montrer que, pour tout \(m \in \intervalleoo{0}{2}\), l'équation différentielle~:
    \begin{equation}
        y'' + (1-2m)y' - 2my = \e^{2x}
    \end{equation}
    admet une et une seule solution \(f_m\) telle que \(f_m(0)=f_m'(0)=0\). Montrer que pour tout réel \(x\) \(\lim\limits_{m \to 1} f_m(x) = f_1(x)\).
\end{exercice}
\begin{exercice}
    Résoudre sur \(\R\) l'équation différentielle~:
    \begin{equation}
        y'' - 4y' + 3y = x^2\e^x + x\e^{2x}\cos x
    \end{equation}
\end{exercice}
\begin{exercice}
    Déterminer l'ensemble des fonctions \(f\) de \(\R\) dans \(\R\) dérivables telles que~:
    \begin{equation}
        \forall x \in \R \quad f'(x) = f(-x)
    \end{equation}
\end{exercice}
\begin{exercice}
    Déterminer l'ensemble des fonctions \(f\) de \(\R\) dans \(\R\) deux fois dérivables telles que~:
    \begin{equation}
        \forall x \in \R \quad f''(x) + f(-x) = x\e^{-x}
    \end{equation}
\end{exercice}
\begin{exercice}
    On considère l'équation différentielle non linéaire 
    \begin{equation}
        \label{eq:nonlin}
        y' = 1 + y^2.
    \end{equation}
    \begin{enumerate}
        \item Trouver une solution particulière de cette équation sur l'intervalle \(\intervalleoo{-\frac{\pi}{2}}{\frac{\pi}{2}}\).
        \item Soient \(a<b\) deux réels, \(I = \intervalleoo{a}{b}\) et \(f : I \longmapsto \R\) une fonction solution de l'équation~\eqref{eq:nonlin}. On définit une fonction \(g\) de \(I\) dans \(\R\) définit pour tout réel \(x\) de \(I\) par~: \(g(x) = \arctan f(x) \). Montrer que \(g\) est dérivable et que c'est une solution d'une équation différentielle linéaire du premier ordre notée \(\mathcal{H}\).
        \item Résoudre \(\mathcal{H}\).
        \item En déduire que l'équation~\eqref{eq:nonlin} possède une unique solution sur l'intervalle \(\intervalleoo{-\frac{\pi}{2}}{\frac{\pi}{2}}\).
    \end{enumerate}
\end{exercice}
\begin{exercice}    
    On cherche des solutions sur \(\intervalleoo{0}{+\infty}\) de l'équation~:
    \begin{equation}
        \label{eq:eqH}
        x^2y'' + 2xy' + y = \frac{1}{x}
    \end{equation}
    \begin{enumerate}
        \item Soit \(y\) une solution de \eqref{eq:eqH}. On définit la fonction \(Y \in \R^{\R}\) en posant pour tout réel \(t\), \(Y(t) = y(\e^t)\). Montrer que la fonction \(Y\) vérifie une équation différentielle linéaire à coefficients constants que l'on notera \(\mathcal{E}'\).
        \item Résoudre \(\mathcal{E}'\). 
        \item En déduire l'ensemble des solutions de l'équation~\eqref{eq:eqH}.
        \item Montrer qu'il existe une unique solution \(y\) de \eqref{eq:eqH} telle que \(y(1)=y'(1)=0\).
    \end{enumerate}
\end{exercice}
\begin{exercice}
    \begin{enumerate}
        \item Soient un intervalle réel \(I\), \(a \in \R^I\) et \(b \in \C^I\) deux fonctions continues. On suppose que la fonction \(f \in \C^I\) est une solution de l'équation différentielle \(y' + a(x)y = b(x)\). Notons \(c = \Re(b)\) et \(d = \Im(b)\). Montrer que la fonction \(g = \Re(f)\) est une solution de l'équation différentielle \(y' + a(x)y = c(x)\) et que la fonction \(h = \Im(f)\) est une solution de l'équation différentielle \(y' + a(x)y = d(x)\).
        \item Déterminer les fonctions de \(\R\) dans \(\C\) solutions de l'équation différentielle \(2y'-y = x\e^{(3\ii-1)x}\).
        \item En déduire les fonctions de \(\R\) dans \(\R\) solutions de l'équation différentielle \(2y'-y = x\e^{-x}\sin(3x)\).
    \end{enumerate}
\end{exercice}
