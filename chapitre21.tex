\chapter{Intégration et dérivation}
\label{chap:integrationetderivation}
\minitoc
\minilof
\minilot

Dans tout le chapitre, \(I\) est un intervalle réel contenant au moins deux 
points.

\section{Primitives et intégrales}

\subsection{Notion de primitive}

\subsubsection{Définition}

\begin{defdef}
  Soit \(f \in \R^I\). On appelle primitive de \(f\) toute application \(F \in 
  \R^I\) telle que~:
  \begin{enumerate}
    \item \(F\) est dérivable sur \(I\);
    \item \(F'=f\).
  \end{enumerate}
\end{defdef}

\subsubsection{Détermination des primitives d'une fonction \(f\) si on en 
connaît une}

\begin{theo}
  Soit \(f \in \R^I\). On suppose qu'il existe \(F \in \R^I\) une primitive de 
  \(f\). Alors l'ensemble des primitives de \(f\) est \(\enstq{F+\tilde{c}}{c 
  \in \R}\).
\end{theo}
\begin{proof}
  Soit une fonction \(G \in \R^I\). Alors
  \begin{align}
    G \text{~est une primitive de~} f &\iff \begin{cases} G \in \Dr(I,\R) \\ 
    G'=f \end{cases} \\
    &\iff \begin{cases} G \in \Dr(I,\R) \\ G'=F' \end{cases} \\ &\iff 
      \begin{cases} G \in \Dr(I,\R) \\ \exists c \in \R \quad G=F+\tilde{c} 
      \end{cases} \\
      &\iff \exists c \in \R \quad G=F+\tilde{c}.
  \end{align}
  La première équivalence est la définition d'une primitive. La deuxième est 
  claire par définition de \(F\). La troisième est due à l'intégration. Enfin la 
  quatrième est vraie puisque pour toute constante \(c\), \(F+\tilde{c}\) est 
  dérivable.
\end{proof}

\begin{corth}
  Soit \(f \in \R^I\). On suppose qu'il existe une primitive \(F \in \R^I\) de 
  \(f\). Soient \(x_0 \in I\) et \(y_0 \in \R\). Alors il existe une et une 
  seule primitive \(G\) de \(f\) telle que \(G(x_0)=y_0\). C'est l'application 
  \(\fonction{G}{I}{\R}{x}{F(x)+y_0-F(x_0)}\).
\end{corth}

\subsection{Étude des applications \(x \longmapsto \int_a^x f(t)\,\diff t\)}

Soit une fonction \(f \in \R^I\). On suppose que \(f\) est continue par morceaux 
sur tout segement contenu dans \(I\). Soit \(a \in I\). Pour tout \(x \in I\), 
\(\int_a^xf(t)\,\diff t\) a bien un sens (cf chapitre~
\ref{chap:integration}).

On peut définir l'application \(\fonction{\Phi_f}{I}{\R}{x}{\int_a^x f(t)\,\diff 
t}\).

\begin{prop}
  Soit \(f \in \CM(I,\R)\). Si on suppose que \(f \geqslant 0\) alors 
  l'application \(\Phi_f\) définie comme ci-dessus est croissante.
\end{prop}
\begin{proof}
  Soit \((x_1,x_2)\in I^2\) tel que \(x_1<x_2\). Alors
  \begin{align}
    \Phi_f(x_2)-\Phi_f(x_1) &= \int_a^{x_2} f(t)\,\diff t - \int_a^{x_1} 
    f(t)\,\diff t \\
    &= \int_a^{x_2} f(t)\,\diff t + \int_{x_1}^a f(t)\,\diff t \\
    &= \int_{x_1}^{x_2} f(t)\,\diff t
  \end{align}
  et comme \(f\) est positive alors \(\Phi_f(x_2)-\Phi_f(x_1) \geqslant 0\) par 
  positivité de l'intagrale. Finalement \(\Phi_f\) est croissante.
\end{proof}

\begin{prop}
Soit \(f \in \CM(I,\R)\). Si on suppose que \(f\) est bornée alors l'application 
\(\Phi_f\) définie comme ci-dessus est lipschitzienne. \end{prop}
\begin{proof}
  Comme \(f\) est bornée, il existe un réel \(k\) positif tel que pour tout \(x 
  \in I\), \(\abs{f(x)}\leqslant k\). Soient \((x_1,x_2) \in I^2\), alors
  \begin{equation}
    \abs{\Phi_f(x_2)-\Phi_f(x_1)} = \abs{\int_{x_1}^{x_2} f(t)\,\diff t} 
    \leqslant \abs{x_2-x_1} \norme{f}_{\infty} \leqslant k \abs{x_2 -x_1}.
  \end{equation}
Ce qui montre que \(\Phi\) est lipschitzienne. \end{proof}

\begin{prop}
  Soit \(f \in \CM(I,\R)\). L'application \(\Phi_f\) défini comme ci-avant est 
  continue.
\end{prop}
\begin{proof}
  Soit \(x_0 \in I\). Deux cas de figures se présentent~:
  \begin{itemize}
    \item Soit \(x_0 \in \mathring{I}\) (ce qui signifie qu'il est dans 
      l'intérieur et que ce n'est pas une borne, ou plus formellement qu'il 
      existe un réel \(\eta>0\) tel que \(\intervalleoo{x_0-\eta}{x_0+\eta} 
      \subset I\)). On peut alors choisir \((\alpha,\beta) \in I^2\) tel que 
      \(\alpha < x_0 < \beta\) et on pose \(J=\intervalleff{\alpha}{\beta}\).
    \item Soit \(x_0 \in I\setminus\mathring{I}\) (ce qui signifie que c'est une 
      borne) alors deux sous-cas se présentent~:
      \begin{itemize}
        \item soit \(x_0 = \min I\) alors il existe \(\beta \in \R\) tel que 
          \(\beta > x_0\) et donc on pose \(J=\intervalleff{x_0}{\beta} \subset 
          I\);
        \item soit \(x_0 = \max I\) alors il existe \(\alpha \in \R\) tel que 
          \(\alpha < x_0\) et donc on pose \(J=\intervalleff{\alpha}{x_0} 
          \subset I\);
      \end{itemize}
  \end{itemize}

  Dans tous les cas et sous-cas, \(J\) est un segment inclus dans \(I\). Alors 
  \(f\) est continue par morceaux sur \(J\). Par conséquent \(f\) est bornée sur 
  \(J\) donc l'application \({\Phi_f}_{|J}\) est lipschitzienne et donc 
  continue.

  Or \(x_0 \in J\), donc \(\Phi_f\) est continue en \(x_0\) pour tout \(x_0 \in 
  I\). Donc \(\Phi\) est continue sur \(I\).
\end{proof}

\begin{prop}
  Soit \(f \in \CM(I,\R)\) et \(x_0 \in I\). Si on suppose que \(f\) est 
  continue en \(x_0\) alors \(\Phi_f\) est dérivable en \(x_0\) et 
  \(\Phi'(x_0)=f(x_0)\).
\end{prop}
\begin{proof}
  Soit \(x \in I \setminus\{x_0\}\). Alors le taux d'accroissement de \(\Phi_f\) 
  est
  \begin{equation}
    \frac{\Phi_f(x)-\Phi_f(x_0)}{x-x_0} = \frac{1}{x-x_0} \int_{x_0}^x 
    f(t)\,\diff t.
  \end{equation}
  L'application \(f\) est continue en \(x_0\) alors il existe une application 
  \(\epsilon \in \R^I\) telle que
  \begin{equation}
    \begin{cases} \forall x \in I \quad f(x)=f(x_0)+\epsilon(x) \\ 
    \lim\limits_{x_0} \epsilon =0 \end{cases}.
  \end{equation}
  Alors pour tout \(x \in I \setminus\{x_0\}\), on a
  \begin{align}
    \frac{\Phi_f(x)-\Phi_f(x_0)}{x-x_0} &= \frac{1}{x-x_0} \int_{x_0}^x 
    (f(x_0)+\epsilon(t))\,\diff t \\
    &=  \frac{1}{x-x_0} \int_{x_0}^x f(x_0)\,\diff t + \frac{1}{x-x_0} 
    \int_{x_0}^x\epsilon(t)\,\diff t.
  \end{align}

  Comme la fonction \(f\) est continue en \(x_0\), on a par définition
  \begin{equation}
    \forall \epsilon >0 \ \exists \eta>0 \ \forall x \in I \quad \abs{x-x_0} 
    \leqslant \eta \implies \abs{f(x)-f(x_0)} = \abs{\epsilon(x)} \leqslant 
    \epsilon.
  \end{equation}
  Or pour tout \(x \in I \setminus\{x_0\}\) si \(\abs{x-x_0}\leqslant \eta\), 
  pour tout \(t \in [(x_0;x)]\), on a \(\abs{t-x_0} \leqslant \eta\) et par 
  conséquent \(\abs{\epsilon(t)}\leqslant \epsilon\).

  Ainsi pour tout \(x \in I \setminus\{x_0\}\), si \(\abs{x-x_0}\leqslant \eta\) 
  alors on a \(\abs{\frac{1}{x-x_0} \int_{x_0}^x\epsilon(t)\,\diff t} \leqslant 
  \frac{\abs{x-x_0}}{\abs{x-x_0}}\epsilon=\epsilon\). Donc
  \begin{equation}
    \lim\limits_{x \to x_0} \frac{1}{x-x_0} \int_{x_0}^x\epsilon(t)\,\diff t=0
  \end{equation}
  et donc
  \begin{equation}
    \lim\limits_{x \to x_0}\frac{\Phi_f(x)-\Phi_f(x_0)}{x-x_0} =f(x_0).
  \end{equation}
  On vient de démontrer que \(\Phi\) est dérivable en \(x_0\) et que 
  \(\Phi'(x_0)=f(x_0)\).
\end{proof}

\subsection{Primitive d'une fonction continue sur un intervalle}

\begin{theo}[Théorème fondamental]
  Soit \(f \in \R^I\) \emph{continue} sur \(I\). Alors \(f\) admet des 
  primitives. De plus~:
  \begin{enumerate}
    \item pour tout \(x_0 \in I\), l'unique primitive de \(f\) qui s'annule en 
      \(x_0\) est l'application
      \begin{equation}
        \fonction{F}{I}{\R}{x}{\int_{x_0}^x f(t)\,\diff t};
      \end{equation}
    \item toutes les primitives de \(f\) sont de classe \(\classe{1}\).
  \end{enumerate}
\end{theo}
\begin{proof}
  Soit \(a \in I\) et soit l'application 
  \(\fonction{\Phi_f}{I}{\R}{x}{\int_a^xf(t)\,\diff t}\). La fonction \(f\) est 
  continue sur \(I\), alors pour tout \(x_0 \in I\), \(\Phi_f\) est dérivable en 
  \(x_0\) avec \(\Phi_f'(x_0)=f(x_0)\). Alors comme c'est pour tout \(x_0\), 
  \(\Phi\) est dérivable sur \(I\) et \(\Phi_f'=f\). Comme \(f\) est continue, 
  l'application \(\Phi_f\) est de classe \(\classe{1}\).

  Les autres primitives de \(f\) différent de \(\Phi_f\) d'une constantes, donc 
  elles sont aussi de classe \(\classe{1}\).

  Enfin, on a déjà vu l'unicité de la primitive de \(f\) qui s'annule en 
  \(x_0\).
\end{proof}

\begin{prop}
  Soit \(f \in \courbe(I,\R)\). Pour tout couple \((a,b) \in I^2\) et toute 
  primitive \(F\) de \(f\), on a~:
  \begin{equation}
    \int_a^b f(t)\,\diff t=F(b)-F(a)
  \end{equation}
\end{prop}
\begin{proof}
  Soit l'application \(\fonction{\Phi_f}{I}{\R}{x}{\int_a^xf(t)\,\diff t}\). 
  C'est la primitive de \(f\) qui s'annule en \(a\) et \(\int_a^b f(t)\,\diff 
  t=\Phi(b)-\Phi(a)\). Soit \(F\) un autre primitive de \(f\), alors il existe 
  un réel \(c\) tel que \(F=\Phi_f +\tilde{c}\) et dans ce cas 
  \(F(b)-F(a)=\Phi_f(b)+c - \Phi_f(a)-c = \int_a^bf(t)\,\diff t\).
\end{proof}

\emph{Notation}~: Dans un calcul d'intégrale, si \(f\) est une fonction 
continue,  et si \(F\) une primitive de \(F\) sur l'intervalle \(I\), alors on 
écrira \(\int_a^b f(t)\,\diff t=[F(t)]_a^b\).

\begin{corth}
  Soient une application \(f \in \classe{1}(I,\R)\) et \((a,b) \in I^2\), alors
  \begin{equation}
    \int_a^b f'(t)\,\diff t=f(b)-f(a)
  \end{equation}
\end{corth}

\begin{prop}
  Soient \(I\) et \(J\) deux intervalles réels, et deux applications \(u,v \in 
  \classe{1}(I,\R)\) telles que \(v(J) \subset I\) et \(u(J) \subset I\). Soit 
  \(f \in \courbe(I,\R)\), alors l'application
  \begin{equation}
    \fonction{\Phi}{J}{\R}{x}{\int_{u(x)}^{v(x)}f(t)\,\diff t}
  \end{equation}
  est dérivable sur \(J\).
\end{prop}
\begin{proof}
  Soit une primitive \(F\) de \(f\) sur \(I\). Pour tout \(x \in I\), \(u(x) \in 
  I\) et \(v(x) \in I\) alors
  \begin{equation}
    \Phi(x) = F(v(x))-F(u(x)).
  \end{equation}
  Les applications \(u\) et \(v\) sont dérivables sur \(I\), et \(F\) est 
  dérivable sur \(I\). Alors, par composition et différence d'applications 
  dérivables, \(\Phi\) est dérivable sur \(J\). Ainsi pour tout \(x \in J\),
  \begin{equation}
    \Phi'(x) = v'(x)f(v(x)) -u'(x)f(u(x)).
  \end{equation}
  L'application \(f\) est continue et les applications \(u'\) et \(v'\) aussi. 
  Alors \(\Phi'\) est continue. Alors \(\Phi\) est de classe \(\classe{1}\).
\end{proof}

\subsection{Intégration par parties}

\begin{theo}
  Soient \(I\) un intervalle réel, et deux applications \(u\) et \(v\) de 
  \(\classe{1}(I,\R)\). ALors pour tout couple \((a,b) \in I^2\) on a
  \begin{equation}
    \int_a^b u(t)v'(t)\,\diff t = [u(t)v(t)]_a^b - \int_a^b u'(t)v(t)\,\diff t.
  \end{equation}
\end{theo}
\begin{proof}
  Comme les applications \(u\) et \(v\) sont de classe \(\classe{1}\), 
  l'application \(uv\) est de classe \(\classe{1}\). Alors
  \begin{align}
    [u(t)v(t)]_a^b = u(b)v(b)-u(a)v(a) &=\int_a^b (uv)'(t)\,\diff t \\
    &=\int_a^b[u'(t)v(t)u(t)v'(t)]\,\diff t \\
    &=\int_a^b u'(t)v(t)\,\diff t + \int_a^b u(t)v'(t)\,\diff t.
  \end{align}
\end{proof}

En pratique, on utilise l'intégration par parties pour~:
\begin{enumerate}
  \item se ramener à des fonctions plus simples. Comme par exemple pour calculer 
    pour tout \(x>0\), l'intégrale \(\int_1^x \ln(t)\,\diff t\).
  \item calculer des intégrales par récurrence, comme par exemple les intégrales 
    de Wallis.
\end{enumerate}

\subsection{Changement de variable}

\begin{theo}
  Soient \(I\) et \(J\) deux intervalles réels, \(f \in \courbe(I,\R)\) et 
  \(\varphi \in \classe{1}(J,\R)\) tels que \(\varphi(J) \subset I\). Alors pour 
  tout réels \(a, b \in J\) on a
  \begin{equation}
    \int_a^b f(\varphi(t)) \varphi'(t)\,\diff t = \int_{\varphi(a)}^{\varphi(b)} 
    f(u)\,\diff u.
  \end{equation}
\end{theo}
\begin{proof}
  L'application \(f\) est continue sur \(I\) alors il existe une primitive \(F\) 
  de \(f\) sur l'intervalle \(I\). Pour tout \((a,b) \in J^2\), on a 
  \((\varphi(a), \varphi(b)) \in I^2\). Donc
  \begin{equation}
    \int_{\varphi(a)}^{\varphi(b)}f(u)\,\diff u = F(\varphi(b)) - F(\varphi(a)).
  \end{equation}
  Comme \(\varphi \in \classe{1}(J,I)\) et \(F \in \classe{1}(J,\R)\), alors \(F 
  \circ \varphi\) est dans  \(\classe{1}(J,\R)\) et \((F\circ \varphi)'=\varphi' 
  \cdot f \circ \varphi\). Alors
  \begin{align}
    \int_a^b f(\varphi(t)) \varphi'(t)\,\diff t &= \int_a^b (F\circ 
    \varphi)'\,\diff t\\
    &=F\circ \varphi(b) - F\circ \varphi(a) \\
    &=\int_{\varphi(a)}^{\varphi(b)}f(u)\,\diff u.
  \end{align}
\end{proof}

\danger \emph{Il ne faut pas oublier les hypothèses, c-à-d \(\varphi\) est 
\(\classe{1}\).}

\danger \emph{Ne pas oublier de changer les bornes de l'intégrale.}

\danger \emph{Ne pas oublier le '\(\varphi'(t)\)'.}

\emph{Exemple}~: Calculer \(\int_0^{\pi/2} \cos^2 t \sin t\,\diff t\). 

Soit l'application \(\fonction{\varphi}{\intervalleff{0}{\pi/2}}{\R}{t}{\cos 
t}\). Alors \(\varphi\) est bien de classe \(\classe{1}\) et
\begin{equation}
  \int_0^{\pi/2} \cos^2 t \sin t\,\diff t = -\int_0^{\pi/2} 
  \varphi(t)^2\varphi'(t)\,\diff t = -\int_1^0 u^2\,\diff u =\frac{1}{3}.
\end{equation}


\emph{Cas particulier}~: Changement de variable affine. C'est-à-dire que 
l'application \(\varphi\) est de la forme \(\varphi : x \longmapsto ax+b\) avec 
\(a\) et \(b\) deux réels. Ces applications sont toujours de classe 
\(\classe{1}\).

\emph{Applications}~: Notons trois applications singulières
\begin{enumerate}
  \item Les fonctions paires et les fonctions impaires. Soit un réel \(a\) 
    positif strictement et une application \(f : \intervalleff{-a}{a} 
    \longmapsto \R\) continue. Alors
    \begin{align}
      \int_{-a}^a f(t)\,\diff t &=\int_{-a}^0 f(t)\,\diff t + \int_{0}^a 
      f(t)\,\diff t \\
      &=\int_a^0 f(-s)(-\diff s) + \int_{0}^a f(t)\,\diff t \\
      &=\int_0^a f(-s)\,\diff s + \int_{0}^a f(t)\,\diff t \\
    \end{align}
    Si \(f\) est pair alors \(\int_{-a}^a f(t)\,\diff t = 2 \int_{0}^a 
    f(t)\,\diff t\). Si \(f\) est impaire alors l'intégrale est nulle.

  \item Les fonctions périodiques. Soit \(T >0\) et \(f \in \R^{\R}\) continue 
    est \(T\)-périodique. Pour tout réel \(a\) et \(b\), on a
    \begin{equation}
      \int_{a+T}^{b+T} f(t)\,\diff t = \int_a^bf(x+T)\,\diff x = 
      \int_a^bf(x)\,\diff x
    \end{equation}
  \item Passage de \(\intervalleff{a}{b}\) à \(\intervalleff{0}{1}\). Soient 
    deux réels \(a\) et \(b\) tels que \(b>a\) et \(f:\intervalleff{a}{b} 
    \longmapsto \R\) une application continue. Soit l'application \(\varphi\) 
    telle que
    \begin{equation}
      \fonction{\intervalleff{a}{b}}{\intervalleff{0}{1}}{t}{\frac{t-a}{b-a}}.
    \end{equation}
    Pour tout \(t \in \intervalleff{a}{b}\), on a \(\varphi'(t) = 
    \frac{1}{b-a}\), \(\varphi(a)=0\) et \(\varphi(b)=1\).
    \begin{equation}
      \forall u \in \intervalleff{0}{1} \ \forall t \in \intervalleff{a}{b} 
      \quad u=\varphi(t) \iff t=(b-a)u +a = bu +(1-u)a
    \end{equation}
    et alors
    \begin{equation}
      \int_a^b f(t)\,\diff t = \int_0^1 f(bu+(1-u)a)(b-a)\,\diff u = (b-a) 
      \int_0^1 f(bu+(1-u))a\,\diff u.
    \end{equation}
\end{enumerate}

De façon générale, les changements de variables affines servent à exploiter des 
symétries de la fonctions à intégrer.

\section{Formules de Taylor}

\subsection{Notion de développement de Taylor}

\begin{defdef}
  Soit \(I\) un intervalle réel, \(a \in I\), \(f \in \R^I\) et \(n \in \N\). 
  Supposons que \(f\) est dérivable au moins \(n\) fois en \(a\).

  On appelle développement de Taylor de \(f\) au point \(a\) à l'ordre \(n\), 
  l'application polynomiale
  \begin{equation}
    \fonction{T_{n,f,a}}{\R}{\R}{x}{\sum_{k=0}^n \frac{f^{(k)}(a)}{k!}(x-a)^k}
  \end{equation}

  Pour tout réel \(x\),
  \begin{equation}
    T_{n}(x)= f(a) + f'(a)(x-a) + \dotsb +\frac{f^{(n)}(a)}{n!}(x-a)^n.
  \end{equation}
Si \(0 \in I\) et si \(a=0\) on parle de développement de Mac-Laurin. 
\end{defdef}

\emph{Remarque}~: Pour tout naturel \(k\) non nul, on a \(f^{(k)} = 
(f')^{(k-1)}\), donc pour tout réel \(x\)
\begin{equation}
  T_{n,f,a}'(x)=\sum_{k=1}^n \frac{k f^{(k)}(a)}{k!}(x-a)^{k-1} = 
  \sum_{j=0}^{n-1} \frac{f^{(j+1)}(a)}{j!}(x-a)^j = T_{n-1,f',a}(x)
\end{equation}

\emph{Développements de Mac-Laurin des fonctions usuelles}

Soit un naturel \(n\). On pose \(n'=E\left(\frac{n}{2}\right)\) et 
\(n''=E\left(\frac{n-1}{2}\right)\). C'est-à-dire que ce sont les entiers tels 
que
\begin{align}
  2n' \leqslant n \leqslant 2n'+2 \\
  2n'' +1 \leqslant n \leqslant 2n''+3
\end{align}
\(2n'\) est le plus grand entier pair inférieur à \(n\) et \(2n''+1\) est le 
plus grand entier impair inférieur à \(n\). Alors la table~
\ref{tab:MacLaurin} donne les développements.

\begin{table}[!h]
  \centering
  \begin{tabular}{|c|c|c|c|}\hline
    \(f(x)\) &  \(f^{(k)}(0), k\geqslant 1\) & \(T_n(x)\) & \(I\) \\ \hline
    \(\e^x\) &  \(1\) & \(\sum_{k=0}^n \frac{1}{k!}x^k\) & \(\R\) \\ \hline
    \(\cosh x\)  & \(1\) si \(k\) pair & \(\sum_{k=0}^{n'} 
    \frac{1}{(2k)!}x^{2k}\) & \(\R\) \\
    &  \(0\) si \(k\) impair & & \\ \hline
    \(\sinh x\)  & \(0\) si \(k\) pair & \(\sum_{k=0}^{n''} 
    \frac{1}{(2k+1)!}x^{2k+1}\) & \(\R\) \\
    &  \(1\) si \(k\) impair & & \\ \hline
    \(\cos x\)  & \(\cos\left(\frac{k\pi}{2}\right)\) & \(\sum_{k=0}^{n'} 
    \frac{(-1)^k}{(2k)!}x^{2k}\) & \(\R\) \\ \hline
    \(\sin x\)  & \(\sin\left(\frac{k\pi}{2}\right)\) & \(\sum_{k=0}^{n''} 
    \frac{(-1)^k}{(2k+1)!}x^{2k+1}\) & \(\R\) \\ \hline
    \(\ln(1+x)\)  & \((-1)^{k-1}(k-1)\)! & \(\sum_{k=1}^n \frac{(-1)^{k-1}}{k} 
    x^k\) & \(\intervalleoo{-1}{+\infty}\) \\ \hline
    \((1+x)^\alpha\) & \(\prod_{p=0}^{k-1} (\alpha-p)\) & \(\sum_{k=0}^n 
    \frac{\prod_{p=0}^{k-1} (\alpha-p)}{k!} x^k\) & 
    \(\intervalleoo{-1}{+\infty}\) \\ \hline
  \end{tabular}
  \caption{Développements de Mac-Laurin usuels}
  \label{tab:MacLaurin}
\end{table}

\subsection{Formule de Taylor avec reste intégral}

\begin{theo}
  Soient \(I\) un intervalle réel, \(a \in I\), \(n \in \N\) et \(f \in \R^I\). 
  Supposons que \(f\) est de classe \(\classe{n+1}\). Alors pour tout \(x \in 
  I\),
  \begin{equation}
    f(x) = \sum_{k=0}^n \frac{f^{(k)}(a)}{k!} (x-a)^k + \int_a^x 
    \frac{(x-t)^n}{n!} f^{(n+1)}(t)\,\diff t.
  \end{equation}

  C'est la formule de Taylor avec reste intégral de \(f\) à l'ordre \(n\) au 
  point \(a\).
\end{theo}

\begin{proof}
  Posons pour tout x\( \in I\),
  \begin{equation}
    R_n(x) = f(x) -\sum_{k=0}\frac{f^{(k)}(a)}{k!} (x-a)^k
  \end{equation}
  et montrons par récurrence sur \(n \in \N\) l'assertion \(\P(n)\) ``si \(f \in 
  \classe{n+1}(I,\R)\) alors \(\R_n(x)=\int_a^x \frac{(x-t)^n}{n!} 
  f^{(n+1)}(t)\,\diff t\).''

  \emph{Initialisation}~: \(n=0\), suppsons que \(f\) soit \(\classe{1}\), alors 
  pour tout \(x \in I\), on a
  \begin{equation}
    R_0(x) = f(x)-f(a) = \int_a^x f'(t)\,\diff t = \int_a^x 
    \frac{(x-t)^0}{0!}f^{(0+1)}(t)\,\diff t.
  \end{equation}
  Alors \(\P(0)\) est vraie.

  \emph{Hérédité}~: Soit \(n \in \N\) et supposon \(\P(n)\). Si \(f\) est de 
  classe \(\classe{n+2}\) alors pour tout \(x \in I\), on a
  \begin{equation}
    R_{n+1}(x)=f(x) - \sum_{k=0}\frac{f^{(k)}(a)}{k!} (x-a)^k = f(x) - T_n(x) - 
    \frac{f^{(n+1)}(a)}{(n+1)!}(x-a)^{n+1}.
  \end{equation}

  La fonction \(f\) est de classe \(\classe{2}\), donc elle est de classe 
  \(\classe{1}\) et l'hypothèse de récurrence nous donne que pour tout \(x \in 
  I\), on a
  \begin{equation}
    f(x) - T_n(x) = \int_a^x \frac{(x-t)^n}{n!} f^{(n+1)}(t)\,\diff t
  \end{equation}
  et alors
  \begin{equation}
    R_{n+1}(x) = int_a^x \frac{(x-t)^n}{n!} f^{(n+1)}(t)\,\diff t - 
    \frac{f^{(n+1)}(a)}{(n+1)!}(x-a)^{n+1}.
  \end{equation}

  Soit les applications \(\fonction{u}{I}{\R}{t}{f^{(n+1)}(t)}\) et 
  \(\fonction{v}{I}{\R}{t}{-\frac{(x-t)^{n+1}}{(n+1)!}}\). La fonction \(v\) est 
  de classe \(\classe{1}\) puisqu'elle est polynomiale, et \(u\) est aussi de 
  classe \(\classe{1}\) car \(f\) est de classe \(\classe{n+2}\). De plus, pour 
  \(t \in I\), on a
  \begin{equation}
    u'(t) = f^{(n+2)}(t) \quad v'(t)=\frac{(x-t)^n}{n!},
  \end{equation}
  alors pour tout \(x \in I\), en intégrant par parties on a
  \begin{align}
    R_{n+1}(x)& = \left[ -\frac{(x-t)^{n+1}}{(n+1)!}f^{(n+1)}(t)\right]_a^x - 
    \int_a^x f^{(n+2)}(t)\left(-\frac{(x-t)^{n+1}}{(n+1)!}\right)\,\diff t \\ &- 
    \frac{f^{(n+1)}(a)}{(n+1)!}(x-a)^{n+1} \notag \\
    &= \int_a^x f^{(n+2)}(t) \frac{(x-t)^{n+1}}{(n+1)!}\,\diff t
  \end{align}
  donc \(\P(n+1)\) est vraie.

  \emph{Conclusion}~: On a montré \(\P(0)\) et pour tout naturel \(n\), \(\P(n) 
  \Rightarrow \P(n+1)\). Alors par théorème de récurrence, pour tout naturel 
  \(n\), \(\P(n)\).
\end{proof}

\subsection{Inégalité de Taylor-Lagrange}

\begin{theo}
  Soient un intervalle \(I\), \(a \in I\), \(n \in \N\) et \(f \in 
  \classe{n+1}(I,\R)\). La fonction \(f^{(n+1)}\) est continue sur un segment 
  donc bornée. Pour tout réel \(x \in I\) alors
  \begin{equation}
    \abs{f(x)-T_{n,f,a}(x)} \leqslant \frac{\abs{x-a}^{n+1}}{(n+1)!} \sup 
    \abs{f^{(n+1)}}.
  \end{equation}

  C'est l'inégalité de Taylor-Lagrange à l'ordre \(n\) appliquée à \(f\) au 
  point \(a\).
\end{theo}
\begin{proof}
  La fonction \(f\) est de classe \(\classe{n+2}\), alors d'après la formule de 
  Taylor-Lagrange avec reste intégral, on a pour tout \(x \in I\)
  \begin{equation}
    f(x)-T_{n,f,a}(x) = \int_a^x \frac{(x-t)^{n}}{n!} f^{(n+1)}(t)\,\diff t
  \end{equation}
  Deux cas se présentent~:
  \begin{enumerate}
    \item Si \(x \geqslant a\) alors
      \begin{align}
        \abs{R_n(x)} &\leqslant \int_a^x 
        \frac{\abs{x-t}^n}{n!}\abs{f^{(n+1)}(t)}\,\diff t  \\
        &\leqslant \sup\limits_{\intervalleff{a}{x}}\abs{f^{(n+1)}} \int_a^x 
        \frac{(x-t)^n}{n!}\,\diff t\\
        &\leqslant \sup\limits_{\intervalleff{a}{x}}\abs{f^{(n+1)}} 
        \frac{(x-a)^{n+1}}{(n+1)!};
      \end{align}
    \item Si \(x \leqslant a\) alors
      \begin{align}
        \abs{R_n(x)} &\leqslant \int_x^a 
        \frac{\abs{x-t}^n}{n!}\abs{f^{(n+1)}(t)}\,\diff t  \\
        &\leqslant \sup\limits_{\intervalleff{a}{x}}\abs{f^{(n+1)}} \int_a^x 
        \frac{(t-x)^n}{n!}\,\diff t\\
        &\leqslant \sup\limits_{\intervalleff{a}{x}}\abs{f^{(n+1)}} 
        \frac{(a-x)^{n+1}}{(n+1)!}.
      \end{align}
  \end{enumerate}
  Dans les deux cas,
  \begin{equation}
    \abs{R_n(x)} \sup\limits_{[(a,x)]}\abs{f^{(n+1)}} 
    \frac{\abs{x-a}^{n+1}}{(n+1)!}.
  \end{equation}
\end{proof}

\section{Retour sur les développements limités}

\subsection{Développement limité d'une primitive}

\begin{theo}
  Soient un intervalle réel \(I\), \(a \in I\), \(n \in \N\) et \(f \in \R^I\). 
  Supposons que~:
  \begin{enumerate}
    \item \(f\) admet un développement limité à l'ordre \(n\) au voisinage de 
      \(a\);
    \item \(f\) admet une primitive \(F\) sur \(I\).
  \end{enumerate}
  Alors \(F\) admet un développement limité à l'ordre \(n+1\) au voisinage de 
  \(a\). De plus la partie régulière \(P_{n+1}(F)\) de ce développement limité 
  est la primitive de \(P_n(f)\) qui vaut \(F(a)\) en \(a\).

  Si au voisinage de \(a\) on a
  \begin{equation}
    f(x) = \sum_{k=0}^n \alpha_k(x-a)^k +\petito{(x-a)^n}
  \end{equation}
  alors au voisinage de \(a\), on a aussi
  \begin{equation}
    F(x) = \sum_{k=0}^n \frac{\alpha_k}{k+1}(x-a)^{k+1} +F(a) + \petito{(x-a)^n}
  \end{equation}
\end{theo}
\begin{proof}
  Soit pour tout naturel \(n\), l'application \(A_n : x \longmapsto \sum_{k=0}^n 
  \alpha_k(x-a)^k\). Soit \(B_{n+1}\) une primitive de \(A_n\) qui vaut \(F(a)\) 
  en \(a\). Il existe une application \(\epsilon: x \longmapsto \R\) telle que 
  pour tout \(x \in I\) on a
  \begin{equation}
    \begin{cases}
      f(x) = A_n(x)+(x-a)^n\epsilon(x) \\
      \lim\limits_{0} \epsilon = 0
    \end{cases}.
  \end{equation}
  Alors
  \begin{equation}
    F(x) - B_{n+1}(x) = [F(x)-B_{n+1}(x)] - [F(a)-B_{n+1}(a)]
  \end{equation}
  \(F-B_{n+1}\) est continue sur \([(a,x)]\) et dérivable sur \(](a,x)[\) alors 
  d'après le théorème de Rolle, il existe un réel \(c_x \in [(a,x)]\) tel que
  \begin{align}
    F(x)-B_{n+1}(x) &= (F-B_{n+1})'(c_x)(x-a)\\
    & = (f(c_x)-A_n(c_x))(x-a)\\
    &=(c_x-a)^n\epsilon(c_x)(x-a)\\
    &=(x-a)^{n+1} \left(\frac{c_x-a}{x-a}\right)^n \epsilon(c_x).
  \end{align}
  On note \(\epsilon_1(x)=\left(\frac{c_x-a}{x-a}\right)^n \epsilon(c_x)\). 
  Puisque \(c_x \in [(a,x)]\), le théorème des gendarmes nous dit que 
  \(\lim\limits_{x \to a} c_x =a\). De plus on sait que \(\lim\limits_{x \to 
  a}\epsilon(x)=0\). Alors par composition de limites, \(\lim\limits_{x \to a} 
  \epsilon(c_x)=0\). 

  De plus pour tout \(x \in [(a,x)], \abs{\frac{c_x-a}{x-a}} \leqslant 1\) alors 
  finalement \(F(x)-B_{n+1}(x) = \petito{(x-a)^{n+1}}\).
\end{proof}

\begin{corth}
  Soient \(I\) un intervalle réel, \(a \in I\), \(n \in \N\), \(f \in \R^I\) et 
  \(F\) une primitive de \(f\) sur \(I\). Si au voisinage de \(a\) on a 
  \(f(x)=\petito{(x-a)^n}\) alors \(F(x)-F(a)=\petito{(x-a)^n}\) au voisinage de 
  \(a\). 

En particulier si \(f\) est continue, et si au voisinage de \(a\) on a  
\(f(x)=\petito{(x-a)^n}\) alors \(\int_a^x f(t)\,\diff t=\petito{(x-a)^n}\) au 
voisinage de \(a\). \end{corth}

\begin{corth}
  Soient \(I\) un intervalle réel, \(a \in I\), \(n \in \N\), \(f \in \R^I\) et 
  \(F\) une primitive de \(f\) sur \(I\). Soit aussi \(\alpha \in \R^{*}\). 
  Alors
  \begin{equation}
    f(x) \sim_a \alpha(x-a)^n \Rightarrow F(x)-F(a) \sim_a \alpha 
    \frac{(x-a)^{n+1}}{n+1}
  \end{equation}

  En particulier si \(f\) est continue alors
  \begin{equation}
    f(x) \sim_a \alpha(x-a)^n \Rightarrow \int_a^x f(t)\,\diff t \sim_a \alpha 
    \frac{(x-a)^{n+1}}{n+1}
  \end{equation}
\end{corth}

\begin{prop}
  Soient \(I\) un intervalle réel, \(a \in I\), \(n \in \N^*\) et \(f \in 
  \R^I\). On suppose que~:
  \begin{enumerate}
    \item \(f\) admet un \(DL_n(a)\);
    \item \(f\) est dérivable sur \(I\);
    \item \(f'\) admet un \(DL_{n-1}(a)\).
  \end{enumerate}
  Alors si on note \(P\) les parties régulières on a
  \begin{equation}
    P_{n-1}(f')=(P_n(f))'.
  \end{equation}
\end{prop}
\begin{proof}
  Il suffit d'appliquer le théorème sur les primitives de \(f'\).
\end{proof}

\subsection[Développement limité d'une fonction \(\classe{n}\)]{Développement 
limité d'une fonction \(\classe{n}\), Taylor-Young}

\begin{theo}
  Soient \(I\) un intervalle réel, \(a \in I\), \(n \in \N\), \(f \in 
  \classe{n}(I,\R)\). Alors la fonction \(f\) admet le \(DL_n(a)\) suivant~:
  \begin{equation}
    f(x) = \sum_{k=0}^n \frac{f^{(k)}(a)}{k!} (x-a)^k + \petito{(x-a)^n} = 
    T_{n,f,a}(x)+\petito{(x-a)^n}.
  \end{equation}
  C'est la formule de Taylor-Young à l'ordre \(n\) appliquée à \(f\) en \(a\).
\end{theo}

\begin{proof}
  On montre cette assertion, qu'on note \(\P(n)\) pour tout naturel \(n\), par 
  récurrence sur \(n\).

  \emph{Initialisation}~: \(n=0\). Si \(f\) est continue alors on a déjà vue que 
  \(f(x)=f(a)+o(1)\). Alors \(\P(0)\) est vraie.

  \emph{Hérédité}~: Soit \(n \in \N\) et supposons \(\P(n)\). Comme \(f\) est de 
  classe \(\classe{n}\), on peut lui appliquer la formule de Taylor avec reste 
  intégral à l'ordre \(n-1\) en \(a\), c'est-à-dire que
  \begin{equation}
    \forall x \in I \quad f(x) = T_{n-1,f,a}(x) + \int_a^x 
    \frac{(x-t)^{n-1}}{(n-1)!} f^{(n)}(t)\,\diff t.
  \end{equation}

  La fonction \(f^{(n)}\) est continue donc il existe une application 
  \(\epsilon: I \longmapsto \R\) telle que
  \begin{equation}
    \begin{cases}
      \forall t \in I \quad f^{(n)}(t) = f^{(n)}(a)+\epsilon(t) \\
      \lim\limits_{t \to 0}\epsilon(t)=0
    \end{cases}
  \end{equation}
  Alors pour tout \(x \in I\), on a
  \begin{align}
    f(x) &= T_{n-1,f,a}(x) + f^{(n)}(a) \int_a^x \frac{(x-t)^{n-1}}{(n-1)!} 
    f^{(n)}(t)\,\diff t + \int_a^x \frac{(x-t)^{n-1}}{(n-1)!} \epsilon(t)\,\diff 
    t \\
    &= T_{n-1,f,a}(x) + f^{(n)}(a) \frac{(x-a)^n}{n!} + \int_a^x 
    \frac{(x-t)^{n-1}}{(n-1)!} \epsilon(t)\,\diff t \\
    &= T_{n,f,a}(x) + \int_a^x \frac{(x-t)^{n-1}}{(n-1)!} \epsilon(t)\,\diff t.
  \end{align}
  Du coup pour tout \(x \in I\setminus\{a\}\) on a
  \begin{align}
    f(x) - T_{n,f,a}(x) &= \int_a^x \frac{(x-t)^{n-1}}{(n-1)!} 
    \epsilon(t)\,\diff t \\
    &= (x-a)^{n-1} \int_a^x \left(\frac{x-t}{x-a}\right)^{n-1} \frac{1}{(n-1)!} 
    \epsilon(t)\,\diff t.
  \end{align}
  Au voisinage de \(a\), on a \(\left(\frac{x-t}{x-a}\right)^{n-1} 
  \frac{1}{(n-1)!} \epsilon(t) = o(1)\). Donc au voisinage de \(a\),
  \begin{equation}
    \int_a^x \left(\frac{x-t}{x-a}\right)^{n-1} \frac{1}{(n-1)!} 
    \epsilon(t)\,\diff t = o(x-a).
  \end{equation}
  Par conséquent \(f(x)-T_{n,f,a}(x) = \petito{(x-a)^n})\).
\end{proof}

\emph{Remarque}~: Si \(f\) est de classe \(\classe{n}\) et admet un \(DL_n(a)\)  
qui est
\begin{equation}
  f(x) = \sum_{k=0}^n \alpha_k(x-a)^k +\petito{(x-a)^n}
\end{equation}
alors par unicité du \(DL_n(a)\), on peut identifier les coefficients, 
c'est-à-dire que pour tout \(k \in \llbracket 0,n \rrbracket\), \(\alpha_k = 
\frac{f^{(k)}(a)}{k!}\).

\subsection{Retour sur les courbes paramétrées}

\subsubsection{Étude locale}

Soient \(R=(O,\vi,\vj)\) un repère orthonormal direct du plan, \((I,f)\) un arc 
paramétré de classe \(\classe{n}\) (\(n \in \N^*\)). On note 
\(M(t)(x(t),y(t))\)le point courant et \(\Gamma\) la trajectoire de ce point.

\emph{Rappel}~: Si \(t_0 \in I\), on dit que le point \(M(t_0)\) est régulier, 
ou que l'arc \((I,f)\) est régulier en \(t_0\), si \(\derived{\vect{OM}}{t} \neq 
\vect{0}\).

\begin{prop}
  Soit \(t_0 \in I\). Supposons que \(M(t_0)\) est régulier. Alors \(\Gamma\) 
  admet une tangente en \(M(t_0)\) dirigée par \(\derived{\vect{OM}}{t}\).
\end{prop}
\begin{proof}
  On note \(M_0=M(t_0)\). Ainsi pour tout \(t \in I\setminus\{t_0\}\) on a
  \begin{align}
    \vect{M_0M(t)} &= (x(t)-x_0)\vi + (y(t)-y_0)\vj \\
    &= (x'(t_0)(t-t_0) +o(t-t_0))\vi + (y'(t_0)(t-t_0) +o(t-t_0))\vj\\
    &=(t-t_0)\vect{f'(t_0)} +(t-t_0)o(1)(\vi+\vj)
  \end{align}
  Ainsi
  \begin{equation}
    \frac{\vect{M_0M(t)}}{||\vect{M_0M(t)}||} = \pm \frac{\vect{f'(t_0)} 
    +o(1)(\vi+\vj)}{||\vect{f'(t_0)} +o(1)(\vi+\vj)||} \to 
    \frac{\vect{f'(t_0)}}{||\vect{f'(t_0)}||}.
  \end{equation}
  Donc \(\Gamma\) admet une tangente en \(M_0\) dirigée par \(\vect{f'(t_0)}\).
\end{proof}

Les fonctions \(x\) et \(y\) sont de classe \(\classe{n}\) et admettent les 
\(DL_n(t_0)\) suivants~:
\begin{align}
  x(t) &= \sum_{i=0}^n \frac{x^{(i)}(t_0)}{i!} (t-t_0)^i + o((t-t_0)^n) \\
y(t) &= \sum_{i=0}^n \frac{y^{(i)}(t_0)}{i!} (t-t_0)^i + o((t-t_0)^n) 
\end{align}
Si on note pour tout \(i \in \llbracket 0,n \rrbracket\) 
\(x_i=\frac{x^{(i)}(t_0)}{i!}\) et \(y=\frac{y^{(i)}(t_0)}{i!}\), et 
\(\vect{A_i}(x_i,y_i)\).

Alors pour tout \(t\) au voisinage de \(t_0\) on a
\begin{equation}
  \vect{M_0M(t)} = (t-t_0)\vect{A_i} +\vect{o}((t-t_0)^n)
\end{equation}


Supposons qu'il existe au moins deux vecteurs non colinéaires parmi les 
\(\vect{A_i}\). Cela permet de définir
\begin{align}
  p &= \min \{i \in \llbracket 1,n \rrbracket \vect{A_i} \neq \vect{0} \} \\
  q &= \min \{i \in \llbracket p+1,n \rrbracket (\vect{A_i},\vect{A_p}) 
  \text{~est libre}\}
\end{align}

Ainsi~: pour tout \(i \in \llbracket 1, p-1 \rrbracket\), 
\(\vect{A_i}=\vect{0}\) et pour tout \(i \in \llbracket p+1, q-1 \rrbracket\) il 
existe un réel \(\alpha_i\) tel que \(\vect{A_i}=\alpha_i \vect{A_p}\).

Le couple \((\vect{Ap}, \vect{A_q})\) est libre, c'est donc une base du plan. 
Alors
\begin{align}
  \vect{M_0M}(t) &= (t-t_0)^p \vect{A_p} + \sum_{i=p+1}^{q-1} (t-t_0)^i \alpha_i 
  \vect{A_p} +(t-t_0)^q \vect{A_q} +\vect{o}((t-t_0)^q) \\
  &=[(t-t_0)^q + \sum_{i=p+1}^{q-1} (t-t_0)^i \alpha_i]\vect{A_p} + (t-t_0)^q 
  \vect{A_q} +\vect{o}((t-t_0)^q)
\end{align}

Soit \((X(t),Y(t))\) les coordonnées de \(M(t)\) le repère 
\((M_0,\vect{A_p},\vect{A_q})\). Alors
\begin{equation}
  X(t) = (t-t_0)^p+\sum_{i=p+1}^{q-1} (t-t_0)^i \alpha_i + o((t-t_0)^q) = 
  (t-t_0)^p + o((t-t_0)^p)
\end{equation}
et alors \(X(t) \sim_{t_0} (t-t_0)^p\) et de la même manière \(Y(t) 
\sim_{t_0}(t-t_0)^q\).

Au voisinage de \(t_0\), \(X(t)\) a le signe de \((t-t_0)^p\) et \(Y(t)\) a le 
signe de \((t-t_0)^q\). L'allure de la corube dépend de la parité des entiers 
\(p\) et \(q\).

\emph{Tangente}~: Pour tout réel \(t \in I\setminus\{t_0\}\) alors
\begin{equation}
  \frac{\vect{M_0M(t)}}{||\vect{M_0M(t)}||} = \pm \frac{\vect{A_p} 
  +\vect{o}(1)}{||\vect{A_p} +\vect{o}(1)|} \to \pm 
  \frac{\vect{A_p}}{||\vect{A_p}||}.
\end{equation}
La tangente est dirigée par \(\vect{A_p}\).

\emph{Allure en fonction de la parité de \(p\) et de \(q\)}~:
\begin{itemize}
  \item si \(p\) est impair et si \(q\) est pair alors \(M_0\) est un point 
    birrégulier;
  \item si \(p\) est impair et si \(q\) est impair alors \(M_0\) est un point 
    d'inflexion;
  \item si \(p\) est pair et si \(q\) est impair alors \(M_0\) est un point de 
    rebroussement de première espèce;
  \item si \(p\) est pair et si \(q\) est pair alors \(M_0\) est un point de 
    rebroussement de deuxième espèce;
\end{itemize}

\subsubsection{Asymptotes}

Soit \(\alpha\) une borne de \(I\), (\(\alpha \notin I\)) avec \(\alpha \in 
\bar{\R}\). Supposons que \(\lim\limits_{t \to \alpha} \frac{Y(t)}{X(t)} = p \in 
\R\).

Alors la courbe admet une direction asymptotique de pente \(p\). Deux cas se 
présentent selon que \(\alpha \in \R\) ou pas.
\begin{itemize}
  \item Si \(\alpha \in \R\), on suppose que l'on peut faire un développement 
    limité de \(y(t)-px(t)\) au voisinage de \(\alpha\). Si on montre que
    \begin{equation}
      y(t)=px(t)+m +c(t-\alpha)^q +o((t-\alpha)^q)
    \end{equation}
    alors la droite d'équation \(y=px+m\) est asympote à la courbe. On peut 
    connaître la position de l'asymptote par rapport à la courbe à l'aide du 
    signe de \(c(t-\alpha)^q\).
  \item Si \(\alpha =\pm \infty\), on pose \(u=\frac{1}{t}\) qui tend vers zéro. 
    On fait un développement limité au voisinage de zéro de \(y(t)-px(t) = 
    y\left(\frac{1}{u}\right) -px \left(\frac{1}{u}\right)\). Si on montre que
    \begin{equation}
      y\left(\frac{1}{u}\right)=px\left(\frac{1}{u}\right)+m +c(u)^q +o(u^q)
    \end{equation}
    alors la droite d'équation \(y=px+m\) est asymptote à la courbe et la 
    position de la courbe par rapport à l'asymptote est donnée par le signe de 
    \(cu^q\).
\end{itemize}

\section{Calcul approché par la méthode des trapèzes}

\subsection{Principe de la méthode}

Soient deux réels \(a\) et \(b\) tels que \(a<b\), une fonction continue de 
\(\intervalleff{a}{b}\) vers \(\R\). On dispose alors de \(I=\int_a^b 
f(x)\,\diff x\) qu'on ne sait pas forcèment calculer. On va approcher \(I\) par 
\(\int_a^bg(x)\,\d x\) où \(g\) sera une fonction continue ``plus simple'' que 
\(f\).

En un pas, \((n=1)\). On choisit pour \(g\) la fonction affine qui prend les 
mêmes valeurs que \(f\) en \(a\) et en \(b\), c'est-à-dire
\begin{equation}
  \forall x \in \intervalleff{a}{b} \quad g(x) = f(a) 
  +\frac{f(b)-f(a)}{b-a}(x-a).
\end{equation}
On approche \(I\) par \(T_1=\int_a^bg(x)\,\diff x\). Lorsqu'on calcule on a
\begin{align}
  T_1&=\int_a^bg(x)\,\diff x = (b-a)f(a) + \frac{f(b)-f(a)}{b-a} \int_a^b 
  (x-a)\,\diff x \\
  &=(b-a)f(a) + (f(b)-f(a))\left(\frac{b-a}{2}\right)\\
  &=(b-a) \frac{f(a)+f(b)}{2}.
\end{align}

En \(n\) pas, \(n \in \N\setminus\{0,1\}\). On considère la subdivision 
régulière \(\sigma\) de \(\intervalleff{a}{b}\) à \(n+1\) points.
\begin{equation}
  \sigma = (a_k)_{0 \leqslant k \leqslant n} \ \forall k \in \llbracket 0,n 
  \rrbracket \quad a_k = a+ k \frac{b-a}{n}
\end{equation}

Sur chaque intervalle \(\intervalleff{a_k}{a_{k+1}}\) on définit la fonction 
affine \(g_k\) qui prend les mêmes valeurs que \(f\) en \(a_k\) et \(a_{k+1}\). 
C'est-à-dire
\begin{equation}
  \forall x \in \intervalleff{a_k}{a_{k+1}} \quad g_k(x) = f(a_k) + 
  \frac{f(a_{k+1})-f(a_k)}{a_{k+1}-a_k}(x-a_{k}).
\end{equation}

On approche \(I\) par le réel
\begin{align}
  T_n &= \sum_{k=0}^{n-1} \int_{a_k}^{a_{k+1}} g_k(x)\,\diff x \\
  &= \sum_{k=0}^{n-1} \frac{a_{k+1}-a_k}{2}(f(a_k)+f(a_{k+1}))\\
  &=\frac{b-a}{2n}  \sum_{k=0}^{n-1}(f(a_k)+f(a_{k+1})) \\
  &=\frac{b-a}{n} \left(\frac{f(a)+f(b)}{2} + 
  \sum_{k=1}^{n-1}(f(a_k)+f(a_{k+1})) \right).
\end{align}

\subsection{Majoration de l'erreur lorsque \(f\) est \(\classe{2}\)}

On cherche à majorer la quantité \(\int_a^bf(x)\,\diff x -T_n\). Lorsque 
\(n=1\), il s'agit de majorer \(\int_a^bf(x)\,\diff x - 
(b-a)\frac{f(a)+f(b)}{2}\)

\begin{lemme}
  Soit une fonction \(f \in \classe{2}(\intervalleff{a}{b},\R)\). La fonction 
  \(f''\) est continue sur \(\intervalleff{a}{b}\) et donc bornée. Alors
  \begin{equation}
    \abs{\int_a^bf(x)\,\diff x - (b-a)\frac{f(a)+f(b)}{2}} \leqslant 
    \frac{\abs{b-a}^3}{12} \sup\limits_{\intervalleff{a}{b}}\abs{f''}.
  \end{equation}
\end{lemme}
\begin{proof}
  Soit \(M_2=\sup\limits_{\intervalleff{a}{b}}\abs{f''}\) et soit l'application
  \begin{equation}
    \fonction{g}{\intervalleff{a}{b}}{\R}{x}{f(a) + \frac{f(b)-f(a)}{b-a}(x-a)}.
  \end{equation}
  Soit aussi \(x_0 \in \intervalleoo{a}{b}\) et la 
  fonction\(\fonction{\varphi}{\intervalleff{a}{b}}{\R}{x}{f(x)-g(x)-k(x-a)(x-b)}\). 
  On choisit la constante \(k\) de sorte que \(\varphi(x_0)=0\).
  Alors \(k=\frac{f(x_0)-g(x_0)}{(x_0-a)(x_0-b)}\).

  La fonction \(f\) est de classe \(\classe{2}\) alors \(\varphi\) aussi est de 
  classe \(\classe{2}\). De plus \(\varphi(a)=\varphi(x_0)=\varphi(b)=0\).

  La fonction \(\varphi\) est continue sur \(\intervalleff{a}{x_0}\), dérivable 
  sur \(\intervalleoo{a}{x_0}\) et \(\varphi(a)=\varphi(x_0)\) alors le théorème 
  de Rolle nous dit qu'il existe un point \(d \in \intervalleoo{a}{x_0}\) tel 
  que \(\varphi'(d)=0\). De la même manière  \(\varphi\) est continue sur 
  \(\intervalleff{x_0}{b}\), dérivable sur \(\intervalleoo{x_0}{b}\) et 
  \(\varphi(x_0)=\varphi(b)\) alors le théorème de Rolle nous dit qu'il existe 
  un point \(e \in \intervalleoo{x_0}{b}\) tel que \(\varphi'(e)=0\). 

  La fonction \(\varphi'\) est continue sur \(\intervalleff{d}{e}\), dérivable 
  su \(\intervalleoo{d}{e}\) et \(\varphi'(d)=\varphi'(e)\) alors le théorème de 
  Rolle nous dit qu'il existe un poinbt \(c_{x_0} \in \intervalleoo{d}{e} 
  \subset ]a,b[\) tel que \(\varphi''(c_{x_0})=0\). 

  Or pour tout \(x \in \intervalleff{a}{b}\), \(\varphi''(x)=f''(x)-2k\).

  On vient de montrer que pour tout \(x_0 \in ]a,b[\) il existe \(c_{x_0} \in 
  \intervalleoo{a}{b}\) tel que 
  \(f''(c_{x_0})=2\frac{f(x_0)-g(x_0)}{(x_0-a)(x_0-b)}\). C'est-à-dire
  \begin{equation}
    f(x_0)-g(x_0)=\frac{1}{2} f''(c_{x_0})(x_0-a)(x_0-b).
  \end{equation}
  Donc
  \begin{equation}
    \int_a^b (f(x)-g(x))\,\diff x = \int_a^b \frac{1}{2} 
    f''(c_{x})(x-a)(x-b)\,\diff x.
  \end{equation}
  Alors
  \begin{equation}
    \abs{\int_a^b (f(x)-g(x))\,\diff x} \leqslant \frac{M_2}{2} \int_a^b 
    (x-a)(x-b)\,\diff x.
  \end{equation}
  L'intégrale vaut \(\frac{(b-a)^3}{6}\) donc
  \begin{equation}
    \abs{\int_a^b f(x)\,\diff x -\frac{b-a}{2}(f(a)+f(b))} = \abs{\int_a^b 
    (f(x)-g(x))\,\diff x} \leqslant \frac{M_2}{12} (b-a)^3.
  \end{equation}
\end{proof}

Pour démontrer plus génralement cette majoration, suppososn que \(n \in 
\N\setminus\{0,1\}\). Alors
\begin{align}
  \left|\int_a^b f(x)\,\diff x-T_{n}\right| &= \left| \sum_{k=0}^{n-1} 
  \int_{a_k}^{a_{k+1}} f(x)\,\diff x -  \sum_{k=0}^{n-1} \int_{a_k}^{a_{k+1}} 
  g_k(x)\,\diff x \right| \\
  &= \left| \sum_{k=0}^{n-1} \int_{a_k}^{a_{k+1}} (f(x) -g_k(x))\,\diff x 
  \right| \\
  &\leqslant \sum_{k=0}^{n-1} \left|\int_{a_k}^{a_{k+1}} (f(x) -g_k(x))\,\diff x 
  \right| \\
  &\leqslant \sum_{k=0}^{n-1} \frac{M_2}{12} (a_{k+1}-a_k)^3 \\
  &\leqslant \frac{M_2}{12} \sum_{k=0}^{n-1} \left(\frac{b-a}{n}\right)^3 \\
  &\leqslant \frac{M_2}{12} \frac{(b-a)^3}{n^2}.
\end{align}

On a ainsi montré que si \(f\) est de classe \(\classe{2}\) sur 
\(\intervalleff{a}{b}\) alors
\begin{equation}
  \left|\int_a^b f(x)\,\diff x-T_n \right| = O\left(\frac{1}{n^2}\right),
\end{equation}
où \(n+1\) est le nombre de points de la subdivision. On admet que ce résultat 
est encore valable lorsque \(f\) est seulement de classe \(\classe{1}\).

\subsection{Intérêt du procédé de dichotomie}

Soit \(n \in \N\setminus\{0,1\}\), alors \(T_n = 
\frac{b-a}{n}\left(\frac{f(a)+f(b)}{2} + \sum_{k=1}^{n-1}f(a_k)\right)\). Pour 
calculer \(T_n\), on a besoin de calculer la valeur de \(f\) en \(n+1=2+(n-1)\) 
points. Pour passer de \(T_n\) à \(T_{n+1}\), on a besoin de calculer \(f\) en 
\(n\) nouveaux points.
\begin{align}
  T_{2n} &= \frac{b-a}{2n}\left(\frac{f(a)+f(b)}{2} + \sum_{k=1}^{2n-1}f(a+k 
  \frac{b-a}{2n}\right)\\
  &=\frac{T_n}{2} + \frac{b-a}{2n}  
  \sum_{k=1}^{n-1}f\left(a+(2k+1)\frac{b-a}{2n}\right).
\end{align}
Donc pour passer de \(T_n\) à \(T_{2n}\), il suffit de calculer la valeur de 
\(f\) en \(n\) nouveaux points. C'est donc plus intéressant de considérer des 
subdivisions dichotomiques, c'est-à-dire en divisant l'intervalle en deux 
parties égales à chaque étape.

\section{Calcul de primitives}

\subsection{Primitives des fonctions usuelles}

On commence par donner une table, la table~
\ref{tab:primitivesusuelles}, de primitives des fonctions usuelles.

\begin{table}[!h]
  \centering
  \begin{tabular}{|l|l|l|} \hline
    \(f(x)\) & \(F(x)\) & \(I\) \\ \hline
    \(x^n, n \in\Z\setminus\{-1\}\)& \(\frac{x^{n+1}}{n+1}\) & \(\R\) \\
    \(\frac{1}{x}\) & \(\ln\abs{x}\) & \(\R^*_-\) ou \(\R^*_+\) \\
    \(x^\alpha, \alpha \in\R\setminus\{-1\}\)& \(\frac{x^{\alpha+1}}{\alpha+1}\) 
    & \(]0,+\infty\) \\
    \(\ln\abs{x}\) & \(x\ln{x}-x\) & \(\R^*_-\) ou \(\R^*_+\) \\
    \(\frac{1}{1+x^2}\) & \(\arctan(x)\) & \(\R\) \\
    \(\frac{1}{\sqrt{x^2+a}}\) & \(\ln|x+\sqrt{x^2+a}|\) & \(\begin{cases}a>0 & 
    \R \\ a<0 & \intervalleoo{-\infty}{-\sqrt{-a}} \text{~ou~}  
    \intervalleoo{-\sqrt{-a}}{+\infty} \end{cases}\) \\
    \(\e^x\) & \(\e^x\) & \(\R\) \\
    \(\cos x\) & \(\sin x\) & \(\R\) \\
    \(\sinh x\) & \(\cosh x\) & \(\R\) \\
    \(\cosh x\) & \(\sinh x\) & \(\R\) \\
    \(\frac{1}{\cos^2 x}\) & \(\tan x\) & 
    \(\intervalleoo{-\pi/2+k\pi}{\pi/2+k\pi}, k \in \Z\) \\
    \(\frac{1}{\sin^2 x}\) & \(-\cotan x\) & \(\intervalleoo{k\pi}{(k+1)\pi}, k 
    \in \Z\) \\
    \(\frac{1}{\cosh^2 x}\) & \(\tanh x\) & \(\R\) \\
    \(\frac{1}{\sinh^2 x}\) & \(-\frac{1}{\tanh x}\) & \(\R^*_-\) ou \(\R^*_+\) 
    \\
    \(\tan x\) & \(-\ln\abs{\cos x}\) & 
    \(\intervalleoo{-\pi/2+k\pi}{\pi/2+k\pi}, k \in \Z\) \\
    \(\cotan x\)& \(\ln\abs{\sin x}\) & \(\intervalleoo{k\pi}{(k+1)\pi}, k \in 
    \Z\) \\
    \(\tanh x\)& \(\ln\cosh x\) & \(\R\)\\
    \(\frac{1}{1-x^2}\) & \(\frac{1}{2} \ln\abs{\frac{1+x}{1-x}}\) & 
    \(\intervalleoo{-\infty}{-1}\), ou \(\intervalleoo{-1}{1}\) ou alors 
    \(\intervalleoo{1}{+\infty}\)\\
  \hline\end{tabular}
  \caption{Primitives de fonctions usuelles}
  \label{tab:primitivesusuelles}
\end{table}

\subsection{Exponentielle polynôme}

\begin{prop}
  Soient \(a \in R^*\) et \(P \in \K[X]\). Il existe un polynôme \(Q \in \K[X]\) 
  tel que \(\deg(Q)=\deg(P)\) et tel que \(x \longmapsto Q(x)\e^{ax}\) soit une 
  primitive de \(x \longmapsto P(x)\e^{ax}\).
\end{prop}
\begin{proof}
  On montre par récurrence sur \(n \in \N\) l'assertion \(\P(n)\) ``Pour tout 
  polynôme \(P\) de \(R_n[X]\) il existe un polynôme \(Q \in \K[X]\) tel que 
  \(\deg(Q)=\deg(P)\) et tel que \(x \longmapsto Q(x)\e^{ax}\) soit une 
  primitive de \(x \longmapsto P(x)\e^{ax}\).''

  \emph{Initialisation}~: \(n=0\) Pour tout polynôme \(P\) dans \(\R_0[X]\), il 
  existe un réel \(\alpha\) tel que \(P=\alpha\). Alors \(Q=\frac{\alpha}{a}\) à 
  le même degré que \(P\). \(\P(0)\) est vraie.

  \emph{Hérédité}~: Soit \(n \in \N\) et on suppose \(\P(n)\). Soit \(P \in 
  \R_{n+1}[X]\). En intégrant par parties, pour tout \(x \in \R\),
  \begin{align}
    \int_a^x P(t)\e^{at}\,\diff t &= \left[P(t)\frac{\e^{at}}{a}\right]_0^x - 
    \int_0^x \frac{P'(t)\e^{at}}{a}\,\diff t \\
    & = \frac{1}{a}(P(x)\e^{ax} -P(0) - \int_0^x P'(t)\e^{at}\,\diff t ).
  \end{align}
  Comme \(P'\in\R_n[X]\), d'après l'hypothèse de récurrence, il existe un 
  polynôme \(Q_0 \in \R_n[X]\) tel que \(\deg(Q_0)=\deg(P)\) et tel que \(x 
  \longmapsto Q_0(x)\e^{ax}\) est une primitive de \(P'(x)\e^{ax}\). Pour tout 
  réel \(x \in \R\), on a donc
  \begin{align}
    \int_a^x P(t)\e^{at}\,\diff t &= \frac{1}{a}(P(x)\e^{ax} -P(0) - 
    (Q_0(x)\e^{ax}-Q_0(0))\\
    &=(P(x)-Q_0(x))\frac{\e^{ax}}{a} - \frac{P(0)-Q_0(0)}{a}.
  \end{align}
  On pose \(Q = \frac{P-Q_0}{a}\) et on a \(\deg(Q)=\deg(P)\) et \(x\longmapsto 
  Q(x)\e^{ax}\) est une primitive de \(x\longmapsto P(x)\e^{ax}\). \(\P(n+1)\) 
  est vraie.

  \emph{Conclusion}~: On a montré que \(\P(0)\) était vraie et que pour tout 
  naturel \(n\), \(\P(n)\) entraîne \(\P(n+1)\). Alors par théorème de 
  récurrence pour tout naturel \(n\), \(\P(n)\) est vraie.
\end{proof}

En pratique, on peut~:
\begin{itemize}
  \item soit chercher \(Q\) par une méthode de coefficients indéterminés;
  \item soit faire des intégrations par parties successives, cependant à éviter 
    si \(\deg(P) >2\).
\end{itemize}

\emph{Remarque}~: Cette méthode permet également de trouver des primitives pour 
des applications de la formes~:
\begin{itemize}
  \item polynôme \(\times \sinh\) ou \(\cosh\), par passage à l'exponentielle;
  \item polynôme \(\times \sin\) ou \(\cos\), par passage à l'exponentielle 
    complexe.
\end{itemize}

\subsection{Fractions rationelles}

Toute fraction rationelle peut être décomposée en éléments simples et il est 
possible de trouver des primitives des éléments simples à l'aide des fonctions 
usuelles. On utilise le logarithme népérien et l'arctangente.

\subsection{Fonctions polynomiale en \(\cos\) ou \(\sin\)}

On veut déterminer des primitives de fonctions qui sont somme de termes de la 
forme \(t \longmapsto \cos^p t \sin^q t\) avec \(p\) et \(q\) des naturels.

Si \(p\) est impair alors il esite \(k \in \N\) tel que \(p=2k+1\). On note 
\(\int \cos^p x \sin^q x\,\diff x\) une primitive quelconque de la fonction \(t 
\longmapsto \cos^p t \sin^q t\) et alors
\begin{equation}
  \int \cos^{2k} x \sin^q x \cos x\,\diff x = \int (1-\sin^2x)^k\sin^q x (\cos 
  x\,\diff x).
\end{equation}
En effectuant le changement de variable \(\classe{1}\)~: \(u=\sin x\) alors
\begin{equation}
  \int \cos^p x \sin^q x\,\diff x = \int (1-u^2)^k u^q \,\diff u.
\end{equation}
On s'est ramené à une intégrale de fonction polynomiale que l'on sait calculer.

Si \(q\) est impair, alors il existe un naturel \(k\) tel que \(q=2k+1\) et de 
la même manière on a
\begin{equation}
  \int \cos^p x \sin^q x\,\diff x = -\int u^p(1-u^2)^k \,\diff u.
\end{equation}
en faisant le changement de variable \(\classe{1}\)~: \(u=\cos x\).

Si \(p\) et \(q\) sont pairs, ces méthodes ne fonctionnent pas. On doit 
linéariser les expressions.

\subsection{Fractions rationelles en \(\cos\) et \(\sin\)}

Soit \(F(X,Y)\) une fraction rationelle à deux indéterminées. On veut calculer 
des primitives de la fonction \(x \longmapsto F(\cos x,\sin x)\). On doit faire 
un changement de variable, ce changement de variable est indiqué par les 
\emph{régles de Bioche}.

Soit pour ``tout'' \(x\), \(w(x)=F(\cos x,\sin x)\diff x\). Alors les régles 
sont données par la table~
\ref{tab:bioche}.
\begin{table}[!h]
  \centering
  \begin{tabular}{|c|c|}\hline
    Si \(w(x)\) est invariant par & on fait le changement de variable \\
    un changement de \(x\) en &  \\ \hline
    \(-x\) & \(t \to \cos x\)\\
    \(\pi-x\) & \(t \to \sin x\)\\
    \(\pi+x\) & \(t \to \tan x\)\\
  \hline\end{tabular}
  \caption{Régles de Bioche}
  \label{tab:bioche}
\end{table}

Dans tous les cas, on peut utiliser le changement de variable universel \(t \to 
\tan\left(\frac{t}{2}\right)\). Cependant, il conduit bien souvent à des calculs 
compliqués. On essaye d'abord les régles de Bioche.

\danger Ne pas oublier de mettre l'élément différentiel ``\(\diff x\)'' 
lorsqu'on vérifie l'invariance.

\danger Avec la tangete, il faut faire attention aux intervalles sur lesquels on 
se place pour que le changement de variable soit \(\classe{1}\), voire même bien 
défini.

\subsection{Fractions rationelles en exponentielles, \ldots}

Soit \(F(X)\) une fraction rationelle. Soit \(G(X,Y)\) une fraction rationelle à 
deux indéterminées. On veut calculer des primitives des fractions \(x 
\longmapsto F(\e^x)\) et \(x \longmapsto G(\sinh x,\cosh x)\).

Il existe deux méthodes~:
\begin{enumerate}
  \item On fait le changement de variable \(t=\e^x\) pour se ramener à une 
    fraction rationelle;
  \item On utilise la ``cousine trigonométrique''. On applique les régles de 
    Bioche avec \(\int G(\cos x,\sin x)\diff x\). On pose \(w(x) = G(\cos x,\sin 
    x) \diff x\) et les régles sont données par la table~
    \ref{tab:cousinetrigo}

    \begin{table}[!h]
      \centering
      \begin{tabular}{|c|c|}\hline
        Si \(w(x)\) est invariant par & on fait le changement de variable \\
        un changement de \(x\) en &  \\ \hline
        \(-x\) & \(t \to \cosh x\)\\
        \(\pi-x\) & \(t \to \sinh x\)\\
        \(\pi+x\) & \(t \to \tanh x\)\\
      \hline\end{tabular}
      \caption{Cousine trigonométrique}
      \label{tab:cousinetrigo}
    \end{table}
\end{enumerate}

Dans tous les cas, on peut prendre \(t=\tanh\left(\frac{x}{2}\right)\) mais les 
claculs sont souvent plus compliqués. Alors on essaie les régles avant.

\section{Extension aux fonctions à valeur complexe}

On considère des fonctions définies sur un intervalle réel \(I\) à valeur dans 
\(\C\).

\subsection{Primitives}

\begin{defdef}
  Soit \(f \in \C^I\). Une application \(F \in \C^I\) est appelée primitive de 
  \(f\) si et seulement si~:
  \begin{enumerate}
    \item \(F\) est dérivable;
    \item \(F'=f\).
  \end{enumerate}
\end{defdef}

De l'étude de la dérivation des fonctions à valeurs complexes, il vient~:
\begin{prop}
  Soient deux applications \(f\) et \(F\) de \(I\) vers \(\C\), alors
  \begin{itemize}
    \item \(F\) est une primitive de \(f\) si et seulement si \(\Re(F)\) et 
      \(\Im(F)\) sont des primitives respectives de \(\Re(f)\) et \(\Im(f)\).
    \item Si \(F\) est une primitive de \(f\), alors l'ensemble des primitives 
      de \(f\) est \(\{F+\tilde{c}, c \in \C\}\).
  \end{itemize}
\end{prop}

\begin{theo}[Théorème fondamental]
  Soit \(f\) une fonction continue de \(I\) vers \(\C\). Alors \(f\) admet des 
  primitives, de plus
  \begin{enumerate}
    \item pour tout \(t_0 \in I\), il existe une seule primitve \(\Phi_0\) de 
      \(f\) qui s'annule en \(t_0\);
    \item toutes les primitives de \(f\) sont de classe \(\classe{1}\).
  \end{enumerate}
\end{theo}
\begin{proof}
  La fonction \(f\) est continue de \(I\) vers \(\C\) donc \(\Im(f)\) et 
  \(\Re(f)\) sont aussi continues de \(I\) vers \(\R\). Elles admettent des 
  primitives donc \(f\) admet des primitives.
  \begin{enumerate}
    \item l'application \(t\longmapsto \int_{t_0}^t \Re(f)\) est une primitive 
      de \(\Re(f)\) quiu s''annule en \(t_0\), idem pour \(t\longmapsto 
      \int_{t_0}^t \Im(f)\). Alors \(t\longmapsto \int_{t_0}^t f\) est une 
      primitive de \(f\) qui s'annule en \(t_0\). L'unicité découle du fait que 
      les primitives ne différent que d'une constante.
    \item Les primitives de \(f\) sont de classe \(\classe{1}\) car leurs 
      parties imaginaires et réelles sont de classe \(\classe{1}\) en tant que 
      primitives de fonctions continues réelles \(\Re(f)\) et \(\Im(f)\).
  \end{enumerate}
\end{proof}

\begin{prop}
  Soient \((a,b) \in I^2\) et \(f \in \courbe(\intervalleff{a}{b},\C)\). Si 
  \(F\) est une primitive de \(f\), alors
  \begin{equation}
    \int_a^b f(t)\,\diff t = F(b)-F(a)
  \end{equation}
\end{prop}

\begin{cor}
  Soient \((a,b) \in I^2\) et \(f \in \classe{1}(\intervalleff{a}{b},\C)\), 
  alors
  \begin{equation}
    \int_a^b f'(t)\,\diff t = f(b)-f(a)
  \end{equation}
\end{cor}

\subsection{Théorème du relévement}

\begin{theo}
  Soient un intervalle réel \(I\) et \(f \in \classe{k}(I,\C)\) avec \(k \in 
  \N^*\). Supposons que pour tout \(t \in I\), \(|f(t)|=1\) alors il existe une 
  application \(\theta \in \classe{k}(I,\R)\) telle que pour tout \(t \in I\) on 
  ait \(f(t) = \e^{\ii\theta(t)}\).
\end{theo}

\begin{proof}[Analyse]
  Supposons qu'il existe une telle application \(\theta\). Alors comme les 
  applications \(f\) et \(t \longmapsto \e^{\ii\theta(t)}\) sont dérivables. 
  Alors pour tout \(t\in I\) on a
  \begin{equation}
    f'(t) = \ii \theta'(t)f(t),
  \end{equation}
  et puisque \(f\) est de module \(1\), c'est équivalent à
  \begin{equation}
    \theta'(t) = \frac{f'(t)}{\ii f(t)}.
  \end{equation}
L'application \(\theta\) est donc une primitive de l'application \(t \longmapsto 
\frac{f'(t)}{\ii f(t)}\) sur \(I\). \end{proof}
\begin{proof}[Synthèse]
  Soit l'application \(\fonction{g}{I}{\C}{t}{\frac{f'(t)}{\ii f(t)}}\). Cette 
  application est définie et elle est continue.

  Soit \(t_0 \in I\) et \(\theta_0\) un argument de \(f(t_0)\). On peut alors 
  définir la primitive \(\theta\) de \(g\) qui vaut \(\theta_0\) en \(t_0\), 
  c'est à dire
  \begin{equation}
    \forall t \in I \quad \theta(t) = \int_{t_0}^t g(u)\,\diff u +\theta_0.
  \end{equation}
  Cette application est de classe \(\classe{k}\). Soit 
  \(\fonction{\psi}{I}{\C}{t}{f(t)\e^{-\ii\theta(t)}}\). Par produit et 
  composée, \(\psi\) est de classe \(\classe{k}\) et donc au moins de classe 
  \(\classe{1}\). Alors pour tout \(t \in I\), on a
  \begin{align}
    \psi'(t) &= f'(t)\e^{-\ii \theta(t)}-\ii \theta'(t)f(t) \e^{-\ii\theta(t)} 
    \\
    &=\left(f'(t)-\ii f(t) \frac{f'(t)}{\ii f(t)} \right)\e^{-\ii \theta(t)} =0.
  \end{align}
  Donc \(\psi\) est constante sur \(I\) et comme \(\psi(t_0)=f(t_0)\e^{-\ii 
  \theta_0}=1\) (puisque \(f(t_0)=\e^{\ii \theta_0}\). Donc pour tout \(t \in 
  I\), \(\psi(t)=1\), c'est-à-dire que \(f(t)=\e^{\ii \theta(t)}\).
\end{proof}

\emph{Remarque}~: Si \(f \in \classe{k}(I,\C)\) et si elle ne s'annule pas, 
alors on peut appliquer le théorème du relévement à \(t \longmapsto 
\frac{f(t)}{|f(t)|}\). Il existe alors une application \(\theta \in 
\classe{k}(I,\C)\) telle que pour tout \(t \in I\), on a \(f(t) = |f(t)|\e^{\ii 
\theta(t)}\).

\subsection{Inégalité des accroissements finis}

\danger Le théorème de Rolle et le théorème des accroissements finis ne 
s'applique pas aux fonctions à valeurs complexes.

\begin{prop}
  Soit \(f \in \classe{1}(I,\R)\) et deux réels \(a<b\). Alors \(f'\) est bornée 
  sur \(\intervalleff{a}{b}\) et
  \begin{equation}
    |f(b)-f(a)| \leqslant \sup\limits_{\intervalleff{a}{b}}|f'|(b-a)
  \end{equation}
\end{prop}
\begin{proof}
  \begin{equation}
    |f(b)-f(a)| = \left|\int_a^b f'(t)\,\diff t\right| \leqslant \int_a^b 
    |f'(t)\,\diff t| \leqslant (b-a) \sup\limits_{\intervalleff{a}{b}}|f'|.
  \end{equation}
\end{proof}

\subsection{Formules de Taylor}

\subsubsection{Formule de Taylor avec reste intégral}

\begin{theo}
  Soient un intervalle réel \(I\), \(a \in I\), \(n \in \N\) et \(f \in 
  \classe{{n+1}}(I,\C)\). Alors
  \begin{equation}
    \forall x \in I \quad f(x) = \sum_{k=0}^n \frac{f^{(k)}(a)}{k!}(x-a)^k + 
    \int_a^x \frac{(x-t)^n}{n!} f^{(n+1)}(t)\,\diff t
  \end{equation}
  C'est la formule de Taylor avec reste intégral à l'ordre \(n\) appliquée à 
  \(f\) au point \(a\).
\end{theo}
\begin{proof}
  Il suffit de prendre \(\Re(f)\) et \(\Im(f)\).
\end{proof}

\subsubsection{Inégalité de Taylor-Lagrange}

\begin{theo}
  Soient un intervalle \(I\), \(a \in I\), \(n \in \N\) et \(f \in 
  \classe{{n+1}} (I,\C)\). La fonction \(f^{(n+1)}\) est continue sur un segment 
  donc bornée. Pour tout réel \(x \in I\) alors
  \begin{equation}
    \left|f(x)-\sum_{k=0}^n \frac{f^{(k)}(a)}{k!}(x-a)^k\right| \leqslant 
    \frac{|x-a|^{n+1}}{(n+1)!} \sup \abs{f^{(n+1)}}.
  \end{equation}

  C'est l'inégalité de Taylor-Lagrange à l'ordre \(n\) appliquée à \(f\) au 
  point \(a\).
\end{theo}
\begin{proof}
  Idem.
\end{proof}

\subsubsection{Formule de Taylor-Young et développements limités}

\begin{theo}
  Soient \(I\) un intervalle réel, \(a \in I\), \(n \in \N\), \(f \in 
  \classe{n}(I,\C)\). Alors la fonction \(f\) admet le \(DL_n(a)\) suivant~:
  \begin{equation}
    f(x) = \sum_{k=0}^n \frac{f^{(k)}(a)}{k!} (x-a)^k + \petito{(x-a)^n}.
  \end{equation}
  C'est la formule de Taylor-Young à l'ordre \(n\) appliquée à \(f\) en \(a\).
\end{theo}
\begin{proof}
  Idem.
\end{proof}

\begin{defdef}
  Soient \(I\) un intervalle réel, \(a \in I\), \(n \in \N\), \(f \in \C^I\). La 
  fonction \(f\) admet un \(DL_n(a)\) si et seulement s'il existe une famille 
  \((a_k)_{0\leqslant k\leqslant n} \in \C^{n+1}\) et une application \(\epsilon 
  \in \R^I\) telles que
  \begin{equation}
    \forall x \in I \quad f(x)=\sum_{k=0}^n a_k (x-a)^k + \epsilon(x)(x-a)^n.
  \end{equation}

  On vérifie facilement que \(f\in \C^I\) admet un \(DL_n(a)\) si et seulement 
  si \(\Re(f)\) et \(\Im(f)\) admettent des \(DL_n(a)\) et auquel cas~:
  \begin{equation}
    P_n(f) = P_n(\Re(f)) + \ii P_n(\Im(f)).
  \end{equation}
\end{defdef}
