\chapter{Groupe, anneau, corps \& arithmétique}\label{chap:groupes}
\minitoc%
\minilof%
\minilot%
\section{Groupe}
\subsection{Notion de loi de composition interne}
\subsubsection{Définition}
\begin{defdef}
  Soit E un ensemble quelconque, on appelle loi de composition interne sur E
  toute application \(*:E\times E \longmapsto E\). Pour \(x,y \in E\) on note
  \(x*y\).
\end{defdef}
\subsubsection{Partie stable par une loi de composition interne}
\begin{defdef}
  Soit \((E,*)\) un couple où E est un ensemble et \(*\) une loi de composition
  interne sur E. On dit que c'est un magma. Soit \(A\) une partie de \(E\). On
  dit que la partie \(A\) est stable pour la loi \(*\) si et seulement si pour
  tout \(x,y \in A\) \(x*y \in A\). On dispose alors de la restriction \(*_{A} :
  A\times A \longmapsto A\). On l'appelle loi induite sur \(A\) par la loi
  \(*\).
\end{defdef}
\subsubsection{Qualités éventuelles d'une loi de composition interne}
Soit le magma \((E,*)\). Les qualités éventuelles sont les suivantes
\begin{enumerate}
  \item la loi \(*\) est commutative si et seulement si
    \begin{equation}
      \forall (x,y) \in E^2 \quad x*y=y*x;
    \end{equation}
  \item la loi \(*\) est associative si et seulement si
    \begin{equation}
      \forall (x,y,z) \in E^3 \quad x*(y*z)=(x*y)*z,
    \end{equation}
    qu'on pourra noter \(x*y*z\).
\end{enumerate}
On dit que E admet un élément neutre pour \(*\) lorsqu'il existe un élément \(e
\in E\) tel que pour tout élément \(x \in E\), \(x*e=e*x=x\). Soit \(e \in E\),
alors \begin{itemize}
  \item \(e\) est neutre à gauche si et seulement si \(\forall x \in E \quad e *
    x =x\);
  \item \(e\) est neutre à droite si et seulement si \(\forall x \in E \quad x *
    e =x\).
\end{itemize}
\(e\) est un élément neutre si et seulement si \(e\) est neutre à gauche et
\(e\) est neutre à droite.
\begin{prop}
  Un élément neutre, s'il existe, est unique.
\end{prop}
\begin{proof}
  Supposons, qu'il en existe deux \(e_1, e_2\). Alors \(e_1*e_2 =e_1\) puisque
  \(e_2\) est neutre donc neutre à droite. Puisque \(e_1\) est neutre donc
  neutre à gauche, on a aussi \(e_1*e_2=e_2\) donc finalement \(e_1=e_2\).
\end{proof}
\subsubsection{Propriétés  élémentaires}
Soit le magma \((E,*)\). On suppose que \(*\) est associative et que \(e\) est
le neutre de \(E\).
\begin{defdef}
  Soit \(x \in E\). On dit que \(x\) est symétrisable à gauche s'il existe \(x'
  \in E\) tel que \(x' * x=e\). De la même manière \(x\) est symétrisable à
  droite s'il existe \(x'' \in E\) tel que \(x * x''=e\).
\end{defdef}
\begin{prop}
  Soit \(x \in E\). Si \(x\) admet un symétrique à droite \(x'\) et un
  symétrique à gauche \(x''\) alors \(x'=x''\). On a donc \(x*x'=x'*x=e\). On
  dit dans ce cas que \(x\) est inversible et \(x'\) est l'inverse de \(x\).
\end{prop}
\begin{proof}
  On sait par hypothèse que \(x'*x=e\) et que \(x*x''=e\). Alors
  \(x'*(x*x'')=x'*e=x'\) et comme \(*\) est associative on a
  \begin{equation}
    x'=x'*(x*x'')=(x'*x)*x''=e*x''=x''.
  \end{equation}
\end{proof}
\begin{prop}
  Soit \(x \in E\). Si \(x\) est inversible alors l'inverse est unique et on le
  note \(x^{-1}\).
\end{prop}
\begin{proof}
  Soient \(x'\) et \(x''\) deux inverses de \(x\), alors x' est inversible à
  gauche et on a \(x'*x=e\). On compose par \(x''\) à droite et on a
  \(x'*(x*x'')=e*x''\). Alors \(x'*e=x''\) donc \(x'=x''\).
\end{proof}
\begin{prop}
  Soit \(x \in E\). Si \(x\) est inversible d'inverse \(x^{-1}\) alors celui-ci
  est inversible d'inverse \(x\), \((x^{-1})^{-1}=x\).
\end{prop}
\begin{proof}
  On sait que \(x*x^{-1}=x^{-1}*x=e\) donc \(x^{-1}\) est inversible et
  \(x^{-1}*(x^{-1})^{-1}=e\) en composant par \(x\) à gauche on a
  \((x*x^{-1})*(x^{-1})^{-1}=x*e\) d'où le résultat.
\end{proof}
\begin{prop}
  Soient \(x,y \in E\) inversibles, d'inverse respectif \(x^{-1}\) et
  \(y^{-1}\). Alors \(x*y\) est inversible d'inverse \(y^{-1}*x^{-1}\).
\end{prop}
\begin{proof}
  \begin{gather}
    (x*y)*(y^{-1}*x^{-1}) = x*(y*y^{-1})*x^{-1}=x*x^{-1}=e; \\
    (y^{-1}*x^{-1})*(x*y) = y^{-1}*(x^{-1}*x)*y=y^{-1}*y=e.
  \end{gather}
\end{proof}
\begin{defdef}
  Soit \(a \in E\). On dit que \(a\) est simplifiable à droite dans \(E\) si
  \begin{equation}
    \forall x,y \in E \quad x*a=y*a \implies x=y.
  \end{equation}
  On dit que \(a\) est simplifiable à gauche dans \(E\) si
  \begin{equation}
    \forall x,y \in E \quad a*x=a*y \implies x=y.
  \end{equation}
  On dit que \(a\) est simplifiable ou régulier s'il est simplifiable à droite
  et à gauche.
\end{defdef}
\begin{prop}
  Tout élément inversible est régulier.
\end{prop}
\begin{proof}
  Soit \(a \in E\) inversible, et notons \(a^{-1}\) son inverse. Soient \(x,y
  \in E\) tels que \(x*a=y*a\). Alors
  \begin{align}
    (x*a)*a^{-1}&=(y*a)*a^{-1} \\
    x*(a*a^{-1})&=y*(a*a^{-1})\\
    x*e&=y*e\\
    x&=y.
  \end{align}
  On a montré que \(a\) est simplifiable à droite, on peut monter de la même
  manière que \(a\) est simplifiable à gauche.
\end{proof}

\subsubsection{Notations usuelles}
\paragraph{Notation multiplicative}
La loi est notée \(\cdot\), pour \(a,b \in E\) on a \(a*b=ab\). Le neutre est
noté \(1\) et le symétrique de \(a\) est noté \(a^{-1}\).
\paragraph{Notation additive}
La loi est notée \(+\), pour \(a,b \in E\) on a \(a*b=a+b\). Le neutre est noté
\(0\) et le symétrique de \(a\) est noté \(-a\). La notation additive est en
général réservée au cas où la loi est commutative.
\paragraph{Itérés d'un élément}

Soit \(x \in E\). Pour un naturel \(n\) non nul, si la loi \(*\) est commutative
on définit \(x * \ldots *x = *^{n} x = x^n\). Par convention on pose \(x^0=e\)
et par récurrence \(x^{n+1}=x*x^{n}\). Si \(x\) est inversible, on peut définir
les itérés pour \(n \in \Z\setminus\N\) par \(x^n=(x^{-1})^{-n}\).

\subsubsection{Premiers exemples}

Soit un ensemble \(E\), \(E^E\) est l'ensemble des applications de \(E\) dans
\(E\). La loi de composition \(\circ\) est une loi de composition interne sur
\(E^E\). Elle est associative et \(\Id\) est son élément neutre. Elle n'est en
revanche pas commutative et tous les élément de \(E^E\) ne sont pas inversibles.
Les lois habituelles \(+\) et \(\times\) sont des lois associatives et
commutatives sur \(\N\), \(\Z\), \(\Q\), \(\R\) et \(\C\). \(0\) est le neutre
de \(+\) et \(1\) est le neutre de \(\times\).

Soit \(E\) un ensemble. \(\Part(E)\) est l'ensemble des parties de \(E\). Les
opérations \(\bigcap\) et \(\bigcup\) sont des lois de composition internes sur
\(\Part(E)\). Elles sont toutes deux associatives et commutatives. \(E\) est le
neutre pour \(\bigcap\) et \(\emptyset\) est le neutre de \(\bigcup\).

\subsection{Notion de groupe}

\subsubsection{Définition}

\begin{defdef}
  Soit \((G,*)\) un magma. On dit que \((G,*)\) est un groupe si
  \begin{itemize}
    \item la loi \(*\) est associative;
    \item \(G\) admet un neutre pour \(*\), noté \(e\);
    \item tout élément de \(G\) est inversible par \(*\).
  \end{itemize}
\end{defdef}
\begin{defdef}
  Un groupe commutatif ou abélien est un groupe pour lequel la loi est
  commutative.
\end{defdef}

\subsubsection{Propriétés}

\begin{prop}
  Un groupe n'est jamais vide. Puisqu'il contient au moins son neutre.
\end{prop}
\begin{prop}
  L'élément neutre d'un groupe est unique.
\end{prop}
\begin{prop}
  Tout élément d'un groupe admet un et un seul inverse.
\end{prop}
\begin{prop}
  Tout élément d'un groupe est régulier. Puisque tout élément d'un groupe est
  inversible.
\end{prop}

\subsubsection{Exemples}

\((\Z,+)\), \((\Q,+)\), \((\R,+)\), \((\C,+)\), \((\Q^*,\times)\),
\((\R^*,\times)\), \((\C^*,\times)\) sont des groupes. Par contre \((\N,+)\),
\((\R,\times)\) ne sont pas des groupes.

Soit \(E\) un ensemble et \(\sigma(E)\) l'ensemble des permutations de \(E\).
Alors \((\sigma(E), \circ)\) est un groupe. Mais \((E^E, \circ)\) n'est pas un
groupe puisqu'il existe des application de E dans E qui ne sont pas inversibles.

Si \((G,*_1)\) et \((H,*_2)\) sont deux groupes alors \((G \times H, *)\) est un
groupe, le groupe produit, où \(*\) est la loi de composition interne définie
par
\begin{equation}
  \forall (g_1,g_2) \in G^2 \ \forall (h_1,h_2) \in H^2 \quad
  (g_1,h_1)*(g_2,h_2) = (g_1*_1g_2, h_1 *_2 h_2).
\end{equation}
La loi \(*\) est appelée la loi produit.

Soit \(E\) un ensemble. Si on note \(\Delta\) la différence symétrique alors
\((\Part(E), \Delta)\) est un groupe abélien. Avec
\begin{equation}
  \forall A,B \in \Part(E) \quad A\Delta B = (A \cup B) \setminus (A \cap B).
\end{equation}
L'ensemble vide est le neutre de \(\Delta\) et l'inverse de chaque élément est
l'élément lui-même
\begin{align}
  \forall A \in \Part(E) \quad & A \Delta \emptyset = A \setminus \emptyset = A;
  \\
  \forall A \in \Part(E) \quad & A \Delta A = A \setminus A = \emptyset.
\end{align}

\subsection{Sous-groupe}

\subsubsection{Définition}

\begin{defdef}
  Soit \((G,*)\) un groupe. Une partie \(H\) de \(G\) est un sous-groupe si
  \begin{itemize}
    \item \(H\) est stable pour la loi \(*\);
    \item \(H\) muni de la loi induite \(*_H\) est un groupe.
  \end{itemize}
\end{defdef}

\subsubsection{Propriétés}

Soit \((G,*)\) un groupe et \(H\) un sous-groupe de G.
\begin{prop}
  Si \(G\) est un groupe abélien, alors \(H\) est un groupe abélien.
\end{prop}
\begin{proof}
  Soient \(g,h \in H\) alors \(h*_H g=h*g\) par définition de \(*_H\) et comme
  \(G\) est abélien \(h*g=g*h\) donc \(h*_H g=g*_Hh\).
\end{proof}
\begin{prop}
  \(e_H=e_G\).
\end{prop}
\begin{proof}
  Comme \(H\) et \(G\) sont des groupes, on a \(e_H * e_H = e_H*_H e_H =e_H=e_H
  *e_G\). Donc \(e_H * e_H = e_H *e_G\) et comme \(e_H\) est simplifiable on a
  \(e_H=e_G\).
\end{proof}
\begin{prop}
  \label{prop:ssgroupesym}
  Tout élément de H admet le même symétrique dans \(H\) et dans \(G\).
\end{prop}
\begin{proof}
  Soit \(h \in H\). \(H\) est un groupe donc il existe un élément \(h' \in H\)
  qui est le symétrique de \(h\) dans \(H\). \(h \in G\) et \(G\) est un groupe
  donc il existe un élément \(h'' \in H\) qui est le symétrique de \(h\) dans
  \(G\). Ainsi
  \begin{align}
    h *_H h' &= h' *_H h = e_H; \\
    h * h' & = h' * h = e_H.
  \end{align}
  De plus
  \begin{equation}
    h *h'' = h'' * h=e_H.
  \end{equation}
  Par unicité du symétrique dans \(G\) on a forcément \(h'=h''\).
\end{proof}

\subsubsection{Caractérisation des sous-groupes}

\begin{theo}
  Soit \((G,*)\) un groupe et \(H\) une partie de \(G\). Il y a équivalence
  entre
  \begin{enumerate}
    \item \(H\) est un sous-groupe de \(G\);
    \item \(H \neq \emptyset\), \(H \subset G\), \(H\) est stable pour la loi
      \(*\) et stable par passage au symétrique;
    \item \(H \neq \emptyset\), \(H \subset G\), \(\forall x,y \in H \ x*y^{-1}
      \in H\).
  \end{enumerate}
\end{theo}
\begin{proof}
  \begin{itemize}
    \item[\(1 \implies 2\)] Si \(H\) est un sous-groupe de \(G\) alors par
      définition H est non vide, \(H \subset G\) et \(H\) est stable par la loi
      de \(G\). Soit \(x \in H\) alors son symétrique dans \(H\) est le même que
      dans \(G\) (d'après la proposition~\ref{prop:ssgroupesym}). Donc son symétrique dans \(G\) est dans \(H\).
      \(H\) est donc stable par passage au symétrique.
    \item[\(2 \implies 3\)] Il faut juste démontrer le troisième sous-point.
      Soit \((x, y) \in H^2\), \(H\) est stable par symétrie donc \(y^{-1} \in
      H\). \(x\) et \(y^{-1}\) sont dans \(H\) et comme \(H\) est stable pour la
      loi \(*\), on a \(x*y^{-1} \in H\).
    \item[\(3 \implies 1\)] Déjà \(H \subset G\) et \(H\) est non vide. Alors il
      existe \(h \in H\) tel que \(e_G = h*h^{-1}\) alors \(e_G \in H\). Soit
      \(y \in H\) alors \(e_G*y^{-1} = y^{-1} \in H\). Tout élément de \(H\)
      admet un symétrique dans \(H\). Soit \(x \in H\) alors comme \(y^{-1} \in
      H\) on a \(x*y=x*(y^{-1})^{-1} \in H\). \(H\) est stable par \(*\) donc on
      peut considérer le couple \((H,*_H)\) où \(*_H\) est la loi induite.
      Montrons que \((H,*_H)\) est un groupe. La loi \(*_H\) est associative
      puisque \(*\) l'est.  Soit \(h \in H\) alors
      \begin{align}
        h *_H e_G &= h * e_G =h; \\
        e_G *_H h &= e_G * h =h.
      \end{align}
      Donc \(e_G\) est neutre dans \(H\). Tout élément de \(H\) admet un
      symétrique dans \(H\). Alors \((H,*_H)\) est un groupe.
  \end{itemize}
\end{proof}

\subsubsection{Exemples de sous-groupes}

Soit \((G,*)\) un groupe, alors les sous-groupes dits triviaux sont \(G\) et
\(\{e_G\}\). \(\Z\) est un sous-groupe de \((\R,+)\). Les sous-groupes de
\((\Z,+)\) sont les \(n\Z\) où \(n \in \N\), c'est à dire \(n\Z=\enstq{np}{p \in
\Z}\). \(\U\) et \(\U_n\) sont des sous-groupes de \((\C^{*},\times)\). Soit
\((G,*)\) un groupe, \(H_1\) et \(H_2\) deux sous-groupes de \((G,*)\). Soit \(H
= H_1 \cap H_2\) alors c'est aussi un sous-groupe de \(G\).
\begin{proof}
  Puisque \(H_1 \subset G\) et \(H_2 \subset G\) alors \(H \subset G\). On sait
  que \(e_G \in H_1\) et \(e_G \in H_2\) donc \(e_G \in H\). Il est non vide.
  Soit ensuite \((x,y) \in H^2\) alors ils sont dans \(H_1\) et dans \(H_2\).
  Alors \(x*y^{-1}\) est dans \(H_1\) et dans \(H_2\) donc il est dans \(H\).
  Par caractérisation, H est un sous-groupe de \((G,*)\).
\end{proof}

En général l'union de sous-groupes n'est pas un sous-groupe.

\subsection{Morphismes de groupes}

\subsubsection{Définition}

\begin{defdef}
  Soient \((G,*)\) et \((H,\#)\) deux groupes. Une application \(f:(G,*)
  \longrightarrow (H,\#)\) est un morphisme (ou homo-morphisme) de groupes du
  groupe \(G\) sur le groupe \(H\) si et seulement si
  \begin{equation}
    \forall x,y \in G \quad f(x*g) = f(x)\#f(y).
  \end{equation}
\end{defdef}
Soit \(f:(G,*) \longrightarrow (H,\#)\) un morphisme de groupes. On dit que
\begin{itemize}
  \item \(f\) est un isomorphisme si \(f\) est bijective;
  \item \(f\) est un endomorphisme si \(G=H\) et \(*=\#\);
  \item \(f\) est un automorphisme si \(G=H\), \(*=\#\) et si \(f\) est
    bijective.
\end{itemize}

\begin{defdef}
  Soient \((G,*)\) et \((H,\#)\) deux groupes. On dit que \((G,*)\) et
  \((H,\#)\) sont isomorphes si et seulement s'il existe un isomorphisme de
  groupe de \((G,*)\) sur \((H,\#)\).
\end{defdef}

\subsubsection{Propriétés}

\begin{prop}
  \label{prop:compmorph}
  Soient \((G,*)\), \((H,\#)\) et \((I,\%)\) trois groupes. Soient \(f:(G,*)
  \longrightarrow (H,\#)\) et \(g: (H,\#) \longrightarrow (I,\%)\) deux
  morphismes de groupes. Alors la composée \(g \circ f\) est un morphisme de
  groupe de \((G,*)\) sur \((I,\%)\).
\end{prop}
\begin{proof}
  Déjà \(g \circ f\) es bien définie et pour tout \((x,y) \in G^2\) on a
  \begin{equation}
    g \circ f (x * y) = g(f(x) \# f(y)) = g\circ f(x) \% g\circ f(y)
  \end{equation}
\end{proof}
\begin{prop}
  \label{prop:morphinv}
  Soit \(f\) un isomorphisme de \((G,*)\) sur \((H,\#)\). Alors la réciproque
  \(f^{-1}\) est un isomorphisme de \((H,\#)\) sur \((G,*)\).
\end{prop}
\begin{proof}
  \(f\) est un isomorphisme donc \(f\) est bijective et sa réciproque existe. On
  sait que \(f \circ f^{-1}=\Id_{H}\) donc pour tout \((x,y) \in H^2\) on a
  \begin{align}
    f^{-1}(x \# y) &=f^{-1}(f \circ f^{-1}(x) \# f \circ f^{-1}(y))\\
    &=f^{-1}[f(f^{-1}(x) *  f^{-1}(y))]\\
    &=f^{-1}(x) *  f^{-1}(y).
  \end{align}
  C'est donc bien un morphisme de \((H,\#)\) sur \((G,*)\). Comme \(f\) est
  bijective alors \(f^{-1}\) est aussi bijective. Finalement \(f^{-1}\) est bien
  un isomorphisme.
\end{proof}
\begin{prop}
  Soit \((G,*)\) un groupe. Alors \(\Auto{G}\) (ensemble des automorphismes de
  \(G\)) est un sous-groupe de l'ensemble des permutations de \(G\) muni de la
  composition, \((\sigma(G), \circ)\).
\end{prop}
\begin{proof}
  Déjà \(\Auto{G} \subset \sigma(G)\) car par définition les automorphismes de
  \(G\) sont des bijections de \(G\) dans \(G\).
  \begin{itemize}
    \item L'application identité est une bijection de \(G\) dans lui-même et
      \begin{equation}
        \forall (x,y) \in G^2 \quad \Id_G(x * y) = x*y=\Id_G(x) * \Id_G(y),
      \end{equation}
      alors l'identité est un endomorphisme de \((G,*)\). Comme c'est une
      bijection, alors \(\Id_G  \in \Auto{G}\).
    \item Soient \(f\) et \(g\) deux automorphismes de \(G\). Déjà \(f \circ
      g^{-1}\) est une bijection de \(G\) dans \(G\). D'après la proposition~\ref{prop:morphinv}, \(g^{-1}\) est un isomorphisme. D'après la
      proposition~\ref{prop:compmorph}, \(f \circ g^{-1}\) est un morphisme. Comme \(f\) et
      \(g^{-1}\) sont bijectives alors \(f \circ g^{-1}\) est bijective donc
      c'est un isomorphisme de \(G\) dans \(G\). Finalement c'est un
      automorphisme.
  \end{itemize}
  Par caractérisation des sous-groupes, on conclue en écrivant que \(\Auto{G}\)
  est un sous-groupe de \(\sigma(G)\).
\end{proof}
\begin{theo}
  Soit \(f\) un morphisme de groupe de \((G,*)\) vers \((H,\#)\), alors
  \begin{gather}
    f(e_G)=e_H; \\
    \forall x \in G \quad f(x^{-1})=f(x)^{-1}; \\
    \forall (x,y) \in G^2 \quad f(x*y^{-1})=f(x) \# f(y)^{-1}; \\
    \forall n \in \N^{*} \ \forall (x_1, \ldots, x_n) \in G^n \quad f(x_1 *
    \ldots * x_n) = f(x_1) \# \ldots \# f(x_n); \\
    \forall k \in \Z \ \forall x \in G \quad f(x^k)=f(x)^k.
  \end{gather}
\end{theo}
\begin{proof}
  \begin{enumerate}
    \item \(f\) est un morphisme donc \(f(e_G * e_G)=f(e_G)\#f(e_G)\). Comme
      \(e_G\) est le neutre, on a \(f(e_G * e_G)=f(e_G)\). Comme \(e_{H}\) est
      le neutre de H on a \(f(e_G*e_G)=f(e_G) \# e_H\). Donc \(f(e_G)\#f(e_G) =
      f(e_G) \# e_H\). Comme H est un sous-groupe, tous ses éléments sont
      simplifiables alors \(f(e_G)=e_H\).
    \item Soit \(x \in G\), puisque \(f\) est un morphisme alors
      \begin{equation}
        f(x) \# f(x^{-1})=f(x * x^{-1})=f(e_G)=e_H,
      \end{equation}
      d'après le premier point. De même on  montre \(f(x^{-1}) \# f(x)=e_H\).
      Alors \(f(x^{-1})=f(x)^{-1}\).
    \item Soient \((x,y) \in G^2\) alors comme \(f\) est un morphisme, on a
      \begin{equation}
        f(x*y^{-1})=f(x)\#f(y^{-1}) = f(x)\#f(y)^{-1},
      \end{equation}
      d'après le deuxième point.
    \item On démontre par récurrence à partir de la définition
      \begin{equation}
        \forall n \in \N^* \quad P(n) "\forall (x_1, \ldots, x_n) \in G^n \quad
        f(x_1 * \ldots * x_n) = f(x_1) \# \ldots \# f(x_n)"
      \end{equation}
    \item Si \(k \in \N\) on applique le quatrième point pour \(x_k=x\), sinon
      on a
      \begin{equation}
        f(x^k)=f((x^{-1})^{-k})=f(x^{-1})^{-k}=(f(x)^{-1})^{-k}=f(x)^k.
      \end{equation}
  \end{enumerate}
\end{proof}

\subsubsection{Image et noyau d'un morphisme de groupes}

Soient \((G,*)\) et \((H,\#)\) deux groupes et \(f\) un morphisme de \(G\) sur
\(H\).

\begin{theo}
  \label{theo:theo1sousgroupe}
Soit \(G'\) un sous-groupe de G. Alors l'image directe \(f(G')\) est un
sous-groupe de \(H\). \end{theo}
\begin{proof}
  \begin{itemize}
    \item \(f(G') \subset H\) par définition;
    \item \(G'\) est un sous-groupe de G donc \(e_G \in G'\). \(f(e_G)=e_H \in
      f(G')\) alors \(f(G')\) est non vide;
    \item Soient \((x,y) \in f(G')^2\) il existe \((z,t) \in G'^2\) tels que
      \(x=f(z)\) et \(y=f(t)\) et
      \begin{equation}
        x\#y^{-1}  = f(z*y^{-1}).
      \end{equation}
      \(G'\) est un sous-groupe de \(G\) et comme \(z\) et \(t\) sont dans
      \(G'\) alors \(z*y^{-1} \in G'\) d'où \(f(z*y^{-1}) \in f(G')\). On a
      montré que \(x\#y^{-1} \in f(G')\).
  \end{itemize}
  Par caractérisation des sous-groupes, \(f(G')\) est un sous-groupe de \(H\).
\end{proof}

\begin{theo}\label{theo:theo2sousgroupe}
  Soit \(H'\) un sous-groupe de \(H\). Alors l'image réciproque \(f^{-1}(H')\)
  est un sous-groupe de \(G\).
\end{theo}
\begin{proof}
  \(f^{-1}(H')\) est un image réciproque et \(f^{-1}(H')=\enstq{x \in G}{f(x)
  \in H'}\).
  \begin{itemize}
    \item Par définition \(f^{-1}(H') \subset G\);
    \item \(f(e_G)=e_H\), \(H'\) est un sous-groupe de H donc \(e_H \in H'\) et
      donc \(f(e_G) \in H'\), alors \(e_G \in f^{-1}(H')\) donc \(f^{-1}(H')\)
      est non-vide;
    \item soient \((x,y) \in f^{-1}(H')^2\), montrons que \(x*y^{-1} \in
      f^{-1}(H')\); comme \(f\) est un morphisme
      \begin{equation}
        f(x*y^{-1}) = f(x) \# f(y)^{-1},
      \end{equation}
      et comme \((x,y) \in f^{-1}(H')^2\) alors \((f(x), f(y)) \in H'^2\) et
      puisque \(H'\) est un sous-groupe \(f(y)^{-1} \in H'\) et \(f(x) \#
      f(y)^{-1} \in H'\). Finalement \(f(x*y^{-1}) \in H'\) donc \(x*y^{-1} \in
      f^{-1}(H')\).
  \end{itemize}
  Par caractérisation des sous-groupes, \(f^{-1}(H')\) est un sous-groupe de
  \(G\).
\end{proof}
\begin{defdef}
  Soient \((G,*)\) et \((H,\#)\) deux groupes et \(f\) un morphisme de \(G\) sur
  \(H\). On pose
  \begin{align}
    \Image(f) &= \enstq{f(x)}{x \in G} = \enstq{y \in H}{\exists x \in G \quad
    y=f(x)}=f(G)\\
    \Ker(f) &= \enstq{x \in G}{f(x)=e_{G'}}=f^{-1}(\{e_{G'}\}).
  \end{align}
\end{defdef}
\begin{prop}
  L'image \(\Image(f)\) est un sous-groupe de \(H\) et le noyau \(\Ker(f)\) est
  un sous-groupe de \(G\).
\end{prop}
\begin{proof}
  On sait que \(\Image(f)=f(G)\) et comme \(G\) est un sous-groupe de \(G\)
  alors d'après le théorème~\ref{theo:theo1sousgroupe} \(\Image(f)\) est un sous-groupe. On sait aussi que
  \(\Ker(f)=f^{-1}(\{e_H\})\) et \(\{e_H\}\) est un sous-groupe de \(H\) donc
  d'après le théorème~\ref{theo:theo2sousgroupe} \(\Ker(f)\) est un sous-groupe.
\end{proof}
\begin{prop}
  Soit \(f:(G,*) \longrightarrow (H,\#)\) un morphisme de groupes. Alors \(f\)
  est surjective si et seulement si \(\Image(f)=H\).
\end{prop}
\begin{proof}
  Déjà faite au chapitre~\ref{chap:ensembles}.
\end{proof}
\begin{prop}
  Soit \(f:(G,*) \longrightarrow (H,\#)\) un morphisme de groupes. Alors \(f\)
  est injective si et seulement si \(\Ker(f)=\{e_G\}\).
\end{prop}
\begin{proof}
  Supposons \(f\) injective, alors comme c'est un morphisme de groupes
  \(f(e_G)=e_H\) alors \(\{e_G\} \subset \Ker(f)\). Soit \(x \in \Ker(f)\) alors
  \(f(x)=e_H=f(e_G)\) or \(f\) est injective donc \(x=e_G\) donc \(\Ker(f)
  \subset \{e_G\}\). Finalement \(\Ker(f)=\{e_G\}\).

  Supposons que \(\Ker(f)=\{e_G\}\). Soit \((x,y) \in G^2\) tels que
  \(f(x)=f(y)\) alors \(f(x)\#f(y)^{-1}=f(y)\#f(y)^{-1} = e_H\). Comme \(f\) est
  un morphisme on a \(f(x)\#f(y)^{-1} = f(x*y^{-1})\). Alors \(f(x*y^{-1}) =
  e_H\) et donc \(x*y^{-1} \in \Ker(f)\). Comme \(\Ker(f)=\{e_G\}\) on a
  \(x*y^{-1} = e_G\). Par conséquent \((x*y^{-1})*y = e_G*y\) et par
  associativité et grâce aux qualités du neutre on a \(x=y\). Alors \(f\) est
  injective.
\end{proof}

\subsubsection{Exemples}

La fonction logarithme est un isomorphisme de groupes de \((\R^{*}_+,\times)\)
sur \((\R,+)\), c'est-à-dire
\begin{equation}
  \forall (x,y) \in {\Rplusetoile}^2 \quad \ln(xy) = \ln x + \ln y.
\end{equation}
La fonction exponentielle est un isomorphisme de groupes de \((\R,+)\) sur
\((\R^{*}_+,\times)\), c'est-à-dire
\begin{equation}
  \forall (x,y) \in \R^2 \quad \e^{x+y} = \e^x \times \e^y.
\end{equation}
La fonction \(\fonction{\psi}{\Rplusetoile \times
\R}{\C^{*}}{(r,\theta)}{r\e^{\ii \theta}}\) est un morphisme de groupes du
groupe produit des groupes \((\Rplusetoile,\times)\) et \((\R,+)\) sur le groupe
\((\C^{*},\times)\)
\begin{equation}
  \forall (r,r') \in {\Rplusetoile}^2 \ \forall \theta \in \R \quad
  \psi(rr',\theta+\theta')=\psi(r,\theta) \times \psi(r',\theta').
\end{equation}
C'est un morphisme surjectif, \(\Image(f)=\C^{*}\) mais non injectif puisque
\(\Ker(f)=\{1\}\times 2\pi \Z\)~\footnote{cf.\ chapitre~\ref{chap:complexes}}.

\section{Anneau et corps}

\subsection{Structure d'anneau}

\subsubsection{Définition}

\begin{defdef}
  On appelle anneau tout triplet \((A,+,\times)\) où \(+\) et \(\times\) deux
  lois de composition internes sur \(A\) telles que
  \begin{enumerate}
    \item \((A,+)\) est un groupe abélien;
    \item \(\times\) est associative;
    \item \(\times\) admet un élément neutre noté \(1\);
    \item \(\times\) est distributive sur \(+\) :
      \begin{equation}
        \forall (a,b,c) \in A^3 \qquad a \times (b+c)= a \times b + a \times c
        \quad (a+b) \times c = a \times c + b \times c.
      \end{equation}
  \end{enumerate}
  Si de plus la loi \(\times\) est commutative, on dira que l'anneau
  \((A,+,\times)\) est commutatif.
\end{defdef}

\subsubsection{Propriétés}

Soit \((A,+,\times)\) un anneau. Déjà \((A,+)\) est un groupe abélien, donc
\begin{itemize}
  \item il existe un unique élément neutre pour la loi \(+\) noté \(0\) et
    appelé élément nul de A;
  \item tous les éléments de A admettent un unique symétrique pour \(+\) dans A
    noté \(-x\);
  \item tous les éléments de A sont réguliers pour \(+\);
  \item il existe un unique élément neutre pour \(\times\) noté \(1\) et c'est
    l'élément unité de A.
\end{itemize}

Mais \((A,\times)\) n'est pas un groupe. Les éléments de \(A\) n'admettent pas
en général d'inverse pour \(\times\). Les éléments de \(A\) ne sont en général
pas réguliers pour \(\times\).

\subsection{Exemples d'anneaux}

\subsubsection{Anneaux usuels de nombres}

Les quatre anneaux suivants sont commutatifs : \((\Z,+,\times)\),
\((\Q,+,\times)\), \((\R,+,\times)\), \((\C,+,\times)\). Pour \(\Z\) les
éléments inversibles sont 1 et \(-1\) et sont non nuls pour \(\Q\), \(\R\) et
\(\C\).

\subsubsection{Ensemble quelconque}

Soit \(E\) un ensemble quelconque, \((P(E), \Delta, \bigcap)\) est un anneau.

\subsubsection{Anneau produit}

Soient \((A,+_1,\times_1)\) et \((B,+_2,\times_2)\) deux anneaux. On munit \(A_1
\times A_2\) des lois produits \(+\) et \(\times\) définies par
\begin{align}
  \forall (a_1,b_1) \in A_1^2 \ \forall (a_2,b_2) \in A_2^2 \quad
  (a_1,a_2)+(b_1,b_2) = (a_1 +_1 b_1, a_2 +_2 b_2);\\
  \forall (a_1,b_1) \in A_1^2 \ \forall (a_2,b_2) \in A_2^2 \quad (a_1,a_2)
  \times (b_1,b_2) = (a_1 \times_1 b_1, a_2 \times_2 b_2).
\end{align}
Alors \((A_1 \times A_2, +, \times)\) est un anneau commutatif appelé anneau
produit des anneaux \(A_1\) et \(A_2\). S'ils sont commutatifs alors \(A_1
\times A_2\) est commutatif.

\subsubsection{Anneau nul}

Soit \(A=\{a\}\). Alors \(a+a=a\) et \(a \times a=a\) alors \(0=1=a\) donc
\(A=\{0\}\). \(A\) est un anneau commutatif appelé anneau nul.

\subsubsection{Anneau de fonctions}

Soient \(X\) un ensemble non nul, \((A,+,\times)\) un anneau. On munit \(A^X\)
de deux lois de composition internes :
\begin{itemize}
  \item une addition définie par
    \begin{equation}
      \forall (f,g) \in \left(A^X\right)^2 \ \forall x \in X \quad
      (f+g)(x)=f(x)+g(x);
    \end{equation}
  \item une multiplication définie par
    \begin{equation}
      \forall (f,g) \in \left(A^X\right)^2 \ \forall x \in X \quad (f \times
      g)(x)=f(x) \times g(x).
    \end{equation}
\end{itemize}
Alors \((A^X,+,\times)\) est un anneau. Si \(A\) est commutatif alors \(A^X\)
est aussi commutatif. L'élément nul est l'application nulle et l'élément unité
est la fonction constante égale à 1.

\subsection{Anneau intègre}

\begin{defdef}
  Soit \((A,+,\times)\) un anneau. S'il existe \((a,b) \in A^2\) tous non nuls
  tels que \(ab=0\). On dit alors que \((a,b)\) est un couple diviseur de zéro
  dans A.
\end{defdef}
Comme \(A\) n'est pas commutatif, on peut avoir \(ab=0\) et \(ba \neq 0\).
\begin{defdef}
  Soit \((A,+,\times)\) un anneau. On dit que \(A\) est un anneau intègre si
  \begin{enumerate}
    \item la multiplication est commutative;
    \item \(A\) n'est pas réduit à l'anneau nul;
    \item \(A\) n'admet pas de diviseur de zéro.
  \end{enumerate}
\end{defdef}
\begin{theo}
  Si \((A,+,\times)\) est un anneau intègre. Alors
  \begin{equation}
    \forall (a,b) \in A^2 \quad ab=0 \implies a=0 \text{~ou~} b=0.
  \end{equation}
\end{theo}
\begin{proof}
  l'anneau \(A\) n'admet pas de diviseurs de zéro, alors il n'existe pas de
  couple \((a,b)\) tel que \(a \neq 0\) et \(b \neq 0\) et \(ab=0\). Alors
  lorsqu'on prend la négation de cette assertion, on a bien
  \begin{equation}
    \forall (a,b) \in A^2 \quad ab=0 \implies a=0 \text{~ou~} b=0.
  \end{equation}
\end{proof}
\begin{theo}
  Dans un anneau intègre, tout élément non nul est simplifiable par la
  multiplication
\end{theo}
\begin{proof}
  Soient \(a \in A^*\) et \((x,y) \in A^2\) alors
  \begin{align}
    ax=ay &\implies ax + (-ay) = ay - ay \\
    &\implies a(x-y)=0.
  \end{align}
  Alors \(a=0\) ou \(x=y\) puisque A est intègre. Cependant \(a\) n'est pas nul
  donc \(x=y\).

  On a montré que \(a\) est simplifiable à gauche. On peut faire de même pour
  montrer qu'il est simplifiable à droite. Alors \(a\) est simplifiable.
\end{proof}

Les anneaux \(\Z\), \(\Q\), \(\R\) et \(\C\) munis des lois usuelles sont
intègres. Si l'ensemble \(X\) a au moins deux éléments, l'anneau \(\R^X\) n'est
pas intègre. Il existe deux éléments différents \(a,b \in A\) tel que \(f\) est
nulle partout sauf en \(a\) et \(g\) est nulle partout sauf en \(b\). Alors
\(f\) et \(g\) sont non nulles et pourtant \(fg\) est nulle. De la même manière
l'ensemble des suites réelles \(\R^N\) n'est pas intègre. L'anneau
\((\R^2,+,\cdot)\) n'est pas intègre puisque \((1,0) \cdot (0,1)=(0,0)\).

\subsection{Règles de calcul dans un anneau}

\begin{theo}
  Soit \((A,+,\cdot)\) un anneau alors
  \begin{gather}
    \forall x \in A \quad x \cdot 0 = 0 \cdot x = 0; \\
    \forall (x,y) \in A^2 \qquad x(-y)=-(xy) \quad (-y)x=-(yx); \\
    \forall (x,y,y') \in A^3 \qquad x(y-y')=xy-xy' \quad (y-y')x=yx-y'x; \\
    \forall (x,y) \in A^2 \ \forall k \in \Z \qquad x(ky)=k(xy) \quad
    (ky)x=k(xy); \\
    \forall n \in \N^* \forall (x, y_1, \ldots, y_n) \in A^{n+1} \qquad x
    \sum_{i=1}^n y_i = \sum_{i=1}^n xy_i \quad \left(\sum_{i=1}^n y_i\right)x =
    \sum_{i=1}^n y_ix.
  \end{gather}
\end{theo}
\begin{proof}
  Soit \(x \in A\), on définit \(\fonction{f}{A}{A}{y}{xy}\). La fonction \(f\)
  est bien définie puisque \((A,+,\cdot)\) est un anneau donc il est stable. De
  plus \(f\) est un endomorphisme du groupe \((A,+)\)
  \begin{equation}
    \forall (y,y') \in A^2 \quad f(y+y')=x(y+y')=xy+xy'=f(y)+f(y').
  \end{equation}
  Alors les propriétés découlent des propriétés des endomorphismes de groupe.
\end{proof}
\begin{theo}
  Soit \((A,+,\cdot)\) un anneau non nul, alors
  \begin{enumerate}
    \item \(1 \neq 0\)
    \item \(0\) n'est pas inversible pour la multiplication.
  \end{enumerate}
\end{theo}
\begin{proof}
  Si \(0=1\) alors pour tout \(x \in A\) on a \(x=x\cdot 1=x\cdot 0=0\), alors
  l'anneau serait nul. Contradiction.

  Pour tout \(x \in A\) \(x \cdot 0=0 \neq 1\) alors \(0\) n'est pas inversible
  dans A pour la multiplication.
\end{proof}

\begin{lemme}
  Soit \((A,+,\cdot)\) un anneau. Pour un couple d'éléments \((a,b) \in A^2\)
  qui commutent, c'est-à-dire tels que \(ab=ba\), on a \(\forall p,q \in \N
  \quad a^pb^q=b^qa^p\).
\end{lemme}
\begin{proof}
  On démontre ce résultat par récurrence sur \(p \in \N\) la propriété \(\P(p)
  "\forall (a,b) \in A^2 \quad ab=ba \implies a^pb=ba^p"\).

  L'initialisation est vraie. Soit \(p \in \N\) supposons \(\P(p)\) alors pour
  tout \(a,b \in A\) tels que \(ab=ba\). Alors
  \begin{align}
    a^{p+1}b=(a \cdot a^p)b&=a(a^p \cdot b) \\
    &=a(b\cdot a^p)\\
    &=(ab) \cdot a^p\\
    &=(ba) \cdot a^p\\
    &=b(a \cdot a^p)\\
    &=b a^{p+1}.
  \end{align}
  Ainsi \(\P(p+1)\) est vraie. Alors par théorème de récurrence La propriété
  \(\P\) est vraie pour tout \(p\). Les éléments \(a^p\) et \(b\) commutent. Il
  suffit d'appliquer ce résultat en posant \(p=q\), \(a=b\) et \(b=a^p\) et on
  obtient le résultat \(b^qa^p=a^pb^q\).
\end{proof}
\begin{theo}[Formule du binôme de Newton]
  Soit \((A,+,\cdot)\) un anneau. Pour deux éléments \((a,b) \in A^2\) qui
  commutent on a
  \begin{equation}
    \forall n \in \N^* \quad (a+b)^n=\sum_{k=0}^n \binom{n}{k} a^k b^{n-k}.
  \end{equation}
\end{theo}
\begin{proof}
  Il suffit d'appliquer le lemme et la démonstration est la même qu'au chapitre~\ref{chap:naturels}.
\end{proof}
\begin{theo}
  Soient \((A,+,\cdot)\) un anneau, \(n \in \N\) et \(a,b \in A\) commutant
  alors
  \begin{equation}
    a^{n+1}-b^{n+1}=(a-b)\sum_{k=0}^n a^kb^{n-k}.
  \end{equation}
\end{theo}
\begin{proof}
  \begin{align}
    (a-b)\sum_{k=0}^n a^kb^{n-k} &= a \sum_{k=0}^n a^kb^{n-k} - b \sum_{k=0}^n
    a^kb^{n-k}\\
    &=\sum_{k=0}^n a^{k+1}b^{n-k} - \sum_{k=0}^n ba^kb^{n-k}\\
    &=\sum_{j=1}^n a^jb^{n+1-j} - \sum_{k=0}^n b^{n-k+1}a^k \\
    &=\sum_{j=1}^n b^{n+1-j}a^j - \sum_{k=0}^n b^{n-k+1}a^k \\
    &=b^0a^{n+1}-b^{n+1}a^0\\
    &=a^{n+1}-b^{n+1}.
  \end{align}
\end{proof}
\begin{corth}
  Soit \((A,+,\cdot)\) un anneau et \(a \in A\). Alors pour tout naturel \(n\)
  \begin{equation}
    1-a^{n+1}=(1-a)\sum_{j=0}^n a^j.
  \end{equation}
  Si \(1-a\) est inversible dans l'anneau alors
  \begin{equation}
    \sum_{j=0}^n a^j = (1-a)^{-1}(1-a^{n+1}).
  \end{equation}
\end{corth}
Dans un corps, tous les éléments non nuls sont réguliers. Le corollaire sera
valable pour tout \(a\) différent de 1.

\begin{theo}[Distributivité généralisée]
  Soit \((A,+,\cdot)\) un anneau. Pour tout \(I\) et \(J\) des ensembles finis
  non vides, toutes familles \((x_i)_{i \in I}\) et \((y_j)_{j \in J}\)
  d'éléments de \(A\) on a
  \begin{equation}
    \left(\sum_{i \in I} x_i \right)\left(\sum_{j \in J} y_j \right) =
    \sum_{(i,j)\in I\times J} x_iy_j.
  \end{equation}
\end{theo}
\begin{proof}[Démonstration par récurrence]
  On pose par récurrence sur \(n \in \N^*\) la propriété \(\P(n)\)~: quelque
  soient les ensembles \(I\) fini de cardinal \(n\) et \(J\) fini et non vide et
  pour toute famille d'éléments de A \((x_i)_{i \in I}\) et \((y_j)_{j \in J}\)
  on a
  \begin{equation} \left(\sum_{i \in I} x_i \right)\left(\sum_{j \in J} y_j
    \right) = \sum_{(i,j)\in I\times J} x_iy_j.
  \end{equation}

  L'initialisation est vraie puisque
  \begin{equation}
    \forall x \in A \ \forall (y_j)_{j \in J} \in A^J \quad x \left(\sum_{j \in
    J} y_j \right) = \sum_{j \in J} xy_j.
  \end{equation}

  Supposons ensuite que la propriété est vraie au rang \(n\), montrons la au
  rang \(n+1\). Soient \(I\) fini de cardinal \(n+1\) et \(J\) fini non vide,
  soient aussi deux familles \((x_i)_{i \in I}\) et \((y_j)_{j \in J}\)
  d'éléments de \(A\). Soit \(i_0 \in I\) et \(I'=I\setminus\{i_0\}\) alors
  \(I'\) est de cardinal \(n\). Alors
  \begin{align}
    \left(\sum_{i \in I} x_i \right)\left(\sum_{j \in J} y_j \right) &=x_{i_0}
    \sum_{j \in J} y_j + \left(\sum_{i \in I'} x_i \right)\left(\sum_{j \in J}
    y_j \right) \\
    &=\sum_{j \in J} x_{i_0}y_j +  \sum_{(i,j)\in I'\times J} x_iy_j.
  \end{align}
  Puisque \(\P(1)\) et \(\P(n)\) sont vraies. Alors en regroupant
  \begin{equation}
    \left(\sum_{i \in I} x_i \right)\left(\sum_{j \in J} y_j \right) =
    \sum_{(i,j)\in I\times J} x_iy_j.
  \end{equation}
  Donc \(\P(n+1)\) est vraie. Le théorème de récurrence nous affirme donc que
  \(\P(n)\) est vraie pour tout naturel non nul \(n\).
\end{proof}
\begin{theo}\label{theo:uinversible}
  Soit \((A,+,\cdot)\) un anneau. On note \begin{equation}
  \uinv{A}=\enstq{a \in A}{a \text{~inversible pour la loi~} \cdot}.
  \end{equation}
  Alors \((\uinv{A},\cdot)\) est un groupe appelé groupe des inversibles (ou des
  unités) de \(A\).
\end{theo}
\begin{proof}
  La loi \(\cdot\) est une loi de composition interne sur \(\uinv{A}\) puisque
  si \(a\) et \(b\) sont des éléments de A inversibles, leur produit \(ab\) est
  inversible d'inverse \(b^{-1}a^{-1}\). La loi \(\cdot\) est associative car A
  est un anneau. On a \(1 \in \uinv{A}\), donc il est non vide. Soit \(a \in
  \uinv{A}\) et notons \(a^{-1}\) l'inverse de A, alors \(a^{-1} \in \uinv{A}\).

  Tout élément de \(\uinv{A}\) admet un inverse dans \(\uinv{A}\). Donc
  finalement \((\uinv{A},\cdot)\) est un groupe.
\end{proof}

\subsection{Sous-anneau}

\subsubsection{Définition}

\begin{defdef}
  Soit \((A,+,\cdot)\) un anneau. On appelle sous-anneau de \(A\), toute partie
  \(B\) de \(A\) telle que~:
  \begin{enumerate}
    \item \(B\) est stable pour les lois \(+\) et \(\cdot\);
    \item muni des lois induites, \(B\) est un anneau;
    \item \(1 \in B\).
  \end{enumerate}
\end{defdef}

\subsubsection{Propriétés préliminaires}

Soit \((A,+,\cdot)\) un anneau et \(B\) un sous-anneau de \(A\), alors
\begin{enumerate}
  \item \(B\) et \(A\) ont le même \(0\);
  \item \(B\) et \(A\) ont le même \(1\);
  \item Si \((A,+,\cdot)\) est un anneau commutatif, alors \(B\) est aussi
    commutatif;
  \item Si \((A,+,\cdot)\) est un anneau intègre, alors \(B\) est aussi intègre.
\end{enumerate}

\begin{proof}
  \((B,+)\) est un sous-groupe de \((A,+)\). \((B,+,\cdot)\) est un anneau, il
  admet donc un neutre pour \(\cdot\). C'est aussi un neutre pour la loi
  \(\cdot\) de \(A\). Alors a fortiori par unicité du neutre ils sont égaux.
  Idem pour \(0\). Si la loi sur \(A\) est commutative, alors elle est
  commutative sur B. Si maintenant A est intègre, alors il est commutatif, donc
  \(B\) est commutatif. \(A\) est un anneau non nul donc \(0 \neq A\) sur \(A\).
  Comme ils ont le même zéro et le même un on a \(O \neq 1\) sur \(B\). Si on
  considère un couple \((a,b)\) diviseur de zéro dans \(B\) alors c'est aussi un
  couple diviseurs de zéro dans \(A\), et comme \(A\) est intègre, il n'en a pas
  donc le couple \((a,b)\) en question n'existe pas et alors \(B\) n'en a pas
  non plus. Finalement \(B\) est un anneau intègre.
\end{proof}

\subsubsection{Caractérisation des sous-anneaux}

\begin{theo}
  Soit \((A,+,\cdot)\) un anneau et \(B\) une partie de \(A\). Alors il y a
  équivalence entre
  \begin{enumerate}
    \item \(B\) est un sous-anneau de \(A\);
    \item \((B,+)\) est un sous-groupe de \((A,+)\), \(B\) est stable pour
      \(\cdot\) et \(1 \in B\);
    \item \(B \subset A\) et \(\forall (x,y) \in B^2 \quad x-y \in B\)  et  \(xy
      \in B\) et \(1 \in B\).
  \end{enumerate}
\end{theo}
\begin{proof}
  2 et 3 sont équivalents grâce à la caractérisation des sous-groupes. 1
  entraîne 2 par définition. Montrons ensuite que 1 entraîne 2. \(B\) est stable
  par \(\cdot\) par hypothèse et comme \((B,+)\) est un sous-groupe de \((A,+)\)
  il est stable par \(+\). On a \(1 \in B\). Il reste à montrer que B, muni des
  lois induites, est un anneau
  \begin{itemize}
    \item \((B,+)\) est un groupe abélien car c'est un sous-groupe de \(A\);
    \item \(\cdot\) sur \(B\) est un restriction de la loi associative \(\cdot\)
      sur \(A\), donc elle est associative;
    \item \(\cdot\) est aussi distributive sur \(+\) sur \(B\) puisqu'elle l'est
      sur \(A\);
    \item \(1 \in B\).
  \end{itemize}
  Alors \(B\) est un anneau.
\end{proof}

\subsection{Exemples}

Si \((A,+,\cdot)\) est un anneau, alors \(A\) est un sous-anneau mais en général
\(\{0\}\) n'est pas un sous-anneau de \(A\).

\((\Z,+,\cdot)\) est un sous-anneau de \((\R,+,\cdot)\).
\(\Z[\sqrt{2}]=\{a+b\sqrt{2}, (a,b) \in \Z^2\}\) est un sous-anneau de \(\R\).

\subsection{Morphisme d'anneaux}

\begin{defdef}
  Soient \((A,+,\cdot)\) et \((B,+,\cdot)\) deux anneaux. On appelle morphisme
  d'anneaux de \(A\) sur \(B\) toute application \(\fonctionR{f}{A}{B}\) telle
  que
  \begin{enumerate}
    \item \(\forall (a,b) \in A^2 \quad f(a+b)=f(a)+f(b)\);
    \item \(\forall (a,b )\in A^2 \quad f(a \cdot b)=f(a) \cdot f(b)\);
    \item \(f(1)=1\).
  \end{enumerate}
\end{defdef}

Si \(f\) est un morphisme d'anneaux de \((A,+,\cdot)\) et \((B,+,\cdot)\) alors
c'est un morphisme de groupe de \((A,+)\) sur \((B,+)\).

\subsection{Corps et sous-corps}

\subsubsection{Corps}

\begin{defdef}
  On appelle corps tout anneau \((\K,+,\cdot)\) commutatif, non nul, dans lequel
  tous les éléments non nul admettent un inverse pour \(\cdot\).
\end{defdef}
\begin{prop}
  Soit \((\K,+,\cdot)\) un corps. Alors
  \begin{enumerate}
    \item \(\K\) admet au moins deux éléments distincts;
    \item \(\uinv{\K}=\K\setminus\{0\}\);
    \item \((\K\setminus\{0\}, \cdot)\) est un groupe;
    \item \(\K\) est un anneau intègre.
  \end{enumerate}
\end{prop}
\begin{proof}
  Par définition tous les éléments non nuls sont inversibles. Attention à la
  notation étoilé. \(A^*\) peut signifier \(\uinv{A}\) ou \(A\setminus\{0\}\)
  qui peuvent être différents. Comme par exemple \(\uinv{\Z}=\{-1,1\}\) et
  \(\Z^*=\Z\setminus\{0\}\)
  Puisque \(\K\setminus\{0\} = \uinv{\K}\) alors c'est un groupe.

  Soit \(a,b \in K\) tous non nuls. Si \(ab=0\) et comme \(a\) est non nul alors
  il est inversible donc \(b=0\) (Contradiction). Donc \(\K\) n'admet pas de
  diviseur de zéro. En plus il est commutatif et non nul donc il est intègre.
\end{proof}

\subsubsection{Sous-corps}

\begin{defdef}
  Soit \((\K,+,\cdot)\) un corps. On appelle sous-corps de \(\K\) tout
  sous-anneau \(\L\) de \(\K\) qui, muni des lois induites par celles de \(\K\),
  est un corps.
\end{defdef}
\begin{theo}[Caractérisation des sous-corps]
  Soit \((\K,+,\cdot)\) un corps. Soit \(\L\) une partie de \(\K\). Il y a
  équivalence entre
  \begin{enumerate}
    \item \(\L\) est un sous-corps de \(\K\);
    \item \(\L\) est un sous-anneau de \(\K\) et pour tout élément non nul \(x
      \in \L\) on a \(x^{-1} \in \L\).
  \end{enumerate}
\end{theo}
\begin{proof}
  Si \(\L\) est un sous-corps de \(\K\) alors \(\L\) est un sous-anneau de
  \(\K\). Muni des lois induites, \(\L\) est un corps et dans un corps tous les
  éléments non nuls sont inversibles.

Si \(\L\) est un sous-anneau de \(\K\), il faut juste vérifier que, muni des
lois induites par celles de \(\K\), c'est un corps. Déjà \(\L\) est un
sous-anneau commutatif, ensuite c'est un sous-anneau de \(\K\) donc \(0 \in \L\)
et \(1 \in \L\). Comme \(\K\) est un corps \(0 \neq 1\). Donc \(\L\) n'est pas
réduit à \(\{0\}\). Tout \(x\) non nul dans \(\L\) admet un inverse dans \(\K\)
et c'est le même dans \(\L\). \(\L\) muni des lois induites est un corps. C'est
donc un sous-corps de \(\K\). \end{proof}

\subsubsection{Exemples}

Les ensembles de nombres \(\Q\), \(\R\) et \(\C\) sont des corps. Un corps à
deux éléments est \(\K=\{0,1\}\) tel que \(0+0=0\), \(0+1=1+0=1\),  \(1+1=0\),
\(0 \cdot 0= 1 \cdot 0=0 \cdot 1=0\) et \(1 \cdot 1=1\).

\subsubsection{Corps des fractions d'un anneau intègre}

\begin{theo}[Admis]
  Soit \((A,+,\cdot)\) un anneau intègre. Il existe un corps \(\K\), unique à
  isomorphisme près, tel que \(A\) soit un sous-anneau de \(\K\) et tel que les
  éléments de \(\K\) soient de la forme \(\frac{a}{b}\), avec \(a,b \in A\) et
  \(b \neq 0\). Ce corps \(\K\) est appelé corps des fractions de l'anneau
  intègre \(A\).
\end{theo}

``Unique à isomorphisme près'' signifie que si deux corps \(\K\) et \(\K'\)
vérifient ces hypothèses alors ils sont isomorphes. Comme par exemple, \(\Q\)
est le corps des fractions de l'anneau intègre \(\Z\). \(\K(X)\), l'ensemble des
fractions rationnelles, est le corps des fractions de l'anneau des polynômes
\(K[X]\).

\section{Arithmétique dans \(\Z\)}

\subsection{Entiers relatifs et division euclidienne}

\subsubsection{Anneau \((\Z,+,\cdot)\)}

Les éléments de \(\Z\) sont les entiers relatifs. On admet les résultats
suivants

\paragraph{Structure de \(\Z\)}

\((\Z,+,\cdot)\) est un anneau intègre. L'ensemble des inversibles de \(\Z\),
\(\uinv{\Z}=\{-1,1\}\), muni de la multiplication est un groupe.

\paragraph{Ordre sur \(\Z\)}

Muni de l'ordre usuel \(\leqslant\), \(\Z\) est un ensemble totalement ordonné.
L'ordre est compatible avec la structure d'anneau de \(\Z\).
\begin{align}
  \forall (p,p',q,q') \in \Z^4 &\quad p \leqslant p' \ q \leqslant q' \implies
  p+q \leqslant p'+q';\\
  \forall (p,p',q,q') \in \Z^4 &\quad p \leqslant p' \ 0 \leqslant q \implies pq
  \leqslant p'q.
\end{align}

Toute partie non vide et majorée de \(\Z\) admet un plus grand élément. De la
même manière, toute partie non vide et minorée de \(\Z\) admet un plus petit
élément. \(\Z\) n'admet ni de plus petit ni de plus grand élément.

\paragraph{\(\Z\) est archimédien}

C'est-à-dire que
\begin{equation}
  \forall a \in \Z \ \forall b \in \N^* \ \exists n \in \N \quad a \neq nb.
\end{equation}
\begin{proof}
  Si \(a \leqslant 0\) on peut prendre \(n=0\) puisque \(0 \cdot b = 0 \geqslant
  a\). Sinon \(a \geqslant 1\) et \(b \in \N^*\) donc \(b \geqslant 1\) alors
  \(ab \geqslant a\). On peut prendre \(n=a \in \N\).
\end{proof}

\subsubsection{Division euclidienne}

La division euclidienne d'un entier relatif par un entier relatif non nul est
telle que
\begin{theo}
  Pour tout couple \((a,b) \in \Z \times \N\) avec \(b \neq 0\). Il existe un
  unique couple \((q,r) \in \Z^2\) tel que
  \begin{equation}
    \begin{cases}
      a=bq+r \\ 0 \leqslant r < b
    \end{cases}.
  \end{equation}
\end{theo}
L'entier \(q\) est appelé le quotient de la division euclidienne de \(a\) par
\(b\). L'entier \(r\) est appelé le reste de la division euclidienne de \(a\)
par \(b\).

\begin{proof}[Unicité]
  Soient \((q,r)\) et \((q',r')\) deux couples de \(\Z^2\) tels que
  \begin{equation}
    \begin{cases}
      a=bq+r \\ 0 \leqslant r < b
    \end{cases}
    \begin{cases}
      a=bq'+r' \\ 0 \leqslant r' < b
    \end{cases}.
  \end{equation}
  Alors \(bq+r=bq'+r'\) et donc \(b(q-q')=r'-r\). De plus \(\abs{r'-r} < b\).
  Supposons que \(q-q' \neq 0\), alors \(\abs{q-q'} \geqslant 1\) alors
  \(\abs{r'-r}=b\abs{q-q'} \geqslant b\). Contradiction. Donc \(q=q'\). Par
  conséquent \(r=r'\).
\end{proof}
\begin{proof}[Existence]
  Soit l'ensemble \(A=\enstq{k \in \Z}{bk \leqslant a}\). Alors \begin{itemize}
    \item \(A\) est une partie de \(\Z\);
    \item \(A\) est non vide~:
      \begin{itemize}
        \item si \(a \geqslant 0\) alors \(0b \leqslant a\) donc \(0 \in A\);
        \item si \(a < 0\) alors pour tout \(k \in \Z\) \(bk \leqslant a \iff -a
          \leqslant b(-k)\);
        \item \(\Z\) est archimédien et \(b>0\) donc il existe \(m \in \N\) tel
          que \(-a \leqslant bm\) on pose \(k=-m \in \Z\) alors \(bk \leqslant
          a\) donc \(k \in A\).
      \end{itemize}
    \item \(A\) est majorée, puisque pour tout \(k \in \Z\), \(bk \leqslant a
      \leqslant \abs{a} \leqslant b \abs{a}\) car \(b \geqslant 1\). Alors
      \(b(k-\abs{a}) \leqslant 0\) or \(b \geqslant 0\) donc \(k-\abs{a}
      \geqslant 0\) alors \(k \leqslant \abs{a}\). Donc \(\abs{a}\) est un
      majorant de \(A\).
  \end{itemize}

  Finalement \(A\) est une partie non vide et majorée de \(\Z\) donc \(A\) admet
  un plus grand élément et on le note \(q\). On pose \(z=ab-q\). Alors \((q,r)
  \in \Z^2\) par définition. L'élément \(q\) est le plus grand de \(A\) donc \(q
  \in A\) et \(q+1 \notin A\). Ainsi \(bq \leqslant a\) et \(b(q+1) > a\). Alors
  \(0 \leqslant r < b\).

  Le couple \((q,r)\) convient.
\end{proof}

\paragraph{Division euclidienne dans \(\N\)}

\begin{theo}
  Pour tout couple \((a,b) \in \N^2\) avec \(b \neq 0\). Il existe un unique
  couple \((q,r) \in \N^2\) tel que
  \begin{equation}
    \begin{cases}
      a=bq+r \\ 0 \leqslant r < b
    \end{cases}.
  \end{equation}
\end{theo}
\begin{proof}
  On applique le théorème précédent. La seule chose à vérifier est alors \(q \in
  \N\). Supposons que \(q \notin \N\) alors \(q \leqslant -1\). Puisque \(b>0\)
  donc \(bq \leqslant -b\). Alors \(a=bq+r \leqslant r-b <0\), ce qui contredit
  l'hypothèse de départ. Donc \(q \in \N\).
\end{proof}

\paragraph{Autre division euclidienne dans \(\Z\)}

\begin{theo}
  Pour tout couple \((a,b) \in \Z^2\) avec \(b \neq 0\). Il existe un unique
  couple \((q,r) \in \Z^2\) tel que
  \begin{equation}
    \begin{cases}
      a=bq+r \\ 0 \leqslant r < \abs{b}
    \end{cases}.
  \end{equation}
\end{theo}
\begin{proof}[Existence]
  Si \(b>0\), on applique le premier théorème et on a \(b=|b|\). Si \(b<0\), on
  applique le premier théorème à \((a,-b) \in \Z \times \N\) et
  \begin{equation}
    \exists (q^*, r^*) \in \Z^2 \
    \begin{cases}
      a=-bq^* + r^* \\ 0 \leqslant r^* < -b=\abs{b}
    \end{cases}.
  \end{equation}
  On pose \(r=r^*\) et \(q=-q^*\) et on a
  \begin{equation}
    \begin{cases}
      a=bq+r \\ 0 \leqslant r < b
    \end{cases}.
  \end{equation}
\end{proof}
\begin{proof}[Unicité]
  Si on a deux couples \((q,r)\) et \((q',r')\) d'entiers relatifs avec
  \begin{equation}
    \begin{cases}
      a=bq + r \\ 0 \leqslant r < b
    \end{cases}
    \begin{cases}
      a=bq' + r' \\ 0 \leqslant r' < b
    \end{cases}.
  \end{equation}
  On a \(b(q-q')=r'-r\) et on a \(\abs{r'-r} < \abs{b}\). Si \(q \neq q'\) alors
  \(\abs{q-q'} \geqslant 1\) et donc \(\abs{r'-r}=\abs{b}\abs{q-q'} \geqslant
  \abs{b}\). On a une contradiction donc \(q=q'\) et alors \(r=r'\).
\end{proof}

\paragraph{Algorithme de la division euclidienne dans \(\N\)}

Soit \(a,b \in \N\) avec \(b \neq 0\). On notera \(q(a,b)\) et \(r(a,b)\) le
quotient et le reste de la division euclidienne de \(a\) par \(b\).

\begin{prop}
  Si \(a \leqslant b\) alors \(q(a,b)=0\) et \(r(a,b)=a\). Si \(a \geqslant b\)
  alors \(q(a,b)=q(a-b,b)+1\) et \(r(a,b)=r(a-b,b)\).
\end{prop}
\begin{proof}
  Si \(a < b\) alors \(a=0 \cdot b + a\). On a \(0 \in \N\) et \(a \in \N\) et
  \(0 \leqslant a <b\) par unicité du reste et du quotient \(q(a,b)=0\) et
  \(r(a,b)=a\).

  Si \(a > b\) alors \(a=q(a,b)b+r(a,b)\) et donc
  \(a-b=q(a,b)b+r(a,b)-b=(q(a,b)-1)b+r(a,b)\).

  On sait que \(q(a,b)-1 \in \N\) puisque sinon on aurait \(q(a,b) =0\) et alors
  on aurait \(a=r(a,b) \leqslant b\) ce qui contredit l'hypothèse.

  Par unicité de la division euclidienne de \(a-b\) par \(b\).
  \begin{equation}
    \begin{cases}
      q(a-b,b) = q(a,b)-1 \\ r(a-b)=r(a,b)
    \end{cases}.
  \end{equation}
\end{proof}

\subsubsection{Divisibilité}

\begin{prop}
  Pour tout entier \(a\), l'ensemble \(a\Z=\enstq{b \in \Z}{\exists q \in \Z
  b=aq}\) est un sous-groupe de \((\Z,+)\).
\end{prop}
\begin{proof}
  \begin{enumerate}
    \item \(a\Z \subset \Z\);
    \item \(0 \in a\Z\) donc \(a\Z\) est non vide;
    \item soient \((x,y) \in a\Z^2\) alors il existe deux entiers relatifs
      \(q,q'\) tels que \(x=aq\) et \(y=aq'\), alors \(x-y=a(q-q')\) et comme
      \(q-q' \in \Z\) on a \(x-y \in a\Z\).
  \end{enumerate}
  Par caractérisation des sous-groupes, \(a\Z\) est un sous-groupe de
  \((\Z,+)\).
\end{proof}

\paragraph{Divisibilité dans \(\Z\)}

\begin{defdef}
  Soient \((a,b) \in \Z^2\), on dit que \(b\) divise \(a\) dans \(\Z\), ou que
  \(b\) est un multiple de \(a\) dans \(\Z\) et on note \(b\mid a\) si et
  seulement s'il existe \(q \in \Z\) tel que \(a=bq\).
  \begin{equation}
    b\mid a \iff a \in b\Z.
  \end{equation}
\end{defdef}
On a \(0\mid 0\) puisque pour tout \(q \in \Z\), \(0=0 \times q\). Pour tout \(q
\in \Z\) \(q\mid 0\) car \(0=0 \times q\). Pour tout \(q \in \Z\), si \(0\mid
q\) alors \(q=0\).

\begin{prop}
  La relation de divisibilité est réflexive et transitive, cependant elle n'est
  ni symétrique ni antisymétrique.
\end{prop}
\begin{proof}
  Pour tout entier \(a \in \Z\), on a \(a\mid a\). Soient \((a,b,c) \in \Z^2\)
  tels que \(a\mid b\) et \(b\mid c\) alors il existe \(k\) et \(l\) dans \(\Z\)
  tels que \(b=ak\) \(c=lb\) donc \(c=(lk)a\) donc \(a\mid c\). Ainsi la
  relation de divisibilité est transitive. Elle n'est pas symétrique puisque 2
  divise 4 mais 4 ne divise pas 2. Elle n'est pas antisymétrique puisque \(5\mid
  -5\) et \(-5 \mid 5\) mais \(5 \neq -5\).
\end{proof}
\begin{prop}
  \begin{equation}
    \forall (a,b) \in \Z^2 \quad b\mid a \iff a\Z \subset b\Z.
  \end{equation}
\end{prop}
\begin{proof}
  \begin{itemize}
    \item[\(\impliedby\)] Si \(a\Z \subset b\Z\) alors \(a \in b\Z\) donc
      \(b\mid a\).
    \item[\(\implies\)] Si \(b\mid a\) alors il existe \(k \in \Z\) tel que
      \(a=bk\). Soit \(n \in a\Z\), alors il existe \(q \in \Z\) tel que
      \(n=aq\). Alors \(n=aq=b(kq)\) donc \(n \in b\Z\). Alors \(a\Z \subset
      b\Z\).
  \end{itemize}
\end{proof}
\begin{prop}
  \begin{equation}
    \forall (a,b) \in \Z^2 \ b\neq 0 \quad b\mid a \iff r(a,b)=0.
  \end{equation}
\end{prop}
\begin{proof}
  \begin{itemize}
    \item[\(\implies\)] Si \(b\mid a\) alors il existe \(q \in \Z\) tel que
      \(a=bq+0\) et on a \(0\leqslant \abs{b}\) donc par unicité de la division
      euclidienne \(r(a,b)=0\).
    \item[\(\impliedby\)] Si \(r(a,b)=0\) alors \(a=q(a,b) b+r(a,b)\) avec
      \(q(a,b) \in \Z\) et \(r(a,b)=0\) et \(q=q(a,b)\) par définition \(b|a\).
  \end{itemize}
\end{proof}

Si \(b \neq 0\) et si \(b\mid a\), le quotient \(q\) de la division euclidienne
de \(a\) par \(b\) alors \(a=bq\). Il est appelé le quotient exact de \(a\) par
\(b\) et noté parfois \(q=\frac{a}{b}\). Cependant cette notation est un peu
abusive et elle est à éviter.

\paragraph{Entiers relatifs associés}

\begin{defdef}
  Pour tout couple \((a,b) \in \Z^2\) on dit que \(a\) et \(b\) sont associés si
  et seulement si \(a\mid b\) et \(b\mid a\).
\end{defdef}
\begin{prop}
  Pour tout \(a,b \in \Z\), \(a\) et \(b\) sont associés si et seulement si
  \(a\Z=b\Z\), c'est à dire si et seulement si \(a=b\) ou \(a=-b\).
\end{prop}
\begin{proof}
  On a déjà vu que \(a\mid b \iff b\Z \subset a\Z\). Alors \(a\) et \(b\) sont
  associés si et seulement si \(a\mid b\) et \(b\mid a\) donc si et seulement si
  (par double inclusion) \(a\Z=b\Z\).

  Il faut monter la deuxième équivalence. Si \(a=b\) ou \(a=-b\), on a
  clairement \(a\Z=b\Z\). Si \(a\) et \(b\) sont associées alors \(b\mid a\) et
  \(a\mid b\) donc il existe \(k\) et \(l\) dans \(\Z\) tels que \(b=ak\) et
  \(a=bl\). Donc \(b=blk\) donc \(b(1-lk)=0\). L'anneau \(\Z\) est intègre, donc
  \(b=0\) ou \(1-lk=0\).

  Si \(b=0\) alors \(a=bl=0\) donc \(a=b\). Sinon alors \(1-lk=0\) alors
  \((l,k)=(1,1)\) ou \((l,k)=(-1,-1)\). Alors \(a=b\) ou \(a=-b\).
\end{proof}

La relation d'association dans \(\Z\) est une relation d'équivalence. En effet,
elle est réflexive, \(a\) est associé à \(a\)~:
\begin{equation}
  \forall a \in \Z \quad a\Z=a\Z;
\end{equation}
elle est symétrique, si \(a\) est associé à \(b\) alors \(b\) est associé à
\(a\)~:
\begin{equation}
  \forall a,b \in \Z \quad a\Z=b\Z \iff b\Z=a\Z;
\end{equation}
elle est transitive, si \(a\) est associé à \(b\) et si \(b\) est associé à
\(c\) alors \(a\) est associé à \(c\)~:
\begin{equation}
  \forall a,b,c \in \Z \quad a\Z=b\Z \ b\Z=c\Z \iff a\Z=c\Z.
\end{equation}

\subsubsection{Sous groupes additifs de \(\Z\)}

\begin{theo}
  les sous-groupes additifs de \((\Z,+)\) sont les parties de la forme \(a\Z\),
  \(a \in \N\). De plus si \(H\) est un sous-groupe de \((\Z,+)\), il existe un
  unique entier \(a \in \N\) tel que \(H=a\Z\).
\end{theo}
\begin{proof}
  Les \(a\Z\) \(a \in \N\) sont des sous-groupes de \((\Z,+)\). Soit \(H\) un
  sous-groupe de \((\Z,+)\). Si \(H\) est le sous-groupe nul alors c'est un
  \(a\Z\) avec \(a=0\). Sinon, il y a dans \(H\) au moins un entier relatif
  \(n_0\) non nul. Soit alors \(A=\enstq{\abs{h}}{h \in H\setminus\{0\}}\),
  alors c'est une partie de \(\N\) qui est non vide (puisque \(\abs{n_0} \in
  A\)). Alors \(A\) admet un plus petit élément noté \(a\). Alors \(a \in \N\)
  donc \(\abs{a}=a\) et \(a \in H\setminus\{0\}\). Montrons que \(H=a\Z\), par
  double inclusion. Déjà \(a \in H\). Puisque \(H\) est un sous-groupe, alors il
  est stable par addition. Ainsi par récurrence on obtient
  \begin{equation}
    \forall n \in \N^* \quad na \in H.
  \end{equation}
  Si \(n=0\) alors \(na=0 \in H\) puisque c'est un sous-groupe de \((\Z,+)\).
  %Il est aussi stable par passage au symétrique alors
  \begin{equation}
    \forall n \in \Z\setminus\N \quad na=-(-na) \quad -n \in \N,
  \end{equation}
  comme \(-na \in H\) d'après le premier point et comme \(H\) est stable par
  passage au symétrique \(na \in H\). Alors
  \begin{equation}
    \forall n \in \Z \quad na \in H.
  \end{equation}
  On vient de montrer que \(a\Z \subset H\). Soit \(h \in H\), montrons que \(h
  \in a\Z\). On sait que \(a \neq 0\) donc on peut effectuer la division
  euclidienne de \(h\) par \(a\)
  \begin{equation}
    \exists! (q,r) \in \Z^2 \quad \begin{cases} h=aq+r \\ 0 \leqslant r < a
    \end{cases}.
  \end{equation}
  Déjà \(h \in H\) et comme \(aq \in a\Z \subset H\) on a \(r=h-aq \in H\)
  puisque \(H\) est un sous-groupe de \((\Z,+)\). Or \(0 \leqslant r < a\), si
  \(r \neq 0\) alors \(r \in H\setminus\{0\}\) donc \(\abs{r}=r \in A\) or \(r <
  a\) et \(a\) est le plus petit élément de \(H\), c'est impossible.
  Nécessairement \(r=0\). Donc \(a\mid h\) et \(h \in a\Z\). Ainsi \(H \subset
  a\Z\). Finalement par double inclusion \(H=a\Z\). Soit \((a,b) \in \N^2\) tels
  que \(a\Z=b\Z\) alors \(a\) et \(b\) sont associés alors \(a=b\) ou \(a=-b\).
  Or \(a \geqslant 0\) et \(b \geqslant 0\) donc \(a=b\). Si \(H\) est un
  sous-groupe de \((\Z,+)\) alors il existe un unique naturel \(a\) tel que
  \(H=a\Z\).
\end{proof}

\subsection{PGCD \& PPCM}

\subsubsection{Diviseurs communs de deux entiers}

\paragraph{Ensemble des diviseurs d'un entier}

\begin{defdef}
  Soit \(a \in \Z\), on pose
  \begin{equation}
    \Div(a)=\enstq{b \in \Z}{b\mid a}=\enstq{b \in \Z}{\exists k \in \Z \quad
    a=bk}.
  \end{equation}
\end{defdef}
\begin{prop}
  \begin{equation}
    \forall a \in \Z \quad \{1,-1,a,-a\} \subset \Div(a).
  \end{equation}
  Ce sont les diviseurs triviaux. S'il existe d'autres diviseurs, ils sont
  appelés les diviseurs propres de \(a\).
  \begin{align}
    \forall a \in \Z &\quad \Div(a)=\Div(-a)=\Div(\abs{a}); \\
    \forall (a,x) \in \Z^2 &\quad -x \in \Div(a) \iff x \in \Div(a).
  \end{align}
\end{prop}
\begin{prop}
  \begin{align}
    \forall (a,b) \in \Z^2 &\quad b\in \Div(a) \iff b\mid a \iff a \in b\Z;\\
    \forall (a,b) \in \Z^2 &\quad \Div(b) \subset \Div(a) \iff b\mid a \iff a\Z
    \subset b\Z; \\
    \forall (a,b) \in \Z^2 &\quad \Div(a)=\Div(b) \iff a\mid b \text{~et~} b
    \mid a \iff a=b \text{~et~} a=-b.
  \end{align}
\end{prop}

\(b\Z\) est un sous-groupe de \((\Z,+)\) mais pas \(\Div(a)\). En effet, par
exemple \(\Div(1)=\{-1,1\}\) n'est pas un sous-groupe puisque \(0 \notin
\Div(1)\) et puisqu'il n'est pas stable par l'addition.

\begin{prop}
  \(\Div(0)=\Z\) et \(b \in \Div(0) \iff \exists k \in \Z \ 0=bk \quad \forall b
  \in \Z \ 0=b\cdot 0\).
\end{prop}
Si \(a \neq 0\), \(\abs{a}\) est le plus grand élément de \(\Div(a)\) pour
l'ordre usuel (et \(-\abs{a}\) est le plus petit).
\begin{gather}
  \forall a \in \Z^* \quad (a,-a) \in \Div(a)^2 \\
  \forall x \in \Div(a) \quad x\mid a \text{~et~} x\mid -a.
\end{gather}
\(a\) et \(-a\) sont deux plus grand éléments pour la relation de divisibilité.
La relation de divisibilité n'est pas une relation d'ordre dans \(\Z\) car elle
n'est pas antisymétrique, c'est pourquoi il n'y a pas unicité du plus grand
élément.

Si \(a=0\); \(0\) est le plus grand élément de \(\Z\) qui divise \(a\).

\paragraph{Diviseurs communs de deux entiers}

Étant donnés deux entiers relatifs \(a\) et \(b\). L'ensemble des diviseurs
communs de \(a\) et \(b\) est \(\Div(a) \cap \Div(b)\),
\begin{equation}
  \forall (a,b) \in \Z^2 \quad \{-1,1\} \subset \Div(a) \cap \Div(b).
\end{equation}

\begin{defdef}\label{def:premiersentreeux}
  Soit \((a,b) \in \Z^2\). Si \(\Div(a) \cap \Div(b) = \{-1, 1\}\). On dit que
  \(a\) et \(b\) sont premiers entre eux. Cela signifie que les seuls diviseurs
  communs de \(a\) et \(b\) sont \(1\) et \(-1\).
\end{defdef}

\subsubsection{PGCD de deux entiers}

Soit \((a,b) \in \Z^2\).

\paragraph{Sous-groupe \(a\Z+b\Z\)}

\begin{defdef}
  On définit
  \begin{equation}
    a\Z+b\Z = \enstq{ak+bl}{(k,l) \in \Z^2} = \enstq{n \in \Z}{\exists (k,l) \in
    \Z^2 \quad n=ak+bl}.
  \end{equation}
\end{defdef}
\begin{theo}
  \(a\Z+b\Z\) est un sous-groupe de \((\Z,+)\).
\end{theo}
\begin{proof}
  \begin{itemize}
    \item Déjà \(a\Z+b\Z \subset \Z\) par définition;
    \item \(a\Z+b\Z\) n'est pas vide et il contient \(0\);
    \item soient \((x,y) \in (a\Z+b\Z)^2\), alors il existe des entiers relatifs
      \(k,l,m,n\) tels que \(x=ak+bl\) et \(y=am+bn\); alors \(x-y=a(k-m)
      +b(l-n)\) donc \(x-y \in a\Z+b\Z\).
  \end{itemize}
  Par caractérisation des sous-groupes, \(a\Z+b\Z\) est un sous-groupe de
  \((\Z,+)\).
\end{proof}
\begin{corth}
  Il existe un unique naturel \(d\) tel que \(a\Z+b\Z=d\Z\). On l'appelle le
  PGCD de \(a\) et de \(b\). On note \(d=a \wedge b = \pgcd(a,b)\).
\end{corth}
\begin{theo}\label{th:presqueBezout}
  Quelque soient les relatifs \(a\) et \(b\), il existe deux entiers relatifs
  \(u_0,v_0\) tels que \(\pgcd(a,b)=au_0+bv_0\).
\end{theo}
\begin{proof}
  On sait que \(\pgcd(a,b) \in \pgcd(a,b) \Z =a\Z+b\Z\), alors il existe deux
  entiers relatifs \(u_0,v_0\) tels que \(\pgcd(a,b)=au_0+bv_0\).
\end{proof}

Si \(a=b=0\) alors \(a\Z+b\Z=\{0\}\) alors \(\pgcd(a,b)=0\). Si \(a \neq 0\) ou
\(b\neq 0\) alors \(a\) et \(b\) sont dans \(a\Z+b\Z\) et donc \(a\Z+b\Z \neq
\{0\}\) alors \(\pgcd(a,b) \neq 0\).

\begin{theo}
  Quelque soient \(a,b \in \Z\), on a \begin{equation}
    \Div(a) \cap \Div(b) = \Div(\pgcd(a,b)).
  \end{equation}
\end{theo}
\begin{proof}
  On montre cette égalité par deux inclusions. On sait qu'il existe \(u_0, v_0
  \in \Z\) tels que \(\pgcd(a,b)=au_0+bv_0\).

  Soit \(x \in \Div(a) \cap \Div(b)\). Alors \(x\mid a\) et \(x\mid b\). Alors
  \(x \mid au_0\) et \(x\mid bv_0\) et donc \(x\mid au_0+bv_0\). Finalement
  \(x\mid \pgcd(a,b)\) et donc \(x \in \Div(\pgcd(a,b))\). Alors \(\Div(a) \cap
  \Div(b) \subset \Div(\pgcd(a,b))\).

  Soit \(x \in \Div(\pgcd(a,b))\). On sait que \(a\Z+b\Z=\pgcd(a,b)\Z\) et comme
  \(a \in a\Z+b\Z\) alors \(a \in \pgcd(a,b)\Z\) et donc \(\pgcd(a,b) \mid a\).
  De la même manière \(\pgcd(a,b) \mid b\). Or \(x\mid \pgcd(a,b)\), alors par
  transitivité \(x\mid a\) et \(x\mid b\), donc \(x \in \Div(a) \cap \Div(b)\).
  Au final \(\Div(\pgcd(a,b)) \subset \Div(a) \cap \Div(b)\).

  Finalement par double inclusion, \(\Div(a) \cap \Div(b) = \Div(\pgcd(a,b))\).
\end{proof}

\paragraph{Interprétation}

Les diviseurs communs de \(a\) et \(b\) sont les diviseurs de \(\pgcd(a,b)\).
L'entier \(\pgcd(a,b)\) est le plus grand naturel au sens de la divisibilité de
l'ensemble des diviseurs communs de \(a\) et \(b\). D'où son nom de ``Plus Grand
Commun Diviseur''.

On définit l'application \(\fonction{\bigwedge}{\Z\times
\Z}{\Z}{(a,b)}{\pgcd(a,b)}\). Elle vérifie les propriétés suivantes
\begin{enumerate}
  \item \(\bigwedge\) est une loi de composition interne sur \(\Z\);
  \item \(\bigwedge\) est commutative et commutative;
  \item \(\bigwedge\) est pseudo-distributive;
    \begin{equation}
      \forall (a,b,c) \in \Z^2 \quad \pgcd(ca,cb) = \abs{c} \pgcd(a,b);
    \end{equation}
  \item Pour tout relatif \(a\), \(a \wedge a=\abs{a}\), \(a \wedge 0=\abs{a}\)
    et \(a \wedge 1=1\);
  \item Quelque soient \(a,b \in \Z\)
    \begin{equation}
      b\mid a \iff \pgcd(a,b)=\abs{b}.
    \end{equation}
\end{enumerate}

\begin{proof}
  \begin{enumerate}
    \item En effet, le PGCD de deux entiers relatifs est un entier relatif;
    \item Comme \(+\) est commutative \(a\Z+b\Z=b\Z+a\Z\) donc
      \(\pgcd(a,b)=\pgcd(b,a)\);
    \item
      \begin{align}
        \Div((a \wedge b) \wedge c)&= (\Div(a) \cap \Div(b)) \cap \Div(c)\\
        &=\Div(a) \cap (\Div(b) \cap \Div(c))\\
        &=\Div(a \wedge (b \wedge c)),
      \end{align}
      donc \((a \wedge b) \wedge c=a \wedge (b \wedge c)\);
    \item Soient \(d= a \wedge b\) et \(\delta=(ca) \wedge (cb)\), alors \(d\mid
      a\) et \(d\mid b\). Ainsi \(\abs{c}d\mid ca\) et \(\abs{c}d\mid cb\).
      L'entier \(\abs{c}d\) est un diviseur commun de \(ca\) et de \(cb\) donc
      \(\abs{c}d\mid \delta\).

      Il existe deux relatifs tels que \(d=au_0+bv_0\) alors
      \(\abs{c}d=acu_0+bcv_0\). Puisque \(\delta = (ca) \wedge (cb)\) on a
      \(\delta \mid ca\) et \(\delta \mid  bc\). Ainsi \(\delta\mid acu_0\) et
      \(\delta \mid  bcv_0\). D'où \(\delta \mid  acu_0+bcv_0 =cd\). Alors
      \(\delta \mid  \abs{c}d\).

      Puisque \(\abs{c}d\mid \delta\) et \(\delta \mid  \abs{c}d\) donc ils sont
      associés et comme ils sont tous les deux positifs ou nul on a
      \(\abs{c}d=\delta\);
    \item On a
      \begin{align}
        \Div(a) \cap \Div(a) &= \Div(a) = \Div(\abs{a}) \\
        \Div(a) \cap \Div(0) &= \Div(a) \cap \Z = \Div(a) =\Div(\abs{a}) \\
        \Div(a) \cap \Div(1) &= \Div(a) \cap \{-1, 1\} = \{-1, 1\} = \Div(1);
      \end{align}
    \item
      \begin{align}
        b\mid a &\iff \Div(b) \subset \Div(a) \\
        &\iff \Div(b) \cap \Div(a) = \Div(b) = \Div(\abs{b})\\
        &\iff a \wedge b = \abs{b}.
      \end{align}
  \end{enumerate}
\end{proof}

\subsubsection{Algorithme d'Euclide pour la détermination du PGCD}


Nous savons que pour tout relatifs \(a\) et \(b\) on a \(a \wedge 0 = \abs{a}\)
et \(\pgcd(a,-b) = \pgcd(-a,-b) = \pgcd(a,b)\). Alors on peut supposer, sans
perte de généralité, que \((a,b) \in \left(\N^{*}\right)^{2}\). De plus
\(\pgcd(a,b) =\pgcd(b,a)\), alors on peut supposer de plus que \(a \geqslant
b\).

\begin{lemme}
  Soit \(r(a,b)\) le reste de la division euclidienne de \(a\) par \(b\), \(b
  \neq 0\), alors
  \begin{equation}
    \pgcd(a,b) = \pgcd(b,r(a,b)).
  \end{equation}
\end{lemme}
\begin{proof}
  Soit \(q\) le quotient de la division et \(r\) le reste, alors \(a=bq+r\).
  \begin{itemize}
    \item Si \(x \in \Div(b) \cap \Div(r)\) alors \(x \mid b\) et donc \(x \mid
      bq\) et \(x \mid r\), \(x \mid a\) et alors \(x \in \Div(a)\); ainsi \(x
      \in \Div(a) \cap \Div(b)\) et finalement \(\Div(b) \cap \Div(r) \subset
      \Div(a) \cap \Div(b)\);
    \item Si \(x \in \Div(a) \cap \Div(b)\) alors comme \(r=a-bq\), \(x \mid a\)
      et \(x \mid b\) donc \(x \mid r\); alors \(x \in \Div(r)\); ainsi \(x \in
      \Div(r) \cap \Div(b)\) et finalement \(\Div(a) \cap \Div(b) \subset
      \Div(b) \cap \Div(r)\).
  \end{itemize}
  Alors par double inclusion \(\Div(a) \cap \Div(b)= \Div(b) \cap \Div(r)\).
  Donc \(\Div(\pgcd(a,b)) = \Div(\pgcd(b,r))\) et alors \(\pgcd(a,b) =
  \pgcd(b,r)\).
\end{proof}

\paragraph{Algorithme d'Euclide}

Soit \(a \geqslant b\) on pose \(r_0=a\), \(r_1=b\) et \(r_2=r(a,b)\). Alors
\(\pgcd(a,b) = \pgcd(r_0, r_1) = \pgcd(r_1,r_2)\). Tant que \(r_k \neq 0\) on
effectue la division euclidienne de \(r_{k-1}\) par \(r_k\) et on pose
\(r_{k+1}=r(r_{k-1},r_k)\). Alors \(r_{k+1} < r_k\).

La suite \((r_k)_{k \in \N}\) est strictement décroissante, à valeurs dans
\(\N\). La suite s'annule forcément à partir d'un certain rang. Donc il existe
un naturel \(n\) tel que \(r_n \neq 0\) et \(r_{n+1}=0\). Alors \(\pgcd(a,b)\)
est égal au dernier reste non nul dans la suite des division euclidiennes de
l'algorithme d'Euclide.

\subsubsection{PPCM de deux entiers}

Soient \(a\) et \(b\) deux entiers relatifs. L'ensemble \(a\Z \cap b\Z\) est
l'ensemble des entiers multiples communs de \(a\) et \(b\).
\begin{theo}
  pour tout relatifs \(a\) et \(b\), \(a\Z \cap b\Z\) est un sous-groupe de
  \((\Z,+)\).
\end{theo}
\begin{proof}
  On sait que \(a\Z\) et \(b\Z\) sont des sous-groupes de \((\Z,+)\) et comme
  l'intersection de deux sous-groupes est un sous-groupe alors \(a\Z \cap b\Z\)
  est un sous-groupe.
\end{proof}
\begin{corth}
  pour tout relatifs \(a\) et \(b\) il existe un unique naturel \(m\) tel que
  \(a\Z \cap b\Z= m \Z\). L'entier \(m\) est appelé le PPCM de \(a\) et de
  \(b\), on le note \(a \vee b = \ppcm(a,b)\).
\end{corth}
\begin{theo}
  Quelque soient les relatifs \(a\) et \(b\)
  \begin{equation}
    a\Z \cap b\Z = \ppcm(a,b) \Z
  \end{equation}
\end{theo}

\paragraph{Interprétation}

Les multiples communs de \(a\) et de \(b\) sont exactement les multiples de
\(\ppcm(a,b)\). \(\ppcm(a,b)\) est le plus petit naturel, au sens de la
divisibilité, de l'ensemble des multiples communs de \(a\) et de \(b\). D'où le
nom de Plus Petit Commun Multiple.

Si \(a=0\) ou \(b=0\) alors \(a\Z \cap b\Z = \{0\}=0\Z\) donc \(\ppcm(a,b)=0\).
Si \(a\) et \(b\) sont tous les deux non nuls alors \(\ppcm(a,b)\) est non nul
car \(ab \neq 0\) et \(ab \in a\Z \cap b\Z\).

\begin{prop}
  Comme le PPCM est défini de manière unique, on peut définir une application
  \(\fonction{\bigvee}{\Z\times\Z}{\Z}{(a,b)}{\ppcm(a,b)}\). Cette application
  vérifie les propriétés suivantes
  \begin{enumerate}
    \item c'est une loi de composition interne;
    \item elle est commutative, \(\ppcm(a,b)=\ppcm(b,a)\);
    \item elle est associative, \(\ppcm((\ppcm(a,b),c))=\ppcm(a,\ppcm(b,c))\);
    \item elle est pseudo-distributive, \(\ppcm(ca,cb)=\abs{c}\ppcm(a,b)\);
    \item \(\forall a \in \Z \quad \ppcm(a,a)=\abs{a} \ \ppcm(a,0) = 0 \
      \ppcm(a,1)=\abs{a}\);
    \item \(\forall (a,b) \in \Z^2 \quad b\mid a \iff \ppcm(a,b)=\abs{a}\).
  \end{enumerate}
\end{prop}

\begin{proof}
  Soient \(a,b\) et \(c\) trois relatifs
  \begin{enumerate}
    \item le PPCM est un entier relatif;
    \item \(a\Z \cap b\Z = b\Z \cap a\Z\);
    \item
      \begin{align}
        \ppcm((\ppcm(a,b),c) \Z &= a\Z \cap \ppcm(b,c) \Z \\ &= a\Z \cap (b\Z
        \cap c\Z)\\ &= (a\Z \cap b\Z) \cap c\Z \\ &= \ppcm(a,b) \Z \cap c\Z \\
        &=\ppcm(\ppcm(a,b),c) \Z
      \end{align}
      d'où \(\ppcm((\ppcm(a,b),c)=\ppcm(a,\ppcm(b,c))\).
    \item On note \(m=\ppcm(a,b)\) et \(\mu=\ppcm(ca,cb)\). Alors \(a \mid m\)
      et \(b \mid m\). Par conséquent \(ca \mid \abs{c}m\) et \(cb \mid
      \abs{c}m\). Alors \(\abs{c}m\) est un multiple commun de \(ca\) et \(cb\).
      D'où \(\mu=\ppcm(ca,cb)\) divise \(\abs{c}m\).

      On sait que \(ca \mid \mu\) et \(cb \mid \mu\). Alors il existe deux
      relatifs \(p\) et \(q\) tels que \(\mu = cap\) et \(\mu=cbq\). Alors
      \(cap=cbq\) et donc \(c(ap-bq)=0\). Comme \(\Z\) est intègre donc soit
      \(c=0\) ou soit \(ap-bq=0\).
      \begin{itemize}
        \item Si \(c=0\) alors \(\mu=0=\abs{c}=m\);
      \item Si \(c \neq 0\) alors \(ap-bq=0\) donc \(ap=bq\) est un multiple
      commun de \(a\) et de \(b\). C'est donc un multiple du PPCM \(m\). Alors
      \(m \mid ap=bq\) et \(\abs{c}m \mid cap = bqc=\mu\). \end{itemize}
      On a montré que \(\mu \mid \abs{c}m\) et \(\abs{c} m \mid \mu\), or ce
      sont des naturels donc ils sont égaux. \(\mu=\abs{c} m\), soit
      \(\ppcm(ca,cb)=\abs{c}\ppcm(a,b)\).
    \item \(a\Z \cap a\Z = a\Z\), \(a\Z \cap 0\Z = 0\Z\) et \(a\Z \cap 1\Z = a\Z
      \cap \Z=a\Z=|a|\Z\).
    \item \(b \mid a \iff a\Z \subset b\Z \iff a\Z \cap b\Z = a\Z=|a|\Z\).
  \end{enumerate}
\end{proof}

\subsection{Entiers premiers entre eux}

Soient \(a\) et \(b\) deux entiers relatifs. On a défini plus tôt dans ce cours,
cf.\ définition~\ref{def:premiersentreeux}, que \(a\) et \(b\) sont premiers entre eux si et
seulement si \(\Div(a) \cap \Div(b)=\{-1,1\}\) or \(\Div(a) \cap \Div(b) =
\Div(\pgcd(a,b))\). Donc \(a\) et \(b\) sont premiers entre eux si et seulement
si leur PGCD vaut \(1\).

\begin{theo}[Théorème de Bezout]
  Étant donnés deux relatifs \(a\) et \(b\). Ils sont premiers entre eux si et
  seulement s'il existe deux relatifs \(u\) et \(v\) tels que \(au+bv=1\).
\end{theo}
\begin{proof}
  Si \(a\) et \(b\) sont premiers entre eux alors \(\pgcd(a,b) =1\). %D'après le
  théorème~\ref{th:presqueBezout}
  On sait que \(a\Z+b\Z=\pgcd(a,b)\Z=\Z\). On a \(1 \in \Z\) donc \(1 \in
  a\Z+b\Z\) donc il existe deux relatifs \(u\) et \(v\) tels que \(au+bv=1\).

  S'il existe deux relatifs \(u\) et \(v\) tels que \(au+bv=1\). Si \(d \in
  \Div(a) \cap \Div(b)\) alors \(d \mid au\) et \(d \mid bv\) donc \(d \mid
  au+bv=1\) donc \(d \in \{-1,1\}\). Alors \(\Div(a) \cap \Div(b) \subset
  \{-1,1\}\). L'autre inclusion est vraie aussi. Alors \(a\) et \(b\) sont
  premiers entre eux.
\end{proof}

\begin{defdef}
  Soient \(a\) et \(b\) deux entiers relatifs premiers entre eux. Tout couple
  \((u,v)\) de relatifs tel que \(au+bv=1\) est un couple de coefficients de
  Bezout de \((a,b)\).
\end{defdef}

\paragraph{Algorithme de Bezout}

On tente de trouver le couple de Bezout de \((a,b)\). Il s'agit de ``remonter''
l'algorithme d'Euclide. Soient deux naturels \(a\) et \(b\) non nuls avec
\(a>b\). On pose \(r_0=a\), \(r_1=b\) et \(r_2=r(a,b)\) et par récurrence
\(r_k=r(r_{k-1}, r_{k-2})\) tant que \(r_k \neq 0\). Il existe alors un naturel
\(n\) tel que \(r_n \neq 0\) et \(r_{n+1}=0\). On a \(r_n=\pgcd(a,b)=1\).

Il existe une suite \((q_k)_{k \in \llbracket 0, n \rrbracket}\) telle que
\begin{align}
  r_0 &= r_1 q_1+r_2 \\
  r_1 &= r_2 q_2+r_3 \\
  \vdots &= \vdots \notag \\
  r_{n-3} &=r_{n-2} q_{n-2}+ r_{n-1} \\
  r_{n-2} &=r_{n-1} q_{n-1}+ r_{n} \\
r_{n-1} &=r_{n} q_{n}+ r_{n+1}. \end{align}
Alors
\begin{align}
  1=r_n&=r_{n-2} -r_{n-1}q_{n-1} \\
  &=r_{n-2} - (r_{n-3} -r_{n-2}q_{n-2})q_{n-1}\\
  &= \vdots \notag \\
  &=ur_0 + vr_1\\
  &=au+bv.
\end{align}

\begin{theo}[Théorème de Gau\ss{}]
  pour tout relatifs \(a,b\) et \(c\), si \(a \mid bc\) et \(\pgcd(a,b)=1\)
  alors \(a \mid c\).
\end{theo}
\begin{proof}[Première démonstration]
  Les entiers \(a\) et \(b\) sont premiers entre eux alors on peut définir un
  couple de coefficients de Bezout \((u,v)\) tel que \(au+bv=1\) donc
  \(cau+cbv=c\). Par hypothèse \(a \mid bc\) donc \(a \mid bcv\) et de plus \(a
  \mid acu\) donc \(a \mid acu+bcv =c\).
\end{proof}
\begin{proof}[Deuxième démonstration]
  Comme \(\pgcd(a,b)=1\) alors \(\pgcd(ca,cb)=\abs{c}\). Or \(a \mid bc\), par
  hypothèse et \(a \mid ac\) donc \(a \mid \abs{c}\) soit \(a \mid c\).
\end{proof}

\paragraph{Propriétés}

\begin{prop}\label{prop:prop1}
  Soient \(n \in \N^*\) et \(a\), \(b_1\), \ldots, \(b_n\) des entiers relatifs.
  On suppose que \(a\) est premier avec chacun des \(b_i\) (pour \(i\) variant
  de \(1\) à \(n\)). Alors \(a\) est premiers avec \(b_1 \dotsm b_n\).
\end{prop}
\begin{proof}
  On démontre par récurrence sur \(n \in \N^*\) la propriété \(\P(n)\)
  ``\(\pgcd(a,b_1 \dotsm  b_n)=1\)''.

  \emph{Initialisation}. Pour \(n=1\) le résultat est évident.

  \emph{Hérédité}. Soit \(n \in \N^*\). Supposons que \(\P(n)\) soit vraie alors
  par hypothèse de récurrence il existe un couple de coefficients de Bezout
  \((u,v)\) tel que \(au + (b_1 \dotsm b_n) v=1\). On sait aussi que \(b_{n+1}\)
  et \(a\) sont premiers entre eux alors il existe un deuxième couple de
  coefficients de Bezout \((w,z)\) tel que \(aw+b_{n+1}z=1\).

  En multipliant les deux relations on a
  \begin{align}
    1&=(au + (b_1\dotsm b_n)v)(aw+b_{n+1}z)\\
    1&=a^2uw +aub_{n+1}z+ (b_1\dotsm b_n)vaw + (b_1\dotsm b_{n+1})vz\\
    1&=a[auw +ub_{n+1}z+(b_1\dotsm b_n)vw] + (b_1\dotsm b_{n+1})vz
  \end{align}
  On pose \(\mathfrak{U}=auw +ub_{n+1}z+(b_1\dotsm b_n)vw \in \Z\) et
  \(\mathfrak{V}=vz \in \Z\) donc
  \begin{equation}
    1=a\mathfrak{U} + (b_1\dotsm b_n) \mathfrak{V}
  \end{equation}
  Alors d'après le théorème de Bezout, \(\P(n+1)\) est vraie.

  \emph{Conclusion}. La propriété \(\P\) est donc vraie pour tout naturel \(n\)
  non nul.
\end{proof}
\begin{prop}[Généralisation du théorème de Gau\ss{}]\label{prop:gengauss}
  Soient un naturel \(n\) non nul et des relatifs \(a\), \(c\), \(b_1\), \ldots,
  \(b_n\). On suppose que \(a \mid b_1\dotsm b_n\cdot c\) et que pour tout \(i
  \in \intervalleentier{1}{n}\), \(a\) et \(b_i\) sont premiers, alors \(a \mid
  c\).
\end{prop}
\begin{proof}
  D'après la proposition~\ref{prop:prop1} on sait que \(\pgcd(a, b_1 \dotsm b_n)=1\) et d'après le
  théorème de Gau\ss{} on a \(a \mid c\).
\end{proof}

\begin{prop}\label{prop:prop3}
  Soient un naturel \(n\) non nul un relatif \(a\) et \(b_1, \dotsc, b_n\) des
  diviseurs de \(a\). On suppose que les \(b_i\) sont premiers entre eux deux à
  deux. Alors \(b_1 \dotsm b_n \mid a\).
\end{prop}
\begin{proof}
  On montre par récurrence sur \(n \in \N^*\) la propriété \(\P(n)\) ``\(\forall
  a \in \Z \quad \forall b_1, \ldots b_n\) diviseurs de \(a\) premiers antre eux
  deux à deux, \(b_1 \dotsm b_n \mid a\)''

  \emph{Initialisation}. Pour \(n=1\) c'est évident.

  \emph{Hérédité}. Soit \(n \in \N^*\) et on suppose que \(\P(n)\) est vraie.
  Montrons que \(\P(n+1)\) est vraie. Soient des relatifs \(a, b_1, \dotsc,
  b_{n+1}\) tels que les \(b_i\) divisent \(a\) et sont premiers entre eux deux
  à deux. D'après l'hypothèse de récurrence \(b_1 \dotsm b_n  \mid a\). Il
  existe un relatif \(q\) tel que \(a=b_1 \ldots b_n q\). Or \(b_{n+1} \mid a\)
  donc \(b_{n+1} \mid b_1 \dotsm b_n q\). Cependant \(\forall i \in \llbracket
  1,n \rrbracket \ \pgcd(b_{n+1}, b_i)=1\). Donc d'après la proposition~\ref{prop:gengauss} on a \(b_{n+1} \mid q\). Il existe donc un relatif \(r\)
  tel que \(q=b_{n+1} r\). Alors \(a=b_1 \dotsm b_n q=b_1 \dotsm b_{n+1} r\).
  Alors \(b_1 \ldots b_{n+1} \mid a\). \(\P(n+1)\) est donc vraie.

  \emph{Conclusion} Par théorème de récurrence, la propriété \(\P\) est vraie
  pour tout naturel \(n\) non nul.
\end{proof}

\paragraph{Retour sur le PPCM et le PGCD}

Soient deux relatifs \(a\) et \(b\). Si \(x\) est un diviseur non nul de \(a\)
il existe un unique entier relatif \(q\) tel que \(a=xq\)
\(\left("\frac{a}{x}"\right)\).

\begin{prop}\label{prop:premiereprop}
  Soient deux relatifs \(a\) et \(b\) et un naturel non nul \(d\). Alors
  \begin{equation}
    \pgcd(a,b)=d \iff d \mid a \ \text{~et~} \ d \mid b \ \text{~et~} \
    \pgcd\left("\frac{a}{b}", "\frac{b}{a}" \right) =1
  \end{equation}
\end{prop}
\begin{proof}
  Si \(\pgcd(a,b)=d\) alors \(d\mid a\), \(d\mid b\) et il existe deux relatifs
  \(a_1\) et \(b_1\) tels que \(a=a_1d\) et \(b=b_1d\). Or \(\pgcd(a,b)=d\) donc
  \(\pgcd(a_1d,b_1d)=d\). Donc comme \(d \in \N^*\) alors \(\pgcd(a_1,b_1)=1\).

  Si on suppose que \(d \mid a\) et \(d \mid b\) alors il existe deux relatifs
  \(a_1\) et \(b_1\) tels que \(a=a_1d\) et \(b=b_1d\). De plus par hypothèse
  ils sont premiers entre eux. Alors \(\pgcd(a,b)=\abs{d}\pgcd(a_1,b_1)=d\),
  puisque \(d \in \N^*\) et puisque \(a_1\) et \(b_1\) sont premiers entre eux.
\end{proof}
\begin{prop}
  Soient deux entiers relatifs \(a\) et \(b\), alors \(\pgcd(a,b) \cdot
  \ppcm(a,b)= \abs{ab}\)
\end{prop}
\begin{proof}
  \begin{itemize}
    \item Cas 1, \(\pgcd(a,b)=1\). Comme \(ab\) est un multiple commun de \(a\)
      et de \(b\) on a \(\ppcm(a,b) \mid ab\). De plus \(a \mid \ppcm(a,b)\),
      \(b \mid \ppcm(a,b)\) et \(\pgcd(a,b)=1\) alors d'après la proposition~\ref{prop:prop3} on a \(ab \mid \ppcm(a,b)\). Ainsi \(\ppcm(a,b)\) et
      \(ab\) sont associés, c'est-à-dire que \(\ppcm(a,b)=\abs{ab}\). Finalement
      \(\ppcm(a,b)\pgcd(a,b)=\abs{ab}\).
    \item Cas 2, \(\pgcd(a,b) \neq 1\)~:
      \begin{itemize}
        \item si \(\pgcd(a,b) = 0\) alors \(a=b=0\) donc \(ab=0=\pgcd(a,b)\);
        \item si \(\pgcd(a,b)=d \in \N\setminus\{0,1\}\), en utilisant la
          proposition~\ref{prop:premiereprop}, comme \(d  \mid a\) et \(d \mid b\) alors il
          existe deux entiers relatifs \(a_1\) et \(b_1\) tels que \(a=a_1 d\)
          et \(b=b_1d\) et \(\pgcd(a_1,b_1)=1\); si on applique le premier cas
          au couple \((a_1,b_1)\) on obtient
          \begin{equation}
            \pgcd(a_1,b_1) \ppcm(a_1,b_1) = \abs{a_1b_1},
          \end{equation}
          en multipliant par \(d^2\), on obtient
          \begin{equation}
            d\pgcd(a_1,b_1) d\ppcm(a_1,b_1) = d\abs{a_1} d\abs{b_1},
          \end{equation}
          et comme \(d\geqslant 0\) alors \(d=\abs{d}\) et donc
          \begin{equation}
            \pgcd(da_1,db_1) \ppcm(da_1,db_1) = \abs{da_1} \abs{db_1},
          \end{equation}
          en remplaçant, on a bien
          \begin{equation}
            \pgcd(a,b) \cdot \ppcm(a,b)= \abs{ab}.
          \end{equation}
      \end{itemize}
  \end{itemize}
\end{proof}

\subsection{Nombres premiers}

\subsubsection{Notion de nombre premier}

\begin{defdef}
  Un naturel \(n\) est dit nombre premier lorsqu'il a exactement deux diviseurs
  distincts (\(1\) et lui-même) dans \(\N\).
\end{defdef}

\emph{Remarque}~: Le nombre \(1\) n'est pas premier, puisqu'il n'a qu'un seul
diviseur~: lui-même.

\begin{defdef}
  Un entier relatif \(n \in \Z\) est dit premier ou irréductible si et seulement
  si
  \begin{itemize}
    \item \(n\) n'est pas une unité de \(\Z\) (\(n \notin \{-1,1\}\));
    \item \(n\) n'a pas de diviseurs propres dans \(\Z\) (\(\Div(n) = \{-1, 1,
      -n, n\}\)).
  \end{itemize}
  Autrement dit, \(n\) admet exactement quatre diviseurs dans \(\Z\).
\end{defdef}

\subsubsection{Propriétés}

\begin{prop}\label{prop:prepoa1}
  Soient \(a\) et \(p\) deux relatifs tels que \(p\) est premier. Alors
  \begin{equation}
    \pgcd(a,p)=1 \iff p \nmid a.
  \end{equation}
\end{prop}
\begin{proof}
  Si \(p \mid a\) alors \(\pgcd(a,p)=\abs{p} \neq 1\) car \(p\) est premier.
  Alors par contraposée on a montré \(\pgcd(a,p)=1 \implies p \nmid a\).

  Si \(p \nmid a\) alors \(\pgcd(a,p)\) divise \(p\) or \(p\) est premier donc
  \(\pgcd(a,p) \in \{1, \abs{p}\}\). Si on avait \(\pgcd(a,p)=\abs{p}\) alors on
  aurait \(\abs{p} \mid a\), ce qui contredit l'hypothèse. Donc
  \(\pgcd(a,p)=1\).
\end{proof}
\begin{cor}
  Deux entiers relatifs premiers \(p\) et \(q\) sont premiers entre eux dès que
  \(\abs{p} \neq \abs{q}\).
\end{cor}
\begin{prop}
  Soit \(p\) un entier premier, \(n\) un naturel et \(a_1, \dotsc, a_n\) des
  entiers relatifs quelconques. Alors
  \begin{equation}
    p \mid a_1\dotsm a_n \iff \exists k \in \intervalleentier{1}{n} \quad p \mid
    a_k.
  \end{equation}
\end{prop}
\begin{proof}
  Si \(p\) divise l'un des \(a_k\) alors il divise le produit des \(a_k\), c'est
  évident.

  Sinon alors pour tout \(k \in \intervalleentier{1}{n}\), \(p \nmid a_k\).
  Alors d'après la proposition~\ref{prop:prepoa1} pour tout \(k \in \intervalleentier{1}{n}\), on a
  \(\pgcd(p,a_k)=1\). D'après la proposition~\ref{prop:prop1}, pour tout \(k \in \intervalleentier{1}{n}\), \(\pgcd(p, a_1
  \dotsm a_k)=1\). On a montré par contraposée
  \begin{equation}
    p \mid a_1 \dotsm a_n \implies \exists k \in \intervalleentier{1}{n} \quad p
    \mid a_k.
  \end{equation}
\end{proof}
\begin{prop}\label{prop:distinctsunites}
  Tout entier relatif distinct d'une unité admet parmi ses diviseurs au moins un
  nombre premier.
\end{prop}
\begin{proof}
  Soit \(a \in \Z \setminus \{-1,1\}\). Si \(a\) est nul alors \(a=2 \times 0\)
  et \(2\) est premier. S'il est non nul alors \(\abs{a} \geqslant 2\). On
  définit \(\epsilon(a)=\enstq{\abs{b}}{b \mid a \text{~et~} b\notin
  \{0,1,-1\}}\). Alors \(\epsilon(a) \subset \N\) et il est non vide (\(a \in
  \epsilon(a)\)). Alors \(\epsilon(a)\) admet un plus petit élément noté \(p\).

  Alors \(p \in \N\), \(p \mid a\) et \(p \geqslant 2\). Soit \(q \in \N\) tel
  que \(q \mid p\). Alors par transitivité \(q \mid a\).

  Si \(q<p\) alors \(q \notin \epsilon(a)\) car \(p\) est le plus petit élément
  de \(\epsilon(a)\). Alors au final \(q \notin \epsilon(a)\), \(q \mid a\) et
  \(q \in \N\) alors \(q=1\). Les seuls diviseurs de \(p\) dans \(\N\) sont donc
  \(1\) et \(p\). Alors \(p\) est un nombre premier.

  On a trouvé un nombre premier parmi les diviseurs de \(a\).
\end{proof}
\begin{prop}
  L'ensemble des nombres premiers est infini.
\end{prop}
\begin{proof}
  Notons \(\P\) l'ensemble des nombres premiers. Si \(\P\) est fini, alors on
  peut écrire pour tout naturel non nul \(N\) \(\P=\enstq{p_i}{i \in
  \intervalleentier{1}{N}}\). Il est non vide puisque \(2\) est premier. Soit
  \(a=p_1 \dotsm  p_N +1\) et comme \(\P\) est non vide on a \(a \in \N
  \setminus \{0,1\}\). D'après la proposition~\ref{prop:distinctsunites} \(a\) admet au moins un nombre premier parmi ses
  diviseurs. Ce qui s'écrit \(\exists k \in \intervalleentier{1}{n} \quad p_k
  \mid a\). Alors \(p_k \mid a\) et \(p_k \mid p_1 \dotsm p_N\) ce qui implique
  que \(p_k \mid 1\). Ce qui est absurde puisque le seul diviseur de 1 est 1
  lui-même, alors \(p_k\) est premier. Donc \(\P\) est infini.
\end{proof}

\begin{prop}[Crible d'Ératosthéne]
  Tout naturel \(n \geqslant 4\) non premier admet au moins un diviseur \(q\)
  compris entre \(2\) et \(E(\sqrt{n})\) (partie entière).
\end{prop}
\begin{proof}
  Soit un naturel \(n \geqslant 4\) non  premier. Il admet des diviseurs
  propres, donc il existe \((x,y) \in \N^2\) tels que \(n=xy\) avec \(x
  \geqslant 2\) et \(y \geqslant 2\).

  Quitte à intervertir les rôles de \(x\) et \(y\), on peut supposer que \(x
  \leqslant y\). D'où \(x^2 \leqslant n\) et donc \(x \leqslant \sqrt{n}\). Or
  \(x \in \N\) donc \(x \leqslant E(\sqrt{n})\). On a donc trouvé un diviseur
  \(x\) de \(n\) compris entre \(2\) et \(E(\sqrt{n})\).
\end{proof}

Pour l'application pratique du crible, on utilise la contraposée. Si \(n\) est
un entier plus grand que \(4\) qui n'admet pas de diviseurs compris entre 2 et
\(\sqrt{n}\) alors \(n\) est un nombre premier.

Soit \(n \geqslant 4\). Supposons connus tous les nombres premiers inférieurs à
\(E(\sqrt{n})\). Pour chaque \(p\) premier tel que \(p \leqslant E(\sqrt{n})\),
on garde \(p\) et on barre tous les multiples de \(p\) inférieur ou égaux à
\(n\).

Les nombres restants sont les nombres premiers inférieurs à \(n\). Par exemple
pour \(n=100\) \(E(\sqrt{n})=10\). L'ensemble des nombres premiers inférieurs à
10 est \(\{2,3,5,7\}\). Alors on barre tous les multiples de 2, de 3, de 5 et de
7. Au final l'ensemble des nombres premiers inférieurs à 100 est \begin{align}
  \P_{\leqslant 100}&=\{2,3,5,7,11,13,17,19,23,29,31,37,41,43,47,53, \notag \\
  & \phantom{=} 61,67,71,73,79,83,89,97\}.
\end{align}

\subsubsection{Décomposition d'un entier en produit de nombres premiers}

On suppose que \(n\) est un entier naturel.

\begin{theo}
  Soit \(n \in \N\), \(n \geqslant 2\). Il existe \(m \in \N^*\), \(p_1, \dotsc,
  p_m\) des nombres premiers distincts, \(k_1, \dotsc, k_m\) des naturels non
  nuls tels que
  \begin{equation}
    n = \prod_{i=1}^m p_i^{k_i}.
  \end{equation}
  De plus cette décomposition est unique à l'ordre des facteurs près. On dit que
  c'est la décomposition primaire du naturel \(n\). L'ensemble \(\enstq{p_i}{i
  \in \intervalleentier{1}{m}}\) est appelé le support primaire de \(n\).
\end{theo}
\begin{proof}[Existence]
  On démontre par récurrence forte sur \(n \geqslant 2\) la propriété \(\P(n)\)
  ``\(n\) est produit de facteurs premiers''.

  \emph{Initialisation}. Pour \(n=2\). \(2=2\), \(m=1\), \(p_1=2\) et \(k_1=1\).
  Donc \(\P(2)\) est vraie.

  \emph{Hérédité}. Soit \(n \geqslant 2\) et on suppose que \(\P(2), \dotsc,
  \P(n)\) sont vraies. Alors \(n+1\) admet au moins un diviseur de \(p\) qui est
  un nombre premier. Deux cas se présentent
  \begin{enumerate}
    \item \(p=n+1\). Alors \(n+1\) est premier, il est égal à un produit de
      facteurs premiers, \(m=1\), \(p_1=n+1\) et \(k_1=1\).
    \item \(p \neq n+1\) et alors \(p < n+1\). Il existe un naturel \(q\) tel
      que \(n+1=pq\). Le nombre \(p\) est premier donc \(p \geqslant 2\). Alors
      \(2 \leqslant q \leqslant n\). Comme \(\P(q)\) est vraie alors \(q\) est
      égal à un produit de nombres premiers.

      Alors \(n+1=pq\) est lui aussi égal à un produit de facteurs premiers,
      puisque \(p\) est premier. Donc \(\P(n+1)\) est vraie.
  \end{enumerate}

  \emph{Conclusion}. Par théorème de récurrence forte, la propriété \(\P\) est
  vraie pour tout naturel \(n \geqslant 2\).

  Il reste à mettre le produit sous la bonne forme.
\end{proof}
\begin{proof}[Unicité]
  Soient deux décompositions primaires de \(n\)
  \begin{equation}
    n=\prod_{i=1}^m p_i^{k_i} = \prod_{j=1}^r q_j^{l_j}.
  \end{equation}
  Soit \(i \in \intervalleentier{1}{m}\), alors \(p_i \mid n\) et \(p_i\) est
  premier donc il existe un naturel \(j \in \intervalleentier{1}{r}\) tel que
  \(p_i \mid q_j\). Le nombre \(q_j\) est premier et c'est un naturel donc
  \(p_i=1\) ou \(p_i=q_j\). Or \(p_i\) est aussi premier donc \(p_i=q_j\). On
  vient de montrer que \(\{p_1, \ldots, p_m\} \subset \{q_1, \ldots q_r\}\). Par
  symétrie des rôle on a l'autre inclusion et donc l'égalité \(\{p_1, \ldots,
  p_m\} = \{q_1, \ldots q_r\}\) et \(m=r\).

  Quitte à ré-indexer les \(q_j\), on peut supposer que pour tout \(i \in
  \intervalleentier{1}{m}\), \(p_i=q_i\). Alors
  \begin{equation}
    n=\prod_{i=1}^m p_i^{k_i} = \prod_{i=1}^m p_i^{l_i}.
  \end{equation}
  Montrons que pour tout \(i \in \intervalleentier{1}{m}\), \(k_i=l_i\). S'il
  existe \(i_0 \intervalleentier{1}{m}\) tel que \(k_{i_0} \neq l_{i_0}\) (par
  exemple \(k_{i_0} > l_{i_0}\)) alors \(p_{i_0}^{k_{i_0}} \geqslant
  p_{i_0}^{k_{i_0}-l_{i_0}}\). Alors
  \begin{align}
    n&=\prod_{i=1}^m p_i^{k_i} = \prod_{i=1}^m p_i^{l_i} \\
    &p_{i_0}^{l_{i_0}} \cdot p_{i_0}^{k_{i_0}-l_{i_0}} \prod_{i \neq i_0}^m
    p_i^{k_i} = p_{i_0}^{l_{i_0}} \cdot \prod_{i \neq i_0}^m p_i^{k_i}\\
    &p_{i_0}^{k_{i_0}-l_{i_0}} \prod_{i \neq i_0}^m p_i^{k_i} =\prod_{i \neq
    i_0}^m p_i^{k_i},
  \end{align}
  puisque \(p_{i_0}^{l_{i_0}} \neq 0\). Alors \(p_{i_0}\) divise le membre de
  gauche mais pas celui de droite. C'est absurde. Donc pour tout \(i \in
  \intervalleentier{1}{m}\),  \(k_i=l_i\).
\end{proof}
\begin{corth}
  Soit \(n=\prod_{i=1}^m p_i^{k_i}\) une décomposition primaire. Alors
  \begin{enumerate}
    \item pour \(i \in \intervalleentier{1}{m}\) alors \(k_i\) est le plus grand
      entier \(k\) tel que \(p_i^k \mid n\);
    \item les diviseurs de \(n\) dans \(\N\) sont les \(\prod_{i=1}^m
      p_i^{l_i}\) avec \(0 \leqslant l_i \leqslant k_i\) (pas tout à fait une
      décomposition primaire);
    \item Soient deux naturels \(a\) et \(b\) et leurs décompositions primaires
      \begin{equation}
        a=\prod_{i=1}^m p_i^{k_i}, \quad b=\prod_{i=1}^m p_i^{l_i},
      \end{equation}
      ce sont des décompositions primaires dans lesquelles on autorise les
      puissances \(k_i\) et \(l_i\) à être nulles.

      Alors
      \begin{equation}
        \pgcd(a,b) = \prod_{i=1}^m p_i^{\min(k_i,l_i)}, \quad \ppcm(a,b) =
        \prod_{i=1}^m p_i^{\max(k_i,l_i)}.
      \end{equation}
  \end{enumerate}
\end{corth}
\begin{proof}
  \begin{enumerate}
    \item Déjà \(p_i^{k_i} \mid n\). Ensuite, si \(k > k_i\) et si \(p_i^k\)
      divisait \(n\) alors il existerait \(q \in \N\) tel que \(n=q p_i^k\).
      Alors
      \begin{equation}
        qp_i^k = \prod_{j=1}^m p_j^{k_j} = p_i^{k_i} \prod_{j \neq i} p_j^{k_j},
      \end{equation}
      donc
      \begin{equation}
        qp_i^{k-k_i}p_i^{k_i} =  p_i^{k_i} \prod_{j \neq i} p_j^{k_j},
      \end{equation}
      et comme \(p_i^{k_i} \neq 0\) on simplifie et on obtient
      \begin{equation}
        qp_i^{k-k_i} = \prod_{j \neq i} p_j^{k_j}.
      \end{equation}
      Alors \(p_i\) diviserait le terme de gauche mais pas celui de droite,
      c'est absurde donc \(p_i^k\) ne divise pas \(n\).
    \item Déjà les \(p_i^{k_i}\) sont bien des diviseurs de \(n\). Soit \(q \in
      \N\) qui divise \(n\). Si \(p\) est un nombre premier qui divise \(q\),
      alors par transitivité \(p\) divise \(n\). Puisque \(n=\prod_{i=1}^m
      p_i^{k_i}\) il existe \(i \in \intervalleentier{1}{m}\) tel que \(p=p_i\).

      Les diviseurs premiers de \(q\) sont donc à choisir dans \(\{p_1, \ldots,
      p_m\}\). On peut alors écrire \(q=\prod_{i=1}^m p_i^{l_i}\) avec \(\forall
      i \in \intervalleentier{1}{m}\), \(l_i \in \N\). Donc pour tout \(i\in
      \intervalleentier{1}{m}\), \(p_i^{l_i} \mid q\) donc \(p_i^{l_i} \mid n\)
      et d'après le premier point \(l_i \leqslant k_i\).
    \item On pose \(d= \pgcd(a,b)\) et \(\delta = \prod_{i=1}^m
      p_i^{\min(k_i,l_i)}\). Comme \(\min(k_i, l_i) \leqslant k_i\) alors
      d'après le deuxième point \(\delta \mid a\) et de la même manière
      \(\min(k_i, l_i) \leqslant l_i\) donc \(\delta \mid b\). Alors \(\delta
      \mid d\).

      D'autre part \(d \mid a\) et \(d \mid b\) donc on peut écrire, d'après le
      deuxième point, \(d\) sous la forme
      \begin{equation}
        d = \prod_{i=1}^m p_i^{\alpha_i} \quad \alpha_i \leqslant k_i
        \text{~et~} \alpha_i \leqslant l_i.
      \end{equation}
      Alors \(\alpha_i \leqslant \min(k_i,l_i)\) et donc, toujours d'après le
      deuxième point, \(d \mid \prod_{i=1}^m p_i ^{\min(l_i,k_i)}=\delta\).
      Finalement \(d\) et \(\delta\) sont associés et comme ils sont positifs
      \(d=\delta\). On sait que
      \begin{equation}
        ab=\abs{ab} = \pgcd(a,b) \cdot \ppcm(a,b),
      \end{equation}
      alors
      \begin{equation}
        ab = \prod_{i=1}^m p_i^{l_i+k_i}= \prod_{i=1}^m
        p_i^{\min(k_i,l_i)+\max(k_i,l_i)} = \pgcd(a,b) \prod_{i=1}^m
        p_i^{\max(k_i,l_i)},
      \end{equation}
      et comme le PPCM est unique, on en conclue que la formule est vraie.
  \end{enumerate}
\end{proof}

\subsection{Nombres rationnels}

\subsubsection{Corps des nombres rationnels : \(\Q\)}

\begin{defdef}
  On appelle \(\Q\) le corps des fractions de l'anneau intègre \(\Z\). \(\Q\)
  est appelé le corps des rationnels, les éléments de \(\Q\) sont les nombres
  rationnels. Par définition, les éléments de \(\Q\) s'écrivent sous la forme
  \(\frac{a}{b}\) avec \(a\) et \(b\) des relatifs et \(b\) non nul. Par
  définition, pour tout relatifs \(a\) et \(a'\) et tous relatifs non nuls \(b\)
  et \(b'\) on a
  \begin{equation}
    \frac{a}{b} = \frac{a'}{b'} \iff ab'=ba'.
  \end{equation}
\end{defdef}
\begin{defdef}
  Soit \(x \in \Q\). Si \((a,b) \in \Z \times \Z^*\) vérifie \(x=\frac{a}{b}\),
  on dit que \(\frac{a}{b}\) est une représentation du rationnel \(x\).
\end{defdef}

On peut toujours trouver un représentant tel que \(a \in \Z\) et \(b \in \N^*\).
Si \(b <0\) alors \(\frac{a}{b}=\frac{-a}{-b}\).

\((\Q, +, \cdot)\) est un corps où pour tout relatifs \((a,a')\) et tout
relatifs non nul \((b,b')\) on a
\begin{align}
  \frac{a}{b}+\frac{a'}{b'} &= \frac{ab'+ba'}{bb'}; \\
  \frac{a}{b} \cdot \frac{a'}{b'} &= \frac{aa'}{bb'}.
\end{align}

\paragraph{Immersion de \(\Z\) dans \(\Q\)}

L'application \(\fonction{j}{\Z}{\Q}{a}{\frac{a}{1}}\) est un morphisme injectif
d'anneaux. \(j\) induit un isomorphisme de \(\Z\) sur \(\Image{j}\) et on
choisit d'identifier \(\Z\) et \(\Image{j}\). Autrement dit, pour tout \(a \in
\Z\), on identifie \(a\) avec le rationnel \(\frac{a}{1}\).

Ainsi \(\Z\) est un sous-anneau de \(\Q\).

\subsubsection{Relation d'ordre sur \(\Q\)}

\begin{defdef}
  Soient \(\frac{a}{b}\) et \(\frac{c}{d}\) deux éléments de \(\Q\). On définit
  \(\frac{a}{b} \leqslant \frac{c}{d}\) par \((ad-bc)bd \leqslant 0\).
\end{defdef}

C'est légitime car~:
\begin{enumerate}
  \item Le signe de \((ad-bc)bd\) ne dépend pas du choix du représentants des
    rationnels \(\frac{a}{b}\) et \(\frac{c}{d}\) : si
    \(\frac{a}{b}=\frac{a'}{b'}\) et \(\frac{c}{d} = \frac{c'}{d'}\) alors
    \(ab'=b'a\) et \(c'd=cd'\) et
    \begin{align}
      (a'd'-b'c')(b'd')(bd)^2&=(a'd'bd -b'c'bd)(b'd')(bd)\\
      &=(ab'dd'-cd'b'b)(b'd')(bd)\\
      &=(ad-bc)(b'd')^2(bd),
    \end{align}
    les relatifs \((bd)^2\) et \((b'd')^2\) sont positifs ou nuls donc
    \((a'd'-b'c')(b'd')\) a le même signe que \((ad-bc)(bd)\);
  \item soient \(a\) et \(c\) deux relatifs (ce sont aussi des rationnels) et
    \(a \leqslant c\) dans \(\Q\) est équivalent à \(\frac{a}{1} \leqslant
    \frac{c}{1}\) qui est équivalent à \((a\cdot 1-c \cdot 1)\cdot 1 \cdot 1
    \leqslant 0\) dans \(\Z\) qui est équivalent à \(a \leqslant c\) dans
    \(\Z\).
\end{enumerate}

La relations d'ordre définie sur \(\Q\) prolonge la relation définie sur \(\Z\).

\paragraph{Propriétés}

\begin{prop}
  Pour tout relatif \(a\) et tout relatif non nul \(b\)
  \begin{equation}
    \frac{a}{b} \geqslant 0 \iff ab \geqslant_\Z 0.
  \end{equation}
\end{prop}
\begin{proof}
  \begin{align}
    \frac{a}{b} \geqslant 0 &\iff \frac{a}{b} \geqslant \frac{0}{1} \\
    & \iff (a \cdot 1 - 0 \cdot b)\cdot b \cdot 1 \geqslant 0 \\
    & \iff ab \geqslant 0.
  \end{align}
\end{proof}
\begin{prop}
  La relation \(\leqslant\) définie sur \(\Q\) est bien une relation d'ordre,
  c'est-à-dire qu'elle est réflexive, antisymétrique et transitive.
\end{prop}
\begin{proof}
  \begin{equation}
    \forall \frac{a}{b} \in \Q \quad \frac{a}{b} \leqslant \frac{a}{b},
  \end{equation}
  puisque \((ab-ba)b^2=0 \leqslant 0\). La relation d'ordre sur \(\Q\) est
  réflexive.
  \begin{equation}
    \forall (r,s) \in \Q^2 \ \exists (a,c) \in \Z^2 \ \exists (b,d) \in (\Z^*)^2
    \quad r=\frac{a}{b} \ s=\frac{c}{d}.
  \end{equation}
  Si \(r\leqslant s\) et \(s \leqslant r\) alors \((ad-bc)bd \leqslant 0\) et
  \((ad-bc)bd \geqslant 0\) et comme \(\leqslant\) est antisymétrique sur \(\Z\)
  alors \((ad-bc)bd=0\) or \(b\) et \(d\) sont non nuls et comme \(\Z\) est
  intègre alors \(ad=bc\) donc \(r=s\). La relation d'ordre sur \(\Q\) est
  antisymétrique.

  Soient trois rationnels \(r,s\) et \(t\) et leur représentants
  \(r=\frac{a}{b}\), \(s=\frac{d}{c}\) et \(t=\frac{e}{f}\) tels que \(r
  \leqslant s\) et \(s \leqslant t\). Alors
  \begin{align}
    (ad-bc)bd \leqslant 0, \\
    (cf-ed)df \leqslant 0.
  \end{align}
  Comme \(f^2\) et \(b^2\) sont positifs alors
  \begin{align}
    abd^2f^2 \leqslant cdb^2f^2, \\
    cdf^2b^2 \leqslant efd^2b^2.
  \end{align}
  Par transitivité de la relation d'ordre sur \(\Z\)
  \begin{align}
    abd^2f^2 \leqslant efd^2b^2, \\
    (af-eb)bd^2f \leqslant 0, \\
    (af-eb)bf \leqslant 0.
  \end{align}
  Donc \(\frac{a}{b} \leqslant \frac{e}{f}\). La relation d'ordre sur \(\Q\) est
  transitive.
\end{proof}
\begin{prop}
  L'ordre \(\leqslant \) est total sur \(\Q\).
\end{prop}
\begin{proof}
  L'ordre \(\leqslant \) est un ordre total sur \(\Z\) donc \(\forall \left(
  \frac{a}{b}, \frac{c}{d} \right) \in \Q^2\) on a soit \((ad-bc)bd \geqslant
  0\) ou soit \((ad-bc)bd \leqslant 0\). Donc soit \(\frac{a}{b} \leqslant
  \frac{c}{d}\) ou soit \(\frac{a}{b} \geqslant \frac{c}{d}\).
\end{proof}
\begin{prop}
  \((\Q,+,\cdot, \leqslant)\) est un corps totalement ordonné, c'est-à-dire que
  \begin{itemize}
    \item \(\Q\) est un corps;
    \item l'ordre \(\leqslant\) est total sur \(\Q\);
    \item l'ordre \(\leqslant\) est compatible avec les lois \(+\) et \(\cdot\)
      de \(\Q\).
  \end{itemize}
\end{prop}
\begin{proof}
  Soient \(r,r'\) et \(s\) des rationnels. Soient \((a,b)\), \((c,d)\) et
  \((a',b')\) des représentants respectifs de \(r\), \(s\) et \(r'\). On suppose
  que \(r \leqslant r'\). Alors \((ab'-a'b)bb' \leqslant 0\). On sait aussi que
  \(r+s = \frac{ad+cb}{bd}\) et \(r'+s=\frac{a'd+cb'}{b'd}\). Alors
  \begin{align}
    [(ad+bc)b'd-(a'd+cb')bd](bdb'd) &= (ab'd^2+bb'cd-a'bd^2-bb'cd)(bb'd^2) \\
    &=(ab'-a'b)(bb')d^4 \leqslant 0,
  \end{align}
  puisque \(d^4 \leqslant 0\) et \((ab'-a'b)bb' \leqslant 0\). Donc \(r+r'
  \leqslant r'+s\).

  On suppose maintenant que \(s \geqslant 0\), c'est à dire \(cd \geqslant 0\).
  Alors \(rs = \frac{ac}{bd}\) et \(r's = \frac{a'c}{b'd}\). Donc
  \begin{equation}
    (acb'd-a'cbd)(bdb'd) = (ab'-a'b)(bb')(cd)d^2 \leqslant 0,
  \end{equation}
  puisque \(cd \geqslant 0\), \(d^2 \geqslant 0\) et \((ab'-a'b)bb' \leqslant
  0\). D'où \(rs \leqslant r's\).
\end{proof}
\begin{prop}
  Le corps totalement ordonné \(\Q\) est archimédien. C'est-à-dire que
  \begin{equation}
    \forall r \in \Q \ \forall s \in \Q^*_+ \ \exists n \in \N \quad r \leqslant
    ns
  \end{equation}
\end{prop}
\begin{proof}
  Soient \(\frac{a}{b}\) et \(\frac{c}{d}\) des représentants respectifs de
  \(r\) et \(s\). Comme \(s \geqslant 0\) on a \(cd \geqslant 0\). Soit un
  naturel \(n\) alors
  \begin{align}
    r \leqslant ns &\iff \frac{a}{b} \leqslant \frac{nc}{d} \\
    &\iff (ad-bnc)bd \leqslant 0 \\
    &\iff abd^2 \leqslant n(b^2cd).
  \end{align}
  Or l'anneau \((\Z,+,\cdot)\) est archimédien, \(b^2cd \in \Z^*_+\) donc le
  naturel \(n\) existe bien. Alors \(\Q\) est archimédien.
\end{proof}

\subsubsection{Représentation irréductible d'un rationnel}

\begin{defdef}
  Soit un rationnel \(r\). On appelle représentant irréductible du rationnel
  \(r\) tout couple \((a,b) \in \Z \times \Z^*\) tel que
  \begin{enumerate}
    \item \(r=\frac{a}{b}\);
    \item \(\pgcd(a,b)=1\).
  \end{enumerate}
\end{defdef}
\begin{theo}
  Tout rationnel admet au moins un représentant irréductible.
\end{theo}
\begin{proof}
  Soit un rationnel \(r\). Il existe alors \((c,d) \in \Z \times \Z^*\) tel que
  \(r = \frac{a}{b}\). Soit \(d= \pgcd(a,b)\) comme \(b \neq 0\) on a
  \(\pgcd(a,b) \neq 0\). Soient \(a_1\) et \(b_1\) les quotients exacts de \(a\)
  et de \(b\) par \(d\). C'est-à-dire que \(a=a_1d\) et \(b=b_1d\). On sait
  alors que \(\pgcd(a_1,b_1)=1\). Alors \(ab_1=a_1b_1d=a_1b\). D'où
  \(\frac{a}{b}=\frac{a_1}{b_1}\).

  On a trouvé un représentant irréductible de \(r\). On dit que \(r\) est mis
  sous forme irréductible.
\end{proof}

\begin{theo}
  Soient \((a,b) \in \Z \times \Z^*\) et \((p,q) \in \Z \times \Z^*\) tels que
  \(\pgcd(p,q)=1\). Alors
  \begin{equation}
    \frac{a}{b} = \frac{p}{q} \iff \exists d \in \Z^* \ a=dp \text{~et~} b=dp
    \text{~et~} \abs{d}=\pgcd(a,b).
  \end{equation}
\end{theo}
\begin{proof}
  \begin{enumerate}
    \item[\(\impliedby\)] Soit \(d=\pgcd(a,b)\) tel que \(a=dp\) et \(b=dq\)
      alors \(aq=dpq=bp\) donc \(\frac{a}{b} = \frac{p}{q}\).
    \item[\(\implies\)] Si \(\frac{a}{b} = \frac{p}{q}\) alors \(aq=bp\) dans
      \(\Z\) et \(\pgcd(p,q)=1\). Le théorème de Gau\ss{} nous affirme donc que
      \(q \mid b\). Il existe donc \(d \in \Z^*\) tel que \(b=dq\). Alors
      \(aq=bp=dpq\) donc \((a-dp)q=0\) or \(q \neq 0\) donc \(a=dp\) puisque
      \(\Z\) est intègre. Ainsi
      \(\pgcd(a,b)=\pgcd(dp,dq)=\abs{d}\pgcd(p,q)=\abs{d}\).
  \end{enumerate}
\end{proof}
\begin{corth}
  Pour tout \((p,q) \in \Z \times \Z^*\) et \((p',q') \in \Z \times \Z^*\) tels
  que \(\pgcd(p,q)=1\) et \(\pgcd(p',q')=1\), on a
  \begin{equation}
    \frac{p}{q} = \frac{p'}{q'} \iff \exists \mu \in \{-1, 1\} \
    \begin{cases}
      p'=\mu p \\
      q' =\mu q
    \end{cases}.
  \end{equation}
\end{corth}
Tout rationnel admet exactement deux représentants irréductibles et un unique
représentant irréductible dont le dénominateur est un naturel non nul.
