\chapter{Nombres réels}
\label{chap:reels}
\minitoc
\minilof
\minilot
\section{Ensemble ordonné \((\R,\leqslant)\)}

\subsection{Notion d'ordre, éléments remarquables}

\subsubsection{Relation d'ordre \(\leqslant\)}

On admet l'existence d'un ensemble \(\R\) dont les éléments sont appelés réels, 
muni d'une relation d'ordre notée \(\leqslant\), c'est-à-dire d'une relation 
binaire qui est~:
\begin{itemize}
  \item reflexive, \(\forall a \in \R \quad a\leqslant a\);
  \item antisymétrique, \(\forall (a,b) \in \R^2 \quad a\leqslant b\) et \( 
    b\leqslant a \implies a=b\);
  \item transitive, \(\forall (a,b,c) \in \R^2 \quad a\leqslant b\) et 
    \(b\leqslant c \implies a\leqslant c\).
\end{itemize}
L'ensemble \(\R\) est totalement ordonné par \(\leqslant\), c'est-à-dire que 
deux réels quelconques sont toujours comparables.

\subsubsection{Relation d'ordre stricte \(<\)}

\begin{defdef}
  On définit sur \(\R\) une relation binaire, appelée ``ordre'' stricte et notée 
  \(<\) par~:
  \begin{equation}
    \forall (a,b) \in \R^2 \quad x<y \iff x\leqslant y \text{~et~} x\neq y.
  \end{equation}
\end{defdef}
Ce n'est pas une relation d'ordre puisqu'elle n'est pas symétrique.

\subsubsection{Éléments remarquables}

Soit \(A\) une partie de \(\R\), alors
\begin{defdef}
  \begin{enumerate}
    \item on appelle majorant de \(A\) dans \(\R\) tout élément \(a\) de \(\R\) 
      tel que \(\forall x \in A \ x\leqslant a\). On dit que \(A\) est majorée 
      si elle admet un majorant;
    \item on appelle minorant de \(A\) dans \(\R\) tout élément \(b\) de \(\R\) 
      tel que \(\forall x \in A \ b\leqslant x\). On dit que \(A\) est minorée 
      si elle admet un minorant;
    \item la partie \(A\) est bornée si et seulement si elle est majorée et 
      minorée.
  \end{enumerate}
\end{defdef}
\begin{defdef}
  \begin{enumerate}
    \item On appelle un plus grand élément de \(A\) tout élément de \(A\) qui 
      majore \(A\) et s'il existe, il est unique et on l'appelle le plus grand 
      élément de \(A\);
    \item on appelle un plus petit élément de \(A\) tout élément de \(A\) qui 
      minore \(A\) et s'il existe, il est unique et on l'appelle le plus petit 
      élément de \(A\).
  \end{enumerate}
\end{defdef}
L'unicité est une conséquence de l'antisymétrie de la relation d'ordre.

\subsection{Caractérisation d'une borne supérieure dans \(\R\)}

Soit \(A\) une partie de \(\R\).
\begin{defdef}
  \begin{enumerate}
    \item On appelle borne supérieure (ou supremum) de \(A\) dans \(\R\), le 
      plus petit élément, s'il existe, de l'ensemble des majorants de \(A\) et 
      si elle existe elle est unique et on la note \(\sup A\) ou 
      \(\sup\limits_{x\in A} x\);
    \item On appelle borne inférieure (ou infimum) de \(A\) dans \(\R\), le plus 
      grand élément, s'il existe, de l'ensemble des minorants de \(A\) et si 
      elle existe elle est unique et on la note \(\inf A\) ou 
      \(\inf\limits_{x\in A} x\);
  \end{enumerate}
\end{defdef}
\begin{theo}
  Soient \(A\) une partie de \(\R\) et un réel \(a\), alors
  \begin{equation}
    a=\sup A \iff \begin{cases} \forall x \in A \ x\leqslant a \\  \forall 
    \epsilon >0 \ \exists x_0\in A \ a-\epsilon<x_0\end{cases}.
  \end{equation}
\end{theo}
\begin{proof}
  \begin{itemize}
    \item[\(\implies\)] si \(a=\sup A\), alors par définition \(a\) majore \(A\) 
      donc pour tout \(x \in A\) on a \(x\leqslant a\) et s'il existe un 
      \(\epsilon\) positif tel que pour tout \(x\in A\) \(a-\epsilon\geqslant 
      x\) alors \(a-\epsilon\) est un majorant de \(A\), or \(a-\epsilon<a\) et 
      c'est impossible puisque \(a\) est le plus petit des majorants de \(A\). 
      Donc pour tout \(\epsilon >0\) il existe \(x_0\in A \ a-\epsilon<x_0\).
    \item [\(\impliedby\)] Déjà \(a\) est un majorant de \(A\). Supposons qu'il 
      existe un majorant \(b\) qui soit plus petit que \(a\). Soit 
      \(\epsilon=a-b>0\), par hypothèse il existe \(x_0\in A\) tel que 
      \(x_0>a-\epsilon=b\), alors \(b\) n'est pas un majorant de \(A\). On 
      arrive donc à une contradiction. Il n'existe donc pas de majorant de \(A\) 
      qui soit strictement inférieur à \(a\). Le réel \(a\) est donc la borne 
      supérieure de la partie \(A\).
  \end{itemize}
\end{proof}
On peut voir une représentation illustrant un ensemble réel majoré, ses majorant 
et sa borne supérieure sur la figure~
\ref{fig:BorneSup}.
\begin{figure}[h]
  \centering
  \includegraphics[scale=0.7]{./BorneSup.png}
  \caption{Illustration de la borne supérieure d'un ensemble réel}
  \label{fig:BorneSup}
\end{figure}
\begin{theo}
  Soient \(A\) une partie de \(\R\) et \(a\) un réel, alors
  \begin{equation}
    a=\inf A \iff \begin{cases} \forall x \in A \ x\geqslant a \\  \forall 
    \epsilon >0 \ \exists x_0\in A \ x_0<a+\epsilon\end{cases}.
  \end{equation}
\end{theo}
\begin{proof}
  Soit la partie \(B=-A=\{x\in\R; -x\in A\}\). L'ensemble des minorants de \(A\) 
  correspond à l'ensemble des majorants de B. S'il existe le plus grand élément 
  de l'ensemble des minorants de \(A\) sera égal à l'opposé du plus petit 
  élément de l'ensemble des majorants de \(B\).
  \begin{equation}
    \inf A=-\sup B=-\sup(-A).
  \end{equation}
  Alors,
  \begin{align}
    a=\inf A &\iff -a=\sup(-A)\\
             &\iff \begin{cases} \forall x \in -A \ x\leqslant -a \\ \forall 
             \epsilon>0 \ \exists x_0 \in -A \ x_0>-a-\epsilon \end{cases}\\
             &\iff \begin{cases} \forall y \in A \ -y\leqslant -a \\ \forall 
             \epsilon>0 \ \exists y_0 \in A \ -y_0>-a-\epsilon \end{cases}\\
             &\iff \begin{cases} \forall y \in A \ y\geqslant a \\ \forall 
             \epsilon>0 \ \exists y_0 \in A \ y_0<a+\epsilon \end{cases}.
  \end{align}
\end{proof}

Si \(A\) admet un plus petit élément \(a\), alors \(A\) admet une borne 
inférieure et \(\inf A=a\). La réciproque est fausse. De la même manière, Si 
\(A\) admet un plus grand élément \(a\), alors \(A\) admet une borne supérieure 
et \(\sup A=a\) et la réciproque est aussi fausse.

\section{Corps des réels \((\R,+,\times)\)}

\subsection{Définition}

L'ensemble \(\R\) est muni de deux lois de compositions internes, appelées 
addition et multiplication notées respectivement \(+\) et \(\times\) ou 
\(\cdot\) ou rien.
\begin{prop}
  L'ensemble \((\R,+)\) est un groupe abélien (ou commutatif), c'est-à-dire 
  que~:
  \begin{enumerate}
    \item La loi \(+\) est une loi de composition interne associative et 
      commutative
      \begin{equation}
        \forall (x,y,z) \in \R^3 \quad (x+y)+z=x+(y+z) \ x+y=y+x;
      \end{equation}
    \item \(\R\) admet \(0\) comme élément neutre pour \(+\)
      \begin{equation}
        \forall x \in \R \quad x+0=0+x=x;
      \end{equation}
    \item Tout élément \(x\in\R\) admet un symétrique pour la loi \(+\) noté 
      \(-x\)
      \begin{equation}
        \forall x \in \R \quad x+(-x)=-x+x=0.
      \end{equation}
  \end{enumerate}
\end{prop}
\begin{prop}
  L'ensemble \((R,+,\times)\) est un anneau commutatif, c'est-à-dire que~:
  \begin{enumerate}
    \item L'ensemble \((\R,+)\) est un groupe abélien;
    \item La loi \(\times\) est une loi de composition interne sur \(\R\) qui 
      est associative et commutative
      \begin{equation}
        \forall (x,y,z) \in \R^3 \quad (x\times y)\times z=x\times (y\times z) \ 
        x\times y=y\times x;
      \end{equation}
    \item La loi \(\times\) est distributive par rapport à la loi \(+\)
      \begin{equation}
        \forall (x,y,z) \in \R^3 \quad x\times(y+z)=x\times y+x\times z \ 
        (x+y)\times z=x\times z+y\times z;
      \end{equation}
    \item \(\R\) admet \(1\) comme élément neutre pour la loi \(\times\)
      \begin{equation}
        \forall x \in \R \quad x\times 1=1\times x=x.
      \end{equation}
  \end{enumerate}
\end{prop}
\begin{prop}
  L'ensemble \((\R,+,\times)\) est un corps (commutatif), c'est-à-dire que~:
  \begin{enumerate}
    \item L'ensemble \((R,+,\times)\) est un anneau commutatif;
    \item tout élément \(x\) non nul de \(\R\) admet un symétrique pour la loi 
      \(\times\) noté \(\frac{1}{x}\) tel que \(x\cdot 
      \frac{1}{x}=\frac{1}{x}\cdot x=1\);
    \item \(0\neq 1\).
  \end{enumerate}
\end{prop}

\subsection{Propriétés de compatibilité avec l'ordre}

\begin{prop}
  Pour tout réels \(x\), \(y\), et \(z\) on a~:
  \begin{align}
    x\leqslant y &\implies x+z\leqslant y+z \\
    \begin{cases} x\leqslant y \\ 0\leqslant z \end{cases} &\implies xz 
    \leqslant yz.
  \end{align}
  On dit que \((R,+,\times)\) muni de la relation d'ordre \(\leqslant\) est un 
  corps totalement ordonné. La multiplication d'une inégalité par un nombre 
  négatif change le sens de cette inégalité.
\end{prop}
\begin{enumerate}
  \item pour tout réels \(x\), \(y\), \(z\) et \(t\) on a~:
    \begin{equation}
      \begin{cases}x\leqslant y \\ z \leqslant t \end{cases} \implies 
      x+z\leqslant y+t;
    \end{equation}
  \item pour tout réels \(x\), \(y\), \(z\) et \(t\) on a~:
    \begin{equation}
      \begin{cases}0\leqslant x\leqslant y \\ 0 \leqslant z \leqslant t 
      \end{cases} \implies 0\leqslant xz \leqslant yt;
    \end{equation}
  \item pour tout réel \(x\) on a~:
    \begin{equation}
      x>0 \iff \frac{1}{x}>0;
    \end{equation}
  \item pour tout réels \(x\) et \(y\) on a~:
    \begin{equation}
      0<x\leqslant y \iff 0<\frac{1}{y}\leqslant \frac{1}{x}.
    \end{equation}
\end{enumerate}

\subsection{Valeur absolue et distance}

\begin{defdef}
  Pour tout réel \(x\) on définit un réel appelé valeur absolue de \(x\) noté 
  \(\abs{x}\) par
  \begin{equation}
    \abs{x}=\begin{cases} x & \text{si } x\geqslant 0 \\ -x & \text{si } 
    x<0\end{cases},
    \end{equation}
    ou encore \(\abs{x}=\max(x,-x)\).
  \end{defdef}
  \begin{prop}
    Pour tout réel \(x\)
    \begin{gather}
      \abs{x}\geqslant 0 \\
      \abs{x}=0 \iff x=0.
    \end{gather}
  \end{prop}
  \begin{proof}
    si \(x\geqslant 0\) alors \(\abs{x}=x\geqslant 0\) et \(\abs{x}=0 \iff 
    x=0\). Sinon si \(x<0\) alors \(\abs{x}=-x>0\) donc \(\abs{x}\geqslant 0\) 
    et \(\abs{x}\neq 0\).
  \end{proof}
  \begin{prop}
    pour tout réels \(x\) et \(y\) et tout réel non nul \(z\), on a~:
    \begin{gather}
      \abs{xy}=\abs{x}\abs{y}, \\ \abs{\frac{1}{z}}=\frac{1}{\abs{z}}.
    \end{gather}
  \end{prop}
  \begin{prop}[Inégalité triangulaire]
    Soient \(x\) et \(y\) deux réels, alors
    \begin{equation}
      \abs{\abs{x}-\abs{y}}\leqslant \abs{x+y} \leqslant \abs{x}+\abs{y}.
    \end{equation}
  \end{prop}
  \begin{proof}
    D'une part,
    \begin{align}
      (\abs{x}+\abs{y})^2-\abs{x+y}^2 &=\abs{x}^2+\abs{y}^2+2\abs{xy}-(x+y)^2\\
                                      &=x^2+y^2+2\abs{xy}-x^2-y^2-2xy\\
                                      &=2(\abs{xy}-xy)\geqslant 0.
    \end{align}
    Donc \(\abs{x+y}^2\leqslant (\abs{x}+\abs{y})^2\) et puisque les deux 
    membres sont positifs \(\abs{x+y}\leqslant \abs{x}+\abs{y}\). D'autre part,
    \begin{equation}
      \abs{x}=\abs{x+y-y}\leqslant \abs{x+y}+\abs{y}.
    \end{equation}
    Donc \(\abs{x}-\abs{y}\leqslant \abs{x+y}\) et par symétrie des rôles de 
    \(x\) et \(y\) on complète en écrivant \(\abs{\abs{x}-\abs{y}}\leqslant 
    \abs{x+y}\).
  \end{proof}
  \begin{defdef}
    Soit une application \(d:\R\times\R\longmapsto \R\) appelée distance définie 
    comme suit~:
    \begin{equation}
      \forall (x,y) \in \R^2 \quad d(x,y)=\abs{x-y}.
    \end{equation}
  \end{defdef}
  \begin{prop}
    \(d\) est une distance sur \(\R\), c'est-à-dire que pour tout réels \(x\), 
    \(y\) et \(z\) on a
    \begin{enumerate}
      \item \(d(x,y)\geqslant 0\), positivité;
      \item \(d(x,y)=0 \iff x=y\), séparation;
      \item \(d(x,y)=d(y,x)\), symétrie;
      \item \(d(x,y)\leqslant d(x,z)+d(z,y)\), inégalité triangulaire.
    \end{enumerate}
  \end{prop}
  \begin{proof}
    Ce sont des conséquences immédiates des propriétés de la valeur absolue.
  \end{proof}
  \begin{prop}
    pour tout réels \(x\) et \(y\), on a
    \begin{equation}
      \max(x,y)=\frac{x+y+\abs{x-y}}{2} \quad \min(x,y)=\frac{x+y-\abs{x-y}}{2}.
    \end{equation}
  \end{prop}
  \begin{proof}
    Si d'une part \(x\geqslant y\) alors \(\abs{x-y}=x-y\) et donc 
    \(\max(x,y)=\frac{x+y+\abs{x-y}}{2}=\frac{2x}{2}=x\) et 
    \(\min(x,y)=\frac{x+y-\abs{x-y}}{2}=\frac{2y}{2}=y\). Si d'autre part 
    \(x\leqslant y\) alors \(\abs{x-y}=y-x\) et donc 
    \(\max(x,y)=\frac{x+y+\abs{x-y}}{2}=\frac{2y}{2}=y\) et 
    \(\min(x,y)=\frac{x+y-\abs{x-y}}{2}=\frac{2x}{2}=x\).
  \end{proof}
  \begin{defdef}
    Pour tout réel \(x\), on note \(x^+=\max(x,0)\) et \(x^-=\max(-x,0)\).
  \end{defdef}
  \begin{prop}
    Pour tout réel \(x\), on a
    \begin{equation}
      x^+=\frac{x+\abs{x}}{2} \quad x^-=\frac{\abs{x}-x}{2},
    \end{equation}
    ou alors
    \begin{equation}
      \abs{x}=x^++x^- \quad x=x^+-x^-.
    \end{equation}
  \end{prop}
  \begin{prop}
    pour tout réels \(x\) et \(y\) et tout réel \(h>0\)
    \begin{align}
      \abs{x-y}\leqslant h \iff & -h \leqslant x-y \leqslant h\\
                                &\iff y-h \leqslant x \leqslant y+h
    \end{align}
  \end{prop}
  \begin{prop}
    Soit \(A\) une partie de \(\R\), alors \(A\) est bornée si et seulement s'il 
    existe un réel \(M\geqslant 0\) tel que pour tout élément \(a\) de \(A\) on 
    a \(\abs{a}\leqslant M\).
  \end{prop}
  \begin{proof}
    \begin{itemize}
      \item[\(\impliedby\)] Supposons qu'il existe un réel \(M\) positif tel que 
        pour tout élément \(a\) de \(A\) on ait \(\abs{a}\leqslant M\). Alors 
        \(-M\leqslant a\leqslant M\) donc \(-M\) est un minorant de \(A\) et 
        \(M\) en est un majorant. Ainsi \(A\) est bornée.
      \item[\(\implies\)] Supposons que \(A\) soit bornée, c'est-à-dire majorée 
        et minorée, alors il existe deux réels \(m_1\) et \(m_2\) tels que pour 
        tout élément \(a\) de \(A\) on ait \(m_1\leqslant a\leqslant m_2\). Si 
        on pose \(M=\max(m_2,-m_1,0)\) alors \(\abs{a}\leqslant M\).
    \end{itemize}
  \end{proof}

  \subsection{Droite numérique achevée}

  \begin{defdef}
  On adjoint à \(\R\) deux éléments distincts notés \(+\infty\) et \(-\infty\) 
et on note \(\bar{\R}=\R\cup\{+\infty\}\cup\{-\infty\}\). \(\bar{\R}\) est la 
droite numérique achevée.  \end{defdef}
On prolonge la relation d'ordre \(\leqslant\) définie sur \(\R\) à \(\bar{\R}\) 
en posant
\begin{equation}
  \forall x\in \bar{\R} \quad -\infty\leqslant x\leqslant +\infty.
\end{equation}
Le plus petit (grand) élément de \(\bar{\R}\) est \(-\infty\) (\(+\infty\)). On 
prolonge aussi partiellement les lois \(\times\) et \(+\) à \(\bar{\R}\).
\begin{table}[!h]
  \centering
  \begin{tabular}{|c|c|c|c|}\hline
    + & \(-\infty\) & \(y\in\R\) & \(+\infty\) \\ \hline
    \(-\infty\) & \(-\infty\) & \(-\infty\)& Ind \\ \hline
    \(x\in\R\) & \(-\infty\) & \(x+y\) & \(+\infty\) \\ \hline
  \(+\infty\) & Ind & \(+\infty\) & \(+\infty\) \\ \hline \end{tabular}
  \caption{Prolongement de la loi \(+\) à la droite numérique achevée}
\end{table}
\begin{table}[!h]
  \centering
  \begin{tabular}{|c|c|c|c|c|c|}\hline
    \(\times\) & \(-\infty\) & \(x<0\) & \(x=0\) & \(x>0\) & \(+\infty\) \\ 
    \hline
    \(-\infty\) & \(+\infty\)& \(+\infty\)& Ind& \(-\infty\)& \(-\infty\)\\ 
    \hline
    \(y<0\) &\(+\infty\) & xy& 0& xy& \(-\infty\)\\ \hline
    \(y=0\) & Ind& 0& 0& 0&Ind \\ \hline
    \(y>0\) & \(-\infty\)& xy& 0& xy& \(+\infty\)\\ \hline
  \(+\infty\) & \(-\infty\)& \(-\infty\)& Ind& \(+\infty\) & \(+\infty\)\\ 
\hline \end{tabular}
\caption{Prolongement de la loi \(\times\) à la droite numérique achevée}
\end{table}

\subsection{Intervalles de \(\R\)}

\begin{defdef}
  Soient deux réels, \(a\) et \(b\) tels que \(a<b\), les intervalles de \(\R\) 
  sont les parties suivantes~:
  \begin{itemize}
    \item \(\R=\intervalleoo{-\infty}{+\infty}\);
    \item \(\intervalleff{a}{b}=\enstq{x \in \R}{a\leqslant x \leqslant b}\) 
      segment;
    \item \(\intervallefo{a}{b}=\enstq{x \in \R}{a\leqslant x < b}\) intervalle 
      semi-ouvert ou semi-fermé;
    \item \(\intervalleof{a}{b}=\enstq{x \in \R}{a< x \leqslant b}\) idem;
    \item \(\intervalleoo{a}{b}=\enstq{x \in \R}{a< x < b}\) intervalle ouvert;
    \item \(\intervallefo{a}{+\infty}=\enstq{x \in \R}{a\leqslant x}\) 
      demi-droite fermée;
    \item \(\intervalleoo{a}{+\infty}=\enstq{x \in \R}{a< x}\) demi-droite 
      ouverte;
    \item \(\intervalleof{-\infty}{b}=\enstq{x \in \R}{x \leqslant b}\) 
      demi-droite fermée;
    \item \(\intervalleoo{-\infty}{b}=\enstq{x \in \R}{x <b}\) demi-droite 
      ouverte;
  \end{itemize}
\end{defdef}
On note que l'ensemble vide est aussi un intervalle 
\(\emptyset=\intervalleoo{a}{a}\).
\begin{defdef}
  On notera que
  \begin{itemize}
    \item \(\Rpluss=\intervallefo{0}{+\infty}\);
    \item \(\R-=\intervalleof{-\infty}{0}\);
    \item \({\Rpluss}^*=\intervalleoo{0}{+\infty}\);
    \item \(\R-^*=\intervalleoo{-\infty}{0}\);
    \item \(\R^*=\R-^* \cup {\Rpluss}^*\).
  \end{itemize}
  \(\R^*\) n'est pas un intervalle.
\end{defdef}
\begin{defdef}
  Soit \(A\) une partie de \(\R\). On dit que \(A\) est convexe si et seulement 
  si
  \begin{equation}
    \forall (x,y) \in A^2 \quad x\leqslant y \implies \intervalleff{x}{y}\subset 
    A.
  \end{equation}
\end{defdef}
\begin{theo}
  \label{theo:partieconvexe1}
  Soit \(I\) une partie de \(\R\), alors si \(I\) est un intervalle alors \(I\) 
  est convexe.
\end{theo}
\begin{proof}
  Soient \(a\) et \(b\) deux réels tels que \(a\leqslant b\), alors
  \begin{itemize}
    \item \(\R\) est convexe, puisque pour tout réels \(x\) et \(y\), si 
      \(x\leqslant y\) alors \([x,y]\subset \R\);
    \item \(\intervalleff{a}{b}\) est convexe puisque pour tout réels \(x\) et 
      \(y\) de \(\intervalleff{a}{b}\) si \(a\leqslant x\leqslant y\leqslant b\) 
      alors \(\intervalleff{x}{y}\subset \intervalleff{a}{b}\). Idem pour les 
      intervalles semi-ouverts et semi-fermés;
    \item \(\intervallefo{a}{+\infty}\) est convexe, puisque pour tout réels 
      \(x\) et \(y\) de \(\intervallefo{a}{+\infty}\) tels que \(a\leqslant 
      x\leqslant y\) \(\intervalleff{x}{y}\subset \intervallefo{a}{+\infty}\). 
      Idem pour les autres demi-droites.
  \end{itemize}
\end{proof}

\section{Propriété de la borne supérieure}

\subsection{Axiome}

\begin{theo}[Axiome de la borne supérieure]
  \label{theo:bornesup}
  Toute partie non vide et majorée de \(\R\) admet une borne supérieure.
\end{theo}
\begin{theo}
  Toute partie non vide et minorée de \(\R\) admet une borne inférieure.
\end{theo}
\begin{proof}
  Soit \(A\) une partie non vide et minorée de \(\R\). Soit \(B=-A=\enstq{x\in 
  \R}{-x \in A}\). \(B\) est donc non vide et majorée, alors en appliquant 
  l'axiome de la borne supérieure, \(B\) admet une borne supérieure. D'après la 
  caractérisation de la borne inférieure, on en déduit que \(A\) admet une borne 
  inférieure et de plus \(\inf A=-\sup B\).
\end{proof}

\subsection{Intervalles et parties convexes}

\begin{theo}
  Soit \(I\) une partie de \(\R\), alors \(I\) est un intervalle si et seulement 
  si c'est une partie convexe.
\end{theo}
\begin{proof}
  \begin{itemize}
    \item[\(\implies\)] déjà vue au théorème~
      \ref{theo:partieconvexe1}.
    \item[\(\impliedby\)] Supposons \(I\) convexe. Si \(I\) est vide, alors 
      c'est un intervalle. Sinon, soit \(a\in I\) et on définit 
      \(G_a=\intervalleof{-\infty}{a}\cap I\) et 
      \(D_a=\intervallefo{a}{+\infty}\cap I\).
      \begin{itemize}
        \item Si \(D_a\) n'est pas majoré, on va montrer que 
          \(D_a=\intervallefo{a}{+\infty}\).
          \begin{equation}
            \forall M \geqslant a \ \exists b \in D_a \quad b\geqslant M.
          \end{equation}
          Donc \(b\in \intervallefo{a}{+\infty}\) et \(b\in I\). On note que 
          \(\intervalleff{a}{b}\subset \intervallefo{a}{+\infty}\). Comme I est 
          convexe et que \(a,b \in I \ a \leqslant b\) alors 
          \(\intervalleff{a}{b}\subset I\) et \(M\in \intervalleff{a}{b} \subset 
          D_a\). On a donc montré que \(\intervallefo{a}{+\infty} \subset D_a\) 
          et par définition de \(D_a\) l'inclusion réciproque est triviale donc 
          \(D_a=\intervallefo{a}{+\infty}\).
        \item Si \(D_a\) est majorée et non vide (puisque \(a\in D_a\)) alors 
          \(D_a\) admet une borne supérieure notée \(b\). la borne supérieure 
          \(b\) est un majorant de \(D_a\) donc \(D_a \subset 
          \intervalleff{a}{b}\). D'après la caractérisation de la borne 
          supérieure
          \begin{equation}
            \forall \epsilon >0 \ \exists x_0 \in D_a \ b-\epsilon < 
            x_0\leqslant b.
          \end{equation}
          Soit \(x\in \intervallefo{a}{b}\), on pose \(\epsilon=b-x>0\). Il 
          existe un \(x_0\in D_a\) tel que \(b-\epsilon<x_0 \leqslant b\), qui 
          est équivalent à ce que pour tout \(x\in[a,b[\) il existe un \(x_0\in 
          D_a\) tel que \(x<x_0 \leqslant b\). Comme 
          \(\intervalleff{a}{x_0}\subset \intervallefo{a}{+\infty}\) et 
          \(\intervalleff{a}{x_0}\subset I\) (car \(a,x_0 \in I\) et I est 
          convexe). Alors \(\intervalleff{a}{x_0} \subset D_a\). Comme 
          \(a\leqslant x < x_0\leqslant b\) alors \(x\in 
          \intervalleff{a}{x_0}\subset D_a\). Finalement 
          \(\intervallefo{a}{b}\subset D_a \subset \intervalleff{a}{b}\), alors 
          \(D_a=\intervallefo{a}{b}\) ou \(D_a=\intervalleff{a}{b}\).
      \end{itemize}
      De la même manière, on peut montrer que soit 
      \(G_a=\intervalleof{-\infty}{a}\), soit \(G_a=\intervalleff{c}{a}\), soit 
      \(G_a=\intervalleof{c}{a}\).
  \end{itemize}
  Donc comme \(I=G_a\cup D_a\), c'est un intervalle de \(\R\).
\end{proof}

\subsection{Propriété d'Archimède}
\begin{prop}
  Le corps des réels \(\R\) est archimèdien, c'est-à-dire
  \begin{equation}
    \forall x>0 \ \forall y\in \R \ \exists n\in \N \quad nx\geqslant y.
  \end{equation}
\end{prop}
\begin{proof}
  On peut démontrer ce résultat par l'absurde. Supposons qu'il existe un réel 
  \(x>0\) et un réel \(y\) et que pour tout entier naturel \(n\) on ait 
  \(nx<y\). Soit \(A=\enstq{nx}{n\in \N}\), alors \(A\) est non vide (car \(0\in 
  A\)) et majorée par \(y\) d'après l'hypothèse. La partie \(A\) admet donc une 
  borne supérieure noté \(S\). Soit un naturel \(n\), alors \((n+1)x \in A\) 
  donc \((n+1)x\leqslant S\) donc \(nx \leqslant S-x\). Ceci montre que \(S-x\) 
  est un majorant de \(A\), or \(S-x<S\) puisque \(x>0\) et \(S\) est censé être 
  le plus grand des majorants de \(A\). On aboutit donc à une contradiction. 
  L'hypothèse de départ est fausse.
\end{proof}

\subsection{Partie entière d'un réel}\index{partie entiere}

\begin{prop}
  Pour tout réel \(x\), il existe un unique entier \(n\) relatif, tel que 
  \(n\leqslant x<n+1\), c'est la partie entière de \(x\) et on la note \(E(x)\).
\end{prop}
\begin{proof}[Unicité]
  Soient deux entiers relatifs \(n\) et \(m\) tels que \(E(x)=m=n\), alors
  \begin{align}
    n\leqslant x< n+1 \\  m\leqslant x< m+1.
  \end{align}
  Alors
  \begin{align}
    n<m+1 \\  m<n+1,
  \end{align}
  et donc \(-1<n-m<1\) et comme c'est un entier, forcément \(m=n\).
\end{proof}
\begin{proof}[Existence]
  Le corps \(\R\) est archimédien, il existe donc deux naturels \(n\) et \(m\) 
  tels ques \(x\leqslant n\) et  \(-x\leqslant m\). Soit la partie \(A=\enstq{k 
  \in \Z}{k\leqslant x}\), alors \(A\) est majorée par \(n\) et non vide puisque 
  \(-m\in A\). \(A\) est une partie de \(\Z\) non vide et majorée, elle admet 
  donc un plus grand élément noté \(n_0\). Alors \(n_0\in \Z\) et \(n_0\leqslant 
  x\). Puisque \(n_0\) est le plus grand élément de \(A\), \(n_0+1\) n'est pas 
  dans \(A\). donc \(n_0+1>x\). Au final \(n_0\) est la partie entière de \(x\) 
  : \(n_0\leqslant x<n_0+1\).
\end{proof}

\subsection{Ensembles \(\Q\) et \(\R\setminus\Q\)}

\begin{defdef}[Densité]
  Soit \(A\) une partie de \(\R\). On dit que \(A\) est dense dans \(\R\) si et 
  seulement si pour tout couple de réels \((x,y)\) tels que \(x<y\), on ait 
  \(\intervalleff{x}{y}\cap A \neq \emptyset\).
\end{defdef}
\begin{prop}
  L'ensemble des irrationnels \(\R\setminus\Q\) est non vide. Autrement dit, il 
  existe des réels non rationnels.
\end{prop}
\begin{proof}
  On prouve  qu'il existe des réels non rationnels en prenant l'exemple du 
  nombre \(\sqrt{2}\). Supposons que \(\sqrt{2}\) est rationnel, alors il existe 
  deux entiers \(p\) et \(q \neq 0\) tels que \(\sqrt{2}=\frac{p}{q}\) (fraction 
  irréductible), donc \(2q^2=p^2\). L'entier \(p^2\) est pair, donc cela 
  entraine qu'il existe un entier \(k\) tel que \(p=2k\). D'où \(2q^2=p^2=4k^2\) 
  donc \(q^2=2k^2\), alors \(q^2\) est pair, donc \(q\) est pair. Alors \(p\) et 
  \(q\) ne sont pas premiers enre eux. Contradiction. Alors \(\sqrt{2}\) n'est 
  pas rationnel.
\end{proof}

L'ensemble \(\Q\) est un sous-corps de \(\R\), c'est-à-dire que \(\Q\) est 
stable par addition et multiplication et même quotient. Cependant, l'ensemble 
des irrationnels n'est pas un sous-corps, puisqu'il n'est stable ni pour la 
multiplication ni pour l'addition.

\begin{prop}
  \begin{enumerate}
    \item La somme d'un rationnel et d'un irrationnel quelconques est 
      irrationnelle
      \begin{equation}
        \forall x \in \Q \ \forall y \in \R\setminus\Q \quad x+y \in 
        \R\setminus\Q.
      \end{equation}
    \item Le produit d'un rationnel non nul et d'un irrationnel quelconques est 
      irrationnel
      \begin{equation}
        \forall x \in \Q\setminus\{0\} \ \forall y \in \R\setminus\Q \quad xy 
        \in \R\setminus\Q.
      \end{equation}
  \end{enumerate}
\end{prop}
\begin{proof}
  \begin{enumerate}
    \item Par l'absurde, si \(x+y\in\Q\) alors \(y=(x+y)-y \in \Q\), 
      contradiction, donc \(x+y\in\R\setminus\Q\);
    \item idem, par l'absurde, si \(xy\in \Q\) alors \(y=\frac{1}{x}xy\in\Q\). 
      contradiction, donc \(xy\in\R\setminus\Q\).
  \end{enumerate}
\end{proof}

La conséquence est que ces deux ensembles sont infinis.

\begin{theo}
  Les ensembles \(\Q\) et \(\R\setminus\Q\) sont denses dans \(\R\).
\end{theo}
\begin{proof}
  D'après la définition de la densité, il faut montrer que pour tout couple de 
  réels \((x,y)\) avec \(x<y\) il existe un rationnel \(a\) qui soit dans 
  \(\intervalleff{x}{y}\) et un irrationnel \(b\) qui soit aussi dans 
  \(\intervalleff{x}{y}\). Montrons d'abord la densité de \(\Q\). Soit le 
  rationnel \(q=E\left(\frac{1}{y-x}\right)+1\), donc \(0<\frac{1}{q}<y-x\). 
  Soit \(p=E(qx)\), alors \(p\leqslant qx<p+1\) et donc \(\frac{p}{q}\leqslant x 
  <\frac{p+1}{q}\). Donc \(x<\frac{p}{q}+\frac{1}{q}<x+y-x\). Finalement 
  \(x<\frac{p+1}{q}<y\), donc \(\frac{p+1}{q}\) est le rationnel recherché, ce 
  qui prouve que \(\Q\) est dense dans \(\R\). Montrons la densité de 
  \(\R\setminus\Q\). Comme \(\Q\) est dense dans \(\R\), il existe un rationnel 
  \(r\) tel que \(x+\sqrt{2}<r<y+\sqrt{2}\) et \(r-\sqrt{2}\in\R\setminus\Q\) 
  avec \(x<r-\sqrt{2}<y\), on montre que  \(\R\setminus\Q\) est dense dans 
  \(\R\).
\end{proof}

\section{Exercices}
\begin{exercice}
  Soient \(A\) et \(B\) deux parties bornées et non vides de \(\R\). On suppose 
  que leur intersection est non vide. Montrer que l'intersection est bornées et 
  que~:
  \[ \max(\inf A, \inf B) \leqslant \inf(A \cap B) \leqslant \sup(A \cap B) 
  \leqslant \min(\sup A, \sup B)\]
\end{exercice}
\begin{exercice}
  Les parties de \(\R\) suivantes sont-elles majorées, minorées ? Si oui, 
  déterminer leur bornes supérieures, inférieures, et dire s'il s'agit d'un 
  maximum, d'un minimum.
  \begin{align*}
    E &= \enstq{\frac{1}{2^n}+\frac{(-1)^n}{n}}{n \in \N^*} \\
    F &= \enstq{\frac{1+(-1)^n}{n}-n^2}{n \in \N^*}
  \end{align*}
\end{exercice}
\begin{exercice}
  Soit \(A\) une partie non vide de \(\R+\) et \(B = \enstq{x \in \R}{\exists a 
  \in A \ ax=1}\). On suppose que \(A\) est majorée et non réduite au singleton 
  nul. Montrer que \(B\) admet une borne inférieure telle que \(\inf B = 
  \frac{1}{\sup A}\).
\end{exercice}
\begin{exercice}
  Soit \(A = \enstq{r \in \Q}{r^2 \leqslant 2}\). Montrer que \(A\) n'admet pas 
  de borne supérieure dans \(\Q\).
\end{exercice}
\begin{exercice}
  Calculer les bornes supérieures et inférieures de \(\mathcal{E} = 
  \enstq{\frac{3p-q}{2p+q+1}}{(p,q) \in \N^2 \ 0 <q<p}\).
\end{exercice}
\begin{exercice}
  Déterminer \(\inf \enstq{\Ent(x) + \Ent(1/x)}{x \in \R^*+}\).
\end{exercice}
\begin{exercice}
  Soient \(a\) et \(b\) deux réels tels que \(b-a>3\). Démontrer qu'entre \(a\) 
  et \(b\) il existe au moins trois entiers relatifs distincts.
\end{exercice}
\begin{exercice}
  Soit \(f\) une application croissante de \(I = \intervalleff{0}{1}\) vers 
  lui-même. Démontrer qu'il existe au moins un réel \(a \in I\) tel que 
  \(f(a)=a\).
  \emph{Indication}: on pourra utiliser la partie \(A = \enstq{x \in I}{f(x) 
  \geqslant x}\).
\end{exercice}
\begin{exercice}
  Soient \(I\) et \(J\) des intervalles de \(\R\) non vides. Démontrer qu'une 
  condition suffisante pour que \(I \cup J\) soit un intervalle est \(I \cap J 
  \neq \emptyset\). Est-ce une condition nécessaire ?
\end{exercice}
\begin{exercice}
  Soient un réel \(x\) et un naturel non nul \(p\).
  \begin{enumerate}
    \item Démontrer qu'il existe un seul naturel \(k\) avec \(0 \leqslant k 
      \leqslant p-1\) tel que \(\Ent(px) = p\Ent(x)+k\).
    \item Soit un naturel \(q\) tel que \(0 \leqslant q \leqslant p-1\). Prouver 
      que \(\Ent(x+q/p)\) est égal à \(\Ent(x)\) lorsque \(0 \leqslant q 
      \leqslant p-k-1\) et à \(\Ent(x)+1\) lorsque \(p-k\leqslant q \leqslant 
      p-1\).
    \item En déduire que \(\sum_{q=0}^{p-1} \Ent(x+q/p) = \Ent(px)\).
  \end{enumerate}
\end{exercice}

