\chapter{Ensembles finis -- Quelques démonstrations}
\label{chap:ensemblesFinis}
\begin{theo}
  Soient $p$ et $q$ deux naturels non nuls. S'il existe une bijection $\varphi : \intervalleentier{1}{p} \longmapsto \intervalleentier{1}{q}$ alors $p=q$.
\end{theo}
\begin{proof}
  On démontre par récurrence sur $p \in \N^*$ la propriété $\P(p)$ ``pour tout $q \in \N^*$ s'il existe une bijection $\varphi : \intervalleentier{1}{p} \longmapsto \intervalleentier{1}{q}$ alors $p=q$''.

\emph{Initialisation}. Soit $q \in \N^*$. On suppose qu'il existe une bijection $\varphi : \{1\} \longmapsto \intervalleentier{1}{q}$. Si on a $q \geq 2$, comme $\varphi$ est surjective, il existe $x \in \{1\}$ (donc $x=1$) tel que $\varphi(x)=1$. Il existe aussi $y \in \{1\}$ (donc $y=1$) tel que $\varphi(y) = 2$. C'est impossible donc $q=1$. $\P(1)$ est vraie

\emph{Hérédité}. Soit $p \in \N^*$. Supposons que $\P(p)$ soit vraie et montrons $\P(p+1)$. Soit un naturel $q$ non nul. On suppose l'existence d'une bijection $\varphi : \intervalleentier{1}{p+1} \longmapsto \intervalleentier{1}{q}$.

Notons déjà que $q$ est nécessairement supérieur ou égal à 2~\footnote{car sinon $q=1$ or $p \geq 1$ donc $p+1 \geq 2$} d'où $\varphi(1)=1=\varphi(2)$ ce qui est impossible puisque $\varphi$ est injective. L'astuce consiste à construire une bijection $\psi : \intervalleentier{1}{p} \longmapsto \intervalleentier{1}{q-1}$ afin d'utiliser l'hypothèse de récurrence.

On distingue deux cas :
\begin{itemize}
\item Cas 1. $\varphi(p+1)=q$. L'application $\varphi$ est injective, donc pour tout $i \in \intervalleentier{1}{p}$ on a $\varphi(i) \neq q$. Ainsi
  \begin{equation}
    \varphi(\intervalleentier{1}{p}) \subset \intervalleentier{1}{q-1}.
  \end{equation}
  On définit donc $\psi$ comme l'application induite par $\varphi$ de $\intervalleentier{1}{p}$ sur $\intervalleentier{1}{q-1}$. Alors
  \begin{itemize}
  \item $\psi$ est injective : Puisque $\varphi$ est injective
    \begin{equation}
      \forall (i, j) \in \intervalleentier{1}{p}^2 \quad i \neq j \implies \varphi(i) \neq \varphi(j),
    \end{equation}
    et comme $\psi$ et $\varphi$ sont égales sur $\intervalleentier{1}{p}$ alors $\psi$ est injective.
  \item $\psi$ est surjective : Puisque $\varphi$ est surjective,
    \begin{equation}
      \forall i \in \intervalleentier{1}{q-1} \ \exists \alpha \in \intervalleentier{1}{p+1} \quad i=\varphi(\alpha).
    \end{equation}
    On a $\alpha \neq p+1$~\footnote{car $\varphi(p+1)=q \neq i$} donc $\alpha \in \intervalleentier{1}{p}$ et $i= \varphi(\alpha)=\psi(\alpha)$. L'application $\psi$ est surjective.
  \end{itemize}
  Finalement $\psi$ est bijective.
\item Cas 2. $\varphi(p+1) \neq q$. On pose $a=\varphi(p+1)$ et $b=\varphi^{-1}(q)$~\footnote{c'est possible puisque $\varphi$ est bijective}. On définit $    \psi :\intervalleentier{1}{p} \longmapsto \intervalleentier{1}{q-1}$
  par
  \begin{equation}
    \forall i \in \intervalleentier{1}{p} \setminus \{b\} \quad \psi(i)=\varphi(i) \ \psi(b)=a.
  \end{equation}
  
  Cette application est bien à valeurs dans $\intervalleentier{1}{q-1}$, car $\varphi$ est injective et $\varphi(b)=q$. Donc
  \begin{equation}
    \forall i \in \intervalleentier{1}{p} \setminus \{b\} \quad \varphi(i) \in \intervalleentier{1}{q-1} \text{~et~} a=\varphi(p+1) \neq q.
  \end{equation}
  Alors
  \begin{itemize}
  \item $\psi$ est injective~: Soient $(i,j) \in \intervalleentier{1}{p}^2$ et $i \neq j$. Deux sous-cas se présentent~:
    \begin{itemize}
    \item si $i \neq b$ et $j \neq b$ alors comme $\varphi$ est injective $\psi(i)=\varphi(i) \neq \varphi(j) \neq \psi(j)$;
    \item si $i = b$ et $j \neq b$ ou si $j=b$ et $i \neq b$ alors $\psi(i)=a=\varphi(p+1)$ et $\psi(j)=\varphi(j)$ et comme $j \neq p+1$ et que $\varphi$ est injective donc $\varphi(j) \neq \varphi(p+1)$ donc $\psi(j) \neq \psi(i)$;
    \end{itemize}
  \item $\psi$ est surjective~: Soit $i \in \intervalleentier{1}{q-1}$, si $i=a$ alors $i=\psi(b)$ et sinon, il existe $j \in \intervalleentier{1}{p+1}$ tel que $i=\psi(j)$; puisque $\varphi(p+1)=a \neq i$ alors $j \in \intervalleentier{1}{p}$ et $\varphi(b)=q \neq i$, donc $j \in \intervalleentier{1}{p} \setminus \{b\}$ et alors $i=\varphi(j)=\psi(j)$.
  \end{itemize}
  Au final $\psi$ est bijective.
\end{itemize}

Dans les deux cas, il a été possible de construire une bijection $\psi : \intervalleentier{1}{p} \longmapsto \intervalleentier{1}{q-1}$. Par hypothèse de récurrence, on en déduit que $p=q-1$ et donc que $q=p+1$. L'assertion $\P(p+1)$ est vérifiée.

\emph{Conclusion} Le théorème de récurrence nous permet donc d'affirmer que, pour tout naturel $p$ non nul, l'assertion $\P(p)$ est vraie.
\end{proof}

\begin{theo}
  Soit $P$ une partie finie non vide et majorée de $\N$. Alors $P$ est un ensemble fini et il existe un naturel $n$ non nul et une application bijective croissante (donc strictement croissante) $\varphi : \intervalleentier{1}{n} \longmapsto P$. De plus un tel couple $(n,  \varphi)$ vérifiant ces hypothèse est unique.
\end{theo}

\begin{proof}
  On démontre par récurrence sur $M \in \N$ la propriété $\P(M)$ ``Le théorème est vérifié pour toute partie $P$ majorée par $M$, c'est-à-dire telle que $\forall x \in P \ x \leq M$''.

\emph{Initialisation}. Si $M=0$, la seule partie majorée par $0$ de $\N$ est $P=\{0\}$. $P$ est donc finie de cardinal égal à $1$. Il existe une seule application de $\{1\}$ dans $\{0\}$ qui se trouve être bijective et croissante. $\P(0)$ est vraie.

\emph{Hérédité}. Soit $M \in \N$. On suppose que $\P(M)$ est vérifiée. Soit $P$ une partie de $\N$ non vide et majorée par $M+1$. On sait alors que $P$ possède un plus grand élément noté $a$. Alors $a \in P$ et $a \leq M+1$. Deux cas se présentent :
\begin{itemize}
\item Cas 1. Si $P=\{a\}$, alors $P$ est de cardinal fini égal à 1. Comme pour l'initialisation, l'existence et l'unicité du couple $(n, \varphi)$ sont immédiates.
\item Cas 2. Si $P \neq \{a\}$ on pose $P'=P\setminus\{a\}$. $P'$ est non vide. Pour tout $x \in P'$, $x \in P$ donc $x \leq a$ et $x \neq a$ alors $x \leq a-1 \leq M$. $P'$ est donc une partie non vide de $\N$ majorée par $M$. L'hypothèse de récurrence nous dit alors que $P'$ est finie et donc que $P=P' \cup \{a\}$ est finie aussi. Si on note $n = \Card(P)$ alors $\Card(P')=n-1$. L'hypothèse de récurrence nous assure l'existence d'une unique bijection croissante $\psi : \intervalleentier{1}{n-1}\longmapsto P'$. Montrons alors l'existence et l'unicité de la bijection croissante $\varphi : \intervalleentier{1}{n} \longmapsto P$.
\end{itemize}

Unicité. Soit $\varphi : \intervalleentier{1}{n} \longmapsto P$ une bijection croissante. Alors pour tout $k \in \intervalleentier{1}{n}$, $k \leq n$. Ainsi $\varphi(k) \leq \varphi(n)$. Alors $\varphi(n)$ est un majorant de $P=\varphi(\intervalleentier{1}{n})$ et $\varphi(n) \in P$ : $\varphi(n)$ est le plus grand élément de $P$. Par unicité du plus grand élément $a=\varphi(n)$. L'application $\varphi$ étant injective, on  $\varphi(\intervalleentier{1}{n}) \subset P'$ et on peut définir la restriction $\tilde{\varphi}$ de $\varphi$ à $\intervalleentier{1}{n-1}$ au départ et à $P'$ à l'arrivée. Comme $\varphi$ est croissante et bijective, sa restriction $\tilde{\varphi}$ est aussi croissante et bijective. Alors par hypothèse de récurrence, on a donc nécessairement $\psi = \tilde{\varphi}$. On a donc
\begin{equation}
  \forall k \in \intervalleentier{1}{n-1}\ \varphi(k)=\psi(k) \quad \varphi(n)=a
\end{equation}
L'application $\varphi$ est unique.

Existence. L'application $\varphi$ ainsi définie est une bijection croissante de $\intervalleentier{1}{n}$ sur $P$.

\emph{Conclusion}. Par théorème de récurrence, on a l'existence et l'unicité du couple $(n, \varphi)$ pour toute partie non vide et majorée de $\N$.
\end{proof}
